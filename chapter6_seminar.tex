% Chapter 6: Seminar - Cointegration and VECM
% Quizzes, Practice Problems, and Discussion
% Bachelor program, Bucharest University of Economic Studies

\documentclass[9pt, aspectratio=169, t]{beamer}

%=============================================================================
% THEME AND STYLE CONFIGURATION
%=============================================================================
\usetheme{Madrid}
\usecolortheme{seahorse}

% IDA-Inspired Color Palette
\definecolor{MainBlue}{RGB}{26, 58, 110}
\definecolor{AccentBlue}{RGB}{42, 82, 140}
\definecolor{IDAred}{RGB}{220, 53, 69}
\definecolor{DarkGray}{RGB}{51, 51, 51}
\definecolor{MediumGray}{RGB}{128, 128, 128}
\definecolor{LightGray}{RGB}{248, 248, 248}
\definecolor{VeryLightGray}{RGB}{235, 235, 235}
\definecolor{Crimson}{RGB}{220, 53, 69}
\definecolor{Forest}{RGB}{46, 125, 50}
\definecolor{Amber}{RGB}{181, 133, 63}
\definecolor{Orange}{RGB}{230, 126, 34}

\setbeamercolor{palette primary}{bg=MainBlue, fg=white}
\setbeamercolor{palette secondary}{bg=MainBlue!85, fg=white}
\setbeamercolor{palette tertiary}{bg=MainBlue!70, fg=white}
\setbeamercolor{structure}{fg=MainBlue}
\setbeamercolor{title}{fg=MainBlue}
\setbeamercolor{frametitle}{fg=MainBlue, bg=white}
\setbeamercolor{block title}{bg=MainBlue, fg=white}
\setbeamercolor{block body}{bg=VeryLightGray, fg=DarkGray}
\setbeamercolor{block title alerted}{bg=Crimson, fg=white}
\setbeamercolor{block body alerted}{bg=Crimson!8, fg=DarkGray}
\setbeamercolor{block title example}{bg=Forest, fg=white}
\setbeamercolor{block body example}{bg=Forest!8, fg=DarkGray}
\setbeamercolor{item}{fg=MainBlue}

\setbeamertemplate{navigation symbols}{}

\setbeamertemplate{footline}{
    \leavevmode%
    \hbox{%
        \begin{beamercolorbox}[wd=.333333\paperwidth,ht=2.5ex,dp=1ex,center]{author in head/foot}%
            \usebeamerfont{author in head/foot}\insertshortauthor
        \end{beamercolorbox}%
        \begin{beamercolorbox}[wd=.333333\paperwidth,ht=2.5ex,dp=1ex,center]{title in head/foot}%
            \usebeamerfont{title in head/foot}\insertshorttitle
        \end{beamercolorbox}%
        \begin{beamercolorbox}[wd=.333333\paperwidth,ht=2.5ex,dp=1ex,right]{date in head/foot}%
            \usebeamerfont{date in head/foot}\insertshortdate{}\hspace*{2em}
            \insertframenumber{} / \inserttotalframenumber\hspace*{2ex}
        \end{beamercolorbox}}%
    \vskip0pt%
}

%=============================================================================
% PACKAGES
%=============================================================================
\usepackage{amsmath, amssymb, amsthm}
\usepackage{mathtools}
\usepackage{bm}
\usepackage{tikz}
\usetikzlibrary{arrows.meta, positioning, shapes, calc}
\usepackage{booktabs}
\usepackage{multirow}
\usepackage{array}
\usepackage{graphicx}
\usepackage{hyperref}
\hypersetup{colorlinks=false, pdfborder={0 0 0}}
\graphicspath{{logos/}{charts/}}

%=============================================================================
% CUSTOM COMMANDS
%=============================================================================
\newcommand{\E}{\mathbb{E}}
\newcommand{\Var}{\text{Var}}
\newcommand{\Cov}{\text{Cov}}
\newcommand{\bY}{\mathbf{Y}}
\newcommand{\bA}{\mathbf{A}}
\newcommand{\bepsilon}{\boldsymbol{\varepsilon}}
\newcommand{\bvarepsilon}{\boldsymbol{\varepsilon}}
\newcommand{\bc}{\mathbf{c}}
\newcommand{\bPi}{\boldsymbol{\Pi}}
\newcommand{\balpha}{\boldsymbol{\alpha}}
\newcommand{\bbeta}{\boldsymbol{\beta}}
\newcommand{\bGamma}{\boldsymbol{\Gamma}}

%=============================================================================
% TITLE INFORMATION
%=============================================================================
\title[Chapter 6: Seminar]{Chapter 6: Seminar --- Cointegration \& VECM}
\subtitle{Bachelor program Faculty of Cybernetics, Statistics and Economic Informatics, Bucharest University of Economic Studies, Romania}
\author[Prof. dr. Daniel Traian Pele]{Prof. dr. Daniel Traian Pele\\[0.2cm]\footnotesize\texttt{danpele@ase.ro}}
\institute{Bucharest University of Economic Studies}
\date{Academic Year 2025--2026}

\begin{document}

%=============================================================================
% TITLE SLIDE
%=============================================================================
\begin{frame}[plain]
    \begin{tikzpicture}[remember picture, overlay]
        \fill[IDAred] (current page.north west) rectangle ([yshift=-0.15cm]current page.north east);
        \node[anchor=north west] at ([xshift=0.5cm, yshift=-0.3cm]current page.north west) {
            \href{https://www.ase.ro}{\includegraphics[height=1.1cm]{ase_logo.png}}
        };
        \node[anchor=north] at ([yshift=-0.3cm]current page.north) {
            \href{https://ai4efin.ase.ro}{\includegraphics[height=1.1cm]{ai4efin_logo.png}}
        };
        \node[anchor=north east] at ([xshift=-0.5cm, yshift=-0.3cm]current page.north east) {
            \href{https://www.digital-finance-msca.com}{\includegraphics[height=1.1cm]{msca_logo.png}}
        };
    \end{tikzpicture}
    \vfill
    \begin{center}
        {\Large\textcolor{MediumGray}{Time Series Analysis and Forecasting}}\\[0.3cm]
        {\Huge\textbf{\textcolor{MainBlue}{Chapter 6: Cointegration \& VECM}}}\\[0.5cm]
        {\Large\textcolor{IDAred}{Seminar}}
    \end{center}
    \vfill

    \begin{tikzpicture}[remember picture, overlay]
        \fill[IDAred] (current page.south west) rectangle ([yshift=0.15cm]current page.south east);
        \node[anchor=south west] at ([xshift=0.5cm, yshift=0.8cm]current page.south west) {
            \href{https://theida.net}{\includegraphics[height=0.9cm]{ida_logo.png}}
        };
        \node[anchor=south] at ([xshift=-3cm, yshift=0.8cm]current page.south) {
            \href{https://blockchain-research-center.com}{\includegraphics[height=0.9cm]{brc_logo.png}}
        };
        \node[anchor=south] at ([yshift=0.8cm]current page.south) {
            \href{https://quantinar.com}{\includegraphics[height=0.9cm]{qr_logo.png}}
        };
        \node[anchor=south] at ([xshift=3cm, yshift=0.8cm]current page.south) {
            \href{https://quantlet.com}{\includegraphics[height=0.9cm]{ql_logo.png}}
        };
        \node[anchor=south east] at ([xshift=-0.5cm, yshift=0.8cm]current page.south east) {
            \href{https://ipe.ro/new}{\includegraphics[height=0.9cm]{acad_logo.png}}
        };
    \end{tikzpicture}
\end{frame}

%=============================================================================
% OUTLINE
%=============================================================================
\begin{frame}{Seminar Outline}
    \tableofcontents
\end{frame}

%=============================================================================
% SECTION 1: REVIEW QUIZ
%=============================================================================
\section{Review Quiz}

\begin{frame}{Quiz 1: Cointegration Definition}
    \begin{alertblock}{Question}
        Two I(1) variables $X_t$ and $Y_t$ are cointegrated if:
    \end{alertblock}

    \vspace{0.3cm}

    \begin{enumerate}[A)]
        \item They are both stationary
        \item Their sum is I(2)
        \item A linear combination of them is I(0)
        \item They have the same mean
    \end{enumerate}

    \vspace{0.5cm}
    \begin{flushright}\textit{Answer on next slide...}\end{flushright}
\end{frame}

\begin{frame}{Quiz 1: Answer}
    \begin{exampleblock}{Answer: C -- A linear combination is I(0)}
        \begin{center}
            \includegraphics[width=0.85\textwidth, height=0.5\textheight, keepaspectratio]{charts/ch6_quiz1_cointegration.pdf}
        \end{center}
        \vspace{-0.2cm}
        {\footnotesize
        \textbf{Key}: $Y_t - \beta X_t \sim I(0)$ means they share a common stochastic trend. The linear combination (spread) is stationary even though both series are non-stationary.
        }
    \end{exampleblock}
\end{frame}

\begin{frame}{Quiz 2: Spurious Regression}
    \begin{alertblock}{Question}
        When regressing one independent random walk on another, you typically get:
    \end{alertblock}

    \vspace{0.3cm}

    \begin{enumerate}[A)]
        \item Low $R^2$ and insignificant coefficients
        \item High $R^2$ and significant coefficients (spurious!)
        \item Zero coefficients
        \item Undefined results
    \end{enumerate}

    \vspace{0.5cm}
    \begin{flushright}\textit{Answer on next slide...}\end{flushright}
\end{frame}

\begin{frame}{Quiz 2: Answer}
    \begin{exampleblock}{Answer: B -- High $R^2$ and significant coefficients (spurious!)}
        \begin{center}
            \includegraphics[width=0.85\textwidth, height=0.5\textheight, keepaspectratio]{charts/ch6_quiz2_spurious.pdf}
        \end{center}
        \vspace{-0.2cm}
        {\footnotesize
        \textbf{Granger-Newbold (1974)}: Regressing unrelated I(1) series gives misleading results. Rule of thumb: If $R^2 > DW$, suspect spurious regression!
        }
    \end{exampleblock}
\end{frame}

\begin{frame}{Quiz 3: Engle-Granger Test}
    \begin{alertblock}{Question}
        In the Engle-Granger two-step method, what do you test in step 2?
    \end{alertblock}

    \vspace{0.3cm}

    \begin{enumerate}[A)]
        \item Whether the original variables are stationary
        \item Whether the regression residuals have a unit root
        \item Whether the coefficients are significant
        \item Whether the $R^2$ is high enough
    \end{enumerate}

    \vspace{0.5cm}
    \begin{flushright}\textit{Answer on next slide...}\end{flushright}
\end{frame}

\begin{frame}{Quiz 3: Answer}
    \begin{exampleblock}{Answer: B -- Whether residuals have unit root}
        \textbf{Step 1}: Run OLS: $Y_t = \alpha + \beta X_t + e_t$, save residuals $\hat{e}_t$

        \vspace{0.2cm}
        \textbf{Step 2}: ADF test on residuals: $\Delta \hat{e}_t = \rho \hat{e}_{t-1} + \ldots$
        \begin{itemize}
            \item $H_0$: $\rho = 0$ (unit root $\Rightarrow$ no cointegration)
            \item $H_1$: $\rho < 0$ (stationary $\Rightarrow$ cointegration!)
        \end{itemize}

        \vspace{0.2cm}
        \textbf{Important}: Use Engle-Granger critical values, not standard ADF!
    \end{exampleblock}
\end{frame}

\begin{frame}{Quiz 4: Johansen Test Advantage}
    \begin{alertblock}{Question}
        The main advantage of Johansen over Engle-Granger is:
    \end{alertblock}

    \vspace{0.3cm}

    \begin{enumerate}[A)]
        \item It's simpler to compute
        \item It can detect multiple cointegrating relationships
        \item It doesn't require data
        \item It always finds cointegration
    \end{enumerate}

    \vspace{0.5cm}
    \begin{flushright}\textit{Answer on next slide...}\end{flushright}
\end{frame}

\begin{frame}{Quiz 4: Answer}
    \begin{exampleblock}{Answer: B -- Can detect multiple cointegrating relationships}
        \begin{center}
            \includegraphics[width=0.85\textwidth, height=0.45\textheight, keepaspectratio]{charts/ch6_quiz4_johansen.pdf}
        \end{center}
        \vspace{-0.2cm}
        {\footnotesize
        \textbf{Johansen advantages}:
        \begin{itemize}
            \item Tests for $r = 0, 1, 2, \ldots, k-1$ cointegrating vectors
            \item Maximum likelihood (more efficient)
            \item No need to choose dependent variable
        \end{itemize}
        }
    \end{exampleblock}
\end{frame}

\begin{frame}{Quiz 5: Rank of $\bPi$}
    \begin{alertblock}{Question}
        In a VECM with $k=3$ variables, if $\text{rank}(\bPi) = 2$, this means:
    \end{alertblock}

    \vspace{0.3cm}

    \begin{enumerate}[A)]
        \item No cointegration
        \item One cointegrating relationship
        \item Two cointegrating relationships
        \item All variables are stationary
    \end{enumerate}

    \vspace{0.5cm}
    \begin{flushright}\textit{Answer on next slide...}\end{flushright}
\end{frame}

\begin{frame}{Quiz 5: Answer}
    \begin{exampleblock}{Answer: C -- Two cointegrating relationships}
        \textbf{Rank interpretation} for $k$ variables:
        \begin{itemize}
            \item $\text{rank}(\bPi) = 0$: No cointegration (use VAR in differences)
            \item $0 < \text{rank}(\bPi) = r < k$: $r$ cointegrating vectors (use VECM)
            \item $\text{rank}(\bPi) = k$: All variables are I(0) (use VAR in levels)
        \end{itemize}

        \vspace{0.2cm}
        \textbf{With $k=3$ and $r=2$}:
        \begin{itemize}
            \item Two equilibrium relationships
            \item Only $k - r = 1$ common stochastic trend
        \end{itemize}
    \end{exampleblock}
\end{frame}

\begin{frame}{Quiz 6: VECM Structure}
    \begin{alertblock}{Question}
        In the VECM equation $\Delta \bY_t = \bc + \balpha\bbeta'\bY_{t-1} + \ldots$, what does $\balpha$ represent?
    \end{alertblock}

    \vspace{0.3cm}

    \begin{enumerate}[A)]
        \item The cointegrating vectors
        \item The adjustment (loading) coefficients
        \item The short-run dynamics
        \item The error variance
    \end{enumerate}

    \vspace{0.5cm}
    \begin{flushright}\textit{Answer on next slide...}\end{flushright}
\end{frame}

\begin{frame}{Quiz 6: Answer}
    \begin{exampleblock}{Answer: B -- The adjustment (loading) coefficients}
        \begin{center}
            \includegraphics[width=0.85\textwidth, height=0.45\textheight, keepaspectratio]{charts/ch6_quiz6_adjustment.pdf}
        \end{center}
        \vspace{-0.2cm}
        {\footnotesize
        \textbf{$\bPi = \balpha\bbeta'$}:
        \begin{itemize}
            \item $\bbeta$ = cointegrating vectors (define equilibrium)
            \item $\balpha$ = adjustment speeds (how fast each variable corrects)
        \end{itemize}
        }
    \end{exampleblock}
\end{frame}

\begin{frame}{Quiz 7: Error Correction Term}
    \begin{alertblock}{Question}
        If $Y_t - \beta X_t$ is the cointegrating relation and this term is positive, what happens?
    \end{alertblock}

    \vspace{0.3cm}

    \begin{enumerate}[A)]
        \item $Y$ is above equilibrium; $Y$ should decrease (if $\alpha < 0$)
        \item $Y$ is below equilibrium; $Y$ should increase
        \item Nothing, error correction doesn't affect levels
        \item Both variables increase
    \end{enumerate}

    \vspace{0.5cm}
    \begin{flushright}\textit{Answer on next slide...}\end{flushright}
\end{frame}

\begin{frame}{Quiz 7: Answer}
    \begin{exampleblock}{Answer: A -- $Y$ above equilibrium; decreases if $\alpha < 0$}
        \textbf{Error correction mechanism}:
        $$\Delta Y_t = \alpha(Y_{t-1} - \beta X_{t-1}) + \ldots$$

        \begin{itemize}
            \item If $Y_{t-1} - \beta X_{t-1} > 0$: $Y$ is ``too high''
            \item With $\alpha < 0$: $\Delta Y_t < 0$ (Y decreases toward equilibrium)
            \item This is the ``error correction'' pulling $Y$ back
        \end{itemize}

        \vspace{0.2cm}
        \textbf{Sign convention}: $\alpha$ should be negative for the dependent variable to move back toward equilibrium.
    \end{exampleblock}
\end{frame}

\begin{frame}{Quiz 8: Weak Exogeneity}
    \begin{alertblock}{Question}
        If $\alpha_2 = 0$ in a bivariate VECM, this means:
    \end{alertblock}

    \vspace{0.3cm}

    \begin{enumerate}[A)]
        \item There is no cointegration
        \item Variable 2 does not adjust to disequilibrium (weakly exogenous)
        \item Variable 1 does not adjust
        \item Both variables are stationary
    \end{enumerate}

    \vspace{0.5cm}
    \begin{flushright}\textit{Answer on next slide...}\end{flushright}
\end{frame}

\begin{frame}{Quiz 8: Answer}
    \begin{exampleblock}{Answer: B -- Variable 2 is weakly exogenous}
        \textbf{Weak exogeneity}: Variable doesn't respond to disequilibrium.

        \vspace{0.2cm}
        \textbf{Example: Interest rates}
        \begin{itemize}
            \item Long rate ($R_t$) often weakly exogenous ($\alpha_R \approx 0$)
            \item Short rate ($r_t$) adjusts to spread ($\alpha_r < 0$)
            \item Interpretation: Central bank adjusts short rate to maintain term structure
        \end{itemize}

        \vspace{0.2cm}
        \textbf{Implication}: Can estimate single equation for the adjusting variable.
    \end{exampleblock}
\end{frame}

\begin{frame}{Quiz 9: Trace Test}
    \begin{alertblock}{Question}
        The Johansen trace test with $H_0: r \leq 1$ vs $H_1: r > 1$ tests whether:
    \end{alertblock}

    \vspace{0.3cm}

    \begin{enumerate}[A)]
        \item There is exactly one cointegrating vector
        \item There are at most one cointegrating vectors
        \item There are more than one cointegrating vectors
        \item All eigenvalues are zero
    \end{enumerate}

    \vspace{0.5cm}
    \begin{flushright}\textit{Answer on next slide...}\end{flushright}
\end{frame}

\begin{frame}{Quiz 9: Answer}
    \begin{exampleblock}{Answer: B/C -- $H_0$: at most 1; $H_1$: more than 1}
        \textbf{Sequential testing procedure}:
        \begin{enumerate}
            \item Test $H_0: r = 0$ vs $H_1: r > 0$
            \item If rejected, test $H_0: r \leq 1$ vs $H_1: r > 1$
            \item Continue until fail to reject...
        \end{enumerate}

        \vspace{0.2cm}
        \textbf{Trace statistic}:
        $$\lambda_{\text{trace}}(r) = -T \sum_{i=r+1}^{k} \ln(1 - \hat{\lambda}_i)$$

        Reject $H_0$ if trace statistic $>$ critical value.
    \end{exampleblock}
\end{frame}

\begin{frame}{Quiz 10: VECM vs VAR in Differences}
    \begin{alertblock}{Question}
        If variables are cointegrated, using VAR in first differences instead of VECM:
    \end{alertblock}

    \vspace{0.3cm}

    \begin{enumerate}[A)]
        \item Gives identical results
        \item Is more efficient
        \item Loses long-run information (misspecified)
        \item Is the preferred approach
    \end{enumerate}

    \vspace{0.5cm}
    \begin{flushright}\textit{Answer on next slide...}\end{flushright}
\end{frame}

\begin{frame}{Quiz 10: Answer}
    \begin{exampleblock}{Answer: C -- Loses long-run information}
        \textbf{Granger Representation Theorem}: If cointegrated, VECM representation exists and should be used.

        \vspace{0.2cm}
        \begin{center}
        \begin{tabular}{lcc}
            \toprule
            & \textbf{VAR($\Delta$)} & \textbf{VECM} \\
            \midrule
            Long-run equilibrium & Lost & Preserved \\
            Error correction & No & Yes \\
            Forecasts (long-run) & Poor & Better \\
            \bottomrule
        \end{tabular}
        \end{center}

        \vspace{0.2cm}
        \textbf{Bottom line}: Differencing removes the long-run relationship that cointegration represents!
    \end{exampleblock}
\end{frame}

%=============================================================================
% TRUE/FALSE QUESTIONS
%=============================================================================
\section{True/False Questions}

\begin{frame}{True/False Questions}
    Determine if each statement is True or False:

    \vspace{0.3cm}
    \begin{enumerate}
        \item Cointegration requires all variables to be I(1).
        \item The cointegrating vector is unique.
        \item Spurious regression has low Durbin-Watson statistic.
        \item In VECM, both $\alpha$ coefficients must be non-zero.
        \item Johansen test requires choosing a dependent variable.
        \item The number of common trends = $k - r$.
    \end{enumerate}

    \vspace{0.3cm}
    \begin{flushright}\textit{Answers on next slide...}\end{flushright}
\end{frame}

\begin{frame}{True/False: Solutions}
    {\small
    \begin{enumerate}\setlength{\itemsep}{1pt}
        \item Cointegration requires all variables to be I(1). \hfill \textcolor{Forest}{\textbf{TRUE}}

        {\footnotesize \textcolor{MediumGray}{Standard CI(1,1) case: all variables I(1), linear combination I(0).}}

        \item The cointegrating vector is unique. \hfill \textcolor{Crimson}{\textbf{FALSE}}

        {\footnotesize \textcolor{MediumGray}{Unique only up to scalar multiplication. Usually normalized ($\beta_1 = 1$).}}

        \item Spurious regression has low Durbin-Watson statistic. \hfill \textcolor{Forest}{\textbf{TRUE}}

        {\footnotesize \textcolor{MediumGray}{$DW \approx 0$ indicates highly autocorrelated residuals (non-stationary).}}

        \item In VECM, both $\alpha$ coefficients must be non-zero. \hfill \textcolor{Crimson}{\textbf{FALSE}}

        {\footnotesize \textcolor{MediumGray}{One can be zero (weak exogeneity). At least one must be non-zero.}}

        \item Johansen test requires choosing a dependent variable. \hfill \textcolor{Crimson}{\textbf{FALSE}}

        {\footnotesize \textcolor{MediumGray}{That's Engle-Granger. Johansen treats all variables symmetrically.}}

        \item The number of common trends = $k - r$. \hfill \textcolor{Forest}{\textbf{TRUE}}

        {\footnotesize \textcolor{MediumGray}{$k$ variables, $r$ cointegrating relations $\Rightarrow$ $k-r$ common stochastic trends.}}
    \end{enumerate}
    }
\end{frame}

%=============================================================================
% SECTION 2: PRACTICE PROBLEMS
%=============================================================================
\section{Practice Problems}

\begin{frame}{Problem 1: Cointegration Identification}
    \begin{block}{Exercise}
        You have quarterly data on consumption ($C_t$) and income ($Y_t$). ADF tests show both are I(1). The regression $C_t = 0.85 Y_t + e_t$ gives residuals with ADF statistic $= -3.92$. The 5\% Engle-Granger critical value for 2 variables is $-3.34$.

        \vspace{0.2cm}
        Are $C_t$ and $Y_t$ cointegrated?
    \end{block}

    \vspace{0.5cm}
    \begin{flushright}\textit{Answer on next slide...}\end{flushright}
\end{frame}

\begin{frame}{Problem 1: Solution}
    \begin{exampleblock}{Solution: Yes, they are cointegrated}
        \textbf{Test}: $H_0$: No cointegration (residuals have unit root)

        \vspace{0.2cm}
        \textbf{ADF statistic}: $-3.92$

        \textbf{Critical value (5\%)}: $-3.34$

        \vspace{0.2cm}
        Since $-3.92 < -3.34$, we \textbf{reject} $H_0$ at 5\% level.

        \vspace{0.2cm}
        \textbf{Conclusion}: Residuals are stationary $\Rightarrow$ Cointegration exists!

        \vspace{0.2cm}
        \textbf{Interpretation}: Consumption and income share a common trend. The cointegrating vector is approximately $(1, -0.85)$, consistent with permanent income hypothesis.
    \end{exampleblock}
\end{frame}

\begin{frame}{Problem 2: VECM Interpretation}
    \begin{block}{Exercise}
        A VECM for short rate ($r_t$) and long rate ($R_t$) gives:
        \begin{align*}
            \Delta r_t &= 0.01 - 0.25(r_{t-1} - R_{t-1}) + \ldots \\
            \Delta R_t &= 0.005 - 0.02(r_{t-1} - R_{t-1}) + \ldots
        \end{align*}
        Interpret the adjustment coefficients.
    \end{block}

    \vspace{0.5cm}
    \begin{flushright}\textit{Answer on next slide...}\end{flushright}
\end{frame}

\begin{frame}{Problem 2: Solution}
    \begin{exampleblock}{Solution}
        \textbf{Error correction term}: $(r_{t-1} - R_{t-1})$ = spread

        \vspace{0.2cm}
        \textbf{Short rate} ($\alpha_r = -0.25$):
        \begin{itemize}
            \item When spread is positive (short $>$ long), short rate decreases
            \item 25\% of disequilibrium corrected per period
            \item Short rate actively adjusts
        \end{itemize}

        \vspace{0.2cm}
        \textbf{Long rate} ($\alpha_R = -0.02$):
        \begin{itemize}
            \item Very small adjustment coefficient
            \item Long rate is nearly weakly exogenous
            \item Mostly driven by expectations, not error correction
        \end{itemize}

        \vspace{0.2cm}
        \textbf{Economic interpretation}: Central bank (short rate) adjusts to maintain yield curve.
    \end{exampleblock}
\end{frame}

\begin{frame}{Problem 3: Johansen Test Results}
    \begin{block}{Exercise}
        Johansen trace test for 3 variables gives:
        \begin{center}
        \begin{tabular}{lcc}
            \toprule
            $H_0$ & Trace Stat & 5\% CV \\
            \midrule
            $r = 0$ & 45.2 & 29.8 \\
            $r \leq 1$ & 18.1 & 15.5 \\
            $r \leq 2$ & 3.2 & 3.8 \\
            \bottomrule
        \end{tabular}
        \end{center}
        What is the cointegrating rank?
    \end{block}

    \vspace{0.5cm}
    \begin{flushright}\textit{Answer on next slide...}\end{flushright}
\end{frame}

\begin{frame}{Problem 3: Solution}
    \begin{exampleblock}{Solution: Rank = 2}
        \textbf{Sequential testing}:
        \begin{enumerate}
            \item $H_0: r = 0$: $45.2 > 29.8$ $\Rightarrow$ \textbf{Reject} (at least 1)
            \item $H_0: r \leq 1$: $18.1 > 15.5$ $\Rightarrow$ \textbf{Reject} (at least 2)
            \item $H_0: r \leq 2$: $3.2 < 3.8$ $\Rightarrow$ \textbf{Fail to reject}
        \end{enumerate}

        \vspace{0.2cm}
        \textbf{Conclusion}: $r = 2$ cointegrating relationships

        \vspace{0.2cm}
        \textbf{Implications}:
        \begin{itemize}
            \item Two equilibrium relationships among 3 variables
            \item Only $3 - 2 = 1$ common stochastic trend
            \item Use VECM with 2 error correction terms
        \end{itemize}
    \end{exampleblock}
\end{frame}

\begin{frame}{Problem 4: Testing Weak Exogeneity}
    \begin{block}{Exercise}
        In a VECM for prices ($P$) and exchange rate ($E$), you estimate $\alpha_P = -0.15$ (s.e. = 0.04) and $\alpha_E = 0.02$ (s.e. = 0.03).

        Test whether the exchange rate is weakly exogenous at 5\%.
    \end{block}

    \vspace{0.5cm}
    \begin{flushright}\textit{Answer on next slide...}\end{flushright}
\end{frame}

\begin{frame}{Problem 4: Solution}
    \begin{exampleblock}{Solution: Exchange rate is weakly exogenous}
        \textbf{Test}: $H_0: \alpha_E = 0$ (weak exogeneity)

        \vspace{0.2cm}
        \textbf{t-statistic}: $t = \frac{0.02}{0.03} = 0.67$

        \textbf{Critical value (5\%, two-tailed)}: $\pm 1.96$

        \vspace{0.2cm}
        Since $|0.67| < 1.96$, \textbf{fail to reject} $H_0$.

        \vspace{0.2cm}
        \textbf{Conclusion}: Exchange rate does not respond to PPP disequilibrium.

        \vspace{0.2cm}
        \textbf{Implication}: Prices do all the adjusting to restore PPP equilibrium. Can estimate single-equation model for prices.
    \end{exampleblock}
\end{frame}

%=============================================================================
% SECTION 3: WORKED EXAMPLES
%=============================================================================
\section{Worked Examples}

\begin{frame}{Example: Term Structure of Interest Rates}
    {\small
    \begin{block}{Economic Theory}
        Expectations hypothesis: $R_t^{(n)} = \frac{1}{n}\sum_{i=0}^{n-1} E_t[r_{t+i}] + \text{premium}$

        If premium is constant $\Rightarrow$ spread $(R_t - r_t)$ should be stationary.
    \end{block}
    \begin{exampleblock}{Typical Findings}
        \begin{itemize}\setlength{\itemsep}{0pt}
            \item Both rates are I(1) (confirmed by ADF)
            \item Johansen test: $r = 1$ cointegrating vector
            \item Cointegrating vector $\approx (1, -1)$: spread is stationary
            \item Short rate adjusts ($\alpha_r < 0$), long rate weakly exogenous
        \end{itemize}
    \end{exampleblock}
    \begin{block}{Policy Implication}
        Central bank controls short rate; long rate driven by expectations.
    \end{block}
    }
\end{frame}

\begin{frame}{Example: Purchasing Power Parity}
    {\small
    \begin{block}{PPP Theory}
        $e_t = p_t - p_t^*$ (log exchange rate = price differential)

        Real exchange rate: $q_t = e_t - p_t + p_t^*$ should be stationary (long-run PPP)
    \end{block}
    \begin{exampleblock}{Empirical Challenges}
        \begin{itemize}\setlength{\itemsep}{0pt}
            \item Unit root tests: $e_t$, $p_t$, $p_t^*$ all I(1)
            \item Cointegration tests: Mixed results depending on sample
            \item Half-life of PPP deviations: 3-5 years (slow adjustment)
            \item Weak exogeneity: Exchange rate often doesn't adjust
        \end{itemize}
    \end{exampleblock}
    \begin{alertblock}{PPP Puzzle}
        Real exchange rate is highly persistent---slow mean reversion is hard to explain with standard models.
    \end{alertblock}
    }
\end{frame}

\begin{frame}{Example: Pairs Trading Strategy}
    {\small
    \begin{block}{Idea}
        Find cointegrated stocks $\Rightarrow$ trade the stationary spread
    \end{block}
    \begin{exampleblock}{Implementation Steps}
        \begin{enumerate}\setlength{\itemsep}{0pt}
            \item \textbf{Identify pairs}: Test cointegration (e.g., Coca-Cola \& Pepsi)
            \item \textbf{Estimate spread}: $z_t = P_A - \beta P_B$
            \item \textbf{Trading rules}:
            \begin{itemize}
                \item $z_t > \mu + 2\sigma$: Sell A, Buy B (spread too wide)
                \item $z_t < \mu - 2\sigma$: Buy A, Sell B (spread too narrow)
                \item Exit when $z_t \approx \mu$
            \end{itemize}
        \end{enumerate}
    \end{exampleblock}
    \begin{alertblock}{Risks}
        Cointegration can break down; spread may not revert; transaction costs.
    \end{alertblock}
    }
\end{frame}

\begin{frame}{Python Cointegration Analysis: Key Functions}
    {\footnotesize
    \begin{block}{Essential Libraries}
        \texttt{from statsmodels.tsa.stattools import coint, adfuller} \\
        \texttt{from statsmodels.tsa.vector\_ar.vecm import coint\_johansen, VECM}
    \end{block}

    \begin{block}{Workflow}
        \begin{enumerate}\setlength{\itemsep}{0pt}
            \item Unit root tests: \texttt{adfuller(series)}
            \item Engle-Granger: \texttt{coint(y, x)} returns test stat \& p-value
            \item Johansen: \texttt{coint\_johansen(data, det\_order, k\_ar\_diff)}
            \item Fit VECM: \texttt{model = VECM(data, k\_ar\_diff=2, coint\_rank=1)}
            \item Results: \texttt{results = model.fit()}
        \end{enumerate}
    \end{block}

    \begin{alertblock}{Note}
        Complete working examples are provided in the Jupyter notebooks.
    \end{alertblock}
    }
\end{frame}

%=============================================================================
% SECTION 4: DISCUSSION TOPICS
%=============================================================================
\section{Discussion Topics}

\begin{frame}{Discussion: Cointegration vs Correlation}
    {\small
    \begin{alertblock}{Key Question}
        Two series are highly correlated. Are they cointegrated?
    \end{alertblock}
    \begin{block}{Answer: Not necessarily!}
        \begin{itemize}\setlength{\itemsep}{0pt}
            \item \textbf{Correlation}: Measures co-movement (can be spurious for I(1))
            \item \textbf{Cointegration}: Requires stationary linear combination
        \end{itemize}
    \end{block}
    \begin{exampleblock}{Example}
        Two independent random walks can have correlation $> 0.9$ purely by chance (spurious correlation). But they're NOT cointegrated---their spread is also I(1).

        \vspace{0.2cm}
        \textbf{Cointegration} implies a meaningful long-run equilibrium relationship.
    \end{exampleblock}
    }
\end{frame}

\begin{frame}{Discussion: Choosing Deterministic Components}
    {\small
    \begin{alertblock}{Key Question}
        Johansen test has 5 cases for deterministics. Which to choose?
    \end{alertblock}
    \begin{block}{Guidelines}
        \begin{enumerate}\setlength{\itemsep}{0pt}
            \item \textbf{No constant, no trend}: Rarely used (requires mean-zero data)
            \item \textbf{Constant in CE only}: Level series, no drift
            \item \textbf{Constant unrestricted}: Most common for economic data
            \item \textbf{Trend in CE}: Series have deterministic trends
            \item \textbf{Trend unrestricted}: Trending differences (uncommon)
        \end{enumerate}
    \end{block}
    \begin{exampleblock}{Practical Advice}
        Start with Case 3 (constant unrestricted). Check sensitivity to specification. Use economic reasoning: do levels have trends?
    \end{exampleblock}
    }
\end{frame}

%=============================================================================
% SECTION 5: EXERCISES
%=============================================================================
\section{Exercises for Self-Study}

\begin{frame}{Take-Home Exercises}
    {\footnotesize
    \begin{enumerate}\setlength{\itemsep}{2pt}
        \item \textbf{Theoretical}: Show that if $Y_t$ and $X_t$ are both random walks with the same innovation, they are cointegrated.

        \item \textbf{Computation}: Given VECM estimates:
            \begin{align*}
                \Delta Y_t &= 0.5 - 0.3(Y_{t-1} - 2X_{t-1}) + 0.2\Delta Y_{t-1} \\
                \Delta X_t &= 0.1 + 0.1(Y_{t-1} - 2X_{t-1}) + 0.4\Delta X_{t-1}
            \end{align*}
            \begin{itemize}\setlength{\itemsep}{0pt}
                \item What is the cointegrating vector?
                \item Which variable adjusts more quickly?
                \item What is the long-run equilibrium relationship?
            \end{itemize}

        \item \textbf{Applied}: Download 10-year and 3-month Treasury rates:
            \begin{itemize}\setlength{\itemsep}{0pt}
                \item Test for unit roots; Test for cointegration (Engle-Granger and Johansen)
                \item Estimate VECM; Interpret adjustment coefficients
            \end{itemize}

        \item \textbf{Critical Thinking}: Why might PPP hold in the long run but not short run?
    \end{enumerate}
    }
\end{frame}

\begin{frame}{Exercise Solutions Hints}
    {\footnotesize
    \begin{block}{Hints}
        \begin{enumerate}\setlength{\itemsep}{1pt}
            \item If $Y_t = Y_{t-1} + \varepsilon_t$ and $X_t = X_{t-1} + \varepsilon_t$ (same shock), then $Y_t - X_t = Y_0 - X_0$ is constant (stationary).

            \item From the VECM:
                \begin{itemize}\setlength{\itemsep}{0pt}
                    \item Cointegrating vector: $(1, -2)$ (normalized on $Y$)
                    \item $Y$ adjusts faster: $|\alpha_Y| = 0.3 > |\alpha_X| = 0.1$
                    \item Long-run: $Y = 2X$ (when EC term = 0)
                \end{itemize}

            \item For interest rates:
                \begin{itemize}\setlength{\itemsep}{0pt}
                    \item Both typically I(1); spread usually stationary
                    \item Expect one cointegrating vector with $(1, -1)$
                    \item Short rate typically adjusts; long rate often weakly exogenous
                \end{itemize}

            \item PPP deviations: Transportation costs, non-traded goods, sticky prices, tariffs, market segmentation all slow adjustment but don't prevent long-run convergence.
        \end{enumerate}
    \end{block}
    }
\end{frame}

%=============================================================================
% SUMMARY
%=============================================================================
\begin{frame}{Key Takeaways from This Seminar}
    {\footnotesize
    \begin{block}{Main Points}
        \begin{enumerate}\setlength{\itemsep}{0pt}
            \item \textbf{Cointegration}: I(1) variables with stationary linear combination
            \item \textbf{Spurious regression}: High $R^2$ without cointegration is meaningless
            \item \textbf{Engle-Granger}: Simple, but only one cointegrating vector
            \item \textbf{Johansen}: Multiple vectors, MLE, more powerful
        \end{enumerate}
    \end{block}
    \begin{block}{VECM Insights}
        \begin{itemize}\setlength{\itemsep}{0pt}
            \item $\bbeta$ defines equilibrium; $\balpha$ determines adjustment speed
            \item Weak exogeneity ($\alpha = 0$): Variable doesn't respond to disequilibrium
            \item Always use VECM (not VAR in differences) when cointegrated
        \end{itemize}
    \end{block}
    \begin{alertblock}{Remember}
        Cointegration is about \textbf{long-run equilibrium}, not just correlation!
    \end{alertblock}
    }
\end{frame}

\end{document}
