% Capitolul 3: Seminar - Modele ARIMA
% Teste, Probleme Practice și Discuții
% Program de licență, Academia de Studii Economice din București

\documentclass[9pt, aspectratio=169, t]{beamer}

% Ensure content fits on slides
\setbeamersize{text margin left=8mm, text margin right=8mm}

%=============================================================================
% THEME AND STYLE CONFIGURATION
%=============================================================================
\usetheme{Madrid}
\usecolortheme{seahorse}

% IDA-Inspired Color Palette
\definecolor{MainBlue}{RGB}{26, 58, 110}
\definecolor{AccentBlue}{RGB}{42, 82, 140}
\definecolor{IDAred}{RGB}{220, 53, 69}
\definecolor{DarkGray}{RGB}{51, 51, 51}
\definecolor{MediumGray}{RGB}{128, 128, 128}
\definecolor{LightGray}{RGB}{248, 248, 248}
\definecolor{VeryLightGray}{RGB}{235, 235, 235}
\definecolor{Crimson}{RGB}{220, 53, 69}
\definecolor{Forest}{RGB}{46, 125, 50}
\definecolor{Amber}{RGB}{181, 133, 63}
\definecolor{Orange}{RGB}{230, 126, 34}

\setbeamercolor{palette primary}{bg=MainBlue, fg=white}
\setbeamercolor{palette secondary}{bg=MainBlue!85, fg=white}
\setbeamercolor{palette tertiary}{bg=MainBlue!70, fg=white}
\setbeamercolor{structure}{fg=MainBlue}
\setbeamercolor{title}{fg=MainBlue}
\setbeamercolor{frametitle}{fg=MainBlue, bg=white}
\setbeamercolor{block title}{bg=MainBlue, fg=white}
\setbeamercolor{block body}{bg=VeryLightGray, fg=DarkGray}
\setbeamercolor{block title alerted}{bg=Crimson, fg=white}
\setbeamercolor{block body alerted}{bg=Crimson!8, fg=DarkGray}
\setbeamercolor{block title example}{bg=Forest, fg=white}
\setbeamercolor{block body example}{bg=Forest!8, fg=DarkGray}
\setbeamercolor{item}{fg=MainBlue}

\setbeamertemplate{navigation symbols}{}

\setbeamertemplate{footline}{
    \leavevmode%
    \hbox{%
        \begin{beamercolorbox}[wd=.333333\paperwidth,ht=2.5ex,dp=1ex,center]{author in head/foot}%
            \usebeamerfont{author in head/foot}\insertshortauthor
        \end{beamercolorbox}%
        \begin{beamercolorbox}[wd=.333333\paperwidth,ht=2.5ex,dp=1ex,center]{title in head/foot}%
            \usebeamerfont{title in head/foot}\insertshorttitle
        \end{beamercolorbox}%
        \begin{beamercolorbox}[wd=.333333\paperwidth,ht=2.5ex,dp=1ex,right]{date in head/foot}%
            \usebeamerfont{date in head/foot}\insertshortdate{}\hspace*{2em}
            \insertframenumber{} / \inserttotalframenumber\hspace*{2ex}
        \end{beamercolorbox}}%
    \vskip0pt%
}

%=============================================================================
% PACKAGES
%=============================================================================
\usepackage[utf8]{inputenc}
\usepackage[T1]{fontenc}
\usepackage{amsmath, amssymb, amsthm}
\usepackage{mathtools}
\usepackage{bm}
\usepackage{tikz}
\usetikzlibrary{arrows.meta, positioning, shapes, calc}
\usepackage{booktabs}
\usepackage{multirow}
\usepackage{array}
\usepackage{graphicx}
\usepackage{hyperref}
\hypersetup{colorlinks=false, pdfborder={0 0 0}}
\graphicspath{{logos/}{charts/}}

%=============================================================================
% CUSTOM COMMANDS
%=============================================================================
\newcommand{\E}{\mathbb{E}}
\newcommand{\Var}{\text{Var}}
\newcommand{\Cov}{\text{Cov}}
\newcommand{\Corr}{\text{Corr}}
\newcommand{\R}{\mathbb{R}}

%=============================================================================
% TITLE INFORMATION
%=============================================================================
\title[Capitolul 3: Seminar]{Capitolul 3: Seminar --- Modele ARIMA}
\subtitle{Program de licență, Facultatea de Cibernetică, Statistică și Informatică Economică, Academia de Studii Economice din București}
\author[Prof. dr. Daniel Traian Pele]{Prof. dr. Daniel Traian Pele\\[0.2cm]\footnotesize\texttt{danpele@ase.ro}}
\institute{Academia de Studii Economice din București}
\date{An Universitar 2025--2026}

\begin{document}

%=============================================================================
% TITLE SLIDE
%=============================================================================
\begin{frame}[plain]
    \begin{tikzpicture}[remember picture, overlay]
        \fill[IDAred] (current page.north west) rectangle ([yshift=-0.15cm]current page.north east);
        \node[anchor=north west] at ([xshift=0.5cm, yshift=-0.3cm]current page.north west) {
            \href{https://www.ase.ro}{\includegraphics[height=1.1cm]{ase_logo.png}}
        };
        \node[anchor=north] at ([yshift=-0.3cm]current page.north) {
            \href{https://ai4efin.ase.ro}{\includegraphics[height=1.1cm]{ai4efin_logo.png}}
        };
        \node[anchor=north east] at ([xshift=-0.5cm, yshift=-0.3cm]current page.north east) {
            \href{https://www.digital-finance-msca.com}{\includegraphics[height=1.1cm]{msca_logo.png}}
        };
    \end{tikzpicture}
    \vfill
    \begin{center}
        {\Large\textcolor{MediumGray}{Analiza și Prognoza Seriilor de Timp}}\\[0.3cm]
        {\Huge\textbf{\textcolor{MainBlue}{Capitolul 3: Modele ARIMA}}}\\[0.5cm]
        {\Large\textcolor{IDAred}{Seminar}}
    \end{center}
    \vfill

    \begin{tikzpicture}[remember picture, overlay]
        \fill[IDAred] (current page.south west) rectangle ([yshift=0.15cm]current page.south east);
        \node[anchor=south west] at ([xshift=0.5cm, yshift=0.8cm]current page.south west) {
            \href{https://theida.net}{\includegraphics[height=0.9cm]{ida_logo.png}}
        };
        \node[anchor=south] at ([xshift=-3cm, yshift=0.8cm]current page.south) {
            \href{https://blockchain-research-center.com}{\includegraphics[height=0.9cm]{brc_logo.png}}
        };
        \node[anchor=south] at ([yshift=0.8cm]current page.south) {
            \href{https://quantinar.com}{\includegraphics[height=0.9cm]{qr_logo.png}}
        };
        \node[anchor=south] at ([xshift=3cm, yshift=0.8cm]current page.south) {
            \href{https://quantlet.com}{\includegraphics[height=0.9cm]{ql_logo.png}}
        };
        \node[anchor=south east] at ([xshift=-0.5cm, yshift=0.8cm]current page.south east) {
            \href{https://ipe.ro/new}{\includegraphics[height=0.9cm]{acad_logo.png}}
        };
    \end{tikzpicture}
\end{frame}

%=============================================================================
% OUTLINE
%=============================================================================
\begin{frame}{Cuprins Seminar}
    \tableofcontents
\end{frame}

%=============================================================================
% SECTION 1: REVIEW QUIZ
%=============================================================================
\section{Test de Recapitulare}

\begin{frame}{Test 1: Ordinul de Integrare}
    \begin{alertblock}{Întrebare}
        O serie de timp $Y_t$ necesită două diferențe pentru a deveni staționară. Care este ordinul ei de integrare?
    \end{alertblock}

    \vspace{0.3cm}

    \begin{enumerate}[A)]
        \item $I(0)$
        \item $I(1)$
        \item $I(2)$
        \item Nu poate fi determinat
    \end{enumerate}

    \vspace{0.5cm}
    \pause
    \begin{exampleblock}{Răspuns: C -- $I(2)$}
        \textbf{Definiție}: $Y_t \sim I(d)$ dacă $\Delta^d Y_t$ este staționară dar $\Delta^{d-1} Y_t$ nu este.

        \textbf{Exemplu}: Dacă $Y_t$ urmează $\Delta^2 Y_t = \varepsilon_t$, atunci:
        \begin{itemize}
            \item $\Delta Y_t = \Delta Y_{t-1} + \varepsilon_t$ (încă are rădăcină unitară)
            \item $\Delta^2 Y_t = \varepsilon_t$ (zgomot alb, staționară)
        \end{itemize}

        \textbf{Lumea reală}: Nivelurile prețurilor pot fi $I(2)$ când inflația însăși este nestaționară.
    \end{exampleblock}
\end{frame}

\begin{frame}{Vizual: Procese Integrate}
    \begin{center}
        \includegraphics[width=0.95\textwidth]{charts/ch3_def_integrated.pdf}
    \end{center}
    \vspace{-0.2cm}
    \small $I(0)$: staționară. $I(1)$: o diferență necesară. $I(2)$: două diferențe necesare pentru a deveni staționară.
\end{frame}

\begin{frame}{Test 2: Proprietățile Mersului Aleatoriu}
    \begin{alertblock}{Întrebare}
        Pentru un mers aleatoriu $Y_t = Y_{t-1} + \varepsilon_t$ cu $\Var(\varepsilon_t) = \sigma^2$, care este $\Var(Y_t)$?
    \end{alertblock}

    \vspace{0.3cm}

    \begin{enumerate}[A)]
        \item $\sigma^2$
        \item $t \cdot \sigma^2$
        \item $\sigma^2 / t$
        \item $\sigma^2 / (1-\phi^2)$
    \end{enumerate}

    \vspace{0.5cm}
    \pause
    \begin{exampleblock}{Răspuns: B -- $t \cdot \sigma^2$}
        \vspace{-0.2cm}
        \begin{center}
            \includegraphics[width=0.9\textwidth, height=0.45\textheight, keepaspectratio]{charts/sem3_rw_variance.pdf}
        \end{center}
        \vspace{-0.2cm}
        {\footnotesize
        \textbf{Demonstrație}: $Y_t = \sum_{i=1}^{t}\varepsilon_i \Rightarrow \Var(Y_t) = t\sigma^2$ (crește liniar $\Rightarrow$ nestaționară!)
        }
    \end{exampleblock}
\end{frame}

\begin{frame}{Test 3: Ipotezele Testului ADF}
    \begin{alertblock}{Întrebare}
        În testul Augmented Dickey-Fuller, care este ipoteza nulă?
    \end{alertblock}

    \vspace{0.3cm}

    \begin{enumerate}[A)]
        \item Seria este staționară
        \item Seria are o rădăcină unitară
        \item Seria nu are autocorelație
        \item Seria este distribuită normal
    \end{enumerate}

    \vspace{0.5cm}
    \pause
    \begin{exampleblock}{Răspuns: B -- Seria are o rădăcină unitară}
        \textbf{Regresia ADF}: $\Delta Y_t = \alpha + \gamma Y_{t-1} + \sum_{j=1}^{p}\delta_j \Delta Y_{t-j} + \varepsilon_t$

        \textbf{Ipoteze}:
        \begin{itemize}
            \item $H_0: \gamma = 0$ (rădăcină unitară, nestaționară)
            \item $H_1: \gamma < 0$ (staționară)
        \end{itemize}

        \textbf{Decizie}: Respingem $H_0$ dacă statistica $t <$ valoarea critică (de ex., $-2.86$ la 5\%)

        \textbf{Notă}: Folosește distribuția specială Dickey-Fuller, nu $t$ standard.
    \end{exampleblock}
\end{frame}

\begin{frame}{Vizual: Testul ADF}
    \begin{center}
        \includegraphics[width=0.95\textwidth]{charts/ch3_def_adf.pdf}
    \end{center}
    \vspace{-0.2cm}
    \small Stânga: staționară -- ADF respinge rădăcina unitară. Dreapta: nestaționară -- ADF nu respinge.
\end{frame}

\begin{frame}{Test 4: Notația ARIMA}
    \begin{alertblock}{Întrebare}
        Ce înseamnă ARIMA(2,1,1)?
    \end{alertblock}

    \vspace{0.3cm}

    \begin{enumerate}[A)]
        \item AR(2) pe date diferențiate cu erori MA(1)
        \item AR(1) cu 2 diferențe și MA(1)
        \item MA(2) cu 1 diferență și AR(1)
        \item 2 lag-uri, 1 trend, 1 componentă sezonieră
    \end{enumerate}

    \vspace{0.5cm}
    \pause
    \begin{exampleblock}{Răspuns: A -- AR(2) pe date diferențiate cu erori MA(1)}
        \textbf{ARIMA($p,d,q$)}: $\phi(L)(1-L)^d Y_t = \theta(L)\varepsilon_t$

        \textbf{ARIMA(2,1,1) se expandează la}:
        \[
        (1-\phi_1 L - \phi_2 L^2)(1-L)Y_t = (1+\theta_1 L)\varepsilon_t
        \]
        Sau echivalent: $(1-\phi_1 L - \phi_2 L^2)\Delta Y_t = (1+\theta_1 L)\varepsilon_t$

        \textbf{Interpretare}: Mai întâi diferențiem seria, apoi ajustăm ARMA(2,1) pe $\Delta Y_t$.
    \end{exampleblock}
\end{frame}

\begin{frame}{Vizual: Procesul ARIMA}
    \begin{center}
        \includegraphics[width=0.95\textwidth]{charts/ch3_def_arima.pdf}
    \end{center}
    \vspace{-0.2cm}
    \small Sus: seria ARIMA originală. Jos: după diferențiere, folosim ACF/PACF pentru a identifica ordinele AR și MA.
\end{frame}

\begin{frame}{Test 5: Operatorul de Diferență}
    \begin{alertblock}{Întrebare}
        Care este $(1-L)^2 Y_t$ expandat?
    \end{alertblock}

    \vspace{0.3cm}

    \begin{enumerate}[A)]
        \item $Y_t - Y_{t-1}$
        \item $Y_t - 2Y_{t-1} + Y_{t-2}$
        \item $Y_t + 2Y_{t-1} + Y_{t-2}$
        \item $Y_t - Y_{t-2}$
    \end{enumerate}

    \vspace{0.5cm}
    \pause
    \begin{exampleblock}{Răspuns: B -- $Y_t - 2Y_{t-1} + Y_{t-2}$}
        \textbf{Expandare folosind teorema binomială}:
        \[
        (1-L)^2 = 1 - 2L + L^2
        \]
        \textbf{Aplicare lui $Y_t$}:
        \[
        (1-L)^2 Y_t = Y_t - 2L \cdot Y_t + L^2 \cdot Y_t = Y_t - 2Y_{t-1} + Y_{t-2}
        \]

        \textbf{Notă}: Aceasta este egală cu $\Delta(\Delta Y_t) = \Delta Y_t - \Delta Y_{t-1}$, ``schimbarea schimbărilor''.
    \end{exampleblock}
\end{frame}

\begin{frame}{Test 6: KPSS vs ADF}
    \begin{alertblock}{Întrebare}
        Cum diferă testul KPSS de testul ADF?
    \end{alertblock}

    \vspace{0.3cm}

    \begin{enumerate}[A)]
        \item KPSS testează sezonalitatea, ADF testează trenduri
        \item KPSS are staționaritatea ca nulă, ADF are rădăcina unitară ca nulă
        \item KPSS este mai puternic decât ADF
        \item Nu există diferență
    \end{enumerate}

    \vspace{0.5cm}
    \pause
    \begin{exampleblock}{Răspuns: B -- Ipoteze nule inversate}
        \vspace{-0.2cm}
        \begin{center}
            \includegraphics[width=0.85\textwidth, height=0.5\textheight, keepaspectratio]{charts/sem3_adf_kpss.pdf}
        \end{center}
        \vspace{-0.2cm}
        {\footnotesize
        \textbf{Strategie}: Folosiți ambele teste împreună pentru inferență robustă!
        }
    \end{exampleblock}
\end{frame}

\begin{frame}{Test 7: Supradiferențierea}
    \begin{alertblock}{Întrebare}
        Dacă $Y_t \sim I(1)$ și calculăm $\Delta^2 Y_t$, ce se întâmplă?
    \end{alertblock}

    \vspace{0.3cm}

    \begin{enumerate}[A)]
        \item Obținem o serie staționară mai bună
        \item Introducem autocorelație negativă artificială
        \item Varianța scade
        \item Nu se schimbă nimic
    \end{enumerate}

    \vspace{0.5cm}
    \pause
    \begin{exampleblock}{Răspuns: B -- Autocorelație negativă artificială}
        \vspace{-0.2cm}
        \begin{center}
            \includegraphics[width=0.95\textwidth, height=0.48\textheight, keepaspectratio]{charts/sem3_overdifferencing.pdf}
        \end{center}
        \vspace{-0.2cm}
        {\footnotesize
        \textbf{Diagnostic}: ACF la lag 1 $\approx -0.5$ semnalează supradiferențiere. Reduceți $d$ cu 1!
        }
    \end{exampleblock}
\end{frame}

\begin{frame}{Test 8: Varianța Prognozei}
    \begin{alertblock}{Întrebare}
        Pentru un model ARIMA(0,1,0) (mers aleatoriu), cum se comportă varianța prognozei când orizontul $h$ crește?
    \end{alertblock}

    \vspace{0.3cm}

    \begin{enumerate}[A)]
        \item Rămâne constantă
        \item Scade la zero
        \item Crește liniar cu $h$
        \item Convergă la o limită finită
    \end{enumerate}

    \vspace{0.5cm}
    \pause
    \begin{exampleblock}{Răspuns: C -- Crește liniar cu $h$}
        \textbf{Prognoza mersului aleatoriu}: $\hat{Y}_{T+h|T} = Y_T$ (cea mai bună prognoză este valoarea curentă)

        \textbf{Eroarea de prognoză}: $Y_{T+h} - \hat{Y}_{T+h|T} = \sum_{i=1}^{h} \varepsilon_{T+i}$

        \textbf{Varianță}:
        \[
        \Var(Y_{T+h} - \hat{Y}_{T+h|T}) = h\sigma^2
        \]

        \textbf{IC 95\%}: $Y_T \pm 1.96\sqrt{h}\sigma$ (se lărgește cu $\sqrt{h}$)
    \end{exampleblock}
\end{frame}

\begin{frame}{Test 9: Puterea Testului de Rădăcină Unitară}
    \begin{alertblock}{Întrebare}
        Testul ADF are putere scăzută când:
    \end{alertblock}

    \vspace{0.3cm}

    \begin{enumerate}[A)]
        \item Dimensiunea eșantionului este foarte mare
        \item Rădăcina adevărată este aproape dar nu egală cu 1
        \item Seria nu are trend
        \item Seria este clar staționară
    \end{enumerate}

    \vspace{0.5cm}
    \pause
    \begin{exampleblock}{Răspuns: B -- Rădăcina aproape dar nu egală cu 1}
        \textbf{Exemplu}: AR(1) cu $\phi = 0.95$ vs mers aleatoriu ($\phi = 1$)

        \textbf{Problemă}: Ambele au modele ACF similare (descreștere lentă), dar una este staționară!

        \textbf{Putere scăzută înseamnă}: Probabilitate mare de eroare de tip II (eșec în respingerea lui $H_0$ fals)

        \textbf{Soluții}:
        \begin{itemize}
            \item Dimensiuni mai mari ale eșantionului
            \item Testul Phillips-Perron (robust la heteroscedasticitate)
            \item Teste de rădăcină unitară pe paneluri (serii multiple)
        \end{itemize}
    \end{exampleblock}
\end{frame}

\begin{frame}{Test 10: Selecția Modelului ARIMA}
    \begin{alertblock}{Întrebare}
        După o diferențiere, ACF arată un vârf doar la lag 1, și PACF descrește. Modelul potrivit este:
    \end{alertblock}

    \vspace{0.3cm}

    \begin{enumerate}[A)]
        \item ARIMA(1,1,0)
        \item ARIMA(0,1,1)
        \item ARIMA(1,1,1)
        \item ARIMA(0,2,1)
    \end{enumerate}

    \vspace{0.5cm}
    \pause
    \begin{exampleblock}{Răspuns: B -- ARIMA(0,1,1)}
        \vspace{-0.2cm}
        \begin{center}
            \includegraphics[width=0.95\textwidth, height=0.5\textheight, keepaspectratio]{charts/sem3_arima_flowchart.pdf}
        \end{center}
        \vspace{-0.2cm}
        {\footnotesize
        \textbf{Model}: ACF se întrerupe la lag 1, PACF descrește $\Rightarrow$ MA(1) pentru seria diferențiată. Model complet: ARIMA(0,1,1) = IMA(1,1)
        }
    \end{exampleblock}
\end{frame}

\begin{frame}{Test 11: Staționaritate în Trend vs Staționaritate în Diferențe}
    \begin{alertblock}{Întrebare}
        Un proces staționar în trend devine staționar prin:
    \end{alertblock}

    \vspace{0.3cm}

    \begin{enumerate}[A)]
        \item Luarea diferențelor de ordinul întâi
        \item Eliminarea trendului determinist prin regresie
        \item Luarea diferențelor de ordinul doi
        \item Aplicarea ajustării sezoniere
    \end{enumerate}

    \vspace{0.5cm}
    \pause
    \begin{exampleblock}{Răspuns: B -- Eliminarea trendului determinist prin regresie}
        \vspace{-0.2cm}
        \begin{center}
            \includegraphics[width=0.95\textwidth, height=0.5\textheight, keepaspectratio]{charts/sem3_trend_vs_diff.pdf}
        \end{center}
        \vspace{-0.2cm}
        {\footnotesize
        \textbf{Staționar în trend}: Eliminarea trendului prin regresie (șocurile sunt temporare). \textbf{Staționar în diferențe}: Diferențiere (șocurile sunt permanente). Tratamentul greșit afectează modelul!
        }
    \end{exampleblock}
\end{frame}

\begin{frame}{Test 12: Invertibilitatea ARIMA}
    \begin{alertblock}{Întrebare}
        ARIMA(0,1,1) cu $\theta_1 = 1.2$ este:
    \end{alertblock}

    \vspace{0.3cm}

    \begin{enumerate}[A)]
        \item Staționară și invertibilă
        \item Nestaționară dar invertibilă
        \item Nestaționară și neinvertibilă
        \item Staționară dar neinvertibilă
    \end{enumerate}

    \vspace{0.5cm}
    \pause
    \begin{exampleblock}{Răspuns: C -- Nestaționară și neinvertibilă}
        \textbf{Verificare staționaritate}: $d=1$ înseamnă o rădăcină unitară $\Rightarrow$ \textcolor{Crimson}{Nestaționară}

        \textbf{Verificare invertibilitate}: Polinomul MA este $\theta(z) = 1 + 1.2z$
        \begin{itemize}
            \item Rădăcină: $z = -1/1.2 = -0.833$ (în interiorul cercului unitate)
            \item Invertibilitatea necesită rădăcina în afara cercului unitate
            \item $|\theta_1| = 1.2 > 1$ $\Rightarrow$ \textcolor{Crimson}{Neinvertibilă}
        \end{itemize}

        \textbf{Corecție}: Rescrieți cu $\theta^* = 1/1.2 = 0.833$ și ajustați varianța.
    \end{exampleblock}
\end{frame}

\begin{frame}{Test 13: Regresia Falsă}
    \begin{alertblock}{Întrebare}
        Regresând un mers aleatoriu pe un alt mers aleatoriu independent, de obicei se obține:
    \end{alertblock}

    \vspace{0.3cm}

    \begin{enumerate}[A)]
        \item Nicio relație semnificativă
        \item $R^2$ ridicat și statistici t semnificative (fals)
        \item Corelație negativă
        \item Multicolinearitate perfectă
    \end{enumerate}

    \vspace{0.5cm}
    \pause
    \begin{exampleblock}{Răspuns: B -- $R^2$ ridicat și statistici t semnificative (fals)}
        \textbf{Granger \& Newbold (1974)}: Fenomenul regresiei false

        \textbf{Simptome}:
        \begin{itemize}
            \item $R^2$ ridicat (adesea $>$ 0.9) între serii neînrudite
            \item Statistici $t$ semnificative
            \item Statistică Durbin-Watson foarte scăzută ($\ll 2$)
            \item Reziduuri nestaționare
        \end{itemize}

        \textbf{Soluții}: (1) Diferențiați ambele serii, sau (2) Testați pentru cointegrare
    \end{exampleblock}
\end{frame}

\begin{frame}{Test 14: Prognoza pe Termen Lung}
    \begin{alertblock}{Întrebare}
        Prognoza pe termen lung din ARIMA(1,1,0) cu $\phi_1 = 0.7$ convergă la:
    \end{alertblock}

    \vspace{0.3cm}

    \begin{enumerate}[A)]
        \item Zero
        \item Media necondiționată
        \item O extrapolare liniară a trendului
        \item Ultima valoare observată
    \end{enumerate}

    \vspace{0.5cm}
    \pause
    \begin{exampleblock}{Răspuns: C -- O extrapolare liniară a trendului}
        \textbf{Model}: $(1-\phi_1 L)(1-L)Y_t = c + \varepsilon_t$

        \textbf{Prognoza pe termen lung}: Pentru modelele I(1) cu derivă $c$:
        \[
        \hat{Y}_{T+h} \approx Y_T + h \cdot \frac{c}{1-\phi_1}
        \]

        \textbf{Diferențe cheie}:
        \begin{itemize}
            \item ARMA staționară: Prognozele $\to$ media necondiționată
            \item I(1) fără derivă: Prognozele $\to$ ultima valoare (plată)
            \item I(1) cu derivă: Prognozele $\to$ extrapolare liniară
        \end{itemize}
    \end{exampleblock}
\end{frame}

%=============================================================================
% TRUE/FALSE QUESTIONS
%=============================================================================
\section{Întrebări Adevărat/Fals}

\begin{frame}{Întrebări Adevărat/Fals}
    \begin{alertblock}{Întrebare}
        Determinați dacă fiecare afirmație este Adevărată sau Falsă:
    \end{alertblock}

    \vspace{0.3cm}
    \begin{enumerate}
        \item Un proces I(2) necesită două diferențe pentru a deveni staționar.
        \item Testul ADF include întotdeauna un termen constant.
        \item ARIMA(0,1,0) este un alt nume pentru un mers aleatoriu.
        \item Diferențierea unei serii staționare o face ``mai staționară.''
        \item Testul KPSS are staționaritatea ca ipoteză nulă.
        \item Modelele ARIMA pot captura doar modele liniare.
    \end{enumerate}

    \vspace{0.3cm}
    \begin{center}
        \textit{Răspunsul pe slide-ul următor...}
    \end{center}
\end{frame}

\begin{frame}{Adevărat/Fals: Soluții}
    \begin{exampleblock}{Răspunsuri}
    \begin{enumerate}
        \item Un proces I(2) necesită două diferențe pentru a deveni staționar. \hfill \textcolor{Forest}{\textbf{ADEVĂRAT}}

        {\small \textcolor{MediumGray}{I($d$) înseamnă că $d$ diferențe sunt necesare. I(2) = două rădăcini unitare.}}

        \vspace{0.15cm}
        \item Testul ADF include întotdeauna un termen constant. \hfill \textcolor{Crimson}{\textbf{FALS}}

        {\small \textcolor{MediumGray}{Alegeți: fără constantă, doar constantă, sau constantă + trend.}}

        \vspace{0.15cm}
        \item ARIMA(0,1,0) este un alt nume pentru un mers aleatoriu. \hfill \textcolor{Forest}{\textbf{ADEVĂRAT}}

        {\small \textcolor{MediumGray}{$(1-L)Y_t = \varepsilon_t \Rightarrow Y_t = Y_{t-1} + \varepsilon_t$.}}

        \vspace{0.15cm}
        \item Diferențierea unei serii staționare o face ``mai staționară.'' \hfill \textcolor{Crimson}{\textbf{FALS}}

        {\small \textcolor{MediumGray}{Supradiferențierea creează MA neinvertibil; afectează performanța modelului.}}

        \vspace{0.15cm}
        \item Testul KPSS are staționaritatea ca ipoteză nulă. \hfill \textcolor{Forest}{\textbf{ADEVĂRAT}}

        {\small \textcolor{MediumGray}{KPSS: $H_0$ = staționară. Opus testului ADF.}}

        \vspace{0.15cm}
        \item Modelele ARIMA pot captura doar modele liniare. \hfill \textcolor{Forest}{\textbf{ADEVĂRAT}}

        {\small \textcolor{MediumGray}{ARIMA este liniar în parametri. Modelele neliniare necesită GARCH, rețele neuronale, etc.}}
    \end{enumerate}
    \end{exampleblock}
\end{frame}

%=============================================================================
% SECTION 2: PRACTICE PROBLEMS
%=============================================================================
\section{Probleme Practice}

\begin{frame}{Problema 1: Testarea Rădăcinii Unitare}
    \begin{block}{Exercițiu}
        Aveți date trimestriale de PIB pentru 80 de trimestre. Testul ADF (cu constantă și trend) dă o statistică de test de $-2.85$. Valoarea critică la 5\% este $-3.41$.

        \vspace{0.3cm}
        \begin{enumerate}
            \item Care este concluzia dumneavoastră despre staționaritate?
            \item Ce ați face în continuare?
        \end{enumerate}
    \end{block}

    \vspace{0.3cm}
    \pause
    \begin{exampleblock}{Soluție}
        \begin{enumerate}
            \item Deoarece $-2.85 > -3.41$, \textbf{nu respingem} $H_0$. Datele par să aibă o rădăcină unitară (nestaționare).
            \item Luați prima diferență $\Delta Y_t$ și repetați testul ADF pe seria diferențiată pentru a confirma că este acum staționară.
        \end{enumerate}
    \end{exampleblock}
\end{frame}

\begin{frame}{Problema 2: Identificarea Modelului}
    \begin{block}{Exercițiu}
        După diferențierea o dată a unei serii de timp, ACF arată:
        \begin{itemize}
            \item Vârf semnificativ la lag 1 ($\rho_1 = 0.4$)
            \item Toate celelalte lag-uri nesemnificative
        \end{itemize}
        PACF arată descreștere graduală.

        Ce model ARIMA este sugerat?
    \end{block}

    \vspace{0.3cm}
    \pause
    \begin{exampleblock}{Soluție}
        \begin{itemize}
            \item ACF se întrerupe după lag 1 $\Rightarrow$ componentă MA(1)
            \item PACF descrește $\Rightarrow$ Confirmă structura MA
            \item Deoarece am diferențiat o dată: $d = 1$
        \end{itemize}
        \textbf{Model sugerat: ARIMA(0,1,1) sau IMA(1,1)}
    \end{exampleblock}
\end{frame}

\begin{frame}{Problema 3: Ecuația ARIMA}
    \begin{block}{Exercițiu}
        Scrieți ecuația completă pentru ARIMA(1,1,1):
        $$(1-\phi_1 L)(1-L)Y_t = c + (1+\theta_1 L)\varepsilon_t$$

        Expandați complet în termenii $Y_t$, $Y_{t-1}$, $Y_{t-2}$, etc.
    \end{block}

    \vspace{0.3cm}
    \pause
    \begin{exampleblock}{Soluție}
        Expandând $(1-\phi_1 L)(1-L) = 1 - L - \phi_1 L + \phi_1 L^2 = 1 - (1+\phi_1)L + \phi_1 L^2$:

        $$Y_t - (1+\phi_1)Y_{t-1} + \phi_1 Y_{t-2} = c + \varepsilon_t + \theta_1 \varepsilon_{t-1}$$

        Sau echivalent:
        $$Y_t = c + (1+\phi_1)Y_{t-1} - \phi_1 Y_{t-2} + \varepsilon_t + \theta_1 \varepsilon_{t-1}$$
    \end{exampleblock}
\end{frame}

\begin{frame}{Problema 4: Calculul Prognozei}
    \begin{block}{Exercițiu}
        Dat ARIMA(0,1,1): $\Delta Y_t = \varepsilon_t + 0.3\varepsilon_{t-1}$

        La momentul $T$: $Y_T = 100$, $\hat{\varepsilon}_T = 2$, $\sigma^2 = 4$

        Calculați:
        \begin{enumerate}
            \item $\hat{Y}_{T+1|T}$ (prognoza la un pas)
            \item $\hat{Y}_{T+2|T}$ (prognoza la doi pași)
        \end{enumerate}
    \end{block}

    \vspace{0.3cm}
    \pause
    \begin{exampleblock}{Soluție}
        \begin{enumerate}
            \item $\hat{Y}_{T+1|T} = Y_T + 0.3\hat{\varepsilon}_T = 100 + 0.3(2) = \mathbf{100.6}$
            \item $\hat{Y}_{T+2|T} = \hat{Y}_{T+1|T} + 0.3 \cdot 0 = 100.6 + 0 = \mathbf{100.6}$

            (Șocurile viitoare $\varepsilon_{T+1}, \varepsilon_{T+2}$ sunt prognozate ca 0)
        \end{enumerate}
    \end{exampleblock}
\end{frame}

\begin{frame}{Problema 5: Intervale de Încredere}
    \begin{block}{Exercițiu}
        Continuând de la Problema 4, calculați intervalele de încredere de 95\% pentru $\hat{Y}_{T+1|T}$ și $\hat{Y}_{T+2|T}$.

        Reamintim: $\sigma^2 = 4$, $\theta_1 = 0.3$
    \end{block}

    \vspace{0.3cm}
    \pause
    \begin{exampleblock}{Soluție}
        Pentru IMA(1,1), ponderile MA($\infty$) sunt $\psi_0 = 1$, $\psi_j = 1 + \theta_1$ pentru $j \geq 1$.

        \textbf{1 pas:} $\Var(e_{T+1}) = \sigma^2 \psi_0^2 = 4$, deci $SE = 2$

        $100.6 \pm 1.96(2) = \mathbf{[96.68, 104.52]}$

        \textbf{2 pași:} $\Var(e_{T+2}) = \sigma^2(\psi_0^2 + \psi_1^2) = 4(1 + 1.3^2) = 10.76$, $SE = 3.28$

        $100.6 \pm 1.96(3.28) = \mathbf{[94.17, 107.03]}$
    \end{exampleblock}
\end{frame}

%=============================================================================
% SECTION 3: WORKED EXAMPLES
%=============================================================================
\section{Exemple Rezolvate}

\begin{frame}{Exemplu: Testarea Rădăcinii Unitare în Prețurile Acțiunilor}
    \begin{block}{Scenariu}
        Aveți prețuri de închidere zilnice pentru o acțiune pe parcursul a 500 de zile. Vreți să determinați dacă prețurile urmează un mers aleatoriu.
    \end{block}

    \vspace{0.3cm}

    \begin{exampleblock}{Abordare Pas cu Pas}
        \begin{enumerate}
            \item \textbf{Inspecție vizuală}: Reprezentați grafic prețurile -- probabil arată trend
            \item \textbf{Testul ADF pe prețuri}: Așteptați să nu respingeți $H_0$ (rădăcină unitară)
            \item \textbf{Luați randamentele logaritmice}: $r_t = \ln(P_t/P_{t-1}) = \Delta \ln(P_t)$
            \item \textbf{Testul ADF pe randamente}: Ar trebui să respingeți $H_0$ (staționară)
            \item \textbf{Concluzie}: Log prețurile sunt $I(1)$, randamentele sunt $I(0)$
        \end{enumerate}
    \end{exampleblock}
\end{frame}

\begin{frame}{Exemplu: Box-Jenkins pentru Date de Inflație}
    \begin{block}{Scenariu}
        Rate lunare ale inflației pentru 10 ani. Construiți un model ARIMA.
    \end{block}

    \begin{exampleblock}{Flux de lucru}
        \begin{enumerate}
            \item \textbf{Reprezentare grafică și test}: ADF sugerează limită -- încercați atât $d=0$ cât și $d=1$
            \item \textbf{Dacă $d=0$}: Ajustați modele ARMA, comparați AIC
            \item \textbf{Dacă $d=1$}: Examinați ACF/PACF ale lui $\Delta Y_t$
                \begin{itemize}
                    \item ACF: vârf la lag 1, apoi se întrerupe
                    \item PACF: descrește
                    \item $\Rightarrow$ Încercați ARIMA(0,1,1)
                \end{itemize}
            \item \textbf{Estimare}: Ajustați ARIMA(0,1,1), verificați coeficienții
            \item \textbf{Diagnostic}: Ljung-Box pe reziduuri (vrem $p > 0.05$)
            \item \textbf{Comparare}: AIC al ARIMA(0,1,1) vs ARMA(1,1) pe niveluri
        \end{enumerate}
    \end{exampleblock}
\end{frame}

\begin{frame}[fragile]{Exemplu: Interpretarea Rezultatelor Python}
    \begin{block}{Rezultate ARIMA din statsmodels}
        \small
        \begin{verbatim}
                            ARIMA Model Results
==============================================================
Dep. Variable:           D.y   No. Observations:    99
Model:             ARIMA(1,1,1)   AIC                 285.32
                                  BIC                 295.63
==============================================================
                 coef    std err     z     P>|z|
--------------------------------------------------------------
const          0.0521    0.048    1.085   0.278
ar.L1          0.4532    0.102    4.443   0.000
ma.L1         -0.2891    0.118   -2.450   0.014
sigma2         1.2340    0.176    7.011   0.000
        \end{verbatim}
    \end{block}

    \begin{exampleblock}{Interpretare}
        \begin{itemize}
            \item Coeficientul AR (0.45) este semnificativ, coeficientul MA (-0.29) este semnificativ
            \item Constanta (0.052) nesemnificativă -- am putea seta $c=0$
            \item Verificare: $|\phi_1| < 1$ (staționară), $|\theta_1| < 1$ (invertibilă) -- OK!
        \end{itemize}
    \end{exampleblock}
\end{frame}

%=============================================================================
% SECTION 4: REAL DATA ANALYSIS
%=============================================================================
\section{Analiză pe Date Reale}

\begin{frame}{Studiu de Caz: PIB Real SUA (1990--2024)}
    \vspace{-0.3cm}
    \begin{center}
        \includegraphics[width=0.85\textwidth, height=0.55\textheight, keepaspectratio]{charts/ch3_gdp_levels.pdf}
    \end{center}
    \vspace{-0.2cm}
    {\footnotesize
    \begin{itemize}
        \item PIB Real SUA în miliarde de dolari 2017 (date trimestriale)
        \item \textbf{Trend ascendent} clar -- tipic pentru seriile macroeconomice
        \item Scăderi notabile în timpul recesiunilor (2008-2009, 2020)
        \item Nestaționară: necesită diferențiere înainte de modelarea ARIMA
    \end{itemize}
    }
\end{frame}

\begin{frame}{Staționaritate Prin Diferențiere}
    \vspace{-0.3cm}
    \begin{center}
        \includegraphics[width=0.85\textwidth, height=0.55\textheight, keepaspectratio]{charts/ch3_differencing.pdf}
    \end{center}
    \vspace{-0.2cm}
    {\footnotesize
    \begin{itemize}
        \item \textbf{Stânga}: PIB în niveluri -- trend ascendent clar (nestaționară)
        \item \textbf{Dreapta}: Rata de creștere a PIB $= \Delta \log(Y_t) \times 100$ -- staționară
        \item Prima diferențiere a log PIB elimină trendul stochastic
        \item Rata de creștere fluctuează în jurul unei medii constante ($\approx 0.6\%$ trimestrial)
    \end{itemize}
    }
\end{frame}

\begin{frame}{ACF/PACF: Niveluri vs Diferențiate}
    \vspace{-0.3cm}
    \begin{center}
        \includegraphics[width=0.85\textwidth, height=0.55\textheight, keepaspectratio]{charts/ch3_acf_pacf.pdf}
    \end{center}
    \vspace{-0.2cm}
    {\footnotesize
    \begin{itemize}
        \item \textbf{Rândul de sus}: ACF/PACF ale nivelurilor PIB -- descreștere lentă indică nestaționaritate
        \item \textbf{Rândul de jos}: ACF/PACF ale creșterii PIB -- mai ales în limitele de încredere
        \item Modelul sugerează că un model ARIMA de ordin mic este potrivit
    \end{itemize}
    }
\end{frame}

\begin{frame}{Rezultate Estimare ARIMA: Creșterea PIB SUA}
    {\small
    \begin{block}{Model: ARIMA$(1,1,1)$ pe $\log(\text{PIB})$}
        \begin{center}
        \begin{tabular}{lcccc}
            \toprule
            \textbf{Parametru} & \textbf{Estimat} & \textbf{Eroare Std.} & \textbf{z-stat} & \textbf{valoare-p} \\
            \midrule
            $\phi_1$ (AR.L1) & $0.312$ & $0.185$ & $1.69$ & $0.091$ \\
            $\theta_1$ (MA.L1) & $-0.087$ & $0.203$ & $-0.43$ & $0.668$ \\
            $\sigma^2$ & $0.00012$ & -- & -- & -- \\
            \bottomrule
        \end{tabular}
        \end{center}
    \end{block}

    \vspace{0.2cm}

    \begin{exampleblock}{Interpretare}
        \begin{itemize}
            \item ARIMA de ordin mic captează rezonabil dinamica PIB
            \item Coeficientul AR(1) pozitiv -- creșterea PIB arată persistență
            \item Alternativă: mersul aleatoriu simplu (ARIMA(0,1,0)) adesea competitiv
        \end{itemize}
    \end{exampleblock}
    }
\end{frame}

\begin{frame}{Prognoză: ARIMA vs Real}
    \vspace{-0.3cm}
    \begin{center}
        \includegraphics[width=0.85\textwidth, height=0.55\textheight, keepaspectratio]{charts/ch3_arima_forecast.pdf}
    \end{center}
    \vspace{-0.2cm}
    {\footnotesize
    \begin{itemize}
        \item Albastru: date istorice de antrenare; Verde: date reale de test
        \item Roșu întrerupt: prognoze ARIMA cu interval de încredere 95\%
        \item Prognozele captează direcția generală a trendului
        \item Intervalele de încredere se lărgesc pe măsură ce orizontul de prognoză crește
    \end{itemize}
    }
\end{frame}

\begin{frame}{Diagnostice Model: Analiza Reziduurilor}
    \vspace{-0.3cm}
    \begin{center}
        \includegraphics[width=0.85\textwidth, height=0.55\textheight, keepaspectratio]{charts/ch3_diagnostics.pdf}
    \end{center}
    \vspace{-0.2cm}
    {\footnotesize
    \begin{itemize}
        \item Reziduurile nu arată modele sistematice în timp
        \item Distribuție aproximativ normală (histogramă și grafic Q-Q)
        \item ACF-ul reziduurilor în limite -- fără autocorelație semnificativă rămasă
        \item Modelul captează adecvat procesul generator de date
    \end{itemize}
    }
\end{frame}

%=============================================================================
% SECTION 5: DISCUSSION TOPICS
%=============================================================================
\section{Subiecte de Discuție}

\begin{frame}{Discuție: Trenduri Deterministe vs Stochastice}
    \begin{block}{Întrebare Cheie}
        De ce este important să distingem între trendurile deterministe și stochastice?
    \end{block}

    \vspace{0.3cm}

    \begin{block}{Puncte de Discuție}
        \begin{itemize}
            \item \textbf{Consecințele tratamentului greșit}:
                \begin{itemize}
                    \item Eliminarea trendului prin regresie la o rădăcină unitară $\Rightarrow$ staționaritate falsă
                    \item Diferențierea unei serii staționare în trend $\Rightarrow$ supradiferențiere
                \end{itemize}
            \item \textbf{Interpretare economică}:
                \begin{itemize}
                    \item Trend determinist: șocurile sunt temporare
                    \item Trend stochastic: șocurile au efecte permanente
                \end{itemize}
            \item \textbf{Implicații de politică}:
                \begin{itemize}
                    \item O recesiune reduce permanent PIB-ul, sau economia revine la trend?
                \end{itemize}
        \end{itemize}
    \end{block}
\end{frame}

\begin{frame}{Discuție: Criterii de Selecție a Modelului}
    \begin{block}{Întrebare Cheie}
        Când ar trebui să folosiți AIC vs BIC pentru selecția modelului ARIMA?
    \end{block}

    \vspace{0.3cm}

    \begin{block}{Considerații}
        \begin{itemize}
            \item \textbf{AIC}: Minimizează eroarea de predicție, poate supraajusta
                \begin{itemize}
                    \item Mai bun pentru prognoză
                    \item Tinde să selecteze modele mai mari
                \end{itemize}
            \item \textbf{BIC}: Selecție consistentă a modelului, mai simplu
                \begin{itemize}
                    \item Mai bun pentru identificarea modelului ``adevărat''
                    \item Penalizează complexitatea mai puternic
                \end{itemize}
            \item \textbf{Sfat practic}: Raportați ambele, preferați BIC dacă diferă substanțial
        \end{itemize}
    \end{block}
\end{frame}

\begin{frame}{Discuție: Limitările ARIMA}
    \begin{block}{Întrebare Cheie}
        Care sunt principalele limitări ale modelelor ARIMA?
    \end{block}

    \vspace{0.3cm}

    \begin{block}{Puncte de Discuție}
        \begin{itemize}
            \item \textbf{Liniaritate}: Nu poate captura dinamici neliniare
            \item \textbf{Varianță constantă}: Presupune homoscedasticitate (fără efecte GARCH)
            \item \textbf{Fără rupturi structurale}: Parametrii presupuși constanți
            \item \textbf{Univariat}: Ignoră relațiile cu alte variabile
            \item \textbf{Simetric}: Tratează șocurile pozitive și negative la fel
            \item \textbf{Prognoze pe termen lung}: Incertitudinea crește rapid
        \end{itemize}
    \end{block}

    \vspace{0.3cm}

    \begin{alertblock}{Extensii}
        Aceste limitări motivează GARCH (volatilitate), VAR (multivariat), modele cu schimbări de regim, etc.
    \end{alertblock}
\end{frame}

%=============================================================================
% SECTION 5: SUMMARY
%=============================================================================
\section{Rezumat}

\begin{frame}{Puncte Cheie din Seminarul de Astăzi}
    \begin{block}{Ce am Acoperit}
        \begin{enumerate}
            \item \textbf{Integrare și diferențiere}: Procesele $I(d)$ necesită $d$ diferențe
            \item \textbf{Testarea rădăcinii unitare}: ADF testează $H_0$: rădăcină unitară; KPSS testează $H_0$: staționară
            \item \textbf{ARIMA(p,d,q)}: Combină ARMA cu diferențierea
            \item \textbf{Identificarea modelului}: Folosiți modelele ACF/PACF și criteriile informaționale
            \item \textbf{Prognoză}: Prognoze punctuale și intervale de încredere în creștere
        \end{enumerate}
    \end{block}

    \vspace{0.3cm}

    \begin{exampleblock}{Următorul Seminar}
        Exerciții practice Python cu date economice reale:
        \begin{itemize}
            \item Testarea rădăcinii unitare cu \texttt{statsmodels}
            \item Auto-ARIMA cu \texttt{pmdarima}
            \item Prognoză și diagnostice model
        \end{itemize}
    \end{exampleblock}
\end{frame}

\end{document}
