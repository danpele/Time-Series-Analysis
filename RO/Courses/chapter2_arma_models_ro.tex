% Capitolul 2: Modele ARMA
% Prezentare academică de calitate Harvard
% Program de licență, Academia de Studii Economice din București

\documentclass[9pt, aspectratio=169, t]{beamer}
%=============================================================================
% SHARED PREAMBLE - Time Series Analysis and Forecasting
% Harvard-quality academic presentations
% Bachelor program, Bucharest University of Economic Studies
%
% Usage: \documentclass[9pt, aspectratio=169, t]{beamer}
%            %=============================================================================
% SHARED PREAMBLE - Time Series Analysis and Forecasting
% Harvard-quality academic presentations
% Bachelor program, Bucharest University of Economic Studies
%
% Usage: \documentclass[9pt, aspectratio=169, t]{beamer}
%            %=============================================================================
% SHARED PREAMBLE - Time Series Analysis and Forecasting
% Harvard-quality academic presentations
% Bachelor program, Bucharest University of Economic Studies
%
% Usage: \documentclass[9pt, aspectratio=169, t]{beamer}
%            \input{preamble}
%            \subtitle{Seminar X: Seminar Title}
%            \begin{document} ...
%=============================================================================

% Ensure content fits on slides
\setbeamersize{text margin left=8mm, text margin right=8mm}

%=============================================================================
% THEME AND STYLE CONFIGURATION
%=============================================================================
\usetheme{default}
% Using default theme for clean header/footer control

% Color Palette (matching Redispatch PDF)
\definecolor{MainBlue}{RGB}{26, 58, 110}
\definecolor{AccentBlue}{RGB}{26, 58, 110}
\definecolor{IDAred}{RGB}{205, 0, 0}
\definecolor{DarkGray}{RGB}{51, 51, 51}
\definecolor{MediumGray}{RGB}{128, 128, 128}
\definecolor{LightGray}{RGB}{248, 248, 248}
\definecolor{VeryLightGray}{RGB}{235, 235, 235}
\definecolor{KeynoteGray}{RGB}{218, 218, 218}
\definecolor{SectionGray}{RGB}{120, 120, 120}
\definecolor{FooterGray}{RGB}{100, 100, 100}
\definecolor{Crimson}{RGB}{220, 53, 69}
\definecolor{Forest}{RGB}{46, 125, 50}
\definecolor{Amber}{RGB}{181, 133, 63}
\definecolor{Orange}{RGB}{230, 126, 34}
\definecolor{Purple}{RGB}{142, 68, 173}

% Gradient background (exact Keynote 315° gradient: white to RGB 218,218,218)
\setbeamertemplate{background}{%
    \begin{tikzpicture}[remember picture, overlay]
        \shade[shading=axis, shading angle=315,
        top color=white, bottom color=KeynoteGray]
        (current page.south west) rectangle (current page.north east);
    \end{tikzpicture}%
}
% Fallback solid color for compatibility
\setbeamercolor{background canvas}{bg=}

\setbeamercolor{palette primary}{bg=MainBlue, fg=white}
\setbeamercolor{palette secondary}{bg=MainBlue!85, fg=white}
\setbeamercolor{palette tertiary}{bg=MainBlue!70, fg=white}
\setbeamercolor{structure}{fg=MainBlue}
\setbeamercolor{title}{fg=IDAred}
\setbeamercolor{frametitle}{fg=IDAred, bg=}
\setbeamercolor{block title}{bg=MainBlue, fg=white}
\setbeamercolor{block body}{bg=VeryLightGray, fg=DarkGray}
\setbeamercolor{block title alerted}{bg=Crimson, fg=white}
\setbeamercolor{block body alerted}{bg=Crimson!8, fg=DarkGray}
\setbeamercolor{block title example}{bg=Forest, fg=white}
\setbeamercolor{block body example}{bg=Forest!8, fg=DarkGray}
\setbeamercolor{item}{fg=MainBlue}

% Smaller institute font to avoid overfull hbox on title page
\setbeamerfont{institute}{size=\footnotesize}

% Footer colors (override Madrid theme blue)
\setbeamercolor{author in head/foot}{fg=FooterGray, bg=}
\setbeamercolor{title in head/foot}{fg=FooterGray, bg=}
\setbeamercolor{date in head/foot}{fg=FooterGray, bg=}
\setbeamercolor{section in head/foot}{fg=FooterGray, bg=}
\setbeamercolor{subsection in head/foot}{fg=FooterGray, bg=}

% Bullet styles (apply everywhere including blocks)
\setbeamertemplate{itemize item}{\color{MainBlue}$\boxdot$}
\setbeamertemplate{itemize subitem}{\color{MainBlue}$\blacktriangleright$}
\setbeamertemplate{itemize subsubitem}{\color{MainBlue}\tiny$\bullet$}
\setbeamertemplate{itemize/enumerate body begin}{\normalsize}
\setbeamertemplate{itemize/enumerate subbody begin}{\normalsize}

% Item spacing - compact style
\setlength{\leftmargini}{10pt}       % Level 1: minimal indent
\setlength{\leftmarginii}{10pt}      % Level 2: minimal additional indent
% Compact list spacing (zero extra space before/after lists in blocks)
\makeatletter
\def\@listi{\leftmargin\leftmargini \topsep 0pt \parsep 0pt \itemsep 0pt}
\def\@listii{\leftmargin\leftmarginii \topsep 0pt \parsep 0pt \itemsep 0pt}
\makeatother

\setbeamertemplate{navigation symbols}{}

%=============================================================================
% CUSTOM HEADLINE
%=============================================================================
\setbeamertemplate{headline}{%
    \vskip10pt%
    \hbox to \paperwidth{%
        \hskip0.5cm%
        {\small\color{FooterGray}\renewcommand{\hyperlink}[2]{##2}\insertsectionhead}%
        \hfill%
        \textcolor{FooterGray}{\small\insertframenumber}%
        \hskip0.5cm%
    }%
    \vskip4pt%
    {\color{FooterGray}\hrule height 0.4pt}%
}

%=============================================================================
% CUSTOM FOOTER
%=============================================================================
\usepackage{fontawesome5}

\setbeamertemplate{footline}{%
    {\color{FooterGray}\hrule height 0.4pt}%
    \vskip4pt%
    \hbox to \paperwidth{%
        \hskip0.5cm%
        \textcolor{FooterGray}{\small Time Series Analysis and Forecasting}%
        \hfill%
        \raisebox{-0.1em}{%
            \begin{tikzpicture}[x=0.08em, y=0.08em, line width=0.4pt]
                \draw[FooterGray] (0,3) -- (1,4) -- (2,3.5) -- (3,5) -- (4,4) -- (5,6) -- (6,5.5) -- (7,4) -- (8,5) -- (9,7) -- (10,6) -- (11,5) -- (12,6.5) -- (13,8) -- (14,7) -- (15,6) -- (16,7.5) -- (17,9) -- (18,8) -- (19,7) -- (20,8.5) -- (21,10) -- (22,9) -- (23,8) -- (24,9.5);
            \end{tikzpicture}%
        }%
        \hskip0.5cm%
    }%
    \vskip6pt%
}

%=============================================================================
% PACKAGES
%=============================================================================
\usepackage[utf8]{inputenc}
\usepackage[T1]{fontenc}
\usepackage[english]{babel}
\usepackage{amsmath, amssymb, amsthm}
\usepackage{mathtools}
\usepackage{bm}
\usepackage{tikz}
\usetikzlibrary{arrows.meta, positioning, shapes, calc, decorations.pathreplacing, shadings}
\usepackage{booktabs}
\usepackage{multirow}
\usepackage{array}
\usepackage{graphicx}
\usepackage{hyperref}
\usepackage{colortbl}
\usepackage{listings}
\lstset{basicstyle=\ttfamily\small, breaklines=true, frame=single, backgroundcolor=\color{VeryLightGray}}
\hypersetup{colorlinks=true, linkcolor=MainBlue, urlcolor=MainBlue}
\graphicspath{{../../logos/}{../../charts/}{../../photos/}}
\hfuzz=2pt  % Suppress tiny overfull warnings (<2pt)
\vfuzz=2pt  % Suppress tiny vertical overfull warnings (<2pt)

%=============================================================================
% QUANTLET COMMAND
%=============================================================================
\newcommand{\quantlet}[2]{%
    \hfill\href{#2}{%
        \raisebox{-0.15em}{\includegraphics[height=0.7em]{ql_logo.png}}%
        \textcolor{MainBlue}{\tiny\ #1}%
    }%
}

%=============================================================================
% CUSTOM TITLE PAGE
%=============================================================================
\defbeamertemplate*{title page}{hybrid}[1][]
{
    \vspace{0.2cm}
    % Logos row - top header (with clickable links)
    \begin{center}
        \href{https://www.ase.ro}{\includegraphics[height=1.0cm]{ase_logo.png}}\hspace{0.25cm}%
        \href{https://theida.net}{\includegraphics[height=1.0cm]{ida_logo.png}}\hspace{0.25cm}%
        \href{https://blockchain-research-center.com}{\includegraphics[height=1.0cm]{brc_logo.png}}\hspace{0.25cm}%
        \href{https://www.ai4efin.ase.ro}{\includegraphics[height=1.0cm]{ai4efin_logo.png}}\hspace{0.25cm}%
        \href{https://ipe.ro/new}{\includegraphics[height=1.0cm]{acad_logo.png}}\hspace{0.25cm}%
        \href{https://www.digital-finance-msca.com}{\includegraphics[height=1.0cm]{msca_logo.png}}%
    \end{center}

    \vspace{0.6cm}

    % Main title with Q logos on sides (with clickable links)
    \begin{center}
        \begin{minipage}{0.1\textwidth}
            \centering
            \href{https://quantlet.com}{\includegraphics[height=1.1cm]{ql_logo.png}}
        \end{minipage}%
        \begin{minipage}{0.78\textwidth}
            \centering
            {\LARGE\bfseries\usebeamercolor[fg]{title}\inserttitle}

            \vspace{0.3cm}

            {\usebeamerfont{subtitle}\usebeamercolor[fg]{title}\insertsubtitle}
        \end{minipage}%
        \begin{minipage}{0.1\textwidth}
            \centering
            \href{https://quantinar.com}{\includegraphics[height=1.1cm]{qr_logo.png}}
        \end{minipage}
    \end{center}

    \vspace{0.6cm}

    % Authors (left aligned)
    \hspace{0.5cm}{\usebeamerfont{author}\insertauthor}

    \vspace{0.3cm}

    % Institute/Affiliations (left aligned)
    \hspace{0.5cm}\begin{minipage}[t]{0.9\textwidth}
        \raggedright\small\insertinstitute
    \end{minipage}
}

%=============================================================================
% THEOREM ENVIRONMENTS
%=============================================================================
\theoremstyle{definition}
\setbeamertemplate{theorems}[numbered]
\newtheorem{defn}{Definition}
\newtheorem{thm}{Theorem}
\newtheorem{prop}{Proposition}
\newtheorem{rmk}{Remark}

%=============================================================================
% CENTRED MINIPAGE (no extra vertical space)
%=============================================================================
\newenvironment{cminipage}[1]{%
    \par\noindent\hfill\begin{minipage}{#1}\ignorespaces
}{%
    \end{minipage}\hfill\null\par
}

%=============================================================================
% CUSTOM COMMANDS
%=============================================================================
\newcommand{\E}{\mathbb{E}}
\newcommand{\Var}{\text{Var}}
\newcommand{\Cov}{\text{Cov}}
\newcommand{\Corr}{\text{Corr}}
\newcommand{\R}{\mathbb{R}}
\newcommand{\N}{\mathbb{N}}
\newcommand{\Z}{\mathbb{Z}}
\newcommand{\B}{\mathbf{B}}
\newcommand{\imark}{\textcolor{MainBlue}{\textbullet}}
\newcommand{\RMSE}{\text{RMSE}}
\newcommand{\MAE}{\text{MAE}}
\newcommand{\MAPE}{\text{MAPE}}
\newcommand{\correct}{\textcolor{Forest}{\checkmark}}
\newcommand{\incorrect}{\textcolor{Crimson}{\texttimes}}

% Boldface vector/matrix commands
\newcommand{\bY}{\mathbf{Y}}
\newcommand{\bX}{\mathbf{X}}
\newcommand{\bA}{\mathbf{A}}
\newcommand{\bB}{\mathbf{B}}
\newcommand{\bepsilon}{\boldsymbol{\varepsilon}}
\newcommand{\bvarepsilon}{\boldsymbol{\varepsilon}}
\newcommand{\bSigma}{\boldsymbol{\Sigma}}
\newcommand{\bPhi}{\boldsymbol{\Phi}}
\newcommand{\bGamma}{\boldsymbol{\Gamma}}
\newcommand{\bPi}{\boldsymbol{\Pi}}
\newcommand{\bc}{\mathbf{c}}
\newcommand{\balpha}{\boldsymbol{\alpha}}
\newcommand{\bbeta}{\boldsymbol{\beta}}

%=============================================================================
% TITLE INFORMATION
%=============================================================================
\title[Time Series Analysis]{Time Series Analysis and Forecasting}
\author[D.T. Pele]{Daniel Traian PELE}
\institute{Bucharest University of Economic Studies\\
IDA Institute Digital Assets\\
Blockchain Research Center\\
AI4EFin Artificial Intelligence for Energy Finance\\
Romanian Academy, Institute for Economic Forecasting\\
MSCA Digital Finance}
\date{}

%            \subtitle{Seminar X: Seminar Title}
%            \begin{document} ...
%=============================================================================

% Ensure content fits on slides
\setbeamersize{text margin left=8mm, text margin right=8mm}

%=============================================================================
% THEME AND STYLE CONFIGURATION
%=============================================================================
\usetheme{default}
% Using default theme for clean header/footer control

% Color Palette (matching Redispatch PDF)
\definecolor{MainBlue}{RGB}{26, 58, 110}
\definecolor{AccentBlue}{RGB}{26, 58, 110}
\definecolor{IDAred}{RGB}{205, 0, 0}
\definecolor{DarkGray}{RGB}{51, 51, 51}
\definecolor{MediumGray}{RGB}{128, 128, 128}
\definecolor{LightGray}{RGB}{248, 248, 248}
\definecolor{VeryLightGray}{RGB}{235, 235, 235}
\definecolor{KeynoteGray}{RGB}{218, 218, 218}
\definecolor{SectionGray}{RGB}{120, 120, 120}
\definecolor{FooterGray}{RGB}{100, 100, 100}
\definecolor{Crimson}{RGB}{220, 53, 69}
\definecolor{Forest}{RGB}{46, 125, 50}
\definecolor{Amber}{RGB}{181, 133, 63}
\definecolor{Orange}{RGB}{230, 126, 34}
\definecolor{Purple}{RGB}{142, 68, 173}

% Gradient background (exact Keynote 315° gradient: white to RGB 218,218,218)
\setbeamertemplate{background}{%
    \begin{tikzpicture}[remember picture, overlay]
        \shade[shading=axis, shading angle=315,
        top color=white, bottom color=KeynoteGray]
        (current page.south west) rectangle (current page.north east);
    \end{tikzpicture}%
}
% Fallback solid color for compatibility
\setbeamercolor{background canvas}{bg=}

\setbeamercolor{palette primary}{bg=MainBlue, fg=white}
\setbeamercolor{palette secondary}{bg=MainBlue!85, fg=white}
\setbeamercolor{palette tertiary}{bg=MainBlue!70, fg=white}
\setbeamercolor{structure}{fg=MainBlue}
\setbeamercolor{title}{fg=IDAred}
\setbeamercolor{frametitle}{fg=IDAred, bg=}
\setbeamercolor{block title}{bg=MainBlue, fg=white}
\setbeamercolor{block body}{bg=VeryLightGray, fg=DarkGray}
\setbeamercolor{block title alerted}{bg=Crimson, fg=white}
\setbeamercolor{block body alerted}{bg=Crimson!8, fg=DarkGray}
\setbeamercolor{block title example}{bg=Forest, fg=white}
\setbeamercolor{block body example}{bg=Forest!8, fg=DarkGray}
\setbeamercolor{item}{fg=MainBlue}

% Smaller institute font to avoid overfull hbox on title page
\setbeamerfont{institute}{size=\footnotesize}

% Footer colors (override Madrid theme blue)
\setbeamercolor{author in head/foot}{fg=FooterGray, bg=}
\setbeamercolor{title in head/foot}{fg=FooterGray, bg=}
\setbeamercolor{date in head/foot}{fg=FooterGray, bg=}
\setbeamercolor{section in head/foot}{fg=FooterGray, bg=}
\setbeamercolor{subsection in head/foot}{fg=FooterGray, bg=}

% Bullet styles (apply everywhere including blocks)
\setbeamertemplate{itemize item}{\color{MainBlue}$\boxdot$}
\setbeamertemplate{itemize subitem}{\color{MainBlue}$\blacktriangleright$}
\setbeamertemplate{itemize subsubitem}{\color{MainBlue}\tiny$\bullet$}
\setbeamertemplate{itemize/enumerate body begin}{\normalsize}
\setbeamertemplate{itemize/enumerate subbody begin}{\normalsize}

% Item spacing - compact style
\setlength{\leftmargini}{10pt}       % Level 1: minimal indent
\setlength{\leftmarginii}{10pt}      % Level 2: minimal additional indent
% Compact list spacing (zero extra space before/after lists in blocks)
\makeatletter
\def\@listi{\leftmargin\leftmargini \topsep 0pt \parsep 0pt \itemsep 0pt}
\def\@listii{\leftmargin\leftmarginii \topsep 0pt \parsep 0pt \itemsep 0pt}
\makeatother

\setbeamertemplate{navigation symbols}{}

%=============================================================================
% CUSTOM HEADLINE
%=============================================================================
\setbeamertemplate{headline}{%
    \vskip10pt%
    \hbox to \paperwidth{%
        \hskip0.5cm%
        {\small\color{FooterGray}\renewcommand{\hyperlink}[2]{##2}\insertsectionhead}%
        \hfill%
        \textcolor{FooterGray}{\small\insertframenumber}%
        \hskip0.5cm%
    }%
    \vskip4pt%
    {\color{FooterGray}\hrule height 0.4pt}%
}

%=============================================================================
% CUSTOM FOOTER
%=============================================================================
\usepackage{fontawesome5}

\setbeamertemplate{footline}{%
    {\color{FooterGray}\hrule height 0.4pt}%
    \vskip4pt%
    \hbox to \paperwidth{%
        \hskip0.5cm%
        \textcolor{FooterGray}{\small Time Series Analysis and Forecasting}%
        \hfill%
        \raisebox{-0.1em}{%
            \begin{tikzpicture}[x=0.08em, y=0.08em, line width=0.4pt]
                \draw[FooterGray] (0,3) -- (1,4) -- (2,3.5) -- (3,5) -- (4,4) -- (5,6) -- (6,5.5) -- (7,4) -- (8,5) -- (9,7) -- (10,6) -- (11,5) -- (12,6.5) -- (13,8) -- (14,7) -- (15,6) -- (16,7.5) -- (17,9) -- (18,8) -- (19,7) -- (20,8.5) -- (21,10) -- (22,9) -- (23,8) -- (24,9.5);
            \end{tikzpicture}%
        }%
        \hskip0.5cm%
    }%
    \vskip6pt%
}

%=============================================================================
% PACKAGES
%=============================================================================
\usepackage[utf8]{inputenc}
\usepackage[T1]{fontenc}
\usepackage[english]{babel}
\usepackage{amsmath, amssymb, amsthm}
\usepackage{mathtools}
\usepackage{bm}
\usepackage{tikz}
\usetikzlibrary{arrows.meta, positioning, shapes, calc, decorations.pathreplacing, shadings}
\usepackage{booktabs}
\usepackage{multirow}
\usepackage{array}
\usepackage{graphicx}
\usepackage{hyperref}
\usepackage{colortbl}
\usepackage{listings}
\lstset{basicstyle=\ttfamily\small, breaklines=true, frame=single, backgroundcolor=\color{VeryLightGray}}
\hypersetup{colorlinks=true, linkcolor=MainBlue, urlcolor=MainBlue}
\graphicspath{{../../logos/}{../../charts/}{../../photos/}}
\hfuzz=2pt  % Suppress tiny overfull warnings (<2pt)
\vfuzz=2pt  % Suppress tiny vertical overfull warnings (<2pt)

%=============================================================================
% QUANTLET COMMAND
%=============================================================================
\newcommand{\quantlet}[2]{%
    \hfill\href{#2}{%
        \raisebox{-0.15em}{\includegraphics[height=0.7em]{ql_logo.png}}%
        \textcolor{MainBlue}{\tiny\ #1}%
    }%
}

%=============================================================================
% CUSTOM TITLE PAGE
%=============================================================================
\defbeamertemplate*{title page}{hybrid}[1][]
{
    \vspace{0.2cm}
    % Logos row - top header (with clickable links)
    \begin{center}
        \href{https://www.ase.ro}{\includegraphics[height=1.0cm]{ase_logo.png}}\hspace{0.25cm}%
        \href{https://theida.net}{\includegraphics[height=1.0cm]{ida_logo.png}}\hspace{0.25cm}%
        \href{https://blockchain-research-center.com}{\includegraphics[height=1.0cm]{brc_logo.png}}\hspace{0.25cm}%
        \href{https://www.ai4efin.ase.ro}{\includegraphics[height=1.0cm]{ai4efin_logo.png}}\hspace{0.25cm}%
        \href{https://ipe.ro/new}{\includegraphics[height=1.0cm]{acad_logo.png}}\hspace{0.25cm}%
        \href{https://www.digital-finance-msca.com}{\includegraphics[height=1.0cm]{msca_logo.png}}%
    \end{center}

    \vspace{0.6cm}

    % Main title with Q logos on sides (with clickable links)
    \begin{center}
        \begin{minipage}{0.1\textwidth}
            \centering
            \href{https://quantlet.com}{\includegraphics[height=1.1cm]{ql_logo.png}}
        \end{minipage}%
        \begin{minipage}{0.78\textwidth}
            \centering
            {\LARGE\bfseries\usebeamercolor[fg]{title}\inserttitle}

            \vspace{0.3cm}

            {\usebeamerfont{subtitle}\usebeamercolor[fg]{title}\insertsubtitle}
        \end{minipage}%
        \begin{minipage}{0.1\textwidth}
            \centering
            \href{https://quantinar.com}{\includegraphics[height=1.1cm]{qr_logo.png}}
        \end{minipage}
    \end{center}

    \vspace{0.6cm}

    % Authors (left aligned)
    \hspace{0.5cm}{\usebeamerfont{author}\insertauthor}

    \vspace{0.3cm}

    % Institute/Affiliations (left aligned)
    \hspace{0.5cm}\begin{minipage}[t]{0.9\textwidth}
        \raggedright\small\insertinstitute
    \end{minipage}
}

%=============================================================================
% THEOREM ENVIRONMENTS
%=============================================================================
\theoremstyle{definition}
\setbeamertemplate{theorems}[numbered]
\newtheorem{defn}{Definition}
\newtheorem{thm}{Theorem}
\newtheorem{prop}{Proposition}
\newtheorem{rmk}{Remark}

%=============================================================================
% CENTRED MINIPAGE (no extra vertical space)
%=============================================================================
\newenvironment{cminipage}[1]{%
    \par\noindent\hfill\begin{minipage}{#1}\ignorespaces
}{%
    \end{minipage}\hfill\null\par
}

%=============================================================================
% CUSTOM COMMANDS
%=============================================================================
\newcommand{\E}{\mathbb{E}}
\newcommand{\Var}{\text{Var}}
\newcommand{\Cov}{\text{Cov}}
\newcommand{\Corr}{\text{Corr}}
\newcommand{\R}{\mathbb{R}}
\newcommand{\N}{\mathbb{N}}
\newcommand{\Z}{\mathbb{Z}}
\newcommand{\B}{\mathbf{B}}
\newcommand{\imark}{\textcolor{MainBlue}{\textbullet}}
\newcommand{\RMSE}{\text{RMSE}}
\newcommand{\MAE}{\text{MAE}}
\newcommand{\MAPE}{\text{MAPE}}
\newcommand{\correct}{\textcolor{Forest}{\checkmark}}
\newcommand{\incorrect}{\textcolor{Crimson}{\texttimes}}

% Boldface vector/matrix commands
\newcommand{\bY}{\mathbf{Y}}
\newcommand{\bX}{\mathbf{X}}
\newcommand{\bA}{\mathbf{A}}
\newcommand{\bB}{\mathbf{B}}
\newcommand{\bepsilon}{\boldsymbol{\varepsilon}}
\newcommand{\bvarepsilon}{\boldsymbol{\varepsilon}}
\newcommand{\bSigma}{\boldsymbol{\Sigma}}
\newcommand{\bPhi}{\boldsymbol{\Phi}}
\newcommand{\bGamma}{\boldsymbol{\Gamma}}
\newcommand{\bPi}{\boldsymbol{\Pi}}
\newcommand{\bc}{\mathbf{c}}
\newcommand{\balpha}{\boldsymbol{\alpha}}
\newcommand{\bbeta}{\boldsymbol{\beta}}

%=============================================================================
% TITLE INFORMATION
%=============================================================================
\title[Time Series Analysis]{Time Series Analysis and Forecasting}
\author[D.T. Pele]{Daniel Traian PELE}
\institute{Bucharest University of Economic Studies\\
IDA Institute Digital Assets\\
Blockchain Research Center\\
AI4EFin Artificial Intelligence for Energy Finance\\
Romanian Academy, Institute for Economic Forecasting\\
MSCA Digital Finance}
\date{}

%            \subtitle{Seminar X: Seminar Title}
%            \begin{document} ...
%=============================================================================

% Ensure content fits on slides
\setbeamersize{text margin left=8mm, text margin right=8mm}

%=============================================================================
% THEME AND STYLE CONFIGURATION
%=============================================================================
\usetheme{default}
% Using default theme for clean header/footer control

% Color Palette (matching Redispatch PDF)
\definecolor{MainBlue}{RGB}{26, 58, 110}
\definecolor{AccentBlue}{RGB}{26, 58, 110}
\definecolor{IDAred}{RGB}{205, 0, 0}
\definecolor{DarkGray}{RGB}{51, 51, 51}
\definecolor{MediumGray}{RGB}{128, 128, 128}
\definecolor{LightGray}{RGB}{248, 248, 248}
\definecolor{VeryLightGray}{RGB}{235, 235, 235}
\definecolor{KeynoteGray}{RGB}{218, 218, 218}
\definecolor{SectionGray}{RGB}{120, 120, 120}
\definecolor{FooterGray}{RGB}{100, 100, 100}
\definecolor{Crimson}{RGB}{220, 53, 69}
\definecolor{Forest}{RGB}{46, 125, 50}
\definecolor{Amber}{RGB}{181, 133, 63}
\definecolor{Orange}{RGB}{230, 126, 34}
\definecolor{Purple}{RGB}{142, 68, 173}

% Gradient background (exact Keynote 315° gradient: white to RGB 218,218,218)
\setbeamertemplate{background}{%
    \begin{tikzpicture}[remember picture, overlay]
        \shade[shading=axis, shading angle=315,
        top color=white, bottom color=KeynoteGray]
        (current page.south west) rectangle (current page.north east);
    \end{tikzpicture}%
}
% Fallback solid color for compatibility
\setbeamercolor{background canvas}{bg=}

\setbeamercolor{palette primary}{bg=MainBlue, fg=white}
\setbeamercolor{palette secondary}{bg=MainBlue!85, fg=white}
\setbeamercolor{palette tertiary}{bg=MainBlue!70, fg=white}
\setbeamercolor{structure}{fg=MainBlue}
\setbeamercolor{title}{fg=IDAred}
\setbeamercolor{frametitle}{fg=IDAred, bg=}
\setbeamercolor{block title}{bg=MainBlue, fg=white}
\setbeamercolor{block body}{bg=VeryLightGray, fg=DarkGray}
\setbeamercolor{block title alerted}{bg=Crimson, fg=white}
\setbeamercolor{block body alerted}{bg=Crimson!8, fg=DarkGray}
\setbeamercolor{block title example}{bg=Forest, fg=white}
\setbeamercolor{block body example}{bg=Forest!8, fg=DarkGray}
\setbeamercolor{item}{fg=MainBlue}

% Smaller institute font to avoid overfull hbox on title page
\setbeamerfont{institute}{size=\footnotesize}

% Footer colors (override Madrid theme blue)
\setbeamercolor{author in head/foot}{fg=FooterGray, bg=}
\setbeamercolor{title in head/foot}{fg=FooterGray, bg=}
\setbeamercolor{date in head/foot}{fg=FooterGray, bg=}
\setbeamercolor{section in head/foot}{fg=FooterGray, bg=}
\setbeamercolor{subsection in head/foot}{fg=FooterGray, bg=}

% Bullet styles (apply everywhere including blocks)
\setbeamertemplate{itemize item}{\color{MainBlue}$\boxdot$}
\setbeamertemplate{itemize subitem}{\color{MainBlue}$\blacktriangleright$}
\setbeamertemplate{itemize subsubitem}{\color{MainBlue}\tiny$\bullet$}
\setbeamertemplate{itemize/enumerate body begin}{\normalsize}
\setbeamertemplate{itemize/enumerate subbody begin}{\normalsize}

% Item spacing - compact style
\setlength{\leftmargini}{10pt}       % Level 1: minimal indent
\setlength{\leftmarginii}{10pt}      % Level 2: minimal additional indent
% Compact list spacing (zero extra space before/after lists in blocks)
\makeatletter
\def\@listi{\leftmargin\leftmargini \topsep 0pt \parsep 0pt \itemsep 0pt}
\def\@listii{\leftmargin\leftmarginii \topsep 0pt \parsep 0pt \itemsep 0pt}
\makeatother

\setbeamertemplate{navigation symbols}{}

%=============================================================================
% CUSTOM HEADLINE
%=============================================================================
\setbeamertemplate{headline}{%
    \vskip10pt%
    \hbox to \paperwidth{%
        \hskip0.5cm%
        {\small\color{FooterGray}\renewcommand{\hyperlink}[2]{##2}\insertsectionhead}%
        \hfill%
        \textcolor{FooterGray}{\small\insertframenumber}%
        \hskip0.5cm%
    }%
    \vskip4pt%
    {\color{FooterGray}\hrule height 0.4pt}%
}

%=============================================================================
% CUSTOM FOOTER
%=============================================================================
\usepackage{fontawesome5}

\setbeamertemplate{footline}{%
    {\color{FooterGray}\hrule height 0.4pt}%
    \vskip4pt%
    \hbox to \paperwidth{%
        \hskip0.5cm%
        \textcolor{FooterGray}{\small Time Series Analysis and Forecasting}%
        \hfill%
        \raisebox{-0.1em}{%
            \begin{tikzpicture}[x=0.08em, y=0.08em, line width=0.4pt]
                \draw[FooterGray] (0,3) -- (1,4) -- (2,3.5) -- (3,5) -- (4,4) -- (5,6) -- (6,5.5) -- (7,4) -- (8,5) -- (9,7) -- (10,6) -- (11,5) -- (12,6.5) -- (13,8) -- (14,7) -- (15,6) -- (16,7.5) -- (17,9) -- (18,8) -- (19,7) -- (20,8.5) -- (21,10) -- (22,9) -- (23,8) -- (24,9.5);
            \end{tikzpicture}%
        }%
        \hskip0.5cm%
    }%
    \vskip6pt%
}

%=============================================================================
% PACKAGES
%=============================================================================
\usepackage[utf8]{inputenc}
\usepackage[T1]{fontenc}
\usepackage[english]{babel}
\usepackage{amsmath, amssymb, amsthm}
\usepackage{mathtools}
\usepackage{bm}
\usepackage{tikz}
\usetikzlibrary{arrows.meta, positioning, shapes, calc, decorations.pathreplacing, shadings}
\usepackage{booktabs}
\usepackage{multirow}
\usepackage{array}
\usepackage{graphicx}
\usepackage{hyperref}
\usepackage{colortbl}
\usepackage{listings}
\lstset{basicstyle=\ttfamily\small, breaklines=true, frame=single, backgroundcolor=\color{VeryLightGray}}
\hypersetup{colorlinks=true, linkcolor=MainBlue, urlcolor=MainBlue}
\graphicspath{{../../logos/}{../../charts/}{../../photos/}}
\hfuzz=2pt  % Suppress tiny overfull warnings (<2pt)
\vfuzz=2pt  % Suppress tiny vertical overfull warnings (<2pt)

%=============================================================================
% QUANTLET COMMAND
%=============================================================================
\newcommand{\quantlet}[2]{%
    \hfill\href{#2}{%
        \raisebox{-0.15em}{\includegraphics[height=0.7em]{ql_logo.png}}%
        \textcolor{MainBlue}{\tiny\ #1}%
    }%
}

%=============================================================================
% CUSTOM TITLE PAGE
%=============================================================================
\defbeamertemplate*{title page}{hybrid}[1][]
{
    \vspace{0.2cm}
    % Logos row - top header (with clickable links)
    \begin{center}
        \href{https://www.ase.ro}{\includegraphics[height=1.0cm]{ase_logo.png}}\hspace{0.25cm}%
        \href{https://theida.net}{\includegraphics[height=1.0cm]{ida_logo.png}}\hspace{0.25cm}%
        \href{https://blockchain-research-center.com}{\includegraphics[height=1.0cm]{brc_logo.png}}\hspace{0.25cm}%
        \href{https://www.ai4efin.ase.ro}{\includegraphics[height=1.0cm]{ai4efin_logo.png}}\hspace{0.25cm}%
        \href{https://ipe.ro/new}{\includegraphics[height=1.0cm]{acad_logo.png}}\hspace{0.25cm}%
        \href{https://www.digital-finance-msca.com}{\includegraphics[height=1.0cm]{msca_logo.png}}%
    \end{center}

    \vspace{0.6cm}

    % Main title with Q logos on sides (with clickable links)
    \begin{center}
        \begin{minipage}{0.1\textwidth}
            \centering
            \href{https://quantlet.com}{\includegraphics[height=1.1cm]{ql_logo.png}}
        \end{minipage}%
        \begin{minipage}{0.78\textwidth}
            \centering
            {\LARGE\bfseries\usebeamercolor[fg]{title}\inserttitle}

            \vspace{0.3cm}

            {\usebeamerfont{subtitle}\usebeamercolor[fg]{title}\insertsubtitle}
        \end{minipage}%
        \begin{minipage}{0.1\textwidth}
            \centering
            \href{https://quantinar.com}{\includegraphics[height=1.1cm]{qr_logo.png}}
        \end{minipage}
    \end{center}

    \vspace{0.6cm}

    % Authors (left aligned)
    \hspace{0.5cm}{\usebeamerfont{author}\insertauthor}

    \vspace{0.3cm}

    % Institute/Affiliations (left aligned)
    \hspace{0.5cm}\begin{minipage}[t]{0.9\textwidth}
        \raggedright\small\insertinstitute
    \end{minipage}
}

%=============================================================================
% THEOREM ENVIRONMENTS
%=============================================================================
\theoremstyle{definition}
\setbeamertemplate{theorems}[numbered]
\newtheorem{defn}{Definition}
\newtheorem{thm}{Theorem}
\newtheorem{prop}{Proposition}
\newtheorem{rmk}{Remark}

%=============================================================================
% CENTRED MINIPAGE (no extra vertical space)
%=============================================================================
\newenvironment{cminipage}[1]{%
    \par\noindent\hfill\begin{minipage}{#1}\ignorespaces
}{%
    \end{minipage}\hfill\null\par
}

%=============================================================================
% CUSTOM COMMANDS
%=============================================================================
\newcommand{\E}{\mathbb{E}}
\newcommand{\Var}{\text{Var}}
\newcommand{\Cov}{\text{Cov}}
\newcommand{\Corr}{\text{Corr}}
\newcommand{\R}{\mathbb{R}}
\newcommand{\N}{\mathbb{N}}
\newcommand{\Z}{\mathbb{Z}}
\newcommand{\B}{\mathbf{B}}
\newcommand{\imark}{\textcolor{MainBlue}{\textbullet}}
\newcommand{\RMSE}{\text{RMSE}}
\newcommand{\MAE}{\text{MAE}}
\newcommand{\MAPE}{\text{MAPE}}
\newcommand{\correct}{\textcolor{Forest}{\checkmark}}
\newcommand{\incorrect}{\textcolor{Crimson}{\texttimes}}

% Boldface vector/matrix commands
\newcommand{\bY}{\mathbf{Y}}
\newcommand{\bX}{\mathbf{X}}
\newcommand{\bA}{\mathbf{A}}
\newcommand{\bB}{\mathbf{B}}
\newcommand{\bepsilon}{\boldsymbol{\varepsilon}}
\newcommand{\bvarepsilon}{\boldsymbol{\varepsilon}}
\newcommand{\bSigma}{\boldsymbol{\Sigma}}
\newcommand{\bPhi}{\boldsymbol{\Phi}}
\newcommand{\bGamma}{\boldsymbol{\Gamma}}
\newcommand{\bPi}{\boldsymbol{\Pi}}
\newcommand{\bc}{\mathbf{c}}
\newcommand{\balpha}{\boldsymbol{\alpha}}
\newcommand{\bbeta}{\boldsymbol{\beta}}

%=============================================================================
% TITLE INFORMATION
%=============================================================================
\title[Time Series Analysis]{Time Series Analysis and Forecasting}
\author[D.T. Pele]{Daniel Traian PELE}
\institute{Bucharest University of Economic Studies\\
IDA Institute Digital Assets\\
Blockchain Research Center\\
AI4EFin Artificial Intelligence for Energy Finance\\
Romanian Academy, Institute for Economic Forecasting\\
MSCA Digital Finance}
\date{}

\subtitle{Capitolul 2: Modele ARMA}

\begin{document}

% Pagina de titlu (fără antet/subsol)
{
\setbeamertemplate{headline}{}
\setbeamertemplate{footline}{}
\begin{frame}
    \titlepage
\end{frame}
}

%=============================================================================
% OBIECTIVE DE ÎNVĂȚARE
%=============================================================================
\begin{frame}{Obiective de învățare}
    \begin{cminipage}{0.95\textwidth}
        \begin{block}{La finalul acestui capitol, veți fi capabili să:}
            \begin{enumerate}\setlength{\itemsep}{0pt}
                \item[\textcolor{MainBlue}{\textbf{1.}}] \textbf{Definiți} și simulați procese AR (AutoRegresive), MA (Medie Mobilă) și ARMA (AutoRegresive cu Medie Mobilă)
                \item[\textcolor{MainBlue}{\textbf{2.}}] \textbf{Verificați} condițiile de staționaritate și invertibilitate
                \item[\textcolor{MainBlue}{\textbf{3.}}] \textbf{Identificați} ordinele $p$ și $q$ prin analiza ACF (Funcția de Autocorelație) / PACF (Funcția de Autocorelație Parțială)
                \item[\textcolor{MainBlue}{\textbf{4.}}] \textbf{Estimați} parametrii prin Yule-Walker, MLE (Estimarea prin Verosimilitate Maximă) și criterii informaționale -- AIC (Akaike), BIC (Bayesian)
                \item[\textcolor{MainBlue}{\textbf{5.}}] \textbf{Diagnosticați} modelul prin analiza reziduurilor și testul Ljung-Box
                \item[\textcolor{MainBlue}{\textbf{6.}}] \textbf{Prognozați} folosind modele ARMA cu intervale de încredere (IC)
                \item[\textcolor{MainBlue}{\textbf{7.}}] \textbf{Aplicați} metodologia Box-Jenkins pe date reale (pete solare)
            \end{enumerate}
        \end{block}
    \end{cminipage}
\end{frame}

%=============================================================================
% CUPRINS
%=============================================================================
\begin{frame}{Cuprins}
    \setbeamertemplate{section in toc}{\color{MainBlue}$\boxdot$~\inserttocsection}
    {\footnotesize
    \begin{columns}[T]
        \begin{column}{0.48\textwidth}
            \tableofcontents[sections={1-7}, hideallsubsections]
        \end{column}
        \begin{column}{0.48\textwidth}
            \tableofcontents[sections={8-13}, hideallsubsections]
        \end{column}
    \end{columns}
    }
\end{frame}

%=============================================================================
% MOTIVAȚIE
%=============================================================================
\section{Motivație}

\begin{frame}{De ce modele ARMA?}
    \begin{cminipage}{0.95\textwidth}
        \vspace{-0.3cm}
        \begin{center}
            \includegraphics[width=0.92\textwidth, height=0.68\textheight, keepaspectratio]{ch2_motivation_stationary.pdf}
        \end{center}
        \vspace{-0.2cm}
        {\scriptsize
        \begin{exampleblock}{}
            \begin{itemize}\setlength{\itemsep}{0pt}
                \item \textbf{Procese AR}: Valoarea curentă depinde de valorile trecute $\Rightarrow$ comportament de revenire la medie
                \item \textbf{Procese MA}: Valoarea curentă depinde de șocurile trecute $\Rightarrow$ memorie scurtă
                \item \textbf{ARMA}: Combină ambele mecanisme pentru modelare flexibilă
            \end{itemize}
        \end{exampleblock}
        }
    \end{cminipage}
    \quantlet{TSA\_ch2\_motivation}{https://github.com/QuantLet/TSA/tree/main/TSA_ch2/TSA_ch2_motivation}
\end{frame}

\begin{frame}{Identificarea modelului prin tipare ACF}
    \begin{cminipage}{0.95\textwidth}
        \vspace{-0.3cm}
        \begin{center}
            \includegraphics[width=0.92\textwidth, height=0.68\textheight, keepaspectratio]{ch2_motivation_acf.pdf}
        \end{center}
        \vspace{-0.2cm}
        {\scriptsize
        \begin{exampleblock}{ACF reflectă structura modelului}
            \begin{itemize}\setlength{\itemsep}{0pt}
                \item \textbf{Tipare distincte}: AR: descreștere exponențială; MA: anulare bruscă; ARMA: descreștere mixtă
                \item \textbf{Identificare}: Analiza vizuală a ACF/PACF ghidează selecția ordinelor $p$ și $q$
            \end{itemize}
        \end{exampleblock}
        }
    \end{cminipage}
    \quantlet{TSA\_ch2\_motivation}{https://github.com/QuantLet/TSA/tree/main/TSA_ch2/TSA_ch2_motivation}
\end{frame}

%=============================================================================
% SECȚIUNEA 1: INTRODUCERE ȘI OPERATORUL LAG
%=============================================================================
\section{Introducere și operatorul lag}

\begin{frame}{Recapitulare: Staționaritatea}
    \begin{cminipage}{0.95\textwidth}
        \vspace{-0.3cm}
        {\small
        \begin{block}{Din capitolul 1}
            \begin{itemize}\setlength{\itemsep}{0pt}
                \item Un proces $\{X_t\}$ este \textbf{slab staționar} dacă:
                \begin{enumerate}\setlength{\itemsep}{0pt}
                    \item $\E[X_t] = \mu$ (medie constantă)
                    \item $\Var(X_t) = \sigma^2 < \infty$ (varianță constantă, finită)
                    \item $\Cov(X_t, X_{t+h}) = \gamma(h)$ (covarianța depinde doar de lag-ul $h$)
                \end{enumerate}
            \end{itemize}
        \end{block}
        \vspace{-0.3cm}
        \begin{exampleblock}{De ce contează staționaritatea pentru ARMA}
            \begin{itemize}\setlength{\itemsep}{0pt}
                \item \textcolor{Forest}{Modelele ARMA presupun staționaritate} -- parametrii rămân stabili în timp, structura de autocorelație se menține
                \item \textcolor{Crimson}{Date nestaționare} $\Rightarrow$ diferențiați mai întâi (ARIMA, Cap. 3)
            \end{itemize}
        \end{exampleblock}
        \vspace{-0.3cm}
        \begin{alertblock}{Obiectivul capitolului}
            \begin{itemize}\setlength{\itemsep}{0pt}
                \item Modele parametrice pentru serii staționare $\Rightarrow$ combinând dependența de observațiile anterioare (AR) cu influența șocurilor aleatoare (MA)
            \end{itemize}
        \end{alertblock}
        }
    \end{cminipage}
\end{frame}

\begin{frame}{Operatorul lag: ilustrație vizuală}
    \begin{cminipage}{0.95\textwidth}
        \begin{center}
            \includegraphics[width=0.88\textwidth, height=0.55\textheight, keepaspectratio]{ch2_lag_operator.pdf}
        \end{center}
        \vspace{-0.3cm}
        \quantlet{TSA\_ch2\_lag\_operator}{https://github.com/QuantLet/TSA/tree/main/TSA_ch2/TSA_ch2_lag_operator}
        {\scriptsize
        \begin{exampleblock}{Rolul operatorului lag}
            \begin{itemize}\setlength{\itemsep}{0pt}
                \item \textbf{Notație compactă}: ecuații cu diferențe, polinoame lag
                \item \textbf{Utilitate}: manipulare algebrică, condiții de staționaritate
            \end{itemize}
        \end{exampleblock}
        }
    \end{cminipage}
\end{frame}

\begin{frame}{Operatorul lag (operatorul de întârziere)}
    \begin{cminipage}{0.95\textwidth}
        \vspace{-0.2cm}
        \begin{defn}[Operatorul lag]
            \begin{itemize}\setlength{\itemsep}{0pt}
                \item \textbf{Operatorul lag} $L$ (sau operatorul de întârziere $B$) deplasează o serie de timp înapoi cu o perioadă: $L X_t = X_{t-1}$
            \end{itemize}
        \end{defn}
        \vspace{-0.2cm}
        \begin{block}{Proprietăți}
            \begin{itemize}\setlength{\itemsep}{0pt}
                \item $L^k X_t = X_{t-k}$ (deplasare înapoi cu $k$ perioade)
                \item $L^0 X_t = X_t$ (identitate)
                \item $(1-L)X_t = X_t - X_{t-1} = \Delta X_t$ (prima diferență)
                \item $(1-L)^d X_t = \Delta^d X_t$ (diferența de ordin $d$)
            \end{itemize}
        \end{block}
        \vspace{-0.2cm}
        \begin{exampleblock}{Polinoame lag}
            \begin{itemize}\setlength{\itemsep}{0pt}
                \item \textbf{Polinom AR}: $\phi(L) = 1 - \phi_1 L - \phi_2 L^2 - \cdots - \phi_p L^p$
                \item \textbf{Polinom MA}: $\theta(L) = 1 + \theta_1 L + \theta_2 L^2 + \cdots + \theta_q L^q$
            \end{itemize}
        \end{exampleblock}
    \end{cminipage}
    \quantlet{TSA\_ch2\_lag\_operator}{https://github.com/QuantLet/TSA/tree/main/TSA_ch2/TSA_ch2_lag_operator}
\end{frame}

\begin{frame}{Zgomot alb: ilustrare vizuală}
    \vspace{-0.2cm}
    \begin{cminipage}{0.95\textwidth}
        \centering
        \includegraphics[width=0.92\textwidth, height=0.58\textheight, keepaspectratio]{ch2_white_noise.pdf}

        \vspace{0.1cm}
        {\footnotesize
        \begin{block}{Caracteristici}
            \begin{itemize}\setlength{\itemsep}{2pt}
                \item \textbf{Stânga}: Fluctuații aleatorii, fără tipare, varianță constantă
                \item \textbf{Dreapta}: ACF doar un vârf la lag 0; celelalte în limitele de semnificație $\Rightarrow$ fără dependență liniară
            \end{itemize}
        \end{block}
        }
    \end{cminipage}
    \quantlet{TSA\_ch2\_white\_noise}{https://github.com/QuantLet/TSA/tree/main/TSA_ch2/TSA_ch2_white_noise}
\end{frame}

\begin{frame}{Procesul de zgomot alb}
    \begin{cminipage}{0.95\textwidth}
        \vspace{-0.2cm}
        \begin{defn}[Zgomot Alb]
            \begin{itemize}\setlength{\itemsep}{0pt}
                \item Un proces $\{\varepsilon_t\}$ este \textbf{zgomot alb}, notat $\varepsilon_t \sim \text{WN}(0, \sigma^2)$ (WN -- White Noise, zgomot alb), dacă:
                \begin{enumerate}
                    \item $\E[\varepsilon_t] = 0$ pentru toți $t$
                    \item $\Var(\varepsilon_t) = \sigma^2$ pentru toți $t$
                    \item $\Cov(\varepsilon_t, \varepsilon_s) = 0$ pentru toți $t \neq s$
                \end{enumerate}
            \end{itemize}
        \end{defn}
        \vspace{-0.2cm}
        {\footnotesize
        \begin{block}{Proprietăți}
            \begin{itemize}\setlength{\itemsep}{0pt}
                \item \textbf{Element de bază}: Zgomotul alb stă la baza tuturor modelelor ARMA
                \item \textbf{ACF}: $\rho(0) = 1$, $\rho(h) = 0$ pentru $h \neq 0$; PACF: același tipar
                \item \textbf{Zgomot alb Gaussian}: $\varepsilon_t \sim N(0, \sigma^2)$ i.i.d.
                \item \textbf{Impredictibil}: Zgomotul alb \textit{nu} poate fi prezis $\Rightarrow$ este pur aleatoriu
            \end{itemize}
        \end{block}
        }
    \end{cminipage}
\end{frame}

%=============================================================================
% SECȚIUNEA 2: MODELE AR
%=============================================================================
\section{Modele autoregresive (AR)}

\begin{frame}{AR(1): ilustrație vizuală}
    \begin{cminipage}{0.95\textwidth}
        \vspace{-0.3cm}
        \begin{center}
            \includegraphics[width=0.95\textwidth, height=0.60\textheight, keepaspectratio]{ch2_def_ar1.pdf}
        \end{center}
        \vspace{-0.3cm}
        {\scriptsize
        \begin{block}{Interpretarea vizuală}
            \begin{itemize}\setlength{\itemsep}{0pt}
                \item \textbf{$\phi$ pozitiv}: Fluctuații persistente, revenire graduală la medie
                \item \textbf{$\phi$ negativ}: Comportament oscilant, alternând în jurul mediei
                \item $|\phi|$ mai mare $\Rightarrow$ persistență mai mare, revenire mai lentă
            \end{itemize}
        \end{block}
        }
    \end{cminipage}
    \quantlet{TSA\_ch2\_ar1}{https://github.com/QuantLet/TSA/tree/main/TSA_ch2/TSA_ch2_ar1}
\end{frame}

\begin{frame}{Modelul AR(1): definiție}
    \begin{cminipage}{0.95\textwidth}
        \vspace{-0.2cm}
        \begin{defn}[Proces AR(1)]
            \begin{itemize}\setlength{\itemsep}{0pt}
                \item Un \textbf{proces autoregresiv de ordin 1} este: $X_t = c + \phi X_{t-1} + \varepsilon_t$
                \item $\varepsilon_t \sim WN(0, \sigma^2)$ și $|\phi| < 1$ pentru staționaritate
            \end{itemize}
        \end{defn}
        \vspace{-0.2cm}
        \begin{columns}[T]
            \begin{column}{0.48\textwidth}
                \begin{block}{Interpretare}
                    \begin{itemize}\setlength{\itemsep}{0pt}
                        \item $c$: constantă (interceptul)
                        \item $\phi$: coeficient autoregresiv
                        \begin{itemize}
                            \item Măsoară persistența seriei
                        \end{itemize}
                        \item $\varepsilon_t$: inovație (șoc)
                    \end{itemize}
                \end{block}
            \end{column}
            \begin{column}{0.48\textwidth}
                \begin{exampleblock}{Notație cu operatorul lag}
                    \begin{itemize}\setlength{\itemsep}{0pt}
                        \item $(1 - \phi L)X_t = c + \varepsilon_t$
                        \item $\phi(L) X_t = c + \varepsilon_t$
                        \item $\phi(L) = 1 - \phi L$
                    \end{itemize}
                \end{exampleblock}
            \end{column}
        \end{columns}
    \end{cminipage}
\end{frame}

\begin{frame}{Condiția de staționaritate AR(1)}
    \begin{cminipage}{0.95\textwidth}
        \vspace{-0.2cm}
        \begin{alertblock}{Condiție necesară și suficientă: $|\phi| < 1$}
            \begin{itemize}\setlength{\itemsep}{0pt}
                \item Rădăcina ecuației caracteristice trebuie să fie în afara cercului unitate
            \end{itemize}
        \end{alertblock}
        \vspace{-0.2cm}
        \begin{columns}[T]
            \begin{column}{0.48\textwidth}
                \begin{exampleblock}{\textcolor{Forest}{Staționar ($|\phi| < 1$)}}
                    \begin{itemize}\setlength{\itemsep}{0pt}
                        \item Șocurile se diminuează în timp
                        \begin{itemize}
                            \item Procesul revine la medie
                            \item Varianță finită, stabilă
                        \end{itemize}
                    \end{itemize}
                \end{exampleblock}
            \end{column}
            \begin{column}{0.48\textwidth}
                \begin{alertblock}{\textcolor{white}{Nestaționar ($|\phi| \geq 1$)}}
                    \begin{itemize}\setlength{\itemsep}{0pt}
                        \item $|\phi| = 1$: mers aleatoriu
                        \begin{itemize}
                            \item Rădăcină unitate, varianță $\to \infty$
                        \end{itemize}
                        \item $|\phi| > 1$: proces exploziv
                    \end{itemize}
                \end{alertblock}
            \end{column}
        \end{columns}
        \vspace{-0.2cm}
        \begin{block}{Ecuația caracteristică}
            \begin{itemize}\setlength{\itemsep}{0pt}
                \item $\phi(z) = 1 - \phi z = 0 \implies z = 1/\phi$
                \item Staționaritate $\Leftrightarrow$ rădăcina în afara cercului unitate ($|z|>1$)
            \end{itemize}
        \end{block}
    \end{cminipage}
\end{frame}

\begin{frame}{Proprietățile AR(1)}
    \begin{cminipage}{0.95\textwidth}
        \vspace{-0.2cm}
        \begin{block}{AR(1) staționar cu $|\phi| < 1$}
            \begin{itemize}\setlength{\itemsep}{0pt}
                \item Proprietățile momentelor:
            \end{itemize}
            \vspace{-0.2cm}
            \begin{columns}[T]
                \begin{column}{0.48\textwidth}
                    {\small
                    \begin{itemize}\setlength{\itemsep}{0pt}
                        \item \textbf{Media:} $\mu = \E[X_t] = \frac{c}{1-\phi}$
                        \item \textbf{Varianța:} $\gamma(0) = \Var(X_t) = \frac{\sigma^2}{1-\phi^2}$
                    \end{itemize}
                    }
                \end{column}
                \begin{column}{0.48\textwidth}
                    {\small
                    \begin{itemize}\setlength{\itemsep}{0pt}
                        \item \textbf{Autocovarianța:} $\gamma(h) = \frac{\phi^h \sigma^2}{1-\phi^2}$
                        \item \textbf{Autocorelația (ACF):} $\rho(h) = \phi^h$
                    \end{itemize}
                    }
                \end{column}
            \end{columns}
        \end{block}
        \vspace{-0.2cm}
        \begin{alertblock}{Observație}
            \begin{itemize}\setlength{\itemsep}{0pt}
                \item \textbf{Semnătura AR(1)}: ACF scade exponențial cu factorul $\phi$
                \begin{itemize}
                    \item $\phi > 0$: descreștere monotonă spre zero
                    \item $\phi < 0$: oscilații amortizate (semne alternante)
                \end{itemize}
            \end{itemize}
        \end{alertblock}
    \end{cminipage}
    \quantlet{TSA\_ch2\_ar1}{https://github.com/QuantLet/TSA/tree/main/TSA_ch2/TSA_ch2_ar1}
\end{frame}

\begin{frame}{Simulări AR(1): efectul lui $\phi$}
    \begin{cminipage}{0.95\textwidth}
        \vspace{-0.3cm}
        \begin{center}
            \includegraphics[width=0.82\textwidth, height=0.56\textheight, keepaspectratio]{ch2_ar1_simulations.pdf}
        \end{center}
        \vspace{-0.2cm}
        {\scriptsize
        \begin{block}{Interpretare}
            \begin{itemize}\setlength{\itemsep}{0pt}
                \item Valori diferite ale lui $\phi$ produc comportamente distincte: $|\phi|$ mai mare $\Rightarrow$ mai multă persistență
                \item $\phi$ pozitiv creează evoluții netede; $\phi$ negativ creează oscilații
                \item Pe măsură ce $|\phi| \to 1$, procesul se apropie de nestaționaritate
            \end{itemize}
        \end{block}
        }
    \end{cminipage}
    \quantlet{TSA\_ch2\_ar1\_simulation}{https://github.com/QuantLet/TSA/tree/main/TSA_ch2/TSA_ch2_ar1_simulation}
\end{frame}

\begin{frame}{ACF teoretic AR(1)}
    \begin{cminipage}{0.95\textwidth}
        \vspace{-0.2cm}
        \begin{center}
            \includegraphics[width=0.85\textwidth, height=0.60\textheight, keepaspectratio]{ch2_ar1_theoretical_acf.pdf}
        \end{center}
        \vspace{-0.2cm}
        {\scriptsize
        \begin{block}{Tipar ACF}
            \begin{itemize}\setlength{\itemsep}{0pt}
                \item \textbf{Formula}: $\rho(h) = \phi^h$ $\Rightarrow$ descreștere exponențială
                \item $\phi > 0$: descreștere monotonă; $\phi < 0$: semne alternante
            \end{itemize}
        \end{block}
        }
    \end{cminipage}
    \quantlet{TSA\_ch2\_ar1}{https://github.com/QuantLet/TSA/tree/main/TSA_ch2/TSA_ch2_ar1}
\end{frame}

\begin{frame}{ACF și PACF AR(1): teorie vs eșantion}
    \begin{cminipage}{0.95\textwidth}
        \vspace{-0.3cm}
        \begin{center}
            \includegraphics[width=0.82\textwidth, height=0.60\textheight, keepaspectratio]{ch2_ar1_acf_pacf.pdf}
        \end{center}
        \vspace{-0.2cm}
        {\scriptsize
        \begin{block}{Interpretare}
            \begin{itemize}\setlength{\itemsep}{0pt}
                \item \textbf{ACF eșantion vs teorie}: estimările fluctuează în jurul curbei teoretice $\rho(h)=\phi^h$
                \item \textbf{PACF}: un singur vârf semnificativ la lag 1, apoi se anulează $\Rightarrow$ semnătura tipică AR(1)
            \end{itemize}
        \end{block}
        }
    \end{cminipage}
    \quantlet{TSA\_ch2\_ar1}{https://github.com/QuantLet/TSA/tree/main/TSA_ch2/TSA_ch2_ar1}
\end{frame}

\begin{frame}{Demonstrație: media AR(1)}
    \begin{cminipage}{0.95\textwidth}
        \vspace{-0.3cm}
        \begin{block}{Afirmație}
            \begin{itemize}\setlength{\itemsep}{0pt}
                \item Pentru AR(1): $X_t = c + \phi X_{t-1} + \varepsilon_t$, media este $\mu = \frac{c}{1-\phi}$
            \end{itemize}
        \end{block}
        \vspace{-0.3cm}
        {\small
        \begin{exampleblock}{Demonstrație}
            \begin{itemize}\setlength{\itemsep}{0pt}
                \item Luăm speranța ambelor părți: $\E[X_t] = c + \phi \E[X_{t-1}] + \E[\varepsilon_t]$
                \item Prin staționaritate, $\E[X_t] = \E[X_{t-1}] = \mu$, și $\E[\varepsilon_t] = 0$: $\mu = c + \phi \mu$
                \item Rezolvând: $\mu - \phi\mu = c \implies \mu(1-\phi) = c \implies \boxed{\mu = \frac{c}{1-\phi}}$
            \end{itemize}
        \end{exampleblock}
        }
        \vspace{-0.2cm}
        \begin{alertblock}{Cerință}
            \begin{itemize}\setlength{\itemsep}{0pt}
                \item \textbf{Condiție necesară}: $\phi \neq 1$ pentru ca media să fie definită
                \begin{itemize}
                    \item Dacă $\phi = 1$ (rădăcină unitară), media este nedefinită
                    \item Procesul devine mers aleatoriu (nestaționaritate)
                \end{itemize}
            \end{itemize}
        \end{alertblock}
    \end{cminipage}
\end{frame}

\begin{frame}{Demonstrație: varianța AR(1)}
    \begin{cminipage}{0.95\textwidth}
        \vspace{-0.3cm}
        \begin{block}{Afirmație}
            \begin{itemize}\setlength{\itemsep}{0pt}
                \item $\Var(X_t) = \frac{\sigma^2}{1-\phi^2}$
            \end{itemize}
        \end{block}
        \vspace{-0.3cm}
        {\small
        \begin{exampleblock}{Demonstrație}
            \begin{itemize}\setlength{\itemsep}{0pt}
                \item Presupunem $c=0$. Luăm varianța din $X_t = \phi X_{t-1} + \varepsilon_t$:
                \item $\Var(X_t) = \phi^2 \Var(X_{t-1}) + \Var(\varepsilon_t) + 2\phi\underbrace{\Cov(X_{t-1}, \varepsilon_t)}_{=0}$
                \item Prin staționaritate, $\Var(X_t) = \Var(X_{t-1}) = \gamma(0)$:
                \item $\gamma(0) = \phi^2 \gamma(0) + \sigma^2 \implies \gamma(0)(1-\phi^2) = \sigma^2 \implies \boxed{\gamma(0) = \frac{\sigma^2}{1-\phi^2}}$
            \end{itemize}
        \end{exampleblock}
        }
        \vspace{-0.2cm}
        \begin{alertblock}{Notă}
            \begin{itemize}\setlength{\itemsep}{0pt}
                \item Necesită $|\phi| < 1$ pentru varianță pozitivă. Când $|\phi| \to 1$, varianța $\to \infty$
            \end{itemize}
        \end{alertblock}
    \end{cminipage}
\end{frame}

\begin{frame}{Varianța AR(1) ca funcție de $\phi$}
    \begin{cminipage}{0.95\textwidth}
        \vspace{-0.2cm}
        \begin{center}
            \includegraphics[width=0.85\textwidth, height=0.55\textheight, keepaspectratio]{ch2_ar1_variance.pdf}
        \end{center}
        \vspace{-0.3cm}
        {\scriptsize
        \begin{block}{Observații}
            \begin{itemize}\setlength{\itemsep}{0pt}
                \item Pe măsură ce $|\phi| \to 1$, varianța explodează $\Rightarrow$ nestaționaritate
                \item Pentru $\phi = 0$: $\gamma(0) = \sigma^2$ (zgomot alb); varianța crește monoton cu $|\phi|$
            \end{itemize}
        \end{block}
        }
    \end{cminipage}
    \quantlet{TSA\_ch2\_ar1}{https://github.com/QuantLet/TSA/tree/main/TSA_ch2/TSA_ch2_ar1}
\end{frame}

\begin{frame}{Demonstrație: funcția de autocorelație AR(1)}
    \begin{cminipage}{0.95\textwidth}
        \vspace{-0.3cm}
        \begin{block}{Afirmație: $\rho(h) = \phi^h$ pentru $h \geq 0$}
            \begin{itemize}\setlength{\itemsep}{0pt}
                \item Găsim autocovarianța $\gamma(h) = \Cov(X_t, X_{t-h})$
            \end{itemize}
        \end{block}
        \vspace{-0.3cm}
        {\small
        \begin{exampleblock}{Demonstrație}
            \begin{itemize}\setlength{\itemsep}{0pt}
                \item Înmulțim $X_t = \phi X_{t-1} + \varepsilon_t$ cu $X_{t-h}$ și luăm media:
                \item $\E[X_t X_{t-h}] = \phi \E[X_{t-1} X_{t-h}] + \E[\varepsilon_t X_{t-h}]$
                \item Pentru $h \geq 1$: $\E[\varepsilon_t X_{t-h}] = 0$ $\Rightarrow$ $\gamma(h) = \phi \gamma(h-1)$
                \item Relație recursivă de la $\gamma(0)$: $\gamma(1) = \phi\gamma(0)$, $\gamma(2) = \phi^2\gamma(0)$, $\ldots$ $\boxed{\gamma(h) = \phi^h\gamma(0)}$
                \item ACF: $\rho(h) = \frac{\gamma(h)}{\gamma(0)} = \frac{\phi^h\gamma(0)}{\gamma(0)} = \boxed{\phi^h}$
            \end{itemize}
        \end{exampleblock}
        }
    \end{cminipage}
\end{frame}

\begin{frame}{Demonstrație: condiția de staționaritate AR(1)}
    \begin{cminipage}{0.95\textwidth}
        \vspace{-0.3cm}
        \begin{block}{Afirmație}
            \begin{itemize}\setlength{\itemsep}{0pt}
                \item AR(1) este staționar dacă și numai dacă $|\phi| < 1$
            \end{itemize}
        \end{block}
        \vspace{-0.3cm}
        {\small
        \begin{exampleblock}{Demonstrație}
            \begin{itemize}\setlength{\itemsep}{0pt}
                \item Substituție recursivă: $X_t = \phi X_{t-1} + \varepsilon_t = \phi(\phi X_{t-2} + \varepsilon_{t-1}) + \varepsilon_t = \cdots$
                \item După $n$ pași: $X_t = \phi^n X_{t-n} + \sum_{j=0}^{n-1}\phi^j\varepsilon_{t-j}$
                \item Dacă $|\phi| < 1$: $\phi^n \to 0$ când $n \to \infty$, deci $X_t = \sum_{j=0}^{\infty}\phi^j\varepsilon_{t-j}$
                \item Varianță finită: $\Var(X_t) = \sigma^2\sum_{j=0}^{\infty}\phi^{2j} = \frac{\sigma^2}{1-\phi^2} < \infty$ \quad (serie geometrică)
            \end{itemize}
        \end{exampleblock}
        }
        \vspace{-0.2cm}
        \begin{alertblock}{Concluzie}
            \begin{itemize}\setlength{\itemsep}{0pt}
                \item Converge $\iff |\phi| < 1$. Pentru $|\phi| \geq 1$, termenul $\phi^n X_{t-n}$ nu dispare $\Rightarrow$ varianță infinită
            \end{itemize}
        \end{alertblock}
    \end{cminipage}
\end{frame}

\begin{frame}{Funcția de autocorelație parțială (PACF)}
    \begin{cminipage}{0.95\textwidth}
        \vspace{-0.3cm}
        \begin{defn}[PACF]
            \begin{itemize}\setlength{\itemsep}{0pt}
                \item \textbf{Autocorelația parțială} de ordin $k$, notată $\pi_k$, măsoară corelația dintre $X_t$ și $X_{t-k}$
                \item Se elimină efectele liniare ale variabilelor intermediare $X_{t-1}, \ldots, X_{t-k+1}$
            \end{itemize}
        \end{defn}
        \vspace{-0.3cm}
        {\small
        \begin{columns}[T]
            \begin{column}{0.48\textwidth}
                \begin{block}{Definiție formală}
                    \begin{itemize}\setlength{\itemsep}{0pt}
                        \item $\pi_1 = \rho(1)$
                        \item Pentru $k \geq 2$: $\pi_k$ este ultimul coeficient din: $X_t = \alpha_1 X_{t-1} + \cdots + \alpha_k X_{t-k} + e_t$
                        \item $\pi_k = \alpha_k$ (coeficientul lui $X_{t-k}$)
                    \end{itemize}
                \end{block}
            \end{column}
            \begin{column}{0.48\textwidth}
                \begin{exampleblock}{Calculul prin Yule-Walker}
                    \begin{itemize}\setlength{\itemsep}{0pt}
                        \item Se rezolvă ecuațiile Yule-Walker de ordin $k$
                        \item $\pi_k$ = ultimul element al vectorului soluție
                    \end{itemize}
                \end{exampleblock}
                \vspace{-0.2cm}
                \begin{alertblock}{Utilitate}
                    \begin{itemize}\setlength{\itemsep}{0pt}
                        \item \textbf{Identificare}: PACF determină ordinul $p$ al unui model AR
                        \begin{itemize}
                            \item PACF se anulează după lag $p$
                        \end{itemize}
                    \end{itemize}
                \end{alertblock}
            \end{column}
        \end{columns}
        }
    \end{cminipage}
\end{frame}

\begin{frame}{Algoritmul Durbin-Levinson pentru PACF}
    \begin{cminipage}{0.95\textwidth}
        \small
        \begin{block}{Recursie Durbin-Levinson}
            \begin{itemize}\setlength{\itemsep}{0pt}
                \item Calculează PACF ($\pi_k$) recursiv, fără inversarea matricii Toeplitz:
            \end{itemize}
            \vspace{-2mm}
            \begin{enumerate}\setlength{\itemsep}{1pt}
                \item \textbf{Inițializare}: $\pi_1 = \hat{\rho}(1)$, $v_1 = \hat{\gamma}(0)(1 - \pi_1^2)$
                \item \textbf{Recursie} ($k = 2, 3, \ldots$):
                \[
                    \pi_k = \frac{\hat{\rho}(k) - \sum_{j=1}^{k-1} \phi_{k-1,j}\, \hat{\rho}(k-j)}{1 - \sum_{j=1}^{k-1} \phi_{k-1,j}\, \hat{\rho}(j)}
                \]
                \item \textbf{Actualizare coeficienți}: $\phi_{k,j} = \phi_{k-1,j} - \pi_k\, \phi_{k-1,k-j}$ pentru $j = 1, \ldots, k-1$; $\phi_{k,k} = \pi_k$
                \item \textbf{Varianța predicției}: $v_k = v_{k-1}(1 - \pi_k^2)$
            \end{enumerate}
        \end{block}
        \vspace{-0.2cm}
        \begin{columns}[T]
            \begin{column}{0.48\textwidth}
                \begin{exampleblock}{Complexitate}
                    \begin{itemize}\setlength{\itemsep}{0pt}
                        \item $O(k^2)$ vs $O(k^3)$ (inversare directă)
                        \item Exploatează structura Toeplitz a lui $\boldsymbol{\Gamma}_k$
                        \item Garantează $v_k > 0$ dacă procesul e staționar
                    \end{itemize}
                \end{exampleblock}
            \end{column}
            \begin{column}{0.48\textwidth}
                \begin{alertblock}{Identificare AR($p$)}
                    \begin{itemize}\setlength{\itemsep}{0pt}
                        \item $\pi_k = 0$ pentru $k > p$ $\Rightarrow$ ordinul $p$
                        \item Interval de încredere: $|\pi_k| > 1.96/\sqrt{T}$ $\Rightarrow$ semnificativ
                        \item Echivalent cu testul $t$ pe ultimul coeficient OLS
                    \end{itemize}
                \end{alertblock}
            \end{column}
        \end{columns}
    \end{cminipage}
\end{frame}

\begin{frame}{AR(p): ilustrație vizuală}
    \begin{cminipage}{0.95\textwidth}
        \vspace{-0.2cm}
        \begin{center}
            \includegraphics[width=0.92\textwidth, height=0.56\textheight, keepaspectratio]{ch2_def_arp.pdf}
        \end{center}
        \vspace{-0.3cm}
        {\scriptsize
        \begin{block}{Observații}
            \begin{itemize}\setlength{\itemsep}{0pt}
                \item AR(2) poate prezenta comportament pseudo-ciclic (rădăcini complexe); ACF sinusoidală amortizată
                \item PACF se anulează după lag 2 $\Rightarrow$ tiparul distinctiv de identificare
            \end{itemize}
        \end{block}
        }
    \end{cminipage}
    \quantlet{TSA\_ch2\_ar2}{https://github.com/QuantLet/TSA/tree/main/TSA_ch2/TSA_ch2_ar2}
\end{frame}

\begin{frame}{Modelul AR(p): forma generală}
    \begin{cminipage}{0.95\textwidth}
        \vspace{-0.2cm}
        \begin{defn}[Proces AR(p)]
            \begin{itemize}\setlength{\itemsep}{0pt}
                \item Un \textbf{proces autoregresiv de ordin p} este: $X_t = c + \phi_1 X_{t-1} + \phi_2 X_{t-2} + \cdots + \phi_p X_{t-p} + \varepsilon_t$
                \item \textbf{Operator lag}: $\phi(L) X_t = c + \varepsilon_t$, unde $\phi(L) = 1 - \phi_1 L - \phi_2 L^2 - \cdots - \phi_p L^p$
            \end{itemize}
        \end{defn}
        \vspace{-0.2cm}
        \begin{block}{Condiție de staționaritate}
            \begin{itemize}\setlength{\itemsep}{0pt}
                \item Toate rădăcinile lui $\phi(z) = 0$ trebuie să se afle \textbf{în afara} cercului unitate
                \item Echivalent: toate rădăcinile au modul $> 1$
            \end{itemize}
        \end{block}
        \vspace{-0.2cm}
        \begin{exampleblock}{Tiparul PACF}
            \begin{itemize}\setlength{\itemsep}{0pt}
                \item PACF se anulează după lag $p$
                \item ACF scade (exponențial sau cu oscilații amortizate)
            \end{itemize}
        \end{exampleblock}
    \end{cminipage}
\end{frame}

\begin{frame}{Staționaritatea AR(2): vizualizarea cercului unitate}
    \begin{cminipage}{0.95\textwidth}
        \vspace{-0.3cm}
        \begin{center}
            \includegraphics[width=0.85\textwidth, height=0.58\textheight, keepaspectratio]{ch2_unit_circle.pdf}
        \end{center}
        \vspace{-0.3cm}
        {\scriptsize
        \begin{block}{Polinomul caracteristic și condiția cercului unitate}
            \begin{itemize}\setlength{\itemsep}{0pt}
                \item \textbf{Polinomul caracteristic} al unui proces AR($p$): $\phi(z) = 1 - \phi_1 z - \phi_2 z^2 - \cdots - \phi_p z^p$
                \item Toate rădăcinile lui $\phi(z) = 0$ trebuie să se afle \textbf{în afara} cercului unitate ($|z| > 1$)
                \item Rădăcini pe cerc: nestaționar; rădăcini în interior: proces exploziv
            \end{itemize}
        \end{block}
        }

    \end{cminipage}
    \quantlet{TSA\_ch2\_stationarity}{https://github.com/QuantLet/TSA/tree/main/TSA_ch2/TSA_ch2_stationarity}
\end{frame}

\begin{frame}{Triunghiul de staționaritate AR(2)}
    \begin{cminipage}{0.95\textwidth}
        \vspace{-0.3cm}
        \begin{center}
            \includegraphics[width=0.82\textwidth, height=0.52\textheight, keepaspectratio]{ch2_ar2_stationarity.pdf}
        \end{center}
        \vspace{-0.3cm}
        {\scriptsize
        \begin{block}{Regiunea de staționaritate}
            \begin{itemize}\setlength{\itemsep}{0pt}
                \item Regiunea triunghiulară definește combinațiile de parametri AR(2) staționari
                \item \textbf{Granițe}: $\phi_1 + \phi_2 < 1$, $\phi_2 - \phi_1 < 1$ și $|\phi_2| < 1$
                \item Punctele din afara regiunii $\Rightarrow$ procese nestaționare sau explozive
            \end{itemize}
        \end{block}
        }

    \end{cminipage}
    \quantlet{TSA\_ch2\_stationarity}{https://github.com/QuantLet/TSA/tree/main/TSA_ch2/TSA_ch2_stationarity}
\end{frame}

\begin{frame}{Rădăcinile polinomului caracteristic}
    \begin{cminipage}{0.95\textwidth}
        \vspace{-0.3cm}
        \begin{center}
            \includegraphics[width=0.88\textwidth, height=0.60\textheight, keepaspectratio]{ch2_characteristic_roots.pdf}
        \end{center}
        \vspace{-0.3cm}
        {\scriptsize
        \begin{block}{Tipuri de rădăcini}
            \begin{itemize}\setlength{\itemsep}{0pt}
                \item \textbf{Rădăcini reale}: descreștere exponențială în ACF
                \item \textbf{Rădăcini complexe}: oscilații amortizate (pseudo-cicluri)
                \item Toate rădăcinile trebuie să fie în afara cercului unitate
            \end{itemize}
        \end{block}
        }

    \end{cminipage}
    \quantlet{TSA\_ch2\_stationarity}{https://github.com/QuantLet/TSA/tree/main/TSA_ch2/TSA_ch2_stationarity}
\end{frame}

\begin{frame}{Modelul AR(2)}
    \begin{cminipage}{0.95\textwidth}
        \vspace{-0.2cm}
        \begin{defn}[Proces AR(2)]
            \begin{itemize}\setlength{\itemsep}{0pt}
                \item $X_t = c + \phi_1 X_{t-1} + \phi_2 X_{t-2} + \varepsilon_t$
            \end{itemize}
        \end{defn}
        \vspace{-0.2cm}
        \begin{block}{Condiții de staționaritate}
            \begin{itemize}\setlength{\itemsep}{0pt}
                \item $\phi_1 + \phi_2 < 1$; \quad $\phi_2 - \phi_1 < 1$; \quad $|\phi_2| < 1$
            \end{itemize}
        \end{block}
        \vspace{-0.2cm}
        \begin{exampleblock}{Comportamentul ACF}
            \begin{itemize}\setlength{\itemsep}{0pt}
                \item \textbf{Rădăcini reale}: amestec de două descreșteri exponențiale
                \item \textbf{Rădăcini complexe}: tipar sinusoidal amortizat (pseudo-cicluri)
                \item \textbf{PACF}: Se anulează după lag 2 ($\pi_k = 0$ pentru $k > 2$)
            \end{itemize}
        \end{exampleblock}

    \end{cminipage}
    \quantlet{TSA\_ch2\_ar2}{https://github.com/QuantLet/TSA/tree/main/TSA_ch2/TSA_ch2_ar2}
\end{frame}

%=============================================================================
% SECȚIUNEA 3: MODELE MA
%=============================================================================
\section{Modele de medie mobilă (MA)}

\begin{frame}{MA(1): ilustrație vizuală}
    \begin{cminipage}{0.95\textwidth}
        \vspace{-0.2cm}
        \begin{center}
            \includegraphics[width=0.92\textwidth, height=0.55\textheight, keepaspectratio]{ch2_def_ma1.pdf}
        \end{center}
        \vspace{-0.3cm}
        {\scriptsize
        \begin{block}{Interpretare vizuală}
            \begin{itemize}\setlength{\itemsep}{0pt}
                \item \textbf{Panoul stâng}: Serie MA(1) $\Rightarrow$ revenire rapidă la medie
                \item \textbf{Panoul drept}: ACF cu \textbf{anulare după lag 1}; PACF descreștere exponențială
            \end{itemize}
        \end{block}
        }

    \end{cminipage}
    \quantlet{TSA\_ch2\_ma1}{https://github.com/QuantLet/TSA/tree/main/TSA_ch2/TSA_ch2_ma1}
\end{frame}

\begin{frame}{Modelul MA(1): definiție}
    \begin{cminipage}{0.95\textwidth}
        \vspace{-0.2cm}
        \begin{defn}[Proces MA(1)]
            \begin{itemize}\setlength{\itemsep}{0pt}
                \item Un \textbf{proces de medie mobilă de ordin 1} este: $X_t = \mu + \varepsilon_t + \theta \varepsilon_{t-1}$
                \item $\varepsilon_t \sim WN(0, \sigma^2)$
            \end{itemize}
        \end{defn}
        \vspace{-0.2cm}
        \begin{columns}[T]
            \begin{column}{0.48\textwidth}
                \begin{block}{Interpretare}
                    \begin{itemize}\setlength{\itemsep}{0pt}
                        \item $\mu$: media procesului
                        \item $\theta$: coeficient MA
                        \begin{itemize}
                            \item Măsoară impactul șocului trecut
                        \end{itemize}
                        \item Depinde de $\varepsilon_t$ și $\varepsilon_{t-1}$
                    \end{itemize}
                \end{block}
            \end{column}
            \begin{column}{0.48\textwidth}
                \begin{exampleblock}{Notație cu operatorul lag}
                    \begin{itemize}\setlength{\itemsep}{0pt}
                        \item $X_t = \mu + \theta(L)\varepsilon_t$
                        \item $\theta(L) = 1 + \theta L$
                    \end{itemize}
                \end{exampleblock}
                \vspace{-0.2cm}
                \begin{alertblock}{Proprietate}
                    \begin{itemize}\setlength{\itemsep}{0pt}
                        \item \textbf{Staționaritate garantată}: Procesele MA sunt întotdeauna staționare
                        \begin{itemize}
                            \item Nu depinde de valoarea lui $\theta$
                        \end{itemize}
                    \end{itemize}
                \end{alertblock}
            \end{column}
        \end{columns}
    \end{cminipage}
\end{frame}

\begin{frame}{Proprietățile MA(1)}
    \begin{cminipage}{0.95\textwidth}
        \vspace{-0.3cm}
        {\small
        \begin{block}{MA(1): $X_t = \mu + \varepsilon_t + \theta \varepsilon_{t-1}$}
            \begin{itemize}\setlength{\itemsep}{0pt}
                \item \textbf{Media}: $\E[X_t] = \mu$; \quad \textbf{Varianța}: $\gamma(0) = \sigma^2(1 + \theta^2)$
                \item \textbf{Autocovarianța}: $\gamma(1) = \theta\sigma^2$, $\gamma(h) = 0$ $(h > 1)$
                \item \textbf{ACF}: $\rho(1) = \frac{\theta}{1+\theta^2}$, $\rho(h) = 0$ $(h > 1)$
            \end{itemize}
        \end{block}
        \vspace{-0.3cm}
        \begin{alertblock}{De reținut}
            \begin{itemize}\setlength{\itemsep}{0pt}
                \item \textbf{Semnătura MA(1)}: ACF se anulează după lag 1
                \begin{itemize}
                    \item $\rho(1) \neq 0$, dar $\rho(h) = 0$ pentru $h > 1$; tipar opus față de AR(1)
                \end{itemize}
            \end{itemize}
        \end{alertblock}
        }

    \end{cminipage}
    \quantlet{TSA\_ch2\_ma1}{https://github.com/QuantLet/TSA/tree/main/TSA_ch2/TSA_ch2_ma1}
\end{frame}

\begin{frame}{Demonstrație: varianța și autocovarianța MA(1)}
    \begin{cminipage}{0.95\textwidth}
        \vspace{-0.3cm}
        {\small
        \begin{block}{Punct de plecare: $X_t = \varepsilon_t + \theta\varepsilon_{t-1}$ (presupunând $\mu = 0$)}
            \begin{itemize}\setlength{\itemsep}{0pt}
                \item \textbf{Varianța}:
            \end{itemize}
            \vspace{-0.3cm}
            \begin{align*}
            \gamma(0) &= \Var(\varepsilon_t + \theta\varepsilon_{t-1}) = \sigma^2 + \theta^2\sigma^2 + 0 = \boxed{\sigma^2(1+\theta^2)}
            \end{align*}
        \end{block}
        \vspace{-0.4cm}
        \begin{exampleblock}{Autocovarianța la lag 1}
            \begin{itemize}\setlength{\itemsep}{0pt}
                \item $\gamma(1) = \Cov(\varepsilon_t + \theta\varepsilon_{t-1}, \varepsilon_{t-1} + \theta\varepsilon_{t-2})$
                \item $= \Cov(\varepsilon_t, \varepsilon_{t-1}) + \theta\Cov(\varepsilon_t, \varepsilon_{t-2}) + \theta\Cov(\varepsilon_{t-1}, \varepsilon_{t-1}) + \theta^2\Cov(\varepsilon_{t-1}, \varepsilon_{t-2})$
                \item $= 0 + 0 + \theta\sigma^2 + 0 = \boxed{\theta\sigma^2}$
            \end{itemize}
        \end{exampleblock}
        \vspace{-0.3cm}
        \begin{alertblock}{Autocovarianța la lag $h \geq 2$}
            \begin{itemize}\setlength{\itemsep}{0pt}
                \item Niciun termen $\varepsilon$ comun $\Rightarrow$ $\gamma(h) = 0$
            \end{itemize}
        \end{alertblock}
        }
    \end{cminipage}
\end{frame}

\begin{frame}{Demonstrație: maximul ACF pentru MA(1)}
    \begin{cminipage}{0.95\textwidth}
        \vspace{-0.3cm}
        {\small
        \begin{block}{Afirmație: $|\rho(1)| \leq 0.5$ pentru orice valoare a lui $\theta$}
            \begin{itemize}\setlength{\itemsep}{0pt}
                \item ACF la lag 1: $\rho(1) = \frac{\theta}{1+\theta^2}$
                \item Derivăm: $\frac{d\rho(1)}{d\theta} = \frac{1-\theta^2}{(1+\theta^2)^2} = 0$ $\Rightarrow$ $\theta = \pm 1$
                \item La aceste valori: $\rho(1)\big|_{\theta=1} = \frac{1}{2}$, $\rho(1)\big|_{\theta=-1} = -\frac{1}{2}$
            \end{itemize}
        \end{block}
        }
        \begin{exampleblock}{Implicație}
            \begin{itemize}\setlength{\itemsep}{0pt}
                \item \textbf{Test practic}: Dacă $|\hat{\rho}(1)| > 0.5$ din date, procesul \textbf{nu} este MA(1)
                \begin{itemize}
                    \item Maximul $|\rho(1)| = 0.5$ se atinge la $\theta = \pm 1$
                    \item Considerați modele AR sau ARMA ca alternative
                \end{itemize}
            \end{itemize}
        \end{exampleblock}
    \end{cminipage}
\end{frame}

\begin{frame}{Simulări MA(1): efectul lui $\theta$}
    \begin{cminipage}{0.95\textwidth}
        \vspace{-0.3cm}
        \begin{center}
            \includegraphics[width=0.82\textwidth, height=0.56\textheight, keepaspectratio]{ch2_ma1_simulations.pdf}
        \end{center}
        \vspace{-0.2cm}
        {\scriptsize
        \begin{block}{Interpretare}
            \begin{itemize}\setlength{\itemsep}{0pt}
                \item MA(1) este întotdeauna staționar indiferent de $\theta$ $\Rightarrow$ memorie finită de doar un lag
                \item $\theta$ pozitiv netezește seria; $\theta$ negativ creează fluctuații mai rapide
                \item Spre deosebire de AR(1), șocurile MA(1) afectează procesul doar pentru o perioadă
            \end{itemize}
        \end{block}
        }

    \end{cminipage}
    \quantlet{TSA\_ch2\_ma1}{https://github.com/QuantLet/TSA/tree/main/TSA_ch2/TSA_ch2_ma1}
\end{frame}

\begin{frame}{Demonstrație: ACF pentru MA(1)}
    \begin{cminipage}{0.95\textwidth}
        \vspace{-0.3cm}
        \begin{block}{Afirmație: $\rho(1) = \frac{\theta}{1+\theta^2}$ și $\rho(h) = 0$ pentru $h > 1$}
            \begin{itemize}\setlength{\itemsep}{0pt}
                \item MA(1) are autocorelație nenulă \textbf{doar} la lag 1
            \end{itemize}
        \end{block}
        \vspace{-0.3cm}
        {\small
        \begin{exampleblock}{Demonstrație}
            \begin{itemize}\setlength{\itemsep}{0pt}
                \item Fie $X_t = \varepsilon_t + \theta\varepsilon_{t-1}$.
                \item Autocorelația la lag 1:
                $\rho(1) = \frac{\gamma(1)}{\gamma(0)} = \frac{\theta\sigma^2}{\sigma^2(1+\theta^2)} = \boxed{\frac{\theta}{1+\theta^2}}$
                \item Pentru $h > 1$: $\gamma(h) = \Cov(\varepsilon_t + \theta\varepsilon_{t-1},\;\varepsilon_{t-h} + \theta\varepsilon_{t-h-1})$
                \item Termenii $\varepsilon_t, \varepsilon_{t-1}$ nu se suprapun cu $\varepsilon_{t-h}, \varepsilon_{t-h-1}$ când $h > 1$, deci $\boxed{\gamma(h) = 0}$
            \end{itemize}
        \end{exampleblock}
        }
        \vspace{-0.2cm}
        \begin{alertblock}{Consecință practică}
            \begin{itemize}\setlength{\itemsep}{0pt}
                \item ACF se anulează brusc după lag 1 $\Rightarrow$ semn distinctiv al proceselor MA(1)
            \end{itemize}
        \end{alertblock}
    \end{cminipage}
\end{frame}

\begin{frame}{ACF și PACF MA(1): comparație vizuală}
    \begin{cminipage}{0.95\textwidth}
        \vspace{-0.3cm}
        \begin{center}
            \includegraphics[width=0.82\textwidth, height=0.60\textheight, keepaspectratio]{ch2_ma1_acf_pacf.pdf}
        \end{center}
        \vspace{-0.2cm}
        {\scriptsize
        \begin{block}{Interpretare}
            \begin{itemize}\setlength{\itemsep}{0pt}
                \item \textbf{ACF}: Un singur vârf la lag 1, apoi se anulează $\Rightarrow$ semnătura tipică MA(1)
                \item \textbf{PACF}: Descreștere exponențială $\Rightarrow$ tipar opus față de AR(1)
                \item Această inversare diferențiază procesele MA de cele AR
            \end{itemize}
        \end{block}
        }

    \end{cminipage}
    \quantlet{TSA\_ch2\_ma1}{https://github.com/QuantLet/TSA/tree/main/TSA_ch2/TSA_ch2_ma1}
\end{frame}


\begin{frame}{Invertibilitate: ilustrație vizuală}
    \begin{cminipage}{0.95\textwidth}
        \vspace{-0.2cm}
        \begin{center}
            \includegraphics[width=0.92\textwidth, height=0.60\textheight, keepaspectratio]{ch2_def_invertibility.pdf}
        \end{center}
        \vspace{-0.3cm}
        {\scriptsize
        \begin{block}{Interpretare}
            \begin{itemize}\setlength{\itemsep}{0pt}
                \item \textbf{Stânga}: invertibilitatea necesită rădăcini în afara cercului unitate
                \item \textbf{Dreapta}: ponderile AR($\infty$) scad doar când $|\theta| < 1$
            \end{itemize}
        \end{block}
        }

    \end{cminipage}
    \quantlet{TSA\_ch2\_ma1}{https://github.com/QuantLet/TSA/tree/main/TSA_ch2/TSA_ch2_ma1}
\end{frame}

\begin{frame}{Invertibilitatea modelelor MA}
    \begin{cminipage}{0.95\textwidth}
        \vspace{-0.2cm}
        \begin{defn}[Invertibilitate]
            \begin{itemize}\setlength{\itemsep}{0pt}
                \item Un proces MA este \textbf{invertibil} dacă poate fi scris ca un proces AR infinit:
                \item $X_t = \mu + \sum_{j=1}^{\infty} \pi_j (X_{t-j} - \mu) + \varepsilon_t$
            \end{itemize}
        \end{defn}
        \vspace{-0.2cm}
        \begin{block}{Condiții de invertibilitate}
            \begin{itemize}\setlength{\itemsep}{0pt}
                \item \textbf{MA(1)}: Invertibil dacă $|\theta| < 1$
                \item \textbf{MA(q)}: Rădăcinile lui $\theta(z) = 0$ în afara cercului unitate
            \end{itemize}
        \end{block}
        \vspace{-0.2cm}
        \begin{exampleblock}{De ce contează invertibilitatea}
            \begin{itemize}\setlength{\itemsep}{0pt}
                \item Asigură reprezentare unică (fără invertibilitate, mai multe modele MA descriu aceleași date)
                \item Necesară pentru prognoză și estimare
                \item \textbf{Staționaritate} $\Rightarrow$ AR; \textbf{Invertibilitate} $\Rightarrow$ MA
            \end{itemize}
        \end{exampleblock}
    \end{cminipage}
\end{frame}

\begin{frame}{Demonstrație: invertibilitatea MA(1)}
    \begin{cminipage}{0.95\textwidth}
        \vspace{-0.3cm}
        \begin{block}{Afirmație}
            \begin{itemize}\setlength{\itemsep}{0pt}
                \item MA(1) este invertibil dacă și numai dacă $|\theta| < 1$
            \end{itemize}
        \end{block}
        \vspace{-0.3cm}
        {\small
        \begin{exampleblock}{Demonstrație}
            \begin{itemize}\setlength{\itemsep}{0pt}
                \item Din $X_t = \varepsilon_t + \theta\varepsilon_{t-1}$, izolăm: $\varepsilon_t = X_t - \theta\varepsilon_{t-1}$
                \item Substituție recursivă: $\varepsilon_t = X_t - \theta(X_{t-1} - \theta\varepsilon_{t-2}) = X_t - \theta X_{t-1} + \theta^2\varepsilon_{t-2}$
                \item Continuând: $\varepsilon_t = \sum_{j=0}^{n}(-\theta)^j X_{t-j} + (-\theta)^{n+1}\varepsilon_{t-n-1}$
                \item Dacă $|\theta| < 1$: $(-\theta)^{n+1} \to 0$, deci $\boxed{\varepsilon_t = \sum_{j=0}^{\infty}(-\theta)^j X_{t-j}}$
            \end{itemize}
        \end{exampleblock}
        }
        \vspace{-0.2cm}
        \begin{alertblock}{Concluzie}
            \begin{itemize}\setlength{\itemsep}{0pt}
                \item Seria geometrică converge $\iff |\theta| < 1$ $\Rightarrow$ MA(1) se scrie ca AR($\infty$)
            \end{itemize}
        \end{alertblock}
    \end{cminipage}
\end{frame}

\begin{frame}{MA(q): ilustrație vizuală}
    \begin{cminipage}{0.95\textwidth}
        \begin{center}
            \includegraphics[width=0.90\textwidth, height=0.60\textheight, keepaspectratio]{ch2_def_maq.pdf}
        \end{center}
        \vspace{-0.2cm}
        {\scriptsize
        \begin{block}{Observație}
            \begin{itemize}\setlength{\itemsep}{0pt}
                \item Proces MA(3): semnătura distinctivă $\Rightarrow$ ACF se anulează după lag $q$ ($\rho(h) = 0$ pentru $h > 3$)
            \end{itemize}
        \end{block}
        }

    \end{cminipage}
    \quantlet{TSA\_ch2\_acf\_pacf\_patterns}{https://github.com/QuantLet/TSA/tree/main/TSA_ch2/TSA_ch2_acf_pacf_patterns}
\end{frame}

\begin{frame}{Modelul MA(q): forma generală}
    \begin{cminipage}{0.95\textwidth}
        \vspace{-0.3cm}
        {\small
        \begin{defn}[Proces MA(q)]
            \begin{itemize}\setlength{\itemsep}{0pt}
                \item Un \textbf{proces de medie mobilă de ordin q}: $X_t = \mu + \varepsilon_t + \theta_1\varepsilon_{t-1} + \cdots + \theta_q\varepsilon_{t-q}$
                \item \textbf{Operator lag}: $X_t = \mu + \theta(L)\varepsilon_t$, unde $\theta(L) = 1 + \theta_1 L + \cdots + \theta_q L^q$
            \end{itemize}
        \end{defn}
        \vspace{-0.3cm}
        \begin{block}{Proprietăți}
            \begin{itemize}\setlength{\itemsep}{0pt}
                \item Întotdeauna staționar (varianță finită)
                \item ACF se anulează după lag $q$: $\rho(h) = 0$ pentru $h > q$; PACF scade gradual
                \item Invertibil dacă toate rădăcinile lui $\theta(z) = 0$ se află în afara cercului unitate
            \end{itemize}
        \end{block}
        }
    \end{cminipage}
\end{frame}

%=============================================================================
% SECȚIUNEA 4: MODELE ARMA
%=============================================================================
\section{Modele ARMA}

\begin{frame}{ARMA: ilustrație vizuală}
    \begin{cminipage}{0.95\textwidth}
        \vspace{-0.3cm}
        \begin{center}
            \includegraphics[width=0.95\textwidth, height=0.65\textheight, keepaspectratio]{ch2_def_arma.pdf}
        \end{center}
        \vspace{-0.3cm}
        {\scriptsize
        \begin{block}{Interpretare ARMA(1,1)}
            \begin{itemize}\setlength{\itemsep}{0pt}
                \item \textbf{Combină} persistența AR cu răspunsul la șocuri MA; ACF/PACF ambele descresc
                \item Nici ACF nici PACF nu se întrerup $\Rightarrow$ indicator distinctiv pentru modele mixte
            \end{itemize}
        \end{block}
        }

    \end{cminipage}
    \quantlet{TSA\_ch2\_arma}{https://github.com/QuantLet/TSA/tree/main/TSA_ch2/TSA_ch2_arma}
\end{frame}

\begin{frame}{Modelul ARMA(p,q): definiție}
    \begin{cminipage}{0.95\textwidth}
        \vspace{-0.2cm}
        \begin{defn}[Proces ARMA(p,q)]
            \begin{itemize}\setlength{\itemsep}{0pt}
                \item $X_t = c + \phi_1 X_{t-1} + \cdots + \phi_p X_{t-p} + \varepsilon_t + \theta_1\varepsilon_{t-1} + \cdots + \theta_q\varepsilon_{t-q}$
                \item \textbf{Formă compactă}: $\phi(L)X_t = c + \theta(L)\varepsilon_t$, unde $\mu = \frac{c}{1-\phi_1-\cdots-\phi_p}$
            \end{itemize}
        \end{defn}
        \vspace{-0.2cm}
        \begin{exampleblock}{Principiu}
            \begin{itemize}\setlength{\itemsep}{0pt}
                \item \textbf{Flexibilitate}: Combină componentele AR și MA
                \begin{itemize}
                    \item AR captează persistența; MA captează răspunsul la șocuri
                \end{itemize}
                \item \textbf{Parcimonie}: ARMA(1,1) poate fi mai bun decât AR(5) sau MA(5)
                \begin{itemize}
                    \item Mai puțini parametri, mai puțin risc de supraajustare
                \end{itemize}
            \end{itemize}
        \end{exampleblock}
    \end{cminipage}
\end{frame}

\begin{frame}{Structura modelului ARMA}
    \begin{cminipage}{0.95\textwidth}
        \begin{center}
            \includegraphics[width=0.88\textwidth, height=0.55\textheight, keepaspectratio]{ch2_arma_structure.pdf}
        \end{center}
        \vspace{-0.2cm}
        {\scriptsize
        \begin{block}{Componente}
            \begin{itemize}\setlength{\itemsep}{0pt}
                \item \textbf{AR}: valorile trecute $\Rightarrow$ persistență, memorie
                \item \textbf{MA}: șocuri trecute $\Rightarrow$ răspuns la inovații
                \item \textbf{Forma compactă}: $\phi(L)X_t = \theta(L)\varepsilon_t$
            \end{itemize}
        \end{block}
        }
        \quantlet{TSA\_ch2\_arma}{https://github.com/QuantLet/TSA/tree/main/TSA_ch2/TSA_ch2_arma}
    \end{cminipage}
\end{frame}

\begin{frame}{Cum funcționează simularea ARMA}
    \begin{cminipage}{0.95\textwidth}
        \vspace{-0.2cm}
        \begin{center}
            \includegraphics[width=0.82\textwidth, height=0.63\textheight, keepaspectratio]{ch2_arma_simulation.pdf}
        \end{center}
        \vspace{-0.3cm}
        \begin{block}{Pași}
            \begin{itemize}\setlength{\itemsep}{0pt}
                \item Generare zgomot alb, aplicare ecuația ARMA recursiv, obținere serie simulată
            \end{itemize}
        \end{block}

    \end{cminipage}
    \quantlet{TSA\_ch2\_arma}{https://github.com/QuantLet/TSA/tree/main/TSA_ch2/TSA_ch2_arma}
\end{frame}

\begin{frame}{Exemple ARMA}
    \begin{cminipage}{0.95\textwidth}
        \vspace{-0.2cm}
        \begin{center}
            \includegraphics[width=0.85\textwidth, height=0.63\textheight, keepaspectratio]{ch2_arma_examples.pdf}
        \end{center}
        \vspace{-0.3cm}
        \begin{block}{Observație}
            \begin{itemize}\setlength{\itemsep}{0pt}
                \item Diferite combinații de ordine $(p,q)$ produc comportamente distincte
            \end{itemize}
        \end{block}

    \end{cminipage}
    \quantlet{TSA\_ch2\_arma}{https://github.com/QuantLet/TSA/tree/main/TSA_ch2/TSA_ch2_arma}
\end{frame}

\begin{frame}{Modelul ARMA(1,1)}
    \begin{cminipage}{0.95\textwidth}
        \vspace{-0.3cm}
        {\small
        \begin{defn}[Proces ARMA(1,1)]
            \begin{itemize}\setlength{\itemsep}{0pt}
                \item $X_t = c + \phi X_{t-1} + \varepsilon_t + \theta\varepsilon_{t-1}$
            \end{itemize}
        \end{defn}
        \vspace{-0.3cm}
        \begin{block}{Proprietăți (staționaritate și invertibilitate)}
            \begin{itemize}\setlength{\itemsep}{0pt}
                \item \textbf{Media}: $\mu = \frac{c}{1-\phi}$; \textbf{Varianța}: $\gamma(0) = \frac{(1+2\phi\theta+\theta^2)\sigma^2}{1-\phi^2}$
            \end{itemize}
        \end{block}
        \vspace{-0.3cm}
        \begin{exampleblock}{ACF}
            \begin{itemize}\setlength{\itemsep}{0pt}
                \item $\rho(1) = \frac{(1+\phi\theta)(\phi+\theta)}{1+2\phi\theta+\theta^2}$; \quad $\rho(h) = \phi \cdot \rho(h-1)$ pentru $h \geq 2$
                \item ACF scade exponențial după lag 1 (punctul de pornire depinde de $\phi$ și $\theta$)
            \end{itemize}
        \end{exampleblock}
        }

    \end{cminipage}
    \quantlet{TSA\_ch2\_arma}{https://github.com/QuantLet/TSA/tree/main/TSA_ch2/TSA_ch2_arma}
\end{frame}

\begin{frame}{Demonstrație: varianța ARMA(1,1)}
    \begin{cminipage}{0.95\textwidth}
        \vspace{-0.3cm}
        \begin{block}{Afirmație}
            \begin{itemize}\setlength{\itemsep}{0pt}
                \item $\gamma(0) = \frac{(1 + 2\phi\theta + \theta^2)\sigma^2}{1 - \phi^2}$
            \end{itemize}
        \end{block}
        \vspace{-0.3cm}
        {\small
        \begin{exampleblock}{Demonstrație}
            \begin{itemize}\setlength{\itemsep}{0pt}
                \item Fie $Y_t = X_t - \mu$: $Y_t = \phi Y_{t-1} + \varepsilon_t + \theta\varepsilon_{t-1}$
                \item Ridicăm la pătrat: $Y_t^2 = \phi^2 Y_{t-1}^2 + \varepsilon_t^2 + \theta^2\varepsilon_{t-1}^2 + 2\phi Y_{t-1}\varepsilon_t + 2\phi\theta Y_{t-1}\varepsilon_{t-1} + 2\theta\varepsilon_t\varepsilon_{t-1}$
                \item Luăm media; $\E[\varepsilon_t Y_{t-1}] = 0$, $\E[\varepsilon_t\varepsilon_{t-1}] = 0$:
                \item $\gamma(0) = \phi^2\gamma(0) + \sigma^2 + \theta^2\sigma^2 + 2\phi\theta\,\E[\varepsilon_{t-1}Y_{t-1}]$
                \item Din $Y_{t-1} = \phi Y_{t-2} + \varepsilon_{t-1} + \theta\varepsilon_{t-2}$: doar $\varepsilon_{t-1}^2$ contribuie $\Rightarrow$ $\E[\varepsilon_{t-1}Y_{t-1}] = \sigma^2$
                \item $\gamma(0)(1-\phi^2) = (1 + 2\phi\theta + \theta^2)\sigma^2 \implies \boxed{\gamma(0) = \frac{(1+2\phi\theta+\theta^2)\sigma^2}{1-\phi^2}}$
            \end{itemize}
        \end{exampleblock}
        }
    \end{cminipage}
\end{frame}

\begin{frame}{Tipare ACF/PACF: AR vs MA vs ARMA}
    \begin{cminipage}{0.95\textwidth}
        \vspace{-0.2cm}
        \begin{center}
            \includegraphics[width=0.72\textwidth, height=0.75\textheight, keepaspectratio]{ch2_acf_pacf_summary.pdf}
        \end{center}

    \end{cminipage}
    \quantlet{TSA\_ch2\_acf\_pacf\_patterns}{https://github.com/QuantLet/TSA/tree/main/TSA_ch2/TSA_ch2_acf_pacf_patterns}
\end{frame}

\begin{frame}{Demonstrație: ACF la lag 1 pentru ARMA(1,1)}
    \begin{cminipage}{0.95\textwidth}
        \vspace{-0.3cm}
        \begin{block}{Afirmație}
            \begin{itemize}\setlength{\itemsep}{0pt}
                \item $\rho(1) = \frac{(1+\phi\theta)(\phi+\theta)}{1+2\phi\theta+\theta^2}$; \quad $\rho(h) = \phi\,\rho(h-1)$ pentru $h \geq 2$
            \end{itemize}
        \end{block}
        \vspace{-0.3cm}
        {\footnotesize
        \begin{exampleblock}{Demonstrație}
            \begin{itemize}\setlength{\itemsep}{0pt}
                \item Înmulțim $Y_t$ cu $Y_{t-1}$ și luăm media:
                $\gamma(1) = \phi\gamma(0) + \underbrace{\E[\varepsilon_t Y_{t-1}]}_{=0} + \theta\underbrace{\E[\varepsilon_{t-1}Y_{t-1}]}_{=\sigma^2} = \phi\gamma(0) + \theta\sigma^2$
                \item Împărțim la $\gamma(0)$: $\rho(1) = \phi + \frac{\theta\sigma^2}{\gamma(0)}$. Substituim $\gamma(0)$:
                \item $\rho(1) = \phi + \frac{\theta(1-\phi^2)}{1+2\phi\theta+\theta^2} = \frac{\phi(1+2\phi\theta+\theta^2)+\theta(1-\phi^2)}{1+2\phi\theta+\theta^2}$
                \item Numărătorul: $\phi+\theta+\phi^2\theta+\phi\theta^2 = (\phi+\theta)(1+\phi\theta)$, deci $\boxed{\rho(1) = \frac{(1+\phi\theta)(\phi+\theta)}{1+2\phi\theta+\theta^2}}$
            \end{itemize}
        \end{exampleblock}
        }
        \vspace{-0.3cm}
        {\footnotesize
        \begin{alertblock}{Recursivitate}
            \begin{itemize}\setlength{\itemsep}{0pt}
                \item Pentru $h \geq 2$: $\gamma(h) = \phi\gamma(h-1)$, deci $\rho(h) = \phi\,\rho(h-1)$ $\Rightarrow$ descreștere exponențială de la lag 1
            \end{itemize}
        \end{alertblock}
        }
    \end{cminipage}
\end{frame}

\begin{frame}{Funcții de răspuns la impuls}
    \begin{cminipage}{0.95\textwidth}
        \begin{center}
            \includegraphics[width=0.88\textwidth, height=0.68\textheight, keepaspectratio]{ch2_impulse_response.pdf}
        \end{center}
        \vspace{-0.2cm}
        {\scriptsize
        \begin{block}{Propagarea șocurilor}
            \begin{itemize}\setlength{\itemsep}{0pt}
                \item Arată cum se propagă un șoc unitar prin sistem în timp
                \item \textbf{AR}: descreștere exponențială sau oscilantă; \textbf{MA}: efect limitat la $q$ perioade
            \end{itemize}
        \end{block}
        }

    \end{cminipage}
    \quantlet{TSA\_ch2\_arma}{https://github.com/QuantLet/TSA/tree/main/TSA_ch2/TSA_ch2_arma}
\end{frame}

\begin{frame}{Rezumat staționaritate și invertibilitate}
    \begin{cminipage}{0.95\textwidth}
        \vspace{-0.2cm}
        \begin{block}{Condiții pentru un model ARMA(p,q) valid}
            \begin{itemize}\setlength{\itemsep}{0pt}
                \item Rezumat cerințe:
            \end{itemize}
            \vspace{-0.2cm}
            \begin{tabular}{ll}
                \toprule
                \textbf{Condiție} & \textbf{Cerință} \\
                \midrule
                Staționaritate & Rădăcinile lui $\phi(z) = 0$ în afara cercului unitate \\
                Invertibilitate & Rădăcinile lui $\theta(z) = 0$ în afara cercului unitate \\
                \bottomrule
            \end{tabular}
        \end{block}
        \vspace{-0.2cm}
        \begin{exampleblock}{Implicații}
            \begin{itemize}\setlength{\itemsep}{0pt}
                \item \textbf{Staționaritate}: Se poate scrie ca MA($\infty$): $X_t = \mu + \sum_{j=0}^{\infty} \psi_j \varepsilon_{t-j}$
                \item \textbf{Invertibilitate}: Se poate scrie ca AR($\infty$): $X_t = \mu + \sum_{j=1}^{\infty} \pi_j (X_{t-j}-\mu) + \varepsilon_t$
                \item \textbf{Reprezentare cauzală}: $X_t$ depinde doar de șocurile \textit{trecute} $\Rightarrow$ necesară pentru prognoză
            \end{itemize}
        \end{exampleblock}
    \end{cminipage}
\end{frame}

\begin{frame}{Teorema de descompunere a lui Wold}
    \vspace{-0.1cm}
    \begin{cminipage}{0.95\textwidth}
        \vspace{0.1cm}
        \begin{center}
            \includegraphics[width=0.70\textwidth, height=0.55\textheight, keepaspectratio]{ch2_wold_representation.pdf}
        \end{center}
        \vspace{-0.2cm}
        \begin{block}{Teorema Wold (1938)}
            \begin{itemize}\setlength{\itemsep}{0pt}
                \item Orice proces \textbf{staționar} $\{X_t\}$ poate fi descompus unic:
                $X_t = \underbrace{\sum_{j=0}^{\infty} \psi_j \varepsilon_{t-j}}_{\text{componentă stochastică}} + \underbrace{D_t}_{\text{deterministă}}$
                \;\; unde $\psi_0 = 1$, $\sum_{j=0}^{\infty} \psi_j^2 < \infty$, $\varepsilon_t \sim WN(0, \sigma^2)$
                \item \textbf{Implicație}: orice proces staționar se poate scrie ca MA($\infty$)
                \item ARMA este o aproximare \textbf{parcimonioasă} a lui MA($\infty$) cu un număr finit de parametri
            \end{itemize}
        \end{block}
        \quantlet{TSA\_ch2\_arma}{https://github.com/QuantLet/TSA/tree/main/TSA_ch2/TSA_ch2_arma}
    \end{cminipage}
\end{frame}

%=============================================================================
% SECȚIUNEA 5: IDENTIFICAREA MODELULUI
%=============================================================================
\section{Identificarea modelului}

\begin{frame}{Metodologia Box-Jenkins}
    \begin{cminipage}{0.95\textwidth}
        \begin{center}
            \includegraphics[width=0.88\textwidth, height=0.55\textheight, keepaspectratio]{ch2_box_jenkins.pdf}
        \end{center}
        \vspace{-0.2cm}
        {\scriptsize
        \begin{block}{Abordare iterativă}
            \begin{itemize}\setlength{\itemsep}{0pt}
                \item \textbf{Pas 1}: Identificare (ACF/PACF, AIC/BIC)
                \item \textbf{Pas 2}: Estimare (MLE, Yule-Walker)
                \item \textbf{Pas 3}: Validare (reziduuri = zgomot alb?)
                \item \textbf{Pas 4}: Prognoză
            \end{itemize}
        \end{block}
        }
        \quantlet{TSA\_ch2\_case\_study}{https://github.com/QuantLet/TSA/tree/main/TSA_ch2/TSA_ch2_case_study}
    \end{cminipage}
\end{frame}

\begin{frame}{Reguli de identificare ACF/PACF}
    \begin{cminipage}{0.95\textwidth}
        \vspace{-0.2cm}
        {\scriptsize
        \begin{block}{Tipare teoretice pentru procese staționare}
            \vspace{-0.1cm}
            {\small
            \begin{center}
                \begin{tabular}{lll}
                    \toprule
                    \textbf{Model} & \textbf{Tipar ACF} & \textbf{Tipar PACF} \\
                    \midrule
                    AR(1) & Descreștere exponențială & Vârf la lag 1, apoi 0 \\
                    AR(2) & Exp./sinusoidă amortizată & Vârfuri la lag-uri 1-2, apoi 0 \\
                    AR(p) & Scade gradual & Se anulează după lag $p$ \\
                    \midrule
                    MA(1) & Vârf la lag 1, apoi 0 & Descreștere exponențială \\
                    MA(2) & Vârfuri la lag-uri 1-2, apoi 0 & Exp./sinusoidă amortizată \\
                    MA(q) & Se anulează după lag $q$ & Scade gradual \\
                    \midrule
                    ARMA(p,q) & Scade & Scade \\
                    \bottomrule
                \end{tabular}
            \end{center}
            }
        \end{block}
        }
        \vspace{-0.2cm}
        \begin{center}
            \includegraphics[width=0.95\textwidth, height=0.28\textheight, keepaspectratio]{ch2_model_identification.pdf}
        \end{center}
        \quantlet{TSA\_ch2\_acf\_pacf\_patterns}{https://github.com/QuantLet/TSA/tree/main/TSA_ch2/TSA_ch2_acf_pacf_patterns}
    \end{cminipage}
\end{frame}

\begin{frame}{AIC vs BIC: selecția modelului}
    \begin{cminipage}{0.95\textwidth}
        \begin{center}
            \includegraphics[width=0.88\textwidth, height=0.55\textheight, keepaspectratio]{ch2_aic_bic.pdf}
        \end{center}
        \vspace{-0.2cm}
        {\scriptsize
        \begin{block}{Interpretare}
            \begin{itemize}\setlength{\itemsep}{0pt}
                \item \textbf{Heatmap}: verde = AIC/BIC mic, $\square$ = cel mai bun model
                \item \textbf{Practică}: AIC $\Rightarrow$ prognoză, BIC $\Rightarrow$ identificare
            \end{itemize}
        \end{block}
        }
        \quantlet{TSA\_ch2\_model\_selection}{https://github.com/QuantLet/TSA/tree/main/TSA_ch2/TSA_ch2_model_selection}
    \end{cminipage}
\end{frame}

\begin{frame}{Criterii informaționale}
    \begin{cminipage}{0.95\textwidth}
        \vspace{-0.2cm}
        \begin{columns}[T]
            \begin{column}{0.48\textwidth}
                \begin{block}{AIC (Akaike)}
                    \begin{itemize}\setlength{\itemsep}{0pt}
                        \item $\text{AIC} = -2\ln(\hat{L}) + 2k$
                        \item Penalizare moderată
                        \begin{itemize}
                            \item Tinde să aleagă modele mai mari
                            \item Optim pentru prognoză
                        \end{itemize}
                    \end{itemize}
                \end{block}
            \end{column}
            \begin{column}{0.48\textwidth}
                \begin{block}{BIC (Bayesian)}
                    \begin{itemize}\setlength{\itemsep}{0pt}
                        \item $\text{BIC} = -2\ln(\hat{L}) + k\ln(n)$
                        \item Penalizare mai puternică
                        \begin{itemize}
                            \item Preferă modele parcimonioase
                            \item Consistent pentru identificare
                        \end{itemize}
                    \end{itemize}
                \end{block}
            \end{column}
        \end{columns}
        \vspace{-0.1cm}
        {\scriptsize
        \begin{itemize}\setlength{\itemsep}{0pt}
            \item \textbf{unde:} $\hat{L}$ = maximul funcției de verosimilitate, $k$ = nr.\ parametri estimați, $n$ = dimensiunea eșantionului
        \end{itemize}
        }
        \vspace{-0.1cm}
        \begin{alertblock}{Reguli}
            \begin{itemize}\setlength{\itemsep}{0pt}
                \item Valori mai mici = model mai bun. Comparați modele pe \textit{aceleași date}
            \end{itemize}
        \end{alertblock}
    \end{cminipage}
\end{frame}

\begin{frame}{Principiul parcimoniei: echilibrul bias-varianță}
    \begin{cminipage}{0.95\textwidth}
        \begin{center}
            \includegraphics[width=0.88\textwidth, height=0.55\textheight, keepaspectratio]{ch2_parsimony.pdf}
        \end{center}
        \vspace{-0.2cm}
        {\scriptsize
        \begin{alertblock}{Echilibrul bias-varianță}
            \begin{itemize}\setlength{\itemsep}{0pt}
                \item \textbf{Prea simplu}: bias mare (subajustare)
                \item \textbf{Prea complex}: varianță mare (supraajustare)
                \item \textbf{Optim}: intersecția curbelor; AIC/BIC ajută
            \end{itemize}
        \end{alertblock}
        }
        \quantlet{TSA\_ch2\_model\_selection}{https://github.com/QuantLet/TSA/tree/main/TSA_ch2/TSA_ch2_model_selection}
    \end{cminipage}
\end{frame}

\begin{frame}{Selecția automată a modelului}
    \begin{cminipage}{0.95\textwidth}
        \vspace{-0.3cm}
        {\small
        \begin{block}{Abordarea grid search}
            \begin{itemize}\setlength{\itemsep}{0pt}
                \item Estimați ARMA(p,q) pentru $p = 0,\ldots,p_{max}$ și $q = 0,\ldots,q_{max}$
                \item Selectați modelul cu cel mai mic AIC sau BIC; verificați cu teste de validare
            \end{itemize}
        \end{block}
        \vspace{-0.3cm}
        \begin{exampleblock}{În Python}
            \begin{itemize}\setlength{\itemsep}{0pt}
                \item \texttt{pm.auto\_arima()} din pachetul \texttt{pmdarima}
                \item Testează automat staționaritatea, parcurge ordine $(p,q)$, returnează cel mai bun model
            \end{itemize}
        \end{exampleblock}
        \vspace{-0.3cm}
        \begin{alertblock}{Atenție}
            \begin{itemize}\setlength{\itemsep}{0pt}
                \item Selecția automată nu este răspunsul final $\Rightarrow$ verificați validitatea modelului
                \item Auto-ARIMA complet (inclusiv selecția lui $d$) $\Rightarrow$ Capitolul 3
            \end{itemize}
        \end{alertblock}
        }
    \end{cminipage}
\end{frame}

%=============================================================================
% SECȚIUNEA 6: ESTIMAREA PARAMETRILOR
%=============================================================================
\section{Estimarea parametrilor}

\begin{frame}{Metode de estimare}
    \begin{cminipage}{0.95\textwidth}
        \begin{center}
            \includegraphics[width=0.88\textwidth, height=0.45\textheight, keepaspectratio]{ch2_estimation_comparison.pdf}
        \end{center}
        \vspace{-0.2cm}
        {\scriptsize
        \begin{block}{Cele trei abordări principale}
            \begin{itemize}\setlength{\itemsep}{0pt}
                \item \textbf{Yule-Walker}: formă închisă, doar AR; egalează autocorelațiile din eșantion cu cele teoretice
                \item \textbf{MLE}: cea mai eficientă și consistentă; necesită ipoteză de distribuție (Gaussiană)
                \item \textbf{Cele mai mici pătrate condiționate}: compromis; minimizează suma pătratelor reziduurilor
            \end{itemize}
        \end{block}
        }
        \quantlet{TSA\_ch2\_estimation}{https://github.com/QuantLet/TSA/tree/main/TSA_ch2/TSA_ch2_estimation}
    \end{cminipage}
\end{frame}

\begin{frame}{Ecuațiile Yule-Walker pentru AR(p)}
    \begin{cminipage}{0.95\textwidth}
        \begin{center}
            \includegraphics[width=0.88\textwidth, height=0.55\textheight, keepaspectratio]{ch2_yule_walker.pdf}
        \end{center}
        \vspace{-0.2cm}
        {\scriptsize
        \begin{exampleblock}{Ideea principală}
            \begin{itemize}\setlength{\itemsep}{0pt}
                \item \textbf{Relație liniară}: autocorelații $\leftrightarrow$ parametri AR
                \item \textbf{Formă închisă}: fără optimizare numerică, matrice Toeplitz
            \end{itemize}
        \end{exampleblock}
        }
        \quantlet{TSA\_ch2\_estimation}{https://github.com/QuantLet/TSA/tree/main/TSA_ch2/TSA_ch2_estimation}
    \end{cminipage}
\end{frame}

\begin{frame}{Ecuațiile Yule-Walker: forma matriceală}
    \begin{cminipage}{0.95\textwidth}
        \vspace{-0.3cm}
        {\small
        \begin{block}{Ecuațiile Yule-Walker pentru AR(p)}
            \begin{itemize}\setlength{\itemsep}{0pt}
                \item $\rho(k) = \phi_1\rho(k-1) + \phi_2\rho(k-2) + \cdots + \phi_p\rho(k-p)$, \quad $k = 1, 2, \ldots, p$
            \end{itemize}
        \end{block}
        \vspace{-0.3cm}
        \begin{exampleblock}{Forma matriceală}
            \begin{itemize}\setlength{\itemsep}{0pt}
                \item $\begin{pmatrix} \rho(0) & \rho(1) & \cdots & \rho(p-1) \\ \rho(1) & \rho(0) & \cdots & \rho(p-2) \\ \vdots & \vdots & \ddots & \vdots \\ \rho(p-1) & \rho(p-2) & \cdots & \rho(0) \end{pmatrix}
                \begin{pmatrix} \phi_1 \\ \phi_2 \\ \vdots \\ \phi_p \end{pmatrix} =
                \begin{pmatrix} \rho(1) \\ \rho(2) \\ \vdots \\ \rho(p) \end{pmatrix}$
                \item \textbf{Estimare}: Înlocuiți $\rho(k)$ cu $\hat{\rho}(k)$; matricea Toeplitz este simetrică și pozitiv definită
            \end{itemize}
        \end{exampleblock}
        }
    \end{cminipage}
\end{frame}

\begin{frame}{Exemplu numeric: Yule-Walker pentru AR(2)}
    \begin{cminipage}{0.95\textwidth}
        \vspace{-0.3cm}
        {\scriptsize
        \begin{exampleblock}{Date din eșantion ($T = 100$)}
            \begin{itemize}\setlength{\itemsep}{0pt}
                \item \textbf{Autocorelații estimate}: $\hat{\rho}(1) = 0.75$, $\hat{\rho}(2) = 0.65$, \quad varianța estimată: $\hat{\gamma}(0) = 4.0$
            \end{itemize}
        \end{exampleblock}
        \vspace{-0.3cm}
        \begin{block}{Pas 1: Sistemul matriceal}
            \begin{itemize}\setlength{\itemsep}{0pt}
                \item \textbf{Yule-Walker}: $\mathbf{R}\hat{\bm{\phi}} = \bm{\rho}$ \quad $\Rightarrow$ \quad $\begin{pmatrix} 1 & 0.75 \\ 0.75 & 1 \end{pmatrix} \begin{pmatrix} \hat{\phi}_1 \\ \hat{\phi}_2 \end{pmatrix} = \begin{pmatrix} 0.75 \\ 0.65 \end{pmatrix}$
            \end{itemize}
        \end{block}
        \vspace{-0.3cm}
        \begin{block}{Pas 2: Rezolvare (regula lui Cramer)}
            \begin{itemize}\setlength{\itemsep}{0pt}
                \item $\det(\mathbf{R}) = 1 - 0.75^2 = 0.4375$
                \item $\hat{\phi}_1 = \frac{0.75 \times 1 - 0.75 \times 0.65}{0.4375} = \frac{0.2625}{0.4375} = \boxed{0.600}$ \qquad
                $\hat{\phi}_2 = \frac{0.65 \times 1 - 0.75 \times 0.75}{0.4375} = \frac{0.0875}{0.4375} = \boxed{0.200}$
            \end{itemize}
        \end{block}
        \vspace{-0.3cm}
        \begin{block}{Pas 3: Varianța zgomotului}
            \begin{itemize}\setlength{\itemsep}{0pt}
                \item $\hat{\sigma}^2 = \hat{\gamma}(0)(1 - \hat{\phi}_1\hat{\rho}(1) - \hat{\phi}_2\hat{\rho}(2)) = 4.0(1 - 0.45 - 0.13) = \boxed{1.68}$
            \end{itemize}
        \end{block}
        \vspace{-0.3cm}
        \begin{alertblock}{Verificare staționaritate}
            \begin{itemize}\setlength{\itemsep}{0pt}
                \item $\hat{\phi}_1 + \hat{\phi}_2 = 0.8 < 1$ \checkmark \quad $|\hat{\phi}_2| = 0.2 < 1$ \checkmark \quad $\hat{\phi}_2 - \hat{\phi}_1 = -0.4 < 1$ \checkmark
            \end{itemize}
        \end{alertblock}
        }
    \end{cminipage}
\end{frame}

\begin{frame}{Demonstrație: ecuațiile Yule-Walker}
    \begin{cminipage}{0.95\textwidth}
        \vspace{-0.3cm}
        {\small
        \begin{block}{Scop: Derivarea $\rho(k) = \phi_1\rho(k-1) + \cdots + \phi_p\rho(k-p)$}
            \begin{itemize}\setlength{\itemsep}{0pt}
                \item Pornim de la AR(p): $X_t = \phi_1 X_{t-1} + \cdots + \phi_p X_{t-p} + \varepsilon_t$
                \item Înmulțim cu $X_{t-k}$ și luăm media:
                \item $\E[X_t X_{t-k}] = \phi_1 \E[X_{t-1} X_{t-k}] + \cdots + \phi_p \E[X_{t-p} X_{t-k}] + \E[\varepsilon_t X_{t-k}]$
                \item Pentru $k \geq 1$: $\E[\varepsilon_t X_{t-k}] = 0$ $\Rightarrow$ $\gamma(k) = \phi_1 \gamma(k-1) + \cdots + \phi_p \gamma(k-p)$
                \item Împărțind la $\gamma(0)$: $\boxed{\rho(k) = \phi_1 \rho(k-1) + \phi_2 \rho(k-2) + \cdots + \phi_p \rho(k-p)}$
            \end{itemize}
        \end{block}
        }
        \begin{exampleblock}{Cazul special AR(1)}
            \begin{itemize}\setlength{\itemsep}{0pt}
                \item $\rho(k) = \phi_1 \rho(k-1) = \phi_1^k$ (folosind $\rho(0) = 1$)
            \end{itemize}
        \end{exampleblock}
    \end{cminipage}
\end{frame}

\begin{frame}{Estimarea prin metoda verosimilității maxime}
    \begin{cminipage}{0.95\textwidth}
        \vspace{-0.3cm}
        {\small
        \begin{block}{Logaritmul funcției de verosimilitate ARMA(p,q) (erori gaussiene: $\varepsilon_t \sim N(0, \sigma^2)$)}
            \begin{itemize}\setlength{\itemsep}{0pt}
                \item $\ell(\bm{\phi}, \bm{\theta}, \sigma^2) = -\frac{n}{2}\ln(2\pi) - \frac{n}{2}\ln(\sigma^2) - \frac{1}{2\sigma^2}\sum_{t=1}^{n}\varepsilon_t^2$
                \item $\varepsilon_t$ sunt inovațiile calculate recursiv
            \end{itemize}
        \end{block}
        \vspace{-0.3cm}
        \begin{exampleblock}{Procedura de estimare}
            \begin{itemize}\setlength{\itemsep}{0pt}
                \item Inițializare: folosiți metoda momentelor sau OLS (Metoda Celor Mai Mici Pătrate) pentru valori inițiale
                \item Optimizare: metode numerice (BFGS, Newton-Raphson)
                \item Iterare până la convergență
            \end{itemize}
        \end{exampleblock}
        \vspace{-0.3cm}
        \begin{alertblock}{În practică}
            \begin{itemize}\setlength{\itemsep}{0pt}
                \item \texttt{statsmodels.tsa.arima.model.ARIMA} $\Rightarrow$ implementează MLE exact cu inițializare automată
            \end{itemize}
        \end{alertblock}
        }
    \end{cminipage}
\end{frame}

\begin{frame}{Erori standard și inferență}
    \begin{cminipage}{0.95\textwidth}
        \vspace{-0.3cm}
        {\footnotesize
        \begin{block}{Distribuția asimptotică a MLE}
            \vspace{-0.1cm}
            \begin{itemize}\setlength{\itemsep}{0pt}\setlength{\parsep}{0pt}
                \item $\hat{\bm{\theta}} \xrightarrow{d} N\!\left(\bm{\theta}_0, \frac{1}{n}\mathbf{I}(\bm{\theta}_0)^{-1}\right)$, unde $\mathbf{I}(\bm{\theta})$ este \textbf{matricea informațională Fisher}
                \item $\mathbf{I}(\bm{\theta}) = -E\!\left[\frac{\partial^2 \ln L(\bm{\theta})}{\partial \bm{\theta} \, \partial \bm{\theta}'}\right]$ $\Rightarrow$ curbura medie a funcției de verosimilitate
                \item Matricea de varianță-covarianță estimată: $\hat{\mathbf{V}} = \frac{1}{n}\hat{\mathbf{I}}^{-1}$
            \end{itemize}
            \vspace{-0.1cm}
        \end{block}
        \vspace{-0.15cm}
        \begin{exampleblock}{Ce este eroarea standard (SE)?}
            \vspace{-0.1cm}
            \begin{itemize}\setlength{\itemsep}{0pt}\setlength{\parsep}{0pt}
                \item $SE(\hat{\theta}_j) = \sqrt{\hat{\mathbf{V}}_{jj}} = \sqrt{\text{diag}_j\!\left(\frac{1}{n}\hat{\mathbf{I}}^{-1}\right)}$ $\Rightarrow$ măsoară incertitudinea estimării
                \item \textbf{Exemplu AR(1)}: $SE(\hat{\phi}) \approx \sqrt{(1-\hat{\phi}^2)/n}$; pentru $\hat{\phi}=0.8$, $n=100$: $SE \approx 0.06$
                \item \textbf{Interpretare}: un SE mic $\Rightarrow$ parametrul este estimat cu precizie ridicată
            \end{itemize}
            \vspace{-0.1cm}
        \end{exampleblock}
        \vspace{-0.15cm}
        \begin{alertblock}{Testarea semnificației parametrilor}
            \vspace{-0.1cm}
            \begin{itemize}\setlength{\itemsep}{0pt}\setlength{\parsep}{0pt}
                \item $H_0: \theta_j = 0$ \quad Statistică: $z = \frac{\hat{\theta}_j}{SE(\hat{\theta}_j)} \sim N(0,1)$ asimptotic
                \item Respingeți dacă $|z| > 1.96$ la 5\% \quad $\Rightarrow$ \textbf{IC}: $\hat{\theta}_j \pm 1.96 \cdot SE(\hat{\theta}_j)$
            \end{itemize}
            \vspace{-0.1cm}
        \end{alertblock}
        }
    \end{cminipage}
\end{frame}

%=============================================================================
% SECȚIUNEA 7: DIAGNOSTICUL MODELULUI
%=============================================================================
\section{Diagnosticul modelului}

\begin{frame}{Diagnosticarea reziduurilor}
    \begin{cminipage}{0.95\textwidth}
        \vspace{-0.3cm}
        \begin{center}
            \includegraphics[width=0.82\textwidth, height=0.48\textheight, keepaspectratio]{ch2_diagnostics.pdf}
        \end{center}
        \vspace{-0.4cm}
        {\scriptsize
        \begin{block}{Dacă modelul este corect specificat, reziduurile trebuie să fie zgomot alb}
            \begin{itemize}\setlength{\itemsep}{0pt}
                \item \textbf{Graficul reziduurilor}: fluctuații aleatorii în jurul lui zero, varianță constantă
                \item \textbf{ACF reziduurilor}: fără vârfuri semnificative $\Rightarrow$ zgomot alb
                \item \textbf{Graficul Q-Q}: punctele pe diagonală $\Rightarrow$ distribuție normală; cozi groase $\Rightarrow$ erori non-normale
            \end{itemize}
        \end{block}
        \vspace{-0.4cm}
        \begin{alertblock}{Decizie}
            \begin{itemize}\setlength{\itemsep}{0pt}
                \item \textcolor{Forest}{$\checkmark$ Toate verificările OK} $\Rightarrow$ model adecvat \qquad \textcolor{Crimson}{$\times$ Nesatisfăcut} $\Rightarrow$ reveniți la identificare
            \end{itemize}
        \end{alertblock}
        }
    \end{cminipage}
    \quantlet{TSA\_ch2\_diagnostics}{https://github.com/QuantLet/TSA/tree/main/TSA_ch2/TSA_ch2_diagnostics}
\end{frame}

\begin{frame}{Testul Ljung-Box: ilustrație vizuală}
    \begin{cminipage}{0.95\textwidth}
        \vspace{-0.2cm}
        \begin{center}
            \includegraphics[width=0.90\textwidth, height=0.60\textheight, keepaspectratio]{ch2_def_ljungbox.pdf}
        \end{center}
        \vspace{-0.3cm}
        {\scriptsize
        \begin{block}{Interpretare}
            \begin{itemize}\setlength{\itemsep}{0pt}
                \item \textbf{Stânga}: model bun $\Rightarrow$ reziduuri zgomot alb
                \item \textbf{Dreapta}: model inadecvat $\Rightarrow$ autocorelație reziduală $\Rightarrow$ re-specificare necesară
            \end{itemize}
        \end{block}
        }

    \end{cminipage}
    \quantlet{TSA\_ch2\_diagnostics}{https://github.com/QuantLet/TSA/tree/main/TSA_ch2/TSA_ch2_diagnostics}
\end{frame}

\begin{frame}{Testul Ljung-Box}
    \begin{cminipage}{0.95\textwidth}
        \vspace{-0.3cm}
        {\small
        \begin{defn}[Testul Ljung-Box]
            \begin{itemize}\setlength{\itemsep}{0pt}
                \item Testează dacă reziduurile sunt distribuite independent (fără autocorelație)
                \item \textbf{Statistică}: $Q(m) = n(n+2)\sum_{k=1}^{m}\frac{\hat{\rho}_k^2}{n-k}$
            \end{itemize}
        \end{defn}
        \vspace{-0.3cm}
        \begin{block}{Ipoteze și distribuție}
            \begin{itemize}\setlength{\itemsep}{0pt}
                \item $H_0$: Reziduurile sunt zgomot alb; $H_1$: Reziduurile sunt autocorelate
                \item Sub $H_0$, $Q(m) \sim \chi^2(m-p-q)$ aproximativ
            \end{itemize}
        \end{block}
        \vspace{-0.3cm}
        \begin{alertblock}{Decizie}
            \begin{itemize}\setlength{\itemsep}{0pt}
                \item \textcolor{Forest}{p-value $> 0.05$} $\Rightarrow$ nu respingem $H_0$ $\Rightarrow$ reziduurile sunt zgomot alb
                \item \textcolor{Crimson}{p-value $< 0.05$} $\Rightarrow$ autocorelație reziduală $\Rightarrow$ model inadecvat
            \end{itemize}
        \end{alertblock}
        }

    \end{cminipage}
    \quantlet{TSA\_ch2\_diagnostics}{https://github.com/QuantLet/TSA/tree/main/TSA_ch2/TSA_ch2_diagnostics}
\end{frame}

\begin{frame}{Lista de verificare a modelului}
    \begin{cminipage}{0.95\textwidth}
        \vspace{-0.3cm}
        {\footnotesize
        \begin{block}{Un model ARMA bun ar trebui să îndeplinească}
            \begin{itemize}\setlength{\itemsep}{0pt}
                \item \textbf{Staționaritate}: Rădăcinile AR în afara cercului unitate (\texttt{arroots})
                \item \textbf{Invertibilitate}: Rădăcinile MA în afara cercului unitate (\texttt{maroots})
                \item \textbf{Reziduuri zgomot alb}: Fără ACF semnificativ (testul Ljung-Box)
                \item \textbf{Reziduuri normale}: Grafic Q-Q, testul Jarque-Bera
                \item \textbf{Fără heteroscedasticitate}: Varianță constantă (testul ARCH)
                \item \textbf{Simplu}: Cel mai mic AIC/BIC dintre modelele adecvate
            \end{itemize}
        \end{block}
        \vspace{-0.3cm}
        \begin{alertblock}{Dacă verificările nu sunt satisfăcute}
            \begin{itemize}\setlength{\itemsep}{0pt}
                \item Reveniți la identificare, încercați ordine diferite
            \end{itemize}
        \end{alertblock}
        }
    \end{cminipage}
\end{frame}

%=============================================================================
% SECȚIUNEA 8: PROGNOZA
%=============================================================================
\section{Prognoza cu ARMA}

\begin{frame}{Prognoze punctuale}
    \begin{cminipage}{0.95\textwidth}
        \vspace{-0.3cm}
        {\small
        \begin{block}{Prognoză optimă: $\hat{X}_{n+h|n} = \E[X_{n+h} | X_n, X_{n-1}, \ldots]$}
            \begin{itemize}\setlength{\itemsep}{0pt}
                \item Speranța condiționată minimizează MSE (Eroarea Medie Pătratică)
            \end{itemize}
        \end{block}
        \vspace{-0.3cm}
        \begin{exampleblock}{AR(1): $X_t = c + \phi X_{t-1} + \varepsilon_t$}
            \begin{itemize}\setlength{\itemsep}{0pt}
                \item $\hat{X}_{n+1|n} = c + \phi X_n$; \quad $\hat{X}_{n+h|n} = \mu + \phi^h(X_n - \mu)$
                \item Prognozele converg la media $\mu$ când $h \to \infty$ (revenire la medie)
            \end{itemize}
        \end{exampleblock}
        \vspace{-0.3cm}
        \begin{block}{MA(1): $X_t = \mu + \varepsilon_t + \theta\varepsilon_{t-1}$}
            \begin{itemize}\setlength{\itemsep}{0pt}
                \item $\hat{X}_{n+1|n} = \mu + \theta\varepsilon_n$; \quad $\hat{X}_{n+h|n} = \mu$ pentru $h > 1$
            \end{itemize}
        \end{block}
        }
    \end{cminipage}
\end{frame}

\begin{frame}{Incertitudinea prognozei}
    \begin{cminipage}{0.95\textwidth}
        \vspace{-0.3cm}
        {\small
        \begin{block}{MSFE (Eroarea Medie Pătratică de Prognoză)}
            \begin{itemize}\setlength{\itemsep}{0pt}
                \item \textbf{Eroarea}: $e_{n+h|n} = X_{n+h} - \hat{X}_{n+h|n}$
                \item \textbf{MSFE}: $\text{MSFE}(h) = \sigma^2 \sum_{j=0}^{h-1}\psi_j^2$, unde $\psi_j$ sunt coeficienții MA($\infty$)
            \end{itemize}
        \end{block}
        \vspace{-0.3cm}
        \begin{exampleblock}{Pentru AR(1): $\psi_j = \phi^j$}
            \begin{itemize}\setlength{\itemsep}{0pt}
                \item $\text{MSFE}(h) = \sigma^2 \frac{1-\phi^{2h}}{1-\phi^2} \to \frac{\sigma^2}{1-\phi^2} = \Var(X_t)$
            \end{itemize}
        \end{exampleblock}
        \vspace{-0.3cm}
        \begin{alertblock}{Observație}
            \begin{itemize}\setlength{\itemsep}{0pt}
                \item Incertitudinea prognozei crește cu orizontul
                \item Converge la varianța necondiționată $\Var(X_t)$
            \end{itemize}
        \end{alertblock}
        }
    \end{cminipage}
\end{frame}

\begin{frame}{Prognoza ARMA cu intervale de încredere}
    \begin{cminipage}{0.95\textwidth}
        \vspace{-0.2cm}
        \begin{center}
            \includegraphics[width=0.88\textwidth, height=0.65\textheight, keepaspectratio]{ch2_arma_forecast.pdf}
        \end{center}
        \vspace{-0.2cm}
        {\scriptsize
        \begin{block}{Observație}
            \begin{itemize}\setlength{\itemsep}{0pt}
                \item Banda de încredere se lărgește cu orizontul $\Rightarrow$ convergență la intervalul necondiționat
            \end{itemize}
        \end{block}
        }

    \end{cminipage}
    \quantlet{TSA\_ch2\_forecasting}{https://github.com/QuantLet/TSA/tree/main/TSA_ch2/TSA_ch2_forecasting}
\end{frame}

\begin{frame}{Demonstrație: MSFE pentru AR(1)}
    \begin{cminipage}{0.95\textwidth}
        \vspace{-0.3cm}
        \begin{block}{Afirmație}
            \begin{itemize}\setlength{\itemsep}{0pt}
                \item $\text{MSFE}(h) = \sigma^2\frac{1-\phi^{2h}}{1-\phi^2}$ \quad și \quad $\text{MSFE}(\infty) = \gamma(0)$
            \end{itemize}
        \end{block}
        \vspace{-0.3cm}
        {\small
        \begin{exampleblock}{Demonstrație}
            \begin{itemize}\setlength{\itemsep}{0pt}
                \item Eroarea de prognoză la orizontul $h$: $e_{n+h|n} = X_{n+h} - \hat{X}_{n+h|n}$
                \item Prin substituție recursivă: $e_{n+h|n} = \sum_{j=0}^{h-1}\phi^j\varepsilon_{n+h-j}$
                \item $\text{MSFE}(h) = \E[e_{n+h|n}^2] = \sigma^2\sum_{j=0}^{h-1}\phi^{2j} = \boxed{\sigma^2\frac{1-\phi^{2h}}{1-\phi^2}}$
                \item Limita: $\text{MSFE}(\infty) = \frac{\sigma^2}{1-\phi^2} = \gamma(0)$ $\Rightarrow$ prognoza converge la media necondiționată
            \end{itemize}
        \end{exampleblock}
        }
        \vspace{-0.2cm}
        \begin{alertblock}{Interpretare}
            \begin{itemize}\setlength{\itemsep}{0pt}
                \item La orizonturi lungi, nu facem mai bine decât media necondiționată: IC $\to 2 \times 1.96\sqrt{\gamma(0)}$
            \end{itemize}
        \end{alertblock}
    \end{cminipage}
\end{frame}

\begin{frame}{Prognoza AR(1): revenirea la medie}
    \begin{cminipage}{0.95\textwidth}
        \vspace{-0.3cm}
        \begin{center}
            \includegraphics[width=0.82\textwidth, height=0.60\textheight, keepaspectratio]{ch2_ar1_forecast.pdf}
        \end{center}
        \vspace{-0.2cm}
        {\scriptsize
        \begin{block}{Proprietăți}
            \begin{itemize}\setlength{\itemsep}{0pt}
                \item Prognozele converg la media necondiționată $\mu$ pe măsură ce orizontul crește
                \item $|\phi|$ mai mare $\Rightarrow$ revenire mai lentă; IC se lărgesc cu orizontul
            \end{itemize}
        \end{block}
        }

    \end{cminipage}
    \quantlet{TSA\_ch2\_forecasting}{https://github.com/QuantLet/TSA/tree/main/TSA_ch2/TSA_ch2_forecasting}
\end{frame}

\begin{frame}{Varianța erorii de prognoză în funcție de orizont}
    \begin{cminipage}{0.95\textwidth}
        \vspace{-0.2cm}
        \begin{center}
            \includegraphics[width=0.88\textwidth, height=0.68\textheight, keepaspectratio]{ch2_forecast_error.pdf}
        \end{center}
        \vspace{-0.2cm}
        {\scriptsize
        \begin{block}{Observație}
            \begin{itemize}\setlength{\itemsep}{0pt}
                \item MSFE crește monoton cu orizontul $h$ $\Rightarrow$ convergență la $\Var(X_t)$ (limita predictibilității)
            \end{itemize}
        \end{block}
        }

    \end{cminipage}
    \quantlet{TSA\_ch2\_forecasting}{https://github.com/QuantLet/TSA/tree/main/TSA_ch2/TSA_ch2_forecasting}
\end{frame}

\begin{frame}{Intervale de încredere pentru prognoze}
    \begin{cminipage}{0.95\textwidth}
        \vspace{-0.2cm}
        \begin{block}{Formule}
            \begin{itemize}\setlength{\itemsep}{0pt}
                \item $X_{n+h} | X_n, \ldots \sim N\left(\hat{X}_{n+h|n}, \text{MSFE}(h)\right)$
                \item \textbf{IC $(1-\alpha)$}: $\hat{X}_{n+h|n} \pm z_{\alpha/2} \cdot \sqrt{\text{MSFE}(h)}$, unde $z_{\alpha/2} = 1.96$ pentru 95\%
            \end{itemize}
        \end{block}
        \vspace{-0.2cm}
        \begin{exampleblock}{Proprietăți}
            \begin{itemize}\setlength{\itemsep}{0pt}
                \item Intervalele se lărgesc pe măsură ce orizontul crește
                \begin{itemize}
                    \item Converg la intervalul necondiționat: $\mu \pm z_{\alpha/2}\sigma_X$
                \end{itemize}
                \item Lățimea depinde de parametrii modelului
                \begin{itemize}
                    \item Coeficienți AR mai mari $\Rightarrow$ intervale mai largi
                \end{itemize}
                \item \textbf{Python}: \texttt{model.get\_forecast(h).conf\_int()}
            \end{itemize}
        \end{exampleblock}
    \end{cminipage}
\end{frame}

\begin{frame}{Exemplu de prognoză train/validare/test}
    \begin{cminipage}{0.95\textwidth}
        \begin{center}
            \includegraphics[width=0.88\textwidth, height=0.60\textheight, keepaspectratio]{ch2_train_val_test.pdf}
        \end{center}
        \vspace{-0.2cm}
        {\scriptsize
        \begin{alertblock}{Bună practică}
            \begin{itemize}\setlength{\itemsep}{0pt}
                \item Evaluați întotdeauna prognozele pe date neutilizate la estimare
                \item Folosiți împărțire antrenare/validare/test
            \end{itemize}
        \end{alertblock}
        }

    \end{cminipage}
    \quantlet{TSA\_ch2\_forecasting}{https://github.com/QuantLet/TSA/tree/main/TSA_ch2/TSA_ch2_forecasting}
\end{frame}

\begin{frame}{Evaluarea prognozei}
    \begin{cminipage}{0.95\textwidth}
        \begin{columns}[T]
            \begin{column}{0.48\textwidth}
                \begin{block}{Testare în afara eșantionului}
                    \begin{itemize}\setlength{\itemsep}{0pt}
                        \item Împărțiți datele: antrenare + test
                        \item Generați prognoze pe test
                        \item Comparați cu valorile reale
                        \item \textbf{Fereastră mobilă}: re-estimați pe măsură ce sosesc date noi
                    \end{itemize}
                \end{block}
            \end{column}
            \begin{column}{0.48\textwidth}
                \begin{exampleblock}{Metrici de eroare}
                    \begin{itemize}\setlength{\itemsep}{0pt}
                        \item \textbf{MAE} (Eroarea Medie Absolută) $= \frac{1}{n}\sum|e_t|$
                        \begin{itemize}
                            \item Robust la valori extreme
                        \end{itemize}
                        \item \textbf{RMSE} (Rădăcina Erorii Medii Pătratice) $= \sqrt{\frac{1}{n}\sum e_t^2}$
                        \begin{itemize}
                            \item Penalizează erorile mari
                        \end{itemize}
                        \item \textbf{MAPE} (Eroarea Medie Absolută Procentuală) $= \frac{100}{n}\sum\left|\frac{e_t}{X_t}\right|$
                        \begin{itemize}
                            \item Procentual, interpretabil
                        \end{itemize}
                    \end{itemize}
                \end{exampleblock}
            \end{column}
        \end{columns}
    \end{cminipage}
\end{frame}

\begin{frame}{Prognoza cu fereastră mobilă (rolling forecast)}
    \begin{cminipage}{0.95\textwidth}
        \begin{center}
            \includegraphics[width=0.95\textwidth, height=0.55\textheight, keepaspectratio]{ch2_rolling_forecast.pdf}
        \end{center}
        \vspace{-0.2cm}
        {\scriptsize
        \begin{block}{Metodologia prognozei cu fereastră mobilă}
            \begin{itemize}\setlength{\itemsep}{0pt}
                \item \textbf{Fereastră fixă} (ultimele $w$ obs.) vs \textbf{expansivă} (toate datele)
                \item Generează prognoza 1-pas, actualizează fereastra, repetă
            \end{itemize}
        \end{block}
        }

    \end{cminipage}
    \quantlet{TSA\_ch2\_forecasting}{https://github.com/QuantLet/TSA/tree/main/TSA_ch2/TSA_ch2_forecasting}
\end{frame}

\begin{frame}{Fereastră mobilă vs prognoza multi-pas}
    \begin{cminipage}{0.95\textwidth}
        \begin{center}
            \includegraphics[width=0.95\textwidth, height=0.55\textheight, keepaspectratio]{ch2_rolling_vs_multistep.pdf}
        \end{center}
        \vspace{-0.2cm}
        {\scriptsize
        \begin{exampleblock}{Diferențe}
            \begin{itemize}\setlength{\itemsep}{0pt}
                \item \textbf{Fereastră mobilă 1-pas}: cel mai precis, re-estimare la fiecare pas
                \item \textbf{Multi-pas direct}: model separat pentru fiecare orizont
                \item \textbf{Recursiv}: acumulare de erori la orizonturi mari
            \end{itemize}
        \end{exampleblock}
        }

    \end{cminipage}
    \quantlet{TSA\_ch2\_forecasting}{https://github.com/QuantLet/TSA/tree/main/TSA_ch2/TSA_ch2_forecasting}
\end{frame}

\begin{frame}{Aplicație cu date reale: comparație prognoze}
    \begin{cminipage}{0.95\textwidth}
        \begin{center}
            \includegraphics[width=0.95\textwidth, height=0.55\textheight, keepaspectratio]{ch2_real_data_forecast.pdf}
        \end{center}
        \vspace{-0.2cm}
        {\scriptsize
        \begin{block}{Considerații practice}
            \begin{itemize}\setlength{\itemsep}{0pt}
                \item Date reale: posibilă nestaționaritate, rupturi structurale
                \item Comparați mai multe modele; folosiți validare pe fereastră mobilă
            \end{itemize}
        \end{block}
        }

    \end{cminipage}
    \quantlet{TSA\_ch2\_forecasting}{https://github.com/QuantLet/TSA/tree/main/TSA_ch2/TSA_ch2_forecasting}
\end{frame}

%=============================================================================
% SECȚIUNEA 9: IMPLEMENTARE PRACTICĂ
%=============================================================================
\section{Implementare practică}

\begin{frame}{Rezumat flux de lucru}
    \begin{cminipage}{0.95\textwidth}
        \vspace{-0.3cm}
        {\small
        \begin{block}{Pașii metodologiei Box-Jenkins}
            \begin{itemize}\setlength{\itemsep}{1pt}
                \item \textbf{1. Pregătirea datelor}: Verificați valori lipsă, valori aberante
                \begin{itemize}
                    \item Transformați dacă este necesar (logaritm, diferențiere)
                \end{itemize}
                \item \textbf{2. Verificarea staționarității}: Inspecție vizuală, teste formale (ADF -- Augmented Dickey-Fuller, KPSS)
                \begin{itemize}
                    \item Diferențiați dacă seria este nestaționară
                \end{itemize}
                \item \textbf{3. Identificarea modelului}: Tipare ACF/PACF
                \begin{itemize}
                    \item Grid search cu criterii informaționale (AIC, BIC)
                \end{itemize}
                \item \textbf{4. Estimare și validare}: Estimați modelul, verificați semnificația
                \begin{itemize}
                    \item Analiză reziduală, testul Ljung-Box
                \end{itemize}
                \item \textbf{5. Prognoză}: Prognoze punctuale cu IC
                \begin{itemize}
                    \item Validare în afara eșantionului
                \end{itemize}
            \end{itemize}
        \end{block}
        }
    \end{cminipage}
\end{frame}

%=============================================================================
% STUDIU DE CAZ: DATE REALE
%=============================================================================
\section{Studiu de caz: date reale}

\begin{frame}{Studiu de caz: petele solare (sunspots)}
    \begin{cminipage}{0.95\textwidth}
        \vspace{-0.2cm}
        \begin{center}
            \includegraphics[width=0.95\textwidth, height=0.62\textheight, keepaspectratio]{ch2_case_raw_data.pdf}
        \end{center}
        \vspace{-0.3cm}
        {\scriptsize
        \begin{block}{Descrierea datelor}
            \begin{itemize}\setlength{\itemsep}{0pt}
                \item Pete solare anuale (1700--2008): serie staționară cu cicluri de $\sim$11 ani
                \item Aplicăm metodologia Box-Jenkins completă
            \end{itemize}
        \end{block}
        }

    \end{cminipage}
    \quantlet{TSA\_ch2\_case\_study}{https://github.com/QuantLet/TSA/tree/main/TSA_ch2/TSA_ch2_case_study}
\end{frame}

\begin{frame}{Pasul 1: analiza ACF/PACF}
    \begin{cminipage}{0.95\textwidth}
        \begin{center}
            \includegraphics[width=0.95\textwidth, height=0.68\textheight, keepaspectratio]{ch2_case_acf_pacf.pdf}
        \end{center}
        \vspace{-0.2cm}
        {\scriptsize
        \begin{exampleblock}{Identificare}
            \begin{itemize}\setlength{\itemsep}{0pt}
                \item ACF sinusoidală (sugerează AR); PACF cu vârfuri la lag 1, 2, 9
                \item Modele candidate: AR(2) sau AR(9); serie staționară ($d=0$)
            \end{itemize}
        \end{exampleblock}
        }

    \end{cminipage}
    \quantlet{TSA\_ch2\_case\_study}{https://github.com/QuantLet/TSA/tree/main/TSA_ch2/TSA_ch2_case_study}
\end{frame}

\begin{frame}{Pasul 2: compararea modelelor}
    \begin{cminipage}{0.95\textwidth}
        \vspace{-0.2cm}
        \begin{center}
            \includegraphics[width=0.95\textwidth, height=0.60\textheight, keepaspectratio]{ch2_case_model_comparison.pdf}
        \end{center}
        \vspace{-0.3cm}
        \begin{alertblock}{Selecția modelului}
            \begin{itemize}\setlength{\itemsep}{0pt}
                \item Comparăm mai multe modele candidate folosind criteriul AIC
                \item Modelul \textbf{AR(9)} are cel mai mic AIC, capturând ciclul solar de 11 ani
            \end{itemize}
        \end{alertblock}

    \end{cminipage}
    \quantlet{TSA\_ch2\_case\_study}{https://github.com/QuantLet/TSA/tree/main/TSA_ch2/TSA_ch2_case_study}
\end{frame}

\begin{frame}{Pasul 3: diagnosticul modelului}
    \begin{cminipage}{0.95\textwidth}
        \begin{center}
            \includegraphics[width=0.95\textwidth, height=0.60\textheight, keepaspectratio]{ch2_case_diagnostics.pdf}
        \end{center}
        \vspace{-0.2cm}
        {\scriptsize
        \begin{block}{Diagnosticul AR(9)}
            \begin{itemize}\setlength{\itemsep}{0pt}
                \item Reziduuri: zgomot alb, medie zero, varianță constantă
                \item ACF fără structură reziduală; distribuție $\approx$ normală
            \end{itemize}
        \end{block}
        }

    \end{cminipage}
    \quantlet{TSA\_ch2\_case\_study}{https://github.com/QuantLet/TSA/tree/main/TSA_ch2/TSA_ch2_case_study}
\end{frame}

\begin{frame}{Pasul 4: prognoza}
    \begin{cminipage}{0.95\textwidth}
        \vspace{-0.2cm}
        \begin{center}
            \includegraphics[width=0.95\textwidth, height=0.62\textheight, keepaspectratio]{ch2_case_forecast.pdf}
        \end{center}
        \vspace{-0.3cm}
        {\scriptsize
        \begin{exampleblock}{Rezultate}
            \begin{itemize}\setlength{\itemsep}{0pt}
                \item AR(9) captează ciclicitatea; IC 95\% acoperă valorile reale
                \item RMSE $\approx$ 30
            \end{itemize}
        \end{exampleblock}
        }

    \end{cminipage}
    \quantlet{TSA\_ch2\_case\_study}{https://github.com/QuantLet/TSA/tree/main/TSA_ch2/TSA_ch2_case_study}
\end{frame}

%=============================================================================
\section{Utilizare IA}
%=============================================================================

\begin{frame}{Exercițiu AI: Gândire critică}
    \begin{cminipage}{0.95\textwidth}
        \vspace{-3mm}
        \begin{block}{\footnotesize Prompt de testat în ChatGPT / Claude / Copilot}
            {\footnotesize
            ``Descarcă de pe FRED indicele lunar al producției industriale din SUA (seria INDPRO) din 2010-01 până în 2024-12 (180 observații). Calculează diferențele logaritmice lunare (rate de creștere). Estimează un model ARMA, fă diagnosticul reziduurilor și prognozează pe 12 luni. Vreau cod Python complet cu grafice.''
            }
        \end{block}
        \vspace{-2mm}
        {\footnotesize
        \begin{block}{\footnotesize Exercițiu}
            \begin{enumerate}\setlength{\itemsep}{0pt}
                \item Rulați prompt-ul într-un LLM (Large Language Model) la alegere și analizați critic răspunsul.
                \item Verifică staționaritatea datelor \textit{înainte} de a estima ARMA? Justificați.
                \item Cum alege ordinele $p$ și $q$? Folosește ACF/PACF sau AIC/BIC?
                \item Reziduurile sunt testate corect? (Ljung-Box, Q-Q, heteroscedasticitate)
                \item Intervalele de încredere ale prognozei converg la media necondiționată?
            \end{enumerate}
        \end{block}
        }
        \vspace{-2mm}
        \begin{alertblock}{}
            {\footnotesize \textbf{Atenție}: Codul generat de AI poate rula fără erori și arăta profesional. \textit{Asta nu înseamnă că e corect.}}
        \end{alertblock}
    \end{cminipage}
\end{frame}

%=============================================================================
% REZUMAT
%=============================================================================
\section{Rezumat}

\begin{frame}{Rezumat}
    \begin{cminipage}{0.95\textwidth}
        \begin{block}{Ce am învățat în acest capitol}
            \begin{itemize}\setlength{\itemsep}{1pt}
                \item Modele autoregresive AR($p$)
                \begin{itemize}
                    \item Depind de $p$ valori trecute; staționaritate: rădăcini în afara cercului unitate; PACF se anulează la lag $p$
                \end{itemize}
                \item Modele de medie mobilă MA($q$)
                \begin{itemize}
                    \item Depind de $q$ șocuri trecute; întotdeauna staționare; ACF se anulează la lag $q$
                \end{itemize}
                \item Modele ARMA($p,q$) și metodologia Box-Jenkins
                \begin{itemize}
                    \item Combină AR și MA; identificare $\Rightarrow$ estimare $\Rightarrow$ validare $\Rightarrow$ prognoză
                \end{itemize}
                \item Prognoze cu intervale de încredere
                \begin{itemize}
                    \item Converg la media necondiționată; incertitudinea crește cu orizontul
                \end{itemize}
            \end{itemize}
        \end{block}
        \begin{exampleblock}{Concluzie}
            \begin{itemize}\setlength{\itemsep}{0pt}
                \item \textbf{Parcimonie}: Începeți cu modele simple (ordine mici), validați cu testul Ljung-Box și criterii informaționale (AIC, BIC)
            \end{itemize}
        \end{exampleblock}
    \end{cminipage}
\end{frame}
\begin{frame}{Ce urmează?}
    \begin{cminipage}{0.95\textwidth}
        \begin{center}
            \begin{minipage}{0.85\textwidth}
                \begin{block}{Capitolul 3: Modele ARIMA pentru date nestaționare}
                    \begin{itemize}\setlength{\itemsep}{0pt}
                        \item \textbf{Nestaționaritate}: tipuri, teste de rădăcină unitate (ADF, PP, KPSS)
                        \item \textbf{Diferențierea}: operatorul diferență și ordinul de integrare $d$
                        \item \textbf{ARIMA($p,d,q$)}: modele integrate pentru date nestaționare
                        \item \textbf{Auto-ARIMA}: selecție automată a modelului complet
                        \item \textbf{Studiu de caz}: Prognoza PIB SUA
                    \end{itemize}
                \end{block}
            \end{minipage}
        \end{center}

        \vspace{0.3cm}
        \begin{center}
            \Large\textcolor{MainBlue}{Întrebări?}
        \end{center}
    \end{cminipage}
\end{frame}

%=============================================================================
% QUIZ
%=============================================================================
\section{Quiz}

\begin{frame}{Întrebarea 1}
    \begin{cminipage}{0.95\textwidth}
        \begin{alertblock}{Întrebare}
            \begin{itemize}\setlength{\itemsep}{0pt}
                \item Pentru ce valoare a lui $\phi$ este procesul AR(1) $X_t = c + \phi X_{t-1} + \varepsilon_t$ staționar?
            \end{itemize}
        \end{alertblock}

        \vspace{0.3cm}

        \setlength{\leftmargini}{2cm}
        \begin{block}{Variante de răspuns}

            \textcolor{MainBlue}{\textbf{(A)}} $\phi = 1.2$ \qquad
            \textcolor{MainBlue}{\textbf{(B)}} $\phi = 1.0$ \qquad
            \textcolor{MainBlue}{\textbf{(C)}} $\phi = -0.8$ \qquad
            \textcolor{MainBlue}{\textbf{(D)}} $\phi = -1.5$

        \end{block}
    \end{cminipage}
\end{frame}

\begin{frame}{Întrebarea 1: Răspuns}
    \begin{cminipage}{0.95\textwidth}
        \begin{exampleblock}{Răspuns corect: (C) $\phi = -0.8$}
            \begin{itemize}\setlength{\itemsep}{0pt}
                \item AR(1) este staționar dacă și numai dacă $|\phi| < 1$
                \item Doar $|-0.8| = 0.8 < 1$
            \end{itemize}
        \end{exampleblock}
        \vspace{-0.1cm}
        \begin{center}
            \includegraphics[width=0.98\textwidth, height=0.58\textheight, keepaspectratio]{ch2_quiz_ar_stationarity.pdf}
        \end{center}

    \end{cminipage}
    \quantlet{TSA\_ch2\_ar1}{https://github.com/QuantLet/TSA/tree/main/TSA_ch2/TSA_ch2_ar1}
\end{frame}

\begin{frame}{Întrebarea 2}
    \begin{cminipage}{0.95\textwidth}
        \begin{alertblock}{Întrebare}
            \begin{itemize}\setlength{\itemsep}{0pt}
                \item Observați: ACF are vârf la lag 1, apoi se anulează. PACF scade gradual. Ce model?
            \end{itemize}
        \end{alertblock}

        \vspace{0.3cm}

        \setlength{\leftmargini}{2cm}
        \begin{block}{Variante de răspuns}

            \textcolor{MainBlue}{\textbf{(A)}} AR(1) \qquad
            \textcolor{MainBlue}{\textbf{(B)}} MA(1) \qquad
            \textcolor{MainBlue}{\textbf{(C)}} ARMA(1,1) \qquad
            \textcolor{MainBlue}{\textbf{(D)}} Zgomot alb

        \end{block}
    \end{cminipage}
\end{frame}

\begin{frame}{Întrebarea 2: Răspuns}
    \begin{cminipage}{0.95\textwidth}
        \begin{exampleblock}{Răspuns corect: (B) MA(1)}
            \begin{itemize}\setlength{\itemsep}{0pt}
                \item ACF se anulează $\Rightarrow$ proces MA
                \item PACF scade $\Rightarrow$ confirmă MA(1)
            \end{itemize}
        \end{exampleblock}
        \vspace{-0.1cm}
        \begin{center}
            \includegraphics[width=0.98\textwidth, height=0.58\textheight, keepaspectratio]{ch2_quiz_acf_pacf_patterns.pdf}
        \end{center}

    \end{cminipage}
    \quantlet{TSA\_ch2\_ma1}{https://github.com/QuantLet/TSA/tree/main/TSA_ch2/TSA_ch2_ma1}
\end{frame}

\begin{frame}{Întrebarea 3}
    \begin{cminipage}{0.95\textwidth}
        \begin{alertblock}{Întrebare}
            \begin{itemize}\setlength{\itemsep}{0pt}
                \item Este MA(1) $X_t = \varepsilon_t + 1.5\varepsilon_{t-1}$ invertibil?
            \end{itemize}
        \end{alertblock}

        \vspace{0.3cm}

        \setlength{\leftmargini}{2cm}
        \begin{block}{Variante de răspuns}

            \textcolor{MainBlue}{\textbf{(A)}} Da, procesele MA sunt întotdeauna invertibile\\[3pt]

            \textcolor{MainBlue}{\textbf{(B)}} Da, deoarece $1.5 > 0$\\[3pt]

            \textcolor{MainBlue}{\textbf{(C)}} Nu, deoarece $|\theta| = 1.5 > 1$\\[3pt]

            \textcolor{MainBlue}{\textbf{(D)}} Nu, procesele MA nu sunt niciodată invertibile

        \end{block}
    \end{cminipage}
\end{frame}

\begin{frame}{Întrebarea 3: Răspuns}
    \begin{cminipage}{0.95\textwidth}
        \begin{exampleblock}{Răspuns corect: (C) Nu, deoarece $|\theta| = 1.5 > 1$}
            \begin{itemize}\setlength{\itemsep}{0pt}
                \item Invertibilitatea necesită $|\theta| < 1$
                \item Aici $|\theta| = 1.5 > 1$, deci nu este invertibil
            \end{itemize}
        \end{exampleblock}
        \vspace{-0.1cm}
        \begin{center}
            \includegraphics[width=0.98\textwidth, height=0.58\textheight, keepaspectratio]{ch2_def_invertibility.pdf}
        \end{center}
    \end{cminipage}
    \quantlet{TSA\_ch2\_ma1}{https://github.com/QuantLet/TSA/tree/main/TSA_ch2/TSA_ch2_ma1}
\end{frame}

\begin{frame}{Întrebarea 4}
    \begin{cminipage}{0.95\textwidth}
        \begin{alertblock}{Întrebare}
            \begin{itemize}\setlength{\itemsep}{0pt}
                \item Forma compactă $\phi(L)X_t = \theta(L)\varepsilon_t$ reprezintă ce model?
            \end{itemize}
        \end{alertblock}

        \vspace{0.3cm}

        \setlength{\leftmargini}{2cm}
        \begin{block}{Variante de răspuns}

            \textcolor{MainBlue}{\textbf{(A)}} Model AR pur \qquad
            \textcolor{MainBlue}{\textbf{(B)}} Model MA pur \qquad
            \textcolor{MainBlue}{\textbf{(C)}} Model ARMA \qquad
            \textcolor{MainBlue}{\textbf{(D)}} Niciunul de mai sus

        \end{block}
    \end{cminipage}
\end{frame}

\begin{frame}{Întrebarea 4: Răspuns}
    \begin{cminipage}{0.95\textwidth}
        \begin{exampleblock}{Răspuns corect: (C) Model ARMA}
            \begin{itemize}\setlength{\itemsep}{0pt}
                \item $\phi(L)$ este polinomul AR, $\theta(L)$ este polinomul MA $\Rightarrow$ ARMA(p,q)
            \end{itemize}
        \end{exampleblock}
        \vspace{-0.1cm}
        \begin{center}
            \includegraphics[width=0.98\textwidth, height=0.58\textheight, keepaspectratio]{ch2_def_arma.pdf}
        \end{center}
    \end{cminipage}
    \quantlet{TSA\_ch2\_arma}{https://github.com/QuantLet/TSA/tree/main/TSA_ch2/TSA_ch2_arma}
\end{frame}

\begin{frame}{Întrebarea 5}
    \begin{cminipage}{0.95\textwidth}
        \begin{alertblock}{Întrebare}
            \begin{itemize}\setlength{\itemsep}{0pt}
                \item Ce este $(1-L)^2 X_t$?
            \end{itemize}
        \end{alertblock}

        \vspace{0.3cm}

        \setlength{\leftmargini}{2cm}
        \begin{block}{Variante de răspuns}

            \textcolor{MainBlue}{\textbf{(A)}} $X_t - X_{t-1}$ \qquad
            \textcolor{MainBlue}{\textbf{(B)}} $X_t - 2X_{t-1} + X_{t-2}$ \qquad
            \textcolor{MainBlue}{\textbf{(C)}} $X_t + X_{t-1} + X_{t-2}$ \qquad
            \textcolor{MainBlue}{\textbf{(D)}} $X_t - X_{t-2}$

        \end{block}
    \end{cminipage}
\end{frame}

\begin{frame}{Întrebarea 5: Răspuns}
    \begin{cminipage}{0.95\textwidth}
        \begin{exampleblock}{Răspuns corect: (B) $X_t - 2X_{t-1} + X_{t-2}$}
            \begin{itemize}\setlength{\itemsep}{0pt}
                \item $(1-L)^2 = 1 - 2L + L^2$
                \item $(1-L)^2 X_t = X_t - 2X_{t-1} + X_{t-2}$
            \end{itemize}
        \end{exampleblock}
        \vspace{-0.1cm}
        \begin{center}
            \includegraphics[width=0.98\textwidth, height=0.58\textheight, keepaspectratio]{ch2_lag_operator.pdf}
        \end{center}
    \end{cminipage}
    \quantlet{TSA\_ch2\_lag\_operator}{https://github.com/QuantLet/TSA/tree/main/TSA_ch2/TSA_ch2_lag_operator}
\end{frame}

\begin{frame}{Întrebarea 6}
    \begin{cminipage}{0.95\textwidth}
        \begin{alertblock}{Întrebare}
            \begin{itemize}\setlength{\itemsep}{0pt}
                \item Comparând ARMA(1,1) vs ARMA(2,1) folosind BIC, care este corect?
            \end{itemize}
        \end{alertblock}

        \vspace{0.3cm}

        \setlength{\leftmargini}{2cm}
        \begin{block}{Variante de răspuns}

            \textcolor{MainBlue}{\textbf{(A)}} BIC mai mic înseamnă întotdeauna prognoze mai bune\\[3pt]

            \textcolor{MainBlue}{\textbf{(B)}} BIC penalizează complexitatea mai puțin decât AIC\\[3pt]

            \textcolor{MainBlue}{\textbf{(C)}} Modelul cu BIC mai mic este preferat\\[3pt]

            \textcolor{MainBlue}{\textbf{(D)}} BIC poate compara doar modele cu același număr de parametri

        \end{block}
    \end{cminipage}
\end{frame}

\begin{frame}{Întrebarea 6: Răspuns}
    \begin{cminipage}{0.95\textwidth}
        \begin{exampleblock}{Răspuns corect: (C) Modelul cu BIC mai mic este preferat}
            \begin{itemize}\setlength{\itemsep}{0pt}
                \item BIC mai mic indică un echilibru mai bun între calitatea estimării și complexitate
                \item BIC penalizează complexitatea \textit{mai mult} decât AIC
            \end{itemize}
        \end{exampleblock}
        \vspace{-0.1cm}
        \begin{center}
            \includegraphics[width=0.98\textwidth, height=0.58\textheight, keepaspectratio]{ch2_quiz_information_criteria.pdf}
        \end{center}

    \end{cminipage}
    \quantlet{TSA\_ch2\_model\_selection}{https://github.com/QuantLet/TSA/tree/main/TSA_ch2/TSA_ch2_model_selection}
\end{frame}

\begin{frame}{Întrebarea 7}
    \begin{cminipage}{0.95\textwidth}
        \begin{alertblock}{Întrebare}
            \begin{itemize}\setlength{\itemsep}{0pt}
                \item După estimarea unui model ARMA, rulați testul Ljung-Box pe reziduuri și obțineți p-value = 0.03. Ce înseamnă asta?
            \end{itemize}
        \end{alertblock}

        \vspace{0.3cm}

        \setlength{\leftmargini}{2cm}
        \begin{block}{Variante de răspuns}

            \textcolor{MainBlue}{\textbf{(A)}} Modelul este adecvat, reziduurile sunt zgomot alb\\[3pt]

            \textcolor{MainBlue}{\textbf{(B)}} Modelul este inadecvat, reziduurile au autocorelație\\[3pt]

            \textcolor{MainBlue}{\textbf{(C)}} Trebuie să creșteți dimensiunea eșantionului\\[3pt]

            \textcolor{MainBlue}{\textbf{(D)}} Testul este neconcludent

        \end{block}
    \end{cminipage}
\end{frame}

\begin{frame}{Întrebarea 7: Răspuns}
    \begin{cminipage}{0.95\textwidth}
        \begin{exampleblock}{Răspuns corect: (B) Modelul este inadecvat}
            \begin{itemize}\setlength{\itemsep}{0pt}
                \item p-value $< 0.05$ respinge $H_0$ (zgomot alb)
                \item Indică autocorelație reziduală rămasă
            \end{itemize}
        \end{exampleblock}
        \vspace{-0.1cm}
        \begin{center}
            \includegraphics[width=0.98\textwidth, height=0.58\textheight, keepaspectratio]{ch2_quiz_ljung_box.pdf}
        \end{center}

    \end{cminipage}
    \quantlet{TSA\_ch2\_diagnostics}{https://github.com/QuantLet/TSA/tree/main/TSA_ch2/TSA_ch2_diagnostics}
\end{frame}

\begin{frame}{Întrebarea 8}
    \begin{cminipage}{0.95\textwidth}
        \begin{alertblock}{Întrebare}
            \begin{itemize}\setlength{\itemsep}{0pt}
                \item Pentru un model AR(1) staționar, ce se întâmplă cu prognozele când orizontul $h \to \infty$?
            \end{itemize}
        \end{alertblock}

        \vspace{0.3cm}

        \setlength{\leftmargini}{2cm}
        \begin{block}{Variante de răspuns}

            \textcolor{MainBlue}{\textbf{(A)}} Prognozele cresc nelimitat\\[3pt]

            \textcolor{MainBlue}{\textbf{(B)}} Prognozele oscilează la nesfârșit\\[3pt]

            \textcolor{MainBlue}{\textbf{(C)}} Prognozele converg la media necondiționată $\mu$\\[3pt]

            \textcolor{MainBlue}{\textbf{(D)}} Prognozele devin mai precise

        \end{block}
    \end{cminipage}
\end{frame}

\begin{frame}{Întrebarea 8: Răspuns}
    \begin{cminipage}{0.95\textwidth}
        \begin{exampleblock}{Răspuns corect: (C) Prognozele converg la $\mu$}
            \begin{itemize}\setlength{\itemsep}{0pt}
                \item $\hat{X}_{n+h|n} = \mu + \phi^h(X_n - \mu) \to \mu$ când $h \to \infty$ (deoarece $|\phi|<1$)
            \end{itemize}
        \end{exampleblock}
        \vspace{-0.1cm}
        \begin{center}
            \includegraphics[width=0.98\textwidth, height=0.58\textheight, keepaspectratio]{ch2_quiz_forecast_properties.pdf}
        \end{center}

    \end{cminipage}
    \quantlet{TSA\_ch2\_forecasting}{https://github.com/QuantLet/TSA/tree/main/TSA_ch2/TSA_ch2_forecasting}
\end{frame}

\begin{frame}{Întrebarea 9}
    \begin{cminipage}{0.95\textwidth}
        \begin{alertblock}{Întrebare}
            \begin{itemize}\setlength{\itemsep}{0pt}
                \item Fie un proces AR(1) cu $\phi = 0.6$ și $\sigma^2 = 4$. Cât este $\Var(X_t)$?
            \end{itemize}
        \end{alertblock}

        \vspace{0.3cm}

        \setlength{\leftmargini}{2cm}
        \begin{block}{Variante de răspuns}

            \textcolor{MainBlue}{\textbf{(A)}} 4.0 \qquad
            \textcolor{MainBlue}{\textbf{(B)}} 5.56 \qquad
            \textcolor{MainBlue}{\textbf{(C)}} 6.25 \qquad
            \textcolor{MainBlue}{\textbf{(D)}} 10.0

        \end{block}
    \end{cminipage}
\end{frame}

\begin{frame}{Întrebarea 9: Răspuns}
    \begin{cminipage}{0.95\textwidth}
        \begin{exampleblock}{Răspuns corect: (C) 6.25}
            \begin{itemize}\setlength{\itemsep}{0pt}
                \item $\Var(X_t) = \frac{\sigma^2}{1-\phi^2} = \frac{4}{1-0.36} = \frac{4}{0.64} = 6.25$
                \item Varianța procesului este mai mare decât $\sigma^2$ din cauza persistenței
            \end{itemize}
        \end{exampleblock}
        \vspace{-0.1cm}
        \begin{center}
            \includegraphics[width=0.98\textwidth, height=0.58\textheight, keepaspectratio]{ch2_quiz_ar1_variance.pdf}
        \end{center}
    \end{cminipage}
    \quantlet{TSA\_ch2\_ar1}{https://github.com/QuantLet/TSA/tree/main/TSA_ch2/TSA_ch2_ar1}
\end{frame}

\begin{frame}{Întrebarea 10}
    \begin{cminipage}{0.95\textwidth}
        \begin{alertblock}{Întrebare}
            \begin{itemize}\setlength{\itemsep}{0pt}
                \item Fie un proces MA(1) cu $\theta = 0.5$. Cât este $\rho(1)$?
            \end{itemize}
        \end{alertblock}

        \vspace{0.3cm}

        \setlength{\leftmargini}{2cm}
        \begin{block}{Variante de răspuns}

            \textcolor{MainBlue}{\textbf{(A)}} 0.50 \qquad
            \textcolor{MainBlue}{\textbf{(B)}} 0.40 \qquad
            \textcolor{MainBlue}{\textbf{(C)}} 0.25 \qquad
            \textcolor{MainBlue}{\textbf{(D)}} 0.33

        \end{block}
    \end{cminipage}
\end{frame}

\begin{frame}{Întrebarea 10: Răspuns}
    \begin{cminipage}{0.95\textwidth}
        \begin{exampleblock}{Răspuns corect: (B) 0.40}
            \begin{itemize}\setlength{\itemsep}{0pt}
                \item $\rho(1) = \frac{\theta}{1+\theta^2} = \frac{0.5}{1+0.25} = \frac{0.5}{1.25} = 0.40$
                \item Observați că $\rho(1) < \theta$ --- autocorelația este \textbf{întotdeauna} atenuată
            \end{itemize}
        \end{exampleblock}
        \vspace{-0.1cm}
        \begin{center}
            \includegraphics[width=0.98\textwidth, height=0.58\textheight, keepaspectratio]{ch2_quiz_ma1_acf.pdf}
        \end{center}
    \end{cminipage}
    \quantlet{TSA\_ch2\_ma1}{https://github.com/QuantLet/TSA/tree/main/TSA_ch2/TSA_ch2_ma1}
\end{frame}

\begin{frame}{Întrebarea 11}
    \begin{cminipage}{0.95\textwidth}
        \begin{alertblock}{Întrebare}
            \begin{itemize}\setlength{\itemsep}{0pt}
                \item Care afirmație despre ACF-ul unui ARMA(1,1) este \textbf{adevărată}?
            \end{itemize}
        \end{alertblock}

        \vspace{0.3cm}

        \setlength{\leftmargini}{2cm}
        \begin{block}{Variante de răspuns}

            \textcolor{MainBlue}{\textbf{(A)}} Se anulează după lag 1\\[3pt]

            \textcolor{MainBlue}{\textbf{(B)}} Descreștere exponențială începând de la lag 1, cu $\rho(1) \neq \phi$\\[3pt]

            \textcolor{MainBlue}{\textbf{(C)}} Este zero pentru toate lag-urile\\[3pt]

            \textcolor{MainBlue}{\textbf{(D)}} Urmează exact tiparul $\phi^h$ pentru orice $h \geq 0$

        \end{block}
    \end{cminipage}
\end{frame}

\begin{frame}{Întrebarea 11: Răspuns}
    \begin{cminipage}{0.95\textwidth}
        \begin{exampleblock}{Răspuns corect: (B) Descreștere exponențială de la lag 1, cu $\rho(1) \neq \phi$}
            \begin{itemize}\setlength{\itemsep}{0pt}
                \item $\rho(1) = \frac{(1+\phi\theta)(\phi+\theta)}{1+2\phi\theta+\theta^2} \neq \phi$ (componenta MA modifică lag-ul 1)
                \item Pentru $h \geq 2$: $\rho(h) = \phi\,\rho(h-1)$ --- descreștere exponențială ca la AR(1)
            \end{itemize}
        \end{exampleblock}
        \vspace{-0.1cm}
        \begin{center}
            \includegraphics[width=0.98\textwidth, height=0.58\textheight, keepaspectratio]{ch2_quiz_arma11_acf.pdf}
        \end{center}
    \end{cminipage}
    \quantlet{TSA\_ch2\_arma}{https://github.com/QuantLet/TSA/tree/main/TSA_ch2/TSA_ch2_arma}
\end{frame}

\begin{frame}{Întrebarea 12}
    \begin{cminipage}{0.95\textwidth}
        \begin{alertblock}{Întrebare}
            \begin{itemize}\setlength{\itemsep}{0pt}
                \item Un proces AR(2) are $\phi_1 = 0.8$ și $\phi_2 = 0.3$. Este staționar?
            \end{itemize}
        \end{alertblock}

        \vspace{0.3cm}

        \setlength{\leftmargini}{2cm}
        \begin{block}{Variante de răspuns}

            \textcolor{MainBlue}{\textbf{(A)}} Da, este staționar\\[3pt]

            \textcolor{MainBlue}{\textbf{(B)}} Nu, deoarece $\phi_1 + \phi_2 = 1.1 > 1$\\[3pt]

            \textcolor{MainBlue}{\textbf{(C)}} Nu se poate determina fără date\\[3pt]

            \textcolor{MainBlue}{\textbf{(D)}} Depinde de valoarea lui $\sigma^2$

        \end{block}
    \end{cminipage}
\end{frame}

\begin{frame}{Întrebarea 12: Răspuns}
    \begin{cminipage}{0.95\textwidth}
        \begin{exampleblock}{Răspuns corect: (B) Nu, deoarece $\phi_1 + \phi_2 = 1.1 > 1$}
            \begin{itemize}\setlength{\itemsep}{0pt}
                \item Condițiile necesare pentru staționaritate AR(2):
                \item $\phi_1 + \phi_2 < 1$, \; $\phi_2 - \phi_1 < 1$, \; $|\phi_2| < 1$
                \item Aici $0.8 + 0.3 = 1.1 > 1$ $\Rightarrow$ prima condiție este încălcată
            \end{itemize}
        \end{exampleblock}
        \vspace{-0.1cm}
        \begin{center}
            \includegraphics[width=0.98\textwidth, height=0.58\textheight, keepaspectratio]{ch2_quiz_ar2_check.pdf}
        \end{center}
    \end{cminipage}
    \quantlet{TSA\_ch2\_ar2}{https://github.com/QuantLet/TSA/tree/main/TSA_ch2/TSA_ch2_ar2}
\end{frame}

\begin{frame}{Întrebarea 13}
    \begin{cminipage}{0.95\textwidth}
        \begin{alertblock}{Întrebare}
            \begin{itemize}\setlength{\itemsep}{0pt}
                \item Ce garantează teorema descompunerii Wold?
            \end{itemize}
        \end{alertblock}

        \vspace{0.3cm}

        \setlength{\leftmargini}{2cm}
        \begin{block}{Variante de răspuns}

            \textcolor{MainBlue}{\textbf{(A)}} Orice serie de timp este un proces AR\\[3pt]

            \textcolor{MainBlue}{\textbf{(B)}} Orice proces staționar se poate scrie ca MA($\infty$): $X_t = \sum_{j=0}^{\infty}\psi_j\varepsilon_{t-j}$\\[3pt]

            \textcolor{MainBlue}{\textbf{(C)}} Orice proces are varianță finită\\[3pt]

            \textcolor{MainBlue}{\textbf{(D)}} Modelele ARMA sunt întotdeauna invertibile

        \end{block}
    \end{cminipage}
\end{frame}

\begin{frame}{Întrebarea 13: Răspuns}
    \begin{cminipage}{0.95\textwidth}
        \begin{exampleblock}{Răspuns corect: (B) Orice proces staționar $=$ MA($\infty$)}
            \begin{itemize}\setlength{\itemsep}{0pt}
                \item Teorema Wold: $X_t = \sum_{j=0}^{\infty}\psi_j\varepsilon_{t-j} + D_t$, unde $D_t$ este componenta deterministă
                \item Aceasta justifică modelele ARMA: sunt aproximări parcimonioase ale lui MA($\infty$)
            \end{itemize}
        \end{exampleblock}
        \vspace{-0.1cm}
        \begin{center}
            \includegraphics[width=0.98\textwidth, height=0.58\textheight, keepaspectratio]{ch2_quiz_wold.pdf}
        \end{center}
    \end{cminipage}
    \quantlet{TSA\_ch2\_arma}{https://github.com/QuantLet/TSA/tree/main/TSA_ch2/TSA_ch2_arma}
\end{frame}

\begin{frame}{Întrebarea 14}
    \begin{cminipage}{0.95\textwidth}
        \begin{alertblock}{Întrebare}
            \begin{itemize}\setlength{\itemsep}{0pt}
                \item AR(1) cu $\phi = 0.9$, $\sigma^2 = 1$. Ce se întâmplă cu lățimea IC pe măsură ce $h \to \infty$?
            \end{itemize}
        \end{alertblock}

        \vspace{0.3cm}

        \setlength{\leftmargini}{2cm}
        \begin{block}{Variante de răspuns}

            \textcolor{MainBlue}{\textbf{(A)}} Rămâne constantă\\[3pt]

            \textcolor{MainBlue}{\textbf{(B)}} Scade la zero\\[3pt]

            \textcolor{MainBlue}{\textbf{(C)}} Crește spre $2 \times 1.96 \times \sqrt{1/(1-0.81)} \approx 9.0$\\[3pt]

            \textcolor{MainBlue}{\textbf{(D)}} Crește la infinit

        \end{block}
    \end{cminipage}
\end{frame}

\begin{frame}{Întrebarea 14: Răspuns}
    \begin{cminipage}{0.95\textwidth}
        \begin{exampleblock}{Răspuns corect: (C) Crește spre $\approx 9.0$}
            \begin{itemize}\setlength{\itemsep}{0pt}
                \item $\text{MSFE}(\infty) = \frac{\sigma^2}{1-\phi^2} = \frac{1}{1-0.81} = \frac{1}{0.19} \approx 5.26$
                \item Lățimea IC $= 2 \times 1.96\sqrt{5.26} \approx 2 \times 1.96 \times 2.29 \approx 9.0$
            \end{itemize}
        \end{exampleblock}
        \vspace{-0.1cm}
        \begin{center}
            \includegraphics[width=0.98\textwidth, height=0.58\textheight, keepaspectratio]{ch2_quiz_ci_growth.pdf}
        \end{center}
    \end{cminipage}
    \quantlet{TSA\_ch2\_forecasting}{https://github.com/QuantLet/TSA/tree/main/TSA_ch2/TSA_ch2_forecasting}
\end{frame}

%=============================================================================
% BIBLIOGRAFIE
%=============================================================================
\section{Bibliografie}

\begin{frame}{Bibliografie I}
    \vspace{-0.3cm}
    \begin{cminipage}{0.95\textwidth}
        \begin{block}{Lucrări fundamentale ARMA}
            {\small
            \begin{itemize}\setlength{\itemsep}{0pt}
                \item Box, G.E.P., \& Jenkins, G.M. (1970). \textit{Time Series Analysis: Forecasting and Control}, Holden-Day.
                \item Akaike, H. (1974). A New Look at the Statistical Model Identification, \textit{IEEE Transactions on Automatic Control}, 19(6), 716--723.
                \item Schwarz, G. (1978). Estimating the Dimension of a Model, \textit{The Annals of Statistics}, 6(2), 461--464.
                \item Ljung, G.M., \& Box, G.E.P. (1978). On a Measure of Lack of Fit in Time Series Models, \textit{Biometrika}, 65(2), 297--303.
            \end{itemize}
            }
        \end{block}

        \begin{exampleblock}{Manuale clasice}
            {\small
            \begin{itemize}\setlength{\itemsep}{0pt}
                \item Hamilton, J.D. (1994). \textit{Time Series Analysis}, Princeton University Press.
                \item Brockwell, P.J., \& Davis, R.A. (2016). \textit{Introduction to Time Series and Forecasting}, 3rd ed., Springer.
                \item Shumway, R.H., \& Stoffer, D.S. (2017). \textit{Time Series Analysis and Its Applications}, 4th ed., Springer.
            \end{itemize}
            }
        \end{exampleblock}
    \end{cminipage}
\end{frame}

\begin{frame}{Bibliografie II}
    \begin{cminipage}{0.95\textwidth}
        \begin{block}{Referințe moderne}
            {\small
            \begin{itemize}\setlength{\itemsep}{0pt}
                \item Hyndman, R.J., \& Athanasopoulos, G. (2021). \textit{Forecasting: Principles and Practice}, 3rd ed., OTexts.
                \item Box, G.E.P., Jenkins, G.M., Reinsel, G.C., \& Ljung, G.M. (2015). \textit{Time Series Analysis: Forecasting and Control}, 5th ed., Wiley.
                \item Lütkepohl, H. (2005). \textit{New Introduction to Multiple Time Series Analysis}, Springer.
            \end{itemize}
            }
        \end{block}

        \begin{exampleblock}{Resurse online și cod}
            {\small
            \begin{itemize}\setlength{\itemsep}{0pt}
                \item \textbf{Quantlet}: \url{https://quantlet.com} -- Platformă de cod pentru metode cantitative
                \item \textbf{Quantinar}: \url{https://quantinar.com} -- Platformă de învățare pentru metode cantitative
                \item \textbf{GitHub TSA}: \url{https://github.com/QuantLet/TSA/tree/main/TSA_ch2} -- Cod Python pentru acest capitol
            \end{itemize}
            }
        \end{exampleblock}
    \end{cminipage}
\end{frame}

\begin{frame}{}
    \begin{cminipage}{0.95\textwidth}
        \centering
        \vspace{1cm}

        \Huge\textcolor{MainBlue}{Vă Mulțumim!}

        \vspace{0.8cm}

        \Large Întrebări?

        \vspace{1cm}

        \normalsize
        Materialele cursului sunt disponibile la: \url{https://danpele.github.io/Time-Series-Analysis/}

        \vspace{0.3cm}

        \href{https://quantlet.com}{\raisebox{-0.15em}{\includegraphics[height=0.8em]{ql_logo.png}} Quantlet} \hspace{0.5cm}
        \href{https://quantinar.com}{\raisebox{-0.15em}{\includegraphics[height=0.8em]{qr_logo.png}} Quantinar}
    \end{cminipage}
\end{frame}

\end{document}


