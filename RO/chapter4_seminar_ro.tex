% Capitolul 4: Seminar - Modele SARIMA
% Teste, Probleme Practice și Discuții
% Program de licență, Academia de Studii Economice din București

\documentclass[9pt, aspectratio=169, t]{beamer}

% Ensure content fits on slides
\setbeamersize{text margin left=8mm, text margin right=8mm}

%=============================================================================
% THEME AND STYLE CONFIGURATION
%=============================================================================
\usetheme{Madrid}
\usecolortheme{seahorse}

% IDA-Inspired Color Palette
\definecolor{MainBlue}{RGB}{26, 58, 110}
\definecolor{AccentBlue}{RGB}{42, 82, 140}
\definecolor{IDAred}{RGB}{220, 53, 69}
\definecolor{DarkGray}{RGB}{51, 51, 51}
\definecolor{MediumGray}{RGB}{128, 128, 128}
\definecolor{LightGray}{RGB}{248, 248, 248}
\definecolor{VeryLightGray}{RGB}{235, 235, 235}
\definecolor{Crimson}{RGB}{220, 53, 69}
\definecolor{Forest}{RGB}{46, 125, 50}
\definecolor{Amber}{RGB}{181, 133, 63}
\definecolor{Orange}{RGB}{230, 126, 34}

\setbeamercolor{palette primary}{bg=MainBlue, fg=white}
\setbeamercolor{palette secondary}{bg=MainBlue!85, fg=white}
\setbeamercolor{palette tertiary}{bg=MainBlue!70, fg=white}
\setbeamercolor{structure}{fg=MainBlue}
\setbeamercolor{title}{fg=MainBlue}
\setbeamercolor{frametitle}{fg=MainBlue, bg=white}
\setbeamercolor{block title}{bg=MainBlue, fg=white}
\setbeamercolor{block body}{bg=VeryLightGray, fg=DarkGray}
\setbeamercolor{block title alerted}{bg=Crimson, fg=white}
\setbeamercolor{block body alerted}{bg=Crimson!8, fg=DarkGray}
\setbeamercolor{block title example}{bg=Forest, fg=white}
\setbeamercolor{block body example}{bg=Forest!8, fg=DarkGray}
\setbeamercolor{item}{fg=MainBlue}

\setbeamertemplate{navigation symbols}{}

\setbeamertemplate{footline}{
    \leavevmode%
    \hbox{%
        \begin{beamercolorbox}[wd=.333333\paperwidth,ht=2.5ex,dp=1ex,center]{author in head/foot}%
            \usebeamerfont{author in head/foot}\insertshortauthor
        \end{beamercolorbox}%
        \begin{beamercolorbox}[wd=.333333\paperwidth,ht=2.5ex,dp=1ex,center]{title in head/foot}%
            \usebeamerfont{title in head/foot}\insertshorttitle
        \end{beamercolorbox}%
        \begin{beamercolorbox}[wd=.333333\paperwidth,ht=2.5ex,dp=1ex,right]{date in head/foot}%
            \usebeamerfont{date in head/foot}\insertshortdate{}\hspace*{2em}
            \insertframenumber{} / \inserttotalframenumber\hspace*{2ex}
        \end{beamercolorbox}}%
    \vskip0pt%
}

%=============================================================================
% PACKAGES
%=============================================================================
\usepackage[utf8]{inputenc}
\usepackage[T1]{fontenc}
\usepackage{amsmath, amssymb, amsthm}
\usepackage{mathtools}
\usepackage{bm}
\usepackage{tikz}
\usetikzlibrary{arrows.meta, positioning, shapes, calc}
\usepackage{booktabs}
\usepackage{multirow}
\usepackage{array}
\usepackage{graphicx}
\usepackage{hyperref}
\hypersetup{colorlinks=false, pdfborder={0 0 0}}
\graphicspath{{../logos/}{../charts/}}

%=============================================================================
% CUSTOM COMMANDS
%=============================================================================
\newcommand{\E}{\mathbb{E}}
\newcommand{\Var}{\text{Var}}
\newcommand{\Cov}{\text{Cov}}
\newcommand{\Corr}{\text{Corr}}
\newcommand{\R}{\mathbb{R}}

%=============================================================================
% TITLE INFORMATION
%=============================================================================
\title[Capitolul 4: Seminar]{Capitolul 4: Seminar --- Modele SARIMA}
\subtitle{Program de licență, Facultatea de Cibernetică, Statistică și Informatică Economică, Academia de Studii Economice din București}
\author[Prof. dr. Daniel Traian Pele]{Prof. dr. Daniel Traian Pele\\[0.2cm]\footnotesize\texttt{danpele@ase.ro}}
\institute{Academia de Studii Economice din București}
\date{An Universitar 2025--2026}

\begin{document}

%=============================================================================
% TITLE SLIDE
%=============================================================================
\begin{frame}[plain]
    \begin{tikzpicture}[remember picture, overlay]
        \fill[IDAred] (current page.north west) rectangle ([yshift=-0.15cm]current page.north east);
        \node[anchor=north west] at ([xshift=0.5cm, yshift=-0.3cm]current page.north west) {
            \href{https://www.ase.ro}{\includegraphics[height=1.1cm]{ase_logo.png}}
        };
        \node[anchor=north] at ([yshift=-0.3cm]current page.north) {
            \href{https://ai4efin.ase.ro}{\includegraphics[height=1.1cm]{ai4efin_logo.png}}
        };
        \node[anchor=north east] at ([xshift=-0.5cm, yshift=-0.3cm]current page.north east) {
            \href{https://www.digital-finance-msca.com}{\includegraphics[height=1.1cm]{msca_logo.png}}
        };
    \end{tikzpicture}
    \vfill
    \begin{center}
        {\Large\textcolor{MediumGray}{Analiza și Prognoza Seriilor de Timp}}\\[0.3cm]
        {\Huge\textbf{\textcolor{MainBlue}{Capitolul 4: Modele SARIMA}}}\\[0.5cm]
        {\Large\textcolor{IDAred}{Seminar}}
    \end{center}
    \vfill

    \begin{tikzpicture}[remember picture, overlay]
        \fill[IDAred] (current page.south west) rectangle ([yshift=0.15cm]current page.south east);
        \node[anchor=south west] at ([xshift=0.5cm, yshift=0.8cm]current page.south west) {
            \href{https://theida.net}{\includegraphics[height=0.9cm]{ida_logo.png}}
        };
        \node[anchor=south] at ([xshift=-3cm, yshift=0.8cm]current page.south) {
            \href{https://blockchain-research-center.com}{\includegraphics[height=0.9cm]{brc_logo.png}}
        };
        \node[anchor=south] at ([yshift=0.8cm]current page.south) {
            \href{https://quantinar.com}{\includegraphics[height=0.9cm]{qr_logo.png}}
        };
        \node[anchor=south] at ([xshift=3cm, yshift=0.8cm]current page.south) {
            \href{https://quantlet.com}{\includegraphics[height=0.9cm]{ql_logo.png}}
        };
        \node[anchor=south east] at ([xshift=-0.5cm, yshift=0.8cm]current page.south east) {
            \href{https://ipe.ro/new}{\includegraphics[height=0.9cm]{acad_logo.png}}
        };
    \end{tikzpicture}
\end{frame}

%=============================================================================
% OUTLINE
%=============================================================================
\begin{frame}{Cuprins Seminar}
    \tableofcontents
\end{frame}

%=============================================================================
% SECTION 1: REVIEW QUIZ
%=============================================================================
\section{Test de Recapitulare}

\begin{frame}{Test 1: Diferențierea Sezonieră}
    \begin{alertblock}{Întrebare}
        Pentru date lunare cu sezonalitate anuală, ce face operatorul $(1-L^{12})$?
    \end{alertblock}

    \vspace{0.3cm}

    \begin{enumerate}[A)]
        \item Ia 12 diferențe consecutive
        \item Calculează $Y_t - Y_{t-12}$
        \item Face media pe 12 luni
        \item Elimină primele 12 observații
    \end{enumerate}

    \vspace{0.5cm}
    \pause
    \begin{exampleblock}{Răspuns: B -- Calculează $Y_t - Y_{t-12}$}
        \textbf{Operatorul de diferență sezonieră}:
        \[
        (1-L^{12})Y_t = Y_t - L^{12}Y_t = Y_t - Y_{t-12}
        \]

        \textbf{Exemplu} (vânzări ianuarie): $Y_{Ian2025} - Y_{Ian2024}$

        \textbf{Efect}: Elimină modelul sezonier anual stabil

        \textbf{Notă}: $(1-L^s)$ pentru orice perioadă sezonieră $s$ (trimestrial: $s=4$, săptămânal: $s=52$)
    \end{exampleblock}
\end{frame}

\begin{frame}{Vizual: Diferența Sezonieră}
    \begin{center}
        \includegraphics[width=0.95\textwidth]{ch4_def_seasonal_diff.pdf}
    \end{center}
    \vspace{-0.2cm}
    \small Diferențierea sezonieră elimină modelele anuale comparând aceleași perioade între ani.
\end{frame}

\begin{frame}{Test 2: Notația SARIMA}
    \begin{alertblock}{Întrebare}
        Ce reprezintă SARIMA$(1,1,1) \times (1,1,1)_{12}$?
    \end{alertblock}

    \vspace{0.3cm}

    \begin{enumerate}[A)]
        \item 12 modele ARIMA diferite
        \item ARIMA cu 12 termeni AR și 12 termeni MA
        \item ARIMA(1,1,1) cu ARIMA(1,1,1) sezonier la perioada 12
        \item Un model care necesită 12 ani de date
    \end{enumerate}
\end{frame}

\begin{frame}{Test 2: Răspuns}
    \begin{exampleblock}{Răspuns: C -- ARIMA(1,1,1) cu ARIMA(1,1,1) sezonier la perioada 12}
        \begin{center}
            \includegraphics[width=0.95\textwidth, height=0.55\textheight, keepaspectratio]{sem4_sarima_notation.pdf}
        \end{center}
        $(1-\phi_1 L)(1-\Phi_1 L^{12})(1-L)(1-L^{12})Y_t = (1+\theta_1 L)(1+\Theta_1 L^{12})\varepsilon_t$
    \end{exampleblock}
\end{frame}

\begin{frame}{Vizual: Structura Modelului SARIMA}
    \begin{center}
        \includegraphics[width=0.95\textwidth]{ch4_def_sarima.pdf}
    \end{center}
    \vspace{-0.2cm}
    \small SARIMA combină componentele ARIMA obișnuite cu componentele sezoniere la lag-ul $s$.
\end{frame}

\begin{frame}{Test 3: Modelul Airline}
    \begin{alertblock}{Întrebare}
        ``Modelul airline'' se referă la SARIMA$(0,1,1) \times (0,1,1)_{12}$. Câți parametri are (excluzând varianța)?
    \end{alertblock}

    \vspace{0.3cm}

    \begin{enumerate}[A)]
        \item 2 parametri
        \item 4 parametri
        \item 6 parametri
        \item 12 parametri
    \end{enumerate}
\end{frame}

\begin{frame}{Test 3: Răspuns}
    \begin{exampleblock}{Răspuns: A -- 2 parametri ($\theta_1$ și $\Theta_1$)}
        \begin{center}
            \includegraphics[width=0.95\textwidth, height=0.55\textheight, keepaspectratio]{sem4_airline_model.pdf}
        \end{center}
        \textbf{Modelul airline}: $(1-L)(1-L^{12})Y_t = (1+\theta_1 L)(1+\Theta_1 L^{12})\varepsilon_t$

        Se potrivește remarcabil de bine pe multe serii economice sezoniere (Box \& Jenkins, 1970)
    \end{exampleblock}
\end{frame}

\begin{frame}{Test 4: ACF-ul Datelor Sezoniere}
    \begin{alertblock}{Întrebare}
        Pentru date lunare cu sezonalitate puternică, unde vă așteptați să vedeți vârfuri ACF semnificative?
    \end{alertblock}

    \vspace{0.3cm}

    \begin{enumerate}[A)]
        \item Doar la lag 1
        \item Doar la lag 12
        \item La lag-urile 12, 24, 36, ...
        \item Distribuite aleatoriu
    \end{enumerate}

    \vspace{0.5cm}
    \pause
    \begin{exampleblock}{Răspuns: C -- La lag-urile 12, 24, 36, ...}
        {\small
        \textbf{Intuiție}: Ianuarie 2024 este similar cu ianuarie 2023, 2022, etc.

        \textbf{Model ACF}: Vârfuri la lag-urile $s, 2s, 3s, \ldots$ ($\rho_{12}, \rho_{24}, \rho_{36} \neq 0$)

        \textbf{Diagnostic}: Descreștere lentă la lag-urile sezoniere $\Rightarrow$ $D=1$; Întrerupere după lag $s$ $\Rightarrow$ $Q=1$
        }
    \end{exampleblock}
\end{frame}

\begin{frame}{Vizual: Modele de Sezonalitate}
    \begin{center}
        \includegraphics[width=0.95\textwidth]{ch4_def_seasonality.pdf}
    \end{center}
    \vspace{-0.2cm}
    \small Modelele sezoniere se repetă la intervale regulate (lunar, trimestrial, etc.) și pot fi aditive sau multiplicative.
\end{frame}

\begin{frame}{Test 5: Structura Multiplicativă}
    \begin{alertblock}{Întrebare}
        În SARIMA, ce înseamnă ``structură multiplicativă''?
    \end{alertblock}

    \vspace{0.3cm}

    \begin{enumerate}[A)]
        \item Amplitudinea sezonieră crește proporțional
        \item Polinoamele obișnuite și sezoniere sunt înmulțite
        \item Înmulțim datele cu factori sezonieri
        \item Modelul este estimat folosind înmulțirea
    \end{enumerate}

    \vspace{0.5cm}
    \pause
    \begin{exampleblock}{Răspuns: B -- Polinoamele obișnuite și sezoniere sunt înmulțite}
        {\small
        \textbf{SARIMA multiplicativ}: $\phi(L)\Phi(L^s)(1-L)^d(1-L^s)^D Y_t = \theta(L)\Theta(L^s)\varepsilon_t$

        \textbf{Exemplu}: $(1-\phi_1 L)(1-\Phi_1 L^{12}) = 1 - \phi_1 L - \Phi_1 L^{12} + \phi_1\Phi_1 L^{13}$

        \textbf{Termenul încrucișat} $\phi_1\Phi_1 L^{13}$: Captează interacțiunea între dinamica pe termen scurt și lung
        }
    \end{exampleblock}
\end{frame}

\begin{frame}{Test 6: Diferențierea Sezonieră vs Obișnuită}
    \begin{alertblock}{Întrebare}
        Când ați aplica atât diferențierea obișnuită ($d=1$) cât și sezonieră ($D=1$)?
    \end{alertblock}

    \vspace{0.3cm}

    \begin{enumerate}[A)]
        \item Când datele au doar un trend
        \item Când datele au doar sezonalitate
        \item Când datele au atât trend cât și nestaționaritate sezonieră
        \item Niciodată -- se anulează reciproc
    \end{enumerate}

    \vspace{0.5cm}
    \pause
    \begin{exampleblock}{Răspuns: C -- Atât trend cât și nestaționaritate sezonieră}
        {\small
        \textbf{Combinat}: $W_t = (1-L)(1-L^{12})Y_t = Y_t - Y_{t-1} - Y_{t-12} + Y_{t-13}$

        \textbf{Când este necesar}: ACF cu descreștere lentă la lag-urile 1,2,3... $\Rightarrow d=1$; la lag-urile 12,24,36... $\Rightarrow D=1$

        \textbf{Exemple}: Pasageri aerieni, vânzări retail, cerere de energie
        }
    \end{exampleblock}
\end{frame}

\begin{frame}{Test 7: Detectarea Sezonalității din ACF}
    \begin{alertblock}{Întrebare}
        ACF-ul unei serii de timp lunare arată descreștere lentă la lag-urile 12, 24 și 36. Ce sugerează aceasta?
    \end{alertblock}

    \vspace{0.3cm}

    \begin{enumerate}[A)]
        \item Seria este staționară
        \item Seria necesită doar diferențierea obișnuită
        \item Seria are o rădăcină unitară sezonieră necesitând $D=1$
        \item Seria este zgomot alb
    \end{enumerate}
\end{frame}

\begin{frame}{Test 7: Răspuns}
    \begin{exampleblock}{Răspuns: C -- Rădăcină unitară sezonieră necesitând $D=1$}
        \begin{center}
            \includegraphics[width=0.95\textwidth, height=0.55\textheight, keepaspectratio]{sem4_seasonal_acf.pdf}
        \end{center}
        \textbf{Stânga}: Sezonieră staționară (descreștere rapidă la lag-urile sezoniere)

        \textbf{Dreapta}: Rădăcină unitară sezonieră (descreștere lentă $\Rightarrow$ necesită $D=1$)
    \end{exampleblock}
\end{frame}

\begin{frame}{Test 8: Sezonalitate Multiplicativă vs Aditivă}
    \begin{alertblock}{Întrebare}
        Dacă amplitudinea sezonieră a unei serii de timp crește proporțional cu nivelul, aceasta indică:
    \end{alertblock}

    \vspace{0.3cm}

    \begin{enumerate}[A)]
        \item Sezonalitate aditivă -- folosiți $(1-L^s)$
        \item Sezonalitate multiplicativă -- folosiți transformarea $\log$
        \item Fără sezonalitate prezentă
        \item Nevoie doar de diferențiere obișnuită
    \end{enumerate}
\end{frame}

\begin{frame}{Test 8: Răspuns}
    \begin{exampleblock}{Răspuns: B -- Sezonalitate multiplicativă, folosiți transformarea $\log$}
        \begin{center}
            \includegraphics[width=0.95\textwidth, height=0.55\textheight, keepaspectratio]{sem4_mult_add.pdf}
        \end{center}
        \textbf{Multiplicativă}: Amplitudinea sezonieră crește cu nivelul (linii divergente)

        \textbf{Soluție}: Aplicați transformarea $\log$ înainte de a ajusta SARIMA
    \end{exampleblock}
\end{frame}

\begin{frame}{Test 9: Graficul Subseriilor Sezoniere}
    \begin{alertblock}{Întrebare}
        Într-un grafic de subserii sezoniere, ce indică sezonalitatea multiplicativă?
    \end{alertblock}

    \vspace{0.3cm}

    \begin{enumerate}[A)]
        \item Liniile pentru fiecare lună sunt paralele
        \item Liniile pentru fiecare lună divergează (dispersia crește în timp)
        \item Toate lunile au aceeași medie
        \item Liniile sunt orizontale
    \end{enumerate}

    \vspace{0.5cm}
    \pause
    \begin{exampleblock}{Răspuns: B -- Liniile divergează (dispersia crește în timp)}
        {\small
        \textbf{Graficul subseriilor}: Grupează datele pe luni, reprezintă valorile fiecărei luni de-a lungul anilor

        \textbf{Paralele} $\Rightarrow$ Aditivă; \textbf{Divergente} $\Rightarrow$ Multiplicativă; \textbf{Orizontale} $\Rightarrow$ Fără trend

        \textbf{Acțiune}: Dacă multiplicativă, aplicați $\log$ înainte de a ajusta SARIMA
        }
    \end{exampleblock}
\end{frame}

\begin{frame}{Test 10: Invertibilitatea în SARIMA}
    \begin{alertblock}{Întrebare}
        Pentru ca SARIMA$(0,1,1) \times (0,1,1)_{12}$ să fie invertibil, care condiție trebuie îndeplinită?
    \end{alertblock}

    \vspace{0.3cm}

    \begin{enumerate}[A)]
        \item $|\theta_1| < 1$ doar
        \item $|\Theta_1| < 1$ doar
        \item Atât $|\theta_1| < 1$ cât și $|\Theta_1| < 1$
        \item Nu există condiție de invertibilitate pentru modelele MA
    \end{enumerate}

    \vspace{0.5cm}
    \pause
    \begin{exampleblock}{Răspuns: C -- Atât $|\theta_1| < 1$ cât și $|\Theta_1| < 1$}
        {\small
        \textbf{Invertibilitate}: Toate rădăcinile MA în afara cercului unitate

        \textbf{MA multiplicativ}: $(1+\theta_1 L)(1+\Theta_1 L^{12})$

        \textbf{Rădăcini}: Obișnuită $|z| = |{-1}/{\theta_1}| > 1 \Leftrightarrow |\theta_1| < 1$; Sezonieră $|\Theta_1| < 1$

        \textbf{Ambele} condiții necesare pentru invertibilitate generală!
        }
    \end{exampleblock}
\end{frame}

\begin{frame}{Test 11: Testul HEGY}
    \begin{alertblock}{Întrebare}
        Testul HEGY este folosit pentru:
    \end{alertblock}

    \vspace{0.3cm}

    \begin{enumerate}[A)]
        \item Estimarea parametrilor SARIMA
        \item Testarea rădăcinilor unitare la diferite frecvențe (trend și sezoniere)
        \item Verificarea normalității reziduurilor
        \item Compararea modelelor SARIMA folosind criterii informaționale
    \end{enumerate}

    \vspace{0.5cm}
    \pause
    \begin{exampleblock}{Răspuns: B -- Testarea rădăcinilor unitare la diferite frecvențe}
        {\small
        \textbf{Testul HEGY} (Hylleberg-Engle-Granger-Yoo, 1990):

        Testează la: Frecvența zero ($\omega = 0$) $\Rightarrow d = 1$; Nyquist ($\omega = \pi$); Sezonieră $\Rightarrow D = 1$

        \textbf{Decizie}: Respingeți toate $\Rightarrow$ variabile dummy sezoniere; Nu respingeți sezoniera $\Rightarrow$ diferențiere sezonieră
        }
    \end{exampleblock}
\end{frame}

\begin{frame}{Test 12: Identificarea MA Sezonier}
    \begin{alertblock}{Întrebare}
        După aplicarea $(1-L)(1-L^{12})$, ACF arată un singur vârf semnificativ doar la lag 12 (fără vârf la lag 1). PACF descrește la lag-urile sezoniere. Aceasta sugerează:
    \end{alertblock}

    \vspace{0.3cm}

    \begin{enumerate}[A)]
        \item SARIMA$(0,1,0) \times (0,1,1)_{12}$
        \item SARIMA$(0,1,1) \times (0,1,0)_{12}$
        \item SARIMA$(1,1,0) \times (1,1,0)_{12}$
        \item SARIMA$(0,1,1) \times (0,1,1)_{12}$
    \end{enumerate}

    \vspace{0.5cm}
    \pause
    \begin{exampleblock}{Răspuns: A -- SARIMA$(0,1,0) \times (0,1,1)_{12}$}
        {\small
        \textbf{Model}: Lag-uri obișnuite -- fără vârfuri în ACF/PACF; Lag-uri sezoniere -- ACF se anulează la $s$, PACF descrește

        \textbf{Interpretare}: Fără MA obișnuit ($q = 0$); MA(1) sezonier indicat ($Q = 1$)

        \textbf{Model}: $(1-L)(1-L^{12})Y_t = (1 + \Theta_1 L^{12})\varepsilon_t$
        }
    \end{exampleblock}
\end{frame}

\begin{frame}{Test 13: Supradiferențierea}
    \begin{alertblock}{Întrebare}
        După diferențiere, ACF arată un vârf negativ mare la lag 1 sau lag $s$. Aceasta indică de obicei:
    \end{alertblock}

    \vspace{0.3cm}

    \begin{enumerate}[A)]
        \item Modelul necesită mai mulți termeni AR
        \item Seria a fost supradiferențiată
        \item Seria este perfect staționară
        \item Prezența heteroscedasticității
    \end{enumerate}

    \vspace{0.5cm}
    \pause
    \begin{exampleblock}{Răspuns: B -- Seria a fost supradiferențiată}
        {\small
        \textbf{Semnătură}: ACF la lag 1 $\approx -0.5$ $\Rightarrow$ supradif la $d$; ACF la lag $s$ $\approx -0.5$ $\Rightarrow$ supradif la $D$

        \textbf{De ce?} $\Delta^2 Y_t = \varepsilon_t - \varepsilon_{t-1}$ este MA(1) cu $\theta = -1$, dând $\rho_1 = -0.5$

        \textbf{Corecție}: Reduceți $d$ sau $D$ cu unu și re-examinați ACF/PACF
        }
    \end{exampleblock}
\end{frame}

\begin{frame}{Test 14: Orizontul de Prognoză}
    \begin{alertblock}{Întrebare}
        Pentru un model SARIMA cu $D=1$, ce se întâmplă cu intervalele de încredere ale prognozei când orizontul $h \to \infty$?
    \end{alertblock}

    \vspace{0.3cm}

    \begin{enumerate}[A)]
        \item Converg la o lățime fixă
        \item Cresc fără limită
        \item Se micșorează la zero
        \item Oscilează sezonier
    \end{enumerate}

    \vspace{0.5cm}
    \pause
    \begin{exampleblock}{Răspuns: B -- Cresc fără limită}
        {\small
        \textbf{Proprietatea rădăcinii unitare}: Orice rădăcină unitară cauzează varianță de prognoză nemărginită

        \textbf{Pentru SARIMA cu $D=1$}: $\Var(\hat{Y}_{T+h} - Y_{T+h}) \to \infty$ când $h \to \infty$

        \textbf{Intuiție}: Șocurile sezoniere se acumulează; prognozele pe termen lung au IC-uri largi
        }
    \end{exampleblock}
\end{frame}

\begin{frame}{Test 15: Selectarea Perioadei Sezoniere}
    \begin{alertblock}{Întrebare}
        Aveți date zilnice care arată modele săptămânale clare. Ce perioadă sezonieră $s$ ar trebui să folosiți într-un model SARIMA?
    \end{alertblock}

    \vspace{0.3cm}

    \begin{enumerate}[A)]
        \item $s = 12$ (lunar)
        \item $s = 7$ (săptămânal)
        \item $s = 365$ (anual)
        \item $s = 24$ (orar)
    \end{enumerate}

    \vspace{0.5cm}
    \pause
    \begin{exampleblock}{Răspuns: B -- $s = 7$ (săptămânal)}
        {\small
        \begin{center}
        \begin{tabular}{lll}
            \textbf{Date} & \textbf{Model} & \textbf{Perioada $s$} \\
            Zilnice & Săptămânal & 7 \\
            Lunare & Anual & 12 \\
            Trimestriale & Anual & 4
        \end{tabular}
        \end{center}
        \textbf{Regulă}: $s$ = observații per ciclu al modelului dominant
        }
    \end{exampleblock}
\end{frame}

\begin{frame}{Test 16: Componenta AR Sezonieră}
    \begin{alertblock}{Întrebare}
        În componenta sezonieră $\Phi(L^s) = 1 - \Phi_1 L^s$, ce ne spune coeficientul $\Phi_1 = 0.8$?
    \end{alertblock}

    \vspace{0.3cm}

    \begin{enumerate}[A)]
        \item 80\% din valoarea perioadei curente vine din perioada anterioară
        \item Există 80\% corelație între observațiile consecutive
        \item 80\% din valoarea perioadei curente este explicată de aceeași perioadă din anul trecut
        \item Modelul sezonier explică 80\% din varianță
    \end{enumerate}

    \vspace{0.5cm}
    \pause
    \begin{exampleblock}{Răspuns: C -- 80\% explicat de aceeași perioadă din anul trecut}
        {\small
        \textbf{SAR(1)}: $Y_t = \Phi_1 Y_{t-12} + \varepsilon_t$

        \textbf{Cu $\Phi_1 = 0.8$}: $Y_{Ian2024} = 0.8 \cdot Y_{Ian2023} + \varepsilon_t$

        \textbf{Interpretare}: Persistență sezonieră puternică -- 80\% explicat de aceeași lună din anul trecut

        \textbf{Staționaritate}: Necesită $|\Phi_1| < 1$ (satisfăcută aici)
        }
    \end{exampleblock}
\end{frame}

\begin{frame}{Test 17: Staționaritatea Sezonieră}
    \begin{alertblock}{Întrebare}
        Un proces sezonier cu $\Phi_1 = 1$ în SARIMA$(0,0,0) \times (1,0,0)_{12}$ este:
    \end{alertblock}

    \vspace{0.3cm}

    \begin{enumerate}[A)]
        \item Staționar
        \item Are o rădăcină unitară sezonieră (integrat sezonier)
        \item Exploziv
        \item Nedefinit
    \end{enumerate}

    \vspace{0.5cm}
    \pause
    \begin{exampleblock}{Răspuns: B -- Are o rădăcină unitară sezonieră}
        {\small
        \textbf{Model}: $Y_t = Y_{t-12} + \varepsilon_t$ (mers aleatoriu sezonier)

        \textbf{Proprietăți}: Varianța crește cu timpul; fiecare lună urmează propriul său mers aleatoriu; necesită $D = 1$

        \textbf{Analogie}: Ca mersul aleatoriu obișnuit dar la frecvența sezonieră
        }
    \end{exampleblock}
\end{frame}

\begin{frame}{Test 18: Compararea Modelelor}
    \begin{alertblock}{Întrebare}
        Modelul A: SARIMA$(1,1,1) \times (1,1,1)_{12}$ are AIC = 520. Modelul B: SARIMA$(0,1,1) \times (0,1,1)_{12}$ are AIC = 525. Care afirmație este cea mai corectă?
    \end{alertblock}

    \vspace{0.3cm}

    \begin{enumerate}[A)]
        \item Modelul A este întotdeauna mai bun deoarece are AIC mai mic
        \item Modelul B ar trebui preferat datorită parsimoniei în ciuda AIC mai mare
        \item Diferența AIC de 5 sugerează că Modelul A este substanțial mai bun
        \item Nu putem compara modele cu ordine diferite
    \end{enumerate}

    \vspace{0.5cm}
    \pause
    \begin{exampleblock}{Răspuns: C -- Diferența AIC de 5 sugerează că Modelul A este substanțial mai bun}
        {\small
        \textbf{Regulă empirică}: $\Delta$AIC $< 2$: echivalente; $2$--$10$: anumite dovezi; $> 10$: dovezi puternice

        \textbf{Aici}: $\Delta$AIC $= 5$ sugerează Modelul A semnificativ mai bun

        \textbf{Întotdeauna}: Verificați și diagnosticele reziduurilor și performanța prognozei!
        }
    \end{exampleblock}
\end{frame}

\begin{frame}{Test 19: Modele Sezoniere în Reziduuri}
    \begin{alertblock}{Întrebare}
        După ajustarea unui model SARIMA, observați vârfuri ACF semnificative la lag-urile 12 și 24 în reziduuri. Ce indică aceasta?
    \end{alertblock}

    \vspace{0.3cm}

    \begin{enumerate}[A)]
        \item Modelul este corect specificat
        \item Componenta sezonieră este inadecvată
        \item Datele nu sunt sezoniere
        \item A apărut supraajustarea
    \end{enumerate}

    \vspace{0.5cm}
    \pause
    \begin{exampleblock}{Răspuns: B -- Componenta sezonieră este inadecvată}
        {\small
        \textbf{Diagnostice}: Reziduurile bune ar trebui să fie zgomot alb (fără ACF semnificativ)

        \textbf{ACF sezonier în reziduuri}: Modelul nu a capturat structura sezonieră; încercați să creșteți $P$ sau $Q$; verificați că $D$ este corect

        \textbf{Acțiune}: Încercați SARIMA cu ordin sezonier mai mare, verificați Ljung-Box la lag-urile sezoniere
        }
    \end{exampleblock}
\end{frame}

\begin{frame}{Test 20: Prognoză Practică}
    \begin{alertblock}{Întrebare}
        Prognozați vânzări retail lunare cu SARIMA$(0,1,1) \times (0,1,1)_{12}$. Pentru prognoza la 13 luni înainte, care observații istorice sunt cele mai influente?
    \end{alertblock}

    \vspace{0.3cm}

    \begin{enumerate}[A)]
        \item Doar cea mai recentă observație
        \item Observația de aceeași lună din anul trecut
        \item Toate observațiile în mod egal
        \item Doar observațiile din aceeași lună din toți anii anteriori
    \end{enumerate}

    \vspace{0.5cm}
    \pause
    \begin{exampleblock}{Răspuns: B -- Observația de aceeași lună din anul trecut}
        {\small
        \textbf{Pentru 13 luni înainte}: Cea mai influentă este $Y_{T-11}$ (aceeași lună anul trecut), de asemenea $Y_T$ și $Y_{T-12}$

        \textbf{Intuiție}: ``Ianuarie viitor arată ca ianuarie trecut, ajustat pentru trendul recent''
        }
    \end{exampleblock}
\end{frame}

%=============================================================================
% TRUE/FALSE QUESTIONS
%=============================================================================
\begin{frame}{Întrebări Adevărat/Fals (1-6)}
    \begin{alertblock}{Întrebare}
        Determinați dacă fiecare afirmație este Adevărată sau Falsă:
    \end{alertblock}

    \vspace{0.3cm}
    {\small
    \begin{enumerate}
        \item Perioada sezonieră $s$ pentru date trimestriale cu modele anuale este $s=4$.
        \item Modelele SARIMA pot gestiona doar o singură frecvență sezonieră.
        \item Dacă AIC selectează SARIMA$(1,1,1) \times (1,1,1)_{12}$ și BIC selectează modelul airline, BIC greșește întotdeauna.
        \item Testul Kruskal-Wallis poate detecta sezonalitatea fără a presupune normalitate.
        \item După ajustarea unui model SARIMA, reziduurile nu ar trebui să arate ACF semnificativ la lag-urile sezoniere.
        \item Transformarea logaritmică convertește sezonalitatea multiplicativă în aditivă.
    \end{enumerate}
    }

    \vspace{0.3cm}
    \begin{center}
        \textit{Răspunsul pe slide-ul următor...}
    \end{center}
\end{frame}

\begin{frame}{Soluții Adevărat/Fals (1-6)}
    \begin{exampleblock}{Răspunsuri}
    \begin{enumerate}
        \item \textcolor{Forest}{\textbf{ADEVĂRAT}}: Datele trimestriale cu ciclu anual au $s=4$ trimestre pe an.
        \item \textcolor{Forest}{\textbf{ADEVĂRAT}}: SARIMA standard gestionează un $s$; sezonalități multiple necesită TBATS sau termeni Fourier.
        \item \textcolor{Crimson}{\textbf{FALS}}: BIC penalizează complexitatea mai mult; modelul mai simplu poate fi mai bun pentru interpretare/prognoză.
        \item \textcolor{Forest}{\textbf{ADEVĂRAT}}: Kruskal-Wallis este neparametric, comparând distribuțiile între sezoane.
        \item \textcolor{Forest}{\textbf{ADEVĂRAT}}: ACF-ul reziduurilor ar trebui să fie în limitele de încredere la TOATE lag-urile inclusiv cele sezoniere.
        \item \textcolor{Forest}{\textbf{ADEVĂRAT}}: $\log(T \times S \times \varepsilon) = \log T + \log S + \log \varepsilon$ (formă aditivă).
    \end{enumerate}
    \end{exampleblock}
\end{frame}

%=============================================================================
% SECTION 2: PRACTICE PROBLEMS
%=============================================================================
\section{Probleme Practice}

\begin{frame}{Problema 1: Expandarea Diferenței Sezoniere}
    \begin{block}{Exercițiu}
        Expandați complet $(1-L)(1-L^{12})Y_t$. Ce observații sunt implicate?
    \end{block}

    \vspace{0.3cm}
    \pause
    \begin{exampleblock}{Soluție}
        $(1-L)(1-L^{12}) = 1 - L - L^{12} + L^{13}$

        \vspace{0.2cm}
        Prin urmare:
        $(1-L)(1-L^{12})Y_t = Y_t - Y_{t-1} - Y_{t-12} + Y_{t-13}$

        \vspace{0.2cm}
        \textbf{Interpretare}: Aceasta este diferența diferențelor:
        \begin{itemize}
            \item Mai întâi diferența sezonieră: $Y_t - Y_{t-12}$ (anul acesta vs anul trecut)
            \item Apoi diferența obișnuită: $(Y_t - Y_{t-12}) - (Y_{t-1} - Y_{t-13})$
        \end{itemize}
    \end{exampleblock}
\end{frame}

\begin{frame}{Problema 2: Expandarea Modelului Airline}
    \begin{block}{Exercițiu}
        Scrieți ecuația completă pentru modelul airline SARIMA$(0,1,1) \times (0,1,1)_{12}$:
        $(1-L)(1-L^{12})Y_t = (1+\theta_1 L)(1+\Theta_1 L^{12})\varepsilon_t$
    \end{block}

    \vspace{0.2cm}
    \pause
    \begin{exampleblock}{Soluție}
        Expandați partea MA:
        $(1+\theta_1 L)(1+\Theta_1 L^{12}) = 1 + \theta_1 L + \Theta_1 L^{12} + \theta_1 \Theta_1 L^{13}$

        \vspace{0.2cm}
        Modelul complet:
        $Y_t - Y_{t-1} - Y_{t-12} + Y_{t-13} = \varepsilon_t + \theta_1 \varepsilon_{t-1} + \Theta_1 \varepsilon_{t-12} + \theta_1 \Theta_1 \varepsilon_{t-13}$

        \vspace{0.2cm}
        \textbf{Notă}: Termenul încrucișat $\theta_1 \Theta_1 L^{13}$ este interacțiunea multiplicativă între componentele MA obișnuite și sezoniere.
    \end{exampleblock}
\end{frame}

\begin{frame}{Problema 3: Numărarea Parametrilor}
    \begin{block}{Exercițiu}
        Câți parametri (excluzând $\sigma^2$) sunt în SARIMA$(2,1,1) \times (1,0,1)_4$?
    \end{block}

    \vspace{0.3cm}
    \pause
    \begin{exampleblock}{Soluție}
        \begin{itemize}
            \item AR obișnuit($p=2$): $\phi_1, \phi_2$ $\Rightarrow$ 2 parametri
            \item MA obișnuit($q=1$): $\theta_1$ $\Rightarrow$ 1 parametru
            \item AR sezonier($P=1$): $\Phi_1$ $\Rightarrow$ 1 parametru
            \item MA sezonier($Q=1$): $\Theta_1$ $\Rightarrow$ 1 parametru
        \end{itemize}

        \vspace{0.2cm}
        \textbf{Total: 5 parametri}

        \vspace{0.2cm}
        Notă: Ordinele de diferențiere ($d=1$, $D=0$) nu adaugă parametri -- sunt transformări aplicate datelor.
    \end{exampleblock}
\end{frame}

\begin{frame}{Problema 4: Prognoza SARIMA}
    \begin{block}{Exercițiu}
        Dat modelul airline cu $\theta_1 = -0.4$ și $\Theta_1 = -0.6$, și:
        \begin{itemize}
            \item $Y_T = 500$, $Y_{T-1} = 495$, $Y_{T-11} = 480$, $Y_{T-12} = 470$
            \item $\varepsilon_T = 5$, $\varepsilon_{T-11} = -3$, $\varepsilon_{T-12} = 2$
        \end{itemize}
        Prognozați $Y_{T+1}$.
    \end{block}

    \pause
    \begin{exampleblock}{Soluție}
        Din model: $Y_{T+1} = Y_T + Y_{T-11} - Y_{T-12} + \varepsilon_{T+1} + \theta_1 \varepsilon_T + \Theta_1 \varepsilon_{T-11} + \theta_1 \Theta_1 \varepsilon_{T-12}$

        Setând $\E[\varepsilon_{T+1}] = 0$:

        $\hat{Y}_{T+1} = 500 + 480 - 470 + 0 + (-0.4)(5) + (-0.6)(-3) + (-0.4)(-0.6)(2)$

        $= 510 - 2 + 1.8 + 0.48 = \mathbf{510.28}$
    \end{exampleblock}
\end{frame}

\begin{frame}{Problema 5: Identificarea Perioadei Sezoniere}
    \begin{block}{Exercițiu}
        Potriviți fiecare tip de date cu perioada sezonieră tipică $s$:
        \begin{enumerate}
            \item Date trimestriale de PIB
            \item Vânzări retail lunare
            \item Rezervări săptămânale la restaurante
            \item Cerere zilnică de electricitate
        \end{enumerate}
    \end{block}

    \vspace{0.2cm}
    \pause
    \begin{exampleblock}{Soluție}
        \begin{enumerate}
            \item PIB trimestrial: $s = 4$ (ciclu anual pe 4 trimestre)
            \item Vânzări retail lunare: $s = 12$ (ciclu anual pe 12 luni)
            \item Rezervări săptămânale la restaurante: $s = 7$ (ciclu săptămânal) sau $s = 52$ (anual)
            \item Cerere zilnică de electricitate: $s = 7$ (model săptămânal) sau $s = 365$ (anual)
        \end{enumerate}

        \textbf{Notă}: Unele serii au modele sezoniere multiple (de ex., datele zilnice pot avea cicluri săptămânale ȘI anuale).
    \end{exampleblock}
\end{frame}

%=============================================================================
% SECTION 3: WORKED EXAMPLES
%=============================================================================
\section{Exemple Rezolvate}

\begin{frame}{Exemplu: Analiza Vânzărilor Retail Lunare}
    \begin{block}{Scenariu}
        Aveți 5 ani de date de vânzări retail lunare cu vârfuri clare în decembrie și scăderi în ianuarie. Construiți un model SARIMA potrivit.
    \end{block}

    \vspace{0.2cm}

    \begin{exampleblock}{Abordare Pas cu Pas}
        \begin{enumerate}
            \item \textbf{Inspecție vizuală}: Graficul arată trend ascendent + vârfuri puternice în decembrie
            \item \textbf{Perioada sezonieră}: Date lunare cu model anual $\Rightarrow s = 12$
            \item \textbf{Transformare}: Considerați $\log(Y_t)$ dacă amplitudinea sezonieră crește cu nivelul
            \item \textbf{Diferențiere}: Încercați $(1-L)(1-L^{12})Y_t$ -- verificați ACF/PACF
            \item \textbf{Selectarea modelului}: Începeți cu modelul airline, comparați prin AIC
        \end{enumerate}
    \end{exampleblock}
\end{frame}

\begin{frame}{Exemplu: Interpretarea ACF/PACF pentru Date Sezoniere}
    \begin{block}{Modele Observate (după diferențiere)}
        \begin{itemize}
            \item ACF: Semnificativ la lag-urile 1, 12; se anulează după lag 1 și lag 12
            \item PACF: Semnificativ la lag-urile 1, 12, 13; descrește la multiplii de 12
        \end{itemize}
    \end{block}

    \vspace{0.2cm}

    \begin{exampleblock}{Interpretare}
        \textbf{Componenta obișnuită}: ACF se anulează la 1 $\Rightarrow$ MA(1)

        \textbf{Componenta sezonieră}: ACF semnificativ doar la lag 12 $\Rightarrow$ MA(1) sezonier

        \textbf{Model sugerat}: SARIMA$(0,d,1) \times (0,D,1)_{12}$ -- modelul airline!

        \vspace{0.2cm}
        \textbf{Verificare alternativă}: Dacă PACF ar fi arătat anulare la lag-urile sezoniere în loc de ACF, considerați termeni AR sezonieri.
    \end{exampleblock}
\end{frame}

\begin{frame}[fragile]{Exemplu: Implementare Python}
    \begin{block}{Ajustarea SARIMA în Python}
        \small
        \begin{verbatim}
from statsmodels.tsa.statespace.sarimax import SARIMAX
import pmdarima as pm

# Ajustare manuală
model = SARIMAX(y, order=(0,1,1), seasonal_order=(0,1,1,12))
results = model.fit()
print(results.summary())

# Selecție automată
auto_model = pm.auto_arima(y, seasonal=True, m=12,
                           start_p=0, max_p=2,
                           start_q=0, max_q=2,
                           d=1, D=1,
                           trace=True)
        \end{verbatim}
    \end{block}
\end{frame}

\begin{frame}[fragile]{Exemplu: Interpretarea Rezultatelor SARIMA}
    \begin{block}{Rezultate Exemplu statsmodels}
        \footnotesize
        \begin{verbatim}
                         SARIMAX Results
==============================================================
Model:            SARIMAX(0,1,1)x(0,1,1,12)   AIC:    1348.52
                                               BIC:    1358.21
==============================================================
                 coef    std err      z     P>|z|
--------------------------------------------------------------
ma.L1         -0.4018      0.072   -5.58    0.000
ma.S.L12      -0.5521      0.081   -6.82    0.000
sigma2      1254.3201    142.856    8.78    0.000
        \end{verbatim}
    \end{block}

    \begin{exampleblock}{Interpretare}
        \begin{itemize}
            \item $\hat{\theta}_1 = -0.40$: MA negativ înseamnă că șocurile pozitive reduc valoarea perioadei următoare
            \item $\hat{\Theta}_1 = -0.55$: Corelația pentru aceeași sezon este captată
            \item Ambii coeficienți semnificativi $(p < 0.001)$; $|\theta|, |\Theta| < 1$ -- invertibil
        \end{itemize}
    \end{exampleblock}
\end{frame}

%=============================================================================
% SECTION 4: REAL DATA ANALYSIS
%=============================================================================
\section{Analiză pe Date Reale}

\begin{frame}{Studiu de Caz: Pasageri Aerieni (1949--1960)}
    \vspace{-0.3cm}
    \begin{center}
        \includegraphics[width=0.85\textwidth, height=0.55\textheight, keepaspectratio]{ch4_airline_data.pdf}
    \end{center}
    \vspace{-0.2cm}
    {\footnotesize
    \begin{itemize}
        \item Set de date clasic Box-Jenkins: 144 observații lunare
        \item \textbf{Trend ascendent} clar și \textbf{model sezonier} (vârfuri vara)
        \item Amplitudinea sezonieră \textbf{crește cu nivelul} $\Rightarrow$ sezonalitate multiplicativă
        \item Sugerează: transformare logaritmică + modelare SARIMA
    \end{itemize}
    }
\end{frame}

\begin{frame}{Analiza ACF/PACF După Diferențiere}
    \vspace{-0.3cm}
    \begin{center}
        \includegraphics[width=0.85\textwidth, height=0.55\textheight, keepaspectratio]{ch4_acf_pacf.pdf}
    \end{center}
    \vspace{-0.2cm}
    {\footnotesize
    \begin{itemize}
        \item După $(1-L)(1-L^{12})\log(Y_t)$: seria pare staționară
        \item Vârf semnificativ la lag 1 în ACF $\Rightarrow$ componentă MA(1)
        \item Vârf semnificativ la lag 12 în ACF $\Rightarrow$ componentă MA(1) sezonieră
        \item Modelul sugerează: \textbf{SARIMA$(0,1,1)(0,1,1)_{12}$} (modelul airline)
    \end{itemize}
    }
\end{frame}

\begin{frame}{Rezultate Estimare SARIMA: Date Pasageri Aerieni}
    {\small
    \begin{block}{Model: SARIMA$(0,1,1)(0,1,1)_{12}$ pe $\log(\text{Pasageri})$}
        \begin{center}
        \begin{tabular}{lcccc}
            \toprule
            \textbf{Parametru} & \textbf{Estimat} & \textbf{Eroare Std.} & \textbf{z-stat} & \textbf{valoare-p} \\
            \midrule
            $\theta_1$ (MA.L1) & $-0.4018$ & $0.0896$ & $-4.48$ & $<0.001$ \\
            $\Theta_1$ (MA.S.L12) & $-0.5569$ & $0.0731$ & $-7.62$ & $<0.001$ \\
            $\sigma^2$ & $0.00135$ & -- & -- & -- \\
            \bottomrule
        \end{tabular}
        \end{center}
    \end{block}

    \vspace{0.2cm}

    \begin{exampleblock}{Statistici de Ajustare a Modelului}
        \begin{itemize}
            \item Log-Verosimilitate: $244.70$
            \item AIC: $-483.40$, BIC: $-474.53$
            \item Ambii coeficienți MA semnificativi și în limitele de invertibilitate
        \end{itemize}
    \end{exampleblock}
    }
\end{frame}

\begin{frame}{Prognoză: 24 Luni Înainte}
    \vspace{-0.3cm}
    \begin{center}
        \includegraphics[width=0.85\textwidth, height=0.55\textheight, keepaspectratio]{ch4_sarima_forecast.pdf}
    \end{center}
    \vspace{-0.2cm}
    {\footnotesize
    \begin{itemize}
        \item Prognozele captează atât trendul cât și modelul sezonier
        \item Intervalele de încredere de 95\% se lărgesc pe orizontul de prognoză
        \item Vârfurile sezoniere (iulie-august) și scăderile (februarie) clar vizibile
        \item Modelul extrapolează cu succes modelul sezonier multiplicativ
    \end{itemize}
    }
\end{frame}

\begin{frame}{Diagnostice Model}
    \vspace{-0.3cm}
    \begin{center}
        \includegraphics[width=0.85\textwidth, height=0.55\textheight, keepaspectratio]{ch4_diagnostics.pdf}
    \end{center}
    \vspace{-0.2cm}
    {\footnotesize
    \begin{itemize}
        \item Reziduurile par aleatoare fără modele sistematice
        \item Distribuție aproximativ normală (graficul Q-Q aproape de diagonală)
        \item ACF-ul reziduurilor în limitele de încredere -- fără autocorelație semnificativă
        \item Testul Ljung-Box: $p > 0.05$ la toate lag-urile testate $\Rightarrow$ model adecvat
    \end{itemize}
    }
\end{frame}

%=============================================================================
% SECTION 5: DISCUSSION TOPICS
%=============================================================================
\section{Subiecte de Discuție}

\begin{frame}{Discuție: Sezonalitate Deterministă vs Stochastică}
    \begin{block}{Întrebare Cheie}
        Când ar trebui să folosiți variabile dummy sezoniere vs SARIMA pentru date sezoniere?
    \end{block}

    \vspace{0.2cm}

    \begin{block}{Considerații}
        \textbf{Variabile dummy sezoniere} (deterministe):
        \begin{itemize}
            \item Model fix, care se repetă în fiecare an
            \item Același efect decembrie în fiecare an
            \item Potrivite când sezonalitatea este stabilă
        \end{itemize}

        \vspace{0.2cm}
        \textbf{SARIMA} (stochastic):
        \begin{itemize}
            \item Model sezonier în evoluție
            \item Decembrie anul acesta depinde de decembrie anul trecut
            \item Mai bun când amplitudinea sezonieră variază
        \end{itemize}
    \end{block}
\end{frame}

\begin{frame}{Discuție: Transformarea Logaritmică}
    \begin{block}{Întrebare Cheie}
        Când ar trebui să luați logaritmi înainte de a ajusta SARIMA?
    \end{block}

    \vspace{0.2cm}

    \begin{block}{Îndrumări}
        \textbf{Folosiți transformarea log când}:
        \begin{itemize}
            \item Fluctuațiile sezoniere cresc cu nivelul (sezonalitate multiplicativă)
            \item Varianța crește în timp
            \item Datele sunt strict pozitive (prețuri, vânzări, numărători)
        \end{itemize}

        \vspace{0.2cm}
        \textbf{Evitați log când}:
        \begin{itemize}
            \item Modelul sezonier este aditiv (amplitudine constantă)
            \item Datele conțin zerouri sau negative
            \item Deja pe o scală de rate/proporții
        \end{itemize}

        \vspace{0.2cm}
        \textbf{Sfat}: Comparați AIC-ul modelelor cu și fără transformare log.
    \end{block}
\end{frame}

\begin{frame}{Discuție: Sezonalități Multiple}
    \begin{block}{Provocare}
        Datele zilnice de vânzări pot avea atât modele săptămânale (7 zile) cât și anuale (365 zile). Cum gestionați aceasta?
    \end{block}

    \vspace{0.2cm}

    \begin{block}{Abordări}
        \begin{enumerate}
            \item \textbf{SARIMA imbricat}: Modelați la frecvența mai scurtă, includeți mai lungă ca exogenă
            \item \textbf{Modele TBATS/BATS}: Gestionează explicit sezonalități multiple
            \item \textbf{Termeni Fourier}: Adăugați termeni sin/cos pentru fiecare frecvență sezonieră
            \item \textbf{Prophet/similare}: Instrumente moderne proiectate pentru sezonalități multiple
        \end{enumerate}

        \vspace{0.2cm}
        \textbf{Notă}: SARIMA standard gestionează doar o perioadă sezonieră. Pentru sezonalitate complexă, considerați metode specializate.
    \end{block}
\end{frame}

\begin{frame}{Discuție: Prognozarea Datelor Sezoniere}
    \begin{block}{Întrebare Cheie}
        Care sunt provocările unice ale prognozării seriilor de timp sezoniere?
    \end{block}

    \vspace{0.2cm}

    \begin{block}{Provocări și Soluții}
        \begin{itemize}
            \item \textbf{Orizontul contează}: Prognoza pe 12 luni înseamnă prezicerea unui ciclu complet
            \item \textbf{Incertitudinea crește}: Prognozele sezoniere compun incertitudinea obișnuită
            \item \textbf{Puncte de cotitură}: Captarea când sezoanele ating vârf/minim
            \item \textbf{Rupturi structurale}: COVID-19 a perturbat multe modele sezoniere
        \end{itemize}

        \vspace{0.2cm}
        \textbf{Bune practici}:
        \begin{itemize}
            \item Folosiți validare încrucișată cu origine mobilă
            \item Comparați cu benchmark-ul naiv sezonier
            \item Raportați intervale de prognoză, mai ales la orizonturi sezoniere
        \end{itemize}
    \end{block}
\end{frame}

%=============================================================================
% SECTION 5: EXERCISES
%=============================================================================
\section{Exerciții pentru Studiu Individual}

\begin{frame}{Exerciții pentru Acasă}
    {\small
    \begin{enumerate}
        \item \textbf{Teoretic}: Arătați că $(1-L)(1-L^4)$ poate fi scris ca $(1 - L - L^4 + L^5)$ și explicați ce face această transformare datelor trimestriale cu sezonalitate anuală.

        \vspace{0.2cm}
        \item \textbf{Calcul}: Pentru SARIMA$(1,0,0) \times (1,0,0)_4$ cu $\phi_1 = 0.5$ și $\Phi_1 = 0.8$, scrieți polinomul AR complet și identificați toți coeficienții nenuli.

        \vspace{0.2cm}
        \item \textbf{Aplicat}: Descărcați datele lunare despre pasagerii aerieni și:
            \begin{itemize}
                \item Reprezentați grafic seria și identificați trend/sezonalitate
                \item Aplicați transformările potrivite
                \item Ajustați modelul airline și interpretați coeficienții
                \item Generați prognoze pe 24 de luni cu intervale de încredere
            \end{itemize}

        \vspace{0.2cm}
        \item \textbf{Comparație}: Ajustați atât SARIMA$(0,1,1) \times (0,1,1)_{12}$ cât și SARIMA$(1,1,0) \times (1,1,0)_{12}$ pe datele despre pasagerii aerieni. Comparați folosind AIC, BIC și diagnosticele reziduurilor. Care este preferat?
    \end{enumerate}
    }
\end{frame}

\begin{frame}{Indicii pentru Soluții}
    {\small
    \begin{block}{Indicii}
        \begin{enumerate}
            \item Expandați $(1-L)(1-L^4) = 1 \cdot 1 - 1 \cdot L^4 - L \cdot 1 + L \cdot L^4 = 1 - L - L^4 + L^5$

            \vspace{0.1cm}
            \item Polinomul AR: $(1 - \phi_1 L)(1 - \Phi_1 L^4) = 1 - 0.5L - 0.8L^4 + 0.4L^5$

            \vspace{0.1cm}
            \item Pentru datele pasagerilor aerieni:
                \begin{itemize}
                    \item Folosiți transformarea log (sezonalitate multiplicativă)
                    \item Atât $d=1$ cât și $D=1$ sunt necesare
                    \item Estimări tipice: $\theta_1 \approx -0.4$, $\Theta_1 \approx -0.6$
                \end{itemize}

            \vspace{0.1cm}
            \item Modelul airline bazat pe MA se potrivește de obicei mai bine decât modelul AR sezonier pur pentru aceste date (AIC mai mic).
        \end{enumerate}
    \end{block}
    }
\end{frame}

%=============================================================================
% SUMMARY
%=============================================================================
\begin{frame}{Concluzii Cheie din Acest Seminar}
    \begin{block}{Puncte Principale}
        \begin{enumerate}
            \item Diferențierea sezonieră $(1-L^s)$ elimină sezonalitatea stochastică
            \item Notația SARIMA: $(p,d,q) \times (P,D,Q)_s$ separă obișnuitul de sezonier
            \item Modelul airline este surprinzător de eficient pentru multe seturi de date
            \item Structura multiplicativă creează termeni de interacțiune
            \item ACF/PACF arată modele atât la lag-urile obișnuite cât și la cele sezoniere
            \item Transformarea log adesea necesară pentru sezonalitatea multiplicativă
        \end{enumerate}
    \end{block}

    \vspace{0.2cm}
    \begin{alertblock}{Pașii Următori}
        Capitolul 5 va acoperi seriile de timp multivariate: modele VAR, cauzalitatea Granger și cointegrarea.
    \end{alertblock}
\end{frame}

\end{document}
