% Capitolul 1: Seminar - Analiza Seriilor de Timp
% Teste, Probleme Practice și Discuții
% Destinat: Studenți la Statistică și Știința Datelor

\documentclass[9pt, aspectratio=169, t]{beamer}

% Ensure content fits on slides
\setbeamersize{text margin left=8mm, text margin right=8mm}

%=============================================================================
% THEME AND STYLE CONFIGURATION
%=============================================================================
\usetheme{Madrid}
\usecolortheme{seahorse}

% IDA-Inspired Color Palette
\definecolor{MainBlue}{RGB}{26, 58, 110}
\definecolor{AccentBlue}{RGB}{42, 82, 140}
\definecolor{IDAred}{RGB}{220, 53, 69}
\definecolor{DarkGray}{RGB}{51, 51, 51}
\definecolor{MediumGray}{RGB}{128, 128, 128}
\definecolor{LightGray}{RGB}{248, 248, 248}
\definecolor{VeryLightGray}{RGB}{235, 235, 235}
\definecolor{Crimson}{RGB}{220, 53, 69}
\definecolor{Forest}{RGB}{46, 125, 50}
\definecolor{Amber}{RGB}{181, 133, 63}
\definecolor{Orange}{RGB}{230, 126, 34}

\setbeamercolor{palette primary}{bg=MainBlue, fg=white}
\setbeamercolor{palette secondary}{bg=MainBlue!85, fg=white}
\setbeamercolor{palette tertiary}{bg=MainBlue!70, fg=white}
\setbeamercolor{structure}{fg=MainBlue}
\setbeamercolor{title}{fg=MainBlue}
\setbeamercolor{frametitle}{fg=MainBlue, bg=white}
\setbeamercolor{block title}{bg=MainBlue, fg=white}
\setbeamercolor{block body}{bg=VeryLightGray, fg=DarkGray}
\setbeamercolor{block title alerted}{bg=Crimson, fg=white}
\setbeamercolor{block body alerted}{bg=Crimson!8, fg=DarkGray}
\setbeamercolor{block title example}{bg=Forest, fg=white}
\setbeamercolor{block body example}{bg=Forest!8, fg=DarkGray}
\setbeamercolor{item}{fg=MainBlue}

\setbeamertemplate{navigation symbols}{}

\setbeamertemplate{footline}{
    \leavevmode%
    \hbox{%
        \begin{beamercolorbox}[wd=.333333\paperwidth,ht=2.5ex,dp=1ex,center]{author in head/foot}%
            \usebeamerfont{author in head/foot}\insertshortauthor
        \end{beamercolorbox}%
        \begin{beamercolorbox}[wd=.333333\paperwidth,ht=2.5ex,dp=1ex,center]{title in head/foot}%
            \usebeamerfont{title in head/foot}\insertshorttitle
        \end{beamercolorbox}%
        \begin{beamercolorbox}[wd=.333333\paperwidth,ht=2.5ex,dp=1ex,right]{date in head/foot}%
            \usebeamerfont{date in head/foot}\insertshortdate{}\hspace*{2em}
            \insertframenumber{} / \inserttotalframenumber\hspace*{2ex}
        \end{beamercolorbox}}%
    \vskip0pt%
}

%=============================================================================
% PACKAGES
%=============================================================================
\usepackage[utf8]{inputenc}
\usepackage[T1]{fontenc}
\usepackage{amsmath, amssymb, amsthm}
\usepackage{mathtools}
\usepackage{bm}
\usepackage{tikz}
\usetikzlibrary{arrows.meta, positioning, shapes, calc}
\usepackage{booktabs}
\usepackage{multirow}
\usepackage{array}
\usepackage{graphicx}
\usepackage{hyperref}
\hypersetup{colorlinks=false, pdfborder={0 0 0}}
% enumitem conflicts with beamer, using standard enumerate
\graphicspath{{logos/}{charts/}}

%=============================================================================
% CUSTOM COMMANDS
%=============================================================================
\newcommand{\E}{\mathbb{E}}
\newcommand{\Var}{\text{Var}}
\newcommand{\Cov}{\text{Cov}}
\newcommand{\Corr}{\text{Corr}}
\newcommand{\R}{\mathbb{R}}

% Quiz styling
\newcommand{\quizmark}{\textcolor{Orange}{\textbf{[TEST]}}}
\newcommand{\correct}{\textcolor{Forest}{\checkmark}}
\newcommand{\incorrect}{\textcolor{Crimson}{\texttimes}}

%=============================================================================
% TITLE INFORMATION
%=============================================================================
\title[Capitolul 1: Seminar]{Capitolul 1: Seminar \& Practică}
\subtitle{Program de licență, Facultatea de Cibernetică, Statistică și Informatică Economică, Academia de Studii Economice din București}
\author[Prof. dr. Daniel Traian Pele]{Prof. dr. Daniel Traian Pele\\[0.2cm]\footnotesize\texttt{danpele@ase.ro}}
\institute{Academia de Studii Economice din București}
\date{An Universitar 2025--2026}

\begin{document}

%=============================================================================
% TITLE SLIDE
%=============================================================================
\begin{frame}[plain]
    \begin{tikzpicture}[remember picture, overlay]
        \fill[IDAred] (current page.north west) rectangle ([yshift=-0.15cm]current page.north east);
        \node[anchor=north west] at ([xshift=0.5cm, yshift=-0.3cm]current page.north west) {
            \href{https://www.ase.ro}{\includegraphics[height=1.1cm]{ase_logo.png}}
        };
        \node[anchor=north] at ([yshift=-0.3cm]current page.north) {
            \href{https://ai4efin.ase.ro}{\includegraphics[height=1.1cm]{ai4efin_logo.png}}
        };
        \node[anchor=north east] at ([xshift=-0.5cm, yshift=-0.3cm]current page.north east) {
            \href{https://www.digital-finance-msca.com}{\includegraphics[height=1.1cm]{msca_logo.png}}
        };
    \end{tikzpicture}
    \vfill
    \begin{center}
        {\Large\textcolor{MediumGray}{Analiza și Prognoza Seriilor de Timp}}\\[0.3cm]
        {\Huge\textbf{\textcolor{MainBlue}{Capitolul 1: Introducere în Seriile de Timp}}}\\[0.5cm]
        {\Large\textcolor{IDAred}{Seminar}}
    \end{center}
    \vfill

    \begin{tikzpicture}[remember picture, overlay]
        \fill[IDAred] (current page.south west) rectangle ([yshift=0.15cm]current page.south east);
        \node[anchor=south west] at ([xshift=0.5cm, yshift=0.8cm]current page.south west) {
            \href{https://theida.net}{\includegraphics[height=0.9cm]{ida_logo.png}}
        };
        \node[anchor=south] at ([xshift=-3cm, yshift=0.8cm]current page.south) {
            \href{https://blockchain-research-center.com}{\includegraphics[height=0.9cm]{brc_logo.png}}
        };
        \node[anchor=south] at ([yshift=0.8cm]current page.south) {
            \href{https://quantinar.com}{\includegraphics[height=0.9cm]{qr_logo.png}}
        };
        \node[anchor=south] at ([xshift=3cm, yshift=0.8cm]current page.south) {
            \href{https://quantlet.com}{\includegraphics[height=0.9cm]{ql_logo.png}}
        };
        \node[anchor=south east] at ([xshift=-0.5cm, yshift=0.8cm]current page.south east) {
            \href{https://ipe.ro/new}{\includegraphics[height=0.9cm]{acad_logo.png}}
        };
    \end{tikzpicture}
\end{frame}

%=============================================================================
% SEMINAR OUTLINE
%=============================================================================
\begin{frame}{Cuprins Seminar}
    \textbf{\large Activitățile de astăzi:}

    \vspace{0.4cm}

    \begin{enumerate}
        \item[\textcolor{MainBlue}{\textbf{1.}}] \textbf{Recapitulare Rapidă} -- Rezumatul conceptelor cheie
        \vspace{0.15cm}
        \item[\textcolor{MainBlue}{\textbf{2.}}] \textbf{Test Grilă} -- Verifică-ți înțelegerea
        \vspace{0.15cm}
        \item[\textcolor{MainBlue}{\textbf{3.}}] \textbf{Întrebări Adevărat/Fals} -- Verificări conceptuale
        \vspace{0.15cm}
        \item[\textcolor{MainBlue}{\textbf{4.}}] \textbf{Exerciții de Calcul} -- Practică aplicată
        \vspace{0.15cm}
        \item[\textcolor{MainBlue}{\textbf{5.}}] \textbf{Exerciții Python} -- Practică de programare
        \vspace{0.15cm}
        \item[\textcolor{MainBlue}{\textbf{6.}}] \textbf{Întrebări de Discuție} -- Gândire critică
    \end{enumerate}
\end{frame}

%=============================================================================
% PART 1: QUICK REVIEW
%=============================================================================
\section{Recapitulare Rapidă}

\begin{frame}{Formule Cheie de Reținut}
    \begin{columns}[T]
        \begin{column}{0.48\textwidth}
            \textbf{Descompunere:}
            \begin{itemize}
                \item Aditivă: $X_t = T_t + S_t + \varepsilon_t$
                \item Multiplicativă: $X_t = T_t \times S_t \times \varepsilon_t$
            \end{itemize}

            \vspace{0.3cm}

            \textbf{Netezire Exponențială:}
            \begin{itemize}
                \item SES: $\hat{X}_{t+1} = \alpha X_t + (1-\alpha)\hat{X}_t$
                \item Holt: adaugă trend $b_t$
                \item HW: adaugă sezonalitate $S_t$
            \end{itemize}
        \end{column}
        \begin{column}{0.48\textwidth}
            \textbf{Staționaritate:}
            \begin{itemize}
                \item $\E[X_t] = \mu$ (constantă)
                \item $\Var(X_t) = \sigma^2$ (constantă)
                \item $\Cov(X_t, X_{t+h}) = \gamma(h)$
            \end{itemize}

            \vspace{0.3cm}

            \textbf{Mers Aleatoriu:}
            \begin{itemize}
                \item $X_t = X_{t-1} + \varepsilon_t$
                \item $\Var(X_t) = t\sigma^2$ (crește!)
            \end{itemize}
        \end{column}
    \end{columns}
\end{frame}

\begin{frame}{Rezumatul Conceptelor Cheie}
    \begin{center}
    \small
    \begin{tabular}{lll}
        \toprule
        \textbf{Concept} & \textbf{Punct Cheie} & \textbf{Când se Folosește} \\
        \midrule
        Descompunere aditivă & Amplitudine sezonieră constantă & Varianță stabilă \\
        Descompunere multiplicativă & Sezonalitatea crește cu nivelul & Varianță în creștere \\
        SES & Doar nivel ($\alpha$) & Fără trend, fără sezonalitate \\
        Holt & Nivel + Trend ($\alpha, \beta$) & Trend, fără sezonalitate \\
        Holt-Winters & Nivel + Trend + Sezonalitate & Trend și sezonalitate \\
        \midrule
        Testul ADF & $H_0$: rădăcină unitară & Test pentru nestaționaritate \\
        Testul KPSS & $H_0$: staționară & Confirmă staționaritatea \\
        \midrule
        Diferențiere & Elimină trendul stocastic & Mers aleatoriu, rădăcină unitară \\
        Regresie & Elimină trendul determinist & Trend liniar/polinomial \\
        \bottomrule
    \end{tabular}
    \end{center}
\end{frame}

%=============================================================================
% PART 2: MULTIPLE CHOICE QUIZZES
%=============================================================================
\section{Test Grilă}

\begin{frame}{Test 1: Bazele Seriilor de Timp}
    \begin{alertblock}{Întrebare}
        Care dintre următoarele NU este o caracteristică a datelor de tip serie de timp?
    \end{alertblock}

    \vspace{0.4cm}

    \begin{enumerate}[A.]
        \item Observațiile sunt ordonate în timp
        \item Observațiile consecutive sunt de obicei corelate
        \item Observațiile sunt independente și identic distribuite
        \item Datele au o ordonare temporală naturală
    \end{enumerate}

    \vspace{0.5cm}

    \begin{center}
        \textit{Răspunsul pe slide-ul următor...}
    \end{center}
\end{frame}

\begin{frame}{Test 1: Răspuns}
    \begin{exampleblock}{Răspuns: C -- Observațiile sunt independente și identic distribuite}
        \textbf{Întrebare:} Care NU este o caracteristică a datelor de tip serie de timp?

        \vspace{0.3cm}

        \begin{enumerate}[A.]
            \item Observațiile sunt ordonate în timp \incorrect
            \item Observațiile consecutive sunt de obicei corelate \incorrect
            \item \textbf{\textcolor{Forest}{Observațiile sunt independente și identic distribuite}} \correct
            \item Datele au o ordonare temporală naturală \incorrect
        \end{enumerate}

        \vspace{0.3cm}

        Observațiile seriilor de timp sunt de obicei \textbf{dependente} (autocorelate), nu independente. Ipoteza observațiilor i.i.d. este fundamentală pentru analiza transversală, dar este încălcată în seriile de timp. Această dependență temporală este ceea ce face analiza seriilor de timp unică și necesită metode specializate.
    \end{exampleblock}
\end{frame}

\begin{frame}{Test 2: Descompunere}
    \begin{alertblock}{Întrebare}
        Când ar trebui să folosiți descompunerea multiplicativă în loc de cea aditivă?
    \end{alertblock}

    \vspace{0.4cm}

    \begin{enumerate}[A.]
        \item Când modelul sezonier are amplitudine constantă
        \item Când varianța seriei este stabilă în timp
        \item Când fluctuațiile sezoniere cresc proporțional cu nivelul
        \item Când seria de timp nu are componentă de trend
    \end{enumerate}

    \vspace{0.5cm}

    \begin{center}
        \textit{Răspunsul pe slide-ul următor...}
    \end{center}
\end{frame}

\begin{frame}{Test 2: Răspuns}
    \begin{exampleblock}{Răspuns: C -- Când fluctuațiile sezoniere cresc proporțional cu nivelul}
        \vspace{-0.2cm}
        \begin{center}
            \includegraphics[width=0.95\textwidth, height=0.48\textheight, keepaspectratio]{charts/sem1_decomposition.pdf}
        \end{center}
        \vspace{-0.2cm}
        {\footnotesize
        \textbf{Multiplicativă}: $X_t = T_t \times S_t \times \varepsilon_t$, amplitudinea sezonieră \textbf{scalează cu nivelul} (model în evantai)
        }
    \end{exampleblock}
\end{frame}

\begin{frame}{Test 3: Netezire Exponențială}
    \begin{alertblock}{Întrebare}
        În Netezirea Exponențială Simplă cu $\alpha = 0.9$, ce se întâmplă?
    \end{alertblock}

    \vspace{0.4cm}

    \begin{enumerate}[A.]
        \item Prognozele sunt foarte netede și stabile
        \item Observațiile recente au foarte puțină pondere
        \item Prognozele reacționează rapid la schimbările recente
        \item Prognoza este în esență o medie pe termen lung
    \end{enumerate}

    \vspace{0.5cm}

    \begin{center}
        \textit{Răspunsul pe slide-ul următor...}
    \end{center}
\end{frame}

\begin{frame}{Test 3: Răspuns}
    \begin{exampleblock}{Răspuns: C -- Prognozele reacționează rapid la schimbările recente}
        Cu $\alpha = 0.9$: $\hat{X}_{t+1} = 0.9 X_t + 0.1 \hat{X}_t$

        Aceasta înseamnă 90\% pondere pe cea mai recentă observație! Valorile mari ale lui $\alpha$ fac prognozele foarte receptive la date noi. Valorile mici ale lui $\alpha$ (de exemplu, 0.1) produc prognoze mai netede, mai stabile, care mediază peste mai mult istoric.
    \end{exampleblock}
\end{frame}

\begin{frame}{Test 4: Staționaritate}
    \begin{alertblock}{Întrebare}
        Un proces de mers aleatoriu $X_t = X_{t-1} + \varepsilon_t$ este:
    \end{alertblock}

    \vspace{0.4cm}

    \begin{enumerate}[A.]
        \item Strict staționar
        \item Slab staționar
        \item Nestaționar deoarece varianța crește cu timpul
        \item Staționar după adăugarea unei constante
    \end{enumerate}

    \vspace{0.5cm}

    \begin{center}
        \textit{Răspunsul pe slide-ul următor...}
    \end{center}
\end{frame}

\begin{frame}{Test 4: Răspuns}
    \begin{exampleblock}{Răspuns: C -- Nestaționar deoarece varianța crește cu timpul}
        Pentru mersul aleatoriu: $X_t = \sum_{i=1}^{t} \varepsilon_i$
        \begin{itemize}
            \item $\E[X_t] = 0$ (medie constantă -- OK)
            \item $\Var(X_t) = t\sigma^2$ (varianța depinde de $t$ -- NU e OK!)
        \end{itemize}
        Deoarece varianța nu este constantă, procesul încalcă condiția de staționaritate. Soluție: \textbf{diferențierea} dă $\Delta X_t = \varepsilon_t$ care ESTE staționară.
    \end{exampleblock}
\end{frame}

\begin{frame}{Vizual: Mers Aleatoriu vs Staționar}
    \begin{center}
        \includegraphics[width=0.95\textwidth]{charts/ch3_def_random_walk.pdf}
    \end{center}
    \vspace{-0.2cm}
    \small Traiectoriile mersului aleatoriu rătăcesc imprevizibil; varianța crește liniar cu timpul $\Rightarrow$ nestaționar.
\end{frame}

\begin{frame}{Test 5: Teste pentru Rădăcină Unitară}
    \begin{alertblock}{Întrebare}
        Rulați testele ADF și KPSS. ADF nu reușește să respingă $H_0$, iar KPSS respinge $H_0$. Ce concluzie trageți?
    \end{alertblock}

    \vspace{0.4cm}

    \begin{enumerate}[A.]
        \item Seria este staționară
        \item Seria are o rădăcină unitară (nestaționară)
        \item Rezultatele sunt neconcludente
        \item Trebuie să rulați mai multe teste
    \end{enumerate}

    \vspace{0.5cm}

    \begin{center}
        \textit{Răspunsul pe slide-ul următor...}
    \end{center}
\end{frame}

\begin{frame}{Test 5: Răspuns}
    \begin{exampleblock}{Răspuns: B -- Seria are o rădăcină unitară (nestaționară)}
        \begin{itemize}
            \item ADF: $H_0$ = rădăcină unitară. Nu respingem $\Rightarrow$ evidență PENTRU rădăcină unitară
            \item KPSS: $H_0$ = staționară. Respingem $\Rightarrow$ evidență ÎMPOTRIVA staționarității
        \end{itemize}
        Ambele teste sunt de acord: seria este \textbf{nestaționară}. Ar trebui să diferențiați seria înainte de a modela cu ARMA.
    \end{exampleblock}
\end{frame}

\begin{frame}{Test 6: Evaluarea Prognozei}
    \begin{alertblock}{Întrebare}
        Care metrică este cea mai potrivită pentru compararea acurateții prognozei între diferite serii de timp cu scale diferite?
    \end{alertblock}

    \vspace{0.4cm}

    \begin{enumerate}[A.]
        \item Eroarea Absolută Medie (MAE)
        \item Rădăcina Erorii Medii Pătratice (RMSE)
        \item Eroarea Absolută Medie Procentuală (MAPE)
        \item Eroarea Medie Pătratică (MSE)
    \end{enumerate}

    \vspace{0.5cm}

    \begin{center}
        \textit{Răspunsul pe slide-ul următor...}
    \end{center}
\end{frame}

\begin{frame}{Test 6: Răspuns}
    \begin{exampleblock}{Răspuns: C -- Eroarea Absolută Medie Procentuală (MAPE)}
        MAPE $= \frac{100}{n}\sum\left|\frac{e_t}{X_t}\right|$ exprimă erorile ca \textbf{procente}.

        \begin{itemize}
            \item MAE, RMSE, MSE sunt \textbf{dependente de scală} (unități ale lui $X_t$)
            \item MAPE este \textbf{independentă de scală} (întotdeauna în \%)
            \item Avertisment: MAPE eșuează când $X_t$ este aproape de zero
        \end{itemize}
    \end{exampleblock}
\end{frame}

\begin{frame}{Test 7: Tipuri de Trend}
    \begin{alertblock}{Întrebare}
        Un trend determinist poate fi eliminat prin:
    \end{alertblock}

    \vspace{0.4cm}

    \begin{enumerate}[A.]
        \item Diferențiere
        \item Regresie pe timp
        \item Ajustare sezonieră
        \item Netezire cu medie mobilă
    \end{enumerate}

    \vspace{0.5cm}

    \begin{center}
        \textit{Răspunsul pe slide-ul următor...}
    \end{center}
\end{frame}

\begin{frame}{Test 7: Răspuns}
    \begin{exampleblock}{Răspuns: B -- Regresie pe timp}
        \textbf{Trend determinist}: $Y_t = \alpha + \beta t + \varepsilon_t$ unde $\beta$ este fix.

        \textbf{Metoda de eliminare}: Regresați $Y_t$ pe $t$, apoi analizați reziduurile $\hat{\varepsilon}_t = Y_t - \hat{\alpha} - \hat{\beta}t$

        \textbf{De ce nu diferențiere?} Diferențierea unui trend determinist dă: $\Delta Y_t = \beta + \Delta\varepsilon_t$, care elimină trendul dar lasă o constantă. Pentru trenduri \textit{stocastice} (rădăcini unitare), diferențierea este corectă.
    \end{exampleblock}
\end{frame}

\begin{frame}{Test 8: Interpretarea ACF}
    \begin{alertblock}{Întrebare}
        Dacă ACF-ul unei serii de timp descrește foarte lent (rămâne semnificativ pentru multe lag-uri), aceasta sugerează:
    \end{alertblock}

    \vspace{0.4cm}

    \begin{enumerate}[A.]
        \item Seria este zgomot alb
        \item Seria este probabil nestaționară
        \item Seria nu are autocorelație
        \item Seria este perfect predictibilă
    \end{enumerate}

    \vspace{0.5cm}

    \begin{center}
        \textit{Răspunsul pe slide-ul următor...}
    \end{center}
\end{frame}

\begin{frame}{Test 8: Răspuns}
    \begin{exampleblock}{Răspuns: B -- Seria este probabil nestaționară}
        \vspace{-0.2cm}
        \begin{center}
            \includegraphics[width=0.85\textwidth, height=0.45\textheight, keepaspectratio]{charts/sem1_acf_decay.pdf}
        \end{center}
        \vspace{-0.2cm}
        {\footnotesize
        \textbf{Staționară}: ACF descrește rapid ($\rho_k = \phi^k \to 0$) \quad
        \textbf{Nestaționară}: ACF rămâne aproape de 1 $\Rightarrow$ diferențiere necesară
        }
    \end{exampleblock}
\end{frame}

\begin{frame}{Test 9: Metoda Holt}
    \begin{alertblock}{Întrebare}
        Netezirea exponențială Holt diferă de SES prin adăugarea:
    \end{alertblock}

    \vspace{0.4cm}

    \begin{enumerate}[A.]
        \item O componentă sezonieră
        \item O componentă de trend
        \item O componentă ciclică
        \item O componentă neregulată
    \end{enumerate}

    \vspace{0.5cm}

    \begin{center}
        \textit{Răspunsul pe slide-ul următor...}
    \end{center}
\end{frame}

\begin{frame}{Test 9: Răspuns}
    \begin{exampleblock}{Răspuns: B -- O componentă de trend}
        \vspace{-0.2cm}
        \begin{center}
            \includegraphics[width=0.95\textwidth, height=0.48\textheight, keepaspectratio]{charts/sem1_holt_method.pdf}
        \end{center}
        \vspace{-0.2cm}
        {\footnotesize
        \textbf{Holt}: $L_t = \alpha Y_t + (1-\alpha)(L_{t-1} + b_{t-1})$; $b_t = \beta(L_t - L_{t-1}) + (1-\beta)b_{t-1}$ \quad
        \textbf{Prognoză}: $\hat{Y}_{t+h} = L_t + h \cdot b_t$
        }
    \end{exampleblock}
\end{frame}

\begin{frame}{Test 10: Zgomot Alb}
    \begin{alertblock}{Întrebare}
        Care proprietate NU este necesară pentru ca un proces să fie zgomot alb?
    \end{alertblock}

    \vspace{0.4cm}

    \begin{enumerate}[A.]
        \item $\E[\varepsilon_t] = 0$
        \item $\Var(\varepsilon_t) = \sigma^2$ (constantă)
        \item $\Cov(\varepsilon_t, \varepsilon_s) = 0$ pentru $t \neq s$
        \item $\varepsilon_t \sim N(0, \sigma^2)$
    \end{enumerate}

    \vspace{0.5cm}

    \begin{center}
        \textit{Răspunsul pe slide-ul următor...}
    \end{center}
\end{frame}

\begin{frame}{Test 10: Răspuns}
    \begin{exampleblock}{Răspuns: D -- Normalitatea NU este necesară}
        \vspace{-0.2cm}
        \begin{center}
            \includegraphics[width=0.9\textwidth, height=0.45\textheight, keepaspectratio]{charts/sem1_white_noise.pdf}
        \end{center}
        \vspace{-0.2cm}
        {\footnotesize
        \textbf{Zgomot alb}: Medie zero, varianță constantă, necorelat. \textbf{Zgomot alb Gaussian}: Adaugă normalitate $\Rightarrow$ de asemenea independent (nu doar necorelat)
        }
    \end{exampleblock}
\end{frame}

\begin{frame}{Vizual: Proprietățile Zgomotului Alb}
    \begin{center}
        \includegraphics[width=0.95\textwidth]{charts/ch1_def_white_noise.pdf}
    \end{center}
    \vspace{-0.2cm}
    \small Stânga: zgomotul alb fluctuează în jurul lui zero. Dreapta: ACF nu arată autocorelație (toate valorile aproape de zero după lag 0).
\end{frame}

\begin{frame}{Test 11: Orizont de Prognoză}
    \begin{alertblock}{Întrebare}
        Pe măsură ce orizontul de prognoză $h$ crește, ce se întâmplă de obicei cu intervalele de prognoză?
    \end{alertblock}

    \vspace{0.4cm}

    \begin{enumerate}[A.]
        \item Devin mai înguste
        \item Rămân la aceeași lățime
        \item Devin mai largi
        \item Dispar
    \end{enumerate}

    \vspace{0.5cm}

    \begin{center}
        \textit{Răspunsul pe slide-ul următor...}
    \end{center}
\end{frame}

\begin{frame}{Test 11: Răspuns}
    \begin{exampleblock}{Răspuns: C -- Devin mai largi}
        \vspace{-0.2cm}
        \begin{center}
            \includegraphics[width=0.9\textwidth, height=0.45\textheight, keepaspectratio]{charts/sem1_forecast_intervals.pdf}
        \end{center}
        \vspace{-0.2cm}
        {\footnotesize
        \textbf{Mers aleatoriu}: $\Var = h\sigma^2$ (crește liniar) \quad
        \textbf{IC 95\%}: $\hat{Y}_{t+h} \pm 1.96\sqrt{h}\sigma$ (se lărgește cu $\sqrt{h}$)
        }
    \end{exampleblock}
\end{frame}

\begin{frame}{Test 12: Detectarea Sezonalității}
    \begin{alertblock}{Întrebare}
        ACF-ul arată vârfuri semnificative la lag-urile 12, 24 și 36 pentru date lunare. Aceasta sugerează:
    \end{alertblock}

    \vspace{0.4cm}

    \begin{enumerate}[A.]
        \item Fără sezonalitate
        \item Sezonalitate anuală
        \item Sezonalitate săptămânală
        \item Zgomot aleatoriu
    \end{enumerate}

    \vspace{0.5cm}

    \begin{center}
        \textit{Răspunsul pe slide-ul următor...}
    \end{center}
\end{frame}

\begin{frame}{Test 12: Răspuns}
    \begin{exampleblock}{Răspuns: B -- Sezonalitate anuală}
        \textbf{Recunoașterea modelului}:
        \begin{itemize}
            \item Lag 12: corelație cu aceeași lună de anul trecut
            \item Lag 24: corelație cu aceeași lună de acum doi ani
            \item Lag 36: corelație cu aceeași lună de acum trei ani
        \end{itemize}

        \textbf{Perioada sezonieră}: $s = 12$ (date lunare cu ciclu anual)

        \textbf{Modele comune}: Vânzări retail (vârfuri în decembrie), consum de energie (vară/iarnă), date turistice
    \end{exampleblock}
\end{frame}

\begin{frame}{Test 13: Validare Încrucișată în Seriile de Timp}
    \begin{alertblock}{Întrebare}
        De ce nu putem folosi validarea încrucișată standard k-fold pentru seriile de timp?
    \end{alertblock}

    \vspace{0.4cm}

    \begin{enumerate}[A.]
        \item Datele seriilor de timp sunt prea mici
        \item Ar încălca ordonarea temporală (viitorul prezicând trecutul)
        \item Validarea încrucișată este întotdeauna invalidă
        \item Seriile de timp nu au nevoie de validare
    \end{enumerate}

    \vspace{0.5cm}

    \begin{center}
        \textit{Răspunsul pe slide-ul următor...}
    \end{center}
\end{frame}

\begin{frame}{Test 13: Răspuns}
    \begin{exampleblock}{Răspuns: B -- Ar încălca ordonarea temporală}
        \vspace{-0.2cm}
        \begin{center}
            \includegraphics[width=0.95\textwidth, height=0.5\textheight, keepaspectratio]{charts/sem1_timeseries_cv.pdf}
        \end{center}
        \vspace{-0.2cm}
        {\footnotesize
        \textbf{Regulă}: Nu folosiți niciodată date viitoare pentru a prezice trecutul! Folosiți CV cu fereastră mobilă/în expansiune.
        }
    \end{exampleblock}
\end{frame}

\begin{frame}{Test 14: Limitarea MAPE}
    \begin{alertblock}{Întrebare}
        MAPE (Eroarea Absolută Medie Procentuală) NU ar trebui folosită când:
    \end{alertblock}

    \vspace{0.4cm}

    \begin{enumerate}[A.]
        \item Comparați modele pe același set de date
        \item Valorile reale pot fi zero sau aproape de zero
        \item Prognozați prețuri de acțiuni
        \item Datele au un trend
    \end{enumerate}

    \vspace{0.5cm}

    \begin{center}
        \textit{Răspunsul pe slide-ul următor...}
    \end{center}
\end{frame}

\begin{frame}{Test 14: Răspuns}
    \begin{exampleblock}{Răspuns: B -- Când valorile reale pot fi zero sau aproape de zero}
        \textbf{Formula MAPE}: $\text{MAPE} = \frac{100\%}{n}\sum_{t=1}^{n}\left|\frac{Y_t - \hat{Y}_t}{Y_t}\right|$

        \textbf{Problema}: Când $Y_t \approx 0$, împărțirea face MAPE $\to \infty$

        \textbf{Alternative}:
        \begin{itemize}
            \item \textbf{SMAPE}: $\frac{200\%}{n}\sum\frac{|Y_t - \hat{Y}_t|}{|Y_t| + |\hat{Y}_t|}$ (mărginită 0--200\%)
            \item \textbf{MASE}: $\frac{1}{n}\sum\frac{|e_t|}{\frac{1}{n-1}\sum|Y_t - Y_{t-1}|}$ (fără scală)
        \end{itemize}
    \end{exampleblock}
\end{frame}

%=============================================================================
% PART 3: TRUE/FALSE QUESTIONS
%=============================================================================
\section{Întrebări Adevărat/Fals}

\begin{frame}{Adevărat sau Fals? (Setul 1)}
    \begin{alertblock}{Întrebare}
        Marcați fiecare afirmație ca Adevărat (A) sau Fals (F):
    \end{alertblock}

    \vspace{0.3cm}

    \begin{enumerate}
        \item O serie de timp cu medie constantă este întotdeauna staționară. \hfill \underline{\hspace{1cm}}
        \vspace{0.2cm}
        \item Varianța unui mers aleatoriu crește liniar cu timpul. \hfill \underline{\hspace{1cm}}
        \vspace{0.2cm}
        \item Prognozele SES sunt întotdeauna plate (constante pentru toate orizonturile). \hfill \underline{\hspace{1cm}}
        \vspace{0.2cm}
        \item Testele ADF și KPSS au aceeași ipoteză nulă. \hfill \underline{\hspace{1cm}}
        \vspace{0.2cm}
        \item RMSE mai mic înseamnă întotdeauna prognoze mai bune. \hfill \underline{\hspace{1cm}}
        \vspace{0.2cm}
        \item Autocorelația la lag 0 este întotdeauna egală cu 1. \hfill \underline{\hspace{1cm}}
    \end{enumerate}

    \vspace{0.3cm}
    \begin{center}
        \textit{Răspunsul pe slide-ul următor...}
    \end{center}
\end{frame}

\begin{frame}{Adevărat sau Fals: Răspunsuri (Setul 1)}
    \begin{exampleblock}{Răspunsuri}
    \begin{enumerate}
        \item O serie de timp cu medie constantă este întotdeauna staționară. \hfill \textbf{\textcolor{Crimson}{FALS}}

        {\small \textcolor{MediumGray}{De asemenea, este nevoie de varianță constantă și covarianță care depinde doar de lag.}}

        \vspace{0.15cm}

        \item Varianța unui mers aleatoriu crește liniar cu timpul. \hfill \textbf{\textcolor{Forest}{ADEVĂRAT}}

        {\small \textcolor{MediumGray}{$\Var(X_t) = t\sigma^2$ pentru mersul aleatoriu pornind de la $X_0$.}}

        \vspace{0.15cm}

        \item Prognozele SES sunt întotdeauna plate (constante pentru toate orizonturile). \hfill \textbf{\textcolor{Forest}{ADEVĂRAT}}

        {\small \textcolor{MediumGray}{SES nu are componentă de trend, deci $\hat{X}_{t+h} = L_t$ pentru orice $h$.}}

        \vspace{0.15cm}

        \item Testele ADF și KPSS au aceeași ipoteză nulă. \hfill \textbf{\textcolor{Crimson}{FALS}}

        {\small \textcolor{MediumGray}{ADF: $H_0$ = rădăcină unitară. KPSS: $H_0$ = staționară. Ipoteze opuse!}}

        \vspace{0.15cm}

        \item RMSE mai mic înseamnă întotdeauna prognoze mai bune. \hfill \textbf{\textcolor{Crimson}{FALS}}

        {\small \textcolor{MediumGray}{Depinde de context. RMSE este dependent de scală; poate supraajusta la valori extreme.}}

        \vspace{0.15cm}

        \item Autocorelația la lag 0 este întotdeauna egală cu 1. \hfill \textbf{\textcolor{Forest}{ADEVĂRAT}}

        {\small \textcolor{MediumGray}{$\rho(0) = \gamma(0)/\gamma(0) = 1$ prin definiție.}}
    \end{enumerate}
    \end{exampleblock}
\end{frame}

\begin{frame}{Adevărat sau Fals? (Setul 2)}
    \begin{alertblock}{Întrebare}
        Marcați fiecare afirmație ca Adevărat (A) sau Fals (F):
    \end{alertblock}

    \vspace{0.3cm}

    \begin{enumerate}
        \item ACF-ul unui proces AR(1) staționar descrește exponențial. \hfill \underline{\hspace{1cm}}

        \vspace{0.2cm}

        \item Zgomotul alb este întotdeauna distribuit normal. \hfill \underline{\hspace{1cm}}

        \vspace{0.2cm}

        \item Diferențierea poate face o serie nestaționară să devină staționară. \hfill \underline{\hspace{1cm}}

        \vspace{0.2cm}

        \item PACF-ul unui proces MA(1) se întrerupe după lag 1. \hfill \underline{\hspace{1cm}}

        \vspace{0.2cm}

        \item Ar trebui să folosiți întotdeauna setul de test pentru ajustarea hiperparametrilor. \hfill \underline{\hspace{1cm}}

        \vspace{0.2cm}

        \item Holt-Winters este potrivit pentru date fără sezonalitate. \hfill \underline{\hspace{1cm}}
    \end{enumerate}

    \vspace{0.3cm}

    \begin{center}
        \textit{Răspunsul pe slide-ul următor...}
    \end{center}
\end{frame}

\begin{frame}{Adevărat sau Fals: Răspunsuri (Setul 2)}
    \begin{exampleblock}{Răspunsuri}
    \begin{enumerate}
        \item ACF-ul unui AR(1) staționar descrește exponențial. \hfill \textbf{\textcolor{Forest}{ADEVĂRAT}}

        {\small \textcolor{MediumGray}{Pentru AR(1): $\rho(h) = \phi^h$, care descrește exponențial.}}

        \vspace{0.15cm}

        \item Zgomotul alb este întotdeauna distribuit normal. \hfill \textbf{\textcolor{Crimson}{FALS}}

        {\small \textcolor{MediumGray}{Zgomotul alb necesită doar medie zero, varianță constantă, fără autocorelație. Zgomotul alb Gaussian este un caz special.}}

        \vspace{0.15cm}

        \item Diferențierea poate face o serie nestaționară să devină staționară. \hfill \textbf{\textcolor{Forest}{ADEVĂRAT}}

        {\small \textcolor{MediumGray}{Diferențierea elimină trendurile stocastice (rădăcinile unitare).}}

        \vspace{0.15cm}

        \item PACF-ul unui MA(1) se întrerupe după lag 1. \hfill \textbf{\textcolor{Crimson}{FALS}}

        {\small \textcolor{MediumGray}{ACF-ul se întrerupe pentru MA. PACF descrește pentru procesele MA.}}

        \vspace{0.15cm}

        \item Ar trebui să folosiți întotdeauna setul de test pentru ajustarea hiperparametrilor. \hfill \textbf{\textcolor{Crimson}{FALS}}

        {\small \textcolor{MediumGray}{Folosiți setul de validare pentru ajustare. Setul de test este doar pentru evaluarea finală!}}

        \vspace{0.15cm}

        \item Holt-Winters este potrivit pentru date fără sezonalitate. \hfill \textbf{\textcolor{Crimson}{FALS}}

        {\small \textcolor{MediumGray}{Folosiți metoda Holt (fără componentă sezonieră) sau SES pentru date nesezoniere.}}
    \end{enumerate}
    \end{exampleblock}
\end{frame}

%=============================================================================
% PART 4: CALCULATION EXERCISES
%=============================================================================
\section{Exerciții de Calcul}

\begin{frame}{Exercițiu 1: Netezire Exponențială Simplă}
    \textbf{Problemă:} Date fiind următoarele date și $\alpha = 0.3$:

    \begin{center}
    \begin{tabular}{c|ccccc}
        $t$ & 1 & 2 & 3 & 4 & 5 \\
        \hline
        $X_t$ & 10 & 12 & 11 & 14 & 13 \\
    \end{tabular}
    \end{center}

    Începând cu $\hat{X}_1 = X_1 = 10$, calculați:

    \vspace{0.3cm}

    \begin{enumerate}[a)]
        \item Prognozele $\hat{X}_2, \hat{X}_3, \hat{X}_4, \hat{X}_5$
        \item Prognoza pentru $t = 6$: $\hat{X}_6$
        \item Erorile de prognoză $e_t = X_t - \hat{X}_t$ pentru $t = 2, 3, 4, 5$
        \item MAE și RMSE
    \end{enumerate}

    \vspace{0.3cm}

    \textbf{Formula:} $\hat{X}_{t+1} = \alpha X_t + (1-\alpha)\hat{X}_t$
\end{frame}

\begin{frame}{Exercițiu 1: Soluție}
    \textbf{Folosind} $\hat{X}_{t+1} = 0.3 X_t + 0.7 \hat{X}_t$:

    \vspace{0.2cm}

    \begin{center}
    \small
    \begin{tabular}{c|ccccc|c}
        $t$ & 1 & 2 & 3 & 4 & 5 & 6\\
        \hline
        $X_t$ & 10 & 12 & 11 & 14 & 13 & ?\\
        $\hat{X}_t$ & 10 & 10 & 10.6 & 10.72 & 11.70 & \textbf{12.09}\\
        $e_t$ & -- & 2 & 0.4 & 3.28 & 1.30 & --\\
    \end{tabular}
    \end{center}

    \vspace{0.2cm}

    \textbf{Calcule:}
    \begin{itemize}
        \item $\hat{X}_2 = 0.3(10) + 0.7(10) = 10$
        \item $\hat{X}_3 = 0.3(12) + 0.7(10) = 10.6$
        \item $\hat{X}_4 = 0.3(11) + 0.7(10.6) = 10.72$
        \item $\hat{X}_5 = 0.3(14) + 0.7(10.72) = 11.70$
        \item $\hat{X}_6 = 0.3(13) + 0.7(11.70) = \textbf{12.09}$
    \end{itemize}

    \vspace{0.2cm}

    \textbf{MAE} $= \frac{|2|+|0.4|+|3.28|+|1.30|}{4} = 1.745$ \quad \textbf{RMSE} $= \sqrt{\frac{4+0.16+10.76+1.69}{4}} = 2.04$
\end{frame}

\begin{frame}{Exercițiu 2: Autocovarianță}
    \textbf{Problemă:} Pentru un proces staționar cu:
    \begin{itemize}
        \item $\E[X_t] = 5$
        \item $\gamma(0) = 4$ (varianță)
        \item $\gamma(1) = 2$
        \item $\gamma(2) = 1$
    \end{itemize}

    \vspace{0.3cm}

    Calculați:
    \begin{enumerate}[a)]
        \item Funcția de autocorelație $\rho(0), \rho(1), \rho(2)$
        \item $\Cov(X_t, X_{t-1})$
        \item $\Corr(X_5, X_7)$
        \item Dacă $X_t = 6$, care este $\E[X_{t+1} | X_t = 6]$ presupunând AR(1)?
    \end{enumerate}
\end{frame}

\begin{frame}{Exercițiu 2: Soluție}
    \textbf{a) Autocorelații:}
    \[
        \rho(h) = \frac{\gamma(h)}{\gamma(0)}
    \]
    \begin{itemize}
        \item $\rho(0) = \gamma(0)/\gamma(0) = 1$
        \item $\rho(1) = \gamma(1)/\gamma(0) = 2/4 = 0.5$
        \item $\rho(2) = \gamma(2)/\gamma(0) = 1/4 = 0.25$
    \end{itemize}

    \vspace{0.2cm}

    \textbf{b)} $\Cov(X_t, X_{t-1}) = \gamma(1) = 2$ \quad (prin staționaritate, covarianța la lag 1)

    \vspace{0.2cm}

    \textbf{c)} $\Corr(X_5, X_7) = \rho(|7-5|) = \rho(2) = 0.25$

    \vspace{0.2cm}

    \textbf{d)} Pentru AR(1) cu $\phi = \rho(1) = 0.5$:
    \[
        \E[X_{t+1} | X_t] = \mu + \phi(X_t - \mu) = 5 + 0.5(6-5) = 5.5
    \]
\end{frame}

\begin{frame}{Exercițiu 3: Proprietățile Mersului Aleatoriu}
    \textbf{Problemă:} Considerați un mers aleatoriu $X_t = X_{t-1} + \varepsilon_t$ unde $\varepsilon_t \sim WN(0, 4)$ și $X_0 = 100$.

    \vspace{0.4cm}

    Calculați:
    \begin{enumerate}[a)]
        \item $\E[X_{10}]$
        \item $\Var(X_{10})$
        \item $\Cov(X_5, X_{10})$
        \item Intervalul de încredere de 95\% pentru $X_{100}$
        \item După observarea $X_5 = 108$, care este cea mai bună prognoză pentru $X_6$?
    \end{enumerate}
\end{frame}

\begin{frame}{Exercițiu 3: Soluție}
    \textbf{Mers aleatoriu:} $X_t = X_0 + \sum_{i=1}^{t} \varepsilon_i$ cu $\sigma^2 = 4$

    \vspace{0.3cm}

    \textbf{a)} $\E[X_{10}] = X_0 = 100$ \quad (media rămâne la valoarea de pornire)

    \vspace{0.2cm}

    \textbf{b)} $\Var(X_{10}) = 10 \times \sigma^2 = 10 \times 4 = 40$

    \vspace{0.2cm}

    \textbf{c)} $\Cov(X_5, X_{10}) = \min(5, 10) \times \sigma^2 = 5 \times 4 = 20$

    \vspace{0.2cm}

    \textbf{d)} Pentru $X_{100}$:
    \begin{itemize}
        \item $\E[X_{100}] = 100$, $\Var(X_{100}) = 400$, $SD = 20$
        \item IC 95\%: $100 \pm 1.96 \times 20 = [60.8, 139.2]$
    \end{itemize}

    \vspace{0.2cm}

    \textbf{e)} Cea mai bună prognoză: $\hat{X}_6 = X_5 = 108$

    {\small (Mers aleatoriu: cea mai bună prognoză este ultima valoare observată)}
\end{frame}

%=============================================================================
% PART 5: PYTHON EXERCISES
%=============================================================================
\section{Exerciții Python}

\begin{frame}[fragile]{Exercițiu Python 1: Încărcare și Reprezentare Grafică}
    \textbf{Sarcină:} Încărcați datele S\&P 500 și creați un grafic de bază pentru seria de timp.

    \vspace{0.3cm}

    \begin{block}{Cod de Pornire}
    \small
    \begin{verbatim}
import yfinance as yf
import matplotlib.pyplot as plt

# Descărcați datele S&P 500
sp500 = yf.download('^GSPC', start='2020-01-01', end='2025-01-01')

# TODO: Reprezentați grafic prețurile de închidere
# TODO: Adăugați titlu, etichete și grilă
# TODO: Calculați și afișați statistici de bază
    \end{verbatim}
    \end{block}

    \vspace{0.2cm}

    \textbf{Întrebări:}
    \begin{enumerate}
        \item Care este media și deviația standard a randamentelor?
        \item Seria pare staționară? De ce sau de ce nu?
    \end{enumerate}
\end{frame}

\begin{frame}[fragile]{Exercițiu Python 2: Descompunere}
    \textbf{Sarcină:} Efectuați descompunerea STL pe datele pasagerilor aerieni.

    \vspace{0.3cm}

    \begin{block}{Cod de Pornire}
    \small
    \begin{verbatim}
from statsmodels.tsa.seasonal import STL
import pandas as pd

# Încărcați pasagerii aerieni
url = 'https://raw.githubusercontent.com/..../airline.csv'
airline = pd.read_csv(url, parse_dates=['Month'],
                      index_col='Month')

# TODO: Aplicați descompunerea STL cu period=12
# TODO: Reprezentați grafic toate componentele
# TODO: Ce procent din varianță este explicat de trend?
    \end{verbatim}
    \end{block}

    \textbf{Indiciu:} Folosiți \texttt{STL(data, period=12).fit()}
\end{frame}

\begin{frame}[fragile]{Exercițiu Python 3: Netezire Exponențială}
    \textbf{Sarcină:} Comparați SES, Holt și Holt-Winters pe date reale.

    \vspace{0.3cm}

    \begin{block}{Cod de Pornire}
    \small
    \begin{verbatim}
from statsmodels.tsa.holtwinters import (SimpleExpSmoothing,
    ExponentialSmoothing)

# Împărțiți datele: 80% antrenare, 20% test
train = airline[:'1958']
test = airline['1959':]

# TODO: Ajustați SES, Holt și Holt-Winters
# TODO: Generați prognoze pentru perioada de test
# TODO: Calculați RMSE pentru fiecare metodă
# TODO: Care metodă are cele mai bune performanțe? De ce?
    \end{verbatim}
    \end{block}
\end{frame}

\begin{frame}[fragile]{Exercițiu Python 4: Testarea Staționarității}
    \textbf{Sarcină:} Testați staționaritatea folosind testele ADF și KPSS.

    \vspace{0.3cm}

    \begin{block}{Cod de Pornire}
    \small
    \begin{verbatim}
from statsmodels.tsa.stattools import adfuller, kpss

# Testați prețurile S&P 500
prices = sp500['Close']
returns = prices.pct_change().dropna()

# TODO: Rulați testul ADF pe prețuri și randamente
# TODO: Rulați testul KPSS pe prețuri și randamente
# TODO: Interpretați rezultatele

# ADF: adfuller(series)
# KPSS: kpss(series, regression='c')
    \end{verbatim}
    \end{block}

    \textbf{Întrebări:}
    \begin{enumerate}
        \item Prețurile sunt staționare? Randamentele sunt staționare?
        \item ADF și KPSS sunt de acord?
    \end{enumerate}
\end{frame}

%=============================================================================
% PART 6: REAL DATA ANALYSIS
%=============================================================================
\section{Analiză pe Date Reale}

\begin{frame}{Studiu de Caz: Indicele S\&P 500}
    \vspace{-0.3cm}
    \begin{center}
        \includegraphics[width=0.85\textwidth, height=0.55\textheight, keepaspectratio]{charts/sp500_prices_returns.pdf}
    \end{center}
    \vspace{-0.2cm}
    {\footnotesize
    \begin{itemize}
        \item \textbf{Sus}: Nivelul prețului S\&P 500 -- trend ascendent clar (nestaționar)
        \item \textbf{Jos}: Randamente zilnice $r_t = \log(P_t/P_{t-1})$ -- staționare
        \item Randamentele fluctuează în jurul mediei zero fără trend
        \item Gruparea volatilității vizibilă -- perioade de volatilitate ridicată/scăzută
    \end{itemize}
    }
\end{frame}

\begin{frame}{Descompunerea Seriilor de Timp: Exemplu Real}
    \vspace{-0.3cm}
    \begin{center}
        \includegraphics[width=0.85\textwidth, height=0.55\textheight, keepaspectratio]{charts/decomposition.pdf}
    \end{center}
    \vspace{-0.2cm}
    {\footnotesize
    \begin{itemize}
        \item \textbf{Trend}: Direcția pe termen lung a seriei
        \item \textbf{Sezonalitate}: Modele periodice regulate
        \item \textbf{Reziduu}: Ce rămâne după eliminarea trendului și sezonalității
        \item Descompunerea ajută la înțelegerea structurii datelor înainte de modelare
    \end{itemize}
    }
\end{frame}

\begin{frame}{Testarea Staționarității: Rezultate ADF}
    \vspace{-0.3cm}
    \begin{center}
        \includegraphics[width=0.85\textwidth, height=0.55\textheight, keepaspectratio]{charts/adf_test_visualization.pdf}
    \end{center}
    \vspace{-0.2cm}
    {\footnotesize
    \begin{itemize}
        \item Testul ADF compară statistica de test cu valorile critice
        \item Dacă statistica de test $<$ valoarea critică $\Rightarrow$ respingem rădăcina unitară (seria este staționară)
        \item Prețuri: Statistica ADF $> -2.86$ $\Rightarrow$ nestaționară
        \item Randamente: Statistica ADF $< -2.86$ $\Rightarrow$ staționară
    \end{itemize}
    }
\end{frame}

\begin{frame}{Comparație Staționaritate: Prețuri vs Randamente}
    {\small
    \begin{block}{Rezultate Test ADF}
        \begin{center}
        \begin{tabular}{lccc}
            \toprule
            \textbf{Serie} & \textbf{Statistică ADF} & \textbf{valoare-p} & \textbf{Concluzie} \\
            \midrule
            Prețuri S\&P 500 & $-0.82$ & $0.812$ & Nestaționară \\
            Randamente S\&P 500 & $-45.3$ & $<0.001$ & Staționară \\
            \bottomrule
        \end{tabular}
        \end{center}
    \end{block}

    \vspace{0.3cm}

    \begin{exampleblock}{Observație Cheie}
        Prețurile financiare sunt de obicei $I(1)$ -- integrate de ordinul 1.

        Luând diferențe de ordinul întâi (randamente) se obține staționaritatea.

        De aceea modelăm \textbf{randamentele}, nu prețurile!
    \end{exampleblock}
    }
\end{frame}

\begin{frame}{Prognoză cu Netezire Exponențială}
    \vspace{-0.3cm}
    \begin{center}
        \includegraphics[width=0.85\textwidth, height=0.55\textheight, keepaspectratio]{charts/holt_winters.pdf}
    \end{center}
    \vspace{-0.2cm}
    {\footnotesize
    \begin{itemize}
        \item Metoda Holt-Winters pentru date cu trend și sezonalitate
        \item Parametrii de netezire $\alpha$, $\beta$, $\gamma$ controlează adaptabilitatea
        \item Prognozele captează atât continuarea trendului cât și modelul sezonier
        \item Simplă dar eficientă pentru multe aplicații de afaceri
    \end{itemize}
    }
\end{frame}

%=============================================================================
% PART 7: DISCUSSION QUESTIONS
%=============================================================================
\section{Întrebări de Discuție}

\begin{frame}{Întrebare de Discuție 1}
    \begin{block}{Scenariu}
        Analizați date lunare de vânzări pentru o companie de retail. Datele arată sezonalitate clară (vânzări ridicate în decembrie) și un trend ascendent. Vârfurile sezoniere au devenit mai mari în timp.
    \end{block}

    \vspace{0.4cm}

    \textbf{Discutați:}
    \begin{enumerate}
        \item Ar trebui să folosiți descompunere aditivă sau multiplicativă? De ce?
        \item Ce metodă de netezire exponențială ați recomanda?
        \item Cum ați evalua modelul de prognoză?
        \item Ce ar putea merge prost dacă ați folosi descompunerea greșită?
    \end{enumerate}
\end{frame}

\begin{frame}{Întrebare de Discuție 2}
    \begin{block}{Scenariu}
        Un coleg afirmă: „Am rulat testul ADF pe datele mele de prețuri de acțiuni și am obținut o valoare-p de 0.65, deci datele mele sunt staționare și pot ajusta direct un model ARMA."
    \end{block}

    \vspace{0.4cm}

    \textbf{Discutați:}
    \begin{enumerate}
        \item Ce este greșit în această interpretare?
        \item Ce testează de fapt ipotezele ADF?
        \item Ce ar trebui să facă colegul înainte de a ajusta un model ARMA?
        \item Cum ar putea ajuta testul KPSS să clarifice situația?
    \end{enumerate}
\end{frame}

\begin{frame}{Întrebare de Discuție 3}
    \begin{block}{Scenariu}
        Construiți un model de prognoză și obțineți rezultate excelente: MAPE de 2\% pe setul dumneavoastră de date. Managerul este impresionat și vrea să implementeze modelul imediat.
    \end{block}

    \vspace{0.4cm}

    \textbf{Discutați:}
    \begin{enumerate}
        \item Ce întrebări ar trebui să puneți înainte de implementare?
        \item Ați folosit împărțiri corecte antrenare/validare/test?
        \item Ar putea exista scurgere de date în evaluare?
        \item Ce diagnostice suplimentare ați rula?
        \item Cum ați monitoriza modelul în producție?
    \end{enumerate}
\end{frame}

\begin{frame}{Întrebare de Discuție 4}
    \begin{block}{Scenariu}
        Trebuie să prognozați cererea zilnică de electricitate pentru săptămâna următoare. Datele arată: (1) modele zilnice puternice (vârfuri la ora 18), (2) modele săptămânale (mai scăzut în weekend), și (3) modele anuale (mai ridicat vara/iarna).
    \end{block}

    \vspace{0.4cm}

    \textbf{Discutați:}
    \begin{enumerate}
        \item Cum ați gestiona modelele sezoniere multiple?
        \item Ar funcționa Holt-Winters aici? De ce sau de ce nu?
        \item Care este avantajul termenilor Fourier în acest caz?
        \item Cum ați configura împărțirea antrenare/validare/test?
    \end{enumerate}
\end{frame}

%=============================================================================
% SUMMARY
%=============================================================================
\section{Rezumat}

\begin{frame}{Concluzii Cheie de Astăzi}
    \begin{enumerate}
        \item \textbf{Seriile de timp sunt dependente} -- nu i.i.d. ca datele transversale

        \vspace{0.15cm}

        \item \textbf{Alegeți descompunerea cu înțelepciune} -- multiplicativă când amplitudinea sezonieră crește

        \vspace{0.15cm}

        \item \textbf{Înțelegeți parametrii de netezire} -- $\alpha$ mare = reactiv, $\alpha$ mic = neted

        \vspace{0.15cm}

        \item \textbf{Testați staționaritatea} -- folosiți atât ADF cât și KPSS împreună

        \vspace{0.15cm}

        \item \textbf{Evaluare corectă} -- nu ajustați niciodată pe setul de test!

        \vspace{0.15cm}

        \item \textbf{Mersul aleatoriu este nestaționar} -- varianța crește cu timpul
    \end{enumerate}

    \vspace{0.3cm}

    \begin{block}{Următorul Seminar}
        Identificarea, estimarea și prognoza modelelor ARMA/ARIMA
    \end{block}
\end{frame}

\begin{frame}{}
    \centering
    \Huge\textcolor{MainBlue}{Întrebări?}

    \vspace{1cm}

    \Large Succes la exerciții!

    \vspace{1cm}

    \normalsize
    Practica face perfecțiunea.
\end{frame}

\end{document}
