% Capitolul 8: Seminar - Extensii Moderne

\documentclass[9pt, aspectratio=169, t]{beamer}
%=============================================================================
% SHARED PREAMBLE - Time Series Analysis and Forecasting
% Harvard-quality academic presentations
% Bachelor program, Bucharest University of Economic Studies
%
% Usage: \documentclass[9pt, aspectratio=169, t]{beamer}
%            %=============================================================================
% SHARED PREAMBLE - Time Series Analysis and Forecasting
% Harvard-quality academic presentations
% Bachelor program, Bucharest University of Economic Studies
%
% Usage: \documentclass[9pt, aspectratio=169, t]{beamer}
%            %=============================================================================
% SHARED PREAMBLE - Time Series Analysis and Forecasting
% Harvard-quality academic presentations
% Bachelor program, Bucharest University of Economic Studies
%
% Usage: \documentclass[9pt, aspectratio=169, t]{beamer}
%            \input{preamble}
%            \subtitle{Seminar X: Seminar Title}
%            \begin{document} ...
%=============================================================================

% Ensure content fits on slides
\setbeamersize{text margin left=8mm, text margin right=8mm}

%=============================================================================
% THEME AND STYLE CONFIGURATION
%=============================================================================
\usetheme{default}
% Using default theme for clean header/footer control

% Color Palette (matching Redispatch PDF)
\definecolor{MainBlue}{RGB}{26, 58, 110}
\definecolor{AccentBlue}{RGB}{26, 58, 110}
\definecolor{IDAred}{RGB}{205, 0, 0}
\definecolor{DarkGray}{RGB}{51, 51, 51}
\definecolor{MediumGray}{RGB}{128, 128, 128}
\definecolor{LightGray}{RGB}{248, 248, 248}
\definecolor{VeryLightGray}{RGB}{235, 235, 235}
\definecolor{KeynoteGray}{RGB}{218, 218, 218}
\definecolor{SectionGray}{RGB}{120, 120, 120}
\definecolor{FooterGray}{RGB}{100, 100, 100}
\definecolor{Crimson}{RGB}{220, 53, 69}
\definecolor{Forest}{RGB}{46, 125, 50}
\definecolor{Amber}{RGB}{181, 133, 63}
\definecolor{Orange}{RGB}{230, 126, 34}
\definecolor{Purple}{RGB}{142, 68, 173}

% Gradient background (exact Keynote 315° gradient: white to RGB 218,218,218)
\setbeamertemplate{background}{%
    \begin{tikzpicture}[remember picture, overlay]
        \shade[shading=axis, shading angle=315,
        top color=white, bottom color=KeynoteGray]
        (current page.south west) rectangle (current page.north east);
    \end{tikzpicture}%
}
% Fallback solid color for compatibility
\setbeamercolor{background canvas}{bg=}

\setbeamercolor{palette primary}{bg=MainBlue, fg=white}
\setbeamercolor{palette secondary}{bg=MainBlue!85, fg=white}
\setbeamercolor{palette tertiary}{bg=MainBlue!70, fg=white}
\setbeamercolor{structure}{fg=MainBlue}
\setbeamercolor{title}{fg=IDAred}
\setbeamercolor{frametitle}{fg=IDAred, bg=}
\setbeamercolor{block title}{bg=MainBlue, fg=white}
\setbeamercolor{block body}{bg=VeryLightGray, fg=DarkGray}
\setbeamercolor{block title alerted}{bg=Crimson, fg=white}
\setbeamercolor{block body alerted}{bg=Crimson!8, fg=DarkGray}
\setbeamercolor{block title example}{bg=Forest, fg=white}
\setbeamercolor{block body example}{bg=Forest!8, fg=DarkGray}
\setbeamercolor{item}{fg=MainBlue}

% Smaller institute font to avoid overfull hbox on title page
\setbeamerfont{institute}{size=\footnotesize}

% Footer colors (override Madrid theme blue)
\setbeamercolor{author in head/foot}{fg=FooterGray, bg=}
\setbeamercolor{title in head/foot}{fg=FooterGray, bg=}
\setbeamercolor{date in head/foot}{fg=FooterGray, bg=}
\setbeamercolor{section in head/foot}{fg=FooterGray, bg=}
\setbeamercolor{subsection in head/foot}{fg=FooterGray, bg=}

% Bullet styles (apply everywhere including blocks)
\setbeamertemplate{itemize item}{\color{MainBlue}$\boxdot$}
\setbeamertemplate{itemize subitem}{\color{MainBlue}$\blacktriangleright$}
\setbeamertemplate{itemize subsubitem}{\color{MainBlue}\tiny$\bullet$}
\setbeamertemplate{itemize/enumerate body begin}{\normalsize}
\setbeamertemplate{itemize/enumerate subbody begin}{\normalsize}

% Item spacing - compact style
\setlength{\leftmargini}{10pt}       % Level 1: minimal indent
\setlength{\leftmarginii}{10pt}      % Level 2: minimal additional indent
% Compact list spacing (zero extra space before/after lists in blocks)
\makeatletter
\def\@listi{\leftmargin\leftmargini \topsep 0pt \parsep 0pt \itemsep 0pt}
\def\@listii{\leftmargin\leftmarginii \topsep 0pt \parsep 0pt \itemsep 0pt}
\makeatother

\setbeamertemplate{navigation symbols}{}

%=============================================================================
% CUSTOM HEADLINE
%=============================================================================
\setbeamertemplate{headline}{%
    \vskip10pt%
    \hbox to \paperwidth{%
        \hskip0.5cm%
        {\small\color{FooterGray}\renewcommand{\hyperlink}[2]{##2}\insertsectionhead}%
        \hfill%
        \textcolor{FooterGray}{\small\insertframenumber}%
        \hskip0.5cm%
    }%
    \vskip4pt%
    {\color{FooterGray}\hrule height 0.4pt}%
}

%=============================================================================
% CUSTOM FOOTER
%=============================================================================
\usepackage{fontawesome5}

\setbeamertemplate{footline}{%
    {\color{FooterGray}\hrule height 0.4pt}%
    \vskip4pt%
    \hbox to \paperwidth{%
        \hskip0.5cm%
        \textcolor{FooterGray}{\small Time Series Analysis and Forecasting}%
        \hfill%
        \raisebox{-0.1em}{%
            \begin{tikzpicture}[x=0.08em, y=0.08em, line width=0.4pt]
                \draw[FooterGray] (0,3) -- (1,4) -- (2,3.5) -- (3,5) -- (4,4) -- (5,6) -- (6,5.5) -- (7,4) -- (8,5) -- (9,7) -- (10,6) -- (11,5) -- (12,6.5) -- (13,8) -- (14,7) -- (15,6) -- (16,7.5) -- (17,9) -- (18,8) -- (19,7) -- (20,8.5) -- (21,10) -- (22,9) -- (23,8) -- (24,9.5);
            \end{tikzpicture}%
        }%
        \hskip0.5cm%
    }%
    \vskip6pt%
}

%=============================================================================
% PACKAGES
%=============================================================================
\usepackage[utf8]{inputenc}
\usepackage[T1]{fontenc}
\usepackage[english]{babel}
\usepackage{amsmath, amssymb, amsthm}
\usepackage{mathtools}
\usepackage{bm}
\usepackage{tikz}
\usetikzlibrary{arrows.meta, positioning, shapes, calc, decorations.pathreplacing, shadings}
\usepackage{booktabs}
\usepackage{multirow}
\usepackage{array}
\usepackage{graphicx}
\usepackage{hyperref}
\usepackage{colortbl}
\usepackage{listings}
\lstset{basicstyle=\ttfamily\small, breaklines=true, frame=single, backgroundcolor=\color{VeryLightGray}}
\hypersetup{colorlinks=true, linkcolor=MainBlue, urlcolor=MainBlue}
\graphicspath{{../../logos/}{../../charts/}{../../photos/}}
\hfuzz=2pt  % Suppress tiny overfull warnings (<2pt)
\vfuzz=2pt  % Suppress tiny vertical overfull warnings (<2pt)

%=============================================================================
% QUANTLET COMMAND
%=============================================================================
\newcommand{\quantlet}[2]{%
    \hfill\href{#2}{%
        \raisebox{-0.15em}{\includegraphics[height=0.7em]{ql_logo.png}}%
        \textcolor{MainBlue}{\tiny\ #1}%
    }%
}

%=============================================================================
% CUSTOM TITLE PAGE
%=============================================================================
\defbeamertemplate*{title page}{hybrid}[1][]
{
    \vspace{0.2cm}
    % Logos row - top header (with clickable links)
    \begin{center}
        \href{https://www.ase.ro}{\includegraphics[height=1.0cm]{ase_logo.png}}\hspace{0.25cm}%
        \href{https://theida.net}{\includegraphics[height=1.0cm]{ida_logo.png}}\hspace{0.25cm}%
        \href{https://blockchain-research-center.com}{\includegraphics[height=1.0cm]{brc_logo.png}}\hspace{0.25cm}%
        \href{https://www.ai4efin.ase.ro}{\includegraphics[height=1.0cm]{ai4efin_logo.png}}\hspace{0.25cm}%
        \href{https://ipe.ro/new}{\includegraphics[height=1.0cm]{acad_logo.png}}\hspace{0.25cm}%
        \href{https://www.digital-finance-msca.com}{\includegraphics[height=1.0cm]{msca_logo.png}}%
    \end{center}

    \vspace{0.6cm}

    % Main title with Q logos on sides (with clickable links)
    \begin{center}
        \begin{minipage}{0.1\textwidth}
            \centering
            \href{https://quantlet.com}{\includegraphics[height=1.1cm]{ql_logo.png}}
        \end{minipage}%
        \begin{minipage}{0.78\textwidth}
            \centering
            {\LARGE\bfseries\usebeamercolor[fg]{title}\inserttitle}

            \vspace{0.3cm}

            {\usebeamerfont{subtitle}\usebeamercolor[fg]{title}\insertsubtitle}
        \end{minipage}%
        \begin{minipage}{0.1\textwidth}
            \centering
            \href{https://quantinar.com}{\includegraphics[height=1.1cm]{qr_logo.png}}
        \end{minipage}
    \end{center}

    \vspace{0.6cm}

    % Authors (left aligned)
    \hspace{0.5cm}{\usebeamerfont{author}\insertauthor}

    \vspace{0.3cm}

    % Institute/Affiliations (left aligned)
    \hspace{0.5cm}\begin{minipage}[t]{0.9\textwidth}
        \raggedright\small\insertinstitute
    \end{minipage}
}

%=============================================================================
% THEOREM ENVIRONMENTS
%=============================================================================
\theoremstyle{definition}
\setbeamertemplate{theorems}[numbered]
\newtheorem{defn}{Definition}
\newtheorem{thm}{Theorem}
\newtheorem{prop}{Proposition}
\newtheorem{rmk}{Remark}

%=============================================================================
% CENTRED MINIPAGE (no extra vertical space)
%=============================================================================
\newenvironment{cminipage}[1]{%
    \par\noindent\hfill\begin{minipage}{#1}\ignorespaces
}{%
    \end{minipage}\hfill\null\par
}

%=============================================================================
% CUSTOM COMMANDS
%=============================================================================
\newcommand{\E}{\mathbb{E}}
\newcommand{\Var}{\text{Var}}
\newcommand{\Cov}{\text{Cov}}
\newcommand{\Corr}{\text{Corr}}
\newcommand{\R}{\mathbb{R}}
\newcommand{\N}{\mathbb{N}}
\newcommand{\Z}{\mathbb{Z}}
\newcommand{\B}{\mathbf{B}}
\newcommand{\imark}{\textcolor{MainBlue}{\textbullet}}
\newcommand{\RMSE}{\text{RMSE}}
\newcommand{\MAE}{\text{MAE}}
\newcommand{\MAPE}{\text{MAPE}}
\newcommand{\correct}{\textcolor{Forest}{\checkmark}}
\newcommand{\incorrect}{\textcolor{Crimson}{\texttimes}}

% Boldface vector/matrix commands
\newcommand{\bY}{\mathbf{Y}}
\newcommand{\bX}{\mathbf{X}}
\newcommand{\bA}{\mathbf{A}}
\newcommand{\bB}{\mathbf{B}}
\newcommand{\bepsilon}{\boldsymbol{\varepsilon}}
\newcommand{\bvarepsilon}{\boldsymbol{\varepsilon}}
\newcommand{\bSigma}{\boldsymbol{\Sigma}}
\newcommand{\bPhi}{\boldsymbol{\Phi}}
\newcommand{\bGamma}{\boldsymbol{\Gamma}}
\newcommand{\bPi}{\boldsymbol{\Pi}}
\newcommand{\bc}{\mathbf{c}}
\newcommand{\balpha}{\boldsymbol{\alpha}}
\newcommand{\bbeta}{\boldsymbol{\beta}}

%=============================================================================
% TITLE INFORMATION
%=============================================================================
\title[Time Series Analysis]{Time Series Analysis and Forecasting}
\author[D.T. Pele]{Daniel Traian PELE}
\institute{Bucharest University of Economic Studies\\
IDA Institute Digital Assets\\
Blockchain Research Center\\
AI4EFin Artificial Intelligence for Energy Finance\\
Romanian Academy, Institute for Economic Forecasting\\
MSCA Digital Finance}
\date{}

%            \subtitle{Seminar X: Seminar Title}
%            \begin{document} ...
%=============================================================================

% Ensure content fits on slides
\setbeamersize{text margin left=8mm, text margin right=8mm}

%=============================================================================
% THEME AND STYLE CONFIGURATION
%=============================================================================
\usetheme{default}
% Using default theme for clean header/footer control

% Color Palette (matching Redispatch PDF)
\definecolor{MainBlue}{RGB}{26, 58, 110}
\definecolor{AccentBlue}{RGB}{26, 58, 110}
\definecolor{IDAred}{RGB}{205, 0, 0}
\definecolor{DarkGray}{RGB}{51, 51, 51}
\definecolor{MediumGray}{RGB}{128, 128, 128}
\definecolor{LightGray}{RGB}{248, 248, 248}
\definecolor{VeryLightGray}{RGB}{235, 235, 235}
\definecolor{KeynoteGray}{RGB}{218, 218, 218}
\definecolor{SectionGray}{RGB}{120, 120, 120}
\definecolor{FooterGray}{RGB}{100, 100, 100}
\definecolor{Crimson}{RGB}{220, 53, 69}
\definecolor{Forest}{RGB}{46, 125, 50}
\definecolor{Amber}{RGB}{181, 133, 63}
\definecolor{Orange}{RGB}{230, 126, 34}
\definecolor{Purple}{RGB}{142, 68, 173}

% Gradient background (exact Keynote 315° gradient: white to RGB 218,218,218)
\setbeamertemplate{background}{%
    \begin{tikzpicture}[remember picture, overlay]
        \shade[shading=axis, shading angle=315,
        top color=white, bottom color=KeynoteGray]
        (current page.south west) rectangle (current page.north east);
    \end{tikzpicture}%
}
% Fallback solid color for compatibility
\setbeamercolor{background canvas}{bg=}

\setbeamercolor{palette primary}{bg=MainBlue, fg=white}
\setbeamercolor{palette secondary}{bg=MainBlue!85, fg=white}
\setbeamercolor{palette tertiary}{bg=MainBlue!70, fg=white}
\setbeamercolor{structure}{fg=MainBlue}
\setbeamercolor{title}{fg=IDAred}
\setbeamercolor{frametitle}{fg=IDAred, bg=}
\setbeamercolor{block title}{bg=MainBlue, fg=white}
\setbeamercolor{block body}{bg=VeryLightGray, fg=DarkGray}
\setbeamercolor{block title alerted}{bg=Crimson, fg=white}
\setbeamercolor{block body alerted}{bg=Crimson!8, fg=DarkGray}
\setbeamercolor{block title example}{bg=Forest, fg=white}
\setbeamercolor{block body example}{bg=Forest!8, fg=DarkGray}
\setbeamercolor{item}{fg=MainBlue}

% Smaller institute font to avoid overfull hbox on title page
\setbeamerfont{institute}{size=\footnotesize}

% Footer colors (override Madrid theme blue)
\setbeamercolor{author in head/foot}{fg=FooterGray, bg=}
\setbeamercolor{title in head/foot}{fg=FooterGray, bg=}
\setbeamercolor{date in head/foot}{fg=FooterGray, bg=}
\setbeamercolor{section in head/foot}{fg=FooterGray, bg=}
\setbeamercolor{subsection in head/foot}{fg=FooterGray, bg=}

% Bullet styles (apply everywhere including blocks)
\setbeamertemplate{itemize item}{\color{MainBlue}$\boxdot$}
\setbeamertemplate{itemize subitem}{\color{MainBlue}$\blacktriangleright$}
\setbeamertemplate{itemize subsubitem}{\color{MainBlue}\tiny$\bullet$}
\setbeamertemplate{itemize/enumerate body begin}{\normalsize}
\setbeamertemplate{itemize/enumerate subbody begin}{\normalsize}

% Item spacing - compact style
\setlength{\leftmargini}{10pt}       % Level 1: minimal indent
\setlength{\leftmarginii}{10pt}      % Level 2: minimal additional indent
% Compact list spacing (zero extra space before/after lists in blocks)
\makeatletter
\def\@listi{\leftmargin\leftmargini \topsep 0pt \parsep 0pt \itemsep 0pt}
\def\@listii{\leftmargin\leftmarginii \topsep 0pt \parsep 0pt \itemsep 0pt}
\makeatother

\setbeamertemplate{navigation symbols}{}

%=============================================================================
% CUSTOM HEADLINE
%=============================================================================
\setbeamertemplate{headline}{%
    \vskip10pt%
    \hbox to \paperwidth{%
        \hskip0.5cm%
        {\small\color{FooterGray}\renewcommand{\hyperlink}[2]{##2}\insertsectionhead}%
        \hfill%
        \textcolor{FooterGray}{\small\insertframenumber}%
        \hskip0.5cm%
    }%
    \vskip4pt%
    {\color{FooterGray}\hrule height 0.4pt}%
}

%=============================================================================
% CUSTOM FOOTER
%=============================================================================
\usepackage{fontawesome5}

\setbeamertemplate{footline}{%
    {\color{FooterGray}\hrule height 0.4pt}%
    \vskip4pt%
    \hbox to \paperwidth{%
        \hskip0.5cm%
        \textcolor{FooterGray}{\small Time Series Analysis and Forecasting}%
        \hfill%
        \raisebox{-0.1em}{%
            \begin{tikzpicture}[x=0.08em, y=0.08em, line width=0.4pt]
                \draw[FooterGray] (0,3) -- (1,4) -- (2,3.5) -- (3,5) -- (4,4) -- (5,6) -- (6,5.5) -- (7,4) -- (8,5) -- (9,7) -- (10,6) -- (11,5) -- (12,6.5) -- (13,8) -- (14,7) -- (15,6) -- (16,7.5) -- (17,9) -- (18,8) -- (19,7) -- (20,8.5) -- (21,10) -- (22,9) -- (23,8) -- (24,9.5);
            \end{tikzpicture}%
        }%
        \hskip0.5cm%
    }%
    \vskip6pt%
}

%=============================================================================
% PACKAGES
%=============================================================================
\usepackage[utf8]{inputenc}
\usepackage[T1]{fontenc}
\usepackage[english]{babel}
\usepackage{amsmath, amssymb, amsthm}
\usepackage{mathtools}
\usepackage{bm}
\usepackage{tikz}
\usetikzlibrary{arrows.meta, positioning, shapes, calc, decorations.pathreplacing, shadings}
\usepackage{booktabs}
\usepackage{multirow}
\usepackage{array}
\usepackage{graphicx}
\usepackage{hyperref}
\usepackage{colortbl}
\usepackage{listings}
\lstset{basicstyle=\ttfamily\small, breaklines=true, frame=single, backgroundcolor=\color{VeryLightGray}}
\hypersetup{colorlinks=true, linkcolor=MainBlue, urlcolor=MainBlue}
\graphicspath{{../../logos/}{../../charts/}{../../photos/}}
\hfuzz=2pt  % Suppress tiny overfull warnings (<2pt)
\vfuzz=2pt  % Suppress tiny vertical overfull warnings (<2pt)

%=============================================================================
% QUANTLET COMMAND
%=============================================================================
\newcommand{\quantlet}[2]{%
    \hfill\href{#2}{%
        \raisebox{-0.15em}{\includegraphics[height=0.7em]{ql_logo.png}}%
        \textcolor{MainBlue}{\tiny\ #1}%
    }%
}

%=============================================================================
% CUSTOM TITLE PAGE
%=============================================================================
\defbeamertemplate*{title page}{hybrid}[1][]
{
    \vspace{0.2cm}
    % Logos row - top header (with clickable links)
    \begin{center}
        \href{https://www.ase.ro}{\includegraphics[height=1.0cm]{ase_logo.png}}\hspace{0.25cm}%
        \href{https://theida.net}{\includegraphics[height=1.0cm]{ida_logo.png}}\hspace{0.25cm}%
        \href{https://blockchain-research-center.com}{\includegraphics[height=1.0cm]{brc_logo.png}}\hspace{0.25cm}%
        \href{https://www.ai4efin.ase.ro}{\includegraphics[height=1.0cm]{ai4efin_logo.png}}\hspace{0.25cm}%
        \href{https://ipe.ro/new}{\includegraphics[height=1.0cm]{acad_logo.png}}\hspace{0.25cm}%
        \href{https://www.digital-finance-msca.com}{\includegraphics[height=1.0cm]{msca_logo.png}}%
    \end{center}

    \vspace{0.6cm}

    % Main title with Q logos on sides (with clickable links)
    \begin{center}
        \begin{minipage}{0.1\textwidth}
            \centering
            \href{https://quantlet.com}{\includegraphics[height=1.1cm]{ql_logo.png}}
        \end{minipage}%
        \begin{minipage}{0.78\textwidth}
            \centering
            {\LARGE\bfseries\usebeamercolor[fg]{title}\inserttitle}

            \vspace{0.3cm}

            {\usebeamerfont{subtitle}\usebeamercolor[fg]{title}\insertsubtitle}
        \end{minipage}%
        \begin{minipage}{0.1\textwidth}
            \centering
            \href{https://quantinar.com}{\includegraphics[height=1.1cm]{qr_logo.png}}
        \end{minipage}
    \end{center}

    \vspace{0.6cm}

    % Authors (left aligned)
    \hspace{0.5cm}{\usebeamerfont{author}\insertauthor}

    \vspace{0.3cm}

    % Institute/Affiliations (left aligned)
    \hspace{0.5cm}\begin{minipage}[t]{0.9\textwidth}
        \raggedright\small\insertinstitute
    \end{minipage}
}

%=============================================================================
% THEOREM ENVIRONMENTS
%=============================================================================
\theoremstyle{definition}
\setbeamertemplate{theorems}[numbered]
\newtheorem{defn}{Definition}
\newtheorem{thm}{Theorem}
\newtheorem{prop}{Proposition}
\newtheorem{rmk}{Remark}

%=============================================================================
% CENTRED MINIPAGE (no extra vertical space)
%=============================================================================
\newenvironment{cminipage}[1]{%
    \par\noindent\hfill\begin{minipage}{#1}\ignorespaces
}{%
    \end{minipage}\hfill\null\par
}

%=============================================================================
% CUSTOM COMMANDS
%=============================================================================
\newcommand{\E}{\mathbb{E}}
\newcommand{\Var}{\text{Var}}
\newcommand{\Cov}{\text{Cov}}
\newcommand{\Corr}{\text{Corr}}
\newcommand{\R}{\mathbb{R}}
\newcommand{\N}{\mathbb{N}}
\newcommand{\Z}{\mathbb{Z}}
\newcommand{\B}{\mathbf{B}}
\newcommand{\imark}{\textcolor{MainBlue}{\textbullet}}
\newcommand{\RMSE}{\text{RMSE}}
\newcommand{\MAE}{\text{MAE}}
\newcommand{\MAPE}{\text{MAPE}}
\newcommand{\correct}{\textcolor{Forest}{\checkmark}}
\newcommand{\incorrect}{\textcolor{Crimson}{\texttimes}}

% Boldface vector/matrix commands
\newcommand{\bY}{\mathbf{Y}}
\newcommand{\bX}{\mathbf{X}}
\newcommand{\bA}{\mathbf{A}}
\newcommand{\bB}{\mathbf{B}}
\newcommand{\bepsilon}{\boldsymbol{\varepsilon}}
\newcommand{\bvarepsilon}{\boldsymbol{\varepsilon}}
\newcommand{\bSigma}{\boldsymbol{\Sigma}}
\newcommand{\bPhi}{\boldsymbol{\Phi}}
\newcommand{\bGamma}{\boldsymbol{\Gamma}}
\newcommand{\bPi}{\boldsymbol{\Pi}}
\newcommand{\bc}{\mathbf{c}}
\newcommand{\balpha}{\boldsymbol{\alpha}}
\newcommand{\bbeta}{\boldsymbol{\beta}}

%=============================================================================
% TITLE INFORMATION
%=============================================================================
\title[Time Series Analysis]{Time Series Analysis and Forecasting}
\author[D.T. Pele]{Daniel Traian PELE}
\institute{Bucharest University of Economic Studies\\
IDA Institute Digital Assets\\
Blockchain Research Center\\
AI4EFin Artificial Intelligence for Energy Finance\\
Romanian Academy, Institute for Economic Forecasting\\
MSCA Digital Finance}
\date{}

%            \subtitle{Seminar X: Seminar Title}
%            \begin{document} ...
%=============================================================================

% Ensure content fits on slides
\setbeamersize{text margin left=8mm, text margin right=8mm}

%=============================================================================
% THEME AND STYLE CONFIGURATION
%=============================================================================
\usetheme{default}
% Using default theme for clean header/footer control

% Color Palette (matching Redispatch PDF)
\definecolor{MainBlue}{RGB}{26, 58, 110}
\definecolor{AccentBlue}{RGB}{26, 58, 110}
\definecolor{IDAred}{RGB}{205, 0, 0}
\definecolor{DarkGray}{RGB}{51, 51, 51}
\definecolor{MediumGray}{RGB}{128, 128, 128}
\definecolor{LightGray}{RGB}{248, 248, 248}
\definecolor{VeryLightGray}{RGB}{235, 235, 235}
\definecolor{KeynoteGray}{RGB}{218, 218, 218}
\definecolor{SectionGray}{RGB}{120, 120, 120}
\definecolor{FooterGray}{RGB}{100, 100, 100}
\definecolor{Crimson}{RGB}{220, 53, 69}
\definecolor{Forest}{RGB}{46, 125, 50}
\definecolor{Amber}{RGB}{181, 133, 63}
\definecolor{Orange}{RGB}{230, 126, 34}
\definecolor{Purple}{RGB}{142, 68, 173}

% Gradient background (exact Keynote 315° gradient: white to RGB 218,218,218)
\setbeamertemplate{background}{%
    \begin{tikzpicture}[remember picture, overlay]
        \shade[shading=axis, shading angle=315,
        top color=white, bottom color=KeynoteGray]
        (current page.south west) rectangle (current page.north east);
    \end{tikzpicture}%
}
% Fallback solid color for compatibility
\setbeamercolor{background canvas}{bg=}

\setbeamercolor{palette primary}{bg=MainBlue, fg=white}
\setbeamercolor{palette secondary}{bg=MainBlue!85, fg=white}
\setbeamercolor{palette tertiary}{bg=MainBlue!70, fg=white}
\setbeamercolor{structure}{fg=MainBlue}
\setbeamercolor{title}{fg=IDAred}
\setbeamercolor{frametitle}{fg=IDAred, bg=}
\setbeamercolor{block title}{bg=MainBlue, fg=white}
\setbeamercolor{block body}{bg=VeryLightGray, fg=DarkGray}
\setbeamercolor{block title alerted}{bg=Crimson, fg=white}
\setbeamercolor{block body alerted}{bg=Crimson!8, fg=DarkGray}
\setbeamercolor{block title example}{bg=Forest, fg=white}
\setbeamercolor{block body example}{bg=Forest!8, fg=DarkGray}
\setbeamercolor{item}{fg=MainBlue}

% Smaller institute font to avoid overfull hbox on title page
\setbeamerfont{institute}{size=\footnotesize}

% Footer colors (override Madrid theme blue)
\setbeamercolor{author in head/foot}{fg=FooterGray, bg=}
\setbeamercolor{title in head/foot}{fg=FooterGray, bg=}
\setbeamercolor{date in head/foot}{fg=FooterGray, bg=}
\setbeamercolor{section in head/foot}{fg=FooterGray, bg=}
\setbeamercolor{subsection in head/foot}{fg=FooterGray, bg=}

% Bullet styles (apply everywhere including blocks)
\setbeamertemplate{itemize item}{\color{MainBlue}$\boxdot$}
\setbeamertemplate{itemize subitem}{\color{MainBlue}$\blacktriangleright$}
\setbeamertemplate{itemize subsubitem}{\color{MainBlue}\tiny$\bullet$}
\setbeamertemplate{itemize/enumerate body begin}{\normalsize}
\setbeamertemplate{itemize/enumerate subbody begin}{\normalsize}

% Item spacing - compact style
\setlength{\leftmargini}{10pt}       % Level 1: minimal indent
\setlength{\leftmarginii}{10pt}      % Level 2: minimal additional indent
% Compact list spacing (zero extra space before/after lists in blocks)
\makeatletter
\def\@listi{\leftmargin\leftmargini \topsep 0pt \parsep 0pt \itemsep 0pt}
\def\@listii{\leftmargin\leftmarginii \topsep 0pt \parsep 0pt \itemsep 0pt}
\makeatother

\setbeamertemplate{navigation symbols}{}

%=============================================================================
% CUSTOM HEADLINE
%=============================================================================
\setbeamertemplate{headline}{%
    \vskip10pt%
    \hbox to \paperwidth{%
        \hskip0.5cm%
        {\small\color{FooterGray}\renewcommand{\hyperlink}[2]{##2}\insertsectionhead}%
        \hfill%
        \textcolor{FooterGray}{\small\insertframenumber}%
        \hskip0.5cm%
    }%
    \vskip4pt%
    {\color{FooterGray}\hrule height 0.4pt}%
}

%=============================================================================
% CUSTOM FOOTER
%=============================================================================
\usepackage{fontawesome5}

\setbeamertemplate{footline}{%
    {\color{FooterGray}\hrule height 0.4pt}%
    \vskip4pt%
    \hbox to \paperwidth{%
        \hskip0.5cm%
        \textcolor{FooterGray}{\small Time Series Analysis and Forecasting}%
        \hfill%
        \raisebox{-0.1em}{%
            \begin{tikzpicture}[x=0.08em, y=0.08em, line width=0.4pt]
                \draw[FooterGray] (0,3) -- (1,4) -- (2,3.5) -- (3,5) -- (4,4) -- (5,6) -- (6,5.5) -- (7,4) -- (8,5) -- (9,7) -- (10,6) -- (11,5) -- (12,6.5) -- (13,8) -- (14,7) -- (15,6) -- (16,7.5) -- (17,9) -- (18,8) -- (19,7) -- (20,8.5) -- (21,10) -- (22,9) -- (23,8) -- (24,9.5);
            \end{tikzpicture}%
        }%
        \hskip0.5cm%
    }%
    \vskip6pt%
}

%=============================================================================
% PACKAGES
%=============================================================================
\usepackage[utf8]{inputenc}
\usepackage[T1]{fontenc}
\usepackage[english]{babel}
\usepackage{amsmath, amssymb, amsthm}
\usepackage{mathtools}
\usepackage{bm}
\usepackage{tikz}
\usetikzlibrary{arrows.meta, positioning, shapes, calc, decorations.pathreplacing, shadings}
\usepackage{booktabs}
\usepackage{multirow}
\usepackage{array}
\usepackage{graphicx}
\usepackage{hyperref}
\usepackage{colortbl}
\usepackage{listings}
\lstset{basicstyle=\ttfamily\small, breaklines=true, frame=single, backgroundcolor=\color{VeryLightGray}}
\hypersetup{colorlinks=true, linkcolor=MainBlue, urlcolor=MainBlue}
\graphicspath{{../../logos/}{../../charts/}{../../photos/}}
\hfuzz=2pt  % Suppress tiny overfull warnings (<2pt)
\vfuzz=2pt  % Suppress tiny vertical overfull warnings (<2pt)

%=============================================================================
% QUANTLET COMMAND
%=============================================================================
\newcommand{\quantlet}[2]{%
    \hfill\href{#2}{%
        \raisebox{-0.15em}{\includegraphics[height=0.7em]{ql_logo.png}}%
        \textcolor{MainBlue}{\tiny\ #1}%
    }%
}

%=============================================================================
% CUSTOM TITLE PAGE
%=============================================================================
\defbeamertemplate*{title page}{hybrid}[1][]
{
    \vspace{0.2cm}
    % Logos row - top header (with clickable links)
    \begin{center}
        \href{https://www.ase.ro}{\includegraphics[height=1.0cm]{ase_logo.png}}\hspace{0.25cm}%
        \href{https://theida.net}{\includegraphics[height=1.0cm]{ida_logo.png}}\hspace{0.25cm}%
        \href{https://blockchain-research-center.com}{\includegraphics[height=1.0cm]{brc_logo.png}}\hspace{0.25cm}%
        \href{https://www.ai4efin.ase.ro}{\includegraphics[height=1.0cm]{ai4efin_logo.png}}\hspace{0.25cm}%
        \href{https://ipe.ro/new}{\includegraphics[height=1.0cm]{acad_logo.png}}\hspace{0.25cm}%
        \href{https://www.digital-finance-msca.com}{\includegraphics[height=1.0cm]{msca_logo.png}}%
    \end{center}

    \vspace{0.6cm}

    % Main title with Q logos on sides (with clickable links)
    \begin{center}
        \begin{minipage}{0.1\textwidth}
            \centering
            \href{https://quantlet.com}{\includegraphics[height=1.1cm]{ql_logo.png}}
        \end{minipage}%
        \begin{minipage}{0.78\textwidth}
            \centering
            {\LARGE\bfseries\usebeamercolor[fg]{title}\inserttitle}

            \vspace{0.3cm}

            {\usebeamerfont{subtitle}\usebeamercolor[fg]{title}\insertsubtitle}
        \end{minipage}%
        \begin{minipage}{0.1\textwidth}
            \centering
            \href{https://quantinar.com}{\includegraphics[height=1.1cm]{qr_logo.png}}
        \end{minipage}
    \end{center}

    \vspace{0.6cm}

    % Authors (left aligned)
    \hspace{0.5cm}{\usebeamerfont{author}\insertauthor}

    \vspace{0.3cm}

    % Institute/Affiliations (left aligned)
    \hspace{0.5cm}\begin{minipage}[t]{0.9\textwidth}
        \raggedright\small\insertinstitute
    \end{minipage}
}

%=============================================================================
% THEOREM ENVIRONMENTS
%=============================================================================
\theoremstyle{definition}
\setbeamertemplate{theorems}[numbered]
\newtheorem{defn}{Definition}
\newtheorem{thm}{Theorem}
\newtheorem{prop}{Proposition}
\newtheorem{rmk}{Remark}

%=============================================================================
% CENTRED MINIPAGE (no extra vertical space)
%=============================================================================
\newenvironment{cminipage}[1]{%
    \par\noindent\hfill\begin{minipage}{#1}\ignorespaces
}{%
    \end{minipage}\hfill\null\par
}

%=============================================================================
% CUSTOM COMMANDS
%=============================================================================
\newcommand{\E}{\mathbb{E}}
\newcommand{\Var}{\text{Var}}
\newcommand{\Cov}{\text{Cov}}
\newcommand{\Corr}{\text{Corr}}
\newcommand{\R}{\mathbb{R}}
\newcommand{\N}{\mathbb{N}}
\newcommand{\Z}{\mathbb{Z}}
\newcommand{\B}{\mathbf{B}}
\newcommand{\imark}{\textcolor{MainBlue}{\textbullet}}
\newcommand{\RMSE}{\text{RMSE}}
\newcommand{\MAE}{\text{MAE}}
\newcommand{\MAPE}{\text{MAPE}}
\newcommand{\correct}{\textcolor{Forest}{\checkmark}}
\newcommand{\incorrect}{\textcolor{Crimson}{\texttimes}}

% Boldface vector/matrix commands
\newcommand{\bY}{\mathbf{Y}}
\newcommand{\bX}{\mathbf{X}}
\newcommand{\bA}{\mathbf{A}}
\newcommand{\bB}{\mathbf{B}}
\newcommand{\bepsilon}{\boldsymbol{\varepsilon}}
\newcommand{\bvarepsilon}{\boldsymbol{\varepsilon}}
\newcommand{\bSigma}{\boldsymbol{\Sigma}}
\newcommand{\bPhi}{\boldsymbol{\Phi}}
\newcommand{\bGamma}{\boldsymbol{\Gamma}}
\newcommand{\bPi}{\boldsymbol{\Pi}}
\newcommand{\bc}{\mathbf{c}}
\newcommand{\balpha}{\boldsymbol{\alpha}}
\newcommand{\bbeta}{\boldsymbol{\beta}}

%=============================================================================
% TITLE INFORMATION
%=============================================================================
\title[Time Series Analysis]{Time Series Analysis and Forecasting}
\author[D.T. Pele]{Daniel Traian PELE}
\institute{Bucharest University of Economic Studies\\
IDA Institute Digital Assets\\
Blockchain Research Center\\
AI4EFin Artificial Intelligence for Energy Finance\\
Romanian Academy, Institute for Economic Forecasting\\
MSCA Digital Finance}
\date{}

\subtitle{Seminar 8: Extensii Moderne}

\begin{document}

{
\setbeamertemplate{headline}{}
\setbeamertemplate{footline}{}
\begin{frame}
    \titlepage
\end{frame}
}


%=============================================================================
% OUTLINE
%=============================================================================
\begin{frame}{Cuprins Seminar}
    \tableofcontents
\end{frame}

%=============================================================================
% SECTION 1: ARFIMA QUIZ
%=============================================================================
\section{Test: Modele ARFIMA și Memorie Lungă}

\begin{frame}{Test 1: Exponentul Hurst}
    \begin{alertblock}{Întrebare}
        O serie de timp are exponentul Hurst $H = 0.8$. Ce ne indică acest lucru?
    \end{alertblock}

    \vspace{0.3cm}

    \begin{enumerate}[A)]
        \item Seria este un mers aleator pur
        \item Seria are memorie lungă și este persistentă (trend-following)
        \item Seria este anti-persistentă (mean-reverting)
        \item Seria este staționară I(0)
    \end{enumerate}

    \vspace{0.5cm}
    \begin{flushright}\textit{Răspunsul pe slide-ul următor...}\end{flushright}
\end{frame}

\begin{frame}{Test 1: Răspuns}
    \begin{exampleblock}{Răspuns: B -- Memorie lungă și persistență}
        \textbf{Interpretare Exponent Hurst:}
        \begin{itemize}
            \item $H = 0.5$: Mers aleator (fără memorie)
            \item $0.5 < H < 1$: \textbf{Persistență} -- tendința continuă
            \item $0 < H < 0.5$: Anti-persistență -- revenire la medie
        \end{itemize}

        \vspace{0.3cm}
        Cu $H = 0.8 > 0.5$:
        \begin{itemize}
            \item Seria are \textbf{memorie lungă}
            \item Valorile mari tind să fie urmate de valori mari
            \item Autocorelațiile descresc lent (hiperbolic, nu exponențial)
        \end{itemize}
    \end{exampleblock}
    \quantlet{TSA\_ch8\_hurst\_interpretation}{https://github.com/QuantLet/TSA/tree/main/TSA_ch8/TSA_ch8_hurst_interpretation}
\end{frame}

\begin{frame}{Test 2: Parametrul de Diferențiere Fracționară}
    \begin{alertblock}{Întrebare}
        În modelul ARFIMA(p, d, q), parametrul $d$ poate lua valori:
    \end{alertblock}

    \vspace{0.3cm}

    \begin{enumerate}[A)]
        \item Doar valori întregi (0, 1, 2, ...)
        \item Doar $d = 0$ sau $d = 1$
        \item Orice valoare reală, inclusiv fracționară
        \item Doar valori negative
    \end{enumerate}

    \vspace{0.5cm}
    \begin{flushright}\textit{Răspunsul pe slide-ul următor...}\end{flushright}
\end{frame}

\begin{frame}{Test 2: Răspuns}
    \begin{exampleblock}{Răspuns: C -- Orice valoare reală}
        \textbf{Diferențiere fracționară}: $(1-L)^d$ cu $d \in \mathbb{R}$

        \vspace{0.3cm}
        \textbf{Interpretare valori $d$:}
        \begin{itemize}
            \item $d = 0$: Seria staționară (ARMA)
            \item $0 < d < 0.5$: Memorie lungă, staționară
            \item $d = 0.5$: Granița staționar/nestaționară
            \item $0.5 < d < 1$: Memorie lungă, nestaționară
            \item $d = 1$: Diferențiere completă (ARIMA clasic)
        \end{itemize}

        \vspace{0.2cm}
        \textbf{Relația cu Hurst}: $d = H - 0.5$
    \end{exampleblock}
    \quantlet{TSA\_ch8\_arfima\_d\_effect}{https://github.com/QuantLet/TSA/tree/main/TSA_ch8/TSA_ch8_arfima_d_effect}
\end{frame}

\begin{frame}{Test 3: Memoria Lungă în Serii Financiare}
    \begin{alertblock}{Întrebare}
        În ce serie financiară este memoria lungă cel mai frecvent documentată?
    \end{alertblock}

    \vspace{0.3cm}

    \begin{enumerate}[A)]
        \item Prețurile acțiunilor
        \item Randamentele zilnice
        \item Volatilitatea (pătratul randamentelor)
        \item Volumul de tranzacționare
    \end{enumerate}

    \vspace{0.5cm}
    \begin{flushright}\textit{Răspunsul pe slide-ul următor...}\end{flushright}
\end{frame}

\begin{frame}{Test 3: Răspuns}
    \begin{exampleblock}{Răspuns: C -- Volatilitatea}
        \textbf{Fapte stilizate din finanțe:}
        \begin{itemize}
            \item \textbf{Randamentele}: Aproximativ fără memorie ($H \approx 0.5$)
            \item \textbf{Volatilitatea}: Memorie lungă pronunțată ($H \approx 0.7-0.9$)
        \end{itemize}

        \vspace{0.3cm}
        \textbf{De ce?}
        \begin{itemize}
            \item Volatility clustering: perioade turbulente urmate de perioade turbulente
            \item Persistența șocurilor în varianță
            \item FIGARCH: modelează explicit memoria lungă în volatilitate
        \end{itemize}

        \vspace{0.2cm}
        {\footnotesize Acest fapt stilizat este baza modelelor FIGARCH și HAR-RV.}
    \end{exampleblock}
    \quantlet{TSA\_ch8\_volatility\_long\_memory}{https://github.com/QuantLet/TSA/tree/main/TSA_ch8/TSA_ch8_volatility_long_memory}
\end{frame}

%=============================================================================
% SECTION 2: MACHINE LEARNING QUIZ
%=============================================================================
\section{Test: Machine Learning pentru Serii de Timp}

\begin{frame}{Test 4: Feature Engineering}
    \begin{alertblock}{Întrebare}
        Pentru a aplica Random Forest pe serii de timp, trebuie să creăm:
    \end{alertblock}

    \vspace{0.3cm}

    \begin{enumerate}[A)]
        \item Variabile dummy pentru fiecare observație
        \item Caracteristici lag și statistici rolling
        \item Transformări Fourier ale seriei
        \item Numai prima diferență a seriei
    \end{enumerate}

    \vspace{0.5cm}
    \begin{flushright}\textit{Răspunsul pe slide-ul următor...}\end{flushright}
\end{frame}

\begin{frame}{Test 4: Răspuns}
    \begin{exampleblock}{Răspuns: B -- Caracteristici lag și statistici rolling}
        \textbf{Feature Engineering pentru Serii de Timp:}

        \vspace{0.2cm}
        \begin{itemize}
            \item \textbf{Lag features}: $y_{t-1}, y_{t-2}, \ldots, y_{t-k}$
            \item \textbf{Rolling statistics}:
                \begin{itemize}
                    \item Media mobilă: $\bar{y}_{t,w}$
                    \item Deviația standard mobilă: $\sigma_{t,w}$
                    \item Min/Max pe fereastră
                \end{itemize}
            \item \textbf{Caracteristici calendaristice}: ziua săptămânii, luna, etc.
        \end{itemize}

        \vspace{0.2cm}
        \textbf{Important}: Transformă problema de prognoză în problemă de regresie supervizată!
    \end{exampleblock}
    \quantlet{TSA\_ch8\_feature\_engineering}{https://github.com/QuantLet/TSA/tree/main/TSA_ch8/TSA_ch8_feature_engineering}
\end{frame}

\begin{frame}{Test 5: Validarea Încrucișată pentru Serii de Timp}
    \begin{alertblock}{Întrebare}
        De ce \emph{nu} putem folosi k-fold cross-validation standard pentru serii de timp?
    \end{alertblock}

    \vspace{0.3cm}

    \begin{enumerate}[A)]
        \item Este prea lent pentru serii lungi
        \item Încalcă ordinea temporală și cauzează data leakage
        \item Funcționează doar pentru clasificare
        \item Necesită prea multe date
    \end{enumerate}

    \vspace{0.5cm}
    \begin{flushright}\textit{Răspunsul pe slide-ul următor...}\end{flushright}
\end{frame}

\begin{frame}{Test 5: Răspuns}
    \begin{exampleblock}{Răspuns: B -- Încalcă ordinea temporală}
        \textbf{Problema cu k-fold standard:}
        \begin{itemize}
            \item Amestecă ordinea temporală a observațiilor
            \item Antrenează pe date din viitor, testează pe trecut
            \item \textbf{Data leakage} $\Rightarrow$ performanță supraestimată
        \end{itemize}

        \vspace{0.3cm}
        \textbf{Soluția: Time Series Split (Walk-Forward)}
        \begin{center}
        \begin{tikzpicture}[scale=0.7]
            \draw[fill=MainBlue!30] (0,0) rectangle (3,0.4) node[midway] {\tiny Train};
            \draw[fill=Crimson!30] (3,0) rectangle (4,0.4) node[midway] {\tiny Test};
            \draw[fill=MainBlue!30] (0,-0.6) rectangle (4,-0.2) node[midway] {\tiny Train};
            \draw[fill=Crimson!30] (4,-0.6) rectangle (5,-0.2) node[midway] {\tiny Test};
            \draw[fill=MainBlue!30] (0,-1.2) rectangle (5,-0.8) node[midway] {\tiny Train};
            \draw[fill=Crimson!30] (5,-1.2) rectangle (6,-0.8) node[midway] {\tiny Test};
        \end{tikzpicture}
        \end{center}
    \end{exampleblock}
    \quantlet{TSA\_ch8\_timeseries\_cv}{https://github.com/QuantLet/TSA/tree/main/TSA_ch8/TSA_ch8_timeseries_cv}
\end{frame}

\begin{frame}{Test 6: Importanța Variabilelor în Random Forest}
    \begin{alertblock}{Întrebare}
        Importanța variabilelor în Random Forest pentru serii de timp ne ajută să:
    \end{alertblock}

    \vspace{0.3cm}

    \begin{enumerate}[A)]
        \item Să eliminăm toate variabilele cu importanță mică
        \item Să identificăm care lag-uri și caracteristici sunt cele mai predictive
        \item Să determinăm cauzalitatea Granger
        \item Să calculăm intervalele de încredere
    \end{enumerate}

    \vspace{0.5cm}
    \begin{flushright}\textit{Răspunsul pe slide-ul următor...}\end{flushright}
\end{frame}

\begin{frame}{Test 6: Răspuns}
    \begin{exampleblock}{Răspuns: B -- Identifică caracteristicile predictive}
        \textbf{Utilizări ale importanței variabilelor:}
        \begin{itemize}
            \item Înțelegerea structurii temporale
            \item Selectarea numărului optim de lag-uri
            \item Identificarea factorilor relevanți
        \end{itemize}

        \vspace{0.3cm}
        \textbf{Atenție:}
        \begin{itemize}
            \item Importanța \emph{nu} implică cauzalitate
            \item Variabilele corelate pot împărtăși importanța
            \item Folosiți pentru interpretare, nu pentru inferență cauzală
        \end{itemize}
    \end{exampleblock}
    \quantlet{TSA\_ch8\_rf\_prediction}{https://github.com/QuantLet/TSA/tree/main/TSA_ch8/TSA_ch8_rf_prediction}
\end{frame}

%=============================================================================
% SECTION 3: LSTM QUIZ
%=============================================================================
\section{Test: Rețele LSTM}

\begin{frame}{Test 7: Avantajul LSTM}
    \begin{alertblock}{Întrebare}
        Care este principalul avantaj al LSTM față de RNN simple?
    \end{alertblock}

    \vspace{0.3cm}

    \begin{enumerate}[A)]
        \item Este mai rapidă la antrenare
        \item Rezolvă problema gradienților care dispar/explodează
        \item Necesită mai puține date
        \item Este mai ușor de interpretat
    \end{enumerate}

    \vspace{0.5cm}
    \begin{flushright}\textit{Răspunsul pe slide-ul următor...}\end{flushright}
\end{frame}

\begin{frame}{Test 7: Răspuns}
    \begin{exampleblock}{Răspuns: B -- Rezolvă problema gradienților}
        \textbf{Problema RNN Simple:}
        \begin{itemize}
            \item Gradienții scad exponențial cu lungimea secvenței
            \item Nu pot învăța dependențe pe termen lung
        \end{itemize}

        \vspace{0.3cm}
        \textbf{Soluția LSTM:}
        \begin{itemize}
            \item \textbf{Cell state}: Autostradă pentru flux de informație
            \item \textbf{Forget gate}: Decide ce să uite
            \item \textbf{Input gate}: Decide ce să rețină
            \item \textbf{Output gate}: Decide ce să producă
        \end{itemize}

        \vspace{0.2cm}
        Gradienții pot „curge" prin cell state fără degradare!
    \end{exampleblock}
    \quantlet{TSA\_ch8\_lstm\_cell}{https://github.com/QuantLet/TSA/tree/main/TSA_ch8/TSA_ch8_lstm_cell}
\end{frame}

\begin{frame}{Test 8: Pregătirea Datelor pentru LSTM}
    \begin{alertblock}{Întrebare}
        Înainte de antrenarea LSTM, datele trebuie:
    \end{alertblock}

    \vspace{0.3cm}

    \begin{enumerate}[A)]
        \item Transformate logaritmic
        \item Normalizate/standardizate la intervalul [0,1] sau [-1,1]
        \item Diferențiate de ordinul 2
        \item Convertite la numere întregi
    \end{enumerate}

    \vspace{0.5cm}
    \begin{flushright}\textit{Răspunsul pe slide-ul următor...}\end{flushright}
\end{frame}

\begin{frame}{Test 8: Răspuns}
    \begin{exampleblock}{Răspuns: B -- Normalizate/standardizate}
        \textbf{De ce normalizare?}
        \begin{itemize}
            \item Funcțiile de activare (sigmoid, tanh) funcționează în intervale limitate
            \item Convergență mai rapidă
            \item Stabilitate numerică
        \end{itemize}

        \vspace{0.3cm}
        \textbf{Metode comune:}
        \begin{itemize}
            \item \textbf{Min-Max}: $x' = \frac{x - x_{min}}{x_{max} - x_{min}}$ $\succ$ [0, 1]
            \item \textbf{Standard}: $x' = \frac{x - \mu}{\sigma}$ $\succ$ media 0, std 1
        \end{itemize}

        \vspace{0.2cm}
        \textbf{Important}: Fit pe train, transform pe train+test!
    \end{exampleblock}
    \quantlet{TSA\_ch8\_lstm\_architecture}{https://github.com/QuantLet/TSA/tree/main/TSA_ch8/TSA_ch8_lstm_architecture}
\end{frame}

\begin{frame}{Test 9: Hiperparametri LSTM}
    \begin{alertblock}{Întrebare}
        Care \emph{nu} este un hiperparametru tipic pentru LSTM?
    \end{alertblock}

    \vspace{0.3cm}

    \begin{enumerate}[A)]
        \item Numărul de unități (neuroni) pe strat
        \item Lungimea secvenței de intrare
        \item Learning rate
        \item Parametrul de diferențiere $d$
    \end{enumerate}

    \vspace{0.5cm}
    \begin{flushright}\textit{Răspunsul pe slide-ul următor...}\end{flushright}
\end{frame}

\begin{frame}{Test 9: Răspuns}
    \begin{exampleblock}{Răspuns: D -- Parametrul $d$}
        $d$ este specific modelelor ARFIMA, nu LSTM!

        \vspace{0.3cm}
        \textbf{Hiperparametri LSTM:}
        \begin{itemize}
            \item \textbf{Arhitectură}: nr. straturi, unități/strat
            \item \textbf{Secvență}: lungime lookback
            \item \textbf{Training}: learning rate, batch size, epochs
            \item \textbf{Regularizare}: dropout, early stopping
        \end{itemize}

        \vspace{0.2cm}
        \textbf{Tuning}: Căutare pe grilă (\emph{grid search}) sau optimizare bayesiană cu validare încrucișată temporală
    \end{exampleblock}
    \quantlet{TSA\_ch8\_lstm\_architecture}{https://github.com/QuantLet/TSA/tree/main/TSA_ch8/TSA_ch8_lstm_architecture}
\end{frame}

%=============================================================================
% SECTION: TRUE/FALSE QUESTIONS
%=============================================================================
\section{Întrebări Adevărat/Fals}

\begin{frame}{Întrebări Adevărat/Fals}
    Determinați dacă fiecare afirmație este Adevărată sau Falsă:

    \vspace{0.3cm}
    \begin{enumerate}
        \item ARFIMA permite un parametru de diferențiere fracționar $d \in (0, 0.5)$.
        \item Validarea încrucișată k-fold standard este adecvată pentru serii de timp.
        \item Rețelele LSTM pot captura dependențe pe termen lung prin mecanismul de celulă.
        \item Supraajustarea nu este o problemă pentru modelele de învățare automată aplicate pe serii de timp.
        \item Random Forest este un model de tip bagging care agregă mai mulți arbori de decizie.
        \item Temporal Fusion Transformer combină atenția cu variabile statice și cunoscute în viitor.
    \end{enumerate}

    \vspace{0.3cm}
    \begin{flushright}\textit{Răspunsurile pe slide-ul următor...}\end{flushright}
\end{frame}

\begin{frame}{Adevărat/Fals: Soluții}
    {\small
    \begin{enumerate}\setlength{\itemsep}{1pt}
        \item ARFIMA permite un parametru de diferențiere fracționar $d \in (0, 0.5)$. \hfill \textcolor{Forest}{\textbf{ADEVĂRAT}}

        {\footnotesize \textcolor{MediumGray}{Diferențierea fracționară generalizează ARIMA: $d \in (0, 0.5)$ înseamnă memorie lungă și staționaritate.}}

        \item Validarea încrucișată k-fold standard este adecvată pentru serii de timp. \hfill \textcolor{Crimson}{\textbf{FALS}}

        {\footnotesize \textcolor{MediumGray}{K-fold standard amestecă ordinea temporală, cauzând data leakage. Folosiți TimeSeriesSplit.}}

        \item Rețelele LSTM pot captura dependențe pe termen lung prin mecanismul de celulă. \hfill \textcolor{Forest}{\textbf{ADEVĂRAT}}

        {\footnotesize \textcolor{MediumGray}{Cell state-ul LSTM permite fluxul de informație pe distanțe mari, rezolvând problema gradienților care dispar.}}

        \item Supraajustarea nu este o problemă pentru ML pe serii de timp. \hfill \textcolor{Crimson}{\textbf{FALS}}

        {\footnotesize \textcolor{MediumGray}{Supraajustarea este o problemă majoră, mai ales cu date limitate și multe features. Folosiți regularizare și early stopping.}}

        \item Random Forest este un model de tip bagging care agregă mai mulți arbori de decizie. \hfill \textcolor{Forest}{\textbf{ADEVĂRAT}}

        {\footnotesize \textcolor{MediumGray}{RF combină bootstrap aggregating (bagging) cu selecția aleatoare a features la fiecare nod.}}

        \item Temporal Fusion Transformer combină atenția cu variabile statice și cunoscute în viitor. \hfill \textcolor{Forest}{\textbf{ADEVĂRAT}}

        {\footnotesize \textcolor{MediumGray}{TFT folosește self-attention, variable selection networks și gating mechanisms pentru prognoze interpretabile.}}
    \end{enumerate}
    }
\end{frame}

%=============================================================================
% SECTION 4: PRACTICAL PROBLEMS
%=============================================================================
\section{Probleme Practice}

\begin{frame}{Problemă 1: Estimarea Exponentului Hurst}
    \begin{block}{Enunț}
        Fie seria zilnică de randamente Bitcoin. Estimați exponentul Hurst folosind metoda R/S și interpretați rezultatul.
    \end{block}

    \vspace{0.3cm}
    \textbf{Pași de rezolvare:}
    \begin{enumerate}
        \item Calculați media pe subintervale de diferite lungimi $n$
        \item Pentru fiecare $n$: calculați Range($R$) și Std($S$)
        \item Raportul $R/S$ crește ca $n^H$
        \item Fit regresie: $\log(R/S) = H \cdot \log(n) + c$
    \end{enumerate}

    \vspace{0.2cm}
    \textbf{Cod Python}: \texttt{nolds.hurst\_rs(returns)}
    \quantlet{TSA\_ch8\_hurst\_interpretation}{https://github.com/QuantLet/TSA/tree/main/TSA_ch8/TSA_ch8_hurst_interpretation}
\end{frame}

\begin{frame}{Problemă 1: Soluție și Interpretare}
    \begin{exampleblock}{Rezultat Tipic pentru Bitcoin}
        \begin{itemize}
            \item Randamente: $H \approx 0.45-0.55$ (aproape mers aleator)
            \item Volatilitate (|returns|): $H \approx 0.75-0.85$ (memorie lungă!)
        \end{itemize}
    \end{exampleblock}

    \vspace{0.3cm}
    \textbf{Interpretare:}
    \begin{itemize}
        \item Randamentele sunt greu de prognozat (EMH aproximativ validă)
        \item Volatilitatea este predictibilă pe termen lung
        \item Implicații pentru managementul riscului și VaR
    \end{itemize}

    \vspace{0.2cm}
    \textbf{Aplicație}: Modelele FIGARCH pot fi superioare GARCH standard
    \quantlet{TSA\_ch8\_eurron\_long\_memory}{https://github.com/QuantLet/TSA/tree/main/TSA_ch8/TSA_ch8_eurron_long_memory}
\end{frame}

\begin{frame}{Problemă 2: Random Forest pentru Prognoză}
    \begin{block}{Enunț}
        Construiți un model Random Forest pentru prognoza prețului Bitcoin la orizontul de 1 zi. Evaluați folosind TimeSeriesSplit.
    \end{block}

    \vspace{0.3cm}
    \textbf{Pipeline:}
    \begin{enumerate}
        \item \textbf{Feature engineering}:
            \begin{itemize}
                \item Lag-uri: $y_{t-1}, y_{t-2}, \ldots, y_{t-7}$
                \item Rolling mean/std: 7, 14, 30 zile
            \end{itemize}
        \item \textbf{Train/Test split}: TimeSeriesSplit(n\_splits=5)
        \item \textbf{Model}: RandomForestRegressor(n\_estimators=100)
        \item \textbf{Evaluare}: RMSE, MAE, Direction Accuracy
    \end{enumerate}
    \quantlet{TSA\_ch8\_feature\_engineering}{https://github.com/QuantLet/TSA/tree/main/TSA_ch8/TSA_ch8_feature_engineering}
\end{frame}

\begin{frame}{Problemă 2: Cod și Rezultate}
    \begin{exampleblock}{Cod Python}
        {\footnotesize
        \texttt{from sklearn.ensemble import RandomForestRegressor}\\
        \texttt{from sklearn.model\_selection import TimeSeriesSplit}\\[0.2cm]
        \texttt{tscv = TimeSeriesSplit(n\_splits=5)}\\
        \texttt{rf = RandomForestRegressor(n\_estimators=100)}\\[0.2cm]
        \texttt{for train\_idx, test\_idx in tscv.split(X):}\\
        \texttt{~~~~rf.fit(X[train\_idx], y[train\_idx])}\\
        \texttt{~~~~pred = rf.predict(X[test\_idx])}
        }
    \end{exampleblock}

    \vspace{0.2cm}
    \textbf{Rezultate tipice}:
    \begin{itemize}
        \item Direction accuracy: 52-55\% (puțin peste random)
        \item Feature importance: lag-1 și rolling\_std domină
    \end{itemize}
    \quantlet{TSA\_ch8\_rf\_prediction}{https://github.com/QuantLet/TSA/tree/main/TSA_ch8/TSA_ch8_rf_prediction}
\end{frame}

\begin{frame}{Problemă 3: LSTM pentru Serii de Timp}
    \begin{block}{Enunț}
        Implementați un model LSTM simplu pentru prognoza Bitcoin. Comparați cu Random Forest.
    \end{block}

    \vspace{0.3cm}
    \textbf{Arhitectură LSTM simplă:}
    \begin{enumerate}
        \item Input: secvențe de 30 de zile
        \item LSTM layer: 50 unități
        \item Dense output: 1 neuron (prognoza)
        \item Loss: MSE, Optimizer: Adam
    \end{enumerate}

    \vspace{0.2cm}
    \textbf{Pași importanți}:
    \begin{itemize}
        \item Normalizare MinMaxScaler
        \item Reshape la [samples, timesteps, features]
        \item Early stopping pentru a evita supraajustare
    \end{itemize}
    \quantlet{TSA\_ch8\_lstm\_architecture}{https://github.com/QuantLet/TSA/tree/main/TSA_ch8/TSA_ch8_lstm_architecture}
\end{frame}

\begin{frame}{Problemă 3: Cod LSTM}
    \begin{exampleblock}{Cod Keras/TensorFlow}
        {\footnotesize
        \texttt{from tensorflow.keras.models import Sequential}\\
        \texttt{from tensorflow.keras.layers import LSTM, Dense}\\[0.2cm]
        \texttt{model = Sequential([}\\
        \texttt{~~~~LSTM(50, input\_shape=(30, 1)),}\\
        \texttt{~~~~Dense(1)}\\
        \texttt{])}\\[0.2cm]
        \texttt{model.compile(optimizer='adam', loss='mse')}\\
        \texttt{model.fit(X\_train, y\_train, epochs=50,}\\
        \texttt{~~~~~~~~~~validation\_split=0.1, verbose=0)}
        }
    \end{exampleblock}

    \vspace{0.2cm}
    \textbf{Comparație tipică RF vs LSTM}:
    \begin{itemize}
        \item RMSE similar (LSTM ușor mai bun pe date netede)
        \item RF: mai rapid, mai interpretabil
        \item LSTM: captează mai bine pattern-uri complexe
    \end{itemize}
    \quantlet{TSA\_ch8\_case\_lstm\_training}{https://github.com/QuantLet/TSA/tree/main/TSA_ch8/TSA_ch8_case_lstm_training}
\end{frame}

%=============================================================================
% EXERCIȚIU CU ASISTENȚĂ AI
%=============================================================================
\section{Exercițiu cu asistență AI}

\begin{frame}{Exercițiu AI: Gândire critică}
    \vspace{-3mm}
    \begin{block}{\footnotesize Prompt de testat în ChatGPT / Claude / Copilot}
        {\footnotesize
        ``Folosind yfinance, descarcă prețurile zilnice Ethereum (ETH-USD) din 2020-01-01 până în 2024-12-31 (aprox.\ 1.800 observații). Calculează randamentele logaritmice. Compară trei modele: ARIMA, XGBoost (cu lag-uri și variabile calendar ca features) și un LSTM simplu (2 straturi, lookback 30 zile). Folosește walk-forward validation cu fereastră de antrenare de 365 zile și orizont de 14 zile. Evaluează cu RMSE și direcțional accuracy. Vreau cod Python complet cu grafice comparative.''
        }
    \end{block}
    \vspace{-2mm}
    {\footnotesize
    \textbf{Exercițiu}:
    \begin{enumerate}\setlength{\itemsep}{0pt}
        \item Rulați prompt-ul într-un LLM la alegere și analizați critic răspunsul.
        \item Cum construiește features pentru XGBoost? Lag-uri, variabile calendar, termeni Fourier?
        \item LSTM-ul e structurat corect? (scalare MinMax, forma intrărilor 3D, fără data leakage)
        \item Folosește walk-forward validation sau doar un singur split train/test?
        \item Discută compromisul între interpretabilitate, complexitate computațională și acuratețe?
    \end{enumerate}
    }
    \vspace{-2mm}
    \begin{alertblock}{}
        {\footnotesize \textbf{Atenție}: Codul generat de AI poate rula fără erori și arăta profesional. \textit{Asta nu înseamnă că e corect.}}
    \end{alertblock}
    \quantlet{TSA\_ch8\_case\_comparison}{https://github.com/QuantLet/TSA/tree/main/TSA_ch8/TSA_ch8_case_comparison}
\end{frame}

%=============================================================================
% SECTION 5: SUMMARY
%=============================================================================
\section{Recapitulare}

\begin{frame}{Rezumat: Când să Folosim Fiecare Metodă}
    \begin{block}{ARFIMA}
        \begin{itemize}
            \item Serii cu memorie lungă (volatilitate, hidrologie)
            \item Când $0 < d < 0.5$ este teoretic justificat
            \item Interpretabilitate statistică importantă
        \end{itemize}
    \end{block}

    \begin{block}{Random Forest}
        \begin{itemize}
            \item Relații neliniare între caracteristici
            \item Feature importance pentru înțelegere
            \item Date structurate, nu foarte lungi
        \end{itemize}
    \end{block}

    \begin{block}{LSTM}
        \begin{itemize}
            \item Secvențe lungi cu dependențe complexe
            \item Date suficiente pentru deep learning
            \item Pattern-uri dificil de capturat cu metode clasice
        \end{itemize}
    \end{block}
    \quantlet{TSA\_ch8\_case\_comparison}{https://github.com/QuantLet/TSA/tree/main/TSA_ch8/TSA_ch8_case_comparison}
\end{frame}

\begin{frame}{Formule Cheie}
    \begin{block}{ARFIMA și Memorie Lungă}
        \begin{itemize}
            \item Diferențiere fracționară: $(1-L)^d y_t = \varepsilon_t$
            \item Exponent Hurst: $d = H - 0.5$
            \item ACF pentru memorie lungă: $\rho(k) \sim k^{2d-1}$ (descreștere lentă)
        \end{itemize}
    \end{block}

    \begin{block}{Machine Learning}
        \begin{itemize}
            \item Feature lag: $X_t = [y_{t-1}, y_{t-2}, \ldots, y_{t-k}]$
            \item RMSE: $\sqrt{\frac{1}{n}\sum(y_i - \hat{y}_i)^2}$
            \item Direction Accuracy: $\frac{1}{n}\sum \mathbf{1}[\text{sign}(\Delta y) = \text{sign}(\Delta \hat{y})]$
        \end{itemize}
    \end{block}

    \begin{block}{LSTM}
        \begin{itemize}
            \item Forget gate: $f_t = \sigma(W_f \cdot [h_{t-1}, x_t] + b_f)$
            \item Cell update: $C_t = f_t * C_{t-1} + i_t * \tilde{C}_t$
        \end{itemize}
    \end{block}
    \quantlet{TSA\_ch8\_arfima\_d\_effect}{https://github.com/QuantLet/TSA/tree/main/TSA_ch8/TSA_ch8_arfima_d_effect}
\end{frame}

%=============================================================================
% THANK YOU SLIDE
%=============================================================================
\begin{frame}[plain]
    \begin{tikzpicture}[remember picture, overlay]
        \fill[IDAred] (current page.north west) rectangle ([yshift=-0.15cm]current page.north east);
    \end{tikzpicture}
    \vfill
    \begin{center}
        {\Huge\textbf{\textcolor{MainBlue}{Vă mulțumesc!}}}\\[1cm]
        {\Large Întrebări?}\\[0.5cm]
        {\large\texttt{danpele@ase.ro}}
    \end{center}
    \vfill
    \begin{tikzpicture}[remember picture, overlay]
        \fill[IDAred] (current page.south west) rectangle ([yshift=0.15cm]current page.south east);
    \end{tikzpicture}
\end{frame}

\end{document}
