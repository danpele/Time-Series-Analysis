% Capitolul 0: Seminar - Fundamentele Seriilor de Timp
% Teste, Probleme Practice și Discuții
% Program de licență, Academia de Studii Economice din București

\documentclass[9pt, aspectratio=169, t]{beamer}

% Ensure content fits on slides
\setbeamersize{text margin left=8mm, text margin right=8mm}

%=============================================================================
% THEME AND STYLE CONFIGURATION
%=============================================================================
\usetheme{Madrid}
\usecolortheme{seahorse}

% IDA-Inspired Color Palette
\definecolor{MainBlue}{RGB}{26, 58, 110}
\definecolor{AccentBlue}{RGB}{42, 82, 140}
\definecolor{IDAred}{RGB}{220, 53, 69}
\definecolor{DarkGray}{RGB}{51, 51, 51}
\definecolor{MediumGray}{RGB}{128, 128, 128}
\definecolor{LightGray}{RGB}{248, 248, 248}
\definecolor{VeryLightGray}{RGB}{235, 235, 235}
\definecolor{Crimson}{RGB}{220, 53, 69}
\definecolor{Forest}{RGB}{46, 125, 50}
\definecolor{Amber}{RGB}{181, 133, 63}
\definecolor{Orange}{RGB}{230, 126, 34}

\setbeamercolor{palette primary}{bg=MainBlue, fg=white}
\setbeamercolor{palette secondary}{bg=MainBlue!85, fg=white}
\setbeamercolor{palette tertiary}{bg=MainBlue!70, fg=white}
\setbeamercolor{structure}{fg=MainBlue}
\setbeamercolor{title}{fg=MainBlue}
\setbeamercolor{frametitle}{fg=MainBlue, bg=white}
\setbeamercolor{block title}{bg=MainBlue, fg=white}
\setbeamercolor{block body}{bg=VeryLightGray, fg=DarkGray}
\setbeamercolor{block title alerted}{bg=Crimson, fg=white}
\setbeamercolor{block body alerted}{bg=Crimson!8, fg=DarkGray}
\setbeamercolor{block title example}{bg=Forest, fg=white}
\setbeamercolor{block body example}{bg=Forest!8, fg=DarkGray}
\setbeamercolor{item}{fg=MainBlue}

\setbeamertemplate{navigation symbols}{}

\setbeamertemplate{footline}{
    \leavevmode%
    \hbox{%
        \begin{beamercolorbox}[wd=.333333\paperwidth,ht=2.5ex,dp=1ex,center]{author in head/foot}%
            \usebeamerfont{author in head/foot}\insertshortauthor
        \end{beamercolorbox}%
        \begin{beamercolorbox}[wd=.333333\paperwidth,ht=2.5ex,dp=1ex,center]{title in head/foot}%
            \usebeamerfont{title in head/foot}\insertshorttitle
        \end{beamercolorbox}%
        \begin{beamercolorbox}[wd=.333333\paperwidth,ht=2.5ex,dp=1ex,right]{date in head/foot}%
            \usebeamerfont{date in head/foot}\insertshortdate{}\hspace*{2em}
            \insertframenumber{} / \inserttotalframenumber\hspace*{2ex}
        \end{beamercolorbox}}%
    \vskip0pt%
}

%=============================================================================
% PACKAGES
%=============================================================================
\usepackage[utf8]{inputenc}
\usepackage[T1]{fontenc}
\usepackage{amsmath, amssymb, amsthm}
\usepackage{mathtools}
\usepackage{bm}
\usepackage{tikz}
\usetikzlibrary{arrows.meta, positioning, shapes, calc}
\usepackage{booktabs}
\usepackage{multirow}
\usepackage{array}
\usepackage{graphicx}
\usepackage{hyperref}
\hypersetup{colorlinks=false, pdfborder={0 0 0}}
\graphicspath{{../../logos/}{../../charts/}}

%=============================================================================
% QUANTLET COMMAND
%=============================================================================
\newcommand{\quantlet}[2]{%
    \hfill\href{#2}{%
        \raisebox{-0.15em}{\includegraphics[height=0.7em]{ql_logo.png}}%
        \textcolor{MainBlue}{\tiny\ #1}%
    }%
}

%=============================================================================
% CENTRED MINIPAGE (no extra vertical space)
%=============================================================================
\newenvironment{cminipage}[1]{%
    \par\noindent\hfill\begin{minipage}{#1}\ignorespaces
}{%
    \end{minipage}\hfill\null\par
}

%=============================================================================
% CUSTOM COMMANDS
%=============================================================================
\newcommand{\E}{\mathbb{E}}
\newcommand{\Var}{\text{Var}}
\newcommand{\Cov}{\text{Cov}}
\newcommand{\Corr}{\text{Corr}}
\newcommand{\R}{\mathbb{R}}
\newcommand{\RMSE}{\text{RMSE}}
\newcommand{\MAE}{\text{MAE}}
\newcommand{\MAPE}{\text{MAPE}}

\newcommand{\correct}{\textcolor{Forest}{\checkmark}}
\newcommand{\incorrect}{\textcolor{Crimson}{\texttimes}}

%=============================================================================
% TITLE INFORMATION
%=============================================================================
\title[Capitolul 0: Seminar]{Capitolul 0: Seminar --- Fundamente}
\subtitle{Program de licență, Facultatea de Cibernetică, Statistică și Informatică Economică, Academia de Studii Economice din București}
\author[Prof. dr. Daniel Traian Pele]{Prof. dr. Daniel Traian Pele\\[0.2cm]\footnotesize\texttt{danpele@ase.ro}}
\institute{Academia de Studii Economice din București}
\date{An Universitar 2025--2026}

\begin{document}

\begin{frame}[plain]
    \begin{tikzpicture}[remember picture, overlay]
        \fill[IDAred] (current page.north west) rectangle ([yshift=-0.15cm]current page.north east);
        \node[anchor=north west] at ([xshift=0.5cm, yshift=-0.3cm]current page.north west) {
            \href{https://www.ase.ro}{\includegraphics[height=1.1cm]{ase_logo.png}}
        };
        \node[anchor=north] at ([yshift=-0.3cm]current page.north) {
            \href{https://ai4efin.ase.ro}{\includegraphics[height=1.1cm]{ai4efin_logo.png}}
        };
        \node[anchor=north east] at ([xshift=-0.5cm, yshift=-0.3cm]current page.north east) {
            \href{https://www.digital-finance-msca.com}{\includegraphics[height=1.1cm]{msca_logo.png}}
        };
    \end{tikzpicture}
    \vfill
    \begin{center}
        {\Large\textcolor{MediumGray}{Analiza și Prognoza Seriilor de Timp}}\\[0.3cm]
        {\Huge\textbf{\textcolor{MainBlue}{Capitolul 0: Fundamente}}}\\[0.5cm]
        {\Large\textcolor{IDAred}{Seminar}}
    \end{center}
    \vfill

    \begin{tikzpicture}[remember picture, overlay]
        \fill[IDAred] (current page.south west) rectangle ([yshift=0.15cm]current page.south east);
        \node[anchor=south west] at ([xshift=0.5cm, yshift=0.8cm]current page.south west) {
            \href{https://theida.net}{\includegraphics[height=0.9cm]{ida_logo.png}}
        };
        \node[anchor=south] at ([xshift=-3cm, yshift=0.8cm]current page.south) {
            \href{https://blockchain-research-center.com}{\includegraphics[height=0.9cm]{brc_logo.png}}
        };
        \node[anchor=south] at ([yshift=0.8cm]current page.south) {
            \href{https://quantinar.com}{\includegraphics[height=0.9cm]{qr_logo.png}}
        };
        \node[anchor=south] at ([xshift=3cm, yshift=0.8cm]current page.south) {
            \href{https://quantlet.com}{\includegraphics[height=0.9cm]{ql_logo.png}}
        };
        \node[anchor=south east] at ([xshift=-0.5cm, yshift=0.8cm]current page.south east) {
            \href{https://ipe.ro/new}{\includegraphics[height=0.9cm]{acad_logo.png}}
        };
    \end{tikzpicture}
\end{frame}

\begin{frame}{Cuprins Seminar}
    \tableofcontents
\end{frame}

%=============================================================================
% TEST GRILĂ
%=============================================================================
\section{Test grilă}

\begin{frame}{Quiz 1: Bazele seriilor de timp}
    \begin{cminipage}{0.95\textwidth}
    \begin{alertblock}{Întrebare}
        Care dintre următoarele NU este o caracteristică a datelor de tip serie de timp?
    \end{alertblock}

    \vspace{0.4cm}

    \begin{block}{Variante de răspuns}
        \textcolor{MainBlue}{\textbf{(A)}} Observațiile sunt ordonate în timp\\[3pt]
        \textcolor{MainBlue}{\textbf{(B)}} Observațiile consecutive sunt de obicei corelate\\[3pt]
        \textcolor{MainBlue}{\textbf{(C)}} Observațiile sunt independente și identic distribuite\\[3pt]
        \textcolor{MainBlue}{\textbf{(D)}} Datele au o ordonare temporală naturală
    \end{block}

    \vspace{0.5cm}

    \begin{center}
        \textit{Răspunsul pe slide-ul următor...}
    \end{center}
    \end{cminipage}
\end{frame}

\begin{frame}{Quiz 1: Răspuns}
    \begin{cminipage}{0.95\textwidth}
    \begin{exampleblock}{Răspuns: C -- Observațiile sunt independente și identic distribuite}
        \textbf{Întrebare:} Care NU este o caracteristică a datelor de tip serie de timp?

        \vspace{0.3cm}

    \begin{block}{Variante de răspuns}
        \textcolor{MainBlue}{\textbf{(A)}} Observațiile sunt ordonate în timp \incorrect\\[3pt]
        \textcolor{MainBlue}{\textbf{(B)}} Observațiile consecutive sunt de obicei corelate \incorrect\\[3pt]
        \textcolor{MainBlue}{\textbf{(C)}} \textbf{\textcolor{Forest}{Observațiile sunt independente și identic distribuite}} \correct\\[3pt]
        \textcolor{MainBlue}{\textbf{(D)}} Datele au o ordonare temporală naturală \incorrect
    \end{block}

        \vspace{0.3cm}

        \begin{itemize}
            \item Observațiile seriilor de timp sunt \textbf{dependente} (autocorelate), nu independente
            \item Ipoteza i.i.d.\ este fundamentală pentru analiza transversală, dar este \textbf{încălcată} în seriile de timp
            \item Această dependență temporală necesită \textbf{metode specializate}
        \end{itemize}
    \end{exampleblock}
    
    \end{cminipage}
    \quantlet{TSA\_ch0\_definition}{https://github.com/QuantLet/TSA/tree/main/TSA_ch0/TSA_ch0_definition}
\end{frame}

\begin{frame}{Quiz 2: Descompunere}
    \begin{cminipage}{0.95\textwidth}
    \begin{alertblock}{Întrebare}
        Când ar trebui să folosiți descompunerea multiplicativă în loc de cea aditivă?
    \end{alertblock}

    \vspace{0.4cm}

    \begin{block}{Variante de răspuns}
        \textcolor{MainBlue}{\textbf{(A)}} Când modelul sezonier are amplitudine constantă\\[3pt]
        \textcolor{MainBlue}{\textbf{(B)}} Când varianța seriei este stabilă în timp\\[3pt]
        \textcolor{MainBlue}{\textbf{(C)}} Când fluctuațiile sezoniere cresc proporțional cu nivelul\\[3pt]
        \textcolor{MainBlue}{\textbf{(D)}} Când seria nu are componentă de trend
    \end{block}

    \vspace{0.5cm}

    \begin{center}
        \textit{Răspunsul pe slide-ul următor...}
    \end{center}
    \end{cminipage}
\end{frame}

\begin{frame}{Quiz 2: Răspuns}
    \begin{cminipage}{0.95\textwidth}
    \begin{exampleblock}{Răspuns: C -- Când fluctuațiile sezoniere cresc proporțional cu nivelul}
        \vspace{-0.2cm}
        \begin{center}
            \includegraphics[width=0.95\textwidth, height=0.52\textheight, keepaspectratio]{additive_vs_multiplicative.png}
        \end{center}
        \vspace{-0.2cm}
        {\footnotesize
        \begin{itemize}\setlength{\itemsep}{0pt}
            \item \textbf{Multiplicativă}: $X_t = T_t \times S_t \times \varepsilon_t$ --- amplitudinea sezonieră \textbf{scalează cu nivelul}
            \item \textbf{Aditivă}: $X_t = T_t + S_t + \varepsilon_t$ --- amplitudine constantă
        \end{itemize}
        }
    \end{exampleblock}
    
    \end{cminipage}
    \quantlet{TSA\_ch0\_decomposition}{https://github.com/QuantLet/TSA/tree/main/TSA_ch0/TSA_ch0_decomposition}
\end{frame}

\begin{frame}{Quiz 3: Netezire exponențială}
    \begin{cminipage}{0.95\textwidth}
    \begin{alertblock}{Întrebare}
        În Netezirea exponențială Simplă cu $\alpha = 0.9$, ce se întâmplă?
    \end{alertblock}

    \vspace{0.4cm}

    \begin{block}{Variante de răspuns}
        \textcolor{MainBlue}{\textbf{(A)}} Prognozele sunt foarte netede și stabile\\[3pt]
        \textcolor{MainBlue}{\textbf{(B)}} Observațiile recente au foarte puțină pondere\\[3pt]
        \textcolor{MainBlue}{\textbf{(C)}} Prognozele reacționează rapid la schimbările recente\\[3pt]
        \textcolor{MainBlue}{\textbf{(D)}} Prognoza este în esență o medie pe termen lung
    \end{block}

    \vspace{0.5cm}

    \begin{center}
        \textit{Răspunsul pe slide-ul următor...}
    \end{center}
    \end{cminipage}
\end{frame}

\begin{frame}{Quiz 3: Răspuns}
    \begin{cminipage}{0.95\textwidth}
    \begin{exampleblock}{Răspuns: C -- Prognozele reacționează rapid la schimbările recente}
        Cu $\alpha = 0.9$: $\hat{X}_{t+1} = 0.9 X_t + 0.1 \hat{X}_t$
        \begin{itemize}
            \item \textbf{$\alpha$ mare} (ex.\ 0.9): 90\% pondere pe ultima observație
                \begin{itemize}
                    \item Prognoze foarte receptive la date noi
                \end{itemize}
            \item \textbf{$\alpha$ mic} (ex.\ 0.1): prognoze mai netede, mai stabile
                \begin{itemize}
                    \item Ia în calcul mai multe observații din trecut
                \end{itemize}
        \end{itemize}
    \end{exampleblock}
    
    \end{cminipage}
    \quantlet{TSA\_ch0\_smoothing}{https://github.com/QuantLet/TSA/tree/main/TSA_ch0/TSA_ch0_smoothing}
\end{frame}

\begin{frame}{Quiz 4: Medii mobile}
    \begin{cminipage}{0.95\textwidth}
    \begin{alertblock}{Întrebare}
        Ce observații folosește o medie mobilă centrată de ordin 5 (MA-5) pentru a estima trendul la momentul $t$?
    \end{alertblock}

    \vspace{0.4cm}

    \begin{block}{Variante de răspuns}
        \textcolor{MainBlue}{\textbf{(A)}} $X_t, X_{t+1}, X_{t+2}, X_{t+3}, X_{t+4}$\\[3pt]
        \textcolor{MainBlue}{\textbf{(B)}} $X_{t-4}, X_{t-3}, X_{t-2}, X_{t-1}, X_t$\\[3pt]
        \textcolor{MainBlue}{\textbf{(C)}} $X_{t-2}, X_{t-1}, X_t, X_{t+1}, X_{t+2}$\\[3pt]
        \textcolor{MainBlue}{\textbf{(D)}} $X_{t-1}, X_t, X_{t+1}$
    \end{block}

    \vspace{0.5cm}

    \begin{center}
        \textit{Răspunsul pe slide-ul următor...}
    \end{center}
    \end{cminipage}
\end{frame}

\begin{frame}{Quiz 4: Răspuns}
    \begin{cminipage}{0.95\textwidth}
    \begin{exampleblock}{Răspuns: C -- $X_{t-2}, X_{t-1}, X_t, X_{t+1}, X_{t+2}$}
        \vspace{-0.2cm}
        \begin{center}
            \includegraphics[width=0.95\textwidth, height=0.52\textheight, keepaspectratio]{ch1_moving_average.png}
        \end{center}
        \vspace{-0.2cm}
        {\footnotesize
        \begin{itemize}\setlength{\itemsep}{0pt}
            \item \textbf{MA centrată}: folosește $(k-1)/2$ observații de fiecare parte a lui $t$
            \item \textbf{MA-5}: 2 înainte + $t$ + 2 după $\Rightarrow$ fereastră mai mare = mai neted
        \end{itemize}
        }
    \end{exampleblock}
    
    \end{cminipage}
    \quantlet{TSA\_ch0\_definition}{https://github.com/QuantLet/TSA/tree/main/TSA_ch0/TSA_ch0_definition}
\end{frame}

\begin{frame}{Quiz 5: Evaluarea prognozei}
    \begin{cminipage}{0.95\textwidth}
    \begin{alertblock}{Întrebare}
        Care metrică este cea mai potrivită pentru compararea acurateței prognozei între serii cu scale diferite?
    \end{alertblock}

    \vspace{0.4cm}

    \begin{block}{Variante de răspuns}
        \textcolor{MainBlue}{\textbf{(A)}} Eroarea Absolută Medie (MAE)\\[3pt]
        \textcolor{MainBlue}{\textbf{(B)}} Rădăcina Erorii Medii Pătratice (RMSE)\\[3pt]
        \textcolor{MainBlue}{\textbf{(C)}} Eroarea Absolută Medie Procentuală (MAPE)\\[3pt]
        \textcolor{MainBlue}{\textbf{(D)}} Eroarea Medie Pătratică (MSE)
    \end{block}

    \vspace{0.5cm}

    \begin{center}
        \textit{Răspunsul pe slide-ul următor...}
    \end{center}
    \end{cminipage}
\end{frame}

\begin{frame}{Quiz 5: Răspuns}
    \begin{cminipage}{0.95\textwidth}
    \begin{exampleblock}{Răspuns: C -- Eroarea Absolută Medie Procentuală (MAPE)}
        MAPE $= \frac{100}{n}\sum\left|\frac{e_t}{X_t}\right|$ exprimă erorile ca \textbf{procente}.

        \begin{itemize}
            \item MAE, RMSE, MSE sunt \textbf{dependente de scală} (unități ale lui $X_t$)
            \item MAPE este \textbf{independentă de scală} (întotdeauna în \%)
            \item Atenție: MAPE devine instabilă când $X_t \approx 0$
        \end{itemize}
    \end{exampleblock}
    
    \end{cminipage}
    \quantlet{TSA\_ch0\_forecast\_eval}{https://github.com/QuantLet/TSA/tree/main/TSA_ch0/TSA_ch0_forecast_eval}
\end{frame}

\begin{frame}{Quiz 6: Validare încrucișată}
    \begin{cminipage}{0.95\textwidth}
    \begin{alertblock}{Întrebare}
        De ce nu putem folosi validarea încrucișată standard k-fold pentru seriile de timp?
    \end{alertblock}

    \vspace{0.4cm}

    \begin{block}{Variante de răspuns}
        \textcolor{MainBlue}{\textbf{(A)}} Datele seriilor de timp sunt prea mici\\[3pt]
        \textcolor{MainBlue}{\textbf{(B)}} Ar încălca ordonarea temporală (viitorul prezicând trecutul)\\[3pt]
        \textcolor{MainBlue}{\textbf{(C)}} Validarea încrucișată este întotdeauna invalidă\\[3pt]
        \textcolor{MainBlue}{\textbf{(D)}} Seriile de timp nu au nevoie de validare
    \end{block}

    \vspace{0.5cm}

    \begin{center}
        \textit{Răspunsul pe slide-ul următor...}
    \end{center}
    \end{cminipage}
\end{frame}

\begin{frame}{Quiz 6: Răspuns}
    \begin{cminipage}{0.95\textwidth}
    \begin{exampleblock}{Răspuns: B -- Ar încălca ordonarea temporală}
        \vspace{-0.2cm}
        \begin{center}
            \includegraphics[width=0.95\textwidth, height=0.52\textheight, keepaspectratio]{ch8_timeseries_cv.png}
        \end{center}
        \vspace{-0.2cm}
        {\footnotesize
        \textbf{Principiu}: datele viitoare nu pot fi folosite pentru a prezice trecutul! Se recomandă CV cu fereastră mobilă/în expansiune.
        }
    \end{exampleblock}
    
    \end{cminipage}
    \quantlet{TSA\_ch0\_forecast\_eval}{https://github.com/QuantLet/TSA/tree/main/TSA_ch0/TSA_ch0_forecast_eval}
\end{frame}

\begin{frame}{Vizual: Descompunerea seriei de timp}
    \begin{cminipage}{0.95\textwidth}
    \vspace{-0.3cm}
    \begin{center}
        \includegraphics[width=0.88\textwidth, height=0.55\textheight, keepaspectratio]{ch1_decomposition.png}
    \end{center}
    \vspace{-0.3cm}
    {\footnotesize
    \begin{exampleblock}{Componentele descompunerii}
        \begin{itemize}\setlength{\itemsep}{0pt}
            \item \textbf{Trend}: mișcare pe termen lung \quad \textbf{Sezonalitate}: tipar periodic \quad \textbf{Reziduu}: zgomot aleatoriu
        \end{itemize}
    \end{exampleblock}
    }
    
    \end{cminipage}
    \quantlet{TSA\_ch0\_decomposition}{https://github.com/QuantLet/TSA/tree/main/TSA_ch0/TSA_ch0_decomposition}
\end{frame}

%=============================================================================
% ADEVĂRAT/FALS
%=============================================================================
\section{Adevărat/Fals}

\begin{frame}{Adevărat sau Fals? --- Întrebări}
    \begin{cminipage}{0.95\textwidth}
    \footnotesize
    \begin{center}
    \begin{tabular}{p{9cm}c}
        \toprule
        \textbf{Afirmație} & \textbf{A/F?} \\
        \midrule
        1. Prognozele SES sunt plate (constante pentru toate orizonturile). & ? \\[0.15cm]
        2. RMSE penalizează erorile mari mai mult decât MAE. & ? \\[0.15cm]
        3. Descompunerea multiplicativă necesită date pozitive. & ? \\[0.15cm]
        4. Un $\alpha$ mai mare înseamnă mai multă netezire. & ? \\[0.15cm]
        5. Setul de test se folosește pentru ajustarea hiperparametrilor. & ? \\[0.15cm]
        6. Naive sezonier folosește valoarea de acum un sezon. & ? \\[0.15cm]
        7. MAPE poate fi infinit dacă valorile reale sunt zero. & ? \\
        \bottomrule
    \end{tabular}
    \end{center}
    \end{cminipage}
\end{frame}

\begin{frame}{Adevărat sau Fals? --- Răspunsuri}
    \begin{cminipage}{0.95\textwidth}
    \scriptsize
    \begin{center}
    \begin{tabular}{p{7.5cm}cc}
        \toprule
        \textbf{Afirmație} & \textbf{A/F} & \textbf{Explicație} \\
        \midrule
        1. Prognozele SES sunt plate (constante pentru toate orizonturile). & \textcolor{Forest}{\textbf{A}} & {\tiny Fără trend} \\[0.08cm]
        2. RMSE penalizează erorile mari mai mult decât MAE. & \textcolor{Forest}{\textbf{A}} & {\tiny Erori pătratice} \\[0.08cm]
        3. Descompunerea multiplicativă necesită date pozitive. & \textcolor{Forest}{\textbf{A}} & {\tiny Nu se poate $\times$ negativ} \\[0.08cm]
        4. Un $\alpha$ mai mare înseamnă mai multă netezire. & \textcolor{Crimson}{\textbf{F}} & {\tiny $\alpha$ mare = mai puțin neted} \\[0.08cm]
        5. Setul de test se folosește pentru ajustarea hiperparametrilor. & \textcolor{Crimson}{\textbf{F}} & {\tiny Folosiți validare!} \\[0.08cm]
        6. Naive sezonier folosește valoarea de acum un sezon. & \textcolor{Forest}{\textbf{A}} & {\tiny $\hat{X}_{t+h} = X_{t+h-m}$} \\[0.08cm]
        7. MAPE poate fi infinit dacă valorile reale sunt zero. & \textcolor{Forest}{\textbf{A}} & {\tiny Împărțire la zero} \\
        \bottomrule
    \end{tabular}
    \end{center}
    \end{cminipage}
\end{frame}

%=============================================================================
% EXERCIȚII DE CALCUL
%=============================================================================
\section{Exerciții de calcul}

\begin{frame}{Exercițiu 1: Netezire Exponențială Simplă}
    \begin{cminipage}{0.95\textwidth}
    \begin{alertblock}{Problemă}
        \begin{itemize}\setlength{\itemsep}{0pt}
            \item \textbf{Date}: $X = [10, 12, 11, 14, 13]$ cu $\alpha = 0.3$, $\hat{X}_1 = 10$
            \item \textbf{Calculați}: a) Prognozele $\hat{X}_2$ până la $\hat{X}_6$; b) MAE și RMSE
            \item \textbf{Formula}: $\hat{X}_{t+1} = \alpha X_t + (1-\alpha)\hat{X}_t$
        \end{itemize}
    \end{alertblock}

    \vspace{0.2cm}
    \begin{exampleblock}{Soluție}
        \begin{center}
        \small
        \begin{tabular}{c|ccccc|c}
            $t$ & 1 & 2 & 3 & 4 & 5 & 6\\
            \hline
            $X_t$ & 10 & 12 & 11 & 14 & 13 & ?\\
            $\hat{X}_t$ & 10 & 10 & 10.6 & 10.72 & 11.70 & \textbf{12.09}\\
        \end{tabular}
        \end{center}
        \begin{itemize}\setlength{\itemsep}{0pt}
            \item \textbf{MAE} $= 1.745$ \quad \textbf{RMSE} $= 2.04$
        \end{itemize}
    \end{exampleblock}
    \end{cminipage}
\end{frame}

\begin{frame}{Exercițiu 2: Metrici de eroare}
    \begin{cminipage}{0.95\textwidth}
    \begin{alertblock}{Problemă}
        \begin{itemize}\setlength{\itemsep}{0pt}
            \item \textbf{Date}: $X = [100, 110, 105, 120]$, $\hat{X} = [95, 108, 110, 115]$
            \item \textbf{Calculați}: MAE, MSE, RMSE, MAPE
        \end{itemize}
    \end{alertblock}

    \vspace{0.2cm}
    \begin{exampleblock}{Soluție}
        \begin{itemize}\setlength{\itemsep}{0pt}
            \item \textbf{Erori}: $e = [5, 2, -5, 5]$
            \item \textbf{MAE} $= (|5|+|2|+|-5|+|5|)/4 = 4.25$
            \item \textbf{MSE} $= (25+4+25+25)/4 = 19.75$
            \item \textbf{RMSE} $= \sqrt{19.75} = 4.44$
            \item \textbf{MAPE} $= 25 \times (0.05+0.018+0.048+0.042) = 3.95\%$
        \end{itemize}
    \end{exampleblock}
    \end{cminipage}
\end{frame}

\begin{frame}{Exercițiu 3: Indici sezonieri}
    \begin{cminipage}{0.95\textwidth}
    \begin{alertblock}{Problemă}
        \begin{itemize}\setlength{\itemsep}{0pt}
            \item \textbf{Date}: Indici sezonieri: $S = [0.85, 1.05, 0.90, 1.20]$, Trend T4: $T = 1000$
            \item \textbf{Calculați}: a) Verificați normalizarea. b) Prognoza T4. c) Desezonalizați $X_{T4} = 1150$
        \end{itemize}
    \end{alertblock}

    \vspace{0.2cm}
    \begin{exampleblock}{Soluție}
        \begin{itemize}\setlength{\itemsep}{0pt}
            \item \textbf{a) Normalizare}: $\sum S_i = 0.85+1.05+0.90+1.20 = 4.00$ \checkmark
            \item \textbf{b) Prognoză}: $\hat{X}_{T4} = 1000 \times 1.20 = \textbf{1200}$
            \item \textbf{c) Desezonalizare}: $X_{desezonalizat} = 1150/1.20 = \textbf{958.33}$ (sub trend)
        \end{itemize}
    \end{exampleblock}
    \end{cminipage}
\end{frame}

%=============================================================================
% EXERCIȚIU CU ASISTENȚĂ AI
%=============================================================================
\section{Exercițiu cu asistență AI}

\begin{frame}{Exercițiu AI: Gândire critică}
    \begin{cminipage}{0.95\textwidth}
    \vspace{-0.3cm}
    \begin{block}{\footnotesize Prompt de testat în ChatGPT / Claude / Copilot}
        {\footnotesize
        ``Folosind yfinance, descarcă prețurile de închidere ajustate pentru SPY din 2015 până în 2024. Aplică descompunerea sezonieră (aditivă și multiplicativă) și compară rezultatele vizual. Apoi împarte datele în set de antrenare (2015--2023) și set de test (2024). Ajustează trei modele de netezire exponențială (SES, Holt, Holt-Winters) pe setul de antrenare și evaluează-le pe setul de test folosind RMSE și MAPE. Care model e cel mai bun? Arată graficele și tabelul comparativ.''
        }
    \end{block}
    \vspace{-2mm}
    {\footnotesize
    \textbf{Exercițiu}:
    \begin{enumerate}\setlength{\itemsep}{0pt}
        \item Rulați prompt-ul într-un LLM la alegere și analizați critic răspunsul.
        \item AI alege descompunere aditivă sau multiplicativă? Alegerea e justificată?
        \item Cum evaluează modelele --- folosește RMSE pe antrenare sau pe test?
        \item Verificați parametrii de netezire ($\alpha$, $\beta$, $\gamma$). Valori aproape de 1{,}0 sunt problematice?
        \item Codul împarte corect datele în antrenare/test, sau evaluează pe date de antrenare?
    \end{enumerate}
    }
    \vspace{-2mm}
    \begin{alertblock}{}
        {\footnotesize \textbf{Atenție}: Codul generat de AI poate rula fără erori și arăta profesional. \textit{Asta nu înseamnă că e corect.}}
    \end{alertblock}
    \end{cminipage}
\end{frame}

%=============================================================================
% SFÂRȘIT
%=============================================================================
\section{Rezumat}

\begin{frame}{Rezumat: Capitolul 0}
    \begin{cminipage}{0.95\textwidth}
    \begin{exampleblock}{Concepte cheie}
        \begin{itemize}\setlength{\itemsep}{2pt}
            \item[\textcolor{MainBlue}{\textbf{1.}}] \textbf{Serii de timp}: observații ordonate temporal, cu dependență (autocorelație)
            \item[\textcolor{MainBlue}{\textbf{2.}}] \textbf{Descompunere}: aditivă ($X_t = T_t + S_t + \varepsilon_t$) vs multiplicativă ($X_t = T_t \times S_t \times \varepsilon_t$)
            \item[\textcolor{MainBlue}{\textbf{3.}}] \textbf{Netezire exponențială}: SES, Holt, Holt-Winters --- parametrul $\alpha$ controlează reactivitatea
            \item[\textcolor{MainBlue}{\textbf{4.}}] \textbf{Evaluarea prognozei}: MAE, RMSE, MAPE --- alegerea depinde de context
            \item[\textcolor{MainBlue}{\textbf{5.}}] \textbf{Sezonalitate}: indici sezonieri, prognoză și desezonalizare
        \end{itemize}
    \end{exampleblock}

    \vspace{0.5cm}
    \begin{center}
        \Large\textcolor{MainBlue}{Întrebări?}
    \end{center}
    \end{cminipage}
\end{frame}


%=============================================================================
% BIBLIOGRAFIE (aceleași referințe ca în curs)
%=============================================================================
\section{Bibliografie}

\begin{frame}{Bibliografie I}
    \begin{cminipage}{0.95\textwidth}
    \begin{block}{Fundamente ale seriilor de timp}
        {\small
        \begin{itemize}\setlength{\itemsep}{0pt}
            \item Hyndman, R.J., \& Athanasopoulos, G. (2021). \textit{Forecasting: Principles and Practice}, 3rd ed., OTexts.
            \item Shumway, R.H., \& Stoffer, D.S. (2017). \textit{Time Series Analysis and Its Applications}, 4th ed., Springer.
            \item Brockwell, P.J., \& Davis, R.A. (2016). \textit{Introduction to Time Series and Forecasting}, 3rd ed., Springer.
        \end{itemize}
        }
    \end{block}

    \begin{exampleblock}{Serii de timp financiare}
        {\small
        \begin{itemize}\setlength{\itemsep}{0pt}
            \item Tsay, R.S. (2010). \textit{Analysis of Financial Time Series}, 3rd ed., Wiley.
            \item Franke, J., Härdle, W.K., \& Hafner, C.M. (2019). \textit{Statistics of Financial Markets}, 4th ed., Springer.
        \end{itemize}
        }
    \end{exampleblock}
    \end{cminipage}
\end{frame}

\begin{frame}{Bibliografie II}
    \begin{cminipage}{0.95\textwidth}
    \begin{block}{Abordări moderne și Machine Learning}
        {\small
        \begin{itemize}\setlength{\itemsep}{0pt}
            \item Nielsen, A. (2019). \textit{Practical Time Series Analysis}, O'Reilly Media.
            \item Petropoulos, F., et al. (2022). \textit{Forecasting: Theory and Practice}, International Journal of Forecasting.
            \item Makridakis, S., Spiliotis, E., \& Assimakopoulos, V. (2020). The M4 Competition, International Journal of Forecasting.
        \end{itemize}
        }
    \end{block}

    \begin{exampleblock}{Resurse online și cod}
        {\small
        \begin{itemize}\setlength{\itemsep}{0pt}
            \item \textbf{Quantlet}: \url{https://quantlet.com} --- Repository de cod pentru statistică
            \item \textbf{Quantinar}: \url{https://quantinar.com} --- Platformă de învățare metode cantitative
            \item \textbf{GitHub TSA}: \url{https://github.com/QuantLet/TSA/tree/main/TSA_ch0} --- Cod Python pentru fiecare capitol
        \end{itemize}
        }
    \end{exampleblock}
    \end{cminipage}
\end{frame}

\begin{frame}{}
    \begin{cminipage}{0.95\textwidth}
    \centering
    \Huge\textcolor{IDAred}{Vă mulțumim!}

    \vspace{1cm}

    \Large\textcolor{MainBlue}{Întrebări?}

    \vspace{0.8cm}

    \normalsize
    Materialele seminarului sunt disponibile la: \url{https://danpele.github.io/Time-Series-Analysis/}

    \vspace{0.2cm}

    \href{https://quantlet.com}{\raisebox{-0.15em}{\includegraphics[height=0.8em]{ql_logo.png}} Quantlet} \hspace{0.5cm}
    \href{https://quantinar.com}{\raisebox{-0.15em}{\includegraphics[height=0.8em]{qr_logo.png}} Quantinar}
    \end{cminipage}
\end{frame}

\end{document}
