% Capitolul 2: Seminar - Modele ARMA
% Teste, Probleme Practice și Discuții
% Program de licență, Academia de Studii Economice din București

\documentclass[9pt, aspectratio=169, t]{beamer}

% Ensure content fits on slides
\setbeamersize{text margin left=8mm, text margin right=8mm}

%=============================================================================
% THEME AND STYLE CONFIGURATION
%=============================================================================
\usetheme{Madrid}
\usecolortheme{seahorse}

% IDA-Inspired Color Palette
\definecolor{MainBlue}{RGB}{26, 58, 110}
\definecolor{AccentBlue}{RGB}{42, 82, 140}
\definecolor{IDAred}{RGB}{220, 53, 69}
\definecolor{DarkGray}{RGB}{51, 51, 51}
\definecolor{MediumGray}{RGB}{128, 128, 128}
\definecolor{LightGray}{RGB}{248, 248, 248}
\definecolor{VeryLightGray}{RGB}{235, 235, 235}
\definecolor{Crimson}{RGB}{220, 53, 69}
\definecolor{Forest}{RGB}{46, 125, 50}
\definecolor{Amber}{RGB}{181, 133, 63}
\definecolor{Orange}{RGB}{230, 126, 34}

\setbeamercolor{palette primary}{bg=MainBlue, fg=white}
\setbeamercolor{palette secondary}{bg=MainBlue!85, fg=white}
\setbeamercolor{palette tertiary}{bg=MainBlue!70, fg=white}
\setbeamercolor{structure}{fg=MainBlue}
\setbeamercolor{title}{fg=MainBlue}
\setbeamercolor{frametitle}{fg=MainBlue, bg=white}
\setbeamercolor{block title}{bg=MainBlue, fg=white}
\setbeamercolor{block body}{bg=VeryLightGray, fg=DarkGray}
\setbeamercolor{block title alerted}{bg=Crimson, fg=white}
\setbeamercolor{block body alerted}{bg=Crimson!8, fg=DarkGray}
\setbeamercolor{block title example}{bg=Forest, fg=white}
\setbeamercolor{block body example}{bg=Forest!8, fg=DarkGray}
\setbeamercolor{item}{fg=MainBlue}

\setbeamertemplate{navigation symbols}{}

\setbeamertemplate{footline}{
    \leavevmode%
    \hbox{%
        \begin{beamercolorbox}[wd=.333333\paperwidth,ht=2.5ex,dp=1ex,center]{author in head/foot}%
            \usebeamerfont{author in head/foot}\insertshortauthor
        \end{beamercolorbox}%
        \begin{beamercolorbox}[wd=.333333\paperwidth,ht=2.5ex,dp=1ex,center]{title in head/foot}%
            \usebeamerfont{title in head/foot}\insertshorttitle
        \end{beamercolorbox}%
        \begin{beamercolorbox}[wd=.333333\paperwidth,ht=2.5ex,dp=1ex,right]{date in head/foot}%
            \usebeamerfont{date in head/foot}\insertshortdate{}\hspace*{2em}
            \insertframenumber{} / \inserttotalframenumber\hspace*{2ex}
        \end{beamercolorbox}}%
    \vskip0pt%
}

%=============================================================================
% PACKAGES
%=============================================================================
\usepackage[utf8]{inputenc}
\usepackage[T1]{fontenc}
\usepackage{amsmath, amssymb, amsthm}
\usepackage{mathtools}
\usepackage{bm}
\usepackage{tikz}
\usetikzlibrary{arrows.meta, positioning, shapes, calc}
\usepackage{booktabs}
\usepackage{multirow}
\usepackage{array}
\usepackage{graphicx}
\usepackage{hyperref}
\hypersetup{colorlinks=false, pdfborder={0 0 0}}
\graphicspath{{../../logos/}{../../charts/}}

%=============================================================================
% CUSTOM COMMANDS
%=============================================================================
\newcommand{\E}{\mathbb{E}}
\newcommand{\Var}{\text{Var}}
\newcommand{\Cov}{\text{Cov}}
\newcommand{\Corr}{\text{Corr}}
\newcommand{\R}{\mathbb{R}}
\newcommand{\RMSE}{\text{RMSE}}
\newcommand{\MAE}{\text{MAE}}
\newcommand{\MAPE}{\text{MAPE}}

% Quiz styling
\newcommand{\correct}{\textcolor{Forest}{\checkmark}}
\newcommand{\incorrect}{\textcolor{Crimson}{\texttimes}}

%=============================================================================
% QUANTLET COMMAND
%=============================================================================
\newcommand{\quantlet}[2]{%
    \hfill\href{#2}{%
        \raisebox{-0.15em}{\includegraphics[height=0.7em]{ql_logo.png}}%
        \textcolor{MainBlue}{\tiny\ #1}%
    }%
}

%=============================================================================
% TITLE INFORMATION
%=============================================================================
\title[Capitolul 2: Seminar]{Capitolul 2: Seminar --- Modele ARMA}
\subtitle{Program de licență, Facultatea de Cibernetică, Statistică și Informatică Economică, Academia de Studii Economice din București}
\author[Prof. dr. Daniel Traian Pele]{Prof. dr. Daniel Traian Pele\\[0.2cm]\footnotesize\texttt{danpele@ase.ro}}
\institute{Academia de Studii Economice din București}
\date{An Universitar 2025--2026}

\begin{document}

%=============================================================================
% TITLE SLIDE
%=============================================================================
\begin{frame}[plain]
    \begin{tikzpicture}[remember picture, overlay]
        \fill[IDAred] (current page.north west) rectangle ([yshift=-0.15cm]current page.north east);
        \node[anchor=north west] at ([xshift=0.5cm, yshift=-0.3cm]current page.north west) {
            \href{https://www.ase.ro}{\includegraphics[height=1.1cm]{ase_logo.png}}
        };
        \node[anchor=north] at ([yshift=-0.3cm]current page.north) {
            \href{https://ai4efin.ase.ro}{\includegraphics[height=1.1cm]{ai4efin_logo.png}}
        };
        \node[anchor=north east] at ([xshift=-0.5cm, yshift=-0.3cm]current page.north east) {
            \href{https://www.digital-finance-msca.com}{\includegraphics[height=1.1cm]{msca_logo.png}}
        };
    \end{tikzpicture}
    \vfill
    \begin{center}
        {\Large\textcolor{MediumGray}{Analiza și Prognoza Seriilor de Timp}}\\[0.3cm]
        {\Huge\textbf{\textcolor{MainBlue}{Capitolul 2: Modele ARMA}}}\\[0.5cm]
        {\Large\textcolor{IDAred}{Seminar}}
    \end{center}
    \vfill

    \begin{tikzpicture}[remember picture, overlay]
        \fill[IDAred] (current page.south west) rectangle ([yshift=0.15cm]current page.south east);
        \node[anchor=south west] at ([xshift=0.5cm, yshift=0.8cm]current page.south west) {
            \href{https://theida.net}{\includegraphics[height=0.9cm]{ida_logo.png}}
        };
        \node[anchor=south] at ([xshift=-3cm, yshift=0.8cm]current page.south) {
            \href{https://blockchain-research-center.com}{\includegraphics[height=0.9cm]{brc_logo.png}}
        };
        \node[anchor=south] at ([yshift=0.8cm]current page.south) {
            \href{https://quantinar.com}{\includegraphics[height=0.9cm]{qr_logo.png}}
        };
        \node[anchor=south] at ([xshift=3cm, yshift=0.8cm]current page.south) {
            \href{https://quantlet.com}{\includegraphics[height=0.9cm]{ql_logo.png}}
        };
        \node[anchor=south east] at ([xshift=-0.5cm, yshift=0.8cm]current page.south east) {
            \href{https://ipe.ro/new}{\includegraphics[height=0.9cm]{acad_logo.png}}
        };
    \end{tikzpicture}
\end{frame}

%=============================================================================
% OUTLINE
%=============================================================================
\begin{frame}{Cuprins Seminar}
    \tableofcontents
\end{frame}

%=============================================================================
% PART 1: MULTIPLE CHOICE QUIZZES
%=============================================================================
\section{Test Grilă}

\begin{frame}{Test 1: Operatorul Lag}
    \begin{alertblock}{Întrebare}
        Care este rezultatul aplicării $(1-L)^2$ lui $X_t$?
    \end{alertblock}

    \vspace{0.5cm}
    \begin{enumerate}[A.]
        \item $X_t - X_{t-1}$
        \item $X_t - 2X_{t-1} + X_{t-2}$
        \item $X_t + X_{t-1} + X_{t-2}$
        \item $X_t - X_{t-2}$
    \end{enumerate}
\end{frame}

\begin{frame}{Test 1: Răspuns}
    \begin{columns}[T]
        \begin{column}{0.52\textwidth}
            \begin{exampleblock}{Răspuns: B}
                $X_t - 2X_{t-1} + X_{t-2}$
            \end{exampleblock}

            \vspace{0.2cm}

            \textbf{Explicație:}
            \begin{align*}
                (1-L)^2 X_t &= (1 - 2L + L^2)X_t \\
                &= X_t - 2X_{t-1} + X_{t-2}
            \end{align*}

            Aceasta este \textbf{diferența de ordinul doi} a lui $X_t$.
        \end{column}
        \begin{column}{0.45\textwidth}
            \includegraphics[width=\textwidth]{ch2_lag_operator.pdf}
            {\tiny Operatorul lag: $L^k X_t = X_{t-k}$}
        \end{column}
    \end{columns}
    \quantlet{TSA\_ch2\_lag\_operator}{https://github.com/QuantLet/TSA/tree/main/TSA_ch2_lag_operator}
\end{frame}

\begin{frame}{Test 2: Staționaritatea AR(1)}
    \begin{alertblock}{Întrebare}
        Pentru ce valoare a lui $\phi$ procesul AR(1) $X_t = 0.5 + \phi X_{t-1} + \varepsilon_t$ este staționar?
    \end{alertblock}

    \vspace{0.5cm}
    \begin{enumerate}[A.]
        \item $\phi = 1.2$
        \item $\phi = 1.0$
        \item $\phi = -0.8$
        \item $\phi = -1.5$
    \end{enumerate}
\end{frame}

\begin{frame}{Test 2: Răspuns}
    \begin{columns}[T]
        \begin{column}{0.52\textwidth}
            \begin{exampleblock}{Răspuns: C}
                $\phi = -0.8$ (Staționar)
            \end{exampleblock}

            \vspace{0.2cm}

            \textbf{Condiția de staționaritate AR(1):}
            $$|\phi| < 1$$

            \vspace{0.2cm}

            Verificarea fiecărei opțiuni:
            \begin{itemize}
                \item A: $|1.2| = 1.2 > 1$ \incorrect
                \item B: $|1.0| = 1$ (rădăcină unitară) \incorrect
                \item C: $|-0.8| = 0.8 < 1$ \correct
                \item D: $|-1.5| = 1.5 > 1$ \incorrect
            \end{itemize}
        \end{column}
        \begin{column}{0.45\textwidth}
            \includegraphics[width=\textwidth]{ch2_ar1.pdf}
            {\tiny AR(1): regiunea staționară $|\phi| < 1$}
        \end{column}
    \end{columns}
    \quantlet{TSA\_ch2\_ar1}{https://github.com/QuantLet/TSA/tree/main/TSA_ch2_ar1}
\end{frame}

\begin{frame}{Test 3: Modelul ACF}
    \begin{alertblock}{Întrebare}
        Observați următorul model ACF: vârf semnificativ la lag 1, apoi toate lag-urile în benzile de încredere. PACF arată descreștere graduală. Ce model este sugerat?
    \end{alertblock}

    \vspace{0.5cm}
    \begin{enumerate}[A.]
        \item AR(1)
        \item MA(1)
        \item ARMA(1,1)
        \item Zgomot alb
    \end{enumerate}
\end{frame}

\begin{frame}{Test 3: Răspuns}
    \begin{columns}[T]
        \begin{column}{0.52\textwidth}
            \begin{exampleblock}{Răspuns: B}
                MA(1)
            \end{exampleblock}

            \vspace{0.2cm}

            \textbf{Regula cheie de identificare:}
            \begin{itemize}
                \item ACF se anulează după lag $q$ $\Rightarrow$ MA($q$)
                \item PACF se anulează după lag $p$ $\Rightarrow$ AR($p$)
            \end{itemize}

            \vspace{0.2cm}

            Aici: ACF se anulează la lag 1, PACF descrește

            $\Rightarrow$ \textbf{MA(1)}
        \end{column}
        \begin{column}{0.45\textwidth}
            \includegraphics[width=\textwidth]{ch2_ma1.pdf}
            {\tiny MA(1): ACF se anulează după lag 1}
        \end{column}
    \end{columns}
    \quantlet{TSA\_ch2\_ma1}{https://github.com/QuantLet/TSA/tree/main/TSA_ch2_ma1}
\end{frame}

\begin{frame}{Test 4: Invertibilitatea MA}
    \begin{alertblock}{Întrebare}
        Pentru procesul MA(1) $X_t = \varepsilon_t + 1.5\varepsilon_{t-1}$, este procesul invertibil?
    \end{alertblock}

    \vspace{0.5cm}
    \begin{enumerate}[A.]
        \item Da, deoarece procesele MA sunt întotdeauna invertibile
        \item Da, deoarece $1.5 > 0$
        \item Nu, deoarece $|\theta| = 1.5 > 1$
        \item Nu, deoarece procesele MA nu sunt niciodată invertibile
    \end{enumerate}
\end{frame}

\begin{frame}{Test 4: Răspuns}
    \begin{columns}[T]
        \begin{column}{0.52\textwidth}
            \begin{exampleblock}{Răspuns: C}
                Nu este invertibil ($|\theta| = 1.5 > 1$)
            \end{exampleblock}

            \vspace{0.2cm}

            \textbf{Invertibilitatea MA(1):}

            Necesită $|\theta| < 1$

            \vspace{0.2cm}

            Echivalent: rădăcina lui $\theta(z) = 1 + \theta z = 0$ trebuie să fie în afara cercului unitate.

            Aici: $z = -1/1.5 = -0.67$ este \textbf{în interior}!

            \vspace{0.2cm}

            $\Rightarrow$ \textcolor{Crimson}{\textbf{Nu este invertibil}}
        \end{column}
        \begin{column}{0.45\textwidth}
            \includegraphics[width=\textwidth]{ch2_ma1.pdf}
            {\tiny Invertibilitate: rădăcina în afara cercului unitate}
        \end{column}
    \end{columns}
    \quantlet{TSA\_ch2\_ma1}{https://github.com/QuantLet/TSA/tree/main/TSA_ch2_ma1}
\end{frame}

\begin{frame}{Test 5: Reprezentarea ARMA}
    \begin{alertblock}{Întrebare}
        Forma compactă $\phi(L)X_t = \theta(L)\varepsilon_t$ reprezintă ce model?
    \end{alertblock}

    \vspace{0.5cm}
    \begin{enumerate}[A.]
        \item Model AR pur
        \item Model MA pur
        \item Model ARMA
        \item Niciunul dintre cele de mai sus
    \end{enumerate}
\end{frame}

\begin{frame}{Test 5: Răspuns}
    \begin{columns}[T]
        \begin{column}{0.52\textwidth}
            \begin{exampleblock}{Răspuns: C}
                Model ARMA
            \end{exampleblock}

            \vspace{0.2cm}

            \textbf{Notația cu polinoame lag:}
            \begin{itemize}
                \item $\phi(L) = 1 - \phi_1 L - \cdots - \phi_p L^p$
                \item $\theta(L) = 1 + \theta_1 L + \cdots + \theta_q L^q$
            \end{itemize}

            \vspace{0.2cm}

            Cazuri speciale:
            \begin{itemize}
                \item $\theta(L) = 1$: AR pur
                \item $\phi(L) = 1$: MA pur
            \end{itemize}
        \end{column}
        \begin{column}{0.45\textwidth}
            \includegraphics[width=\textwidth]{ch2_arma.pdf}
            {\tiny ARMA(1,1): combină AR și MA}
        \end{column}
    \end{columns}
    \quantlet{TSA\_ch2\_arma}{https://github.com/QuantLet/TSA/tree/main/TSA_ch2_arma}
\end{frame}

\begin{frame}{Test 6: Criterii Informaționale}
    \begin{alertblock}{Întrebare}
        Când comparăm ARMA(1,1) și ARMA(2,1) folosind BIC, care afirmație este corectă?
    \end{alertblock}

    \vspace{0.5cm}
    \begin{enumerate}[A.]
        \item BIC mai mic înseamnă întotdeauna prognoze mai bune
        \item BIC penalizează complexitatea mai puțin decât AIC
        \item Modelul cu BIC mai mic este preferat
        \item BIC poate compara doar modele cu același număr de parametri
    \end{enumerate}
\end{frame}

\begin{frame}{Test 6: Răspuns}
    \begin{columns}[T]
        \begin{column}{0.52\textwidth}
            \begin{exampleblock}{Răspuns: C}
                Modelul cu BIC mai mic este preferat
            \end{exampleblock}

            \vspace{0.2cm}

            \textbf{Criterii Informaționale:}
            \begin{align*}
                \text{AIC} &= -2\ln(\hat{L}) + 2k \\
                \text{BIC} &= -2\ln(\hat{L}) + k\ln(n)
            \end{align*}

            BIC penalizează complexitatea \textbf{mai mult} decât AIC (pentru $n > 7$).

            \vspace{0.2cm}

            $\Rightarrow$ BIC favorizează modele mai simple.
        \end{column}
        \begin{column}{0.45\textwidth}
            \includegraphics[width=\textwidth]{ch2_model_selection.pdf}
            {\tiny Selecția modelului: AIC vs BIC}
        \end{column}
    \end{columns}
    \quantlet{TSA\_ch2\_model\_selection}{https://github.com/QuantLet/TSA/tree/main/TSA_ch2_model_selection}
\end{frame}

\begin{frame}{Test 7: Testul Ljung-Box}
    \begin{alertblock}{Întrebare}
        După ajustarea unui model ARMA(2,1), rulați testul Ljung-Box pe reziduuri și obțineți valoare-p = 0.02. Ce concluzie trageți?
    \end{alertblock}

    \vspace{0.5cm}
    \begin{enumerate}[A.]
        \item Modelul este adecvat
        \item Reziduurile sunt zgomot alb
        \item Există autocorelație semnificativă în reziduuri
        \item Modelul are prea mulți parametri
    \end{enumerate}
\end{frame}

\begin{frame}{Test 7: Răspuns}
    \begin{columns}[T]
        \begin{column}{0.52\textwidth}
            \begin{exampleblock}{Răspuns: C}
                Autocorelație semnificativă în reziduuri
            \end{exampleblock}

            \vspace{0.2cm}

            \textbf{Testul Ljung-Box:}
            \begin{itemize}
                \item $H_0$: Reziduurile sunt zgomot alb
                \item $H_1$: Autocorelație prezentă
            \end{itemize}

            \vspace{0.2cm}

            valoare-p = 0.02 $<$ 0.05

            $\Rightarrow$ \textcolor{Crimson}{\textbf{Respingem $H_0$}}

            Modelul este \textbf{inadecvat} --- încercați alte ordine.
        \end{column}
        \begin{column}{0.45\textwidth}
            \includegraphics[width=\textwidth]{ch2_diagnostics.pdf}
            {\tiny Diagnostice: ACF trebuie să fie zgomot alb}
        \end{column}
    \end{columns}
    \quantlet{TSA\_ch2\_diagnostics}{https://github.com/QuantLet/TSA/tree/main/TSA_ch2_diagnostics}
\end{frame}

\begin{frame}{Test 8: Prognoză}
    \begin{alertblock}{Întrebare}
        Pentru un model AR(1) cu $\phi = 0.6$ și medie $\mu = 10$, ce se întâmplă cu prognozele când orizontul $h \to \infty$?
    \end{alertblock}

    \vspace{0.5cm}
    \begin{enumerate}[A.]
        \item Prognozele cresc fără limită
        \item Prognozele converg la 0
        \item Prognozele converg la $\mu = 10$
        \item Prognozele oscilează pentru totdeauna
    \end{enumerate}
\end{frame}

\begin{frame}{Test 8: Răspuns}
    \begin{columns}[T]
        \begin{column}{0.52\textwidth}
            \begin{exampleblock}{Răspuns: C}
                Prognozele converg la $\mu = 10$
            \end{exampleblock}

            \vspace{0.2cm}

            \textbf{Formula de prognoză AR(1):}
            $$\hat{X}_{n+h|n} = \mu + \phi^h(X_n - \mu)$$

            Deoarece $|\phi| = 0.6 < 1$:
            $$\lim_{h \to \infty} \phi^h = 0$$

            $\Rightarrow$ Prognozele converg la $\mu$.

            \textbf{Revenire la medie!}
        \end{column}
        \begin{column}{0.45\textwidth}
            \includegraphics[width=\textwidth]{ch2_forecasting.pdf}
            {\tiny Prognoze AR(1): revenire la medie}
        \end{column}
    \end{columns}
    \quantlet{TSA\_ch2\_forecasting}{https://github.com/QuantLet/TSA/tree/main/TSA_ch2_forecasting}
\end{frame}

\begin{frame}{Test 9: Rădăcinile AR(2)}
    \begin{alertblock}{Întrebare}
        Un proces AR(2) are rădăcinile caracteristice $z_1 = 0.8$ și $z_2 = -0.5$. Este staționar?
    \end{alertblock}

    \vspace{0.5cm}
    \begin{enumerate}[A.]
        \item Da, deoarece ambele rădăcini sunt în interiorul cercului unitate
        \item Nu, deoarece o rădăcină este negativă
        \item Nu, deoarece rădăcinile trebuie să fie în afara cercului unitate
        \item Nu se poate determina fără mai multe informații
    \end{enumerate}
\end{frame}

\begin{frame}{Test 9: Răspuns}
    \begin{columns}[T]
        \begin{column}{0.52\textwidth}
            \begin{exampleblock}{Răspuns: C}
                Rădăcinile trebuie să fie în afara cercului unitate
            \end{exampleblock}

            \vspace{0.2cm}

            \textbf{Condiția de staționaritate:}

            Rădăcinile lui $\phi(z) = 0$ trebuie să fie \textbf{în afara} cercului unitate ($|z| > 1$).

            \vspace{0.2cm}

            Aici:
            \begin{itemize}
                \item $|z_1| = 0.8 < 1$ \incorrect
                \item $|z_2| = 0.5 < 1$ \incorrect
            \end{itemize}

            Ambele în interior $\Rightarrow$ \textcolor{Crimson}{\textbf{Nestaționar}}
        \end{column}
        \begin{column}{0.45\textwidth}
            \includegraphics[width=\textwidth]{ch2_ar2.pdf}
            {\tiny AR(2): rădăcini și triunghiul de staționaritate}
        \end{column}
    \end{columns}
    \quantlet{TSA\_ch2\_ar2}{https://github.com/QuantLet/TSA/tree/main/TSA_ch2_ar2}
\end{frame}

\begin{frame}{Test 10: Proprietățile MA(q)}
    \begin{alertblock}{Întrebare}
        Pentru un proces MA(2), ACF-ul:
    \end{alertblock}

    \vspace{0.5cm}
    \begin{enumerate}[A.]
        \item Descrește exponențial
        \item Se anulează după lag 2
        \item Se anulează după lag 1
        \item Nu se anulează niciodată
    \end{enumerate}
\end{frame}

\begin{frame}{Test 10: Răspuns}
    \begin{columns}[T]
        \begin{column}{0.52\textwidth}
            \begin{exampleblock}{Răspuns: B}
                Se anulează după lag 2
            \end{exampleblock}

            \vspace{0.2cm}

            \textbf{Proprietatea ACF pentru MA($q$):}

            $$\rho(h) = 0 \text{ pentru } h > q$$

            \vspace{0.2cm}

            \begin{itemize}
                \item MA(1): ACF se anulează după lag 1
                \item MA(2): ACF se anulează după lag 2
                \item MA($q$): ACF se anulează după lag $q$
            \end{itemize}

            Caracteristica cheie de identificare!
        \end{column}
        \begin{column}{0.45\textwidth}
            \includegraphics[width=\textwidth]{ch2_ma1.pdf}
            {\tiny MA: anularea ACF este semnătura}
        \end{column}
    \end{columns}
    \quantlet{TSA\_ch2\_ma1}{https://github.com/QuantLet/TSA/tree/main/TSA_ch2_ma1}
\end{frame}

%=============================================================================
% PART 2: TRUE/FALSE QUESTIONS
%=============================================================================
\section{Întrebări Adevărat/Fals}

\begin{frame}{Adevărat sau Fals? --- Întrebări}
    \footnotesize
    \begin{center}
    \begin{tabular}{p{9cm}c}
        \toprule
        \textbf{Afirmație} & \textbf{A/F?} \\
        \midrule
        1. Un proces AR(2) poate prezenta comportament pseudo-ciclic. & ? \\[0.15cm]
        2. Procesele MA necesită o condiție de staționaritate. & ? \\[0.15cm]
        3. PACF-ul unui proces AR(p) se anulează după lag $p$. & ? \\[0.15cm]
        4. Dacă AIC selectează ARMA(2,1) și BIC selectează ARMA(1,1), nu pot fi ambele corecte. & ? \\[0.15cm]
        5. Intervalele de încredere se îngustează pe măsură ce orizontul crește. & ? \\[0.15cm]
        6. Ecuațiile Yule-Walker pot fi folosite pentru a estima parametrii MA. & ? \\
        \bottomrule
    \end{tabular}
    \end{center}
\end{frame}

\begin{frame}{Adevărat sau Fals? --- Răspunsuri}
    \begin{columns}[T]
        \begin{column}{0.55\textwidth}
            \small
            \begin{enumerate}
                \item \textcolor{Forest}{\textbf{ADEVĂRAT}}: AR(2) cu rădăcini complexe $\Rightarrow$ oscilații amortizate
                \vspace{0.1cm}
                \item \textcolor{Crimson}{\textbf{FALS}}: Procesele MA sunt întotdeauna staționare; au nevoie de \textit{invertibilitate}
                \vspace{0.1cm}
                \item \textcolor{Forest}{\textbf{ADEVĂRAT}}: Caracteristica cheie de identificare a AR($p$)
                \vspace{0.1cm}
                \item \textcolor{Crimson}{\textbf{FALS}}: Ambele sunt „corecte" pentru criteriile lor (AIC: estimare, BIC: parcimonie)
                \vspace{0.1cm}
                \item \textcolor{Crimson}{\textbf{FALS}}: IC se \textit{lărgesc} cu orizontul (mai multă incertitudine)
                \vspace{0.1cm}
                \item \textcolor{Crimson}{\textbf{FALS}}: Yule-Walker este pentru AR; MA folosește MLE
            \end{enumerate}
        \end{column}
        \begin{column}{0.42\textwidth}
            \includegraphics[width=\textwidth]{ch2_ar2.pdf}
            {\tiny AR(2): rădăcini complexe $\Rightarrow$ cicluri}
        \end{column}
    \end{columns}
    \quantlet{TSA\_ch2\_ar2}{https://github.com/QuantLet/TSA/tree/main/TSA_ch2_ar2}
\end{frame}

%=============================================================================
% PART 3: CALCULATION EXERCISES
%=============================================================================
\section{Exerciții de Calcul}

\begin{frame}{Exercițiul 1: Proprietățile AR(1)}
    \textbf{Problemă:} Considerați procesul AR(1):
    $$X_t = 2 + 0.7 X_{t-1} + \varepsilon_t, \quad \varepsilon_t \sim WN(0, 9)$$

    Calculați:
    \begin{enumerate}
        \item Media $\mu$
        \item Varianța $\gamma(0)$
        \item Autocovarianța $\gamma(1)$ și $\gamma(2)$
        \item Autocorelația $\rho(1)$ și $\rho(2)$
    \end{enumerate}
\end{frame}

\begin{frame}{Exercițiul 1: Soluție}
    \begin{columns}[T]
        \begin{column}{0.52\textwidth}
            Dat: $c = 2$, $\phi = 0.7$, $\sigma^2 = 9$

            \vspace{0.2cm}
            \textbf{1. Media:}
            $$\mu = \frac{c}{1-\phi} = \frac{2}{1-0.7} = \frac{2}{0.3} = \textbf{6.67}$$

            \textbf{2. Varianța:}
            $$\gamma(0) = \frac{\sigma^2}{1-\phi^2} = \frac{9}{1-0.49} = \textbf{17.65}$$

            \textbf{3. Autocovarianța:}
            \begin{align*}
                \gamma(1) &= \phi \cdot \gamma(0) = 0.7 \times 17.65 = \textbf{12.35}\\
                \gamma(2) &= \phi^2 \cdot \gamma(0) = 0.49 \times 17.65 = \textbf{8.65}
            \end{align*}

            \textbf{4. Autocorelația:}
            $$\rho(1) = \phi = \textbf{0.7}, \quad \rho(2) = \phi^2 = \textbf{0.49}$$
        \end{column}
        \begin{column}{0.45\textwidth}
            \includegraphics[width=\textwidth]{ch2_seminar_ex1_ar1.png}
            {\tiny Simulare AR(1) și ACF}
        \end{column}
    \end{columns}
    \quantlet{TSA\_ch2\_ex1\_ar1}{https://github.com/QuantLet/TSA/tree/main/TSA_ch2_ex1_ar1}
\end{frame}

\begin{frame}{Exercițiul 2: Proprietățile MA(1)}
    \textbf{Problemă:} Considerați procesul MA(1):
    $$X_t = 5 + \varepsilon_t - 0.4\varepsilon_{t-1}, \quad \varepsilon_t \sim WN(0, 4)$$

    Calculați:
    \begin{enumerate}
        \item Media $\mu$
        \item Varianța $\gamma(0)$
        \item Autocovarianța $\gamma(1)$
        \item Autocorelația $\rho(1)$
        \item Este acest proces invertibil?
    \end{enumerate}
\end{frame}

\begin{frame}{Exercițiul 2: Soluție}
    \begin{columns}[T]
        \begin{column}{0.52\textwidth}
            Dat: $\mu = 5$, $\theta = -0.4$, $\sigma^2 = 4$

            \vspace{0.2cm}
            \textbf{1. Media:}
            $$\E[X_t] = \mu = \textbf{5}$$

            \textbf{2. Varianța:}
            $$\gamma(0) = \sigma^2(1 + \theta^2) = 4(1.16) = \textbf{4.64}$$

            \textbf{3. Autocovarianța:}
            $$\gamma(1) = \theta\sigma^2 = -0.4 \times 4 = \textbf{-1.6}$$

            \textbf{4. Autocorelația:}
            $$\rho(1) = \frac{\gamma(1)}{\gamma(0)} = \frac{-1.6}{4.64} = \textbf{-0.345}$$

            \textbf{5. Invertibilitate:} $|\theta| = 0.4 < 1$ $\Rightarrow$ \textcolor{Forest}{\textbf{Da}}
        \end{column}
        \begin{column}{0.45\textwidth}
            \includegraphics[width=\textwidth]{ch2_seminar_ex2_ma1.png}
            {\tiny MA(1): ACF se anulează după lag 1}
        \end{column}
    \end{columns}
    \quantlet{TSA\_ch2\_ex2\_ma1}{https://github.com/QuantLet/TSA/tree/main/TSA_ch2_ex2_ma1}
\end{frame}

\begin{frame}{Exercițiul 3: Rădăcinile Caracteristice}
    \textbf{Problemă:} Considerați procesul AR(2):
    $$X_t = 0.5X_{t-1} + 0.3X_{t-2} + \varepsilon_t$$

    \begin{enumerate}
        \item Scrieți ecuația caracteristică
        \item Găsiți rădăcinile caracteristice
        \item Este acest proces staționar?
    \end{enumerate}
\end{frame}

\begin{frame}{Exercițiul 3: Soluție}
    \begin{columns}[T]
        \begin{column}{0.52\textwidth}
            \textbf{1. Ecuația caracteristică:}
            $$\phi(z) = 1 - 0.5z - 0.3z^2 = 0$$
            Sau: $0.3z^2 + 0.5z - 1 = 0$

            \vspace{0.2cm}
            \textbf{2. Rădăcinile (formula cuadratică):}
            $$z = \frac{-0.5 \pm \sqrt{0.25 + 1.2}}{0.6}$$
            $$z_1 = \textbf{1.17}, \quad z_2 = \textbf{-2.84}$$

            \vspace{0.2cm}
            \textbf{3. Verificarea staționarității:}

            $|z_1| = 1.17 > 1$ \correct

            $|z_2| = 2.84 > 1$ \correct

            Ambele în afara cercului unitate $\Rightarrow$ \textcolor{Forest}{\textbf{Staționar}}
        \end{column}
        \begin{column}{0.45\textwidth}
            \includegraphics[width=\textwidth]{ch2_seminar_ex3_roots.png}
            {\tiny Rădăcini în afara cercului $\Rightarrow$ staționar}
        \end{column}
    \end{columns}
    \quantlet{TSA\_ch2\_ex3\_roots}{https://github.com/QuantLet/TSA/tree/main/TSA_ch2_ex3_roots}
\end{frame}

\begin{frame}{Exercițiul 4: Prognoză}
    \textbf{Problemă:} Ați ajustat un model AR(1):
    $$X_t = 3 + 0.8X_{t-1} + \varepsilon_t, \quad \sigma^2 = 4$$

    Dat $X_{100} = 20$, calculați:
    \begin{enumerate}
        \item Prognoza la 1 pas înainte $\hat{X}_{101|100}$
        \item Prognoza la 2 pași înainte $\hat{X}_{102|100}$
        \item Prognoza pe termen lung când $h \to \infty$
        \item Intervalul de încredere de 95\% pentru $\hat{X}_{101|100}$
    \end{enumerate}
\end{frame}

\begin{frame}{Exercițiul 4: Soluție}
    \begin{columns}[T]
        \begin{column}{0.52\textwidth}
            Dat: $c = 3$, $\phi = 0.8$, $\sigma^2 = 4$, $X_{100} = 20$

            \textbf{Media:} $\mu = \frac{3}{1-0.8} = \textbf{15}$

            \vspace{0.2cm}
            \textbf{1. Prognoza la un pas:}
            $$\hat{X}_{101|100} = 3 + 0.8 \times 20 = \textbf{19}$$

            \textbf{2. Prognoza la doi pași:}
            $$\hat{X}_{102|100} = 3 + 0.8 \times 19 = \textbf{18.2}$$

            \textbf{3. Prognoza pe termen lung:}
            $$\lim_{h \to \infty} \hat{X}_{100+h|100} = \mu = \textbf{15}$$

            \textbf{4. IC 95\%:}
            $$19 \pm 1.96 \times 2 = \textbf{[15.08, 22.92]}$$
        \end{column}
        \begin{column}{0.45\textwidth}
            \includegraphics[width=\textwidth]{ch2_seminar_ex4_forecast.png}
            {\tiny Prognoza converge la medie}
        \end{column}
    \end{columns}
    \quantlet{TSA\_ch2\_ex4\_forecast}{https://github.com/QuantLet/TSA/tree/main/TSA_ch2_ex4_forecast}
\end{frame}

%=============================================================================
% PART 4: PYTHON EXERCISES
%=============================================================================
\section{Exerciții Python}

\begin{frame}[fragile]{Exercițiu Python 1: Simulare și Ajustare AR(1)}
    \textbf{Sarcină:}
    \begin{enumerate}
        \item Simulați 300 de observații dintr-un AR(1) cu $\phi = 0.6$
        \item Reprezentați grafic seria și ACF/PACF
        \item Ajustați un model AR(1) și comparați $\hat{\phi}$ vs $\phi$ real
        \item Examinați diagnosticele reziduurilor
    \end{enumerate}

    \vspace{0.3cm}
    \textbf{Cod cheie:}
    \begin{verbatim}
from statsmodels.tsa.arima.model import ARIMA
model = ARIMA(x, order=(1, 0, 0)).fit()
print(model.summary())
    \end{verbatim}

    \quantlet{TSA\_ch2\_python\_simulate}{https://github.com/QuantLet/TSA/tree/main/TSA_ch2_python_simulate}
\end{frame}

\begin{frame}{Exercițiu Python 2: Selecția Modelului}
    \textbf{Sarcină:}
    \begin{enumerate}
        \item Încărcați o serie de timp și verificați staționaritatea (testul ADF)
        \item Comparați AIC/BIC pentru AR(1), MA(1), ARMA(1,1), ARMA(2,1)
        \item Selectați cel mai bun model
        \item Generați prognoze cu intervale de încredere
    \end{enumerate}

    \vspace{0.3cm}
    \textbf{Funcții cheie:}
    \begin{itemize}
        \item \texttt{adfuller(x)} pentru testul de staționaritate
        \item \texttt{model.aic}, \texttt{model.bic} pentru criterii
        \item \texttt{model.get\_forecast(h)} pentru predicții
    \end{itemize}

    \quantlet{TSA\_ch2\_python\_selection}{https://github.com/QuantLet/TSA/tree/main/TSA_ch2_python_selection}
\end{frame}

\begin{frame}{Exercițiu Python 3: Verificarea Diagnosticelor}
    \textbf{Sarcină:} După ajustarea unui model, efectuați diagnostice complete:
    \begin{enumerate}
        \item Reprezentați grafic reziduurile în timp
        \item Reprezentați grafic ACF-ul reziduurilor
        \item Creați graficul Q-Q pentru normalitate
        \item Rulați testul Ljung-Box
    \end{enumerate}

    \vspace{0.3cm}
    \textbf{Funcții cheie:}
    \begin{itemize}
        \item \texttt{model.resid} pentru reziduuri
        \item \texttt{plot\_acf(resid)} pentru graficul ACF
        \item \texttt{stats.probplot(resid)} pentru graficul Q-Q
        \item \texttt{acorr\_ljungbox(resid, lags=[10])} pentru test
    \end{itemize}

    \quantlet{TSA\_ch2\_python\_diagnostics}{https://github.com/QuantLet/TSA/tree/main/TSA_ch2_python_diagnostics}
\end{frame}

%=============================================================================
% PART 5: DISCUSSION QUESTIONS
%=============================================================================
\section{Întrebări de Discuție}

\begin{frame}{Discuție 1: Selecția Modelului}
    \textbf{Scenariu:} Modelați rate de inflație lunare. După verificarea staționarității:
    \begin{itemize}
        \item ACF: semnificativ la lag-urile 1, 2, 3, apoi descrește
        \item PACF: semnificativ la lag-urile 1, 2, apoi se anulează
        \item AIC selectează ARMA(2,3)
        \item BIC selectează AR(2)
    \end{itemize}

    \vspace{0.3cm}
    \textbf{Întrebări:}
    \begin{enumerate}
        \item Ce sugerează modelul ACF/PACF?
        \item De ce nu sunt de acord AIC și BIC?
        \item Ce model ați alege și de ce?
        \item Ce verificări suplimentare ați efectua?
    \end{enumerate}
\end{frame}

\begin{frame}{Discuție 2: Evaluarea Prognozei}
    \textbf{Scenariu:} Ajustați un model ARMA(1,1) pe randamente zilnice de acțiuni. Ajustarea în eșantion arată bine (Ljung-Box p = 0.45), dar RMSE în afara eșantionului este mai rău decât mersul aleatoriu.

    \vspace{0.3cm}
    \textbf{Întrebări:}
    \begin{enumerate}
        \item Este aceasta surprinzător? De ce sau de ce nu?
        \item Ce ne spune despre predictibilitatea randamentelor?
        \item Ar trebui să concluzionați că modelul ARMA este inutil?
        \item Ce alternative ați putea considera?
    \end{enumerate}

    \vspace{0.3cm}
    \textbf{Indiciu:} Gândiți-vă la Ipoteza Pieței Eficiente și la ce captează ARMA vs. gruparea volatilității.
\end{frame}

%=============================================================================
% EXERCIȚIU CU ASISTENȚĂ AI
%=============================================================================
\section{Exercițiu cu asistență AI}

\begin{frame}{Exercițiu AI: Gândire critică}
    \vspace{-3mm}
    \begin{block}{\footnotesize Prompt de testat în ChatGPT / Claude / Copilot}
        {\footnotesize
        ``Descarcă de pe FRED numărul lunar de cereri inițiale de șomaj din SUA (seria ICNSA, neajustat sezonier) din 2010-01 până în 2024-12 (180 observații). Calculează diferențele logaritmice pentru a obține rata de creștere. Testează staționaritatea cu ADF și KPSS. Estimează modele ARMA cu ordine $(p,q)$ între $(1,0)$ și $(3,3)$, selectează cel mai bun model după AIC/BIC. Verifică reziduurile (Ljung-Box, normalitate) și prognozează 6 luni. Vreau cod Python complet cu grafice.''
        }
    \end{block}
    \vspace{-2mm}
    {\footnotesize
    \textbf{Exercițiu}:
    \begin{enumerate}\setlength{\itemsep}{0pt}
        \item Rulați prompt-ul într-un LLM la alegere și analizați critic răspunsul.
        \item Verifică staționaritatea \textit{înainte} de a estima ARMA? Folosește ambele teste (ADF + KPSS)?
        \item Cum gestionează vârful COVID-19 din date? Îl tratează ca outlier sau îl ignoră?
        \item Selecția ordinelor $(p,q)$ e justificată prin ACF/PACF sau doar prin AIC?
        \item Reziduurile sunt testate complet? (Ljung-Box, Q-Q plot, heteroscedasticitate)
    \end{enumerate}
    }
    \vspace{-2mm}
    \begin{alertblock}{}
        {\footnotesize \textbf{Atenție}: Codul generat de AI poate rula fără erori și arăta profesional. \textit{Asta nu înseamnă că e corect.}}
    \end{alertblock}
\end{frame}

%=============================================================================
% KEY FORMULAS SUMMARY
%=============================================================================
\section{Formule Cheie}

\begin{frame}{Rezumat Formule Cheie}
    \small
    \begin{center}
    \begin{tabular}{ll}
        \toprule
        \textbf{Concept} & \textbf{Formula} \\
        \midrule
        Media AR(1) & $\mu = c/(1-\phi)$ \\[0.1cm]
        Varianța AR(1) & $\gamma(0) = \sigma^2/(1-\phi^2)$ \\[0.1cm]
        ACF AR(1) & $\rho(h) = \phi^h$ \\[0.1cm]
        Staționaritate AR(1) & $|\phi| < 1$ \\[0.1cm]
        \midrule
        Varianța MA(1) & $\gamma(0) = \sigma^2(1+\theta^2)$ \\[0.1cm]
        ACF MA(1) & $\rho(1) = \theta/(1+\theta^2)$, $\rho(h) = 0$ pentru $h > 1$ \\[0.1cm]
        Invertibilitate MA(1) & $|\theta| < 1$ \\[0.1cm]
        \midrule
        Prognoza AR(1) & $\hat{X}_{n+h|n} = \mu + \phi^h(X_n - \mu)$ \\[0.1cm]
        IC Prognoză & $\hat{X} \pm z_{\alpha/2} \times \sqrt{\text{MSFE}(h)}$ \\[0.1cm]
        \midrule
        AIC & $-2\ln(\hat{L}) + 2k$ \\[0.1cm]
        BIC & $-2\ln(\hat{L}) + k\ln(n)$ \\
        \bottomrule
    \end{tabular}
    \end{center}
\end{frame}

%=============================================================================
% END
%=============================================================================
\begin{frame}{}
    \centering
    \vspace{2cm}
    {\Huge\textcolor{MainBlue}{Întrebări?}}

    \vspace{1cm}

    \normalsize
    Succes la exerciții!

    \vspace{0.5cm}

    \textbf{Următorul Seminar:} ARIMA și Modele Sezoniere
\end{frame}

\end{document}
