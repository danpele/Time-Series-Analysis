% Capitolul 7: Seminar - Cointegrare și VECM
% Teste, Probleme Practice și Discuții
% Program de licență, Academia de Studii Economice din București

\documentclass[9pt, aspectratio=169, t]{beamer}

% Ensure content fits on slides
\setbeamersize{text margin left=8mm, text margin right=8mm}

%=============================================================================
% THEME AND STYLE CONFIGURATION
%=============================================================================
\usetheme{Madrid}
\usecolortheme{seahorse}

% IDA-Inspired Color Palette
\definecolor{MainBlue}{RGB}{26, 58, 110}
\definecolor{AccentBlue}{RGB}{42, 82, 140}
\definecolor{IDAred}{RGB}{220, 53, 69}
\definecolor{DarkGray}{RGB}{51, 51, 51}
\definecolor{MediumGray}{RGB}{128, 128, 128}
\definecolor{LightGray}{RGB}{248, 248, 248}
\definecolor{VeryLightGray}{RGB}{235, 235, 235}
\definecolor{Crimson}{RGB}{220, 53, 69}
\definecolor{Forest}{RGB}{46, 125, 50}
\definecolor{Amber}{RGB}{181, 133, 63}
\definecolor{Orange}{RGB}{230, 126, 34}

\setbeamercolor{palette primary}{bg=MainBlue, fg=white}
\setbeamercolor{palette secondary}{bg=MainBlue!85, fg=white}
\setbeamercolor{palette tertiary}{bg=MainBlue!70, fg=white}
\setbeamercolor{structure}{fg=MainBlue}
\setbeamercolor{title}{fg=MainBlue}
\setbeamercolor{frametitle}{fg=MainBlue, bg=white}
\setbeamercolor{block title}{bg=MainBlue, fg=white}
\setbeamercolor{block body}{bg=VeryLightGray, fg=DarkGray}
\setbeamercolor{block title alerted}{bg=Crimson, fg=white}
\setbeamercolor{block body alerted}{bg=Crimson!8, fg=DarkGray}
\setbeamercolor{block title example}{bg=Forest, fg=white}
\setbeamercolor{block body example}{bg=Forest!8, fg=DarkGray}
\setbeamercolor{item}{fg=MainBlue}

\setbeamertemplate{navigation symbols}{}

\setbeamertemplate{footline}{
    \leavevmode%
    \hbox{%
        \begin{beamercolorbox}[wd=.333333\paperwidth,ht=2.5ex,dp=1ex,center]{author in head/foot}%
            \usebeamerfont{author in head/foot}\insertshortauthor
        \end{beamercolorbox}%
        \begin{beamercolorbox}[wd=.333333\paperwidth,ht=2.5ex,dp=1ex,center]{title in head/foot}%
            \usebeamerfont{title in head/foot}\insertshorttitle
        \end{beamercolorbox}%
        \begin{beamercolorbox}[wd=.333333\paperwidth,ht=2.5ex,dp=1ex,right]{date in head/foot}%
            \usebeamerfont{date in head/foot}\insertshortdate{}\hspace*{2em}
            \insertframenumber{} / \inserttotalframenumber\hspace*{2ex}
        \end{beamercolorbox}}%
    \vskip0pt%
}

%=============================================================================
% PACKAGES
%=============================================================================
\usepackage[utf8]{inputenc}
\usepackage[T1]{fontenc}
\usepackage{amsmath, amssymb, amsthm}
\usepackage{mathtools}
\usepackage{bm}
\usepackage{tikz}
\usetikzlibrary{arrows.meta, positioning, shapes, calc}
\usepackage{booktabs}
\usepackage{multirow}
\usepackage{array}
\usepackage{graphicx}
\usepackage{hyperref}
\hypersetup{colorlinks=false, pdfborder={0 0 0}}
\graphicspath{{../../logos/}{../../charts/}}

%=============================================================================
% CUSTOM COMMANDS
%=============================================================================
\newcommand{\E}{\mathbb{E}}
\newcommand{\Var}{\text{Var}}
\newcommand{\Cov}{\text{Cov}}
\newcommand{\bY}{\mathbf{Y}}
\newcommand{\bA}{\mathbf{A}}
\newcommand{\bepsilon}{\boldsymbol{\varepsilon}}
\newcommand{\bvarepsilon}{\boldsymbol{\varepsilon}}
\newcommand{\bc}{\mathbf{c}}
\newcommand{\bPi}{\boldsymbol{\Pi}}
\newcommand{\balpha}{\boldsymbol{\alpha}}
\newcommand{\bbeta}{\boldsymbol{\beta}}
\newcommand{\bGamma}{\boldsymbol{\Gamma}}

%=============================================================================
% TITLE INFORMATION
%=============================================================================
\title[Capitolul 7: Seminar]{Capitolul 7: Seminar --- Cointegrare \& VECM}
\subtitle{Program de licență, Facultatea de Cibernetică, Statistică și Informatică Economică, Academia de Studii Economice din București}
\author[Prof. dr. Daniel Traian Pele]{Prof. dr. Daniel Traian Pele\\[0.2cm]\footnotesize\texttt{danpele@ase.ro}}
\institute{Academia de Studii Economice din București}
\date{An Universitar 2025--2026}

\begin{document}

%=============================================================================
% TITLE SLIDE
%=============================================================================
\begin{frame}[plain]
    \begin{tikzpicture}[remember picture, overlay]
        \fill[IDAred] (current page.north west) rectangle ([yshift=-0.15cm]current page.north east);
        \node[anchor=north west] at ([xshift=0.5cm, yshift=-0.3cm]current page.north west) {
            \href{https://www.ase.ro}{\includegraphics[height=1.1cm]{ase_logo.png}}
        };
        \node[anchor=north] at ([yshift=-0.3cm]current page.north) {
            \href{https://ai4efin.ase.ro}{\includegraphics[height=1.1cm]{ai4efin_logo.png}}
        };
        \node[anchor=north east] at ([xshift=-0.5cm, yshift=-0.3cm]current page.north east) {
            \href{https://www.digital-finance-msca.com}{\includegraphics[height=1.1cm]{msca_logo.png}}
        };
    \end{tikzpicture}
    \vfill
    \begin{center}
        {\Large\textcolor{MediumGray}{Analiza și Prognoza Seriilor de Timp}}\\[0.3cm]
        {\Huge\textbf{\textcolor{MainBlue}{Capitolul 7: Cointegrare \& VECM}}}\\[0.5cm]
        {\Large\textcolor{IDAred}{Seminar}}
    \end{center}
    \vfill

    \begin{tikzpicture}[remember picture, overlay]
        \fill[IDAred] (current page.south west) rectangle ([yshift=0.15cm]current page.south east);
        \node[anchor=south west] at ([xshift=0.5cm, yshift=0.8cm]current page.south west) {
            \href{https://theida.net}{\includegraphics[height=0.9cm]{ida_logo.png}}
        };
        \node[anchor=south] at ([xshift=-3cm, yshift=0.8cm]current page.south) {
            \href{https://blockchain-research-center.com}{\includegraphics[height=0.9cm]{brc_logo.png}}
        };
        \node[anchor=south] at ([yshift=0.8cm]current page.south) {
            \href{https://quantinar.com}{\includegraphics[height=0.9cm]{qr_logo.png}}
        };
        \node[anchor=south] at ([xshift=3cm, yshift=0.8cm]current page.south) {
            \href{https://quantlet.com}{\includegraphics[height=0.9cm]{ql_logo.png}}
        };
        \node[anchor=south east] at ([xshift=-0.5cm, yshift=0.8cm]current page.south east) {
            \href{https://ipe.ro/new}{\includegraphics[height=0.9cm]{acad_logo.png}}
        };
    \end{tikzpicture}
\end{frame}

%=============================================================================
% OUTLINE
%=============================================================================
\begin{frame}{Cuprins Seminar}
    \tableofcontents
\end{frame}

%=============================================================================
% SECTION 1: REVIEW QUIZ
%=============================================================================
\section{Test de Recapitulare}

\begin{frame}{Test 1: Definiția Cointegrării}
    \begin{alertblock}{Întrebare}
        Două variabile I(1), $X_t$ și $Y_t$, sunt cointegrate dacă:
    \end{alertblock}

    \vspace{0.3cm}

    \begin{enumerate}[A)]
        \item Ambele sunt staționare
        \item Suma lor este I(2)
        \item O combinație liniară a lor este I(0)
        \item Au aceeași medie
    \end{enumerate}

    \vspace{0.5cm}
    \begin{flushright}\textit{Răspunsul pe slide-ul următor...}\end{flushright}
\end{frame}

\begin{frame}{Test 1: Răspuns}
    \begin{exampleblock}{Răspuns: C -- O combinație liniară este I(0)}
        \begin{center}
            \includegraphics[width=0.85\textwidth, height=0.60\textheight, keepaspectratio]{ch6_quiz1_cointegration.pdf}
        \end{center}
        \vspace{-0.2cm}
        {\footnotesize
        \textbf{Cheie}: $Y_t - \beta X_t \sim I(0)$ înseamnă că seriile au un trend stochastic comun. Combinația liniară (spread-ul) este staționară chiar dacă ambele serii sunt nestaționare.
        }
    \end{exampleblock}
\end{frame}

\begin{frame}{Test 2: Regresia Falsă}
    \begin{alertblock}{Întrebare}
        Când regresăm un mers aleator pe alt mers aleator independent, de obicei obținem:
    \end{alertblock}

    \vspace{0.3cm}

    \begin{enumerate}[A)]
        \item $R^2$ mic și coeficienți nesemnificativi
        \item $R^2$ mare și coeficienți semnificativi (fals!)
        \item Coeficienți zero
        \item Rezultate nedefinite
    \end{enumerate}

    \vspace{0.5cm}
    \begin{flushright}\textit{Răspunsul pe slide-ul următor...}\end{flushright}
\end{frame}

\begin{frame}{Test 2: Răspuns}
    \begin{exampleblock}{Răspuns: B -- $R^2$ mare și coeficienți semnificativi (fals!)}
        \begin{center}
            \includegraphics[width=0.85\textwidth, height=0.60\textheight, keepaspectratio]{ch6_quiz2_spurious.pdf}
        \end{center}
        \vspace{-0.2cm}
        {\footnotesize
        \textbf{Granger-Newbold (1974)}: Regresarea seriilor I(1) nerelaționate dă rezultate înșelătoare. Regulă: Dacă $R^2 > DW$, suspectați regresie falsă!
        }
    \end{exampleblock}
\end{frame}

\begin{frame}{Test 3: Testul Engle-Granger}
    \begin{alertblock}{Întrebare}
        În metoda Engle-Granger în doi pași, ce testăm în pasul 2?
    \end{alertblock}

    \vspace{0.3cm}

    \begin{enumerate}[A)]
        \item Dacă variabilele originale sunt staționare
        \item Dacă reziduurile regresiei au rădăcină unitară
        \item Dacă coeficienții sunt semnificativi
        \item Dacă $R^2$ este suficient de mare
    \end{enumerate}

    \vspace{0.5cm}
    \begin{flushright}\textit{Răspunsul pe slide-ul următor...}\end{flushright}
\end{frame}

\begin{frame}{Test 3: Răspuns}
    \begin{exampleblock}{Răspuns: B -- Dacă reziduurile au rădăcină unitară}
        \textbf{Pasul 1}: Rulăm OLS: $Y_t = \alpha + \beta X_t + e_t$, salvăm reziduurile $\hat{e}_t$

        \vspace{0.2cm}
        \textbf{Pasul 2}: Test ADF pe reziduuri: $\Delta \hat{e}_t = \rho \hat{e}_{t-1} + \ldots$
        \begin{itemize}
            \item $H_0$: $\rho = 0$ (rădăcină unitară $\Rightarrow$ fără cointegrare)
            \item $H_1$: $\rho < 0$ (staționar $\Rightarrow$ cointegrare!)
        \end{itemize}

        \vspace{0.2cm}
        \textbf{Important}: Folosiți valorile critice Engle-Granger, nu ADF standard!
    \end{exampleblock}
\end{frame}

\begin{frame}{Test 4: Avantajul Testului Johansen}
    \begin{alertblock}{Întrebare}
        Principalul avantaj al testului Johansen față de Engle-Granger este:
    \end{alertblock}

    \vspace{0.3cm}

    \begin{enumerate}[A)]
        \item Este mai simplu de calculat
        \item Poate detecta relații de cointegrare multiple
        \item Nu necesită date
        \item Găsește întotdeauna cointegrare
    \end{enumerate}

    \vspace{0.5cm}
    \begin{flushright}\textit{Răspunsul pe slide-ul următor...}\end{flushright}
\end{frame}

\begin{frame}{Test 4: Răspuns}
    \begin{exampleblock}{Răspuns: B -- Poate detecta relații de cointegrare multiple}
        \begin{center}
            \includegraphics[width=0.85\textwidth, height=0.58\textheight, keepaspectratio]{ch6_quiz4_johansen.pdf}
        \end{center}
        \vspace{-0.2cm}
        {\footnotesize
        \textbf{Avantaje Johansen}:
        \begin{itemize}
            \item Testează pentru $r = 0, 1, 2, \ldots, k-1$ vectori de cointegrare
            \item Verosimilitate maximă (mai eficient)
            \item Nu necesită alegerea variabilei dependente
        \end{itemize}
        }
    \end{exampleblock}
\end{frame}

\begin{frame}{Test 5: Rangul Matricei $\bPi$}
    \begin{alertblock}{Întrebare}
        Într-un VECM cu $k=3$ variabile, dacă $\text{rang}(\bPi) = 2$, aceasta înseamnă:
    \end{alertblock}

    \vspace{0.3cm}

    \begin{enumerate}[A)]
        \item Fără cointegrare
        \item O relație de cointegrare
        \item Două relații de cointegrare
        \item Toate variabilele sunt staționare
    \end{enumerate}

    \vspace{0.5cm}
    \begin{flushright}\textit{Răspunsul pe slide-ul următor...}\end{flushright}
\end{frame}

\begin{frame}{Test 5: Răspuns}
    \begin{exampleblock}{Răspuns: C -- Două relații de cointegrare}
        \textbf{Interpretarea rangului} pentru $k$ variabile:
        \begin{itemize}
            \item $\text{rang}(\bPi) = 0$: Fără cointegrare (folosiți VAR în diferențe)
            \item $0 < \text{rang}(\bPi) = r < k$: $r$ vectori de cointegrare (folosiți VECM)
            \item $\text{rang}(\bPi) = k$: Toate variabilele sunt I(0) (folosiți VAR în niveluri)
        \end{itemize}

        \vspace{0.2cm}
        \textbf{Cu $k=3$ și $r=2$}:
        \begin{itemize}
            \item Două relații de echilibru
            \item Doar $k - r = 1$ trend stochastic comun
        \end{itemize}
    \end{exampleblock}
\end{frame}

\begin{frame}{Test 6: Structura VECM}
    \begin{alertblock}{Întrebare}
        În ecuația VECM $\Delta \bY_t = \bc + \balpha\bbeta'\bY_{t-1} + \ldots$, ce reprezintă $\balpha$?
    \end{alertblock}

    \vspace{0.3cm}

    \begin{enumerate}[A)]
        \item Vectorii de cointegrare
        \item Coeficienții de ajustare (încărcare)
        \item Dinamica pe termen scurt
        \item Varianța erorilor
    \end{enumerate}

    \vspace{0.5cm}
    \begin{flushright}\textit{Răspunsul pe slide-ul următor...}\end{flushright}
\end{frame}

\begin{frame}{Test 6: Răspuns}
    \begin{exampleblock}{Răspuns: B -- Coeficienții de ajustare (încărcare)}
        \begin{center}
            \includegraphics[width=0.85\textwidth, height=0.58\textheight, keepaspectratio]{ch6_quiz6_adjustment.pdf}
        \end{center}
        \vspace{-0.2cm}
        {\footnotesize
        \textbf{$\bPi = \balpha\bbeta'$}:
        \begin{itemize}
            \item $\bbeta$ = vectorii de cointegrare (definesc echilibrul)
            \item $\balpha$ = vitezele de ajustare (cât de repede corectează fiecare variabilă)
        \end{itemize}
        }
    \end{exampleblock}
\end{frame}

\begin{frame}{Test 7: Termenul de Corecție a Erorilor}
    \begin{alertblock}{Întrebare}
        Dacă $Y_t - \beta X_t$ este relația de cointegrare și acest termen este pozitiv, ce se întâmplă?
    \end{alertblock}

    \vspace{0.3cm}

    \begin{enumerate}[A)]
        \item $Y$ este deasupra echilibrului; $Y$ ar trebui să scadă (dacă $\alpha < 0$)
        \item $Y$ este sub echilibru; $Y$ ar trebui să crească
        \item Nimic, corecția erorilor nu afectează nivelurile
        \item Ambele variabile cresc
    \end{enumerate}

    \vspace{0.5cm}
    \begin{flushright}\textit{Răspunsul pe slide-ul următor...}\end{flushright}
\end{frame}

\begin{frame}{Test 7: Răspuns}
    \begin{exampleblock}{Răspuns: A -- $Y$ deasupra echilibrului; scade dacă $\alpha < 0$}
        \textbf{Mecanismul de corecție a erorilor}:
        $$\Delta Y_t = \alpha(Y_{t-1} - \beta X_{t-1}) + \ldots$$

        \begin{itemize}
            \item Dacă $Y_{t-1} - \beta X_{t-1} > 0$: $Y$ este ``prea sus''
            \item Cu $\alpha < 0$: $\Delta Y_t < 0$ (Y scade spre echilibru)
            \item Aceasta este ``corecția erorilor'' care trage $Y$ înapoi
        \end{itemize}

        \vspace{0.2cm}
        \textbf{Convenție de semn}: $\alpha$ ar trebui să fie negativ pentru ca variabila dependentă să se miște înapoi spre echilibru.
    \end{exampleblock}
\end{frame}

\begin{frame}{Test 8: Exogenitate Slabă}
    \begin{alertblock}{Întrebare}
        Dacă $\alpha_2 = 0$ într-un VECM bivariat, aceasta înseamnă:
    \end{alertblock}

    \vspace{0.3cm}

    \begin{enumerate}[A)]
        \item Nu există cointegrare
        \item Variabila 2 nu se ajustează la dezechilibru (slab exogenă)
        \item Variabila 1 nu se ajustează
        \item Ambele variabile sunt staționare
    \end{enumerate}

    \vspace{0.5cm}
    \begin{flushright}\textit{Răspunsul pe slide-ul următor...}\end{flushright}
\end{frame}

\begin{frame}{Test 8: Răspuns}
    \begin{exampleblock}{Răspuns: B -- Variabila 2 este slab exogenă}
        \textbf{Exogenitate slabă}: Variabila nu răspunde la dezechilibru.

        \vspace{0.2cm}
        \textbf{Exemplu: Ratele dobânzii}
        \begin{itemize}
            \item Rata pe termen lung ($R_t$) adesea slab exogenă ($\alpha_R \approx 0$)
            \item Rata pe termen scurt ($r_t$) se ajustează la spread ($\alpha_r < 0$)
            \item Interpretare: Banca centrală ajustează rata scurtă pentru a menține structura termenelor
        \end{itemize}

        \vspace{0.2cm}
        \textbf{Implicație}: Putem estima o singură ecuație pentru variabila care se ajustează.
    \end{exampleblock}
\end{frame}

\begin{frame}{Test 9: Testul Trace}
    \begin{alertblock}{Întrebare}
        Testul trace Johansen cu $H_0: r \leq 1$ vs $H_1: r > 1$ testează dacă:
    \end{alertblock}

    \vspace{0.3cm}

    \begin{enumerate}[A)]
        \item Există exact un vector de cointegrare
        \item Există cel mult un vector de cointegrare
        \item Există mai mult de un vector de cointegrare
        \item Toate valorile proprii sunt zero
    \end{enumerate}

    \vspace{0.5cm}
    \begin{flushright}\textit{Răspunsul pe slide-ul următor...}\end{flushright}
\end{frame}

\begin{frame}{Test 9: Răspuns}
    \begin{exampleblock}{Răspuns: B/C -- $H_0$: cel mult 1; $H_1$: mai mult de 1}
        \textbf{Procedura de testare secvențială}:
        \begin{enumerate}
            \item Testăm $H_0: r = 0$ vs $H_1: r > 0$
            \item Dacă respingem, testăm $H_0: r \leq 1$ vs $H_1: r > 1$
            \item Continuăm până nu mai respingem...
        \end{enumerate}

        \vspace{0.2cm}
        \textbf{Statistica trace}:
        $$\lambda_{\text{trace}}(r) = -T \sum_{i=r+1}^{k} \ln(1 - \hat{\lambda}_i)$$

        Respingem $H_0$ dacă statistica trace $>$ valoarea critică.
    \end{exampleblock}
\end{frame}

\begin{frame}{Test 10: VECM vs VAR în Diferențe}
    \begin{alertblock}{Întrebare}
        Dacă variabilele sunt cointegrate, folosirea VAR în diferențe prime în loc de VECM:
    \end{alertblock}

    \vspace{0.3cm}

    \begin{enumerate}[A)]
        \item Dă rezultate identice
        \item Este mai eficientă
        \item Pierde informația pe termen lung (model greșit specificat)
        \item Este abordarea preferată
    \end{enumerate}

    \vspace{0.5cm}
    \begin{flushright}\textit{Răspunsul pe slide-ul următor...}\end{flushright}
\end{frame}

\begin{frame}{Test 10: Răspuns}
    \begin{exampleblock}{Răspuns: C -- Pierde informația pe termen lung}
        \textbf{Teorema Reprezentării Granger}: Dacă există cointegrare, reprezentarea VECM există și trebuie folosită.

        \vspace{0.2cm}
        \begin{center}
        \begin{tabular}{lcc}
            \toprule
            & \textbf{VAR($\Delta$)} & \textbf{VECM} \\
            \midrule
            Echilibru pe termen lung & Pierdut & Păstrat \\
            Corecția erorilor & Nu & Da \\
            Prognoze (termen lung) & Slabe & Mai bune \\
            \bottomrule
        \end{tabular}
        \end{center}

        \vspace{0.2cm}
        \textbf{Concluzie}: Diferențierea elimină relația pe termen lung pe care o reprezintă cointegrarea!
    \end{exampleblock}
\end{frame}

%=============================================================================
% TRUE/FALSE QUESTIONS
%=============================================================================
\section{Întrebări Adevărat/Fals}

\begin{frame}{Întrebări Adevărat/Fals}
    Determinați dacă fiecare afirmație este Adevărată sau Falsă:

    \vspace{0.3cm}
    \begin{enumerate}
        \item Cointegrarea necesită ca toate variabilele să fie I(1).
        \item Vectorul de cointegrare este unic.
        \item Regresia falsă are statistică Durbin-Watson mică.
        \item În VECM, ambii coeficienți $\alpha$ trebuie să fie nenuli.
        \item Testul Johansen necesită alegerea unei variabile dependente.
        \item Numărul de trenduri comune = $k - r$.
    \end{enumerate}

    \vspace{0.3cm}
    \begin{flushright}\textit{Răspunsurile pe slide-ul următor...}\end{flushright}
\end{frame}

\begin{frame}{Adevărat/Fals: Soluții}
    {\small
    \begin{enumerate}\setlength{\itemsep}{1pt}
        \item Cointegrarea necesită ca toate variabilele să fie I(1). \hfill \textcolor{Forest}{\textbf{ADEVĂRAT}}

        {\footnotesize \textcolor{MediumGray}{Cazul standard CI(1,1): toate variabilele I(1), combinația liniară I(0).}}

        \item Vectorul de cointegrare este unic. \hfill \textcolor{Crimson}{\textbf{FALS}}

        {\footnotesize \textcolor{MediumGray}{Unic doar până la înmulțirea cu un scalar. De obicei normalizat ($\beta_1 = 1$).}}

        \item Regresia falsă are statistică Durbin-Watson mică. \hfill \textcolor{Forest}{\textbf{ADEVĂRAT}}

        {\footnotesize \textcolor{MediumGray}{$DW \approx 0$ indică reziduuri puternic autocorelate (nestaționare).}}

        \item În VECM, ambii coeficienți $\alpha$ trebuie să fie nenuli. \hfill \textcolor{Crimson}{\textbf{FALS}}

        {\footnotesize \textcolor{MediumGray}{Unul poate fi zero (exogenitate slabă). Cel puțin unul trebuie să fie nenul.}}

        \item Testul Johansen necesită alegerea unei variabile dependente. \hfill \textcolor{Crimson}{\textbf{FALS}}

        {\footnotesize \textcolor{MediumGray}{Asta e pentru Engle-Granger. Johansen tratează toate variabilele simetric.}}

        \item Numărul de trenduri comune = $k - r$. \hfill \textcolor{Forest}{\textbf{ADEVĂRAT}}

        {\footnotesize \textcolor{MediumGray}{$k$ variabile, $r$ relații de cointegrare $\Rightarrow$ $k-r$ trenduri stochastice comune.}}
    \end{enumerate}
    }
\end{frame}

%=============================================================================
% SECTION 2: PRACTICE PROBLEMS
%=============================================================================
\section{Probleme Practice}

\begin{frame}{Problema 1: Identificarea Cointegrării}
    \begin{block}{Exercițiu}
        Aveți date trimestriale pentru consum ($C_t$) și venit ($Y_t$). Testele ADF arată că ambele sunt I(1). Regresia $C_t = 0.85 Y_t + e_t$ dă reziduuri cu statistica ADF $= -3.92$. Valoarea critică Engle-Granger la 5\% pentru 2 variabile este $-3.34$.

        \vspace{0.2cm}
        Sunt $C_t$ și $Y_t$ cointegrate?
    \end{block}

    \vspace{0.5cm}
    \begin{flushright}\textit{Răspunsul pe slide-ul următor...}\end{flushright}
\end{frame}

\begin{frame}{Problema 1: Soluție}
    \begin{exampleblock}{Soluție: Da, sunt cointegrate}
        \textbf{Test}: $H_0$: Fără cointegrare (reziduurile au rădăcină unitară)

        \vspace{0.2cm}
        \textbf{Statistica ADF}: $-3.92$

        \textbf{Valoarea critică (5\%)}: $-3.34$

        \vspace{0.2cm}
        Deoarece $-3.92 < -3.34$, \textbf{respingem} $H_0$ la nivelul de 5\%.

        \vspace{0.2cm}
        \textbf{Concluzie}: Reziduurile sunt staționare $\Rightarrow$ Există cointegrare!

        \vspace{0.2cm}
        \textbf{Interpretare}: Consumul și venitul au un trend comun. Vectorul de cointegrare este aproximativ $(1, -0.85)$, consistent cu ipoteză venitului permanent.
    \end{exampleblock}
\end{frame}

\begin{frame}{Problema 2: Interpretarea VECM}
    \begin{block}{Exercițiu}
        Un VECM pentru rata pe termen scurt ($r_t$) și rata pe termen lung ($R_t$) dă:
        \begin{align*}
            \Delta r_t &= 0.01 - 0.25(r_{t-1} - R_{t-1}) + \ldots \\
            \Delta R_t &= 0.005 - 0.02(r_{t-1} - R_{t-1}) + \ldots
        \end{align*}
        Interpretați coeficienții de ajustare.
    \end{block}

    \vspace{0.5cm}
    \begin{flushright}\textit{Răspunsul pe slide-ul următor...}\end{flushright}
\end{frame}

\begin{frame}{Problema 2: Soluție}
    \begin{exampleblock}{Soluție}
        \textbf{Termenul de corecție a erorilor}: $(r_{t-1} - R_{t-1})$ = spread-ul

        \vspace{0.2cm}
        \textbf{Rata pe termen scurt} ($\alpha_r = -0.25$):
        \begin{itemize}
            \item Când spread-ul este pozitiv (scurt $>$ lung), rata scurtă scade
            \item 25\% din dezechilibru este corectat per perioadă
            \item Rata scurtă se ajustează activ
        \end{itemize}

        \vspace{0.2cm}
        \textbf{Rata pe termen lung} ($\alpha_R = -0.02$):
        \begin{itemize}
            \item Coeficient de ajustare foarte mic
            \item Rata lungă este aproape slab exogenă
            \item Condusă mai mult de așteptări, nu de corecția erorilor
        \end{itemize}

        \vspace{0.2cm}
        \textbf{Interpretare economică}: Banca centrală (rata scurtă) se ajustează pentru a menține curba randamentelor.
    \end{exampleblock}
\end{frame}

\begin{frame}{Problema 3: Rezultatele Testului Johansen}
    \begin{block}{Exercițiu}
        Testul trace Johansen pentru 3 variabile dă:
        \begin{center}
        \begin{tabular}{lcc}
            \toprule
            $H_0$ & Stat. Trace & VC 5\% \\
            \midrule
            $r = 0$ & 45.2 & 29.8 \\
            $r \leq 1$ & 18.1 & 15.5 \\
            $r \leq 2$ & 3.2 & 3.8 \\
            \bottomrule
        \end{tabular}
        \end{center}
        Care este rangul de cointegrare?
    \end{block}

    \vspace{0.5cm}
    \begin{flushright}\textit{Răspunsul pe slide-ul următor...}\end{flushright}
\end{frame}

\begin{frame}{Problema 3: Soluție}
    \begin{exampleblock}{Soluție: Rangul = 2}
        \textbf{Testare secvențială}:
        \begin{enumerate}
            \item $H_0: r = 0$: $45.2 > 29.8$ $\Rightarrow$ \textbf{Respingem} (cel puțin 1)
            \item $H_0: r \leq 1$: $18.1 > 15.5$ $\Rightarrow$ \textbf{Respingem} (cel puțin 2)
            \item $H_0: r \leq 2$: $3.2 < 3.8$ $\Rightarrow$ \textbf{Nu respingem}
        \end{enumerate}

        \vspace{0.2cm}
        \textbf{Concluzie}: $r = 2$ relații de cointegrare

        \vspace{0.2cm}
        \textbf{Implicații}:
        \begin{itemize}
            \item Două relații de echilibru între 3 variabile
            \item Doar $3 - 2 = 1$ trend stochastic comun
            \item Folosiți VECM cu 2 termeni de corecție a erorilor
        \end{itemize}
    \end{exampleblock}
\end{frame}

%=============================================================================
% SECTION 3: WORKED EXAMPLES
%=============================================================================
\section{Exemple Rezolvate}

\begin{frame}{Exemplu: Structura la Termen a Ratelor Dobânzii}
    {\small
    \begin{block}{Teoria Economică}
        Ipoteza așteptărilor: $R_t^{(n)} = \frac{1}{n}\sum_{i=0}^{n-1} E_t[r_{t+i}] + \text{primă}$

        Dacă prima este constantă $\Rightarrow$ spread-ul $(R_t - r_t)$ ar trebui să fie staționar.
    \end{block}
    \begin{exampleblock}{Constatări Tipice}
        \begin{itemize}\setlength{\itemsep}{0pt}
            \item Ambele rate sunt I(1) (confirmat de ADF)
            \item Testul Johansen: $r = 1$ vector de cointegrare
            \item Vector de cointegrare $\approx (1, -1)$: spread-ul este staționar
            \item Rata scurtă se ajustează ($\alpha_r < 0$), rata lungă slab exogenă
        \end{itemize}
    \end{exampleblock}
    \begin{block}{Implicație de Politică}
        Banca centrală controlează rata scurtă; rata lungă este condusă de așteptări.
    \end{block}
    }
\end{frame}

\begin{frame}{Exemplu: Paritatea Puterii de Cumpărare (PPP)}
    {\small
    \begin{block}{Teoria PPP}
        $e_t = p_t - p_t^*$ (log curs de schimb = diferențialul de prețuri)

        Cursul real de schimb: $q_t = e_t - p_t + p_t^*$ ar trebui să fie staționar (PPP pe termen lung)
    \end{block}
    \begin{exampleblock}{Provocări Empirice}
        \begin{itemize}\setlength{\itemsep}{0pt}
            \item Teste rădăcină unitară: $e_t$, $p_t$, $p_t^*$ toate I(1)
            \item Teste de cointegrare: Rezultate mixte în funcție de eșantion
            \item Timp de înjumătățire al deviațiilor PPP: 3-5 ani (ajustare lentă)
            \item Exogenitate slabă: Cursul de schimb adesea nu se ajustează
        \end{itemize}
    \end{exampleblock}
    \begin{alertblock}{Puzzle-ul PPP}
        Cursul real de schimb este foarte persistent---revenirea lentă la medie este greu de explicat cu modelele standard.
    \end{alertblock}
    }
\end{frame}

\begin{frame}{Exemplu: Strategia Pairs Trading}
    {\small
    \begin{block}{Ideea}
        Găsiți acțiuni cointegrate $\Rightarrow$ tranzacționați spread-ul staționar
    \end{block}
    \begin{exampleblock}{Pași de Implementare}
        \begin{enumerate}\setlength{\itemsep}{0pt}
            \item \textbf{Identificați perechile}: Testați cointegrarea (ex., Coca-Cola \& Pepsi)
            \item \textbf{Estimați spread-ul}: $z_t = P_A - \beta P_B$
            \item \textbf{Reguli de tranzacționare}:
            \begin{itemize}
                \item $z_t > \mu + 2\sigma$: Vindeți A, Cumpărați B (spread prea larg)
                \item $z_t < \mu - 2\sigma$: Cumpărați A, Vindeți B (spread prea îngust)
                \item Ieșiți când $z_t \approx \mu$
            \end{itemize}
        \end{enumerate}
    \end{exampleblock}
    \begin{alertblock}{Riscuri}
        Cointegrarea se poate rupe; spread-ul poate să nu revină; costuri de tranzacție.
    \end{alertblock}
    }
\end{frame}

\begin{frame}{Analiza Cointegrării în Python: Funcții Cheie}
    {\footnotesize
    \begin{block}{Biblioteci Esențiale}
        \texttt{from statsmodels.tsa.stattools import coint, adfuller} \\
        \texttt{from statsmodels.tsa.vector\_ar.vecm import coint\_johansen, VECM}
    \end{block}

    \begin{block}{Flux de Lucru}
        \begin{enumerate}\setlength{\itemsep}{0pt}
            \item Teste rădăcină unitară: \texttt{adfuller(serie)}
            \item Engle-Granger: \texttt{coint(y, x)} returnează stat. test \& p-value
            \item Johansen: \texttt{coint\_johansen(data, det\_order, k\_ar\_diff)}
            \item Estimare VECM: \texttt{model = VECM(data, k\_ar\_diff=2, coint\_rank=1)}
            \item Rezultate: \texttt{results = model.fit()}
        \end{enumerate}
    \end{block}

    \begin{alertblock}{Notă}
        Exemple complete de lucru sunt furnizate în notebook-urile Jupyter.
    \end{alertblock}
    }
\end{frame}

%=============================================================================
% SECTION 4: DISCUSSION TOPICS
%=============================================================================
\section{Subiecte de Discuție}

\begin{frame}{Discuție: Cointegrare vs Corelație}
    {\small
    \begin{alertblock}{Întrebare Cheie}
        Două serii sunt puternic corelate. Sunt ele cointegrate?
    \end{alertblock}
    \begin{block}{Răspuns: Nu neapărat!}
        \begin{itemize}\setlength{\itemsep}{0pt}
            \item \textbf{Corelație}: Măsoară co-mișcarea (poate fi falsă pentru I(1))
            \item \textbf{Cointegrare}: Necesită combinație liniară staționară
        \end{itemize}
    \end{block}
    \begin{exampleblock}{Exemplu}
        Două mersuri aleatoare independente pot avea corelație $> 0.9$ pur întâmplător (corelație falsă). Dar NU sunt cointegrate---spread-ul lor este tot I(1).

        \vspace{0.2cm}
        \textbf{Cointegrarea} implică o relație de echilibru pe termen lung semnificativă.
    \end{exampleblock}
    }
\end{frame}

\begin{frame}{Discuție: Alegerea Componentelor Deterministe}
    {\small
    \begin{alertblock}{Întrebare Cheie}
        Testul Johansen are 5 cazuri pentru componentele deterministe. Pe care să îl alegem?
    \end{alertblock}
    \begin{block}{Ghid}
        \begin{enumerate}\setlength{\itemsep}{0pt}
            \item \textbf{Fără constantă, fără trend}: Rar folosit (necesită date cu medie zero)
            \item \textbf{Constantă doar în EC}: Serii în nivel, fără drift
            \item \textbf{Constantă nerestricționată}: Cel mai comun pentru date economice
            \item \textbf{Trend în EC}: Seriile au trenduri deterministe
            \item \textbf{Trend nerestricționat}: Diferențe cu trend (necomun)
        \end{enumerate}
    \end{block}
    \begin{exampleblock}{Sfat Practic}
        Începeți cu Cazul 3 (constantă nerestricționată). Verificați sensibilitatea la specificație. Folosiți raționament economic: au nivelurile trenduri?
    \end{exampleblock}
    }
\end{frame}

%=============================================================================
% EXERCIȚIU CU ASISTENȚĂ AI
%=============================================================================
\section{Exercițiu cu asistență AI}

\begin{frame}{Exercițiu AI: Gândire critică}
    \vspace{-3mm}
    \begin{block}{\footnotesize Prompt de testat în ChatGPT / Claude / Copilot}
        {\footnotesize
        ``Folosind yfinance, descarcă prețurile zilnice de închidere Coca-Cola (KO) și PepsiCo (PEP) din 2015-01-01 până în 2024-12-31 (aprox.\ 2.500 observații). Testează dacă fiecare serie este I(1) cu ADF și KPSS. Testează cointegrarea cu Engle-Granger și Johansen. Estimează un model VECM și interpretează vitezele de ajustare ($\alpha$). Simulează o strategie simplă de pairs trading bazată pe spread-ul cointegrat. Vreau cod Python complet cu grafice.''
        }
    \end{block}
    \vspace{-2mm}
    {\footnotesize
    \textbf{Exercițiu}:
    \begin{enumerate}\setlength{\itemsep}{0pt}
        \item Rulați prompt-ul într-un LLM la alegere și analizați critic răspunsul.
        \item Verifică ordinul de integrare al fiecărei serii \textit{înainte} de cointegrare?
        \item Folosește atât Engle-Granger cât și Johansen? Rezultatele coincid?
        \item Coeficienții $\alpha$ au semnul corect? (ajustare spre echilibru pe termen lung)
        \item Strategia de pairs trading ține cont de costurile de tranzacționare?
    \end{enumerate}
    }
    \vspace{-2mm}
    \begin{alertblock}{}
        {\footnotesize \textbf{Atenție}: Codul generat de AI poate rula fără erori și arăta profesional. \textit{Asta nu înseamnă că e corect.}}
    \end{alertblock}
\end{frame}

%=============================================================================
% SECTION 5: EXERCISES
%=============================================================================
\section{Exerciții pentru Studiu Individual}

\begin{frame}{Exerciții de Făcut Acasă}
    {\footnotesize
    \begin{enumerate}\setlength{\itemsep}{2pt}
        \item \textbf{Teoretic}: Arătați că dacă $Y_t$ și $X_t$ sunt ambele mersuri aleatoare cu aceeași inovație, ele sunt cointegrate.

        \item \textbf{Calcul}: Având estimările VECM:
            \begin{align*}
                \Delta Y_t &= 0.5 - 0.3(Y_{t-1} - 2X_{t-1}) + 0.2\Delta Y_{t-1} \\
                \Delta X_t &= 0.1 + 0.1(Y_{t-1} - 2X_{t-1}) + 0.4\Delta X_{t-1}
            \end{align*}
            \begin{itemize}\setlength{\itemsep}{0pt}
                \item Care este vectorul de cointegrare?
                \item Care variabilă se ajustează mai rapid?
                \item Care este relația de echilibru pe termen lung?
            \end{itemize}

        \item \textbf{Aplicat}: Descărcați ratele trezoreriei la 10 ani și 3 luni:
            \begin{itemize}\setlength{\itemsep}{0pt}
                \item Testați pentru rădăcini unitare; Testați pentru cointegrare
                \item Estimați VECM; Interpretați coeficienții de ajustare
            \end{itemize}

        \item \textbf{Gândire Critică}: De ce ar putea PPP să fie valabilă pe termen lung dar nu pe termen scurt?
    \end{enumerate}
    }
\end{frame}

\begin{frame}{Indicii pentru Soluțiile Exercițiilor}
    {\footnotesize
    \begin{block}{Indicii}
        \begin{enumerate}\setlength{\itemsep}{1pt}
            \item Dacă $Y_t = Y_{t-1} + \varepsilon_t$ și $X_t = X_{t-1} + \varepsilon_t$ (același șoc), atunci $Y_t - X_t = Y_0 - X_0$ este constantă (staționară).

            \item Din VECM:
                \begin{itemize}\setlength{\itemsep}{0pt}
                    \item Vector de cointegrare: $(1, -2)$ (normalizat pe $Y$)
                    \item $Y$ se ajustează mai rapid: $|\alpha_Y| = 0.3 > |\alpha_X| = 0.1$
                    \item Termen lung: $Y = 2X$ (când termenul EC = 0)
                \end{itemize}

            \item Pentru ratele dobânzii:
                \begin{itemize}\setlength{\itemsep}{0pt}
                    \item Ambele sunt de obicei I(1); spread-ul de obicei staționar
                    \item Așteptați un vector de cointegrare cu $(1, -1)$
                    \item Rata scurtă se ajustează de obicei; rata lungă adesea slab exogenă
                \end{itemize}

            \item Deviații PPP: Costurile de transport, bunurile netranzacționate, prețurile rigide, tarifele, segmentarea pieței toate încetinesc ajustarea dar nu previn convergența pe termen lung.
        \end{enumerate}
    \end{block}
    }
\end{frame}

%=============================================================================
% SUMMARY
%=============================================================================
\begin{frame}{Concluzii Cheie din Acest Seminar}
    {\footnotesize
    \begin{block}{Puncte Principale}
        \begin{enumerate}\setlength{\itemsep}{0pt}
            \item \textbf{Cointegrarea}: Variabile I(1) cu combinație liniară staționară
            \item \textbf{Regresia falsă}: $R^2$ mare fără cointegrare este lipsit de sens
            \item \textbf{Engle-Granger}: Simplu, dar doar un vector de cointegrare
            \item \textbf{Johansen}: Vectori multipli, MLE, mai puternic
        \end{enumerate}
    \end{block}
    \begin{block}{Perspective VECM}
        \begin{itemize}\setlength{\itemsep}{0pt}
            \item $\bbeta$ definește echilibrul; $\balpha$ determină viteza de ajustare
            \item Exogenitate slabă ($\alpha = 0$): Variabila nu răspunde la dezechilibru
            \item Folosiți întotdeauna VECM (nu VAR în diferențe) când există cointegrare
        \end{itemize}
    \end{block}
    \begin{alertblock}{Amintiți-vă}
        Cointegrarea este despre \textbf{echilibrul pe termen lung}, nu doar corelație!
    \end{alertblock}
    }
\end{frame}

\end{document}
