% Capitolul 4: Modele SARIMA pentru Serii de Timp Sezoniere
% Prezentare Beamer
% Program de licență, Academia de Studii Economice din București

\documentclass[9pt, aspectratio=169, t]{beamer}

% Ensure content fits on slides
\setbeamersize{text margin left=8mm, text margin right=8mm}

%=============================================================================
% THEME AND STYLE CONFIGURATION
%=============================================================================
\usetheme{Madrid}
\usecolortheme{seahorse}

% IDA-Inspired Color Palette
\definecolor{MainBlue}{RGB}{26, 58, 110}
\definecolor{AccentBlue}{RGB}{42, 82, 140}
\definecolor{IDAred}{RGB}{220, 53, 69}
\definecolor{DarkGray}{RGB}{51, 51, 51}
\definecolor{MediumGray}{RGB}{128, 128, 128}
\definecolor{LightGray}{RGB}{248, 248, 248}
\definecolor{VeryLightGray}{RGB}{235, 235, 235}
\definecolor{Crimson}{RGB}{220, 53, 69}
\definecolor{Forest}{RGB}{46, 125, 50}
\definecolor{Amber}{RGB}{181, 133, 63}

\setbeamercolor{palette primary}{bg=MainBlue, fg=white}
\setbeamercolor{palette secondary}{bg=MainBlue!85, fg=white}
\setbeamercolor{palette tertiary}{bg=MainBlue!70, fg=white}
\setbeamercolor{structure}{fg=MainBlue}
\setbeamercolor{title}{fg=MainBlue}
\setbeamercolor{frametitle}{fg=MainBlue, bg=white}
\setbeamercolor{block title}{bg=MainBlue, fg=white}
\setbeamercolor{block body}{bg=VeryLightGray, fg=DarkGray}
\setbeamercolor{block title alerted}{bg=Crimson, fg=white}
\setbeamercolor{block body alerted}{bg=Crimson!8, fg=DarkGray}
\setbeamercolor{block title example}{bg=Forest, fg=white}
\setbeamercolor{block body example}{bg=Forest!8, fg=DarkGray}
\setbeamercolor{item}{fg=MainBlue}

\setbeamertemplate{navigation symbols}{}

\setbeamertemplate{footline}{
    \leavevmode%
    \hbox{%
        \begin{beamercolorbox}[wd=.333333\paperwidth,ht=2.5ex,dp=1ex,center]{author in head/foot}%
            \usebeamerfont{author in head/foot}\insertshortauthor
        \end{beamercolorbox}%
        \begin{beamercolorbox}[wd=.333333\paperwidth,ht=2.5ex,dp=1ex,center]{title in head/foot}%
            \usebeamerfont{title in head/foot}\insertshorttitle
        \end{beamercolorbox}%
        \begin{beamercolorbox}[wd=.333333\paperwidth,ht=2.5ex,dp=1ex,right]{date in head/foot}%
            \usebeamerfont{date in head/foot}\insertshortdate{}\hspace*{2em}
            \insertframenumber{} / \inserttotalframenumber\hspace*{2ex}
        \end{beamercolorbox}}%
    \vskip0pt%
}

%=============================================================================
% PACKAGES
%=============================================================================
\usepackage[utf8]{inputenc}
\usepackage[T1]{fontenc}
\usepackage{amsmath, amssymb, amsthm}
\usepackage{mathtools}
\usepackage{bm}
\usepackage{tikz}
\usetikzlibrary{arrows.meta, positioning, shapes, calc}
\usepackage{booktabs}
\usepackage{multirow}
\usepackage{array}
\usepackage{graphicx}
\usepackage{hyperref}
\hypersetup{colorlinks=false, pdfborder={0 0 0}}
\graphicspath{{../logos/}{../charts/}}

%=============================================================================
% THEOREM ENVIRONMENTS
%=============================================================================
\theoremstyle{definition}
\setbeamertemplate{theorems}[numbered]
\newtheorem{defn}{Definitie}
\newtheorem{thm}{Teorema}
\newtheorem{prop}{Propozitie}
\newtheorem{rmk}{Observație}

%=============================================================================
% CUSTOM COMMANDS
%=============================================================================
\newcommand{\E}{\mathbb{E}}
\newcommand{\Var}{\text{Var}}
\newcommand{\Cov}{\text{Cov}}
\newcommand{\Corr}{\text{Corr}}
\newcommand{\R}{\mathbb{R}}
\newcommand{\N}{\mathbb{N}}
\newcommand{\Z}{\mathbb{Z}}
\newcommand{\B}{\mathbf{B}}

%=============================================================================
% TITLE INFORMATION
%=============================================================================
\title[Capitolul 4: Modele SARIMA]{Capitolul 4: Modele SARIMA pentru Serii de Timp Sezoniere}
\subtitle{Program de licență, Facultatea de Cibernetică, Statistică și Informatică Economică, Academia de Studii Economice din București}
\author[Prof. dr. Daniel Traian Pele]{Prof. dr. Daniel Traian Pele\\[0.2cm]\footnotesize\texttt{danpele@ase.ro}}
\institute{Academia de Studii Economice din București}
\date{An Universitar 2025--2026}

\begin{document}

%=============================================================================
% TITLE SLIDE
%=============================================================================
\begin{frame}[plain]
    \begin{tikzpicture}[remember picture, overlay]
        \fill[IDAred] (current page.north west) rectangle ([yshift=-0.15cm]current page.north east);
        \node[anchor=north west] at ([xshift=0.5cm, yshift=-0.3cm]current page.north west) {
            \href{https://www.ase.ro}{\includegraphics[height=1.1cm]{ase_logo.png}}
        };
        \node[anchor=north] at ([yshift=-0.3cm]current page.north) {
            \href{https://ai4efin.ase.ro}{\includegraphics[height=1.1cm]{ai4efin_logo.png}}
        };
        \node[anchor=north east] at ([xshift=-0.5cm, yshift=-0.3cm]current page.north east) {
            \href{https://www.digital-finance-msca.com}{\includegraphics[height=1.1cm]{msca_logo.png}}
        };
    \end{tikzpicture}
    \vfill
    \begin{center}
        {\Large\textcolor{MediumGray}{Analiza și Prognoză Seriilor de Timp}}\\[0.3cm]
        {\Huge\textbf{\textcolor{MainBlue}{Capitolul 4: Modele SARIMA}}}\\[0.5cm]
        {\Large\textcolor{IDAred}{Serii de Timp Sezoniere}}
    \end{center}
    \vfill

    \begin{tikzpicture}[remember picture, overlay]
        \fill[IDAred] (current page.south west) rectangle ([yshift=0.15cm]current page.south east);
        \node[anchor=south west] at ([xshift=0.5cm, yshift=0.8cm]current page.south west) {
            \href{https://theida.net}{\includegraphics[height=0.9cm]{ida_logo.png}}
        };
        \node[anchor=south] at ([xshift=-3cm, yshift=0.8cm]current page.south) {
            \href{https://blockchain-research-center.com}{\includegraphics[height=0.9cm]{brc_logo.png}}
        };
        \node[anchor=south] at ([yshift=0.8cm]current page.south) {
            \href{https://quantinar.com}{\includegraphics[height=0.9cm]{qr_logo.png}}
        };
        \node[anchor=south] at ([xshift=3cm, yshift=0.8cm]current page.south) {
            \href{https://quantlet.com}{\includegraphics[height=0.9cm]{ql_logo.png}}
        };
        \node[anchor=south east] at ([xshift=-0.5cm, yshift=0.8cm]current page.south east) {
            \href{https://ipe.ro/new}{\includegraphics[height=0.9cm]{acad_logo.png}}
        };
    \end{tikzpicture}
\end{frame}

%=============================================================================
% TABLE OF CONTENTS
%=============================================================================
\begin{frame}{Cuprins}
    \vspace{-0.3cm}
    {\small
    \begin{columns}[T]
        \begin{column}{0.48\textwidth}
            \tableofcontents[sections={1-5}, hideallsubsections]
        \end{column}
        \begin{column}{0.48\textwidth}
            \tableofcontents[sections={6-9}, hideallsubsections]
        \end{column}
    \end{columns}
    }
\end{frame}

%=============================================================================
% MOTIVATION
%=============================================================================
\begin{frame}{Exemplu motivațional: Sezonalitatea este peste tot}
    \vspace{-0.3cm}
    \begin{center}
        \includegraphics[width=0.88\textwidth, height=0.62\textheight, keepaspectratio]{charts/ch4_motivation_seasonal.pdf}
    \end{center}
    \vspace{-0.2cm}
    {\footnotesize
    \begin{itemize}
        \item Vânzările cu amănuntul prezintă \textbf{tipare anuale} clare: vârfuri în decembrie, minime în ianuarie
        \item Modelele ARIMA standard nu pot captura aceste \textbf{cicluri sezoniere repetitive}
        \item Ignorarea sezonalității duce la erori sistematice de prognoză
    \end{itemize}
    }
\end{frame}

\begin{frame}{Înțelegerea componentelor sezoniere}
    \vspace{-0.3cm}
    \begin{center}
        \includegraphics[width=0.88\textwidth, height=0.62\textheight, keepaspectratio]{charts/ch4_motivation_decomposition.pdf}
    \end{center}
    \vspace{-0.2cm}
    {\footnotesize
    \begin{itemize}
        \item Serie de timp sezonieră = \textbf{Trend} + \textbf{Tipar sezonier} + \textbf{Reziduuri}
        \item Descompunerea ajută la vizualizarea separată a fiecărei componente
        \item Modelele SARIMA captează atât dinamică trendului, cât și comportamentul sezonier
    \end{itemize}
    }
\end{frame}

\begin{frame}{Aplicație reală: Tipare lunare}
    \vspace{-0.3cm}
    \begin{center}
        \includegraphics[width=0.88\textwidth, height=0.62\textheight, keepaspectratio]{charts/ch4_motivation_monthly.pdf}
    \end{center}
    \vspace{-0.2cm}
    {\footnotesize
    \begin{itemize}
        \item Cererea de energie prezintă o \textbf{sezonalitate lunară} puternică (cicluri de încălzire/răcire)
        \item Tiparul se repetă previzibil în fiecare an cu mici variații
        \item Companiile de utilități folosesc prognozele SARIMA pentru planificarea capacitatii
    \end{itemize}
    }
\end{frame}

\begin{frame}{De ce avem nevoie de SARIMA?}
    \vspace{-0.3cm}
    \begin{center}
        \includegraphics[width=0.88\textwidth, height=0.62\textheight, keepaspectratio]{charts/ch4_motivation_why_sarima.pdf}
    \end{center}
    \vspace{-0.2cm}
    {\footnotesize
    \begin{itemize}
        \item \textbf{Stânga}: ACF sezonieră prezintă vârfuri la lag-urile 12, 24, 36... (tipar anual)
        \item \textbf{Dreapta}: Reziduurile ARIMA încă prezintă autocorelație sezonieră --- modelul este incomplet
        \item SARIMA adaugă \textbf{termeni AR și MA sezonieri} pentru a captura aceste tipare
    \end{itemize}
    }
\end{frame}

\begin{frame}{Ce vom învăța astăzi}
    {\small
    \hfill\begin{minipage}{0.9\textwidth}
        \begin{columns}[T]
            \begin{column}{0.48\textwidth}
                \begin{block}{Concepte}
                    \begin{itemize}\setlength{\itemsep}{1pt}
                        \item Identificarea tiparelor sezoniere
                        \item Operatorul de diferențiere sezonieră
                        \item Notația SARIMA$(p,d,q)(P,D,Q)_s$
                        \item Celebrul ``Model Airline''
                        \item Selecția modelului pentru date sezoniere
                    \end{itemize}
                \end{block}
            \end{column}
            \begin{column}{0.48\textwidth}
                \begin{block}{Abilități}
                    \begin{itemize}\setlength{\itemsep}{1pt}
                        \item Diagnosticarea sezonalității din ACF/PACF
                        \item Determinarea perioadei sezoniere $s$
                        \item Alegerea ordinelor sezoniere $(P, D, Q)$
                        \item Implementarea SARIMA în Python/R
                        \item Prognoză seriilor de timp sezoniere
                    \end{itemize}
                \end{block}
            \end{column}
        \end{columns}
    \end{minipage}
    }

    \vspace{0.3cm}

    \begin{alertblock}{Ideea cheie}
        SARIMA = ARIMA aplicată la \textbf{două frecvențe}: nivelul obișnuit (pe termen scurt) și cel sezonier (pe termen lung)
    \end{alertblock}
\end{frame}

%=============================================================================
% SECTION 1: SEASONALITY IN TIME SERIES
%=============================================================================
\section{Sezonalitatea în seriile de timp}

\begin{frame}{Ce este sezonalitatea?}
    \begin{defn}[Sezonalitate]
        O serie de timp prezintă \textbf{sezonalitate} când arată fluctuații regulate, periodice care se repetă pe o perioadă fixă $s$ (perioadă sezonieră).
    \end{defn}

    \vspace{0.1cm}

    \begin{exampleblock}{Perioade sezoniere comune}
        \begin{itemize}
            \item Date lunare: $s = 12$ (ciclu anual)
            \item Date trimestriale: $s = 4$ (ciclu anual)
            \item Date săptămânale: $s = 52$ (anual) sau $s = 7$ (tipar săptămânal)
            \item Date zilnice: $s = 7$ (tipar săptămânal)
        \end{itemize}
    \end{exampleblock}
\end{frame}

\begin{frame}{Sezonalitatea: Ilustrare vizuala}
    \begin{center}
        \includegraphics[width=0.95\textwidth, height=0.58\textheight, keepaspectratio]{charts/ch4_def_seasonality.pdf}
    \end{center}
    \vspace{-0.2cm}
    \begin{exampleblock}{Perioade Sezoniere}
        Stânga: Date lunare cu $s=12$ (ciclu anual). Dreapta: Date trimestriale cu $s=4$. Tiparul se repetă la fiecare $s$ perioade --- această regularitate este exploatată de modelele SARIMA.
    \end{exampleblock}
\end{frame}

\begin{frame}{Exemple de date sezoniere}
	{\small
		\hfill\begin{minipage}{0.9\textwidth}
			\begin{columns}[T]
				\begin{column}{0.48\textwidth}
					\begin{block}{Serii economice}
						\begin{itemize}\setlength{\itemsep}{0pt}
							\item Vânzări cu amănuntul (vârfuri de sarbatori)
							\item Turism (vara/iarna)
							\item Productie agricola
							\item Consum de energie
							\item Ocuparea fortei de munca (cicluri de angajare)
						\end{itemize}
					\end{block}
				\end{column}
				\begin{column}{0.48\textwidth}
					\begin{block}{Alte domenii}
						\begin{itemize}\setlength{\itemsep}{0pt}
							\item Vreme/temperatura
							\item Trafic pe site-uri web
							\item Admisii la spital
							\item Utilizarea transportului
							\item Cererea de electricitate
						\end{itemize}
					\end{block}
				\end{column}
			\end{columns}
		\end{minipage}
	}
	\begin{alertblock}{De ce contează}
		Ignorarea sezonalității duce la prognoze distorsionate și inferența invalidă!
	\end{alertblock}
\end{frame}

\begin{frame}{Exemplu: Datele privind pasagerii companiilor aeriene}
    \vspace{-0.3cm}
    \begin{center}
        \includegraphics[width=0.78\textwidth, height=0.55\textheight, keepaspectratio]{charts/ch4_airline_data.pdf}
    \end{center}
    \vspace{-0.2cm}
    {\small
    \begin{itemize}
        \item Pasageri internaționali lunari ai companiilor aeriene (1949--1960)
        \item \textbf{Trend ascendent} clar și \textbf{amplitudine sezonieră crescătoare}
        \item Vârfurile din vara reflecta tiparele călătoriilor de vacanta
    \end{itemize}
    }
\end{frame}

\begin{frame}{Vizualizarea tiparelor sezoniere}
    \vspace{-0.3cm}
    \begin{center}
        \includegraphics[width=0.78\textwidth, height=0.58\textheight, keepaspectratio]{charts/ch4_seasonal_boxplot.pdf}
    \end{center}
    \vspace{-0.2cm}
    {\footnotesize
    \begin{itemize}
        \item Diagrama box plot relevă un tipar sezonier consistent de-a lungul anilor
        \item Iulie--August prezintă cele mai mari numere de pasageri (călătorii de vară)
        \item Noiembrie--Februarie prezintă cele mai mici numere (lunile de iarnă)
    \end{itemize}
    }
\end{frame}

\begin{frame}{Sezonalitate deterministă vs stochastică}
    {\small
    \hfill\begin{minipage}{0.9\textwidth}
    \begin{columns}[T]
        \begin{column}{0.48\textwidth}
            \begin{block}{Sezonalitate deterministă}
                Tipar sezonier fix:
                $Y_t = \sum_{j=1}^{s} \gamma_j D_{jt} + \varepsilon_t$
                unde $D_{jt}$ sunt variabile dummy sezoniere.

                \textbf{Proprietăți:}
                \begin{itemize}\setlength{\itemsep}{0pt}
                    \item Tiparul constant în timp
                    \item Eliminăt prin regresie
                \end{itemize}
            \end{block}
        \end{column}
        \begin{column}{0.48\textwidth}
            \begin{block}{Sezonalitate stochastică}
                Tipar sezonier în evolutie:
                $\Delta_s Y_t = Y_t - Y_{t-s}$
                prezintă structura de dependență.

                \textbf{Proprietăți:}
                \begin{itemize}\setlength{\itemsep}{0pt}
                    \item Tiparul evoluează în timp
                    \item Necesită diferențiere sezonieră
                \end{itemize}
            \end{block}
        \end{column}
    \end{columns}
    \end{minipage}
    }
\end{frame}

\begin{frame}{Detectarea sezonalității}
    {\small
    \hfill\begin{minipage}{0.9\textwidth}
    \begin{block}{Metode vizuale}
        \begin{itemize}
            \item Graficul seriei de timp -- căutați tipare repetitive
            \item Graficul sub-seriilor sezoniere -- comparați aceleași sezoane de-a lungul anilor
            \item Graficul ACF -- vârfuri la lag-uri sezoniere ($s, 2s, 3s, \ldots$)
        \end{itemize}
    \end{block}

    \vspace{0.1cm}

    \begin{block}{Teste statistice}
        \begin{itemize}
            \item Teste de rădăcină unitară sezonieră (HEGY, CH, OCSB)
            \item Testul F pentru variabile dummy sezoniere
            \item Testul Kruskal-Wallis (neparametric)
        \end{itemize}
    \end{block}

    \vspace{0.1cm}

    \begin{exampleblock}{Semnatura ACF}
        Sezonalitate puternică: ACF prezintă vârfuri semnificâțive la lag-urile $s, 2s, 3s, \ldots$
    \end{exampleblock}
    \end{minipage}
    }
\end{frame}

\begin{frame}{ACF relevă structura sezonieră}
    \vspace{-0.3cm}
    \begin{center}
        \includegraphics[width=0.82\textwidth, height=0.55\textheight, keepaspectratio]{charts/ch4_acf_seasonality.pdf}
    \end{center}
    \vspace{-0.2cm}
    {\footnotesize
    \begin{itemize}
        \item \textbf{Descreștere lentă} la toate lag-urile indică nestaționaritate (trend)
        \item \textbf{Vârfuri la lag-urile 12, 24, 36} confirmă tiparul sezonier ($s=12$)
        \item ACF la lag-urile sezoniere prezintă descreștere lentă $\Rightarrow$ necesită diferențiere sezonieră
    \end{itemize}
    }
\end{frame}

\begin{frame}{Testul F pentru variabile dummy sezoniere: Intuiție}
    {\small
    \hfill\begin{minipage}{0.9\textwidth}
    \begin{block}{Ce face acest test?}
        Testează dacă \textbf{valorile medii diferă semnificâțiv între sezoane}.
        \begin{itemize}\setlength{\itemsep}{0pt}
            \item Dacă media din ianuarie $\neq$ media din februarie $\neq$ ... $\neq$ media din decembrie $\Rightarrow$ sezonalitate
            \item Compară un model CU variabile dummy sezoniere vs. un model FARA
        \end{itemize}
    \end{block}

    \vspace{0.1cm}

    \begin{block}{Modelele comparate}
        \textbf{Restricționat}: $Y_t = \alpha + \varepsilon_t$ \quad \textbf{Nerestictionat}: $Y_t = \alpha + \sum_{j=1}^{s-1} \gamma_j D_{jt} + \varepsilon_t$

        unde $D_{jt} = 1$ dacă observația $t$ este în sezonul $j$, 0 altfel.
    \end{block}

    \vspace{0.1cm}

    \begin{alertblock}{Ideea cheie}
        Dacă adăugarea variabilelor dummy sezoniere \textbf{reduce semnificâțiv} erorile de predicție, atunci sezonalitatea este prezentă.
    \end{alertblock}
    \end{minipage}
    }
\end{frame}

\begin{frame}{Testul F pentru variabile dummy sezoniere: Formula și exemplu}
    \begin{block}{Formula statisticii F}
        $$F = \frac{(SSR_R - SSR_U)/(s-1)}{SSR_U/(n-s)} \sim F_{s-1, n-s}$$
        \begin{itemize}
            \item $SSR_R$ = Suma pătratelor reziduurilor din modelul restricționat (fără dummy)
            \item $SSR_U$ = Suma pătratelor reziduurilor din modelul nerestricționat (cu dummy)
            \item $s-1$ = numărul de restricții (lunar: 11, trimestrial: 3)
        \end{itemize}
    \end{block}

    \vspace{0.1cm}

    \begin{exampleblock}{Exemplu numeric (Date lunare, n=120)}
        $SSR_R = 15000$, $SSR_U = 8500$, $s = 12$

        $$F = \frac{(15000 - 8500)/11}{8500/108} = \frac{590.9}{78.7} = 7.51$$

        Valoarea critică $F_{0.05, 11, 108} \approx 1.87$. Deoarece $7.51 > 1.87$: \textbf{Respingem $H_0$} $\Rightarrow$ Sezonalitate prezentă!
    \end{exampleblock}
\end{frame}

\begin{frame}{Testul Kruskal-Wallis: Intuiție}
    {\small
    \hfill\begin{minipage}{0.9\textwidth}
    \begin{block}{Ce face acest test?}
        Un test \textbf{neparametric} care verifică dacă observațiile din diferite sezoane provin din aceeași distribuție.
        \begin{itemize}\setlength{\itemsep}{0pt}
            \item Funcționează prin \textbf{ordonarea} tuturor observațiilor de la cea mai mică la cea mai mare
            \item Verifică dacă rangurile sunt distribuite uniform între sezoane
            \item Dacă un sezon are în mod constant ranguri mai mari/mici $\Rightarrow$ sezonalitate
        \end{itemize}
    \end{block}

    \vspace{0.1cm}

    \begin{exampleblock}{De ce să-l folosim în locul testului F?}
        \begin{itemize}\setlength{\itemsep}{0pt}
            \item \textbf{Fără ipoteza de normalitate} -- funcționează cu orice distribuție
            \item \textbf{Robust la valori extreme} -- valorile extreme nu distorsionează rezultatele
        \end{itemize}
    \end{exampleblock}

    \vspace{0.1cm}

    \begin{alertblock}{Limitare}
        Mai putin puternic decat testul F când datele SUNT distribuite normal.
    \end{alertblock}
    \end{minipage}
    }
\end{frame}

\begin{frame}{Testul Kruskal-Wallis: Formula și exemplu}
    {\footnotesize
    \hfill\begin{minipage}{0.9\textwidth}
    \begin{block}{Statistică de test}
        $H = \frac{12}{N(N+1)} \sum_{j=1}^{s} \frac{R_j^2}{n_j} - 3(N+1)$ \quad unde $N$ = total obs., $n_j$ = obs. în sezonul $j$, $R_j$ = suma rangurilor.
    \end{block}

    \begin{exampleblock}{Exemplu: Vânzări trimestriale (n=20, s=4)}
        Date ordonate 1-20. Sumele rangurilor: T1: $R_1 = 15$, T2: $R_2 = 35$, T3: $R_3 = 70$, T4: $R_4 = 90$
        $$H = \frac{12}{20 \times 21}\left(\frac{15^2}{5} + \frac{35^2}{5} + \frac{70^2}{5} + \frac{90^2}{5}\right) - 3(21) = 12.6$$
        Valoarea critică $\chi^2_{0.05, 3} = 7.81$. Deoarece $12.6 > 7.81$: \textbf{Respingem $H_0$} $\Rightarrow$ Sezonalitate!
    \end{exampleblock}

    \begin{alertblock}{In Python}
        \texttt{scipy.stats.kruskal(q1, q2, q3, q4)}
    \end{alertblock}
    \end{minipage}
    }
\end{frame}

\begin{frame}{Testul HEGY: Ce problemă rezolvă?}
    {\small
    \hfill\begin{minipage}{0.9\textwidth}
    \begin{block}{Întrebarea cheie}
        Având o serie de timp sezonieră, trebuie să știm:
        \begin{enumerate}\setlength{\itemsep}{0pt}
            \item Are nevoie de \textbf{diferențiere obișnuită} $(1-L)$? $\Rightarrow$ setam $d=1$
            \item Are nevoie de \textbf{diferențiere sezonieră} $(1-L^s)$? $\Rightarrow$ setam $D=1$
        \end{enumerate}
        HEGY testează pentru \textbf{ambele} tipuri de rădăcini unitare simultan!
    \end{block}

    \begin{exampleblock}{De ce să nu folosim doar ADF?}
        ADF testează doar pentru o rădăcină unitară \textbf{obișnuită} la frecvența zero. Datele sezoniere pot avea rădăcini unitare la \textbf{frecvențe sezoniere} pe care ADF le omite!
    \end{exampleblock}

    \begin{alertblock}{HEGY testează frecvențe multiple}
        Trimestrial: testează la 0, $\pi$, $\pm\pi/2$. Lunar: testează la 0, $\pi$, $\pm\pi/6$, $\pm\pi/3$, $\pm\pi/2$, $\pm 2\pi/3$, $\pm 5\pi/6$.
    \end{alertblock}
    \end{minipage}
    }
\end{frame}

\begin{frame}{Testul HEGY: Formula de regresie (Trimestrial)}
    {\footnotesize
    \hfill\begin{minipage}{0.9\textwidth}
    \begin{block}{Regresia auxiliara HEGY}
        Pentru date trimestriale ($s=4$), estimăm:
        $$\Delta_4 y_t = \pi_1 z_{1,t-1} + \pi_2 z_{2,t-1} + \pi_3 z_{3,t-2} + \pi_4 z_{4,t-2} + \sum_{j=1}^{k} \phi_j \Delta_4 y_{t-j} + \varepsilon_t$$
    \end{block}
    \vspace{-0.1cm}
    \begin{block}{Variabile transformate}
        \vspace{-0.3cm}
        \begin{align*}
            z_{1t} &= (1+L+L^2+L^3)y_t = y_t + y_{t-1} + y_{t-2} + y_{t-3} \\[-0.1cm]
            z_{2t} &= -(1-L+L^2-L^3)y_t = -y_t + y_{t-1} - y_{t-2} + y_{t-3} \\[-0.1cm]
            z_{3t} &= -(1-L^2)y_t = -y_t + y_{t-2} \quad;\quad z_{4t} = -(L-L^3)y_t = -y_{t-1} + y_{t-3}
        \end{align*}
        \vspace{-0.3cm}
    \end{block}
    \vspace{-0.1cm}
    \begin{alertblock}{Ipoteze}
        $H_0: \pi_1=0$ (frecv.\ 0), $H_0: \pi_2=0$ (frecv.\ $\pi$), $H_0: \pi_3=\pi_4=0$ (frecv.\ $\pm\pi/2$)
    \end{alertblock}
    \end{minipage}
    }
\end{frame}

\begin{frame}{Testul HEGY: Reguli de decizie cu exemple}
    {\footnotesize
    \hfill\begin{minipage}{0.9\textwidth}
    \begin{block}{Valori critice HEGY (5\%, n=100, cu constanta)}
        \begin{tabular}{lccc}
            \toprule
            Test & Statistică & Valoare critică & Dacă NU este respins... \\
            \midrule
            $t_1$ ($\pi_1=0$) & t-stat & $-2.88$ & Necesită $d=1$ \\
            $t_2$ ($\pi_2=0$) & t-stat & $-2.88$ & Necesită $D=1$ \\
            $F_{34}$ ($\pi_3=\pi_4=0$) & F-stat & $6.57$ & Necesită $D=1$ \\
            \bottomrule
        \end{tabular}
    \end{block}

    \begin{exampleblock}{Exemplu: PIB trimestrial}
        Sa presupunem ca HEGY da: $t_1 = -1.52$, $t_2 = -4.21$, $F_{34} = 2.15$
        \begin{itemize}\setlength{\itemsep}{0pt}
            \item $t_1 = -1.52 > -2.88$: Nu putem respinge $\Rightarrow$ \textbf{necesită $d=1$}
            \item $t_2 = -4.21 < -2.88$: Respingem $\Rightarrow$ fără rădăcină unitară la $\pi$
            \item $F_{34} = 2.15 < 6.57$: Nu putem respinge $\Rightarrow$ \textbf{necesită $D=1$}
        \end{itemize}
        \textbf{Concluzie}: Folosim SARIMA cu $d=1, D=1$
    \end{exampleblock}
    \end{minipage}
    }
\end{frame}

\begin{frame}{Testul Canova-Hansen: Opusul testului HEGY}
    {\footnotesize
    \hfill\begin{minipage}{0.9\textwidth}
    \begin{block}{HEGY vs Canova-Hansen: Ipoteze nule diferite!}
        \begin{center}
        \begin{tabular}{lcc}
            \toprule
            & \textbf{HEGY} & \textbf{Canova-Hansen} \\
            \midrule
            $H_0$ & Rădăcină unitară sezonieră & \textbf{Fără} rădăcină unitară sezonieră \\
            $H_1$ & Fără rădăcină unitară sezonieră & Rădăcină unitară sezonieră \\
            \midrule
            Respingem $H_0$ & Folosim variabile dummy sezoniere & Folosim diferențiere $(1-L^s)$ \\
            Nu respingem & Folosim diferențiere $(1-L^s)$ & Folosim variabile dummy sezoniere \\
            \bottomrule
        \end{tabular}
        \end{center}
    \end{block}

    \begin{alertblock}{De ce contează?}
        \begin{itemize}\setlength{\itemsep}{0pt}
            \item HEGY: ``Demonstrați ca NU există rădăcină unitară'' (conservator fata de diferențiere)
            \item CH: ``Demonstrați ca EXISTA rădăcină unitară'' (conservator fata de variabile dummy)
            \item Folosiți \textbf{ambele} teste pentru concluzii robuste!
        \end{itemize}
    \end{alertblock}
    \end{minipage}
    }
\end{frame}

\begin{frame}{Testul Canova-Hansen: Formula}
    {\footnotesize
    \hfill\begin{minipage}{0.9\textwidth}
    \begin{block}{Procedura de testare}
        1. Regresam $y_t$ pe variabile dummy sezoniere: $y_t = \sum_{j=1}^{s} \gamma_j D_{jt} + u_t$

        2. Calculam sumele partiale la frecvența sezonieră $\lambda_i$:
        $S_{it}^{(c)} = \sum_{j=1}^{t} \hat{u}_j \cos(\lambda_i j)$, \; $S_{it}^{(s)} = \sum_{j=1}^{t} \hat{u}_j \sin(\lambda_i j)$
    \end{block}
    \vspace{-0.1cm}
    \begin{block}{Statistică de test LM}
        $$LM_i = \frac{1}{T^2 \hat{\omega}_i} \left[ \sum_{t=1}^{T} (S_{it}^{(c)})^2 + \sum_{t=1}^{T} (S_{it}^{(s)})^2 \right]$$
        unde $\hat{\omega}_i$ = estimator consistent al densității spectrale la frecvența $\lambda_i$.
    \end{block}
    \vspace{-0.1cm}
    \begin{alertblock}{Decizie}
        Respingem $H_0$ (staționaritate) dacă $LM > $ valoare critică $\Rightarrow$ este necesară diferențierea sezonieră.
    \end{alertblock}
    \end{minipage}
    }
\end{frame}

\begin{frame}{Sumar: Alegerea testului de sezonalitate potrivit}
    {\footnotesize
    \hfill\begin{minipage}{0.9\textwidth}
    \begin{center}
    \begin{tabular}{p{2cm}p{2.5cm}p{2.5cm}p{3cm}}
        \toprule
        \textbf{Test} & \textbf{$H_0$} & \textbf{Dacă respingem} & \textbf{Cel mai bun pentru} \\
        \midrule
        Test F & Fără sezonalitate & Sezonalitate există & Date normale \\
        Kruskal-Wallis & Fără dif. sezonieră & Sezonalitate există & Non-normale, valori extreme \\
        HEGY & Rădăcină unitară există & Folosim dummy & Determinarea $d$, $D$ \\
        Canova-Hansen & Fără rădăcină unitară & Folosim $(1-L^s)$ & Confirmarea stabilității \\
        \bottomrule
    \end{tabular}
    \end{center}

    \vspace{0.1cm}

    \begin{alertblock}{Ideea cheie}
        Test F/Kruskal-Wallis: ``\textit{Există sezonalitate?}'' \\
        HEGY/Canova-Hansen: ``\textit{Ce tip?}'' (deterministă vs stochastică)
    \end{alertblock}
    \end{minipage}
    }
\end{frame}

%=============================================================================
% SECTION 2: SEASONAL DIFFERENCING
%=============================================================================
\section{Diferențierea sezonieră}

\begin{frame}{Operatorul de diferența sezonieră}
    \begin{defn}[Diferenta sezonieră]
        \textbf{Operatorul de diferența sezonieră} $\Delta_s$ este definit ca:
        $$\Delta_s Y_t = (1 - L^s) Y_t = Y_t - Y_{t-s}$$
        unde $L^s Y_t = Y_{t-s}$ este operatorul de lag sezonier.
    \end{defn}

    \vspace{0.1cm}

    \begin{exampleblock}{Exemple}
        \begin{itemize}
            \item Date lunare ($s=12$): $\Delta_{12} Y_t = Y_t - Y_{t-12}$

            Compară fiecare lună cu aceeași lună din anul trecut
            \item Date trimestriale ($s=4$): $\Delta_4 Y_t = Y_t - Y_{t-4}$

            Compară fiecare trimestru cu același trimestru din anul trecut
        \end{itemize}
    \end{exampleblock}
\end{frame}

\begin{frame}{Diferenta sezonieră: Ilustrare vizuala}
    \begin{center}
        \includegraphics[width=0.95\textwidth, height=0.58\textheight, keepaspectratio]{charts/ch4_def_seasonal_diff.pdf}
    \end{center}
    \vspace{-0.2cm}
    \begin{block}{Efectul Diferențierii Sezoniere}
        Stânga: Seria originală cu tipar sezonier clar. Dreapta: După $\Delta_{12} = (1-L^{12})$, tiparul sezonier este eliminat. Comparația an-la-an elimină efectele sezoniere.
    \end{block}
\end{frame}

\begin{frame}{Demonstrație: Diferențierea Sezonieră Elimină Sezonalitatea Deterministă}
    \textbf{Afirmație:} Dacă $Y_t = \mu_t + \varepsilon_t$ unde $\mu_t = \mu_{t-s}$ (medie periodică), atunci $\Delta_s Y_t$ elimină media sezonieră.

    \vspace{0.2cm}
    \textbf{Demonstrație:} Fie $Y_t = \mu_t + \varepsilon_t$ unde $\mu_t$ are perioadă $s$. Aplicăm diferența sezonieră:
    \begin{align*}
    \Delta_s Y_t &= Y_t - Y_{t-s} = (\mu_t + \varepsilon_t) - (\mu_{t-s} + \varepsilon_{t-s}) \\
    &= \mu_t - \mu_{t-s} + \varepsilon_t - \varepsilon_{t-s} \\
    &= 0 + \varepsilon_t - \varepsilon_{t-s} \quad \text{(deoarece } \mu_t = \mu_{t-s}\text{)}
    \end{align*}

    \textbf{Proprietățile lui $\Delta_s Y_t = \varepsilon_t - \varepsilon_{t-s}$:}
    \begin{itemize}
        \item $\E[\Delta_s Y_t] = 0$ (medie constantă)
        \item $\Var(\Delta_s Y_t) = 2\sigma^2$ (varianță constantă)
        \item Autocovarianța: $\gamma(s) = -\sigma^2$, $\gamma(k) = 0$ pentru $k \neq 0, s$
    \end{itemize}

    \begin{exampleblock}{Rezultat}
        Diferențierea sezonieră transformă tiparul sezonier periodic în MA(1) la lag-ul sezonier.
    \end{exampleblock}
\end{frame}

\begin{frame}{Combinarea diferențierii obișnuite și sezoniere}
    \begin{block}{Diferențiere completa}
        Pentru serii cu atât trend cât și sezonalitate:
        $$\Delta \Delta_s Y_t = (1-L)(1-L^s) Y_t$$
    \end{block}

    \vspace{0.1cm}

    \begin{exampleblock}{Dezvoltare}
        $(1-L)(1-L^s) Y_t = Y_t - Y_{t-1} - Y_{t-s} + Y_{t-s-1}$

        Pentru date lunare ($s=12$):
        $$\Delta \Delta_{12} Y_t = Y_t - Y_{t-1} - Y_{t-12} + Y_{t-13}$$
    \end{exampleblock}

    \vspace{0.1cm}

    \begin{alertblock}{Ordinea diferențierii}
        \begin{itemize}
            \item $d$: numărul de diferențe obișnuite (eliminarea trendului)
            \item $D$: numărul de diferențe sezoniere (eliminarea trendului sezonier)
        \end{itemize}
    \end{alertblock}
\end{frame}

\begin{frame}{Efectul operatiilor de diferențiere}
    \vspace{-0.3cm}
    \begin{center}
        \includegraphics[width=0.82\textwidth, height=0.6\textheight, keepaspectratio]{charts/ch4_differencing_effect.pdf}
    \end{center}
    \vspace{-0.2cm}
    {\footnotesize
    \begin{itemize}
        \item Diferențierea obișnuită elimină trendul dar tiparul sezonier ramane
        \item Diferențierea sezonieră elimină sezonalitatea dar tiparul de trend ramane
        \item \textbf{Ambele diferențe} sunt necesare pentru a atinge staționaritatea
    \end{itemize}
    }
\end{frame}

\begin{frame}{ACF înainte și după diferențiere}
    \vspace{-0.3cm}
    \begin{center}
        \includegraphics[width=0.82\textwidth, height=0.6\textheight, keepaspectratio]{charts/ch4_acf_differencing.pdf}
    \end{center}
    \vspace{-0.2cm}
    {\footnotesize
    \begin{itemize}
        \item ACF originală: descreștere lentă indică nestaționaritate
        \item După $\Delta$: vârfuri sezoniere raman la lag-urile 12, 24, 36
        \item După $\Delta_{12}$: descreșterea de trend ramane la lag-urile inițiale
        \item După $\Delta\Delta_{12}$: ACF se opreste brusc $\Rightarrow$ \textbf{staționară}
    \end{itemize}
    }
\end{frame}

\begin{frame}{Integrare sezonieră}
    \begin{defn}[Proces integrat sezonier]
        O serie $Y_t$ este \textbf{integrata sezonier} de ordinul $(d, D)_s$, scrisa $Y_t \sim I(d, D)_s$, dacă:
        $$(1-L)^d (1-L^s)^D Y_t$$
        este staționară.
    \end{defn}

    \vspace{0.1cm}

    \begin{exampleblock}{Cazuri comune}
        \begin{itemize}
            \item $I(1,0)_{12}$: Doar rădăcină unitară obișnuită (lunară)
            \item $I(0,1)_{12}$: Doar rădăcină unitară sezonieră
            \item $I(1,1)_{12}$: Atât rădăcină unitară obișnuită cât și sezonieră
        \end{itemize}
    \end{exampleblock}
\end{frame}

%=============================================================================
% SECTION 3: SARIMA MODEL DEFINITION
%=============================================================================
\section{Modelul SARIMA}

\begin{frame}{Definitia modelului SARIMA}
    \begin{defn}[SARIMA$(p,d,q)\times(P,D,Q)_s$]
        Modelul \textbf{Seasonal ARIMA} este:
        $$\phi(L)\Phi(L^s)(1-L)^d(1-L^s)^D Y_t = c + \theta(L)\Theta(L^s)\varepsilon_t$$
    \end{defn}

    {\footnotesize
    \begin{block}{Componente}
        \begin{itemize}\setlength{\itemsep}{0pt}
            \item $\phi(L) = 1 - \phi_1 L - \cdots - \phi_p L^p$: AR non-sezonier
            \item $\Phi(L^s) = 1 - \Phi_1 L^s - \cdots - \Phi_P L^{Ps}$: AR sezonier
            \item $\theta(L) = 1 + \theta_1 L + \cdots + \theta_q L^q$: MA non-sezonier
            \item $\Theta(L^s) = 1 + \Theta_1 L^s + \cdots + \Theta_Q L^{Qs}$: MA sezonier
            \item $(1-L)^d$: Diferențiere obișnuită; $(1-L^s)^D$: Diferențiere sezonieră
        \end{itemize}
    \end{block}
    }
\end{frame}

\begin{frame}{SARIMA: Ilustrare vizuala}
    \begin{center}
        \includegraphics[width=0.95\textwidth, height=0.58\textheight, keepaspectratio]{charts/ch4_def_sarima.pdf}
    \end{center}
    \vspace{-0.2cm}
    \begin{alertblock}{Strategia de Diferențiere}
        Transformare progresivă: Originală $\to$ diferență obișnuită (elimină trendul) $\to$ diferență sezonieră (elimină sezonalitatea) $\to$ ambele. Aplicați diferențierea minimă necesară pentru a obține staționaritate.
    \end{alertblock}
\end{frame}

\begin{frame}{Demonstrație: Structura Multiplicâțivă Sezonieră}
    \textbf{De ce multiplicâțivă?} Considerăm SARIMA$(1,0,0)\times(1,0,0)_s$:
    $$(1-\phi L)(1-\Phi L^s)Y_t = \varepsilon_t$$

    \vspace{0.15cm}
    \textbf{Dezvoltăm produsul:}
    \begin{align*}
    (1-\phi L)(1-\Phi L^s)Y_t &= Y_t - \phi L Y_t - \Phi L^s Y_t + \phi\Phi L^{s+1} Y_t \\
    &= Y_t - \phi Y_{t-1} - \Phi Y_{t-s} + \phi\Phi Y_{t-s-1}
    \end{align*}

    \textbf{Rezultat:} Modelul include un \textbf{termen de interacțiune} $\phi\Phi Y_{t-s-1}$

    \vspace{0.15cm}
    \begin{exampleblock}{Interpretare (Lunar, $s=12$)}
        $Y_t$ depinde de:
        \begin{itemize}
            \item $Y_{t-1}$: Luna trecută (dinamică pe termen scurt)
            \item $Y_{t-12}$: Aceeași lună anul trecut (efectul sezonier)
            \item $Y_{t-13}$: Interacțiunea ambelor efecte
        \end{itemize}
    \end{exampleblock}

    \begin{block}{Parcimonie}
        Forma multiplicâțivă: 2 parametri ($\phi, \Phi$). Forma aditivă ar necesită 3+ parametri.
    \end{block}
\end{frame}

\begin{frame}{Notația SARIMA}
    \begin{block}{Specificâție completa}
        SARIMA$(p,d,q)\times(P,D,Q)_s$ are 7 parametri de specificăt:
    \end{block}

    \vspace{0.1cm}

    \begin{table}
        \centering
        \small
        \begin{tabular}{ll}
            \toprule
            \textbf{Parametru} & \textbf{Semnificâție} \\
            \midrule
            $p$ & Ordinul AR non-sezonier \\
            $d$ & Ordinul diferențierii non-sezoniere \\
            $q$ & Ordinul MA non-sezonier \\
            $P$ & Ordinul AR sezonier \\
            $D$ & Ordinul diferențierii sezoniere \\
            $Q$ & Ordinul MA sezonier \\
            $s$ & Perioada sezonieră \\
            \bottomrule
        \end{tabular}
    \end{table}

    \vspace{0.1cm}

    \begin{exampleblock}{Exemplu}
        {\small SARIMA$(1,1,1)\times(1,1,1)_{12}$: Date lunare cu AR(1), MA(1), AR sezonier(1), MA sezonier(1), și atât diferențiere obișnuită cât și sezonieră.}
    \end{exampleblock}
\end{frame}

\begin{frame}{Modele SARIMA comune}
    {\small
    \hfill\begin{minipage}{0.9\textwidth}
    \begin{block}{Modelul Airline: SARIMA$(0,1,1)\times(0,1,1)_s$}
        $(1-L)(1-L^s)Y_t = (1+\theta L)(1+\Theta L^s)\varepsilon_t$ -- Model clasic (Box \& Jenkins, 1970)
    \end{block}

    \begin{block}{SARIMA$(1,0,0)\times(1,0,0)_s$}
        $(1-\phi L)(1-\Phi L^s)Y_t = \varepsilon_t$ -- AR sezonier și non-sezonier pur
    \end{block}

    \begin{block}{SARIMA$(0,1,1)\times(0,1,0)_s$}
        $(1-L)(1-L^s)Y_t = (1+\theta L)\varepsilon_t$ -- Random walk + dif. sezonieră + MA(1)
    \end{block}
    \end{minipage}
    }
\end{frame}

\begin{frame}{Structura multiplicâțivă}
    \begin{block}{De ce multiplicâțivă?}
        Părțile sezonieră și non-sezonieră se \textbf{înmulțesc}:
        $$\phi(L)\Phi(L^s) \quad \text{si} \quad \theta(L)\Theta(L^s)$$
    \end{block}

    \vspace{0.1cm}

    \begin{exampleblock}{Exemplu: SARIMA$(1,0,0)\times(1,0,0)_{12}$}
        $(1-\phi L)(1-\Phi L^{12})Y_t = \varepsilon_t$

        Dezvoltand:
        $Y_t - \phi Y_{t-1} - \Phi Y_{t-12} + \phi\Phi Y_{t-13} = \varepsilon_t$

        Termenul incrucist $\phi\Phi Y_{t-13}$ captează interactiunea!
    \end{exampleblock}

    \vspace{0.1cm}

    \begin{alertblock}{Interpretare}
        Structura multiplicâțivă permite modelarea parcimonioasă a tiparelor sezoniere complexe cu puțini parametri.
    \end{alertblock}
\end{frame}

%=============================================================================
% SECTION 4: SEASONAL ACF AND PACF
%=============================================================================
\section{Tipare ACF și PACF sezoniere}

\begin{frame}{ACF/PACF pentru modele sezoniere}
    \begin{block}{Ideea cheie}
        Modelele sezoniere prezintă tipare la ambele:
        \begin{itemize}
            \item Lag-uri non-sezoniere: $1, 2, 3, \ldots$
            \item Lag-uri sezoniere: $s, 2s, 3s, \ldots$
        \end{itemize}
    \end{block}

    \vspace{0.1cm}

    \begin{table}
        \centering
        \small
        \begin{tabular}{lcc}
            \toprule
            \textbf{Model} & \textbf{ACF} & \textbf{PACF} \\
            \midrule
            SAR($P$) & Descreste la $s, 2s, \ldots$ & Se opreste după $Ps$ \\
            SMA($Q$) & Se opreste după $Qs$ & Descreste la $s, 2s, \ldots$ \\
            SARMA & Descreste la lag-uri sezoniere & Descreste la lag-uri sezoniere \\
            \bottomrule
        \end{tabular}
    \end{table}
\end{frame}

\begin{frame}{Exemplu: ACF/PACF pentru modelul Airline}
    \begin{block}{SARIMA$(0,1,1)\times(0,1,1)_{12}$}
        După diferențiere $W_t = (1-L)(1-L^{12})Y_t$:
        $$W_t = (1+\theta L)(1+\Theta L^{12})\varepsilon_t$$
    \end{block}

    \vspace{0.1cm}

    {\small
    \begin{exampleblock}{Tiparul ACF așteptat}
        \begin{itemize}
            \item Vârf la lag-ul 1 (de la $\theta$)
            \item Vârf la lag-ul 12 (de la $\Theta$)
            \item Vârf la lag-ul 13 (de la interactiunea $\theta \cdot \Theta$)
            \item Toate celelalte lag-uri aproape de zero
        \end{itemize}
    \end{exampleblock}

    \begin{exampleblock}{Tiparul PACF așteptat}
        \begin{itemize}
            \item Descreștere exponențială la lag-urile $1, 2, 3, \ldots$
            \item Descreștere exponențială la lag-urile $12, 24, 36, \ldots$
        \end{itemize}
    \end{exampleblock}
    }
\end{frame}

\begin{frame}{Ghid de identificare a modelului}
    {\small
    \hfill\begin{minipage}{0.9\textwidth}
    \begin{block}{Proces pas cu pas}
        \begin{enumerate}
            \item Examinați ACF pentru descreștere lentă la lag-uri sezoniere $\Rightarrow$ diferențiere sezonieră
            \item După diferențiere, verificați tiparele ACF/PACF
            \item Comportamentul non-sezonier la lag-urile $1, 2, \ldots, s-1$
            \item Comportamentul sezonier la lag-urile $s, 2s, 3s, \ldots$
        \end{enumerate}
    \end{block}

    \vspace{0.1cm}

    \begin{alertblock}{Sfaturi practice}
        \begin{itemize}
            \item Începeți cu $d \leq 1$ și $D \leq 1$
            \item De obicei $P, Q \leq 2$ este suficient
            \item Folosiți criterii informationale (AIC, BIC) pentru selecția finala
            \item Algoritmii Auto-SARIMA pot ajută
        \end{itemize}
    \end{alertblock}
    \end{minipage}
    }
\end{frame}

%=============================================================================
% SECTION 5: ESTIMATION AND DIAGNOSTICS
%=============================================================================
\section{Estimare și diagnosticare}

\begin{frame}{Metode de estimare}
    \begin{block}{Estimare prin verosimilitate maxima}
        Abordare standard pentru SARIMA:
        \begin{itemize}
            \item MLE condiționată (condiționată de valorile inițiale)
            \item MLE exacta (prin filtrul Kalman)
        \end{itemize}
    \end{block}

    \vspace{0.1cm}

    \begin{block}{Considerații computationale}
        \begin{itemize}
            \item Mai multi parametri decat ARIMA $\Rightarrow$ mai multe date necesare
            \item Parametrii sezonieri estimați din lag-urile $s, 2s, \ldots$
            \item Necesită suficiente cicluri sezoniere (cel putin 3-4 ani de date lunare)
        \end{itemize}
    \end{block}
\end{frame}

\begin{frame}{Staționaritate și invertibilitate}
    \begin{block}{Condiții de staționaritate}
        Atât polinoamele AR non-sezoniere cât și sezoniere trebuie să aibă rădăcini în afara cercului unitate:
        \begin{itemize}
            \item $\phi(z) = 0 \Rightarrow |z| > 1$
            \item $\Phi(z^s) = 0 \Rightarrow |z| > 1$
        \end{itemize}
    \end{block}

    \vspace{0.1cm}

    \begin{block}{Condiții de invertibilitate}
        Atât polinoamele MA non-sezoniere cât și sezoniere trebuie să aibă rădăcini în afara cercului unitate:
        \begin{itemize}
            \item $\theta(z) = 0 \Rightarrow |z| > 1$
            \item $\Theta(z^s) = 0 \Rightarrow |z| > 1$
        \end{itemize}
    \end{block}
\end{frame}

\begin{frame}{Verificarea diagnosticării}
    \begin{block}{Analiza reziduurilor}
        După ajustarea SARIMA, verificați ca reziduurile sunt zgomot alb:
        \begin{enumerate}
            \item Graficul reziduurilor în timp (fără tipare)
            \item ACF a reziduurilor (fără vârfuri semnificâțive)
            \item Testul Ljung-Box la lag-uri multiple inclusiv sezoniere
            \item Teste de normalitate (grafic Q-Q, Jarque-Bera)
        \end{enumerate}
    \end{block}

    \vspace{0.1cm}

    \begin{alertblock}{Important}
        Verificați ACF la \textbf{ambele} lag-uri non-sezoniere și sezoniere!

        ACF semnificâțivă la lag-ul 12 sugerează modelare sezonieră inadecvată.
    \end{alertblock}
\end{frame}

\begin{frame}{Criterii de selectie a modelului}
    \begin{block}{Criterii informationale}
        Comparați modelele SARIMA concurente folosind:
        \begin{itemize}
            \item AIC = $-2\ln(L) + 2k$
            \item BIC = $-2\ln(L) + k\ln(n)$
            \item AICc = AIC + $\frac{2k(k+1)}{n-k-1}$ (corectat pentru esantioane mici)
        \end{itemize}
        unde $k = p + q + P + Q + 1$ (plus 1 pentru varianța).
    \end{block}

    \vspace{0.1cm}

    \begin{exampleblock}{Auto-SARIMA}
        \texttt{pmdarima.auto\_arima()} din Python cu \texttt{seasonal=True} cauta automat $(p,d,q)\times(P,D,Q)_s$ optim.
    \end{exampleblock}
\end{frame}

%=============================================================================
% SECTION 6: FORECASTING
%=============================================================================
\section{Prognoză cu SARIMA}

\begin{frame}{Prognoze punctuale}
    \begin{block}{Calculul prognozei}
        Prognozele SARIMA sunt calculate recursiv:
        \begin{itemize}
            \item Înlocuiți $\varepsilon_{T+h}$ viitor cu 0
            \item Înlocuiți $Y_{T+h}$ viitor cu prognozele $\hat{Y}_{T+h|T}$
            \item Folosiți valorile trecute cunoscute $Y_T, Y_{T-1}, \ldots$
        \end{itemize}
    \end{block}

    \vspace{0.1cm}

    \begin{exampleblock}{Tiparul sezonier în prognoze}
        Prognozele SARIMA captează în mod natural sezonalitatea:
        \begin{itemize}
            \item Pe termen scurt: influențate de valorile recente
            \item Pe termen lung: revin la tiparul sezonier
        \end{itemize}
    \end{exampleblock}
\end{frame}

\begin{frame}{Intervale de prognoză}
    \begin{block}{Cuantificarea incertitudinii}
        Interval de predicție $(1-\alpha)$\%:
        $$\hat{Y}_{T+h|T} \pm z_{\alpha/2} \sqrt{\Var(e_{T+h})}$$

        Varianta calculata din reprezentarea MA($\infty$).
    \end{block}

    \vspace{0.1cm}

    \begin{alertblock}{Proprietăți cheie}
        \begin{itemize}
            \item Intervalele se lărgesc cu orizontul de prognoză
            \item Pentru serii $I(1,1)_s$: intervalele cresc nelimitat
            \item Tiparul sezonier vizibil în prognozele punctuale
            \item Incertitudinea captează atât variatia de trend cât și cea sezonieră
        \end{itemize}
    \end{alertblock}
\end{frame}

\begin{frame}{Prognoze pe orizont lung}
    \begin{block}{Comportamentul când $h \to \infty$}
        \begin{itemize}
            \item Prognozele punctuale converg la tiparul sezonier determinist
            \item Dacă există deriva: trend linear + tipar sezonier
            \item Intervalele de prognoză continuă să se lărgească
        \end{itemize}
    \end{block}

    \vspace{0.1cm}

    \begin{exampleblock}{Implicație practică}
        \begin{itemize}
            \item Pe termen scurt: SARIMA captează atât nivelul cât și sezonul
            \item Pe termen mediu: Prognoze sezoniere bune, incertitudine crescătoare
            \item Pe termen lung: Reflecta în principal tiparul sezonier, intervale largi
        \end{itemize}
    \end{exampleblock}
\end{frame}

%=============================================================================
% SECTION 7: REAL DATA APPLICATION
%=============================================================================
\section{Aplicație pe date reale: Pasageri companiilor aeriene}

\begin{frame}{Datele privind pasagerii companiilor aeriene}
    \vspace{-0.2cm}
    \begin{center}
        \includegraphics[width=0.9\textwidth, height=0.68\textheight, keepaspectratio]{charts/ch4_airline_data.pdf}
    \end{center}
    \vspace{-0.1cm}
    {\small
    \begin{itemize}
        \item Set de date clasic: Pasageri internaționali lunari ai companiilor aeriene (1949-1960)
        \item Trend ascendent clar și amplitudine sezonieră crescătoare
    \end{itemize}
    }
\end{frame}

\begin{frame}{Descompunerea sezonieră}
    \vspace{-0.2cm}
    \begin{center}
        \includegraphics[width=0.9\textwidth, height=0.68\textheight, keepaspectratio]{charts/ch4_decomposition.pdf}
    \end{center}
    \vspace{-0.1cm}
    {\footnotesize
    \begin{itemize}
        \item Trend: Crestere puternică ascendenta
        \item Sezonalitate: Vârfuri de vară (călătorii de vacanta)
        \item Rezidual: Variație aleatoare după eliminarea trendului și sezonului
    \end{itemize}
    }
\end{frame}

\begin{frame}{Analiza ACF/PACF}
    \vspace{-0.2cm}
    \begin{center}
        \includegraphics[width=0.88\textwidth, height=0.7\textheight, keepaspectratio]{charts/ch4_acf_pacf.pdf}
    \end{center}
    \vspace{-0.1cm}
    {\footnotesize
    \begin{itemize}
        \item După diferențierea $\Delta\Delta_{12}$: vârfuri la lag-urile 1 și 12
        \item Sugerează SARIMA$(0,1,1)\times(0,1,1)_{12}$ (Modelul Airline)
    \end{itemize}
    }
\end{frame}

\begin{frame}{Rezultatele prognozei SARIMA}
    \vspace{-0.2cm}
    \begin{center}
        \includegraphics[width=0.9\textwidth, height=0.68\textheight, keepaspectratio]{charts/ch4_sarima_forecast.pdf}
    \end{center}
    \vspace{-0.1cm}
    {\small
    \begin{itemize}
        \item SARIMA captează atât trendul cât și tiparul sezonier
        \item Prognozele prezintă vârfuri și minime sezoniere corespunzătoare
    \end{itemize}
    }
\end{frame}

\begin{frame}{Diagnosticarea modelului}
    \vspace{-0.2cm}
    \begin{center}
        \includegraphics[width=0.88\textwidth, height=0.7\textheight, keepaspectratio]{charts/ch4_diagnostics.pdf}
    \end{center}
    \vspace{-0.1cm}
    {\footnotesize
    \begin{itemize}
        \item Reziduurile par aleatorii; ACF în limite la toate lag-urile
        \item Modelul captează adecvat structura sezonieră
    \end{itemize}
    }
\end{frame}

\begin{frame}{Implementare în Python}
    {\small
    \begin{block}{Ajustarea SARIMA în Python}
        \texttt{from statsmodels.tsa.statespace.sarimax import SARIMAX}

        \vspace{0.1cm}
        \texttt{model = SARIMAX(y, order=(0,1,1), seasonal\_order=(0,1,1,12))}

        \texttt{results = model.fit()}

        \texttt{forecast = results.get\_forecast(steps=24)}
    \end{block}

    \vspace{0.1cm}

    \begin{alertblock}{Nota}
        Exemple complete în Python cu comentarii sunt furnizate în caietele Jupyter.
    \end{alertblock}
    }
\end{frame}

%=============================================================================
% SECTION 7.5: STUDIU DE CAZ - PASAGERI AERIENI
%=============================================================================
\section{Studiu de caz: Pasageri aerieni}

\begin{frame}{Studiu de caz: Date despre pasagerii aerieni}
    \vspace{-0.2cm}
    \begin{center}
        \includegraphics[width=0.95\textwidth, height=0.65\textheight, keepaspectratio]{charts/ch4_case_raw_data.pdf}
    \end{center}
    \vspace{-0.1cm}
    {\small
    \begin{itemize}
        \item Setul de date clasic Box-Jenkins: pasageri aerieni lunari (1949-1960)
        \item Trend ascendent clar și amplitudine sezonieră crescătoare
        \item Sezonalitatea multiplicâțivă sugerează transformarea logaritmică
    \end{itemize}
    }
\end{frame}

\begin{frame}{Pasul 1: Transformări}
    \vspace{-0.2cm}
    \begin{center}
        \includegraphics[width=0.95\textwidth, height=0.65\textheight, keepaspectratio]{charts/ch4_case_transformations.pdf}
    \end{center}
    \vspace{-0.1cm}
    {\small
    \begin{itemize}
        \item Transformarea log stabilizează varianța (multiplicâțiv $\succ$ aditiv)
        \item Prima diferență elimină trendul; diferența sezonieră elimină sezonalitatea
        \item Seria dublu diferențiată pare staționară
    \end{itemize}
    }
\end{frame}

\begin{frame}{Pasul 2: Analiza ACF/PACF}
    \vspace{-0.2cm}
    \begin{center}
        \includegraphics[width=0.95\textwidth, height=0.65\textheight, keepaspectratio]{charts/ch4_case_acf_pacf.pdf}
    \end{center}
    \vspace{-0.1cm}
    {\small
    \begin{itemize}
        \item ACF: Vârf semnificâțiv la lag 1 și lag 12 $\Rightarrow$ MA(1), SMA(1)
        \item PACF: Tipar de descreștere exponențială confirmă structura MA
        \item Sugerează SARIMA$(0,1,1)\times(0,1,1)_{12}$ (modelul airline)
    \end{itemize}
    }
\end{frame}

\begin{frame}{Pasul 3: Compararea modelelor}
    \vspace{-0.2cm}
    \begin{center}
        \includegraphics[width=0.95\textwidth, height=0.65\textheight, keepaspectratio]{charts/ch4_case_model_comparison.pdf}
    \end{center}
    \vspace{-0.1cm}
    {\small
    \begin{itemize}
        \item Comparăm modelele SARIMA cândidate folosind criteriul AIC
        \item SARIMA$(0,1,1)\times(0,1,1)_{12}$ oferă cea mai bună ajustare (AIC minim)
        \item Acesta este faimosul ``model airline'' identificat de Box \& Jenkins
    \end{itemize}
    }
\end{frame}

\begin{frame}{Pasul 4: Diagnosticarea reziduurilor}
    \vspace{-0.2cm}
    \begin{center}
        \includegraphics[width=0.95\textwidth, height=0.65\textheight, keepaspectratio]{charts/ch4_case_diagnostics.pdf}
    \end{center}
    \vspace{-0.1cm}
    {\small
    \begin{itemize}
        \item Reziduurile par aleatorii fără autocorelație remanentă
        \item Graficul Q-Q arată normalitate aproximativă
        \item Modelul captează adecvat atât trendul cât și structura sezonieră
    \end{itemize}
    }
\end{frame}

\begin{frame}{Pasul 5: Prognoză}
    \vspace{-0.2cm}
    \begin{center}
        \includegraphics[width=0.95\textwidth, height=0.65\textheight, keepaspectratio]{charts/ch4_case_forecast.pdf}
    \end{center}
    \vspace{-0.1cm}
    {\small
    \begin{itemize}
        \item Prognoză pe 24 de luni cu interval de încredere de 95\%
        \item Modelul captează tiparul sezonier și trendul ascendent
        \item Intervalele de predicție se lărgesc corespunzător cu orizontul prognozei
    \end{itemize}
    }
\end{frame}

%=============================================================================
% SECTION 8: SUMMARY
%=============================================================================
\section{Sumar}

\begin{frame}{Concluzii cheie}
    {\small
    \hfill\begin{minipage}{0.9\textwidth}
    \begin{block}{Puncte principale}
        \begin{enumerate}
            \item \textbf{Sezonalitatea} este comuna în datele economice și de afaceri
            \item \textbf{Diferențierea sezonieră} $(1-L^s)$ elimină sezonalitatea stochastică
            \item \textbf{SARIMA}$(p,d,q)\times(P,D,Q)_s$ extinde ARIMA pentru date sezoniere
            \item \textbf{Structura multiplicâțivă} captează interactiunile sezon-trend
            \item \textbf{ACF/PACF} prezintă tipare la ambele lag-uri obișnuite și sezoniere
            \item \textbf{Selecția modelului}: Folosiți AIC/BIC sau algoritmi auto-SARIMA
        \end{enumerate}
    \end{block}

    \vspace{0.1cm}

    \begin{alertblock}{Pașii următorii}
        Capitolul 5 va acoperi seriile de timp multivariate: modele VAR, cauzalitatea Granger și cointegrarea.
    \end{alertblock}
    \end{minipage}
    }
\end{frame}

%=============================================================================
% SECTION 9: QUIZ
%=============================================================================
\section{Quiz}

\begin{frame}{Întrebarea 1}
    \begin{alertblock}{Întrebare}
        Pentru date lunare cu sezonalitate anuala, care este perioadă sezonieră $s$?
    \end{alertblock}

    \vspace{0.3cm}

    \begin{enumerate}[(A)]
        \item $s = 4$
        \item $s = 7$
        \item $s = 12$
        \item $s = 52$
    \end{enumerate}
\end{frame}

\begin{frame}{Întrebarea 1: Răspuns}
    \begin{exampleblock}{Răspuns corect: (C) $s = 12$ (12 luni pe an)}
        Perioade comune: Trimestrial=4, Lunar=12, Săptămânal=52, Zilnic=7, Orar=24
    \end{exampleblock}
    \vspace{0.2cm}
    \begin{center}
        \includegraphics[width=0.95\textwidth, height=0.55\textheight, keepaspectratio]{charts/ch4_quiz1_seasonal_periods.pdf}
    \end{center}
\end{frame}

\begin{frame}{Întrebarea 2}
    \begin{alertblock}{Întrebare}
        Ce face operatorul de diferența sezonieră $(1 - L^{12})$ unei serii lunare?
    \end{alertblock}

    \vspace{0.3cm}

    \begin{enumerate}[(A)]
        \item Calculează $Y_t - Y_{t-1}$ (schimbarea luna-la-luna)
        \item Calculează $Y_t - Y_{t-12}$ (schimbarea an-la-an)
        \item Calculează media mobilă pe 12 luni
        \item Elimină doar componenta de trend
    \end{enumerate}
\end{frame}

\begin{frame}{Întrebarea 2: Răspuns}
    \begin{exampleblock}{Răspuns corect: (B) Schimbarea an-la-an}
        $(1 - L^{12})Y_t = Y_t - Y_{t-12}$ elimină tiparul sezonier prin compararea acelorasi luni.
    \end{exampleblock}
    \vspace{0.2cm}
    \begin{center}
        \includegraphics[width=0.95\textwidth, height=0.55\textheight, keepaspectratio]{charts/ch4_quiz2_seasonal_diff.pdf}
    \end{center}
\end{frame}

\begin{frame}{Întrebarea 3}
    \begin{alertblock}{Întrebare}
        În notația SARIMA$(1,1,1)\times(1,1,1)_{12}$, ce reprezintă partea $(1,1,1)_{12}$?
    \end{alertblock}

    \vspace{0.3cm}

    \begin{enumerate}[(A)]
        \item AR(1), o diferențiere, MA(1) la nivelul obișnuit
        \item AR sezonier(1), o diferențiere sezonieră, MA sezonier(1)
        \item 12 termeni AR, 12 diferențe, 12 termeni MA
        \item Modelul are 12 parametri în total
    \end{enumerate}
\end{frame}

\begin{frame}{Întrebarea 3: Răspuns}
    \begin{exampleblock}{Răspuns corect: (B)}
        AR sezonier(1), o diferențiere sezonieră, MA sezonier(1)
    \end{exampleblock}

    \begin{block}{Descompunerea notației SARIMA}
        SARIMA$(p,d,q)\times(P,D,Q)_s$:

        \vspace{0.2cm}
        \begin{tabular}{ll}
            $(p,d,q)$ & Non-sezonier: AR($p$), $d$ diferențe, MA($q$) \\
            $(P,D,Q)_s$ & Sezonier: SAR($P$), $D$ dif. sezoniere, SMA($Q$) \\
        \end{tabular}

        \vspace{0.3cm}
        Pentru $(1,1,1)\times(1,1,1)_{12}$:
        \begin{itemize}
            \item Non-sezonier: AR(1), o diferența obișnuită, MA(1)
            \item Sezonier: SAR(1) la lag-ul 12, un $\Delta_{12}$, SMA(1) la lag-ul 12
        \end{itemize}
    \end{block}
\end{frame}

\begin{frame}{Întrebarea 4}
    \begin{alertblock}{Întrebare}
        ``Modelul Airline'' este SARIMA$(0,1,1)\times(0,1,1)_{12}$. Câți parametri trebuie estimați (excluzand varianța)?
    \end{alertblock}

    \vspace{0.3cm}

    \begin{enumerate}[(A)]
        \item 1
        \item 2
        \item 4
        \item 12
    \end{enumerate}
\end{frame}

\begin{frame}{Întrebarea 4: Răspuns}
    \begin{exampleblock}{Răspuns corect: (B)}
        2 parametri
    \end{exampleblock}

    \begin{block}{Structura modelului}
        SARIMA$(0,1,1)\times(0,1,1)_{12}$:
        $$(1-L)(1-L^{12})Y_t = (1 + \theta_1 L)(1 + \Theta_1 L^{12})\varepsilon_t$$

        Parametri:
        \begin{itemize}
            \item $\theta_1$: coeficient MA non-sezonier
            \item $\Theta_1$: coeficient MA sezonier
        \end{itemize}

        Total: \textbf{2 parametri} (plus $\sigma^2$)
    \end{block}

    {\footnotesize
    \begin{alertblock}{De ce ``Modelul Airline''?}
        Box \& Jenkins (1970) au folosit acest model pentru a prognoză pasagerii companiilor aeriene internationale. Este remarcabil de eficient pentru multe serii economice sezoniere!
    \end{alertblock}
    }
\end{frame}

\begin{frame}{Întrebarea 5}
    \begin{alertblock}{Întrebare}
        Observați vârfuri ACF semnificâțive la lag-urile 12, 24 și 36 intr-o serie lunară. Ce sugerează aceasta?
    \end{alertblock}

    \vspace{0.3cm}

    \begin{enumerate}[(A)]
        \item Seria are o rădăcină unitară
        \item Seria are sezonalitate anuala care necesită diferențiere sezonieră
        \item Seria urmează un proces AR(36)
        \item Seria este deja staționară
    \end{enumerate}
\end{frame}

\begin{frame}{Întrebarea 5: Răspuns}
    \begin{exampleblock}{Răspuns corect: (B) Necesită diferențiere sezonieră}
        Vârfuri ACF la 12, 24, 36 = sezonalitate stochastică. Aplicați $(1 - L^{12})$ pentru a o elimină.
    \end{exampleblock}
    \vspace{0.2cm}
    \begin{center}
        \includegraphics[width=0.95\textwidth, height=0.55\textheight, keepaspectratio]{charts/ch4_quiz5_seasonal_acf.pdf}
    \end{center}
\end{frame}

\begin{frame}{Întrebarea 6}
    \begin{alertblock}{Întrebare}
        După aplicarea $(1-L)(1-L^{12})$ unei serii lunare, ACF prezintă un vârf semnificâțiv doar la lag-ul 1 și lag-ul 12. Ce model SARIMA este sugerat?
    \end{alertblock}

    \vspace{0.3cm}

    \begin{enumerate}[(A)]
        \item SARIMA$(1,1,0)\times(1,1,0)_{12}$
        \item SARIMA$(0,1,1)\times(0,1,1)_{12}$
        \item SARIMA$(1,1,1)\times(1,1,1)_{12}$
        \item SARIMA$(0,1,0)\times(0,1,0)_{12}$
    \end{enumerate}
\end{frame}

\begin{frame}{Întrebarea 6: Răspuns}
    \begin{exampleblock}{Răspuns corect: (B)}
        SARIMA$(0,1,1)\times(0,1,1)_{12}$ (Modelul Airline)
    \end{exampleblock}

    \begin{block}{Reguli de identificare ACF/PACF}
        Pentru procese MA, ACF \textbf{se opreste brusc} după lag-ul $q$:

        \vspace{0.2cm}
        \begin{tabular}{ll}
            \textbf{Tipar} & \textbf{Sugerează} \\
            \hline
            Vârf ACF doar la lag-ul 1 & MA(1) pentru partea non-sezonieră \\
            Vârf ACF doar la lag-ul 12 & SMA(1) pentru partea sezonieră \\
        \end{tabular}

        \vspace{0.2cm}
        Combinat: MA(1) $\times$ SMA(1) = $(0,d,1)\times(0,D,1)_{12}$

        Cu $d=1$ și $D=1$ (deja diferențiată): $(0,1,1)\times(0,1,1)_{12}$
    \end{block}
\end{frame}

\begin{frame}{Referinte}
    \begin{thebibliography}{9}
        \bibitem{boxjenkins} Box, G.E.P., Jenkins, G.M., Reinsel, G.C., \& Ljung, G.M. (2015). \textit{Time Series Analysis: Forecasting and Control}. 5th ed. Wiley.

        \bibitem{hyndman} Hyndman, R.J. \& Athanasopoulos, G. (2021). \textit{Forecasting: Principles and Practice}. 3rd ed. OTexts.

        \bibitem{hamilton} Hamilton, J.D. (1994). \textit{Time Series Analysis}. Princeton University Press.

        \bibitem{brockwell} Brockwell, P.J. \& Davis, R.A. (2016). \textit{Introduction to Time Series and Forecasting}. 3rd ed. Springer.
    \end{thebibliography}
\end{frame}

\end{document}
