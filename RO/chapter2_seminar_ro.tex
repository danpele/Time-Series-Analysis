% Capitolul 2: Seminar - Modele ARMA
% Prezentare academică de calitate Harvard
% Program de licență, Academia de Studii Economice din București

\documentclass[9pt, aspectratio=169, t]{beamer}

% Ensure content fits on slides
\setbeamersize{text margin left=8mm, text margin right=8mm}

%=============================================================================
% THEME AND STYLE CONFIGURATION
%=============================================================================
\usetheme{Madrid}
\usecolortheme{seahorse}

% IDA-Inspired Color Palette
\definecolor{MainBlue}{RGB}{26, 58, 110}
\definecolor{AccentBlue}{RGB}{42, 82, 140}
\definecolor{IDAred}{RGB}{220, 53, 69}
\definecolor{DarkGray}{RGB}{51, 51, 51}
\definecolor{MediumGray}{RGB}{128, 128, 128}
\definecolor{LightGray}{RGB}{248, 248, 248}
\definecolor{VeryLightGray}{RGB}{235, 235, 235}
\definecolor{Crimson}{RGB}{220, 53, 69}
\definecolor{Forest}{RGB}{46, 125, 50}
\definecolor{Amber}{RGB}{181, 133, 63}
\definecolor{Orange}{RGB}{230, 126, 34}
\definecolor{HarvardCrimson}{RGB}{165, 28, 48}

\setbeamercolor{palette primary}{bg=MainBlue, fg=white}
\setbeamercolor{palette secondary}{bg=MainBlue!85, fg=white}
\setbeamercolor{palette tertiary}{bg=MainBlue!70, fg=white}
\setbeamercolor{structure}{fg=MainBlue}
\setbeamercolor{title}{fg=MainBlue}
\setbeamercolor{frametitle}{fg=MainBlue, bg=white}
\setbeamercolor{block title}{bg=MainBlue, fg=white}
\setbeamercolor{block body}{bg=VeryLightGray, fg=DarkGray}
\setbeamercolor{block title alerted}{bg=Crimson, fg=white}
\setbeamercolor{block body alerted}{bg=Crimson!8, fg=DarkGray}
\setbeamercolor{block title example}{bg=Forest, fg=white}
\setbeamercolor{block body example}{bg=Forest!8, fg=DarkGray}
\setbeamercolor{item}{fg=MainBlue}

\setbeamertemplate{navigation symbols}{}

\setbeamertemplate{footline}{
    \leavevmode%
    \hbox{%
        \begin{beamercolorbox}[wd=.333333\paperwidth,ht=2.5ex,dp=1ex,center]{author in head/foot}%
            \usebeamerfont{author in head/foot}\insertshortauthor
        \end{beamercolorbox}%
        \begin{beamercolorbox}[wd=.333333\paperwidth,ht=2.5ex,dp=1ex,center]{title in head/foot}%
            \usebeamerfont{title in head/foot}\insertshorttitle
        \end{beamercolorbox}%
        \begin{beamercolorbox}[wd=.333333\paperwidth,ht=2.5ex,dp=1ex,right]{date in head/foot}%
            \usebeamerfont{date in head/foot}\insertshortdate{}\hspace*{2em}
            \insertframenumber{} / \inserttotalframenumber\hspace*{2ex}
        \end{beamercolorbox}}%
    \vskip0pt%
}

%=============================================================================
% PACKAGES
%=============================================================================
\usepackage[utf8]{inputenc}
\usepackage[T1]{fontenc}
\usepackage{amsmath, amssymb, amsthm}
\usepackage{mathtools}
\usepackage{bm}
\usepackage{tikz}
\usetikzlibrary{arrows.meta, positioning, shapes, calc, decorations.pathreplacing}
\usepackage{booktabs}
\usepackage{multirow}
\usepackage{array}
\usepackage{colortbl}
\usepackage{graphicx}
\usepackage{hyperref}
\hypersetup{colorlinks=false, pdfborder={0 0 0}}
\graphicspath{{../logos/}{../charts/}}

%=============================================================================
% CUSTOM COMMANDS
%=============================================================================
\newcommand{\E}{\mathbb{E}}
\newcommand{\Var}{\text{Var}}
\newcommand{\Cov}{\text{Cov}}
\newcommand{\Corr}{\text{Corr}}
\newcommand{\R}{\mathbb{R}}
\newcommand{\RMSE}{\text{RMSE}}
\newcommand{\MAE}{\text{MAE}}
\newcommand{\MAPE}{\text{MAPE}}

%=============================================================================
% TITLE INFORMATION
%=============================================================================
\title[Capitolul 2: Seminar]{Capitolul 2: Seminar --- Modele ARMA}
\subtitle{Program de licență, Facultatea de Cibernetică, Statistică și Informatică Economică, Academia de Studii Economice din București}
\author[Prof. dr. Daniel Traian Pele]{Prof. dr. Daniel Traian Pele\\[0.2cm]\footnotesize\texttt{danpele@ase.ro}}
\institute{Academia de Studii Economice din București}
\date{An Universitar 2025--2026}

\begin{document}

%=============================================================================
% TITLE SLIDE
%=============================================================================
\begin{frame}[plain]
    \begin{tikzpicture}[remember picture, overlay]
        \fill[IDAred] (current page.north west) rectangle ([yshift=-0.15cm]current page.north east);
        \node[anchor=north west] at ([xshift=0.5cm, yshift=-0.3cm]current page.north west) {
            \href{https://www.ase.ro}{\includegraphics[height=1.1cm]{ase_logo.png}}
        };
        \node[anchor=north] at ([yshift=-0.3cm]current page.north) {
            \href{https://ai4efin.ase.ro}{\includegraphics[height=1.1cm]{ai4efin_logo.png}}
        };
        \node[anchor=north east] at ([xshift=-0.5cm, yshift=-0.3cm]current page.north east) {
            \href{https://www.digital-finance-msca.com}{\includegraphics[height=1.1cm]{msca_logo.png}}
        };
    \end{tikzpicture}
    \vfill
    \begin{center}
        {\Large\textcolor{MediumGray}{Analiza și Prognoza Seriilor de Timp}}\\[0.3cm]
        {\Huge\textbf{\textcolor{MainBlue}{Capitolul 2: Modele ARMA}}}\\[0.5cm]
        {\Large\textcolor{IDAred}{Seminar}}
    \end{center}
    \vfill

    \begin{tikzpicture}[remember picture, overlay]
        \fill[IDAred] (current page.south west) rectangle ([yshift=0.15cm]current page.south east);
        \node[anchor=south west] at ([xshift=0.5cm, yshift=0.8cm]current page.south west) {
            \href{https://theida.net}{\includegraphics[height=0.9cm]{ida_logo.png}}
        };
        \node[anchor=south] at ([xshift=-3cm, yshift=0.8cm]current page.south) {
            \href{https://blockchain-research-center.com}{\includegraphics[height=0.9cm]{brc_logo.png}}
        };
        \node[anchor=south] at ([yshift=0.8cm]current page.south) {
            \href{https://quantinar.com}{\includegraphics[height=0.9cm]{qr_logo.png}}
        };
        \node[anchor=south] at ([xshift=3cm, yshift=0.8cm]current page.south) {
            \href{https://quantlet.com}{\includegraphics[height=0.9cm]{ql_logo.png}}
        };
        \node[anchor=south east] at ([xshift=-0.5cm, yshift=0.8cm]current page.south east) {
            \href{https://ipe.ro/new}{\includegraphics[height=0.9cm]{acad_logo.png}}
        };
    \end{tikzpicture}
\end{frame}

%=============================================================================
% OUTLINE
%=============================================================================
\begin{frame}{Cuprins Seminar}
    \tableofcontents
\end{frame}

%=============================================================================
% SECTION 1: MULTIPLE CHOICE QUIZ
%=============================================================================
\section{Test Grilă}

\begin{frame}{Test 1: Operatorul Lag}
    \begin{alertblock}{Întrebare}
        Care este rezultatul aplicării $(1-L)^2$ lui $X_t$?
    \end{alertblock}

    \vspace{0.5cm}
    \begin{enumerate}[A.]
        \item $X_t - X_{t-1}$
        \item $X_t - 2X_{t-1} + X_{t-2}$
        \item $X_t + X_{t-1} + X_{t-2}$
        \item $X_t - X_{t-2}$
    \end{enumerate}

    \vspace{0.5cm}
    \begin{center}
        \textit{Răspunsul pe slide-ul următor...}
    \end{center}
\end{frame}

\begin{frame}{Test 1: Soluție}
    \begin{exampleblock}{Răspuns: B -- $X_t - 2X_{t-1} + X_{t-2}$}
        \textbf{Explicație:}
        \begin{align*}
            (1-L)^2 X_t &= (1 - 2L + L^2)X_t \\
            &= X_t - 2LX_t + L^2X_t \\
            &= X_t - 2X_{t-1} + X_{t-2}
        \end{align*}

        Aceasta este \textbf{diferența de ordinul doi} a lui $X_t$.

        \textbf{Notă:} $(1-L)$ este operatorul de diferențiere de ordinul întâi, $(1-L)^2$ este diferența de ordinul doi.
    \end{exampleblock}
\end{frame}

\begin{frame}{Test 2: Staționaritatea AR(1)}
    \begin{alertblock}{Întrebare}
        Pentru ce valoare a lui $\phi$ procesul AR(1) $X_t = 0.5 + \phi X_{t-1} + \varepsilon_t$ este staționar?
    \end{alertblock}

    \vspace{0.5cm}
    \begin{enumerate}[A.]
        \item $\phi = 1.2$
        \item $\phi = 1.0$
        \item $\phi = -0.8$
        \item $\phi = -1.5$
    \end{enumerate}

    \vspace{0.5cm}
    \begin{center}
        \textit{Răspunsul pe slide-ul următor...}
    \end{center}
\end{frame}

\begin{frame}{Test 2: Soluție}
    \begin{exampleblock}{Răspuns: C -- $\phi = -0.8$ (Staționar)}
        \vspace{-0.2cm}
        \begin{center}
            \includegraphics[width=0.95\textwidth, height=0.62\textheight, keepaspectratio]{sem2_stationarity_region.pdf}
        \end{center}
        \vspace{-0.2cm}
        {\footnotesize
        \textbf{Staționaritate AR(1)}: $|\phi| < 1$ (rădăcina în afara cercului unitate). Doar C satisface: $|-0.8| = 0.8 < 1$
        }
    \end{exampleblock}
\end{frame}

\begin{frame}{Vizual: Comportamentul Procesului AR(1)}
    \begin{center}
        \includegraphics[width=0.95\textwidth]{ch2_def_ar1.pdf}
    \end{center}
    \vspace{-0.2cm}
    \small $\phi$ pozitiv: modele persistente, netede. $\phi$ negativ: comportament oscilant în jurul mediei.
\end{frame}

\begin{frame}{Test 3: Modelul ACF}
    \begin{alertblock}{Întrebare}
        Observați următorul model ACF: vârf semnificativ la lag 1, apoi toate celelalte lag-uri sunt în interiorul benzilor de încredere. PACF arată descreștere graduală. Ce model este sugerat?
    \end{alertblock}

    \vspace{0.5cm}
    \begin{enumerate}[A.]
        \item AR(1)
        \item MA(1)
        \item ARMA(1,1)
        \item Zgomot alb
    \end{enumerate}

    \vspace{0.5cm}
    \begin{center}
        \textit{Răspunsul pe slide-ul următor...}
    \end{center}
\end{frame}

\begin{frame}{Test 3: Soluție}
    \begin{exampleblock}{Răspuns: B -- MA(1)}
        \vspace{-0.2cm}
        \begin{center}
            \includegraphics[width=0.95\textwidth, height=0.62\textheight, keepaspectratio]{sem2_acf_pacf_patterns.pdf}
        \end{center}
        \vspace{-0.2cm}
        {\footnotesize
        \textbf{Model}: ACF se întrerupe după lag 1 $\Rightarrow$ MA(1); PACF descrește $\Rightarrow$ confirmă structura MA (nu AR)
        }
    \end{exampleblock}
\end{frame}

\begin{frame}{Vizual: Procesul MA(1) și ACF}
    \begin{center}
        \includegraphics[width=0.95\textwidth]{ch2_def_ma1.pdf}
    \end{center}
    \vspace{-0.2cm}
    \small Procesul MA(1) (stânga). Semnătura cheie: ACF se întrerupe brusc după lag 1 (dreapta).
\end{frame}

\begin{frame}{Test 4: Invertibilitatea MA}
    \begin{alertblock}{Întrebare}
        Pentru procesul MA(1) $X_t = \varepsilon_t + 1.5\varepsilon_{t-1}$, este procesul invertibil?
    \end{alertblock}

    \vspace{0.5cm}
    \begin{enumerate}[A.]
        \item Da, deoarece procesele MA sunt întotdeauna invertibile
        \item Da, deoarece $1.5 > 0$
        \item Nu, deoarece $|\theta| = 1.5 > 1$
        \item Nu, deoarece procesele MA nu sunt niciodată invertibile
    \end{enumerate}

    \vspace{0.5cm}
    \begin{center}
        \textit{Răspunsul pe slide-ul următor...}
    \end{center}
\end{frame}

\begin{frame}{Test 4: Soluție}
    \begin{exampleblock}{Răspuns: C -- Nu este invertibil ($|\theta| = 1.5 > 1$)}
        \vspace{-0.2cm}
        \begin{center}
            \includegraphics[width=0.95\textwidth, height=0.62\textheight, keepaspectratio]{sem2_invertibility.pdf}
        \end{center}
        \vspace{-0.2cm}
        {\footnotesize
        \textbf{Invertibilitate MA}: Rădăcina $z = -1/\theta$ trebuie să fie în afara cercului unitate $\Leftrightarrow |\theta| < 1$. Aici $z = -0.67$ este în interior!
        }
    \end{exampleblock}
\end{frame}

\begin{frame}{Vizual: Conceptul de Invertibilitate}
    \begin{center}
        \includegraphics[width=0.95\textwidth]{ch2_def_invertibility.pdf}
    \end{center}
    \vspace{-0.2cm}
    \small Stânga: invertibilitatea necesită rădăcini în afara cercului unitate. Dreapta: ponderile AR($\infty$) descresc doar când $|\theta| < 1$.
\end{frame}

\begin{frame}{Test 5: Reprezentarea ARMA}
    \begin{alertblock}{Întrebare}
        Forma compactă $\phi(L)X_t = \theta(L)\varepsilon_t$ reprezintă ce model?
    \end{alertblock}

    \vspace{0.5cm}
    \begin{enumerate}[A.]
        \item Model AR pur
        \item Model MA pur
        \item Model ARMA
        \item Niciunul dintre cele de mai sus
    \end{enumerate}

    \vspace{0.5cm}
    \begin{center}
        \textit{Răspunsul pe slide-ul următor...}
    \end{center}
\end{frame}

\begin{frame}{Test 5: Soluție}
    \begin{exampleblock}{Răspuns: C -- Model ARMA}
        \begin{itemize}
            \item $\phi(L) = 1 - \phi_1 L - \cdots - \phi_p L^p$ este polinomul AR
            \item $\theta(L) = 1 + \theta_1 L + \cdots + \theta_q L^q$ este polinomul MA
        \end{itemize}

        Ecuația $\phi(L)X_t = \theta(L)\varepsilon_t$ se expandează la:
        $$X_t - \phi_1 X_{t-1} - \cdots - \phi_p X_{t-p} = \varepsilon_t + \theta_1\varepsilon_{t-1} + \cdots + \theta_q\varepsilon_{t-q}$$

        Acesta este modelul general \textbf{ARMA(p,q)}.

        \textbf{Cazuri speciale:} $\theta(L) = 1$ (fără MA): AR pur; $\phi(L) = 1$ (fără AR): MA pur
    \end{exampleblock}
\end{frame}

\begin{frame}{Vizual: Procesul ARMA}
    \begin{center}
        \includegraphics[width=0.95\textwidth]{ch2_def_arma.pdf}
    \end{center}
    \vspace{-0.2cm}
    \small ARMA(1,1) combină componente AR și MA. ACF arată descreștere după lag-ul inițial.
\end{frame}

\begin{frame}{Test 6: Criterii Informaționale}
    \begin{alertblock}{Întrebare}
        Când comparăm ARMA(1,1) și ARMA(2,1) folosind BIC, care afirmație este corectă?
    \end{alertblock}

    \vspace{0.5cm}
    \begin{enumerate}[A.]
        \item BIC mai mic înseamnă întotdeauna prognoze mai bune
        \item BIC penalizează complexitatea mai puțin decât AIC
        \item Modelul cu BIC mai mic este preferat
        \item BIC poate compara doar modele cu același număr de parametri
    \end{enumerate}

    \vspace{0.5cm}
    \begin{center}
        \textit{Răspunsul pe slide-ul următor...}
    \end{center}
\end{frame}

\begin{frame}{Test 6: Soluție}
    \begin{exampleblock}{Răspuns: C -- BIC mai mic este preferat}
        \vspace{-0.2cm}
        \begin{center}
            \includegraphics[width=0.95\textwidth, height=0.62\textheight, keepaspectratio]{sem2_information_criteria.pdf}
        \end{center}
        \vspace{-0.2cm}
        {\footnotesize
        \textbf{AIC}: $-2\ln(\hat{L}) + 2k$ \quad \textbf{BIC}: $-2\ln(\hat{L}) + k\ln(n)$ \quad BIC penalizează complexitatea mai mult $\Rightarrow$ modele mai simple
        }
    \end{exampleblock}
\end{frame}

\begin{frame}{Test 7: Testul Ljung-Box}
    \begin{alertblock}{Întrebare}
        După ajustarea unui model ARMA(2,1), rulați testul Ljung-Box pe reziduuri și obțineți valoare-p = 0.02. Ce concluzie trageți?
    \end{alertblock}

    \vspace{0.5cm}
    \begin{enumerate}[A.]
        \item Modelul este adecvat
        \item Reziduurile sunt zgomot alb
        \item Există autocorelație semnificativă în reziduuri
        \item Modelul are prea mulți parametri
    \end{enumerate}

    \vspace{0.5cm}
    \begin{center}
        \textit{Răspunsul pe slide-ul următor...}
    \end{center}
\end{frame}

\begin{frame}{Test 7: Soluție}
    \begin{exampleblock}{Răspuns: C -- Există autocorelație semnificativă în reziduuri}
        Testul Ljung-Box are:
        \begin{itemize}
            \item $H_0$: Reziduurile sunt zgomot alb (fără autocorelație)
            \item $H_1$: Reziduurile au autocorelație semnificativă
        \end{itemize}

        Cu valoare-p = 0.02 $<$ 0.05:
        \begin{itemize}
            \item \textbf{Respingem} $H_0$
            \item Concluzie: reziduurile \textbf{nu} sunt zgomot alb
            \item Modelul este \textbf{inadecvat} --- structură semnificativă rămâne
        \end{itemize}

        \textbf{Pasul următor:} Încercați un model diferit (de exemplu, creșteți $p$ sau $q$)
    \end{exampleblock}
\end{frame}

\begin{frame}{Test 8: Prognoză}
    \begin{alertblock}{Întrebare}
        Pentru un model AR(1) cu $\phi = 0.6$ și medie $\mu = 10$, ce se întâmplă cu prognozele când orizontul $h \to \infty$?
    \end{alertblock}

    \vspace{0.5cm}
    \begin{enumerate}[A.]
        \item Prognozele cresc fără limită
        \item Prognozele converg la 0
        \item Prognozele converg la $\mu = 10$
        \item Prognozele oscilează pentru totdeauna
    \end{enumerate}

    \vspace{0.5cm}
    \begin{center}
        \textit{Răspunsul pe slide-ul următor...}
    \end{center}
\end{frame}

\begin{frame}{Test 8: Soluție}
    \begin{exampleblock}{Răspuns: C -- Prognozele converg la $\mu = 10$}
        Pentru AR(1), prognoza la $h$ pași înainte este:
        $$\hat{X}_{n+h|n} = \mu + \phi^h(X_n - \mu)$$

        Deoarece $|\phi| = 0.6 < 1$:
        $$\lim_{h \to \infty} \phi^h = 0$$

        Prin urmare:
        $$\lim_{h \to \infty} \hat{X}_{n+h|n} = \mu + 0 \cdot (X_n - \mu) = \mu = 10$$

        \textbf{Observație cheie:} Prognozele pe termen lung din modele ARMA staționare converg întotdeauna la media necondiționată.
    \end{exampleblock}
\end{frame}

\begin{frame}{Test 9: Rădăcinile AR(2)}
    \begin{alertblock}{Întrebare}
        Un proces AR(2) are rădăcinile caracteristice $z_1 = 0.8$ și $z_2 = -0.5$. Este staționar?
    \end{alertblock}

    \vspace{0.5cm}
    \begin{enumerate}[A.]
        \item Da, deoarece ambele rădăcini sunt în interiorul cercului unitate
        \item Nu, deoarece o rădăcină este negativă
        \item Nu, deoarece rădăcinile trebuie să fie în afara cercului unitate
        \item Nu se poate determina fără mai multe informații
    \end{enumerate}

    \vspace{0.5cm}
    \begin{center}
        \textit{Răspunsul pe slide-ul următor...}
    \end{center}
\end{frame}

\begin{frame}{Test 9: Soluție}
    \begin{exampleblock}{Răspuns: C -- Rădăcinile trebuie să fie în afara cercului unitate}
        Pentru staționaritatea AR, rădăcinile lui $\phi(z) = 0$ trebuie să fie \textbf{în afara} cercului unitate, adică $|z| > 1$.

        Aici: $|z_1| = 0.8 < 1$ și $|z_2| = 0.5 < 1$ -- ambele \textbf{în interior} cercului unitate.

        $\rightarrow$ \textcolor{Crimson}{\textbf{Nestaționar}} (de fapt exploziv)

        \textbf{Notă:} Condiție echivalentă: coeficienții $\phi_1, \phi_2$ trebuie să satisfacă triunghiul de staționaritate.
    \end{exampleblock}
\end{frame}

\begin{frame}{Test 10: Proprietățile MA(q)}
    \begin{alertblock}{Întrebare}
        Pentru un proces MA(2), ACF-ul:
    \end{alertblock}

    \vspace{0.5cm}
    \begin{enumerate}[A.]
        \item Descrește exponențial
        \item Se întrerupe după lag 2
        \item Se întrerupe după lag 1
        \item Nu se întrerupe niciodată
    \end{enumerate}

    \vspace{0.5cm}
    \begin{center}
        \textit{Răspunsul pe slide-ul următor...}
    \end{center}
\end{frame}

\begin{frame}{Test 10: Soluție}
    \begin{exampleblock}{Răspuns: B -- Se întrerupe după lag 2}
        Pentru MA($q$), ACF-ul este exact zero pentru lag-uri $> q$.

        \begin{itemize}
            \item MA(1): ACF se întrerupe după lag 1
            \item MA(2): ACF se întrerupe după lag 2
            \item MA($q$): ACF se întrerupe după lag $q$
        \end{itemize}

        Aceasta este caracteristica cheie de identificare: întreruperea ACF $\Rightarrow$ ordinul MA.

        Între timp, PACF-ul proceselor MA descrește (nu se întrerupe).
    \end{exampleblock}
\end{frame}

\begin{frame}{Test 11: Parsimonia ARMA}
    \begin{alertblock}{Întrebare}
        De ce ar putea fi preferat ARMA(1,1) față de AR(5) chiar dacă ambele se potrivesc la fel de bine?
    \end{alertblock}

    \vspace{0.5cm}
    \begin{enumerate}[A.]
        \item Modelele ARMA sunt întotdeauna mai bune
        \item Mai puțini parametri reduc riscul de supraajustare
        \item Modelele AR nu pot captura trenduri
        \item Componentele MA sunt mai stabile
    \end{enumerate}

    \vspace{0.5cm}
    \begin{center}
        \textit{Răspunsul pe slide-ul următor...}
    \end{center}
\end{frame}

\begin{frame}{Test 11: Soluție}
    \begin{exampleblock}{Răspuns: B -- Mai puțini parametri reduc riscul de supraajustare}
        \textbf{Principiul parsimoniei}: preferați modele mai simple.

        \begin{itemize}
            \item ARMA(1,1): 2 parametri ($\phi_1$, $\theta_1$)
            \item AR(5): 5 parametri ($\phi_1, \ldots, \phi_5$)
        \end{itemize}

        Mai puțini parametri înseamnă:
        \begin{itemize}
            \item Risc mai mic de supraajustare
            \item Prognoze mai bune în afara eșantionului
            \item Model mai interpretabil
        \end{itemize}

        BIC penalizează complexitatea mai mult decât AIC, selectând adesea modele mai simple.
    \end{exampleblock}
\end{frame}

\begin{frame}{Test 12: Diagnosticul Reziduurilor}
    \begin{alertblock}{Întrebare}
        După ajustarea unui model ARMA, ACF-ul reziduurilor arată un vârf semnificativ la lag 5. Aceasta sugerează:
    \end{alertblock}

    \vspace{0.5cm}
    \begin{enumerate}[A.]
        \item Modelul este adecvat
        \item Modelul ar putea avea nevoie de termeni de ordin mai mare
        \item Reziduurile sunt zgomot alb
        \item Datele sunt nestaționare
    \end{enumerate}

    \vspace{0.5cm}
    \begin{center}
        \textit{Răspunsul pe slide-ul următor...}
    \end{center}
\end{frame}

\begin{frame}{Test 12: Soluție}
    \begin{exampleblock}{Răspuns: B -- Modelul ar putea avea nevoie de termeni de ordin mai mare}
        Reziduurile bune ar trebui să fie zgomot alb fără ACF semnificativ.

        Un vârf semnificativ la lag 5 indică structură de autocorelație rămasă necaptată de model.

        \textbf{Acțiuni:}
        \begin{itemize}
            \item Luați în considerare adăugarea termenilor AR sau MA
            \item Verificați dacă componenta AR(5) sau MA(5) ajută
            \item Rulați din nou testul Ljung-Box după modificare
        \end{itemize}
    \end{exampleblock}
\end{frame}

\begin{frame}{Test 13: Descompunerea Wold}
    \begin{alertblock}{Întrebare}
        Teorema descompunerii Wold afirmă că orice proces staționar poate fi scris ca:
    \end{alertblock}

    \vspace{0.5cm}
    \begin{enumerate}[A.]
        \item Un proces AR finit
        \item Un proces MA finit
        \item Un proces MA infinit plus o componentă deterministă
        \item Un proces ARIMA
    \end{enumerate}

    \vspace{0.5cm}
    \begin{center}
        \textit{Răspunsul pe slide-ul următor...}
    \end{center}
\end{frame}

\begin{frame}{Test 13: Soluție}
    \begin{exampleblock}{Răspuns: C -- Un proces MA infinit plus o componentă deterministă}
        Teorema lui Wold: Orice proces staționar poate fi scris ca:
        $$X_t = \sum_{j=0}^{\infty} \psi_j \varepsilon_{t-j} + \eta_t$$
        unde $\eta_t$ este deterministic și $\sum \psi_j^2 < \infty$.

        \textbf{Implicație:} MA($\infty$) este reprezentarea cea mai generală. Modelele ARMA sunt aproximări eficiente ale acestui MA infinit.
    \end{exampleblock}
\end{frame}

\begin{frame}{Test 14: Rădăcină Unitară vs Trend Staționar}
    \begin{alertblock}{Întrebare}
        Cum faceți un proces cu rădăcină unitară să devină staționar?
    \end{alertblock}

    \vspace{0.5cm}
    \begin{enumerate}[A.]
        \item Scădeți un trend liniar
        \item Luați diferențe de ordinul întâi
        \item Aplicați media mobilă
        \item Folosiți ajustare sezonieră
    \end{enumerate}

    \vspace{0.5cm}
    \begin{center}
        \textit{Răspunsul pe slide-ul următor...}
    \end{center}
\end{frame}

\begin{frame}{Test 14: Soluție}
    \begin{exampleblock}{Răspuns: B -- Luați diferențe de ordinul întâi}
        \begin{itemize}
            \item \textbf{Rădăcină unitară} (trend stochastic): Folosiți \textbf{diferențierea}
            \item \textbf{Trend staționar} (trend determinist): Folosiți \textbf{regresia} pentru eliminarea trendului
        \end{itemize}

        Pentru mersul aleatoriu $X_t = X_{t-1} + \varepsilon_t$:
        $$\Delta X_t = X_t - X_{t-1} = \varepsilon_t$$
        care este zgomot alb staționar.
    \end{exampleblock}
\end{frame}

%=============================================================================
% SECTION 2: TRUE/FALSE
%=============================================================================
\section{Întrebări Adevărat/Fals}

\begin{frame}{Întrebări Adevărat/Fals}
    \begin{alertblock}{Întrebare}
        Determinați dacă fiecare afirmație este Adevărată sau Falsă:
    \end{alertblock}

    \vspace{0.3cm}
    \begin{enumerate}
        \item Un proces AR(2) poate prezenta comportament pseudo-ciclic.
        \item Procesele MA necesită o condiție de staționaritate.
        \item PACF-ul unui proces AR(p) se întrerupe după lag $p$.
        \item Dacă AIC selectează ARMA(2,1) și BIC selectează ARMA(1,1), nu pot fi ambele corecte.
        \item Intervalele de încredere ale prognozei se îngustează pe măsură ce orizontul de prognoză crește.
        \item Ecuațiile Yule-Walker pot fi folosite pentru a estima parametrii MA.
    \end{enumerate}

    \vspace{0.3cm}
    \begin{center}
        \textit{Răspunsul pe slide-ul următor...}
    \end{center}
\end{frame}

\begin{frame}{Adevărat/Fals: Soluții}
    \begin{exampleblock}{Răspunsuri}
    \begin{enumerate}
        \item \textcolor{Forest}{\textbf{ADEVĂRAT}}: AR(2) cu rădăcini complexe arată oscilații amortizate
        \item \textcolor{Crimson}{\textbf{FALS}}: Procesele MA sunt întotdeauna staționare; au nevoie de condiția de \textit{invertibilitate}
        \item \textcolor{Forest}{\textbf{ADEVĂRAT}}: Aceasta este caracteristica cheie de identificare a AR(p)
        \item \textcolor{Crimson}{\textbf{FALS}}: Ambele pot fi „corecte" --- optimizează criterii diferite (AIC favorizează potrivirea, BIC favorizează parsimonia)
        \item \textcolor{Crimson}{\textbf{FALS}}: Intervalele de încredere se \textit{lărgesc} pe măsură ce orizontul crește (mai multă incertitudine)
        \item \textcolor{Crimson}{\textbf{FALS}}: Yule-Walker este doar pentru modele AR; MA folosește MLE
    \end{enumerate}
    \end{exampleblock}
\end{frame}

%=============================================================================
% SECTION 3: CALCULATION EXERCISES
%=============================================================================
\section{Exerciții de Calcul}

\begin{frame}{Exercițiu 1: Proprietățile AR(1)}
    \textbf{Problemă:} Considerați procesul AR(1):
    $$X_t = 2 + 0.7 X_{t-1} + \varepsilon_t, \quad \varepsilon_t \sim WN(0, 9)$$

    Calculați:
    \begin{enumerate}
        \item Media $\mu$
        \item Varianța $\gamma(0)$
        \item Autocovarianța $\gamma(1)$ și $\gamma(2)$
        \item Autocorelația $\rho(1)$ și $\rho(2)$
    \end{enumerate}
\end{frame}

\begin{frame}{Exercițiu 1: Soluție}
    Dat: $c = 2$, $\phi = 0.7$, $\sigma^2 = 9$

    \vspace{0.3cm}
    \textbf{1. Media:}
    $$\mu = \frac{c}{1-\phi} = \frac{2}{1-0.7} = \frac{2}{0.3} = 6.67$$

    \textbf{2. Varianța:}
    $$\gamma(0) = \frac{\sigma^2}{1-\phi^2} = \frac{9}{1-0.49} = \frac{9}{0.51} = 17.65$$

    \textbf{3. Autocovarianța:}
    $$\gamma(1) = \phi \cdot \gamma(0) = 0.7 \times 17.65 = 12.35$$
    $$\gamma(2) = \phi^2 \cdot \gamma(0) = 0.49 \times 17.65 = 8.65$$

    \textbf{4. Autocorelația:}
    $$\rho(1) = \phi = 0.7, \quad \rho(2) = \phi^2 = 0.49$$
\end{frame}

\begin{frame}{Exercițiu 2: Proprietățile MA(1)}
    \textbf{Problemă:} Considerați procesul MA(1):
    $$X_t = 5 + \varepsilon_t - 0.4\varepsilon_{t-1}, \quad \varepsilon_t \sim WN(0, 4)$$

    Calculați:
    \begin{enumerate}
        \item Media $\mu$
        \item Varianța $\gamma(0)$
        \item Autocovarianța $\gamma(1)$
        \item Autocorelația $\rho(1)$
        \item Este acest proces invertibil?
    \end{enumerate}
\end{frame}

\begin{frame}{Exercițiu 2: Soluție}
    Dat: $\mu = 5$, $\theta = -0.4$, $\sigma^2 = 4$

    \vspace{0.3cm}
    \textbf{1. Media:}
    $$\E[X_t] = \mu = 5$$

    \textbf{2. Varianța:}
    $$\gamma(0) = \sigma^2(1 + \theta^2) = 4(1 + 0.16) = 4 \times 1.16 = 4.64$$

    \textbf{3. Autocovarianța la lag 1:}
    $$\gamma(1) = \theta\sigma^2 = -0.4 \times 4 = -1.6$$

    \textbf{4. Autocorelația:}
    $$\rho(1) = \frac{\gamma(1)}{\gamma(0)} = \frac{-1.6}{4.64} = -0.345$$

    \textbf{5. Invertibilitate:} $|\theta| = 0.4 < 1$ $\rightarrow$ \textcolor{Forest}{\textbf{Da, invertibil}}
\end{frame}

\begin{frame}{Exercițiu 3: Rădăcinile Caracteristice}
    \textbf{Problemă:} Considerați procesul AR(2):
    $$X_t = 0.5X_{t-1} + 0.3X_{t-2} + \varepsilon_t$$

    \begin{enumerate}
        \item Scrieți ecuația caracteristică
        \item Găsiți rădăcinile caracteristice
        \item Este acest proces staționar?
    \end{enumerate}
\end{frame}

\begin{frame}{Exercițiu 3: Soluție}
    \textbf{1. Ecuația caracteristică:}
    $$\phi(z) = 1 - \phi_1 z - \phi_2 z^2 = 1 - 0.5z - 0.3z^2 = 0$$

    Sau: $0.3z^2 + 0.5z - 1 = 0$

    \vspace{0.3cm}
    \textbf{2. Rădăcinile (folosind formula cuadratică):}
    $$z = \frac{-0.5 \pm \sqrt{0.25 + 1.2}}{0.6} = \frac{-0.5 \pm 1.204}{0.6}$$

    $$z_1 = \frac{0.704}{0.6} = 1.17, \quad z_2 = \frac{-1.704}{0.6} = -2.84$$

    \vspace{0.3cm}
    \textbf{3. Verificarea staționarității:}

    Ambele rădăcini au $|z| > 1$: $|z_1| = 1.17 > 1$ și $|z_2| = 2.84 > 1$

    $\rightarrow$ \textcolor{Forest}{\textbf{Staționar}} (rădăcini în afara cercului unitate)
\end{frame}

\begin{frame}{Exercițiu 4: Prognoză}
    \textbf{Problemă:} Ați ajustat un model AR(1):
    $$X_t = 3 + 0.8X_{t-1} + \varepsilon_t, \quad \sigma^2 = 4$$

    Dat $X_{100} = 20$, calculați:
    \begin{enumerate}
        \item Prognoza la 1 pas înainte $\hat{X}_{101|100}$
        \item Prognoza la 2 pași înainte $\hat{X}_{102|100}$
        \item Prognoza pe termen lung $\hat{X}_{100+h|100}$ când $h \to \infty$
        \item Intervalul de încredere de 95\% pentru $\hat{X}_{101|100}$
    \end{enumerate}
\end{frame}

\begin{frame}{Exercițiu 4: Soluție}
    Dat: $c = 3$, $\phi = 0.8$, $\sigma^2 = 4$, $X_{100} = 20$

    \textbf{Media:} $\mu = \frac{3}{1-0.8} = 15$

    \vspace{0.3cm}
    \textbf{1. Prognoza la un pas:}
    $$\hat{X}_{101|100} = c + \phi X_{100} = 3 + 0.8 \times 20 = 19$$

    \textbf{2. Prognoza la doi pași:}
    $$\hat{X}_{102|100} = c + \phi \hat{X}_{101|100} = 3 + 0.8 \times 19 = 18.2$$

    \textbf{3. Prognoza pe termen lung:}
    $$\lim_{h \to \infty} \hat{X}_{100+h|100} = \mu = 15$$

    \textbf{4. IC 95\% pentru 1 pas:}
    $$\text{MSFE}(1) = \sigma^2 = 4, \quad \sqrt{\text{MSFE}(1)} = 2$$
    $$IC: 19 \pm 1.96 \times 2 = [15.08, 22.92]$$
\end{frame}

%=============================================================================
% SECTION 4: PYTHON EXERCISES
%=============================================================================
\section{Exerciții Python}

\begin{frame}{Exercițiu Python 1: Simulare și Ajustare AR(1)}
    \textbf{Sarcină:}
    \begin{enumerate}
        \item Simulați 500 de observații dintr-un AR(1) cu $\phi = 0.7$
        \item Reprezentați grafic seria și ACF/PACF
        \item Ajustați un model AR(1) și verificați dacă $\hat{\phi} \approx 0.7$
        \item Examinați diagnosticele reziduurilor
    \end{enumerate}

    \vspace{0.3cm}
    \textbf{Cod indiciu:}

    \texttt{np.random.seed(42)}

    \texttt{n = 500}

    \texttt{phi = 0.7}

    \texttt{x = np.zeros(n)}

    \texttt{for t in range(1, n):}

    \texttt{\quad x[t] = phi * x[t-1] + np.random.randn()}
\end{frame}

\begin{frame}{Exercițiu Python 2: Selectarea Modelului}
    \textbf{Sarcină:}
    \begin{enumerate}
        \item Încărcați o serie de timp reală (de exemplu, randamente de acțiuni)
        \item Verificați staționaritatea folosind testul ADF
        \item Comparați AIC/BIC pentru ARMA(1,0), ARMA(0,1), ARMA(1,1), ARMA(2,1)
        \item Selectați cel mai bun model
        \item Generați prognoze cu intervale de încredere
    \end{enumerate}

    \vspace{0.3cm}
    \textbf{Funcții cheie:}
    \begin{itemize}
        \item \texttt{adfuller()} pentru testul de staționaritate
        \item \texttt{ARIMA(data, order=(p,0,q)).fit()} pentru ajustare
        \item \texttt{results.aic}, \texttt{results.bic} pentru criterii
        \item \texttt{results.get\_forecast(h)} pentru predicții
    \end{itemize}
\end{frame}

\begin{frame}{Exercițiu Python 3: Verificarea Diagnosticelor}
    \textbf{Sarcină:} După ajustarea unui model, efectuați diagnostice complete:
    \begin{enumerate}
        \item Reprezentați grafic reziduurile în timp
        \item Reprezentați grafic ACF-ul reziduurilor
        \item Creați graficul Q-Q
        \item Rulați testul Ljung-Box
        \item Verificați dacă rădăcinile AR/MA sunt în afara cercului unitate
    \end{enumerate}

    \vspace{0.3cm}
    \textbf{Funcții cheie:}
    \begin{itemize}
        \item \texttt{results.resid} pentru reziduuri
        \item \texttt{plot\_acf(resid)} pentru graficul ACF
        \item \texttt{stats.probplot(resid)} pentru graficul Q-Q
        \item \texttt{acorr\_ljungbox(resid)} pentru testul portmanteau
        \item \texttt{results.arroots}, \texttt{results.maroots} pentru rădăcini
    \end{itemize}
\end{frame}

%=============================================================================
% SECTION 5: REAL DATA ANALYSIS
%=============================================================================
\section{Analiză pe Date Reale}

\begin{frame}{Studiu de Caz: Indicele Producției Industriale}
    \vspace{-0.3cm}
    \begin{center}
        \includegraphics[width=0.85\textwidth, height=0.55\textheight, keepaspectratio]{arma_examples.pdf}
    \end{center}
    \vspace{-0.2cm}
    {\footnotesize
    \begin{itemize}
        \item Producția industrială SUA: date lunare, deja staționare (rate de creștere)
        \item Arată modele ARMA tipice: revenire la medie cu dependență pe termen scurt
        \item Gruparea volatilității vizibilă -- ARMA captează media condiționată
        \item Potrivit pentru modelarea ARMA fără diferențiere
    \end{itemize}
    }
\end{frame}

\begin{frame}{Recunoașterea Modelului ACF/PACF}
    \vspace{-0.3cm}
    \begin{center}
        \includegraphics[width=0.85\textwidth, height=0.55\textheight, keepaspectratio]{acf_pacf_examples.pdf}
    \end{center}
    \vspace{-0.2cm}
    {\footnotesize
    \begin{itemize}
        \item ACF arată descreștere graduală -- sugerează componentă AR
        \item PACF se întrerupe după lag 2 -- sugerează că AR(2) ar putea fi potrivit
        \item Unele lag-uri semnificative în ACF dincolo de lag 2 -- termenii MA ar putea ajuta
        \item Model consistent cu ARMA(2,1) sau modele similare de ordin mic
    \end{itemize}
    }
\end{frame}

\begin{frame}{Rezultate Estimare ARMA}
    {\small
    \begin{block}{Model: ARMA(2,1) pentru Creșterea Producției Industriale}
        \begin{center}
        \begin{tabular}{lcccc}
            \toprule
            \textbf{Parametru} & \textbf{Estimat} & \textbf{Eroare Std.} & \textbf{z-stat} & \textbf{valoare-p} \\
            \midrule
            $c$ (const) & $0.156$ & $0.048$ & $3.25$ & $0.001$ \\
            $\phi_1$ (AR.L1) & $0.423$ & $0.089$ & $4.75$ & $<0.001$ \\
            $\phi_2$ (AR.L2) & $0.187$ & $0.072$ & $2.60$ & $0.009$ \\
            $\theta_1$ (MA.L1) & $-0.156$ & $0.091$ & $-1.71$ & $0.087$ \\
            \bottomrule
        \end{tabular}
        \end{center}
    \end{block}

    \vspace{0.2cm}

    \begin{exampleblock}{Selecția Modelului}
        AIC: $-412.5$, BIC: $-398.2$. Modelul trece verificările de staționaritate și invertibilitate.
    \end{exampleblock}
    }
\end{frame}

\begin{frame}{Performanța Prognozei}
    \vspace{-0.3cm}
    \begin{center}
        \includegraphics[width=0.85\textwidth, height=0.55\textheight, keepaspectratio]{arma_forecast.pdf}
    \end{center}
    \vspace{-0.2cm}
    {\footnotesize
    \begin{itemize}
        \item Prognozele ARMA revin la media necondiționată
        \item Prognozele pe termen scurt captează dinamica recentă
        \item Intervalele de încredere se extind cu orizontul de prognoză
        \item Comparația cu prognoza naivă arată îmbunătățirea ARMA
    \end{itemize}
    }
\end{frame}

\begin{frame}{Diagnosticele Modelului}
    \vspace{-0.3cm}
    \begin{center}
        \includegraphics[width=0.85\textwidth, height=0.55\textheight, keepaspectratio]{arma_residual_diagnostics.pdf}
    \end{center}
    \vspace{-0.2cm}
    {\footnotesize
    \begin{itemize}
        \item Reziduurile par aleatoare fără modele sistematice
        \item ACF-ul reziduurilor în limitele de încredere
        \item Graficul Q-Q arată normalitate aproximativă
        \item Testul Ljung-Box: $p > 0.05$ -- fără autocorelație semnificativă în reziduuri
    \end{itemize}
    }
\end{frame}

%=============================================================================
% SECTION 6: DISCUSSION
%=============================================================================
\section{Întrebări de Discuție}

\begin{frame}{Discuție 1: Selecția Modelului}
    \textbf{Scenariu:} Modelați rate de inflație lunare. După verificarea staționarității (trecută), găsiți:
    \begin{itemize}
        \item ACF: semnificativ la lag-urile 1, 2, 3, apoi descrește
        \item PACF: semnificativ la lag-urile 1, 2, apoi se întrerupe
        \item AIC selectează ARMA(2,3)
        \item BIC selectează ARMA(2,0) = AR(2)
    \end{itemize}

    \vspace{0.3cm}
    \textbf{Întrebări:}
    \begin{enumerate}
        \item Ce sugerează modelul ACF/PACF?
        \item De ce nu sunt de acord AIC și BIC?
        \item Ce model ați alege și de ce?
        \item Ce verificări suplimentare ați efectua?
    \end{enumerate}
\end{frame}

\begin{frame}{Discuție 2: Evaluarea Prognozei}
    \textbf{Scenariu:} Ajustați un model ARMA(1,1) pe randamente zilnice de acțiuni. Ajustarea în eșantion arată bine (valoare-p Ljung-Box = 0.45), dar RMSE în afara eșantionului este mai rău decât o prognoză simplă de mers aleatoriu.

    \vspace{0.3cm}
    \textbf{Întrebări:}
    \begin{enumerate}
        \item Este aceasta surprinzător? De ce sau de ce nu?
        \item Ce ne spune aceasta despre predictibilitatea randamentelor de acțiuni?
        \item Ar trebui să concluzionați că modelul ARMA este inutil?
        \item Ce alternative ați putea considera?
    \end{enumerate}

    \vspace{0.3cm}
    \textbf{Indiciu:} Gândiți-vă la Ipoteza Pieței Eficiente și la ce captează ARMA vs ce nu captează (de exemplu, gruparea volatilității).
\end{frame}

\begin{frame}{Discuție 3: Aplicație în Lumea Reală}
    \textbf{Scenariu:} Un economist de la banca centrală vă cere să prognozați creșterea trimestrială a PIB-ului pentru planificarea politicii.

    \vspace{0.3cm}
    \textbf{Întrebări:}
    \begin{enumerate}
        \item Ce analiză preliminară ați face înainte de a ajusta ARMA?
        \item PIB-ul este adesea nestaționar --- cum ați gestiona aceasta?
        \item Ați folosi AIC sau BIC pentru selecția modelului? De ce?
        \item Cum ați comunica incertitudinea prognozei factorilor de decizie?
        \item Ce limitări ale modelelor ARMA ar trebui să menționați?
    \end{enumerate}
\end{frame}

%=============================================================================
% SUMMARY
%=============================================================================
\section{Rezumat}

\begin{frame}{Concluzii Cheie din Seminarul de Astăzi}
    \begin{enumerate}
        \item \textbf{Modele AR:} Valoarea curentă depinde de valorile trecute
        \begin{itemize}
            \item Staționaritate: $|\phi| < 1$ pentru AR(1)
            \item PACF se întrerupe la lag $p$
        \end{itemize}

        \item \textbf{Modele MA:} Valoarea curentă depinde de șocurile trecute
        \begin{itemize}
            \item Întotdeauna staționare; invertibilitate: $|\theta| < 1$ pentru MA(1)
            \item ACF se întrerupe la lag $q$
        \end{itemize}

        \item \textbf{Selecția modelului:} Folosiți modelele ACF/PACF + criterii informaționale

        \item \textbf{Diagnostice:} Reziduurile trebuie să fie zgomot alb (testul Ljung-Box)

        \item \textbf{Prognoză:} Prognozele punctuale converg la medie; incertitudinea crește
    \end{enumerate}

    \vspace{0.3cm}
    \textbf{Următorul Seminar:} ARIMA și Modele Sezoniere
\end{frame}

%=============================================================================
% REFERINȚE
%=============================================================================
\begin{frame}{Referințe}
    \footnotesize
    \begin{thebibliography}{99}
        \bibitem{box2015} Box, G.E.P., Jenkins, G.M., Reinsel, G.C., \& Ljung, G.M. (2015). \textit{Time Series Analysis: Forecasting and Control}. 5th ed., Wiley.
        \bibitem{hamilton1994} Hamilton, J.D. (1994). \textit{Time Series Analysis}. Princeton University Press.
        \bibitem{hyndman2021} Hyndman, R.J., \& Athanasopoulos, G. (2021). \textit{Forecasting: Principles and Practice}. 3rd ed., OTexts.
        \bibitem{brockwell2016} Brockwell, P.J., \& Davis, R.A. (2016). \textit{Introduction to Time Series and Forecasting}. 3rd ed., Springer.
    \end{thebibliography}
\end{frame}

\begin{frame}{Surse de Date și Software}
    \textbf{Instrumente Software:}
    \begin{itemize}
        \item \texttt{statsmodels} -- Modele ARIMA pentru Python
        \item \texttt{pmdarima} -- Selecție automată ARIMA
        \item \texttt{pandas} -- Manipulare date serii de timp
        \item \texttt{matplotlib} -- Vizualizare
    \end{itemize}

    \vspace{0.3cm}
    \textbf{Date și Exemple:}
    \begin{itemize}
        \item Procese AR, MA și ARMA simulate
        \item Exemple bazate pe Hyndman \& Athanasopoulos (2021)
    \end{itemize}
\end{frame}

\begin{frame}{}
    \centering
    \Huge\textcolor{MainBlue}{Vă mulțumesc!}

    \vspace{1cm}

    \Large Întrebări?
\end{frame}

\end{document}
