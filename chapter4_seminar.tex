% Chapter 4: Seminar - SARIMA Models
% Quizzes, Practice Problems, and Discussion
% Bachelor program, Bucharest University of Economic Studies

\documentclass[9pt, aspectratio=169, t]{beamer}

% Ensure content fits on slides
\setbeamersize{text margin left=8mm, text margin right=8mm}

%=============================================================================
% THEME AND STYLE CONFIGURATION
%=============================================================================
\usetheme{Madrid}
\usecolortheme{seahorse}

% IDA-Inspired Color Palette
\definecolor{MainBlue}{RGB}{26, 58, 110}
\definecolor{AccentBlue}{RGB}{42, 82, 140}
\definecolor{IDAred}{RGB}{220, 53, 69}
\definecolor{DarkGray}{RGB}{51, 51, 51}
\definecolor{MediumGray}{RGB}{128, 128, 128}
\definecolor{LightGray}{RGB}{248, 248, 248}
\definecolor{VeryLightGray}{RGB}{235, 235, 235}
\definecolor{Crimson}{RGB}{220, 53, 69}
\definecolor{Forest}{RGB}{46, 125, 50}
\definecolor{Amber}{RGB}{181, 133, 63}
\definecolor{Orange}{RGB}{230, 126, 34}

\setbeamercolor{palette primary}{bg=MainBlue, fg=white}
\setbeamercolor{palette secondary}{bg=MainBlue!85, fg=white}
\setbeamercolor{palette tertiary}{bg=MainBlue!70, fg=white}
\setbeamercolor{structure}{fg=MainBlue}
\setbeamercolor{title}{fg=MainBlue}
\setbeamercolor{frametitle}{fg=MainBlue, bg=white}
\setbeamercolor{block title}{bg=MainBlue, fg=white}
\setbeamercolor{block body}{bg=VeryLightGray, fg=DarkGray}
\setbeamercolor{block title alerted}{bg=Crimson, fg=white}
\setbeamercolor{block body alerted}{bg=Crimson!8, fg=DarkGray}
\setbeamercolor{block title example}{bg=Forest, fg=white}
\setbeamercolor{block body example}{bg=Forest!8, fg=DarkGray}
\setbeamercolor{item}{fg=MainBlue}

\setbeamertemplate{navigation symbols}{}

\setbeamertemplate{footline}{
    \leavevmode%
    \hbox{%
        \begin{beamercolorbox}[wd=.333333\paperwidth,ht=2.5ex,dp=1ex,center]{author in head/foot}%
            \usebeamerfont{author in head/foot}\insertshortauthor
        \end{beamercolorbox}%
        \begin{beamercolorbox}[wd=.333333\paperwidth,ht=2.5ex,dp=1ex,center]{title in head/foot}%
            \usebeamerfont{title in head/foot}\insertshorttitle
        \end{beamercolorbox}%
        \begin{beamercolorbox}[wd=.333333\paperwidth,ht=2.5ex,dp=1ex,right]{date in head/foot}%
            \usebeamerfont{date in head/foot}\insertshortdate{}\hspace*{2em}
            \insertframenumber{} / \inserttotalframenumber\hspace*{2ex}
        \end{beamercolorbox}}%
    \vskip0pt%
}

%=============================================================================
% PACKAGES
%=============================================================================
\usepackage{amsmath, amssymb, amsthm}
\usepackage{mathtools}
\usepackage{bm}
\usepackage{tikz}
\usetikzlibrary{arrows.meta, positioning, shapes, calc}
\usepackage{booktabs}
\usepackage{multirow}
\usepackage{array}
\usepackage{graphicx}
\usepackage{hyperref}
\hypersetup{colorlinks=false, pdfborder={0 0 0}}
\graphicspath{{logos/}{charts/}}

%=============================================================================
% CUSTOM COMMANDS
%=============================================================================
\newcommand{\E}{\mathbb{E}}
\newcommand{\Var}{\text{Var}}
\newcommand{\Cov}{\text{Cov}}
\newcommand{\Corr}{\text{Corr}}
\newcommand{\R}{\mathbb{R}}

%=============================================================================
% TITLE INFORMATION
%=============================================================================
\title[Chapter 4: Seminar]{Chapter 4: Seminar --- SARIMA Models}
\subtitle{Bachelor program Faculty of Cybernetics, Statistics and Economic Informatics, Bucharest University of Economic Studies, Romania}
\author[Prof. dr. Daniel Traian Pele]{Prof. dr. Daniel Traian Pele\\[0.2cm]\footnotesize\texttt{danpele@ase.ro}}
\institute{Bucharest University of Economic Studies}
\date{Academic Year 2025--2026}

\begin{document}

%=============================================================================
% TITLE SLIDE
%=============================================================================
\begin{frame}[plain]
    \begin{tikzpicture}[remember picture, overlay]
        \fill[IDAred] (current page.north west) rectangle ([yshift=-0.15cm]current page.north east);
        \node[anchor=north west] at ([xshift=0.5cm, yshift=-0.3cm]current page.north west) {
            \href{https://www.ase.ro}{\includegraphics[height=1.1cm]{ase_logo.png}}
        };
        \node[anchor=north] at ([yshift=-0.3cm]current page.north) {
            \href{https://ai4efin.ase.ro}{\includegraphics[height=1.1cm]{ai4efin_logo.png}}
        };
        \node[anchor=north east] at ([xshift=-0.5cm, yshift=-0.3cm]current page.north east) {
            \href{https://www.digital-finance-msca.com}{\includegraphics[height=1.1cm]{msca_logo.png}}
        };
    \end{tikzpicture}
    \vfill
    \begin{center}
        {\Huge\textbf{\textcolor{MainBlue}{Chapter 4: SARIMA Models}}}\\[0.5cm]
        {\Large\textcolor{IDAred}{Seminar}}
    \end{center}
    \vfill

    \begin{tikzpicture}[remember picture, overlay]
        \fill[IDAred] (current page.south west) rectangle ([yshift=0.15cm]current page.south east);
        \node[anchor=south west] at ([xshift=0.5cm, yshift=0.8cm]current page.south west) {
            \href{https://theida.net}{\includegraphics[height=0.9cm]{ida_logo.png}}
        };
        \node[anchor=south] at ([xshift=-3cm, yshift=0.8cm]current page.south) {
            \href{https://blockchain-research-center.com}{\includegraphics[height=0.9cm]{brc_logo.png}}
        };
        \node[anchor=south] at ([yshift=0.8cm]current page.south) {
            \href{https://quantinar.com}{\includegraphics[height=0.9cm]{qr_logo.png}}
        };
        \node[anchor=south] at ([xshift=3cm, yshift=0.8cm]current page.south) {
            \href{https://quantlet.com}{\includegraphics[height=0.9cm]{ql_logo.png}}
        };
        \node[anchor=south east] at ([xshift=-0.5cm, yshift=0.8cm]current page.south east) {
            \href{https://ipe.ro/new}{\includegraphics[height=0.9cm]{acad_logo.png}}
        };
    \end{tikzpicture}
\end{frame}

%=============================================================================
% OUTLINE
%=============================================================================
\begin{frame}{Seminar Outline}
    \tableofcontents
\end{frame}

%=============================================================================
% SECTION 1: REVIEW QUIZ
%=============================================================================
\section{Review Quiz}

\begin{frame}{Quiz 1: Seasonal Differencing}
    \begin{alertblock}{Question}
        For monthly data with annual seasonality, what does the operator $(1-L^{12})$ do?
    \end{alertblock}

    \vspace{0.3cm}

    \begin{enumerate}[A)]
        \item Takes 12 consecutive differences
        \item Computes $Y_t - Y_{t-12}$
        \item Averages over 12 months
        \item Removes the first 12 observations
    \end{enumerate}

    \vspace{0.5cm}
    \pause
    \begin{exampleblock}{Answer: B -- Computes $Y_t - Y_{t-12}$}
        \textbf{Seasonal difference operator}:
        \[
        (1-L^{12})Y_t = Y_t - L^{12}Y_t = Y_t - Y_{t-12}
        \]

        \textbf{Example} (January sales): $Y_{Jan2025} - Y_{Jan2024}$

        \textbf{Effect}: Removes stable annual seasonal pattern

        \textbf{Note}: $(1-L^s)$ for any seasonal period $s$ (quarterly: $s=4$, weekly: $s=52$)
    \end{exampleblock}
\end{frame}

\begin{frame}{Visual: Seasonal Difference}
    \begin{center}
        \includegraphics[width=0.95\textwidth]{charts/ch4_def_seasonal_diff.pdf}
    \end{center}
    \vspace{-0.2cm}
    \small Seasonal differencing removes annual patterns by comparing same periods across years.
\end{frame}

\begin{frame}{Quiz 2: SARIMA Notation}
    \begin{alertblock}{Question}
        What does SARIMA$(1,1,1) \times (1,1,1)_{12}$ represent?
    \end{alertblock}

    \vspace{0.3cm}

    \begin{enumerate}[A)]
        \item 12 different ARIMA models
        \item ARIMA with 12 AR and 12 MA terms
        \item ARIMA(1,1,1) with seasonal ARIMA(1,1,1) at period 12
        \item A model requiring 12 years of data
    \end{enumerate}
\end{frame}

\begin{frame}{Quiz 2: Answer}
    \begin{exampleblock}{Answer: C -- ARIMA(1,1,1) with seasonal ARIMA(1,1,1) at period 12}
        \begin{center}
            \includegraphics[width=0.95\textwidth, height=0.55\textheight, keepaspectratio]{charts/sem4_sarima_notation.pdf}
        \end{center}
        $(1-\phi_1 L)(1-\Phi_1 L^{12})(1-L)(1-L^{12})Y_t = (1+\theta_1 L)(1+\Theta_1 L^{12})\varepsilon_t$
    \end{exampleblock}
\end{frame}

\begin{frame}{Visual: SARIMA Model Structure}
    \begin{center}
        \includegraphics[width=0.95\textwidth]{charts/ch4_def_sarima.pdf}
    \end{center}
    \vspace{-0.2cm}
    \small SARIMA combines regular ARIMA components with seasonal components at lag $s$.
\end{frame}

\begin{frame}{Quiz 3: The Airline Model}
    \begin{alertblock}{Question}
        The ``airline model'' refers to SARIMA$(0,1,1) \times (0,1,1)_{12}$. How many parameters does it have (excluding variance)?
    \end{alertblock}

    \vspace{0.3cm}

    \begin{enumerate}[A)]
        \item 2 parameters
        \item 4 parameters
        \item 6 parameters
        \item 12 parameters
    \end{enumerate}
\end{frame}

\begin{frame}{Quiz 3: Answer}
    \begin{exampleblock}{Answer: A -- 2 parameters ($\theta_1$ and $\Theta_1$)}
        \begin{center}
            \includegraphics[width=0.95\textwidth, height=0.55\textheight, keepaspectratio]{charts/sem4_airline_model.pdf}
        \end{center}
        \textbf{Airline model}: $(1-L)(1-L^{12})Y_t = (1+\theta_1 L)(1+\Theta_1 L^{12})\varepsilon_t$

        Remarkably fits many seasonal economic series (Box \& Jenkins, 1970)
    \end{exampleblock}
\end{frame}

\begin{frame}{Quiz 4: ACF of Seasonal Data}
    \begin{alertblock}{Question}
        For monthly data with strong seasonality, where would you expect to see significant ACF spikes?
    \end{alertblock}

    \vspace{0.3cm}

    \begin{enumerate}[A)]
        \item Only at lag 1
        \item Only at lag 12
        \item At lags 12, 24, 36, ...
        \item Randomly distributed
    \end{enumerate}

    \vspace{0.5cm}
    \pause
    \begin{exampleblock}{Answer: C -- At lags 12, 24, 36, ...}
        {\small
        \textbf{Intuition}: January 2024 is similar to January 2023, 2022, etc.

        \textbf{ACF pattern}: Spikes at lags $s, 2s, 3s, \ldots$ ($\rho_{12}, \rho_{24}, \rho_{36} \neq 0$)

        \textbf{Diagnostic}: Slow decay at seasonal lags $\Rightarrow$ $D=1$; Cutoff after lag $s$ $\Rightarrow$ $Q=1$
        }
    \end{exampleblock}
\end{frame}

\begin{frame}{Visual: Seasonality Patterns}
    \begin{center}
        \includegraphics[width=0.95\textwidth]{charts/ch4_def_seasonality.pdf}
    \end{center}
    \vspace{-0.2cm}
    \small Seasonal patterns repeat at regular intervals (monthly, quarterly, etc.) and may be additive or multiplicative.
\end{frame}

\begin{frame}{Quiz 5: Multiplicative Structure}
    \begin{alertblock}{Question}
        In SARIMA, what does ``multiplicative structure'' mean?
    \end{alertblock}

    \vspace{0.3cm}

    \begin{enumerate}[A)]
        \item The seasonal amplitude grows proportionally
        \item Regular and seasonal polynomials are multiplied
        \item We multiply the data by seasonal factors
        \item The model is estimated using multiplication
    \end{enumerate}

    \vspace{0.5cm}
    \pause
    \begin{exampleblock}{Answer: B -- Regular and seasonal polynomials are multiplied}
        {\small
        \textbf{Multiplicative SARIMA}: $\phi(L)\Phi(L^s)(1-L)^d(1-L^s)^D Y_t = \theta(L)\Theta(L^s)\varepsilon_t$

        \textbf{Example}: $(1-\phi_1 L)(1-\Phi_1 L^{12}) = 1 - \phi_1 L - \Phi_1 L^{12} + \phi_1\Phi_1 L^{13}$

        \textbf{Cross-term} $\phi_1\Phi_1 L^{13}$: Captures interaction between short and long dynamics
        }
    \end{exampleblock}
\end{frame}

\begin{frame}{Quiz 6: Seasonal vs Regular Differencing}
    \begin{alertblock}{Question}
        When would you apply both regular ($d=1$) and seasonal ($D=1$) differencing?
    \end{alertblock}

    \vspace{0.3cm}

    \begin{enumerate}[A)]
        \item When data has only a trend
        \item When data has only seasonality
        \item When data has both trend and seasonal non-stationarity
        \item Never -- they cancel each other
    \end{enumerate}

    \vspace{0.5cm}
    \pause
    \begin{exampleblock}{Answer: C -- Both trend and seasonal non-stationarity}
        {\small
        \textbf{Combined}: $W_t = (1-L)(1-L^{12})Y_t = Y_t - Y_{t-1} - Y_{t-12} + Y_{t-13}$

        \textbf{When needed}: ACF slow decay at lags 1,2,3... $\Rightarrow d=1$; at lags 12,24,36... $\Rightarrow D=1$

        \textbf{Examples}: Airline passengers, retail sales, energy demand
        }
    \end{exampleblock}
\end{frame}

\begin{frame}{Quiz 7: Detecting Seasonality from ACF}
    \begin{alertblock}{Question}
        The ACF of a monthly time series shows slow decay at lags 12, 24, and 36. What does this suggest?
    \end{alertblock}

    \vspace{0.3cm}

    \begin{enumerate}[A)]
        \item The series is stationary
        \item The series needs regular differencing only
        \item The series has a seasonal unit root requiring $D=1$
        \item The series is white noise
    \end{enumerate}
\end{frame}

\begin{frame}{Quiz 7: Answer}
    \begin{exampleblock}{Answer: C -- Seasonal unit root requiring $D=1$}
        \begin{center}
            \includegraphics[width=0.95\textwidth, height=0.55\textheight, keepaspectratio]{charts/sem4_seasonal_acf.pdf}
        \end{center}
        \textbf{Left}: Stationary seasonal (fast decay at seasonal lags)

        \textbf{Right}: Seasonal unit root (slow decay $\Rightarrow$ need $D=1$)
    \end{exampleblock}
\end{frame}

\begin{frame}{Quiz 8: Multiplicative vs Additive Seasonality}
    \begin{alertblock}{Question}
        If the seasonal amplitude of a time series grows proportionally with the level, this indicates:
    \end{alertblock}

    \vspace{0.3cm}

    \begin{enumerate}[A)]
        \item Additive seasonality -- use $(1-L^s)$
        \item Multiplicative seasonality -- use $\log$ transformation
        \item No seasonality present
        \item Need for regular differencing only
    \end{enumerate}
\end{frame}

\begin{frame}{Quiz 8: Answer}
    \begin{exampleblock}{Answer: B -- Multiplicative seasonality, use $\log$ transformation}
        \begin{center}
            \includegraphics[width=0.95\textwidth, height=0.55\textheight, keepaspectratio]{charts/sem4_mult_add.pdf}
        \end{center}
        \textbf{Multiplicative}: Seasonal amplitude grows with level (diverging lines)

        \textbf{Solution}: Apply $\log$ transformation before fitting SARIMA
    \end{exampleblock}
\end{frame}

\begin{frame}{Quiz 9: Seasonal Subseries Plot}
    \begin{alertblock}{Question}
        In a seasonal subseries plot, what indicates multiplicative seasonality?
    \end{alertblock}

    \vspace{0.3cm}

    \begin{enumerate}[A)]
        \item Lines for each month are parallel
        \item Lines for each month diverge (spread increases over time)
        \item All months have the same mean
        \item Lines are horizontal
    \end{enumerate}

    \vspace{0.5cm}
    \pause
    \begin{exampleblock}{Answer: B -- Lines diverge (spread increases over time)}
        {\small
        \textbf{Subseries plot}: Groups data by month, plots each month's values across years

        \textbf{Parallel} $\Rightarrow$ Additive; \textbf{Diverging} $\Rightarrow$ Multiplicative; \textbf{Horizontal} $\Rightarrow$ No trend

        \textbf{Action}: If multiplicative, apply $\log$ before fitting SARIMA
        }
    \end{exampleblock}
\end{frame}

\begin{frame}{Quiz 10: Invertibility in SARIMA}
    \begin{alertblock}{Question}
        For SARIMA$(0,1,1) \times (0,1,1)_{12}$ to be invertible, which condition must hold?
    \end{alertblock}

    \vspace{0.3cm}

    \begin{enumerate}[A)]
        \item $|\theta_1| < 1$ only
        \item $|\Theta_1| < 1$ only
        \item Both $|\theta_1| < 1$ and $|\Theta_1| < 1$
        \item No invertibility condition exists for MA models
    \end{enumerate}

    \vspace{0.5cm}
    \pause
    \begin{exampleblock}{Answer: C -- Both $|\theta_1| < 1$ and $|\Theta_1| < 1$}
        {\small
        \textbf{Invertibility}: All MA roots outside unit circle

        \textbf{Multiplicative MA}: $(1+\theta_1 L)(1+\Theta_1 L^{12})$

        \textbf{Roots}: Regular $|z| = |{-1}/{\theta_1}| > 1 \Leftrightarrow |\theta_1| < 1$; Seasonal $|\Theta_1| < 1$

        \textbf{Both} conditions required for overall invertibility!
        }
    \end{exampleblock}
\end{frame}

\begin{frame}{Quiz 11: HEGY Test}
    \begin{alertblock}{Question}
        The HEGY test is used to:
    \end{alertblock}

    \vspace{0.3cm}

    \begin{enumerate}[A)]
        \item Estimate SARIMA parameters
        \item Test for unit roots at different frequencies (trend and seasonal)
        \item Check residual normality
        \item Compare SARIMA models using information criteria
    \end{enumerate}

    \vspace{0.5cm}
    \pause
    \begin{exampleblock}{Answer: B -- Test for unit roots at different frequencies}
        {\small
        \textbf{HEGY test} (Hylleberg-Engle-Granger-Yoo, 1990):

        Tests at: Zero freq ($\omega = 0$) $\Rightarrow d = 1$; Nyquist ($\omega = \pi$); Seasonal $\Rightarrow D = 1$

        \textbf{Decision}: Reject all $\Rightarrow$ seasonal dummies; Don't reject seasonal $\Rightarrow$ seasonal differencing
        }
    \end{exampleblock}
\end{frame}

\begin{frame}{Quiz 12: Seasonal MA Identification}
    \begin{alertblock}{Question}
        After applying $(1-L)(1-L^{12})$, the ACF shows a single significant spike at lag 12 only (no spike at lag 1). The PACF decays at seasonal lags. This suggests:
    \end{alertblock}

    \vspace{0.3cm}

    \begin{enumerate}[A)]
        \item SARIMA$(0,1,0) \times (0,1,1)_{12}$
        \item SARIMA$(0,1,1) \times (0,1,0)_{12}$
        \item SARIMA$(1,1,0) \times (1,1,0)_{12}$
        \item SARIMA$(0,1,1) \times (0,1,1)_{12}$
    \end{enumerate}

    \vspace{0.5cm}
    \pause
    \begin{exampleblock}{Answer: A -- SARIMA$(0,1,0) \times (0,1,1)_{12}$}
        {\small
        \textbf{Pattern}: Regular lags -- no spikes in ACF/PACF; Seasonal lags -- ACF cuts off at $s$, PACF decays

        \textbf{Interpretation}: No regular MA ($q = 0$); Seasonal MA(1) indicated ($Q = 1$)

        \textbf{Model}: $(1-L)(1-L^{12})Y_t = (1 + \Theta_1 L^{12})\varepsilon_t$
        }
    \end{exampleblock}
\end{frame}

\begin{frame}{Quiz 13: Over-differencing}
    \begin{alertblock}{Question}
        After differencing, the ACF shows a large negative spike at lag 1 or lag $s$. This typically indicates:
    \end{alertblock}

    \vspace{0.3cm}

    \begin{enumerate}[A)]
        \item The model needs more AR terms
        \item The series has been over-differenced
        \item The series is perfectly stationary
        \item Heteroskedasticity is present
    \end{enumerate}

    \vspace{0.5cm}
    \pause
    \begin{exampleblock}{Answer: B -- The series has been over-differenced}
        {\small
        \textbf{Signature}: ACF at lag 1 $\approx -0.5$ $\Rightarrow$ over-diff at $d$; ACF at lag $s$ $\approx -0.5$ $\Rightarrow$ over-diff at $D$

        \textbf{Why?} $\Delta^2 Y_t = \varepsilon_t - \varepsilon_{t-1}$ is MA(1) with $\theta = -1$, giving $\rho_1 = -0.5$

        \textbf{Fix}: Reduce $d$ or $D$ by one and re-examine ACF/PACF
        }
    \end{exampleblock}
\end{frame}

\begin{frame}{Quiz 14: Forecasting Horizon}
    \begin{alertblock}{Question}
        For a SARIMA model with $D=1$, what happens to forecast confidence intervals as the horizon $h \to \infty$?
    \end{alertblock}

    \vspace{0.3cm}

    \begin{enumerate}[A)]
        \item They converge to a fixed width
        \item They grow without bound
        \item They shrink to zero
        \item They oscillate seasonally
    \end{enumerate}

    \vspace{0.5cm}
    \pause
    \begin{exampleblock}{Answer: B -- They grow without bound}
        {\small
        \textbf{Unit root property}: Any unit root causes unbounded forecast variance

        \textbf{For SARIMA with $D=1$}: $\Var(\hat{Y}_{T+h} - Y_{T+h}) \to \infty$ as $h \to \infty$

        \textbf{Intuition}: Seasonal shocks accumulate; long-range forecasts have wide CIs
        }
    \end{exampleblock}
\end{frame}

\begin{frame}{Quiz 15: Seasonal Period Selection}
    \begin{alertblock}{Question}
        You have daily data showing clear weekly patterns. What seasonal period $s$ should you use in a SARIMA model?
    \end{alertblock}

    \vspace{0.3cm}

    \begin{enumerate}[A)]
        \item $s = 12$ (monthly)
        \item $s = 7$ (weekly)
        \item $s = 365$ (yearly)
        \item $s = 24$ (hourly)
    \end{enumerate}

    \vspace{0.5cm}
    \pause
    \begin{exampleblock}{Answer: B -- $s = 7$ (weekly)}
        {\small
        \begin{center}
        \begin{tabular}{lll}
            \textbf{Data} & \textbf{Pattern} & \textbf{Period $s$} \\
            Daily & Weekly & 7 \\
            Monthly & Annual & 12 \\
            Quarterly & Annual & 4
        \end{tabular}
        \end{center}
        \textbf{Rule}: $s$ = observations per cycle of dominant pattern
        }
    \end{exampleblock}
\end{frame}

\begin{frame}{Quiz 16: Seasonal AR Component}
    \begin{alertblock}{Question}
        In the seasonal component $\Phi(L^s) = 1 - \Phi_1 L^s$, what does the coefficient $\Phi_1 = 0.8$ tell us?
    \end{alertblock}

    \vspace{0.3cm}

    \begin{enumerate}[A)]
        \item 80\% of this period's value comes from the previous period
        \item There is 80\% correlation between consecutive observations
        \item 80\% of this period's value is explained by the same period last year
        \item The seasonal pattern explains 80\% of variance
    \end{enumerate}

    \vspace{0.5cm}
    \pause
    \begin{exampleblock}{Answer: C -- 80\% explained by same period last year}
        {\small
        \textbf{SAR(1)}: $Y_t = \Phi_1 Y_{t-12} + \varepsilon_t$

        \textbf{With $\Phi_1 = 0.8$}: $Y_{Jan2024} = 0.8 \cdot Y_{Jan2023} + \varepsilon_t$

        \textbf{Interpretation}: Strong seasonal persistence -- 80\% explained by same month last year

        \textbf{Stationarity}: Requires $|\Phi_1| < 1$ (satisfied here)
        }
    \end{exampleblock}
\end{frame}

\begin{frame}{Quiz 17: Seasonal Stationarity}
    \begin{alertblock}{Question}
        A seasonal process with $\Phi_1 = 1$ in SARIMA$(0,0,0) \times (1,0,0)_{12}$ is:
    \end{alertblock}

    \vspace{0.3cm}

    \begin{enumerate}[A)]
        \item Stationary
        \item Has a seasonal unit root (seasonally integrated)
        \item Explosive
        \item Undefined
    \end{enumerate}

    \vspace{0.5cm}
    \pause
    \begin{exampleblock}{Answer: B -- Has a seasonal unit root}
        {\small
        \textbf{Model}: $Y_t = Y_{t-12} + \varepsilon_t$ (seasonal random walk)

        \textbf{Properties}: Variance grows with time; each month follows its own RW; need $D = 1$

        \textbf{Analogy}: Like regular random walk but at seasonal frequency
        }
    \end{exampleblock}
\end{frame}

\begin{frame}{Quiz 18: Model Comparison}
    \begin{alertblock}{Question}
        Model A: SARIMA$(1,1,1) \times (1,1,1)_{12}$ has AIC = 520. Model B: SARIMA$(0,1,1) \times (0,1,1)_{12}$ has AIC = 525. Which statement is most accurate?
    \end{alertblock}

    \vspace{0.3cm}

    \begin{enumerate}[A)]
        \item Model A is always better since it has lower AIC
        \item Model B should be preferred due to parsimony despite higher AIC
        \item The AIC difference of 5 suggests Model A is substantially better
        \item We cannot compare models with different orders
    \end{enumerate}

    \vspace{0.5cm}
    \pause
    \begin{exampleblock}{Answer: C -- AIC difference of 5 suggests Model A is substantially better}
        {\small
        \textbf{Rule of thumb}: $\Delta$AIC $< 2$: equivalent; $2$--$10$: some evidence; $> 10$: strong evidence

        \textbf{Here}: $\Delta$AIC $= 5$ suggests Model A meaningfully better

        \textbf{Always}: Also check residual diagnostics and forecast performance!
        }
    \end{exampleblock}
\end{frame}

\begin{frame}{Quiz 19: Seasonal Patterns in Residuals}
    \begin{alertblock}{Question}
        After fitting a SARIMA model, you notice significant ACF spikes at lags 12 and 24 in the residuals. What does this indicate?
    \end{alertblock}

    \vspace{0.3cm}

    \begin{enumerate}[A)]
        \item The model is correctly specified
        \item The seasonal component is inadequate
        \item The data is not seasonal
        \item Overfitting has occurred
    \end{enumerate}

    \vspace{0.5cm}
    \pause
    \begin{exampleblock}{Answer: B -- The seasonal component is inadequate}
        {\small
        \textbf{Diagnostics}: Good residuals should be white noise (no significant ACF)

        \textbf{Seasonal ACF in residuals}: Model hasn't captured seasonal structure; try increasing $P$ or $Q$; verify $D$ is correct

        \textbf{Action}: Try SARIMA with higher seasonal order, check Ljung-Box at seasonal lags
        }
    \end{exampleblock}
\end{frame}

\begin{frame}{Quiz 20: Practical Forecasting}
    \begin{alertblock}{Question}
        You're forecasting monthly retail sales with SARIMA$(0,1,1) \times (0,1,1)_{12}$. For the 13-month-ahead forecast, which historical observations are most influential?
    \end{alertblock}

    \vspace{0.3cm}

    \begin{enumerate}[A)]
        \item Only the most recent observation
        \item The observation from the same month last year
        \item All observations equally
        \item Only observations from the same month in all previous years
    \end{enumerate}

    \vspace{0.5cm}
    \pause
    \begin{exampleblock}{Answer: B -- The observation from the same month last year}
        {\small
        \textbf{For 13-month ahead}: Most influential is $Y_{T-11}$ (same month last year), also $Y_T$ and $Y_{T-12}$

        \textbf{Intuition}: ``Next January looks like last January, adjusted for recent trend''
        }
    \end{exampleblock}
\end{frame}

%=============================================================================
% TRUE/FALSE QUESTIONS
%=============================================================================
\begin{frame}{True/False Questions (1-6)}
    \begin{alertblock}{Question}
        Determine whether each statement is True or False:
    \end{alertblock}

    \vspace{0.3cm}
    {\small
    \begin{enumerate}
        \item The seasonal period $s$ for quarterly data with annual patterns is $s=4$.
        \item SARIMA models can only handle one seasonal frequency.
        \item If AIC selects SARIMA$(1,1,1) \times (1,1,1)_{12}$ and BIC selects the airline model, BIC is always wrong.
        \item The Kruskal-Wallis test can detect seasonality without assuming normality.
        \item After fitting a SARIMA model, residuals should show no significant ACF at seasonal lags.
        \item Log transformation converts multiplicative seasonality to additive.
    \end{enumerate}
    }

    \vspace{0.3cm}
    \begin{center}
        \textit{Answer on next slide...}
    \end{center}
\end{frame}

\begin{frame}{True/False Solutions (1-6)}
    \begin{exampleblock}{Answers}
    \begin{enumerate}
        \item \textcolor{Forest}{\textbf{TRUE}}: Quarterly data with annual cycle has $s=4$ quarters per year.
        \item \textcolor{Forest}{\textbf{TRUE}}: Standard SARIMA handles one $s$; multiple seasonalities need TBATS or Fourier terms.
        \item \textcolor{Crimson}{\textbf{FALSE}}: BIC penalizes complexity more; simpler model may be better for interpretation/forecasting.
        \item \textcolor{Forest}{\textbf{TRUE}}: Kruskal-Wallis is nonparametric, comparing distributions across seasons.
        \item \textcolor{Forest}{\textbf{TRUE}}: Residual ACF should be within confidence bands at ALL lags including seasonal.
        \item \textcolor{Forest}{\textbf{TRUE}}: $\log(T \times S \times \varepsilon) = \log T + \log S + \log \varepsilon$ (additive form).
    \end{enumerate}
    \end{exampleblock}
\end{frame}

%=============================================================================
% SECTION 2: PRACTICE PROBLEMS
%=============================================================================
\section{Practice Problems}

\begin{frame}{Problem 1: Expanding the Seasonal Difference}
    \begin{block}{Exercise}
        Expand $(1-L)(1-L^{12})Y_t$ fully. What observations are involved?
    \end{block}

    \vspace{0.3cm}
    \pause
    \begin{exampleblock}{Solution}
        $(1-L)(1-L^{12}) = 1 - L - L^{12} + L^{13}$

        \vspace{0.2cm}
        Therefore:
        $(1-L)(1-L^{12})Y_t = Y_t - Y_{t-1} - Y_{t-12} + Y_{t-13}$

        \vspace{0.2cm}
        \textbf{Interpretation}: This is the difference of differences:
        \begin{itemize}
            \item First seasonal difference: $Y_t - Y_{t-12}$ (this year vs last year)
            \item Then regular difference: $(Y_t - Y_{t-12}) - (Y_{t-1} - Y_{t-13})$
        \end{itemize}
    \end{exampleblock}
\end{frame}

\begin{frame}{Problem 2: Airline Model Expansion}
    \begin{block}{Exercise}
        Write out the full equation for the airline model SARIMA$(0,1,1) \times (0,1,1)_{12}$:
        $(1-L)(1-L^{12})Y_t = (1+\theta_1 L)(1+\Theta_1 L^{12})\varepsilon_t$
    \end{block}

    \vspace{0.2cm}
    \pause
    \begin{exampleblock}{Solution}
        Expand the MA side:
        $(1+\theta_1 L)(1+\Theta_1 L^{12}) = 1 + \theta_1 L + \Theta_1 L^{12} + \theta_1 \Theta_1 L^{13}$

        \vspace{0.2cm}
        Full model:
        $Y_t - Y_{t-1} - Y_{t-12} + Y_{t-13} = \varepsilon_t + \theta_1 \varepsilon_{t-1} + \Theta_1 \varepsilon_{t-12} + \theta_1 \Theta_1 \varepsilon_{t-13}$

        \vspace{0.2cm}
        \textbf{Note}: The cross-term $\theta_1 \Theta_1 L^{13}$ is the multiplicative interaction between regular and seasonal MA components.
    \end{exampleblock}
\end{frame}

\begin{frame}{Problem 3: Parameter Count}
    \begin{block}{Exercise}
        How many parameters (excluding $\sigma^2$) are in SARIMA$(2,1,1) \times (1,0,1)_4$?
    \end{block}

    \vspace{0.3cm}
    \pause
    \begin{exampleblock}{Solution}
        \begin{itemize}
            \item Regular AR($p=2$): $\phi_1, \phi_2$ $\Rightarrow$ 2 parameters
            \item Regular MA($q=1$): $\theta_1$ $\Rightarrow$ 1 parameter
            \item Seasonal AR($P=1$): $\Phi_1$ $\Rightarrow$ 1 parameter
            \item Seasonal MA($Q=1$): $\Theta_1$ $\Rightarrow$ 1 parameter
        \end{itemize}

        \vspace{0.2cm}
        \textbf{Total: 5 parameters}

        \vspace{0.2cm}
        Note: The differencing orders ($d=1$, $D=0$) don't add parameters -- they're transformations applied to the data.
    \end{exampleblock}
\end{frame}

\begin{frame}{Problem 4: SARIMA Forecasting}
    \begin{block}{Exercise}
        Given the airline model with $\theta_1 = -0.4$ and $\Theta_1 = -0.6$, and:
        \begin{itemize}
            \item $Y_T = 500$, $Y_{T-1} = 495$, $Y_{T-11} = 480$, $Y_{T-12} = 470$
            \item $\varepsilon_T = 5$, $\varepsilon_{T-11} = -3$, $\varepsilon_{T-12} = 2$
        \end{itemize}
        Forecast $Y_{T+1}$.
    \end{block}

    \pause
    \begin{exampleblock}{Solution}
        From the model: $Y_{T+1} = Y_T + Y_{T-11} - Y_{T-12} + \varepsilon_{T+1} + \theta_1 \varepsilon_T + \Theta_1 \varepsilon_{T-11} + \theta_1 \Theta_1 \varepsilon_{T-12}$

        Setting $\E[\varepsilon_{T+1}] = 0$:

        $\hat{Y}_{T+1} = 500 + 480 - 470 + 0 + (-0.4)(5) + (-0.6)(-3) + (-0.4)(-0.6)(2)$

        $= 510 - 2 + 1.8 + 0.48 = \mathbf{510.28}$
    \end{exampleblock}
\end{frame}

\begin{frame}{Problem 5: Identifying Seasonal Period}
    \begin{block}{Exercise}
        Match each data type with its typical seasonal period $s$:
        \begin{enumerate}
            \item Quarterly GDP data
            \item Monthly retail sales
            \item Weekly restaurant reservations
            \item Daily electricity demand
        \end{enumerate}
    \end{block}

    \vspace{0.2cm}
    \pause
    \begin{exampleblock}{Solution}
        \begin{enumerate}
            \item Quarterly GDP: $s = 4$ (annual cycle over 4 quarters)
            \item Monthly retail sales: $s = 12$ (annual cycle over 12 months)
            \item Weekly restaurant reservations: $s = 7$ (weekly cycle) or $s = 52$ (annual)
            \item Daily electricity demand: $s = 7$ (weekly pattern) or $s = 365$ (annual)
        \end{enumerate}

        \textbf{Note}: Some series have multiple seasonal patterns (e.g., daily data may have weekly AND annual cycles).
    \end{exampleblock}
\end{frame}

%=============================================================================
% SECTION 3: WORKED EXAMPLES
%=============================================================================
\section{Worked Examples}

\begin{frame}{Example: Monthly Retail Sales Analysis}
    \begin{block}{Scenario}
        You have 5 years of monthly retail sales data showing clear December peaks and January troughs. Build an appropriate SARIMA model.
    \end{block}

    \vspace{0.2cm}

    \begin{exampleblock}{Step-by-step Approach}
        \begin{enumerate}
            \item \textbf{Visual inspection}: Plot shows upward trend + strong December spikes
            \item \textbf{Seasonal period}: Monthly data with annual pattern $\Rightarrow s = 12$
            \item \textbf{Transformation}: Consider $\log(Y_t)$ if seasonal amplitude grows with level
            \item \textbf{Differencing}: Try $(1-L)(1-L^{12})Y_t$ -- check ACF/PACF
            \item \textbf{Model selection}: Start with airline model, compare via AIC
        \end{enumerate}
    \end{exampleblock}
\end{frame}

\begin{frame}{Example: ACF/PACF Interpretation for Seasonal Data}
    \begin{block}{Observed Patterns (after differencing)}
        \begin{itemize}
            \item ACF: Significant at lags 1, 12; cuts off after lag 1 and lag 12
            \item PACF: Significant at lags 1, 12, 13; decays at multiples of 12
        \end{itemize}
    \end{block}

    \vspace{0.2cm}

    \begin{exampleblock}{Interpretation}
        \textbf{Regular component}: ACF cuts off at 1 $\Rightarrow$ MA(1)

        \textbf{Seasonal component}: ACF significant only at lag 12 $\Rightarrow$ seasonal MA(1)

        \textbf{Suggested model}: SARIMA$(0,d,1) \times (0,D,1)_{12}$ -- the airline model!

        \vspace{0.2cm}
        \textbf{Alternative check}: If PACF showed cutoff at seasonal lags instead of ACF, consider seasonal AR terms.
    \end{exampleblock}
\end{frame}

\begin{frame}[fragile]{Example: Python Implementation}
    \begin{block}{Fitting SARIMA in Python}
        \small
        \begin{verbatim}
from statsmodels.tsa.statespace.sarimax import SARIMAX
import pmdarima as pm

# Manual fit
model = SARIMAX(y, order=(0,1,1), seasonal_order=(0,1,1,12))
results = model.fit()
print(results.summary())

# Automatic selection
auto_model = pm.auto_arima(y, seasonal=True, m=12,
                           start_p=0, max_p=2,
                           start_q=0, max_q=2,
                           d=1, D=1,
                           trace=True)
        \end{verbatim}
    \end{block}
\end{frame}

\begin{frame}[fragile]{Example: Interpreting SARIMA Output}
    \begin{block}{Sample statsmodels Output}
        \footnotesize
        \begin{verbatim}
                         SARIMAX Results
==============================================================
Model:            SARIMAX(0,1,1)x(0,1,1,12)   AIC:    1348.52
                                               BIC:    1358.21
==============================================================
                 coef    std err      z     P>|z|
--------------------------------------------------------------
ma.L1         -0.4018      0.072   -5.58    0.000
ma.S.L12      -0.5521      0.081   -6.82    0.000
sigma2      1254.3201    142.856    8.78    0.000
        \end{verbatim}
    \end{block}

    \begin{exampleblock}{Interpretation}
        \begin{itemize}
            \item $\hat{\theta}_1 = -0.40$: Negative MA means positive shocks reduce next period's value
            \item $\hat{\Theta}_1 = -0.55$: Same-season correlation is captured
            \item Both coefficients significant $(p < 0.001)$; $|\theta|, |\Theta| < 1$ -- invertible
        \end{itemize}
    \end{exampleblock}
\end{frame}

%=============================================================================
% SECTION 4: REAL DATA ANALYSIS
%=============================================================================
\section{Real Data Analysis}

\begin{frame}{Case Study: Airline Passengers (1949--1960)}
    \vspace{-0.3cm}
    \begin{center}
        \includegraphics[width=0.85\textwidth, height=0.55\textheight, keepaspectratio]{charts/ch4_airline_data.pdf}
    \end{center}
    \vspace{-0.2cm}
    {\footnotesize
    \begin{itemize}
        \item Classic Box-Jenkins dataset: 144 monthly observations
        \item Clear \textbf{upward trend} and \textbf{seasonal pattern} (summer peaks)
        \item Seasonal amplitude \textbf{grows with level} $\Rightarrow$ multiplicative seasonality
        \item Suggests: log transformation + SARIMA modeling
    \end{itemize}
    }
\end{frame}

\begin{frame}{ACF/PACF Analysis After Differencing}
    \vspace{-0.3cm}
    \begin{center}
        \includegraphics[width=0.85\textwidth, height=0.55\textheight, keepaspectratio]{charts/ch4_acf_pacf.pdf}
    \end{center}
    \vspace{-0.2cm}
    {\footnotesize
    \begin{itemize}
        \item After $(1-L)(1-L^{12})\log(Y_t)$: series appears stationary
        \item Significant spike at lag 1 in ACF $\Rightarrow$ MA(1) component
        \item Significant spike at lag 12 in ACF $\Rightarrow$ Seasonal MA(1) component
        \item Pattern suggests: \textbf{SARIMA$(0,1,1)(0,1,1)_{12}$} (airline model)
    \end{itemize}
    }
\end{frame}

\begin{frame}{SARIMA Estimation Results: Airline Data}
    {\small
    \begin{block}{Model: SARIMA$(0,1,1)(0,1,1)_{12}$ on $\log(\text{Passengers})$}
        \begin{center}
        \begin{tabular}{lcccc}
            \toprule
            \textbf{Parameter} & \textbf{Estimate} & \textbf{Std. Error} & \textbf{z-stat} & \textbf{p-value} \\
            \midrule
            $\theta_1$ (MA.L1) & $-0.4018$ & $0.0896$ & $-4.48$ & $<0.001$ \\
            $\Theta_1$ (MA.S.L12) & $-0.5569$ & $0.0731$ & $-7.62$ & $<0.001$ \\
            $\sigma^2$ & $0.00135$ & -- & -- & -- \\
            \bottomrule
        \end{tabular}
        \end{center}
    \end{block}

    \vspace{0.2cm}

    \begin{exampleblock}{Model Fit Statistics}
        \begin{itemize}
            \item Log-Likelihood: $244.70$
            \item AIC: $-483.40$, BIC: $-474.53$
            \item Both MA coefficients significant and within invertibility bounds
        \end{itemize}
    \end{exampleblock}
    }
\end{frame}

\begin{frame}{Forecast: 24 Months Ahead}
    \vspace{-0.3cm}
    \begin{center}
        \includegraphics[width=0.85\textwidth, height=0.55\textheight, keepaspectratio]{charts/ch4_sarima_forecast.pdf}
    \end{center}
    \vspace{-0.2cm}
    {\footnotesize
    \begin{itemize}
        \item Forecasts capture both trend and seasonal pattern
        \item 95\% confidence intervals widen over forecast horizon
        \item Seasonal peaks (July-August) and troughs (February) clearly visible
        \item Model successfully extrapolates the multiplicative seasonal pattern
    \end{itemize}
    }
\end{frame}

\begin{frame}{Model Diagnostics}
    \vspace{-0.3cm}
    \begin{center}
        \includegraphics[width=0.85\textwidth, height=0.55\textheight, keepaspectratio]{charts/ch4_diagnostics.pdf}
    \end{center}
    \vspace{-0.2cm}
    {\footnotesize
    \begin{itemize}
        \item Residuals appear random with no systematic patterns
        \item Distribution approximately normal (Q-Q plot close to diagonal)
        \item ACF of residuals within confidence bounds -- no significant autocorrelation
        \item Ljung-Box test: $p > 0.05$ at all tested lags $\Rightarrow$ adequate model
    \end{itemize}
    }
\end{frame}

%=============================================================================
% SECTION 5: DISCUSSION TOPICS
%=============================================================================
\section{Discussion Topics}

\begin{frame}{Discussion: Deterministic vs Stochastic Seasonality}
    \begin{block}{Key Question}
        When should you use seasonal dummies vs SARIMA for seasonal data?
    \end{block}

    \vspace{0.2cm}

    \begin{block}{Considerations}
        \textbf{Seasonal dummies} (deterministic):
        \begin{itemize}
            \item Fixed, repeating pattern each year
            \item Same December effect every year
            \item Appropriate when seasonality is stable
        \end{itemize}

        \vspace{0.2cm}
        \textbf{SARIMA} (stochastic):
        \begin{itemize}
            \item Evolving seasonal pattern
            \item This year's December depends on last year's December
            \item Better when seasonal amplitude varies
        \end{itemize}
    \end{block}
\end{frame}

\begin{frame}{Discussion: Log Transformation}
    \begin{block}{Key Question}
        When should you take logarithms before fitting SARIMA?
    \end{block}

    \vspace{0.2cm}

    \begin{block}{Guidelines}
        \textbf{Use log transformation when}:
        \begin{itemize}
            \item Seasonal fluctuations grow with the level (multiplicative seasonality)
            \item Variance increases over time
            \item Data is strictly positive (prices, sales, counts)
        \end{itemize}

        \vspace{0.2cm}
        \textbf{Avoid log when}:
        \begin{itemize}
            \item Seasonal pattern is additive (constant amplitude)
            \item Data contains zeros or negatives
            \item Already on a rate/ratio scale
        \end{itemize}

        \vspace{0.2cm}
        \textbf{Tip}: Compare AIC of models with and without log transformation.
    \end{block}
\end{frame}

\begin{frame}{Discussion: Multiple Seasonalities}
    \begin{block}{Challenge}
        Daily sales data may have both weekly (7-day) and annual (365-day) seasonal patterns. How do you handle this?
    \end{block}

    \vspace{0.2cm}

    \begin{block}{Approaches}
        \begin{enumerate}
            \item \textbf{Nested SARIMA}: Model at shorter frequency, include longer as exogenous
            \item \textbf{TBATS/BATS models}: Explicitly handle multiple seasonalities
            \item \textbf{Fourier terms}: Add sin/cos terms for each seasonal frequency
            \item \textbf{Prophet/similar}: Modern tools designed for multiple seasonalities
        \end{enumerate}

        \vspace{0.2cm}
        \textbf{Note}: Standard SARIMA handles only one seasonal period. For complex seasonality, consider specialized methods.
    \end{block}
\end{frame}

\begin{frame}{Discussion: Forecasting Seasonal Data}
    \begin{block}{Key Question}
        What are the unique challenges of forecasting seasonal time series?
    \end{block}

    \vspace{0.2cm}

    \begin{block}{Challenges and Solutions}
        \begin{itemize}
            \item \textbf{Horizon matters}: 12-month forecast means predicting a full cycle
            \item \textbf{Uncertainty grows}: Seasonal forecasts compound regular uncertainty
            \item \textbf{Turning points}: Capturing when seasons peak/trough
            \item \textbf{Structural breaks}: COVID-19 disrupted many seasonal patterns
        \end{itemize}

        \vspace{0.2cm}
        \textbf{Best practices}:
        \begin{itemize}
            \item Use rolling-origin cross-validation
            \item Compare against seasonal naive benchmark
            \item Report forecast intervals, especially at seasonal horizons
        \end{itemize}
    \end{block}
\end{frame}

%=============================================================================
% SECTION 5: EXERCISES
%=============================================================================
\section{Exercises for Self-Study}

\begin{frame}{Take-Home Exercises}
    {\small
    \begin{enumerate}
        \item \textbf{Theoretical}: Show that $(1-L)(1-L^4)$ can be written as $(1 - L - L^4 + L^5)$ and explain what this transformation does to quarterly data with annual seasonality.

        \vspace{0.2cm}
        \item \textbf{Computation}: For SARIMA$(1,0,0) \times (1,0,0)_4$ with $\phi_1 = 0.5$ and $\Phi_1 = 0.8$, write out the full AR polynomial and identify all non-zero coefficients.

        \vspace{0.2cm}
        \item \textbf{Applied}: Download monthly airline passenger data and:
            \begin{itemize}
                \item Plot the series and identify trend/seasonality
                \item Apply appropriate transformations
                \item Fit the airline model and interpret coefficients
                \item Generate 24-month forecasts with confidence intervals
            \end{itemize}

        \vspace{0.2cm}
        \item \textbf{Comparison}: Fit both SARIMA$(0,1,1) \times (0,1,1)_{12}$ and SARIMA$(1,1,0) \times (1,1,0)_{12}$ to the airline data. Compare using AIC, BIC, and residual diagnostics. Which is preferred?
    \end{enumerate}
    }
\end{frame}

\begin{frame}{Exercise Solutions Hints}
    {\small
    \begin{block}{Hints}
        \begin{enumerate}
            \item Expand $(1-L)(1-L^4) = 1 \cdot 1 - 1 \cdot L^4 - L \cdot 1 + L \cdot L^4 = 1 - L - L^4 + L^5$

            \vspace{0.1cm}
            \item AR polynomial: $(1 - \phi_1 L)(1 - \Phi_1 L^4) = 1 - 0.5L - 0.8L^4 + 0.4L^5$

            \vspace{0.1cm}
            \item For airline data:
                \begin{itemize}
                    \item Use log transformation (multiplicative seasonality)
                    \item Both $d=1$ and $D=1$ needed
                    \item Typical estimates: $\theta_1 \approx -0.4$, $\Theta_1 \approx -0.6$
                \end{itemize}

            \vspace{0.1cm}
            \item The MA-based airline model typically fits better than pure AR seasonal model for this data (lower AIC).
        \end{enumerate}
    \end{block}
    }
\end{frame}

%=============================================================================
% SUMMARY
%=============================================================================
\begin{frame}{Key Takeaways from This Seminar}
    \begin{block}{Main Points}
        \begin{enumerate}
            \item Seasonal differencing $(1-L^s)$ removes stochastic seasonality
            \item SARIMA notation: $(p,d,q) \times (P,D,Q)_s$ separates regular and seasonal
            \item The airline model is surprisingly effective for many datasets
            \item Multiplicative structure creates interaction terms
            \item ACF/PACF show patterns at both regular and seasonal lags
            \item Log transformation often needed for multiplicative seasonality
        \end{enumerate}
    \end{block}

    \vspace{0.2cm}
    \begin{alertblock}{Next Steps}
        Chapter 5 will cover multivariate time series: VAR models, Granger causality, and cointegration.
    \end{alertblock}
\end{frame}

\end{document}
