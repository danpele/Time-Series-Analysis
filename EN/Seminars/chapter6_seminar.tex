% Seminar 6: VAR and Granger Causality

\documentclass[9pt, aspectratio=169, t]{beamer}
%=============================================================================
% SHARED PREAMBLE - Time Series Analysis and Forecasting
% Harvard-quality academic presentations
% Bachelor program, Bucharest University of Economic Studies
%
% Usage: \documentclass[9pt, aspectratio=169, t]{beamer}
%            %=============================================================================
% SHARED PREAMBLE - Time Series Analysis and Forecasting
% Harvard-quality academic presentations
% Bachelor program, Bucharest University of Economic Studies
%
% Usage: \documentclass[9pt, aspectratio=169, t]{beamer}
%            %=============================================================================
% SHARED PREAMBLE - Time Series Analysis and Forecasting
% Harvard-quality academic presentations
% Bachelor program, Bucharest University of Economic Studies
%
% Usage: \documentclass[9pt, aspectratio=169, t]{beamer}
%            \input{preamble}
%            \subtitle{Seminar X: Seminar Title}
%            \begin{document} ...
%=============================================================================

% Ensure content fits on slides
\setbeamersize{text margin left=8mm, text margin right=8mm}

%=============================================================================
% THEME AND STYLE CONFIGURATION
%=============================================================================
\usetheme{default}
% Using default theme for clean header/footer control

% Color Palette (matching Redispatch PDF)
\definecolor{MainBlue}{RGB}{26, 58, 110}
\definecolor{AccentBlue}{RGB}{26, 58, 110}
\definecolor{IDAred}{RGB}{205, 0, 0}
\definecolor{DarkGray}{RGB}{51, 51, 51}
\definecolor{MediumGray}{RGB}{128, 128, 128}
\definecolor{LightGray}{RGB}{248, 248, 248}
\definecolor{VeryLightGray}{RGB}{235, 235, 235}
\definecolor{KeynoteGray}{RGB}{218, 218, 218}
\definecolor{SectionGray}{RGB}{120, 120, 120}
\definecolor{FooterGray}{RGB}{100, 100, 100}
\definecolor{Crimson}{RGB}{220, 53, 69}
\definecolor{Forest}{RGB}{46, 125, 50}
\definecolor{Amber}{RGB}{181, 133, 63}
\definecolor{Orange}{RGB}{230, 126, 34}
\definecolor{Purple}{RGB}{142, 68, 173}

% Gradient background (exact Keynote 315° gradient: white to RGB 218,218,218)
\setbeamertemplate{background}{%
    \begin{tikzpicture}[remember picture, overlay]
        \shade[shading=axis, shading angle=315,
        top color=white, bottom color=KeynoteGray]
        (current page.south west) rectangle (current page.north east);
    \end{tikzpicture}%
}
% Fallback solid color for compatibility
\setbeamercolor{background canvas}{bg=}

\setbeamercolor{palette primary}{bg=MainBlue, fg=white}
\setbeamercolor{palette secondary}{bg=MainBlue!85, fg=white}
\setbeamercolor{palette tertiary}{bg=MainBlue!70, fg=white}
\setbeamercolor{structure}{fg=MainBlue}
\setbeamercolor{title}{fg=IDAred}
\setbeamercolor{frametitle}{fg=IDAred, bg=}
\setbeamercolor{block title}{bg=MainBlue, fg=white}
\setbeamercolor{block body}{bg=VeryLightGray, fg=DarkGray}
\setbeamercolor{block title alerted}{bg=Crimson, fg=white}
\setbeamercolor{block body alerted}{bg=Crimson!8, fg=DarkGray}
\setbeamercolor{block title example}{bg=Forest, fg=white}
\setbeamercolor{block body example}{bg=Forest!8, fg=DarkGray}
\setbeamercolor{item}{fg=MainBlue}

% Smaller institute font to avoid overfull hbox on title page
\setbeamerfont{institute}{size=\footnotesize}

% Footer colors (override Madrid theme blue)
\setbeamercolor{author in head/foot}{fg=FooterGray, bg=}
\setbeamercolor{title in head/foot}{fg=FooterGray, bg=}
\setbeamercolor{date in head/foot}{fg=FooterGray, bg=}
\setbeamercolor{section in head/foot}{fg=FooterGray, bg=}
\setbeamercolor{subsection in head/foot}{fg=FooterGray, bg=}

% Bullet styles (apply everywhere including blocks)
\setbeamertemplate{itemize item}{\color{MainBlue}$\boxdot$}
\setbeamertemplate{itemize subitem}{\color{MainBlue}$\blacktriangleright$}
\setbeamertemplate{itemize subsubitem}{\color{MainBlue}\tiny$\bullet$}
\setbeamertemplate{itemize/enumerate body begin}{\normalsize}
\setbeamertemplate{itemize/enumerate subbody begin}{\normalsize}

% Item spacing - compact style
\setlength{\leftmargini}{10pt}       % Level 1: minimal indent
\setlength{\leftmarginii}{10pt}      % Level 2: minimal additional indent
% Compact list spacing (zero extra space before/after lists in blocks)
\makeatletter
\def\@listi{\leftmargin\leftmargini \topsep 0pt \parsep 0pt \itemsep 0pt}
\def\@listii{\leftmargin\leftmarginii \topsep 0pt \parsep 0pt \itemsep 0pt}
\makeatother

\setbeamertemplate{navigation symbols}{}

%=============================================================================
% CUSTOM HEADLINE
%=============================================================================
\setbeamertemplate{headline}{%
    \vskip10pt%
    \hbox to \paperwidth{%
        \hskip0.5cm%
        {\small\color{FooterGray}\renewcommand{\hyperlink}[2]{##2}\insertsectionhead}%
        \hfill%
        \textcolor{FooterGray}{\small\insertframenumber}%
        \hskip0.5cm%
    }%
    \vskip4pt%
    {\color{FooterGray}\hrule height 0.4pt}%
}

%=============================================================================
% CUSTOM FOOTER
%=============================================================================
\usepackage{fontawesome5}

\setbeamertemplate{footline}{%
    {\color{FooterGray}\hrule height 0.4pt}%
    \vskip4pt%
    \hbox to \paperwidth{%
        \hskip0.5cm%
        \textcolor{FooterGray}{\small Time Series Analysis and Forecasting}%
        \hfill%
        \raisebox{-0.1em}{%
            \begin{tikzpicture}[x=0.08em, y=0.08em, line width=0.4pt]
                \draw[FooterGray] (0,3) -- (1,4) -- (2,3.5) -- (3,5) -- (4,4) -- (5,6) -- (6,5.5) -- (7,4) -- (8,5) -- (9,7) -- (10,6) -- (11,5) -- (12,6.5) -- (13,8) -- (14,7) -- (15,6) -- (16,7.5) -- (17,9) -- (18,8) -- (19,7) -- (20,8.5) -- (21,10) -- (22,9) -- (23,8) -- (24,9.5);
            \end{tikzpicture}%
        }%
        \hskip0.5cm%
    }%
    \vskip6pt%
}

%=============================================================================
% PACKAGES
%=============================================================================
\usepackage[utf8]{inputenc}
\usepackage[T1]{fontenc}
\usepackage[english]{babel}
\usepackage{amsmath, amssymb, amsthm}
\usepackage{mathtools}
\usepackage{bm}
\usepackage{tikz}
\usetikzlibrary{arrows.meta, positioning, shapes, calc, decorations.pathreplacing, shadings}
\usepackage{booktabs}
\usepackage{multirow}
\usepackage{array}
\usepackage{graphicx}
\usepackage{hyperref}
\usepackage{colortbl}
\usepackage{listings}
\lstset{basicstyle=\ttfamily\small, breaklines=true, frame=single, backgroundcolor=\color{VeryLightGray}}
\hypersetup{colorlinks=true, linkcolor=MainBlue, urlcolor=MainBlue}
\graphicspath{{../../logos/}{../../charts/}{../../photos/}}
\hfuzz=2pt  % Suppress tiny overfull warnings (<2pt)
\vfuzz=2pt  % Suppress tiny vertical overfull warnings (<2pt)

%=============================================================================
% QUANTLET COMMAND
%=============================================================================
\newcommand{\quantlet}[2]{%
    \hfill\href{#2}{%
        \raisebox{-0.15em}{\includegraphics[height=0.7em]{ql_logo.png}}%
        \textcolor{MainBlue}{\tiny\ #1}%
    }%
}

%=============================================================================
% CUSTOM TITLE PAGE
%=============================================================================
\defbeamertemplate*{title page}{hybrid}[1][]
{
    \vspace{0.2cm}
    % Logos row - top header (with clickable links)
    \begin{center}
        \href{https://www.ase.ro}{\includegraphics[height=1.0cm]{ase_logo.png}}\hspace{0.25cm}%
        \href{https://theida.net}{\includegraphics[height=1.0cm]{ida_logo.png}}\hspace{0.25cm}%
        \href{https://blockchain-research-center.com}{\includegraphics[height=1.0cm]{brc_logo.png}}\hspace{0.25cm}%
        \href{https://www.ai4efin.ase.ro}{\includegraphics[height=1.0cm]{ai4efin_logo.png}}\hspace{0.25cm}%
        \href{https://ipe.ro/new}{\includegraphics[height=1.0cm]{acad_logo.png}}\hspace{0.25cm}%
        \href{https://www.digital-finance-msca.com}{\includegraphics[height=1.0cm]{msca_logo.png}}%
    \end{center}

    \vspace{0.6cm}

    % Main title with Q logos on sides (with clickable links)
    \begin{center}
        \begin{minipage}{0.1\textwidth}
            \centering
            \href{https://quantlet.com}{\includegraphics[height=1.1cm]{ql_logo.png}}
        \end{minipage}%
        \begin{minipage}{0.78\textwidth}
            \centering
            {\LARGE\bfseries\usebeamercolor[fg]{title}\inserttitle}

            \vspace{0.3cm}

            {\usebeamerfont{subtitle}\usebeamercolor[fg]{title}\insertsubtitle}
        \end{minipage}%
        \begin{minipage}{0.1\textwidth}
            \centering
            \href{https://quantinar.com}{\includegraphics[height=1.1cm]{qr_logo.png}}
        \end{minipage}
    \end{center}

    \vspace{0.6cm}

    % Authors (left aligned)
    \hspace{0.5cm}{\usebeamerfont{author}\insertauthor}

    \vspace{0.3cm}

    % Institute/Affiliations (left aligned)
    \hspace{0.5cm}\begin{minipage}[t]{0.9\textwidth}
        \raggedright\small\insertinstitute
    \end{minipage}
}

%=============================================================================
% THEOREM ENVIRONMENTS
%=============================================================================
\theoremstyle{definition}
\setbeamertemplate{theorems}[numbered]
\newtheorem{defn}{Definition}
\newtheorem{thm}{Theorem}
\newtheorem{prop}{Proposition}
\newtheorem{rmk}{Remark}

%=============================================================================
% CENTRED MINIPAGE (no extra vertical space)
%=============================================================================
\newenvironment{cminipage}[1]{%
    \par\noindent\hfill\begin{minipage}{#1}\ignorespaces
}{%
    \end{minipage}\hfill\null\par
}

%=============================================================================
% CUSTOM COMMANDS
%=============================================================================
\newcommand{\E}{\mathbb{E}}
\newcommand{\Var}{\text{Var}}
\newcommand{\Cov}{\text{Cov}}
\newcommand{\Corr}{\text{Corr}}
\newcommand{\R}{\mathbb{R}}
\newcommand{\N}{\mathbb{N}}
\newcommand{\Z}{\mathbb{Z}}
\newcommand{\B}{\mathbf{B}}
\newcommand{\imark}{\textcolor{MainBlue}{\textbullet}}
\newcommand{\RMSE}{\text{RMSE}}
\newcommand{\MAE}{\text{MAE}}
\newcommand{\MAPE}{\text{MAPE}}
\newcommand{\correct}{\textcolor{Forest}{\checkmark}}
\newcommand{\incorrect}{\textcolor{Crimson}{\texttimes}}

% Boldface vector/matrix commands
\newcommand{\bY}{\mathbf{Y}}
\newcommand{\bX}{\mathbf{X}}
\newcommand{\bA}{\mathbf{A}}
\newcommand{\bB}{\mathbf{B}}
\newcommand{\bepsilon}{\boldsymbol{\varepsilon}}
\newcommand{\bvarepsilon}{\boldsymbol{\varepsilon}}
\newcommand{\bSigma}{\boldsymbol{\Sigma}}
\newcommand{\bPhi}{\boldsymbol{\Phi}}
\newcommand{\bGamma}{\boldsymbol{\Gamma}}
\newcommand{\bPi}{\boldsymbol{\Pi}}
\newcommand{\bc}{\mathbf{c}}
\newcommand{\balpha}{\boldsymbol{\alpha}}
\newcommand{\bbeta}{\boldsymbol{\beta}}

%=============================================================================
% TITLE INFORMATION
%=============================================================================
\title[Time Series Analysis]{Time Series Analysis and Forecasting}
\author[D.T. Pele]{Daniel Traian PELE}
\institute{Bucharest University of Economic Studies\\
IDA Institute Digital Assets\\
Blockchain Research Center\\
AI4EFin Artificial Intelligence for Energy Finance\\
Romanian Academy, Institute for Economic Forecasting\\
MSCA Digital Finance}
\date{}

%            \subtitle{Seminar X: Seminar Title}
%            \begin{document} ...
%=============================================================================

% Ensure content fits on slides
\setbeamersize{text margin left=8mm, text margin right=8mm}

%=============================================================================
% THEME AND STYLE CONFIGURATION
%=============================================================================
\usetheme{default}
% Using default theme for clean header/footer control

% Color Palette (matching Redispatch PDF)
\definecolor{MainBlue}{RGB}{26, 58, 110}
\definecolor{AccentBlue}{RGB}{26, 58, 110}
\definecolor{IDAred}{RGB}{205, 0, 0}
\definecolor{DarkGray}{RGB}{51, 51, 51}
\definecolor{MediumGray}{RGB}{128, 128, 128}
\definecolor{LightGray}{RGB}{248, 248, 248}
\definecolor{VeryLightGray}{RGB}{235, 235, 235}
\definecolor{KeynoteGray}{RGB}{218, 218, 218}
\definecolor{SectionGray}{RGB}{120, 120, 120}
\definecolor{FooterGray}{RGB}{100, 100, 100}
\definecolor{Crimson}{RGB}{220, 53, 69}
\definecolor{Forest}{RGB}{46, 125, 50}
\definecolor{Amber}{RGB}{181, 133, 63}
\definecolor{Orange}{RGB}{230, 126, 34}
\definecolor{Purple}{RGB}{142, 68, 173}

% Gradient background (exact Keynote 315° gradient: white to RGB 218,218,218)
\setbeamertemplate{background}{%
    \begin{tikzpicture}[remember picture, overlay]
        \shade[shading=axis, shading angle=315,
        top color=white, bottom color=KeynoteGray]
        (current page.south west) rectangle (current page.north east);
    \end{tikzpicture}%
}
% Fallback solid color for compatibility
\setbeamercolor{background canvas}{bg=}

\setbeamercolor{palette primary}{bg=MainBlue, fg=white}
\setbeamercolor{palette secondary}{bg=MainBlue!85, fg=white}
\setbeamercolor{palette tertiary}{bg=MainBlue!70, fg=white}
\setbeamercolor{structure}{fg=MainBlue}
\setbeamercolor{title}{fg=IDAred}
\setbeamercolor{frametitle}{fg=IDAred, bg=}
\setbeamercolor{block title}{bg=MainBlue, fg=white}
\setbeamercolor{block body}{bg=VeryLightGray, fg=DarkGray}
\setbeamercolor{block title alerted}{bg=Crimson, fg=white}
\setbeamercolor{block body alerted}{bg=Crimson!8, fg=DarkGray}
\setbeamercolor{block title example}{bg=Forest, fg=white}
\setbeamercolor{block body example}{bg=Forest!8, fg=DarkGray}
\setbeamercolor{item}{fg=MainBlue}

% Smaller institute font to avoid overfull hbox on title page
\setbeamerfont{institute}{size=\footnotesize}

% Footer colors (override Madrid theme blue)
\setbeamercolor{author in head/foot}{fg=FooterGray, bg=}
\setbeamercolor{title in head/foot}{fg=FooterGray, bg=}
\setbeamercolor{date in head/foot}{fg=FooterGray, bg=}
\setbeamercolor{section in head/foot}{fg=FooterGray, bg=}
\setbeamercolor{subsection in head/foot}{fg=FooterGray, bg=}

% Bullet styles (apply everywhere including blocks)
\setbeamertemplate{itemize item}{\color{MainBlue}$\boxdot$}
\setbeamertemplate{itemize subitem}{\color{MainBlue}$\blacktriangleright$}
\setbeamertemplate{itemize subsubitem}{\color{MainBlue}\tiny$\bullet$}
\setbeamertemplate{itemize/enumerate body begin}{\normalsize}
\setbeamertemplate{itemize/enumerate subbody begin}{\normalsize}

% Item spacing - compact style
\setlength{\leftmargini}{10pt}       % Level 1: minimal indent
\setlength{\leftmarginii}{10pt}      % Level 2: minimal additional indent
% Compact list spacing (zero extra space before/after lists in blocks)
\makeatletter
\def\@listi{\leftmargin\leftmargini \topsep 0pt \parsep 0pt \itemsep 0pt}
\def\@listii{\leftmargin\leftmarginii \topsep 0pt \parsep 0pt \itemsep 0pt}
\makeatother

\setbeamertemplate{navigation symbols}{}

%=============================================================================
% CUSTOM HEADLINE
%=============================================================================
\setbeamertemplate{headline}{%
    \vskip10pt%
    \hbox to \paperwidth{%
        \hskip0.5cm%
        {\small\color{FooterGray}\renewcommand{\hyperlink}[2]{##2}\insertsectionhead}%
        \hfill%
        \textcolor{FooterGray}{\small\insertframenumber}%
        \hskip0.5cm%
    }%
    \vskip4pt%
    {\color{FooterGray}\hrule height 0.4pt}%
}

%=============================================================================
% CUSTOM FOOTER
%=============================================================================
\usepackage{fontawesome5}

\setbeamertemplate{footline}{%
    {\color{FooterGray}\hrule height 0.4pt}%
    \vskip4pt%
    \hbox to \paperwidth{%
        \hskip0.5cm%
        \textcolor{FooterGray}{\small Time Series Analysis and Forecasting}%
        \hfill%
        \raisebox{-0.1em}{%
            \begin{tikzpicture}[x=0.08em, y=0.08em, line width=0.4pt]
                \draw[FooterGray] (0,3) -- (1,4) -- (2,3.5) -- (3,5) -- (4,4) -- (5,6) -- (6,5.5) -- (7,4) -- (8,5) -- (9,7) -- (10,6) -- (11,5) -- (12,6.5) -- (13,8) -- (14,7) -- (15,6) -- (16,7.5) -- (17,9) -- (18,8) -- (19,7) -- (20,8.5) -- (21,10) -- (22,9) -- (23,8) -- (24,9.5);
            \end{tikzpicture}%
        }%
        \hskip0.5cm%
    }%
    \vskip6pt%
}

%=============================================================================
% PACKAGES
%=============================================================================
\usepackage[utf8]{inputenc}
\usepackage[T1]{fontenc}
\usepackage[english]{babel}
\usepackage{amsmath, amssymb, amsthm}
\usepackage{mathtools}
\usepackage{bm}
\usepackage{tikz}
\usetikzlibrary{arrows.meta, positioning, shapes, calc, decorations.pathreplacing, shadings}
\usepackage{booktabs}
\usepackage{multirow}
\usepackage{array}
\usepackage{graphicx}
\usepackage{hyperref}
\usepackage{colortbl}
\usepackage{listings}
\lstset{basicstyle=\ttfamily\small, breaklines=true, frame=single, backgroundcolor=\color{VeryLightGray}}
\hypersetup{colorlinks=true, linkcolor=MainBlue, urlcolor=MainBlue}
\graphicspath{{../../logos/}{../../charts/}{../../photos/}}
\hfuzz=2pt  % Suppress tiny overfull warnings (<2pt)
\vfuzz=2pt  % Suppress tiny vertical overfull warnings (<2pt)

%=============================================================================
% QUANTLET COMMAND
%=============================================================================
\newcommand{\quantlet}[2]{%
    \hfill\href{#2}{%
        \raisebox{-0.15em}{\includegraphics[height=0.7em]{ql_logo.png}}%
        \textcolor{MainBlue}{\tiny\ #1}%
    }%
}

%=============================================================================
% CUSTOM TITLE PAGE
%=============================================================================
\defbeamertemplate*{title page}{hybrid}[1][]
{
    \vspace{0.2cm}
    % Logos row - top header (with clickable links)
    \begin{center}
        \href{https://www.ase.ro}{\includegraphics[height=1.0cm]{ase_logo.png}}\hspace{0.25cm}%
        \href{https://theida.net}{\includegraphics[height=1.0cm]{ida_logo.png}}\hspace{0.25cm}%
        \href{https://blockchain-research-center.com}{\includegraphics[height=1.0cm]{brc_logo.png}}\hspace{0.25cm}%
        \href{https://www.ai4efin.ase.ro}{\includegraphics[height=1.0cm]{ai4efin_logo.png}}\hspace{0.25cm}%
        \href{https://ipe.ro/new}{\includegraphics[height=1.0cm]{acad_logo.png}}\hspace{0.25cm}%
        \href{https://www.digital-finance-msca.com}{\includegraphics[height=1.0cm]{msca_logo.png}}%
    \end{center}

    \vspace{0.6cm}

    % Main title with Q logos on sides (with clickable links)
    \begin{center}
        \begin{minipage}{0.1\textwidth}
            \centering
            \href{https://quantlet.com}{\includegraphics[height=1.1cm]{ql_logo.png}}
        \end{minipage}%
        \begin{minipage}{0.78\textwidth}
            \centering
            {\LARGE\bfseries\usebeamercolor[fg]{title}\inserttitle}

            \vspace{0.3cm}

            {\usebeamerfont{subtitle}\usebeamercolor[fg]{title}\insertsubtitle}
        \end{minipage}%
        \begin{minipage}{0.1\textwidth}
            \centering
            \href{https://quantinar.com}{\includegraphics[height=1.1cm]{qr_logo.png}}
        \end{minipage}
    \end{center}

    \vspace{0.6cm}

    % Authors (left aligned)
    \hspace{0.5cm}{\usebeamerfont{author}\insertauthor}

    \vspace{0.3cm}

    % Institute/Affiliations (left aligned)
    \hspace{0.5cm}\begin{minipage}[t]{0.9\textwidth}
        \raggedright\small\insertinstitute
    \end{minipage}
}

%=============================================================================
% THEOREM ENVIRONMENTS
%=============================================================================
\theoremstyle{definition}
\setbeamertemplate{theorems}[numbered]
\newtheorem{defn}{Definition}
\newtheorem{thm}{Theorem}
\newtheorem{prop}{Proposition}
\newtheorem{rmk}{Remark}

%=============================================================================
% CENTRED MINIPAGE (no extra vertical space)
%=============================================================================
\newenvironment{cminipage}[1]{%
    \par\noindent\hfill\begin{minipage}{#1}\ignorespaces
}{%
    \end{minipage}\hfill\null\par
}

%=============================================================================
% CUSTOM COMMANDS
%=============================================================================
\newcommand{\E}{\mathbb{E}}
\newcommand{\Var}{\text{Var}}
\newcommand{\Cov}{\text{Cov}}
\newcommand{\Corr}{\text{Corr}}
\newcommand{\R}{\mathbb{R}}
\newcommand{\N}{\mathbb{N}}
\newcommand{\Z}{\mathbb{Z}}
\newcommand{\B}{\mathbf{B}}
\newcommand{\imark}{\textcolor{MainBlue}{\textbullet}}
\newcommand{\RMSE}{\text{RMSE}}
\newcommand{\MAE}{\text{MAE}}
\newcommand{\MAPE}{\text{MAPE}}
\newcommand{\correct}{\textcolor{Forest}{\checkmark}}
\newcommand{\incorrect}{\textcolor{Crimson}{\texttimes}}

% Boldface vector/matrix commands
\newcommand{\bY}{\mathbf{Y}}
\newcommand{\bX}{\mathbf{X}}
\newcommand{\bA}{\mathbf{A}}
\newcommand{\bB}{\mathbf{B}}
\newcommand{\bepsilon}{\boldsymbol{\varepsilon}}
\newcommand{\bvarepsilon}{\boldsymbol{\varepsilon}}
\newcommand{\bSigma}{\boldsymbol{\Sigma}}
\newcommand{\bPhi}{\boldsymbol{\Phi}}
\newcommand{\bGamma}{\boldsymbol{\Gamma}}
\newcommand{\bPi}{\boldsymbol{\Pi}}
\newcommand{\bc}{\mathbf{c}}
\newcommand{\balpha}{\boldsymbol{\alpha}}
\newcommand{\bbeta}{\boldsymbol{\beta}}

%=============================================================================
% TITLE INFORMATION
%=============================================================================
\title[Time Series Analysis]{Time Series Analysis and Forecasting}
\author[D.T. Pele]{Daniel Traian PELE}
\institute{Bucharest University of Economic Studies\\
IDA Institute Digital Assets\\
Blockchain Research Center\\
AI4EFin Artificial Intelligence for Energy Finance\\
Romanian Academy, Institute for Economic Forecasting\\
MSCA Digital Finance}
\date{}

%            \subtitle{Seminar X: Seminar Title}
%            \begin{document} ...
%=============================================================================

% Ensure content fits on slides
\setbeamersize{text margin left=8mm, text margin right=8mm}

%=============================================================================
% THEME AND STYLE CONFIGURATION
%=============================================================================
\usetheme{default}
% Using default theme for clean header/footer control

% Color Palette (matching Redispatch PDF)
\definecolor{MainBlue}{RGB}{26, 58, 110}
\definecolor{AccentBlue}{RGB}{26, 58, 110}
\definecolor{IDAred}{RGB}{205, 0, 0}
\definecolor{DarkGray}{RGB}{51, 51, 51}
\definecolor{MediumGray}{RGB}{128, 128, 128}
\definecolor{LightGray}{RGB}{248, 248, 248}
\definecolor{VeryLightGray}{RGB}{235, 235, 235}
\definecolor{KeynoteGray}{RGB}{218, 218, 218}
\definecolor{SectionGray}{RGB}{120, 120, 120}
\definecolor{FooterGray}{RGB}{100, 100, 100}
\definecolor{Crimson}{RGB}{220, 53, 69}
\definecolor{Forest}{RGB}{46, 125, 50}
\definecolor{Amber}{RGB}{181, 133, 63}
\definecolor{Orange}{RGB}{230, 126, 34}
\definecolor{Purple}{RGB}{142, 68, 173}

% Gradient background (exact Keynote 315° gradient: white to RGB 218,218,218)
\setbeamertemplate{background}{%
    \begin{tikzpicture}[remember picture, overlay]
        \shade[shading=axis, shading angle=315,
        top color=white, bottom color=KeynoteGray]
        (current page.south west) rectangle (current page.north east);
    \end{tikzpicture}%
}
% Fallback solid color for compatibility
\setbeamercolor{background canvas}{bg=}

\setbeamercolor{palette primary}{bg=MainBlue, fg=white}
\setbeamercolor{palette secondary}{bg=MainBlue!85, fg=white}
\setbeamercolor{palette tertiary}{bg=MainBlue!70, fg=white}
\setbeamercolor{structure}{fg=MainBlue}
\setbeamercolor{title}{fg=IDAred}
\setbeamercolor{frametitle}{fg=IDAred, bg=}
\setbeamercolor{block title}{bg=MainBlue, fg=white}
\setbeamercolor{block body}{bg=VeryLightGray, fg=DarkGray}
\setbeamercolor{block title alerted}{bg=Crimson, fg=white}
\setbeamercolor{block body alerted}{bg=Crimson!8, fg=DarkGray}
\setbeamercolor{block title example}{bg=Forest, fg=white}
\setbeamercolor{block body example}{bg=Forest!8, fg=DarkGray}
\setbeamercolor{item}{fg=MainBlue}

% Smaller institute font to avoid overfull hbox on title page
\setbeamerfont{institute}{size=\footnotesize}

% Footer colors (override Madrid theme blue)
\setbeamercolor{author in head/foot}{fg=FooterGray, bg=}
\setbeamercolor{title in head/foot}{fg=FooterGray, bg=}
\setbeamercolor{date in head/foot}{fg=FooterGray, bg=}
\setbeamercolor{section in head/foot}{fg=FooterGray, bg=}
\setbeamercolor{subsection in head/foot}{fg=FooterGray, bg=}

% Bullet styles (apply everywhere including blocks)
\setbeamertemplate{itemize item}{\color{MainBlue}$\boxdot$}
\setbeamertemplate{itemize subitem}{\color{MainBlue}$\blacktriangleright$}
\setbeamertemplate{itemize subsubitem}{\color{MainBlue}\tiny$\bullet$}
\setbeamertemplate{itemize/enumerate body begin}{\normalsize}
\setbeamertemplate{itemize/enumerate subbody begin}{\normalsize}

% Item spacing - compact style
\setlength{\leftmargini}{10pt}       % Level 1: minimal indent
\setlength{\leftmarginii}{10pt}      % Level 2: minimal additional indent
% Compact list spacing (zero extra space before/after lists in blocks)
\makeatletter
\def\@listi{\leftmargin\leftmargini \topsep 0pt \parsep 0pt \itemsep 0pt}
\def\@listii{\leftmargin\leftmarginii \topsep 0pt \parsep 0pt \itemsep 0pt}
\makeatother

\setbeamertemplate{navigation symbols}{}

%=============================================================================
% CUSTOM HEADLINE
%=============================================================================
\setbeamertemplate{headline}{%
    \vskip10pt%
    \hbox to \paperwidth{%
        \hskip0.5cm%
        {\small\color{FooterGray}\renewcommand{\hyperlink}[2]{##2}\insertsectionhead}%
        \hfill%
        \textcolor{FooterGray}{\small\insertframenumber}%
        \hskip0.5cm%
    }%
    \vskip4pt%
    {\color{FooterGray}\hrule height 0.4pt}%
}

%=============================================================================
% CUSTOM FOOTER
%=============================================================================
\usepackage{fontawesome5}

\setbeamertemplate{footline}{%
    {\color{FooterGray}\hrule height 0.4pt}%
    \vskip4pt%
    \hbox to \paperwidth{%
        \hskip0.5cm%
        \textcolor{FooterGray}{\small Time Series Analysis and Forecasting}%
        \hfill%
        \raisebox{-0.1em}{%
            \begin{tikzpicture}[x=0.08em, y=0.08em, line width=0.4pt]
                \draw[FooterGray] (0,3) -- (1,4) -- (2,3.5) -- (3,5) -- (4,4) -- (5,6) -- (6,5.5) -- (7,4) -- (8,5) -- (9,7) -- (10,6) -- (11,5) -- (12,6.5) -- (13,8) -- (14,7) -- (15,6) -- (16,7.5) -- (17,9) -- (18,8) -- (19,7) -- (20,8.5) -- (21,10) -- (22,9) -- (23,8) -- (24,9.5);
            \end{tikzpicture}%
        }%
        \hskip0.5cm%
    }%
    \vskip6pt%
}

%=============================================================================
% PACKAGES
%=============================================================================
\usepackage[utf8]{inputenc}
\usepackage[T1]{fontenc}
\usepackage[english]{babel}
\usepackage{amsmath, amssymb, amsthm}
\usepackage{mathtools}
\usepackage{bm}
\usepackage{tikz}
\usetikzlibrary{arrows.meta, positioning, shapes, calc, decorations.pathreplacing, shadings}
\usepackage{booktabs}
\usepackage{multirow}
\usepackage{array}
\usepackage{graphicx}
\usepackage{hyperref}
\usepackage{colortbl}
\usepackage{listings}
\lstset{basicstyle=\ttfamily\small, breaklines=true, frame=single, backgroundcolor=\color{VeryLightGray}}
\hypersetup{colorlinks=true, linkcolor=MainBlue, urlcolor=MainBlue}
\graphicspath{{../../logos/}{../../charts/}{../../photos/}}
\hfuzz=2pt  % Suppress tiny overfull warnings (<2pt)
\vfuzz=2pt  % Suppress tiny vertical overfull warnings (<2pt)

%=============================================================================
% QUANTLET COMMAND
%=============================================================================
\newcommand{\quantlet}[2]{%
    \hfill\href{#2}{%
        \raisebox{-0.15em}{\includegraphics[height=0.7em]{ql_logo.png}}%
        \textcolor{MainBlue}{\tiny\ #1}%
    }%
}

%=============================================================================
% CUSTOM TITLE PAGE
%=============================================================================
\defbeamertemplate*{title page}{hybrid}[1][]
{
    \vspace{0.2cm}
    % Logos row - top header (with clickable links)
    \begin{center}
        \href{https://www.ase.ro}{\includegraphics[height=1.0cm]{ase_logo.png}}\hspace{0.25cm}%
        \href{https://theida.net}{\includegraphics[height=1.0cm]{ida_logo.png}}\hspace{0.25cm}%
        \href{https://blockchain-research-center.com}{\includegraphics[height=1.0cm]{brc_logo.png}}\hspace{0.25cm}%
        \href{https://www.ai4efin.ase.ro}{\includegraphics[height=1.0cm]{ai4efin_logo.png}}\hspace{0.25cm}%
        \href{https://ipe.ro/new}{\includegraphics[height=1.0cm]{acad_logo.png}}\hspace{0.25cm}%
        \href{https://www.digital-finance-msca.com}{\includegraphics[height=1.0cm]{msca_logo.png}}%
    \end{center}

    \vspace{0.6cm}

    % Main title with Q logos on sides (with clickable links)
    \begin{center}
        \begin{minipage}{0.1\textwidth}
            \centering
            \href{https://quantlet.com}{\includegraphics[height=1.1cm]{ql_logo.png}}
        \end{minipage}%
        \begin{minipage}{0.78\textwidth}
            \centering
            {\LARGE\bfseries\usebeamercolor[fg]{title}\inserttitle}

            \vspace{0.3cm}

            {\usebeamerfont{subtitle}\usebeamercolor[fg]{title}\insertsubtitle}
        \end{minipage}%
        \begin{minipage}{0.1\textwidth}
            \centering
            \href{https://quantinar.com}{\includegraphics[height=1.1cm]{qr_logo.png}}
        \end{minipage}
    \end{center}

    \vspace{0.6cm}

    % Authors (left aligned)
    \hspace{0.5cm}{\usebeamerfont{author}\insertauthor}

    \vspace{0.3cm}

    % Institute/Affiliations (left aligned)
    \hspace{0.5cm}\begin{minipage}[t]{0.9\textwidth}
        \raggedright\small\insertinstitute
    \end{minipage}
}

%=============================================================================
% THEOREM ENVIRONMENTS
%=============================================================================
\theoremstyle{definition}
\setbeamertemplate{theorems}[numbered]
\newtheorem{defn}{Definition}
\newtheorem{thm}{Theorem}
\newtheorem{prop}{Proposition}
\newtheorem{rmk}{Remark}

%=============================================================================
% CENTRED MINIPAGE (no extra vertical space)
%=============================================================================
\newenvironment{cminipage}[1]{%
    \par\noindent\hfill\begin{minipage}{#1}\ignorespaces
}{%
    \end{minipage}\hfill\null\par
}

%=============================================================================
% CUSTOM COMMANDS
%=============================================================================
\newcommand{\E}{\mathbb{E}}
\newcommand{\Var}{\text{Var}}
\newcommand{\Cov}{\text{Cov}}
\newcommand{\Corr}{\text{Corr}}
\newcommand{\R}{\mathbb{R}}
\newcommand{\N}{\mathbb{N}}
\newcommand{\Z}{\mathbb{Z}}
\newcommand{\B}{\mathbf{B}}
\newcommand{\imark}{\textcolor{MainBlue}{\textbullet}}
\newcommand{\RMSE}{\text{RMSE}}
\newcommand{\MAE}{\text{MAE}}
\newcommand{\MAPE}{\text{MAPE}}
\newcommand{\correct}{\textcolor{Forest}{\checkmark}}
\newcommand{\incorrect}{\textcolor{Crimson}{\texttimes}}

% Boldface vector/matrix commands
\newcommand{\bY}{\mathbf{Y}}
\newcommand{\bX}{\mathbf{X}}
\newcommand{\bA}{\mathbf{A}}
\newcommand{\bB}{\mathbf{B}}
\newcommand{\bepsilon}{\boldsymbol{\varepsilon}}
\newcommand{\bvarepsilon}{\boldsymbol{\varepsilon}}
\newcommand{\bSigma}{\boldsymbol{\Sigma}}
\newcommand{\bPhi}{\boldsymbol{\Phi}}
\newcommand{\bGamma}{\boldsymbol{\Gamma}}
\newcommand{\bPi}{\boldsymbol{\Pi}}
\newcommand{\bc}{\mathbf{c}}
\newcommand{\balpha}{\boldsymbol{\alpha}}
\newcommand{\bbeta}{\boldsymbol{\beta}}

%=============================================================================
% TITLE INFORMATION
%=============================================================================
\title[Time Series Analysis]{Time Series Analysis and Forecasting}
\author[D.T. Pele]{Daniel Traian PELE}
\institute{Bucharest University of Economic Studies\\
IDA Institute Digital Assets\\
Blockchain Research Center\\
AI4EFin Artificial Intelligence for Energy Finance\\
Romanian Academy, Institute for Economic Forecasting\\
MSCA Digital Finance}
\date{}

\subtitle{Seminar 6: VAR and Granger Causality}

\begin{document}

{
\setbeamertemplate{headline}{}
\setbeamertemplate{footline}{}
\begin{frame}
    \titlepage
\end{frame}
}


\begin{frame}{Seminar Outline}
    \begin{cminipage}{0.95\textwidth}
    \begin{itemize}
        \item \textbf{Multiple Choice Quiz} -- Knowledge check
        \vspace{0.15cm}
        \item \textbf{True/False} -- Conceptual checks
        \vspace{0.15cm}
        \item \textbf{Calculation Exercises} -- Applied practice
        \vspace{0.15cm}
        \item \textbf{Worked Examples} -- Detailed solutions
        \vspace{0.15cm}
        \item \textbf{Real Data Analysis} -- Empirical applications
        \vspace{0.15cm}
        \item \textbf{AI-Assisted Exercise} -- Critical thinking
        \vspace{0.15cm}
        \item \textbf{Summary} -- Key takeaways
    \end{itemize}
    \end{cminipage}
\end{frame}

%=============================================================================
% MULTIPLE CHOICE QUIZ
%=============================================================================
\section{Multiple Choice Quiz}

\begin{frame}{Quiz 1: VAR Definition}
    \begin{cminipage}{0.95\textwidth}
    \begin{alertblock}{Question}
        In a VAR(2) model with 3 variables, how many coefficient matrices $\bA_i$ are there?
    \end{alertblock}

    \vspace{0.4cm}

    \begin{block}{Answer choices}
        \textcolor{MainBlue}{\textbf{(A)}} 2\\[3pt]
        \textcolor{MainBlue}{\textbf{(B)}} 3\\[3pt]
        \textcolor{MainBlue}{\textbf{(C)}} 6\\[3pt]
        \textcolor{MainBlue}{\textbf{(D)}} 9
    \end{block}

    \vspace{0.5cm}

    \begin{center}
        \textit{Answer on next slide...}
    \end{center}
    \end{cminipage}
\end{frame}

\begin{frame}{Quiz 1: Answer}
    \begin{cminipage}{0.95\textwidth}
    \begin{exampleblock}{Answer: A -- 2 coefficient matrices}
        \textbf{VAR($p$) model}: $\bY_t = \bc + \bA_1\bY_{t-1} + \bA_2\bY_{t-2} + \cdots + \bA_p\bY_{t-p} + \bvarepsilon_t$

        \vspace{0.3cm}
        \textbf{VAR(2) with $K=3$}:
        \[
        \begin{pmatrix} Y_{1t} \\ Y_{2t} \\ Y_{3t} \end{pmatrix} = \bc + \underbrace{\bA_1}_{3\times 3}\begin{pmatrix} Y_{1,t-1} \\ Y_{2,t-1} \\ Y_{3,t-1} \end{pmatrix} + \underbrace{\bA_2}_{3\times 3}\begin{pmatrix} Y_{1,t-2} \\ Y_{2,t-2} \\ Y_{3,t-2} \end{pmatrix} + \bvarepsilon_t
        \]

        \textbf{Key}: $p$ = number of lags = number of matrices
    \end{exampleblock}

    \end{cminipage}
    \quantlet{TSA\_ch6\_var\_results}{https://github.com/QuantLet/TSA/tree/main/TSA_ch6/TSA_ch6_var_results}
\end{frame}

\begin{frame}{Quiz 2: Number of Parameters}
    \begin{cminipage}{0.95\textwidth}
    \begin{alertblock}{Question}
        A VAR(2) with $K=3$ variables (including constants) has how many parameters to estimate per equation?
    \end{alertblock}

    \vspace{0.4cm}

    \begin{block}{Answer choices}
        \textcolor{MainBlue}{\textbf{(A)}} 3\\[3pt]
        \textcolor{MainBlue}{\textbf{(B)}} 6\\[3pt]
        \textcolor{MainBlue}{\textbf{(C)}} 7\\[3pt]
        \textcolor{MainBlue}{\textbf{(D)}} 9
    \end{block}

    \vspace{0.5cm}

    \begin{center}
        \textit{Answer on next slide...}
    \end{center}
    \end{cminipage}
\end{frame}

\begin{frame}{Quiz 2: Answer}
    \begin{cminipage}{0.95\textwidth}
    \begin{exampleblock}{Answer: C -- 7 parameters per equation}
        \begin{center}
            \includegraphics[width=0.95\textwidth, height=0.55\textheight, keepaspectratio]{sem6_var_parameters.pdf}
        \end{center}
        \vspace{-0.2cm}
        {\footnotesize
        \textbf{Formula}: Per equation = $1 + K \times p = 1 + 3 \times 2 = 7$. \textbf{Total}: $K(1 + Kp) = 3(1 + 6) = 21$ parameters
        }
    \end{exampleblock}

    \end{cminipage}
    \quantlet{TSA\_ch6\_var\_simulation}{https://github.com/QuantLet/TSA/tree/main/TSA_ch6/TSA_ch6_var_simulation}
\end{frame}

\begin{frame}{Quiz 3: Granger Causality}
    \begin{cminipage}{0.95\textwidth}
    \begin{alertblock}{Question}
        ``$X$ Granger-causes $Y$'' means:
    \end{alertblock}

    \vspace{0.4cm}

    \begin{block}{Answer choices}
        \textcolor{MainBlue}{\textbf{(A)}} $X$ is the economic cause of $Y$\\[3pt]
        \textcolor{MainBlue}{\textbf{(B)}} Past $X$ helps predict future $Y$\\[3pt]
        \textcolor{MainBlue}{\textbf{(C)}} $X$ and $Y$ are contemporaneously correlated\\[3pt]
        \textcolor{MainBlue}{\textbf{(D)}} $X$ always increases when $Y$ increases
    \end{block}

    \vspace{0.5cm}

    \begin{center}
        \textit{Answer on next slide...}
    \end{center}
    \end{cminipage}
\end{frame}

\begin{frame}{Quiz 3: Answer}
    \begin{cminipage}{0.95\textwidth}
    \begin{exampleblock}{Answer: B -- Past $X$ helps predict future $Y$}
        \begin{center}
            \includegraphics[width=0.9\textwidth, height=0.55\textheight, keepaspectratio]{sem6_granger_diagram.pdf}
        \end{center}
        \vspace{-0.2cm}
        {\footnotesize
        \textbf{Key}: Predictive relationship, NOT true causation!
        }
    \end{exampleblock}

    \end{cminipage}
    \quantlet{TSA\_ch6\_case\_granger}{https://github.com/QuantLet/TSA/tree/main/TSA_ch6/TSA_ch6_case_granger}
\end{frame}

\begin{frame}{Quiz 4: Granger Causality Test}
    \begin{cminipage}{0.95\textwidth}
    \begin{alertblock}{Question}
        To test if $Y_2$ Granger-causes $Y_1$ in a VAR(p), we test:
    \end{alertblock}

    \vspace{0.4cm}

    \begin{block}{Answer choices}
        \textcolor{MainBlue}{\textbf{(A)}} All coefficients in the $Y_1$ equation equal zero\\[3pt]
        \textcolor{MainBlue}{\textbf{(B)}} Coefficients on lagged $Y_2$ in the $Y_1$ equation equal zero\\[3pt]
        \textcolor{MainBlue}{\textbf{(C)}} Coefficients on lagged $Y_1$ in the $Y_2$ equation equal zero\\[3pt]
        \textcolor{MainBlue}{\textbf{(D)}} The error covariance equals zero
    \end{block}

    \vspace{0.5cm}

    \begin{center}
        \textit{Answer on next slide...}
    \end{center}
    \end{cminipage}
\end{frame}

\begin{frame}{Quiz 4: Answer}
    \begin{cminipage}{0.95\textwidth}
    \begin{exampleblock}{Answer: B -- Coefficients on lagged $Y_2$ in $Y_1$ equation = 0}
        \textbf{Null hypothesis}: $H_0: a_{12}^{(1)} = a_{12}^{(2)} = \cdots = a_{12}^{(p)} = 0$

        \vspace{0.2cm}
        \textbf{Test statistic}: Wald or F-test with $p$ restrictions

        \vspace{0.2cm}
        \textbf{Interpretation}:
        \begin{itemize}\setlength{\itemsep}{0pt}
            \item Reject $H_0$: $Y_2$ Granger-causes $Y_1$
            \item Don't reject: No evidence of predictive relationship
        \end{itemize}

        \vspace{0.2cm}
        \textbf{Note}: Test $Y_1 \to Y_2$ separately (different coefficients in $Y_2$ equation)
    \end{exampleblock}

    \end{cminipage}
    \quantlet{TSA\_ch6\_case\_granger}{https://github.com/QuantLet/TSA/tree/main/TSA_ch6/TSA_ch6_case_granger}
\end{frame}

\begin{frame}{Quiz 5: VAR Stability}
    \begin{cminipage}{0.95\textwidth}
    \begin{alertblock}{Question}
        A VAR(1) model is stable (stationary) if:
    \end{alertblock}

    \vspace{0.4cm}

    \begin{block}{Answer choices}
        \textcolor{MainBlue}{\textbf{(A)}} All diagonal elements of $\bA_1$ are less than 1\\[3pt]
        \textcolor{MainBlue}{\textbf{(B)}} The determinant of $\bA_1$ is less than 1\\[3pt]
        \textcolor{MainBlue}{\textbf{(C)}} All eigenvalues of $\bA_1$ are less than 1 in absolute value\\[3pt]
        \textcolor{MainBlue}{\textbf{(D)}} The trace of $\bA_1$ equals zero
    \end{block}

    \vspace{0.5cm}

    \begin{center}
        \textit{Answer on next slide...}
    \end{center}
    \end{cminipage}
\end{frame}

\begin{frame}{Quiz 5: Answer}
    \begin{cminipage}{0.95\textwidth}
    \begin{exampleblock}{Answer: C -- All eigenvalues of $\bA_1$ inside unit circle}
        \begin{center}
            \includegraphics[width=0.95\textwidth, height=0.6\textheight, keepaspectratio]{sem6_var_stability.pdf}
        \end{center}
        \vspace{-0.2cm}
        {\footnotesize
        \textbf{Stable}: All $|\lambda_i| < 1$ (inside unit circle) $\Rightarrow$ shocks die out over time
        }
    \end{exampleblock}

    \end{cminipage}
    \quantlet{TSA\_ch6\_var\_results}{https://github.com/QuantLet/TSA/tree/main/TSA_ch6/TSA_ch6_var_results}
\end{frame}

\begin{frame}{Quiz 6: Impulse Response Functions}
    \begin{cminipage}{0.95\textwidth}
    \begin{alertblock}{Question}
        An impulse response function shows:
    \end{alertblock}

    \vspace{0.4cm}

    \begin{block}{Answer choices}
        \textcolor{MainBlue}{\textbf{(A)}} The correlation between two variables\\[3pt]
        \textcolor{MainBlue}{\textbf{(B)}} The effect of a shock to one variable on all variables over time\\[3pt]
        \textcolor{MainBlue}{\textbf{(C)}} The forecast accuracy of the model\\[3pt]
        \textcolor{MainBlue}{\textbf{(D)}} The p-values of coefficient tests
    \end{block}

    \vspace{0.5cm}

    \begin{center}
        \textit{Answer on next slide...}
    \end{center}
    \end{cminipage}
\end{frame}

\begin{frame}{Quiz 6: Answer}
    \begin{cminipage}{0.95\textwidth}
    \begin{exampleblock}{Answer: B -- Effect of shock on all variables over time}
        \begin{center}
            \includegraphics[width=0.95\textwidth, height=0.6\textheight, keepaspectratio]{sem6_irf_example.pdf}
        \end{center}
        \vspace{-0.2cm}
        {\footnotesize
        \textbf{IRF}$_{ij}(h)$: Response of variable $i$ at horizon $h$ to shock in variable $j$
        }
    \end{exampleblock}

    \end{cminipage}
    \quantlet{TSA\_ch6\_irf}{https://github.com/QuantLet/TSA/tree/main/TSA_ch6/TSA_ch6_irf}
\end{frame}

\begin{frame}{Quiz 7: Lag Order Selection}
    \begin{cminipage}{0.95\textwidth}
    \begin{alertblock}{Question}
        Which criterion typically selects the most parsimonious VAR model?
    \end{alertblock}

    \vspace{0.4cm}

    \begin{block}{Answer choices}
        \textcolor{MainBlue}{\textbf{(A)}} AIC (Akaike Information Criterion)\\[3pt]
        \textcolor{MainBlue}{\textbf{(B)}} BIC (Bayesian Information Criterion)\\[3pt]
        \textcolor{MainBlue}{\textbf{(C)}} FPE (Final Prediction Error)\\[3pt]
        \textcolor{MainBlue}{\textbf{(D)}} Adjusted $R^2$
    \end{block}

    \vspace{0.5cm}

    \begin{center}
        \textit{Answer on next slide...}
    \end{center}
    \end{cminipage}
\end{frame}

\begin{frame}{Quiz 7: Answer}
    \begin{cminipage}{0.95\textwidth}
    \begin{exampleblock}{Answer: B -- BIC (Bayesian Information Criterion)}
        \textbf{Penalty comparison} (for $k$ parameters, $n$ observations):
        \begin{itemize}\setlength{\itemsep}{0pt}
            \item AIC: $-2\ln L + 2k$
            \item BIC: $-2\ln L + k\ln n$
        \end{itemize}

        Since $\ln n > 2$ for $n > 8$, BIC penalizes complexity more heavily

        \vspace{0.2cm}
        \textbf{Practical guidance}:
        \begin{itemize}\setlength{\itemsep}{0pt}
            \item Forecasting: AIC may perform better
            \item Inference/parsimony: BIC preferred
            \item Large samples: BIC consistent, AIC tends to overfit
        \end{itemize}
    \end{exampleblock}

    \end{cminipage}
    \quantlet{TSA\_ch6\_lag\_selection}{https://github.com/QuantLet/TSA/tree/main/TSA_ch6/TSA_ch6_lag_selection}
\end{frame}

\begin{frame}{Quiz 8: Granger Causality -- Formal Test}
    \begin{cminipage}{0.95\textwidth}
    \begin{alertblock}{Question}
        ``$X$ Granger-causes $Y$'' means:
    \end{alertblock}

    \vspace{0.4cm}

    \begin{block}{Answer choices}
        \textcolor{MainBlue}{\textbf{(A)}} $X$ is the true cause of $Y$\\[3pt]
        \textcolor{MainBlue}{\textbf{(B)}} Past values of $X$ help predict $Y$ beyond $Y$'s own past\\[3pt]
        \textcolor{MainBlue}{\textbf{(C)}} $X$ and $Y$ are correlated\\[3pt]
        \textcolor{MainBlue}{\textbf{(D)}} $Y$ depends only on $X$
    \end{block}

    \vspace{0.5cm}

    \begin{center}
        \textit{Answer on next slide...}
    \end{center}
    \end{cminipage}
\end{frame}

\begin{frame}{Quiz 8: Answer}
    \begin{cminipage}{0.95\textwidth}
    \begin{exampleblock}{Answer: B -- Past values of $X$ help predict $Y$ beyond $Y$'s own past}
        \begin{itemize}
            \item Granger causality refers to \textbf{predictive} content, not true causation
            \item $X$ Granger-causes $Y$ if lagged $X$ terms are jointly significant in the $Y$ equation, after controlling for lagged $Y$
        \end{itemize}
    \end{exampleblock}

    \end{cminipage}
\end{frame}

\begin{frame}{Quiz 9: Forecast Error Variance Decomposition}
    \begin{cminipage}{0.95\textwidth}
    \begin{alertblock}{Question}
        FEVD (Forecast Error Variance Decomposition) tells us:
    \end{alertblock}

    \vspace{0.4cm}

    \begin{block}{Answer choices}
        \textcolor{MainBlue}{\textbf{(A)}} The correlation between variables\\[3pt]
        \textcolor{MainBlue}{\textbf{(B)}} What proportion of forecast error variance comes from each shock\\[3pt]
        \textcolor{MainBlue}{\textbf{(C)}} The optimal forecast horizon\\[3pt]
        \textcolor{MainBlue}{\textbf{(D)}} Which variables to include in the model
    \end{block}

    \vspace{0.5cm}

    \begin{center}
        \textit{Answer on next slide...}
    \end{center}
    \end{cminipage}
\end{frame}

\begin{frame}{Quiz 9: Answer}
    \begin{cminipage}{0.95\textwidth}
    \begin{exampleblock}{Answer: B -- Proportion of forecast error variance from each shock}
        \begin{center}
            \includegraphics[width=0.95\textwidth, height=0.6\textheight, keepaspectratio]{sem6_fevd_example.pdf}
        \end{center}
        \vspace{-0.2cm}
        {\footnotesize
        \textbf{FEVD}: Shows how much forecast uncertainty comes from each shock at different horizons
        }
    \end{exampleblock}

    \end{cminipage}
    \quantlet{TSA\_ch6\_fevd}{https://github.com/QuantLet/TSA/tree/main/TSA_ch6/TSA_ch6_fevd}
\end{frame}

\begin{frame}{Quiz 10: Structural vs Reduced Form VAR}
    \begin{cminipage}{0.95\textwidth}
    \begin{alertblock}{Question}
        The difference between structural VAR (SVAR) and reduced-form VAR is:
    \end{alertblock}

    \vspace{0.4cm}

    \begin{block}{Answer choices}
        \textcolor{MainBlue}{\textbf{(A)}} SVAR has more variables\\[3pt]
        \textcolor{MainBlue}{\textbf{(B)}} SVAR allows contemporaneous effects between variables\\[3pt]
        \textcolor{MainBlue}{\textbf{(C)}} SVAR uses different estimation methods\\[3pt]
        \textcolor{MainBlue}{\textbf{(D)}} There is no difference
    \end{block}

    \vspace{0.5cm}

    \begin{center}
        \textit{Answer on next slide...}
    \end{center}
    \end{cminipage}
\end{frame}

\begin{frame}{Quiz 10: Answer}
    \begin{cminipage}{0.95\textwidth}
    \begin{exampleblock}{Answer: B -- SVAR allows contemporaneous effects between variables}
        \begin{itemize}
            \item Reduced-form VAR: shocks are correlated, no contemporaneous effects in equations
            \item SVAR: imposes identifying restrictions to recover structural shocks with economic interpretation (e.g., monetary policy shock)
        \end{itemize}
    \end{exampleblock}

    \end{cminipage}
\end{frame}

\begin{frame}{Quiz 11: Cholesky Decomposition}
    \begin{cminipage}{0.95\textwidth}
    \begin{alertblock}{Question}
        Cholesky ordering in IRF analysis assumes:
    \end{alertblock}

    \vspace{0.4cm}

    \begin{block}{Answer choices}
        \textcolor{MainBlue}{\textbf{(A)}} All variables are equally important\\[3pt]
        \textcolor{MainBlue}{\textbf{(B)}} Variables ordered first affect later variables contemporaneously, not vice versa\\[3pt]
        \textcolor{MainBlue}{\textbf{(C)}} Shocks are uncorrelated\\[3pt]
        \textcolor{MainBlue}{\textbf{(D)}} No restrictions are needed
    \end{block}

    \vspace{0.5cm}

    \begin{center}
        \textit{Answer on next slide...}
    \end{center}
    \end{cminipage}
\end{frame}

\begin{frame}{Quiz 11: Answer}
    \begin{cminipage}{0.95\textwidth}
    \begin{exampleblock}{Answer: B -- Variables ordered first affect later ones contemporaneously}
        \begin{center}
            \includegraphics[width=0.95\textwidth, height=0.6\textheight, keepaspectratio]{sem6_cholesky_ordering.pdf}
        \end{center}
        \vspace{-0.2cm}
        {\footnotesize
        \textbf{Cholesky}: Recursive structure. Ordering matters -- justify by economic theory (most exogenous first)!
        }
    \end{exampleblock}

    \end{cminipage}
    \quantlet{TSA\_ch6\_structural\_irf}{https://github.com/QuantLet/TSA/tree/main/TSA_ch6/TSA_ch6_structural_irf}
\end{frame}

\begin{frame}{Quiz 12: VAR Residual Diagnostics}
    \begin{cminipage}{0.95\textwidth}
    \begin{alertblock}{Question}
        In a well-specified VAR, residuals should be:
    \end{alertblock}

    \vspace{0.4cm}

    \begin{block}{Answer choices}
        \textcolor{MainBlue}{\textbf{(A)}} Autocorrelated but homoskedastic\\[3pt]
        \textcolor{MainBlue}{\textbf{(B)}} White noise (no autocorrelation)\\[3pt]
        \textcolor{MainBlue}{\textbf{(C)}} Normally distributed only\\[3pt]
        \textcolor{MainBlue}{\textbf{(D)}} Correlated across equations
    \end{block}

    \vspace{0.5cm}

    \begin{center}
        \textit{Answer on next slide...}
    \end{center}
    \end{cminipage}
\end{frame}

\begin{frame}{Quiz 12: Answer}
    \begin{cminipage}{0.95\textwidth}
    \begin{exampleblock}{Answer: B -- White noise (no autocorrelation)}
        \begin{center}
            \includegraphics[width=0.95\textwidth, height=0.6\textheight, keepaspectratio]{sem6_var_diagnostics.pdf}
        \end{center}
        \vspace{-0.2cm}
        {\footnotesize
        \textbf{Diagnostics}: Residuals should be white noise. Use Portmanteau/LM test. Cross-equation correlation allowed ($\Sigma_u$).
        }
    \end{exampleblock}

    \end{cminipage}
    \quantlet{TSA\_ch6\_diagnostics}{https://github.com/QuantLet/TSA/tree/main/TSA_ch6/TSA_ch6_diagnostics}
\end{frame}

\begin{frame}{Quiz 13: Cointegration and VAR}
    \begin{cminipage}{0.95\textwidth}
    \begin{alertblock}{Question}
        If variables are I(1) and cointegrated, you should use:
    \end{alertblock}

    \vspace{0.4cm}

    \begin{block}{Answer choices}
        \textcolor{MainBlue}{\textbf{(A)}} VAR in levels\\[3pt]
        \textcolor{MainBlue}{\textbf{(B)}} VAR in first differences\\[3pt]
        \textcolor{MainBlue}{\textbf{(C)}} Vector Error Correction Model (VECM)\\[3pt]
        \textcolor{MainBlue}{\textbf{(D)}} Univariate ARIMA models
    \end{block}

    \vspace{0.5cm}

    \begin{center}
        \textit{Answer on next slide...}
    \end{center}
    \end{cminipage}
\end{frame}

\begin{frame}{Quiz 13: Answer}
    \begin{cminipage}{0.95\textwidth}
    \begin{exampleblock}{Answer: C -- Vector Error Correction Model (VECM)}
        \begin{itemize}
            \item With cointegration, VAR in differences loses long-run information, while VAR in levels may be inefficient
            \item VECM incorporates both short-run dynamics and long-run equilibrium relationships through the error correction term
        \end{itemize}
    \end{exampleblock}

    \end{cminipage}
\end{frame}

\begin{frame}{Quiz 14: Instantaneous Causality}
    \begin{cminipage}{0.95\textwidth}
    \begin{alertblock}{Question}
        Instantaneous causality differs from Granger causality because it tests:
    \end{alertblock}

    \vspace{0.4cm}

    \begin{block}{Answer choices}
        \textcolor{MainBlue}{\textbf{(A)}} Lagged relationships only\\[3pt]
        \textcolor{MainBlue}{\textbf{(B)}} Contemporaneous correlation of residuals\\[3pt]
        \textcolor{MainBlue}{\textbf{(C)}} Long-run relationships\\[3pt]
        \textcolor{MainBlue}{\textbf{(D)}} Model stability
    \end{block}

    \vspace{0.5cm}

    \begin{center}
        \textit{Answer on next slide...}
    \end{center}
    \end{cminipage}
\end{frame}

\begin{frame}{Quiz 14: Answer}
    \begin{cminipage}{0.95\textwidth}
    \begin{exampleblock}{Answer: B -- Contemporaneous correlation of residuals}
        \begin{itemize}
            \item Instantaneous causality tests whether shocks to $X$ and $Y$ are correlated within the same period (correlation of VAR residuals)
            \item Granger causality tests whether \textit{lagged} values help predict
        \end{itemize}
    \end{exampleblock}

    \end{cminipage}
\end{frame}

%=============================================================================
% TRUE/FALSE
%=============================================================================
\section{True/False}

\begin{frame}{True or False? --- Questions}
    \begin{cminipage}{0.95\textwidth}
    \footnotesize
    \begin{center}
    \begin{tabular}{p{9cm}c}
        \toprule
        \textbf{Statement} & \textbf{T/F?} \\
        \midrule
        1. VAR models treat all variables as endogenous. & ? \\[0.15cm]
        2. Granger causality proves true economic causation. & ? \\[0.15cm]
        3. A stable VAR always has eigenvalues inside the unit circle. & ? \\[0.15cm]
        4. FEVD results depend on the ordering of variables. & ? \\[0.15cm]
        5. VAR can be estimated by OLS equation by equation. & ? \\[0.15cm]
        6. Impulse responses eventually die out in a stable VAR. & ? \\
        \bottomrule
    \end{tabular}
    \end{center}
    \end{cminipage}
\end{frame}

\begin{frame}{True or False? --- Answers}
    \begin{cminipage}{0.95\textwidth}
    \scriptsize
    \begin{center}
    \begin{tabular}{p{7.5cm}cc}
        \toprule
        \textbf{Statement} & \textbf{T/F} & \textbf{Explanation} \\
        \midrule
        1. VAR models treat all variables as endogenous. & \textcolor{Forest}{\textbf{T}} & {\tiny Each var regressed on all lags} \\[0.08cm]
        2. Granger causality proves true economic causation. & \textcolor{Crimson}{\textbf{F}} & {\tiny Only predictive content} \\[0.08cm]
        3. A stable VAR always has eigenvalues inside the unit circle. & \textcolor{Forest}{\textbf{T}} & {\tiny $|\lambda_i| < 1$ for all $i$} \\[0.08cm]
        4. FEVD results depend on the ordering of variables. & \textcolor{Forest}{\textbf{T}} & {\tiny Cholesky ordering matters} \\[0.08cm]
        5. VAR can be estimated by OLS equation by equation. & \textcolor{Forest}{\textbf{T}} & {\tiny OLS = GLS = ML} \\[0.08cm]
        6. Impulse responses eventually die out in a stable VAR. & \textcolor{Forest}{\textbf{T}} & {\tiny IRFs $\to 0$ as $h \to \infty$} \\
        \bottomrule
    \end{tabular}
    \end{center}
    \end{cminipage}
\end{frame}

%=============================================================================
% CALCULATION EXERCISES
%=============================================================================
\section{Calculation Exercises}

\begin{frame}{Exercise 1: Writing VAR Equations}
    \begin{cminipage}{0.95\textwidth}
    \begin{alertblock}{Problem}
        \begin{itemize}\setlength{\itemsep}{0pt}
            \item Write out the two equations for a bivariate VAR(1) model with variables $Y_t$ (GDP growth) and $X_t$ (inflation).
        \end{itemize}
    \end{alertblock}

    \vspace{0.2cm}
    \begin{exampleblock}{Solution}
        \begin{align*}
            Y_t &= c_1 + a_{11} Y_{t-1} + a_{12} X_{t-1} + \varepsilon_{1t} \\[0.2cm]
            X_t &= c_2 + a_{21} Y_{t-1} + a_{22} X_{t-1} + \varepsilon_{2t}
        \end{align*}

        \textbf{Interpretation}:
        \begin{itemize}\setlength{\itemsep}{0pt}
            \item $a_{12}$: Effect of past inflation on current GDP growth
            \item $a_{21}$: Effect of past GDP growth on current inflation
        \end{itemize}
    \end{exampleblock}
    \end{cminipage}
    \quantlet{TSA\_ch6\_var\_results}{https://github.com/QuantLet/TSA/tree/main/TSA_ch6/TSA_ch6_var_results}
\end{frame}

\begin{frame}{Exercise 2: Parameter Count}
    \begin{cminipage}{0.95\textwidth}
    \begin{alertblock}{Problem}
        \begin{itemize}\setlength{\itemsep}{0pt}
            \item How many total parameters need to be estimated in a VAR(3) with $K=4$ variables (including constants)?
        \end{itemize}
    \end{alertblock}

    \vspace{0.2cm}
    \begin{exampleblock}{Solution}
        Per equation: $1 + K \times p = 1 + 4 \times 3 = 13$ parameters

        \vspace{0.2cm}
        Total for $K=4$ equations: $4 \times 13 = \mathbf{52}$ parameters

        \vspace{0.2cm}
        Plus covariance matrix $\boldsymbol{\Sigma}$: $K(K+1)/2 = 4 \times 5 / 2 = 10$ unique elements

        \vspace{0.2cm}
        \textbf{Grand total: 62 parameters}

        \vspace{0.2cm}
        \textit{This is why VARs can be ``over-parameterized'' with limited data!}
    \end{exampleblock}
    \end{cminipage}
\end{frame}

\begin{frame}{Exercise 3: Granger Causality Interpretation}
    \begin{cminipage}{0.95\textwidth}
    \begin{alertblock}{Problem}
        \begin{itemize}\setlength{\itemsep}{0pt}
            \item A Granger causality test yields:
            \item $H_0$: Money does not Granger-cause GDP. $p$-value = 0.02
            \item $H_0$: GDP does not Granger-cause Money. $p$-value = 0.35
            \item Interpret these results.
        \end{itemize}
    \end{alertblock}

    \vspace{0.15cm}
    \begin{exampleblock}{Solution}
        \begin{itemize}\setlength{\itemsep}{2pt}
            \item \textbf{Reject} $H_0$ at 5\%: Money \textbf{Granger-causes} GDP
            \item \textbf{Fail to reject} $H_0$: GDP does \textbf{not} Granger-cause Money
        \end{itemize}

        \vspace{0.2cm}
        \textbf{Conclusion}: Unidirectional causality: Money $\succ$ GDP

        \vspace{0.2cm}
        \textit{Interpretation}: Past money supply helps predict GDP growth. This is consistent with monetarist views, but remember: Granger causality $\neq$ structural causality!
    \end{exampleblock}
    \end{cminipage}
    \quantlet{TSA\_ch6\_case\_granger}{https://github.com/QuantLet/TSA/tree/main/TSA_ch6/TSA_ch6_case_granger}
\end{frame}

\begin{frame}{Exercise 4: Stability Check}
    \begin{cminipage}{0.95\textwidth}
    \begin{alertblock}{Problem}
        \begin{itemize}\setlength{\itemsep}{0pt}
            \item For VAR(1) with $\bA_1 = \begin{pmatrix} 0.7 & 0.2 \\ 0.1 & 0.5 \end{pmatrix}$, check stability.
        \end{itemize}
    \end{alertblock}

    \vspace{0.2cm}
    \begin{exampleblock}{Solution}
        Find eigenvalues: $\det(\bA_1 - \lambda \mathbf{I}) = 0$

        $(0.7 - \lambda)(0.5 - \lambda) - (0.2)(0.1) = 0$

        $\lambda^2 - 1.2\lambda + 0.33 = 0$

        $\lambda = \frac{1.2 \pm \sqrt{1.44 - 1.32}}{2} = \frac{1.2 \pm 0.346}{2}$

        $\lambda_1 = 0.773, \quad \lambda_2 = 0.427$

        \vspace{0.3cm}
        Both $|\lambda_i| < 1$ $\Rightarrow$ \textbf{Stable!}
    \end{exampleblock}
    \end{cminipage}
    \quantlet{TSA\_ch6\_stability\_roots}{https://github.com/QuantLet/TSA/tree/main/TSA_ch6/TSA_ch6_stability_roots}
\end{frame}

\begin{frame}{Exercise 5: IRF Computation}
    \begin{cminipage}{0.95\textwidth}
    \begin{alertblock}{Problem}
        \begin{itemize}\setlength{\itemsep}{0pt}
            \item For VAR(1) with $\bA = \begin{pmatrix} 0.5 & 0.2 \\ 0 & 0.6 \end{pmatrix}$, compute $\boldsymbol{\Phi}_2$ (response at $h=2$).
        \end{itemize}
    \end{alertblock}

    \vspace{0.2cm}
    \begin{exampleblock}{Solution}
        $\boldsymbol{\Phi}_2 = \bA^2 = \begin{pmatrix} 0.5 & 0.2 \\ 0 & 0.6 \end{pmatrix} \begin{pmatrix} 0.5 & 0.2 \\ 0 & 0.6 \end{pmatrix}$

        \vspace{0.3cm}
        $= \begin{pmatrix} 0.25 + 0 & 0.10 + 0.12 \\ 0 + 0 & 0 + 0.36 \end{pmatrix} = \begin{pmatrix} 0.25 & 0.22 \\ 0 & 0.36 \end{pmatrix}$

        \vspace{0.3cm}
        \textbf{Interpretation}: A unit shock to $Y_2$ at $t$ increases $Y_1$ by 0.22 at $t+2$.
    \end{exampleblock}
    \end{cminipage}
    \quantlet{TSA\_ch6\_irf}{https://github.com/QuantLet/TSA/tree/main/TSA_ch6/TSA_ch6_irf}
\end{frame}

%=============================================================================
% WORKED EXAMPLES
%=============================================================================
\section{Worked Examples}

\begin{frame}{Example: Stock Returns and Trading Volume}
    \begin{cminipage}{0.95\textwidth}
    {\small
    \begin{block}{Scenario}
        Daily data on stock returns ($R_t$) and trading volume ($V_t$). Test Granger causality both directions.
    \end{block}
    \begin{exampleblock}{Typical Findings in Finance Literature}
        \begin{itemize}\setlength{\itemsep}{0pt}
            \item Returns often Granger-cause volume (price changes trigger trading)
            \item Volume sometimes Granger-causes returns (volume as leading indicator)
            \item Results: Often \textbf{bidirectional} causality $R \leftrightarrow V$
        \end{itemize}
    \end{exampleblock}
    \begin{block}{Practical Issue}
        Stock returns are typically stationary, but volume may need transformation (log or difference).
    \end{block}
    }
    \end{cminipage}
\end{frame}

\begin{frame}{Example: Interest Rates and Inflation}
    \begin{cminipage}{0.95\textwidth}
    {\small
    \begin{block}{Taylor Rule Context}
        Central banks set interest rates ($i_t$) in response to inflation ($\pi_t$):
        $i_t = r^* + \pi^* + 1.5(\pi_t - \pi^*) + 0.5(y_t - y^*)$
    \end{block}
    \begin{exampleblock}{VAR Analysis Questions}
        \begin{itemize}\setlength{\itemsep}{0pt}
            \item Does inflation Granger-cause interest rates? (Central bank reaction)
            \item Do interest rates Granger-cause inflation? (Monetary policy transmission)
        \end{itemize}
    \end{exampleblock}
    \begin{block}{Expected Results}
        Bidirectional causality: Quick $\pi \to i$ (policy reaction), Delayed $i \to \pi$ (policy effect)
    \end{block}
    }
    \end{cminipage}
\end{frame}

\begin{frame}{Python VAR Analysis: Key Functions}
    \begin{cminipage}{0.95\textwidth}
    {\footnotesize
    \begin{block}{Essential Libraries}
        \texttt{from statsmodels.tsa.api import VAR} \\
        \texttt{from statsmodels.tsa.stattools import grangercausalitytests}
    \end{block}

    \begin{block}{Workflow}
        \begin{enumerate}\setlength{\itemsep}{0pt}
            \item Create DataFrame: \texttt{data = pd.DataFrame(\{'gdp': ..., 'unemp': ...\})}
            \item Fit VAR: \texttt{model = VAR(data); results = model.fit(maxlags=8, ic='aic')}
            \item Get IRF: \texttt{irf = results.irf(periods=20)}
            \item Get FEVD: \texttt{fevd = results.fevd(periods=20)}
            \item Granger tests: \texttt{grangercausalitytests(data[['y', 'x']], maxlag=4)}
        \end{enumerate}
    \end{block}

    \begin{alertblock}{Note}
        Complete working examples are provided in the Jupyter notebooks.
    \end{alertblock}
    }
    \end{cminipage}
    \quantlet{TSA\_ch6\_var\_simulation}{https://github.com/QuantLet/TSA/tree/main/TSA_ch6/TSA_ch6_var_simulation}
\end{frame}

%=============================================================================
% REAL DATA ANALYSIS
%=============================================================================
\section{Real Data Analysis}

\begin{frame}{Case Study: GDP and Unemployment}
    \begin{cminipage}{0.95\textwidth}
    \vspace{-0.3cm}
    \begin{center}
        \includegraphics[width=0.85\textwidth, height=0.55\textheight, keepaspectratio]{ch6_gdp_unemployment.pdf}
    \end{center}
    \vspace{-0.2cm}
    {\footnotesize
    \begin{itemize}\setlength{\itemsep}{0pt}
        \item \textbf{Top}: US Real GDP growth rate (quarterly)
        \item \textbf{Bottom}: US Unemployment rate
        \item Clear negative relationship (Okun's Law)
        \item VAR model can capture dynamic interactions between these variables
    \end{itemize}
    }
    \end{cminipage}
    \quantlet{TSA\_ch6\_gdp\_unemployment}{https://github.com/QuantLet/TSA/tree/main/TSA_ch6/TSA_ch6_gdp_unemployment}
\end{frame}

\begin{frame}{Cross-Correlation Analysis}
    \begin{cminipage}{0.95\textwidth}
    \vspace{-0.3cm}
    \begin{center}
        \includegraphics[width=0.85\textwidth, height=0.55\textheight, keepaspectratio]{ch6_cross_correlation.pdf}
    \end{center}
    \vspace{-0.2cm}
    {\footnotesize
    \begin{itemize}\setlength{\itemsep}{0pt}
        \item Cross-correlation measures lead-lag relationships
        \item Negative correlation at lag 0: contemporaneous inverse relationship
        \item Asymmetric pattern suggests unemployment responds to GDP with lag
        \item Helps inform VAR lag order selection
    \end{itemize}
    }
    \end{cminipage}
    \quantlet{TSA\_ch6\_motivation\_leadlag}{https://github.com/QuantLet/TSA/tree/main/TSA_ch6/TSA_ch6_motivation_leadlag}
\end{frame}

\begin{frame}{Visual: Cross-Correlation Function}
    \begin{cminipage}{0.95\textwidth}
    \begin{center}
        \includegraphics[width=0.9\textwidth, height=0.65\textheight, keepaspectratio]{ch6_def_ccf.pdf}
    \end{center}
    \vspace{-0.3cm}
    {\small The CCF measures correlation between two series at different lags, revealing lead-lag relationships.}
    \end{cminipage}
    \quantlet{TSA\_ch6\_def\_ccf}{https://github.com/QuantLet/TSA/tree/main/TSA_ch6/TSA_ch6_def_ccf}
\end{frame}

\begin{frame}{VAR Estimation Results}
    \begin{cminipage}{0.95\textwidth}
    {\small
    \begin{block}{Model: VAR(2) for GDP Growth and Unemployment}
        \begin{center}
        \begin{tabular}{lcccc}
            \toprule
            \textbf{Equation} & \textbf{Variable} & \textbf{Coef.} & \textbf{Std. Error} & \textbf{t-stat} \\
            \midrule
            \multirow{4}{*}{$\Delta GDP_t$} & $\Delta GDP_{t-1}$ & $0.312$ & $0.087$ & $3.59$ \\
            & $\Delta GDP_{t-2}$ & $0.145$ & $0.082$ & $1.77$ \\
            & $U_{t-1}$ & $-0.421$ & $0.156$ & $-2.70$ \\
            & $U_{t-2}$ & $0.198$ & $0.148$ & $1.34$ \\
            \midrule
            \multirow{4}{*}{$U_t$} & $\Delta GDP_{t-1}$ & $-0.087$ & $0.032$ & $-2.72$ \\
            & $\Delta GDP_{t-2}$ & $-0.045$ & $0.030$ & $-1.50$ \\
            & $U_{t-1}$ & $1.456$ & $0.058$ & $25.1$ \\
            & $U_{t-2}$ & $-0.521$ & $0.055$ & $-9.47$ \\
            \bottomrule
        \end{tabular}
        \end{center}
    \end{block}
    }
    \end{cminipage}
    \quantlet{TSA\_ch6\_var\_simulation}{https://github.com/QuantLet/TSA/tree/main/TSA_ch6/TSA_ch6_var_simulation}
\end{frame}

\begin{frame}{Impulse Response Functions}
    \begin{cminipage}{0.95\textwidth}
    \vspace{-0.3cm}
    \begin{center}
        \includegraphics[width=0.85\textwidth, height=0.55\textheight, keepaspectratio]{ch6_irf.pdf}
    \end{center}
    \vspace{-0.2cm}
    {\footnotesize
    \begin{itemize}\setlength{\itemsep}{0pt}
        \item IRFs show dynamic response to one-unit shocks
        \item GDP shock: temporary positive effect on GDP, negative on unemployment
        \item Unemployment shock: negative effect on GDP, persistent on unemployment
        \item 95\% confidence bands show uncertainty in responses
    \end{itemize}
    }
    \end{cminipage}
    \quantlet{TSA\_ch6\_irf}{https://github.com/QuantLet/TSA/tree/main/TSA_ch6/TSA_ch6_irf}
\end{frame}

\begin{frame}{Forecast Error Variance Decomposition}
    \begin{cminipage}{0.95\textwidth}
    \vspace{-0.3cm}
    \begin{center}
        \includegraphics[width=0.85\textwidth, height=0.55\textheight, keepaspectratio]{ch6_fevd.pdf}
    \end{center}
    \vspace{-0.2cm}
    {\footnotesize
    \begin{itemize}\setlength{\itemsep}{0pt}
        \item FEVD shows proportion of variance explained by each shock
        \item GDP variance: mostly explained by own shocks, some by unemployment
        \item Unemployment variance: highly persistent (own shocks dominant)
        \item Provides insight into relative importance of different shocks
    \end{itemize}
    }
    \end{cminipage}
    \quantlet{TSA\_ch6\_fevd}{https://github.com/QuantLet/TSA/tree/main/TSA_ch6/TSA_ch6_fevd}
\end{frame}

%=============================================================================
% DISCUSSION TOPICS
%=============================================================================
\section{Discussion Topics}

\begin{frame}{Discussion: Granger Causality vs True Causality}
    \begin{cminipage}{0.95\textwidth}
    {\small
    \begin{alertblock}{Key Question}
        If $X$ Granger-causes $Y$, does that mean $X$ actually causes $Y$? \textbf{NO!}
    \end{alertblock}
    \begin{block}{Why Granger Causality Can Fail}
        \begin{itemize}\setlength{\itemsep}{0pt}
            \item \textbf{Omitted variable bias}: $Z$ might cause both $X$ and $Y$ (e.g., weather $\to$ ice cream \& drownings)
            \item \textbf{Anticipation effects}: Markets anticipate future events (stock prices $\to$ earnings)
            \item \textbf{Aggregation issues}: Timing of data collection matters
        \end{itemize}
    \end{block}
    \begin{exampleblock}{Conclusion}
        Granger causality is about \textbf{prediction}, not \textbf{mechanism}. For structural causality, need theory + identification strategy.
    \end{exampleblock}
    }
    \end{cminipage}
\end{frame}

\begin{frame}{Discussion: Variable Ordering in IRFs}
    \begin{cminipage}{0.95\textwidth}
    \vspace{-0.3cm}
    {\footnotesize
    \begin{alertblock}{Key Question}
        Why does variable ordering matter for orthogonalized IRFs?
    \end{alertblock}
    \begin{block}{Cholesky Decomposition Assumes}
        \begin{itemize}\setlength{\itemsep}{0pt}
            \item First variable: Affects all others contemporaneously
            \item Second variable: Affected by first, affects remaining
            \item Last variable: Affected by all, affects none contemporaneously
        \end{itemize}
    \end{block}
    \begin{exampleblock}{Example: Monetary Policy VAR Ordering}
        1. Oil prices (exogenous) $\to$ 2. GDP (slow) $\to$ 3. Inflation $\to$ 4. Interest rates (fast)
    \end{exampleblock}
    \begin{block}{Rule}
        Order from ``most exogenous'' to ``most endogenous'' --- justify with economic theory!
    \end{block}
    }
    \end{cminipage}
\end{frame}

%=============================================================================
% AI-ASSISTED EXERCISE
%=============================================================================
\section{AI-Assisted Exercise}

\begin{frame}{Take-Home Exercises}
    \begin{cminipage}{0.95\textwidth}
    {\footnotesize
    \begin{enumerate}\setlength{\itemsep}{2pt}
        \item \textbf{Theoretical}: Show that a VAR(1) can be written as MA($\infty$): $\bY_t = \sum_{i=0}^{\infty} \bA^i \bepsilon_{t-i}$ when stable.

        \item \textbf{Computation}: For VAR(1) with $\bA = \begin{pmatrix} 0.8 & -0.1 \\ 0.3 & 0.4 \end{pmatrix}$:
            \begin{itemize}\setlength{\itemsep}{0pt}
                \item Check stability; Compute IRFs for $h = 0, 1, 2, 3$
                \item Plot the response of $Y_1$ to a shock in $Y_2$
            \end{itemize}

        \item \textbf{Applied}: Download US GDP growth and unemployment data:
            \begin{itemize}\setlength{\itemsep}{0pt}
                \item Test stationarity; Estimate VAR (select optimal lag)
                \item Test Granger causality; Compute and interpret IRFs
            \end{itemize}

        \item \textbf{Critical Thinking}: Why might stock prices ``Granger-cause'' GDP even though GDP is determined by real factors?
    \end{enumerate}
    }
    \end{cminipage}
    \quantlet{TSA\_ch6\_var\_forecast}{https://github.com/QuantLet/TSA/tree/main/TSA_ch6/TSA_ch6_var_forecast}
\end{frame}

\begin{frame}{Exercise Solutions Hints}
    \begin{cminipage}{0.95\textwidth}
    {\footnotesize
    \begin{block}{Hints}
        \begin{enumerate}\setlength{\itemsep}{1pt}
            \item Use recursive substitution: $\bY_t = \bA\bY_{t-1} + \bepsilon_t = \bA(\bA\bY_{t-2} + \bepsilon_{t-1}) + \bepsilon_t = \ldots$

            \item Eigenvalues of $\begin{pmatrix} 0.8 & -0.1 \\ 0.3 & 0.4 \end{pmatrix}$:
                \begin{itemize}\setlength{\itemsep}{0pt}
                    \item Characteristic equation: $\lambda^2 - 1.2\lambda + 0.35 = 0$
                    \item $\lambda_1 = 0.7$, $\lambda_2 = 0.5$ (both $< 1$, stable)
                \end{itemize}

            \item For GDP/Unemployment:
                \begin{itemize}\setlength{\itemsep}{0pt}
                    \item GDP growth is usually I(0), unemployment may be I(1)
                    \item Use unemployment rate changes if needed
                    \item Expect GDP growth $\succ$ unemployment (Okun's Law)
                \end{itemize}

            \item Stock prices anticipate future economic conditions---they reflect expectations about future GDP, so they ``lead'' GDP in the data even though causation runs the other way.
        \end{enumerate}
    \end{block}
    }
    \end{cminipage}
\end{frame}

\begin{frame}{AI Exercise: Critical Thinking}
    \begin{cminipage}{0.95\textwidth}
    \vspace{-0.3cm}
    \begin{block}{\footnotesize Prompt to test in ChatGPT / Claude / Copilot}
        {\footnotesize
        ``Download Romania GDP and inflation data. Estimate a VAR model, test Granger causality and compute impulse response functions. Which variable causes the other?''
        }
    \end{block}
    \vspace{-2mm}
    {\footnotesize
    \textbf{Exercise}:
    \begin{enumerate}\setlength{\itemsep}{0pt}
        \item Run the prompt in an LLM of your choice and critically analyze the response.
        \item Does the AI check stationarity of both series before estimating the VAR?
        \item Is the lag order selected via AIC/BIC or chosen arbitrarily?
        \item Is the Granger causality interpretation correct? (it does not imply true causation!)
        \item Are the IRFs orthogonalized? Does the AI mention the Cholesky decomposition ordering?
    \end{enumerate}
    }
    \vspace{-2mm}
    \begin{alertblock}{}
        {\footnotesize \textbf{Warning}: AI-generated code may run without errors and look professional. \textit{That does not mean it is correct.}}
    \end{alertblock}
    \end{cminipage}
\end{frame}

%=============================================================================
% SUMMARY
%=============================================================================
\section{Summary}

\begin{frame}{Summary: Chapter 6}
    \begin{cminipage}{0.95\textwidth}
    \begin{exampleblock}{Key Concepts}
        \begin{itemize}\setlength{\itemsep}{2pt}
            \item[\textcolor{MainBlue}{\textbf{1.}}] \textbf{VAR models}: capture \textbf{interdependencies} between multiple time series
            \item[\textcolor{MainBlue}{\textbf{2.}}] \textbf{Parameter count}: grows quickly ($K^2 p + K$ per system) --- parsimony matters
            \item[\textcolor{MainBlue}{\textbf{3.}}] \textbf{Granger causality}: tests predictive content, not true causation
            \item[\textcolor{MainBlue}{\textbf{4.}}] \textbf{IRFs}: show dynamic propagation of shocks; ordering matters
            \item[\textcolor{MainBlue}{\textbf{5.}}] \textbf{Practical}: check stationarity, use AIC/BIC for lag selection, justify variable ordering
        \end{itemize}
    \end{exampleblock}

    \vspace{0.5cm}
    \begin{center}
        \Large\textcolor{MainBlue}{Questions?}
    \end{center}
    \end{cminipage}
\end{frame}

%=============================================================================
% BIBLIOGRAPHY
%=============================================================================
\section{Bibliography}

\begin{frame}{Bibliography I}
    \begin{cminipage}{0.95\textwidth}
    \begin{block}{Fundamental Textbooks}
        {\small
        \begin{itemize}\setlength{\itemsep}{0pt}
            \item Hyndman, R.J., \& Athanasopoulos, G. (2021). \textit{Forecasting: Principles and Practice}, 3rd ed., OTexts.
            \item Shumway, R.H., \& Stoffer, D.S. (2017). \textit{Time Series Analysis and Its Applications}, 4th ed., Springer.
            \item Brockwell, P.J., \& Davis, R.A. (2016). \textit{Introduction to Time Series and Forecasting}, 3rd ed., Springer.
        \end{itemize}
        }
    \end{block}

    \begin{exampleblock}{Financial Time Series}
        {\small
        \begin{itemize}\setlength{\itemsep}{0pt}
            \item Tsay, R.S. (2010). \textit{Analysis of Financial Time Series}, 3rd ed., Wiley.
            \item Franke, J., H\"ardle, W.K., \& Hafner, C.M. (2019). \textit{Statistics of Financial Markets}, 4th ed., Springer.
        \end{itemize}
        }
    \end{exampleblock}
    \end{cminipage}
\end{frame}

\begin{frame}{Bibliography II}
    \begin{cminipage}{0.95\textwidth}
    \begin{block}{Modern Approaches and Machine Learning}
        {\small
        \begin{itemize}\setlength{\itemsep}{0pt}
            \item Nielsen, A. (2019). \textit{Practical Time Series Analysis}, O'Reilly Media.
            \item Petropoulos, F., et al. (2022). \textit{Forecasting: Theory and Practice}, International Journal of Forecasting.
            \item Makridakis, S., Spiliotis, E., \& Assimakopoulos, V. (2020). The M4 Competition, International Journal of Forecasting.
        \end{itemize}
        }
    \end{block}

    \begin{exampleblock}{Online Resources and Code}
        {\small
        \begin{itemize}\setlength{\itemsep}{0pt}
            \item \textbf{Quantlet}: \url{https://quantlet.com} --- Code repository for statistics
            \item \textbf{Quantinar}: \url{https://quantinar.com} --- Platform for learning quantitative methods
            \item \textbf{GitHub TSA}: \url{https://github.com/QuantLet/TSA/tree/main/TSA_ch6} --- Python code for this seminar
        \end{itemize}
        }
    \end{exampleblock}
    \end{cminipage}
\end{frame}

\begin{frame}{}
    \begin{cminipage}{0.95\textwidth}
    \centering
    \Huge\textcolor{IDAred}{Thank You!}

    \vspace{1cm}

    \Large\textcolor{MainBlue}{Questions?}

    \vspace{0.8cm}

    \normalsize
    Seminar materials are available at: \url{https://danpele.github.io/Time-Series-Analysis/}

    \vspace{0.2cm}

    \href{https://quantlet.com}{\raisebox{-0.15em}{\includegraphics[height=0.8em]{ql_logo.png}} Quantlet} \hspace{0.5cm}
    \href{https://quantinar.com}{\raisebox{-0.15em}{\includegraphics[height=0.8em]{qr_logo.png}} Quantinar}
    \end{cminipage}
\end{frame}

\end{document}
