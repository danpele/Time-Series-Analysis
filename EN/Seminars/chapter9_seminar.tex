% Seminar 9: Prophet and TBATS

\documentclass[9pt, aspectratio=169, t]{beamer}
%=============================================================================
% SHARED PREAMBLE - Time Series Analysis and Forecasting
% Harvard-quality academic presentations
% Bachelor program, Bucharest University of Economic Studies
%
% Usage: \documentclass[9pt, aspectratio=169, t]{beamer}
%            %=============================================================================
% SHARED PREAMBLE - Time Series Analysis and Forecasting
% Harvard-quality academic presentations
% Bachelor program, Bucharest University of Economic Studies
%
% Usage: \documentclass[9pt, aspectratio=169, t]{beamer}
%            %=============================================================================
% SHARED PREAMBLE - Time Series Analysis and Forecasting
% Harvard-quality academic presentations
% Bachelor program, Bucharest University of Economic Studies
%
% Usage: \documentclass[9pt, aspectratio=169, t]{beamer}
%            \input{preamble}
%            \subtitle{Seminar X: Seminar Title}
%            \begin{document} ...
%=============================================================================

% Ensure content fits on slides
\setbeamersize{text margin left=8mm, text margin right=8mm}

%=============================================================================
% THEME AND STYLE CONFIGURATION
%=============================================================================
\usetheme{default}
% Using default theme for clean header/footer control

% Color Palette (matching Redispatch PDF)
\definecolor{MainBlue}{RGB}{26, 58, 110}
\definecolor{AccentBlue}{RGB}{26, 58, 110}
\definecolor{IDAred}{RGB}{205, 0, 0}
\definecolor{DarkGray}{RGB}{51, 51, 51}
\definecolor{MediumGray}{RGB}{128, 128, 128}
\definecolor{LightGray}{RGB}{248, 248, 248}
\definecolor{VeryLightGray}{RGB}{235, 235, 235}
\definecolor{KeynoteGray}{RGB}{218, 218, 218}
\definecolor{SectionGray}{RGB}{120, 120, 120}
\definecolor{FooterGray}{RGB}{100, 100, 100}
\definecolor{Crimson}{RGB}{220, 53, 69}
\definecolor{Forest}{RGB}{46, 125, 50}
\definecolor{Amber}{RGB}{181, 133, 63}
\definecolor{Orange}{RGB}{230, 126, 34}
\definecolor{Purple}{RGB}{142, 68, 173}

% Gradient background (exact Keynote 315° gradient: white to RGB 218,218,218)
\setbeamertemplate{background}{%
    \begin{tikzpicture}[remember picture, overlay]
        \shade[shading=axis, shading angle=315,
        top color=white, bottom color=KeynoteGray]
        (current page.south west) rectangle (current page.north east);
    \end{tikzpicture}%
}
% Fallback solid color for compatibility
\setbeamercolor{background canvas}{bg=}

\setbeamercolor{palette primary}{bg=MainBlue, fg=white}
\setbeamercolor{palette secondary}{bg=MainBlue!85, fg=white}
\setbeamercolor{palette tertiary}{bg=MainBlue!70, fg=white}
\setbeamercolor{structure}{fg=MainBlue}
\setbeamercolor{title}{fg=IDAred}
\setbeamercolor{frametitle}{fg=IDAred, bg=}
\setbeamercolor{block title}{bg=MainBlue, fg=white}
\setbeamercolor{block body}{bg=VeryLightGray, fg=DarkGray}
\setbeamercolor{block title alerted}{bg=Crimson, fg=white}
\setbeamercolor{block body alerted}{bg=Crimson!8, fg=DarkGray}
\setbeamercolor{block title example}{bg=Forest, fg=white}
\setbeamercolor{block body example}{bg=Forest!8, fg=DarkGray}
\setbeamercolor{item}{fg=MainBlue}

% Smaller institute font to avoid overfull hbox on title page
\setbeamerfont{institute}{size=\footnotesize}

% Footer colors (override Madrid theme blue)
\setbeamercolor{author in head/foot}{fg=FooterGray, bg=}
\setbeamercolor{title in head/foot}{fg=FooterGray, bg=}
\setbeamercolor{date in head/foot}{fg=FooterGray, bg=}
\setbeamercolor{section in head/foot}{fg=FooterGray, bg=}
\setbeamercolor{subsection in head/foot}{fg=FooterGray, bg=}

% Bullet styles (apply everywhere including blocks)
\setbeamertemplate{itemize item}{\color{MainBlue}$\boxdot$}
\setbeamertemplate{itemize subitem}{\color{MainBlue}$\blacktriangleright$}
\setbeamertemplate{itemize subsubitem}{\color{MainBlue}\tiny$\bullet$}
\setbeamertemplate{itemize/enumerate body begin}{\normalsize}
\setbeamertemplate{itemize/enumerate subbody begin}{\normalsize}

% Item spacing - compact style
\setlength{\leftmargini}{10pt}       % Level 1: minimal indent
\setlength{\leftmarginii}{10pt}      % Level 2: minimal additional indent
% Compact list spacing (zero extra space before/after lists in blocks)
\makeatletter
\def\@listi{\leftmargin\leftmargini \topsep 0pt \parsep 0pt \itemsep 0pt}
\def\@listii{\leftmargin\leftmarginii \topsep 0pt \parsep 0pt \itemsep 0pt}
\makeatother

\setbeamertemplate{navigation symbols}{}

%=============================================================================
% CUSTOM HEADLINE
%=============================================================================
\setbeamertemplate{headline}{%
    \vskip10pt%
    \hbox to \paperwidth{%
        \hskip0.5cm%
        {\small\color{FooterGray}\renewcommand{\hyperlink}[2]{##2}\insertsectionhead}%
        \hfill%
        \textcolor{FooterGray}{\small\insertframenumber}%
        \hskip0.5cm%
    }%
    \vskip4pt%
    {\color{FooterGray}\hrule height 0.4pt}%
}

%=============================================================================
% CUSTOM FOOTER
%=============================================================================
\usepackage{fontawesome5}

\setbeamertemplate{footline}{%
    {\color{FooterGray}\hrule height 0.4pt}%
    \vskip4pt%
    \hbox to \paperwidth{%
        \hskip0.5cm%
        \textcolor{FooterGray}{\small Time Series Analysis and Forecasting}%
        \hfill%
        \raisebox{-0.1em}{%
            \begin{tikzpicture}[x=0.08em, y=0.08em, line width=0.4pt]
                \draw[FooterGray] (0,3) -- (1,4) -- (2,3.5) -- (3,5) -- (4,4) -- (5,6) -- (6,5.5) -- (7,4) -- (8,5) -- (9,7) -- (10,6) -- (11,5) -- (12,6.5) -- (13,8) -- (14,7) -- (15,6) -- (16,7.5) -- (17,9) -- (18,8) -- (19,7) -- (20,8.5) -- (21,10) -- (22,9) -- (23,8) -- (24,9.5);
            \end{tikzpicture}%
        }%
        \hskip0.5cm%
    }%
    \vskip6pt%
}

%=============================================================================
% PACKAGES
%=============================================================================
\usepackage[utf8]{inputenc}
\usepackage[T1]{fontenc}
\usepackage[english]{babel}
\usepackage{amsmath, amssymb, amsthm}
\usepackage{mathtools}
\usepackage{bm}
\usepackage{tikz}
\usetikzlibrary{arrows.meta, positioning, shapes, calc, decorations.pathreplacing, shadings}
\usepackage{booktabs}
\usepackage{multirow}
\usepackage{array}
\usepackage{graphicx}
\usepackage{hyperref}
\usepackage{colortbl}
\usepackage{listings}
\lstset{basicstyle=\ttfamily\small, breaklines=true, frame=single, backgroundcolor=\color{VeryLightGray}}
\hypersetup{colorlinks=true, linkcolor=MainBlue, urlcolor=MainBlue}
\graphicspath{{../../logos/}{../../charts/}{../../photos/}}
\hfuzz=2pt  % Suppress tiny overfull warnings (<2pt)
\vfuzz=2pt  % Suppress tiny vertical overfull warnings (<2pt)

%=============================================================================
% QUANTLET COMMAND
%=============================================================================
\newcommand{\quantlet}[2]{%
    \hfill\href{#2}{%
        \raisebox{-0.15em}{\includegraphics[height=0.7em]{ql_logo.png}}%
        \textcolor{MainBlue}{\tiny\ #1}%
    }%
}

%=============================================================================
% CUSTOM TITLE PAGE
%=============================================================================
\defbeamertemplate*{title page}{hybrid}[1][]
{
    \vspace{0.2cm}
    % Logos row - top header (with clickable links)
    \begin{center}
        \href{https://www.ase.ro}{\includegraphics[height=1.0cm]{ase_logo.png}}\hspace{0.25cm}%
        \href{https://theida.net}{\includegraphics[height=1.0cm]{ida_logo.png}}\hspace{0.25cm}%
        \href{https://blockchain-research-center.com}{\includegraphics[height=1.0cm]{brc_logo.png}}\hspace{0.25cm}%
        \href{https://www.ai4efin.ase.ro}{\includegraphics[height=1.0cm]{ai4efin_logo.png}}\hspace{0.25cm}%
        \href{https://ipe.ro/new}{\includegraphics[height=1.0cm]{acad_logo.png}}\hspace{0.25cm}%
        \href{https://www.digital-finance-msca.com}{\includegraphics[height=1.0cm]{msca_logo.png}}%
    \end{center}

    \vspace{0.6cm}

    % Main title with Q logos on sides (with clickable links)
    \begin{center}
        \begin{minipage}{0.1\textwidth}
            \centering
            \href{https://quantlet.com}{\includegraphics[height=1.1cm]{ql_logo.png}}
        \end{minipage}%
        \begin{minipage}{0.78\textwidth}
            \centering
            {\LARGE\bfseries\usebeamercolor[fg]{title}\inserttitle}

            \vspace{0.3cm}

            {\usebeamerfont{subtitle}\usebeamercolor[fg]{title}\insertsubtitle}
        \end{minipage}%
        \begin{minipage}{0.1\textwidth}
            \centering
            \href{https://quantinar.com}{\includegraphics[height=1.1cm]{qr_logo.png}}
        \end{minipage}
    \end{center}

    \vspace{0.6cm}

    % Authors (left aligned)
    \hspace{0.5cm}{\usebeamerfont{author}\insertauthor}

    \vspace{0.3cm}

    % Institute/Affiliations (left aligned)
    \hspace{0.5cm}\begin{minipage}[t]{0.9\textwidth}
        \raggedright\small\insertinstitute
    \end{minipage}
}

%=============================================================================
% THEOREM ENVIRONMENTS
%=============================================================================
\theoremstyle{definition}
\setbeamertemplate{theorems}[numbered]
\newtheorem{defn}{Definition}
\newtheorem{thm}{Theorem}
\newtheorem{prop}{Proposition}
\newtheorem{rmk}{Remark}

%=============================================================================
% CENTRED MINIPAGE (no extra vertical space)
%=============================================================================
\newenvironment{cminipage}[1]{%
    \par\noindent\hfill\begin{minipage}{#1}\ignorespaces
}{%
    \end{minipage}\hfill\null\par
}

%=============================================================================
% CUSTOM COMMANDS
%=============================================================================
\newcommand{\E}{\mathbb{E}}
\newcommand{\Var}{\text{Var}}
\newcommand{\Cov}{\text{Cov}}
\newcommand{\Corr}{\text{Corr}}
\newcommand{\R}{\mathbb{R}}
\newcommand{\N}{\mathbb{N}}
\newcommand{\Z}{\mathbb{Z}}
\newcommand{\B}{\mathbf{B}}
\newcommand{\imark}{\textcolor{MainBlue}{\textbullet}}
\newcommand{\RMSE}{\text{RMSE}}
\newcommand{\MAE}{\text{MAE}}
\newcommand{\MAPE}{\text{MAPE}}
\newcommand{\correct}{\textcolor{Forest}{\checkmark}}
\newcommand{\incorrect}{\textcolor{Crimson}{\texttimes}}

% Boldface vector/matrix commands
\newcommand{\bY}{\mathbf{Y}}
\newcommand{\bX}{\mathbf{X}}
\newcommand{\bA}{\mathbf{A}}
\newcommand{\bB}{\mathbf{B}}
\newcommand{\bepsilon}{\boldsymbol{\varepsilon}}
\newcommand{\bvarepsilon}{\boldsymbol{\varepsilon}}
\newcommand{\bSigma}{\boldsymbol{\Sigma}}
\newcommand{\bPhi}{\boldsymbol{\Phi}}
\newcommand{\bGamma}{\boldsymbol{\Gamma}}
\newcommand{\bPi}{\boldsymbol{\Pi}}
\newcommand{\bc}{\mathbf{c}}
\newcommand{\balpha}{\boldsymbol{\alpha}}
\newcommand{\bbeta}{\boldsymbol{\beta}}

%=============================================================================
% TITLE INFORMATION
%=============================================================================
\title[Time Series Analysis]{Time Series Analysis and Forecasting}
\author[D.T. Pele]{Daniel Traian PELE}
\institute{Bucharest University of Economic Studies\\
IDA Institute Digital Assets\\
Blockchain Research Center\\
AI4EFin Artificial Intelligence for Energy Finance\\
Romanian Academy, Institute for Economic Forecasting\\
MSCA Digital Finance}
\date{}

%            \subtitle{Seminar X: Seminar Title}
%            \begin{document} ...
%=============================================================================

% Ensure content fits on slides
\setbeamersize{text margin left=8mm, text margin right=8mm}

%=============================================================================
% THEME AND STYLE CONFIGURATION
%=============================================================================
\usetheme{default}
% Using default theme for clean header/footer control

% Color Palette (matching Redispatch PDF)
\definecolor{MainBlue}{RGB}{26, 58, 110}
\definecolor{AccentBlue}{RGB}{26, 58, 110}
\definecolor{IDAred}{RGB}{205, 0, 0}
\definecolor{DarkGray}{RGB}{51, 51, 51}
\definecolor{MediumGray}{RGB}{128, 128, 128}
\definecolor{LightGray}{RGB}{248, 248, 248}
\definecolor{VeryLightGray}{RGB}{235, 235, 235}
\definecolor{KeynoteGray}{RGB}{218, 218, 218}
\definecolor{SectionGray}{RGB}{120, 120, 120}
\definecolor{FooterGray}{RGB}{100, 100, 100}
\definecolor{Crimson}{RGB}{220, 53, 69}
\definecolor{Forest}{RGB}{46, 125, 50}
\definecolor{Amber}{RGB}{181, 133, 63}
\definecolor{Orange}{RGB}{230, 126, 34}
\definecolor{Purple}{RGB}{142, 68, 173}

% Gradient background (exact Keynote 315° gradient: white to RGB 218,218,218)
\setbeamertemplate{background}{%
    \begin{tikzpicture}[remember picture, overlay]
        \shade[shading=axis, shading angle=315,
        top color=white, bottom color=KeynoteGray]
        (current page.south west) rectangle (current page.north east);
    \end{tikzpicture}%
}
% Fallback solid color for compatibility
\setbeamercolor{background canvas}{bg=}

\setbeamercolor{palette primary}{bg=MainBlue, fg=white}
\setbeamercolor{palette secondary}{bg=MainBlue!85, fg=white}
\setbeamercolor{palette tertiary}{bg=MainBlue!70, fg=white}
\setbeamercolor{structure}{fg=MainBlue}
\setbeamercolor{title}{fg=IDAred}
\setbeamercolor{frametitle}{fg=IDAred, bg=}
\setbeamercolor{block title}{bg=MainBlue, fg=white}
\setbeamercolor{block body}{bg=VeryLightGray, fg=DarkGray}
\setbeamercolor{block title alerted}{bg=Crimson, fg=white}
\setbeamercolor{block body alerted}{bg=Crimson!8, fg=DarkGray}
\setbeamercolor{block title example}{bg=Forest, fg=white}
\setbeamercolor{block body example}{bg=Forest!8, fg=DarkGray}
\setbeamercolor{item}{fg=MainBlue}

% Smaller institute font to avoid overfull hbox on title page
\setbeamerfont{institute}{size=\footnotesize}

% Footer colors (override Madrid theme blue)
\setbeamercolor{author in head/foot}{fg=FooterGray, bg=}
\setbeamercolor{title in head/foot}{fg=FooterGray, bg=}
\setbeamercolor{date in head/foot}{fg=FooterGray, bg=}
\setbeamercolor{section in head/foot}{fg=FooterGray, bg=}
\setbeamercolor{subsection in head/foot}{fg=FooterGray, bg=}

% Bullet styles (apply everywhere including blocks)
\setbeamertemplate{itemize item}{\color{MainBlue}$\boxdot$}
\setbeamertemplate{itemize subitem}{\color{MainBlue}$\blacktriangleright$}
\setbeamertemplate{itemize subsubitem}{\color{MainBlue}\tiny$\bullet$}
\setbeamertemplate{itemize/enumerate body begin}{\normalsize}
\setbeamertemplate{itemize/enumerate subbody begin}{\normalsize}

% Item spacing - compact style
\setlength{\leftmargini}{10pt}       % Level 1: minimal indent
\setlength{\leftmarginii}{10pt}      % Level 2: minimal additional indent
% Compact list spacing (zero extra space before/after lists in blocks)
\makeatletter
\def\@listi{\leftmargin\leftmargini \topsep 0pt \parsep 0pt \itemsep 0pt}
\def\@listii{\leftmargin\leftmarginii \topsep 0pt \parsep 0pt \itemsep 0pt}
\makeatother

\setbeamertemplate{navigation symbols}{}

%=============================================================================
% CUSTOM HEADLINE
%=============================================================================
\setbeamertemplate{headline}{%
    \vskip10pt%
    \hbox to \paperwidth{%
        \hskip0.5cm%
        {\small\color{FooterGray}\renewcommand{\hyperlink}[2]{##2}\insertsectionhead}%
        \hfill%
        \textcolor{FooterGray}{\small\insertframenumber}%
        \hskip0.5cm%
    }%
    \vskip4pt%
    {\color{FooterGray}\hrule height 0.4pt}%
}

%=============================================================================
% CUSTOM FOOTER
%=============================================================================
\usepackage{fontawesome5}

\setbeamertemplate{footline}{%
    {\color{FooterGray}\hrule height 0.4pt}%
    \vskip4pt%
    \hbox to \paperwidth{%
        \hskip0.5cm%
        \textcolor{FooterGray}{\small Time Series Analysis and Forecasting}%
        \hfill%
        \raisebox{-0.1em}{%
            \begin{tikzpicture}[x=0.08em, y=0.08em, line width=0.4pt]
                \draw[FooterGray] (0,3) -- (1,4) -- (2,3.5) -- (3,5) -- (4,4) -- (5,6) -- (6,5.5) -- (7,4) -- (8,5) -- (9,7) -- (10,6) -- (11,5) -- (12,6.5) -- (13,8) -- (14,7) -- (15,6) -- (16,7.5) -- (17,9) -- (18,8) -- (19,7) -- (20,8.5) -- (21,10) -- (22,9) -- (23,8) -- (24,9.5);
            \end{tikzpicture}%
        }%
        \hskip0.5cm%
    }%
    \vskip6pt%
}

%=============================================================================
% PACKAGES
%=============================================================================
\usepackage[utf8]{inputenc}
\usepackage[T1]{fontenc}
\usepackage[english]{babel}
\usepackage{amsmath, amssymb, amsthm}
\usepackage{mathtools}
\usepackage{bm}
\usepackage{tikz}
\usetikzlibrary{arrows.meta, positioning, shapes, calc, decorations.pathreplacing, shadings}
\usepackage{booktabs}
\usepackage{multirow}
\usepackage{array}
\usepackage{graphicx}
\usepackage{hyperref}
\usepackage{colortbl}
\usepackage{listings}
\lstset{basicstyle=\ttfamily\small, breaklines=true, frame=single, backgroundcolor=\color{VeryLightGray}}
\hypersetup{colorlinks=true, linkcolor=MainBlue, urlcolor=MainBlue}
\graphicspath{{../../logos/}{../../charts/}{../../photos/}}
\hfuzz=2pt  % Suppress tiny overfull warnings (<2pt)
\vfuzz=2pt  % Suppress tiny vertical overfull warnings (<2pt)

%=============================================================================
% QUANTLET COMMAND
%=============================================================================
\newcommand{\quantlet}[2]{%
    \hfill\href{#2}{%
        \raisebox{-0.15em}{\includegraphics[height=0.7em]{ql_logo.png}}%
        \textcolor{MainBlue}{\tiny\ #1}%
    }%
}

%=============================================================================
% CUSTOM TITLE PAGE
%=============================================================================
\defbeamertemplate*{title page}{hybrid}[1][]
{
    \vspace{0.2cm}
    % Logos row - top header (with clickable links)
    \begin{center}
        \href{https://www.ase.ro}{\includegraphics[height=1.0cm]{ase_logo.png}}\hspace{0.25cm}%
        \href{https://theida.net}{\includegraphics[height=1.0cm]{ida_logo.png}}\hspace{0.25cm}%
        \href{https://blockchain-research-center.com}{\includegraphics[height=1.0cm]{brc_logo.png}}\hspace{0.25cm}%
        \href{https://www.ai4efin.ase.ro}{\includegraphics[height=1.0cm]{ai4efin_logo.png}}\hspace{0.25cm}%
        \href{https://ipe.ro/new}{\includegraphics[height=1.0cm]{acad_logo.png}}\hspace{0.25cm}%
        \href{https://www.digital-finance-msca.com}{\includegraphics[height=1.0cm]{msca_logo.png}}%
    \end{center}

    \vspace{0.6cm}

    % Main title with Q logos on sides (with clickable links)
    \begin{center}
        \begin{minipage}{0.1\textwidth}
            \centering
            \href{https://quantlet.com}{\includegraphics[height=1.1cm]{ql_logo.png}}
        \end{minipage}%
        \begin{minipage}{0.78\textwidth}
            \centering
            {\LARGE\bfseries\usebeamercolor[fg]{title}\inserttitle}

            \vspace{0.3cm}

            {\usebeamerfont{subtitle}\usebeamercolor[fg]{title}\insertsubtitle}
        \end{minipage}%
        \begin{minipage}{0.1\textwidth}
            \centering
            \href{https://quantinar.com}{\includegraphics[height=1.1cm]{qr_logo.png}}
        \end{minipage}
    \end{center}

    \vspace{0.6cm}

    % Authors (left aligned)
    \hspace{0.5cm}{\usebeamerfont{author}\insertauthor}

    \vspace{0.3cm}

    % Institute/Affiliations (left aligned)
    \hspace{0.5cm}\begin{minipage}[t]{0.9\textwidth}
        \raggedright\small\insertinstitute
    \end{minipage}
}

%=============================================================================
% THEOREM ENVIRONMENTS
%=============================================================================
\theoremstyle{definition}
\setbeamertemplate{theorems}[numbered]
\newtheorem{defn}{Definition}
\newtheorem{thm}{Theorem}
\newtheorem{prop}{Proposition}
\newtheorem{rmk}{Remark}

%=============================================================================
% CENTRED MINIPAGE (no extra vertical space)
%=============================================================================
\newenvironment{cminipage}[1]{%
    \par\noindent\hfill\begin{minipage}{#1}\ignorespaces
}{%
    \end{minipage}\hfill\null\par
}

%=============================================================================
% CUSTOM COMMANDS
%=============================================================================
\newcommand{\E}{\mathbb{E}}
\newcommand{\Var}{\text{Var}}
\newcommand{\Cov}{\text{Cov}}
\newcommand{\Corr}{\text{Corr}}
\newcommand{\R}{\mathbb{R}}
\newcommand{\N}{\mathbb{N}}
\newcommand{\Z}{\mathbb{Z}}
\newcommand{\B}{\mathbf{B}}
\newcommand{\imark}{\textcolor{MainBlue}{\textbullet}}
\newcommand{\RMSE}{\text{RMSE}}
\newcommand{\MAE}{\text{MAE}}
\newcommand{\MAPE}{\text{MAPE}}
\newcommand{\correct}{\textcolor{Forest}{\checkmark}}
\newcommand{\incorrect}{\textcolor{Crimson}{\texttimes}}

% Boldface vector/matrix commands
\newcommand{\bY}{\mathbf{Y}}
\newcommand{\bX}{\mathbf{X}}
\newcommand{\bA}{\mathbf{A}}
\newcommand{\bB}{\mathbf{B}}
\newcommand{\bepsilon}{\boldsymbol{\varepsilon}}
\newcommand{\bvarepsilon}{\boldsymbol{\varepsilon}}
\newcommand{\bSigma}{\boldsymbol{\Sigma}}
\newcommand{\bPhi}{\boldsymbol{\Phi}}
\newcommand{\bGamma}{\boldsymbol{\Gamma}}
\newcommand{\bPi}{\boldsymbol{\Pi}}
\newcommand{\bc}{\mathbf{c}}
\newcommand{\balpha}{\boldsymbol{\alpha}}
\newcommand{\bbeta}{\boldsymbol{\beta}}

%=============================================================================
% TITLE INFORMATION
%=============================================================================
\title[Time Series Analysis]{Time Series Analysis and Forecasting}
\author[D.T. Pele]{Daniel Traian PELE}
\institute{Bucharest University of Economic Studies\\
IDA Institute Digital Assets\\
Blockchain Research Center\\
AI4EFin Artificial Intelligence for Energy Finance\\
Romanian Academy, Institute for Economic Forecasting\\
MSCA Digital Finance}
\date{}

%            \subtitle{Seminar X: Seminar Title}
%            \begin{document} ...
%=============================================================================

% Ensure content fits on slides
\setbeamersize{text margin left=8mm, text margin right=8mm}

%=============================================================================
% THEME AND STYLE CONFIGURATION
%=============================================================================
\usetheme{default}
% Using default theme for clean header/footer control

% Color Palette (matching Redispatch PDF)
\definecolor{MainBlue}{RGB}{26, 58, 110}
\definecolor{AccentBlue}{RGB}{26, 58, 110}
\definecolor{IDAred}{RGB}{205, 0, 0}
\definecolor{DarkGray}{RGB}{51, 51, 51}
\definecolor{MediumGray}{RGB}{128, 128, 128}
\definecolor{LightGray}{RGB}{248, 248, 248}
\definecolor{VeryLightGray}{RGB}{235, 235, 235}
\definecolor{KeynoteGray}{RGB}{218, 218, 218}
\definecolor{SectionGray}{RGB}{120, 120, 120}
\definecolor{FooterGray}{RGB}{100, 100, 100}
\definecolor{Crimson}{RGB}{220, 53, 69}
\definecolor{Forest}{RGB}{46, 125, 50}
\definecolor{Amber}{RGB}{181, 133, 63}
\definecolor{Orange}{RGB}{230, 126, 34}
\definecolor{Purple}{RGB}{142, 68, 173}

% Gradient background (exact Keynote 315° gradient: white to RGB 218,218,218)
\setbeamertemplate{background}{%
    \begin{tikzpicture}[remember picture, overlay]
        \shade[shading=axis, shading angle=315,
        top color=white, bottom color=KeynoteGray]
        (current page.south west) rectangle (current page.north east);
    \end{tikzpicture}%
}
% Fallback solid color for compatibility
\setbeamercolor{background canvas}{bg=}

\setbeamercolor{palette primary}{bg=MainBlue, fg=white}
\setbeamercolor{palette secondary}{bg=MainBlue!85, fg=white}
\setbeamercolor{palette tertiary}{bg=MainBlue!70, fg=white}
\setbeamercolor{structure}{fg=MainBlue}
\setbeamercolor{title}{fg=IDAred}
\setbeamercolor{frametitle}{fg=IDAred, bg=}
\setbeamercolor{block title}{bg=MainBlue, fg=white}
\setbeamercolor{block body}{bg=VeryLightGray, fg=DarkGray}
\setbeamercolor{block title alerted}{bg=Crimson, fg=white}
\setbeamercolor{block body alerted}{bg=Crimson!8, fg=DarkGray}
\setbeamercolor{block title example}{bg=Forest, fg=white}
\setbeamercolor{block body example}{bg=Forest!8, fg=DarkGray}
\setbeamercolor{item}{fg=MainBlue}

% Smaller institute font to avoid overfull hbox on title page
\setbeamerfont{institute}{size=\footnotesize}

% Footer colors (override Madrid theme blue)
\setbeamercolor{author in head/foot}{fg=FooterGray, bg=}
\setbeamercolor{title in head/foot}{fg=FooterGray, bg=}
\setbeamercolor{date in head/foot}{fg=FooterGray, bg=}
\setbeamercolor{section in head/foot}{fg=FooterGray, bg=}
\setbeamercolor{subsection in head/foot}{fg=FooterGray, bg=}

% Bullet styles (apply everywhere including blocks)
\setbeamertemplate{itemize item}{\color{MainBlue}$\boxdot$}
\setbeamertemplate{itemize subitem}{\color{MainBlue}$\blacktriangleright$}
\setbeamertemplate{itemize subsubitem}{\color{MainBlue}\tiny$\bullet$}
\setbeamertemplate{itemize/enumerate body begin}{\normalsize}
\setbeamertemplate{itemize/enumerate subbody begin}{\normalsize}

% Item spacing - compact style
\setlength{\leftmargini}{10pt}       % Level 1: minimal indent
\setlength{\leftmarginii}{10pt}      % Level 2: minimal additional indent
% Compact list spacing (zero extra space before/after lists in blocks)
\makeatletter
\def\@listi{\leftmargin\leftmargini \topsep 0pt \parsep 0pt \itemsep 0pt}
\def\@listii{\leftmargin\leftmarginii \topsep 0pt \parsep 0pt \itemsep 0pt}
\makeatother

\setbeamertemplate{navigation symbols}{}

%=============================================================================
% CUSTOM HEADLINE
%=============================================================================
\setbeamertemplate{headline}{%
    \vskip10pt%
    \hbox to \paperwidth{%
        \hskip0.5cm%
        {\small\color{FooterGray}\renewcommand{\hyperlink}[2]{##2}\insertsectionhead}%
        \hfill%
        \textcolor{FooterGray}{\small\insertframenumber}%
        \hskip0.5cm%
    }%
    \vskip4pt%
    {\color{FooterGray}\hrule height 0.4pt}%
}

%=============================================================================
% CUSTOM FOOTER
%=============================================================================
\usepackage{fontawesome5}

\setbeamertemplate{footline}{%
    {\color{FooterGray}\hrule height 0.4pt}%
    \vskip4pt%
    \hbox to \paperwidth{%
        \hskip0.5cm%
        \textcolor{FooterGray}{\small Time Series Analysis and Forecasting}%
        \hfill%
        \raisebox{-0.1em}{%
            \begin{tikzpicture}[x=0.08em, y=0.08em, line width=0.4pt]
                \draw[FooterGray] (0,3) -- (1,4) -- (2,3.5) -- (3,5) -- (4,4) -- (5,6) -- (6,5.5) -- (7,4) -- (8,5) -- (9,7) -- (10,6) -- (11,5) -- (12,6.5) -- (13,8) -- (14,7) -- (15,6) -- (16,7.5) -- (17,9) -- (18,8) -- (19,7) -- (20,8.5) -- (21,10) -- (22,9) -- (23,8) -- (24,9.5);
            \end{tikzpicture}%
        }%
        \hskip0.5cm%
    }%
    \vskip6pt%
}

%=============================================================================
% PACKAGES
%=============================================================================
\usepackage[utf8]{inputenc}
\usepackage[T1]{fontenc}
\usepackage[english]{babel}
\usepackage{amsmath, amssymb, amsthm}
\usepackage{mathtools}
\usepackage{bm}
\usepackage{tikz}
\usetikzlibrary{arrows.meta, positioning, shapes, calc, decorations.pathreplacing, shadings}
\usepackage{booktabs}
\usepackage{multirow}
\usepackage{array}
\usepackage{graphicx}
\usepackage{hyperref}
\usepackage{colortbl}
\usepackage{listings}
\lstset{basicstyle=\ttfamily\small, breaklines=true, frame=single, backgroundcolor=\color{VeryLightGray}}
\hypersetup{colorlinks=true, linkcolor=MainBlue, urlcolor=MainBlue}
\graphicspath{{../../logos/}{../../charts/}{../../photos/}}
\hfuzz=2pt  % Suppress tiny overfull warnings (<2pt)
\vfuzz=2pt  % Suppress tiny vertical overfull warnings (<2pt)

%=============================================================================
% QUANTLET COMMAND
%=============================================================================
\newcommand{\quantlet}[2]{%
    \hfill\href{#2}{%
        \raisebox{-0.15em}{\includegraphics[height=0.7em]{ql_logo.png}}%
        \textcolor{MainBlue}{\tiny\ #1}%
    }%
}

%=============================================================================
% CUSTOM TITLE PAGE
%=============================================================================
\defbeamertemplate*{title page}{hybrid}[1][]
{
    \vspace{0.2cm}
    % Logos row - top header (with clickable links)
    \begin{center}
        \href{https://www.ase.ro}{\includegraphics[height=1.0cm]{ase_logo.png}}\hspace{0.25cm}%
        \href{https://theida.net}{\includegraphics[height=1.0cm]{ida_logo.png}}\hspace{0.25cm}%
        \href{https://blockchain-research-center.com}{\includegraphics[height=1.0cm]{brc_logo.png}}\hspace{0.25cm}%
        \href{https://www.ai4efin.ase.ro}{\includegraphics[height=1.0cm]{ai4efin_logo.png}}\hspace{0.25cm}%
        \href{https://ipe.ro/new}{\includegraphics[height=1.0cm]{acad_logo.png}}\hspace{0.25cm}%
        \href{https://www.digital-finance-msca.com}{\includegraphics[height=1.0cm]{msca_logo.png}}%
    \end{center}

    \vspace{0.6cm}

    % Main title with Q logos on sides (with clickable links)
    \begin{center}
        \begin{minipage}{0.1\textwidth}
            \centering
            \href{https://quantlet.com}{\includegraphics[height=1.1cm]{ql_logo.png}}
        \end{minipage}%
        \begin{minipage}{0.78\textwidth}
            \centering
            {\LARGE\bfseries\usebeamercolor[fg]{title}\inserttitle}

            \vspace{0.3cm}

            {\usebeamerfont{subtitle}\usebeamercolor[fg]{title}\insertsubtitle}
        \end{minipage}%
        \begin{minipage}{0.1\textwidth}
            \centering
            \href{https://quantinar.com}{\includegraphics[height=1.1cm]{qr_logo.png}}
        \end{minipage}
    \end{center}

    \vspace{0.6cm}

    % Authors (left aligned)
    \hspace{0.5cm}{\usebeamerfont{author}\insertauthor}

    \vspace{0.3cm}

    % Institute/Affiliations (left aligned)
    \hspace{0.5cm}\begin{minipage}[t]{0.9\textwidth}
        \raggedright\small\insertinstitute
    \end{minipage}
}

%=============================================================================
% THEOREM ENVIRONMENTS
%=============================================================================
\theoremstyle{definition}
\setbeamertemplate{theorems}[numbered]
\newtheorem{defn}{Definition}
\newtheorem{thm}{Theorem}
\newtheorem{prop}{Proposition}
\newtheorem{rmk}{Remark}

%=============================================================================
% CENTRED MINIPAGE (no extra vertical space)
%=============================================================================
\newenvironment{cminipage}[1]{%
    \par\noindent\hfill\begin{minipage}{#1}\ignorespaces
}{%
    \end{minipage}\hfill\null\par
}

%=============================================================================
% CUSTOM COMMANDS
%=============================================================================
\newcommand{\E}{\mathbb{E}}
\newcommand{\Var}{\text{Var}}
\newcommand{\Cov}{\text{Cov}}
\newcommand{\Corr}{\text{Corr}}
\newcommand{\R}{\mathbb{R}}
\newcommand{\N}{\mathbb{N}}
\newcommand{\Z}{\mathbb{Z}}
\newcommand{\B}{\mathbf{B}}
\newcommand{\imark}{\textcolor{MainBlue}{\textbullet}}
\newcommand{\RMSE}{\text{RMSE}}
\newcommand{\MAE}{\text{MAE}}
\newcommand{\MAPE}{\text{MAPE}}
\newcommand{\correct}{\textcolor{Forest}{\checkmark}}
\newcommand{\incorrect}{\textcolor{Crimson}{\texttimes}}

% Boldface vector/matrix commands
\newcommand{\bY}{\mathbf{Y}}
\newcommand{\bX}{\mathbf{X}}
\newcommand{\bA}{\mathbf{A}}
\newcommand{\bB}{\mathbf{B}}
\newcommand{\bepsilon}{\boldsymbol{\varepsilon}}
\newcommand{\bvarepsilon}{\boldsymbol{\varepsilon}}
\newcommand{\bSigma}{\boldsymbol{\Sigma}}
\newcommand{\bPhi}{\boldsymbol{\Phi}}
\newcommand{\bGamma}{\boldsymbol{\Gamma}}
\newcommand{\bPi}{\boldsymbol{\Pi}}
\newcommand{\bc}{\mathbf{c}}
\newcommand{\balpha}{\boldsymbol{\alpha}}
\newcommand{\bbeta}{\boldsymbol{\beta}}

%=============================================================================
% TITLE INFORMATION
%=============================================================================
\title[Time Series Analysis]{Time Series Analysis and Forecasting}
\author[D.T. Pele]{Daniel Traian PELE}
\institute{Bucharest University of Economic Studies\\
IDA Institute Digital Assets\\
Blockchain Research Center\\
AI4EFin Artificial Intelligence for Energy Finance\\
Romanian Academy, Institute for Economic Forecasting\\
MSCA Digital Finance}
\date{}

\subtitle{Seminar 9: Prophet and TBATS}

\begin{document}

{
\setbeamertemplate{headline}{}
\setbeamertemplate{footline}{}
\begin{frame}
    \titlepage
\end{frame}
}


\begin{frame}{Seminar Outline}
    \begin{cminipage}{0.95\textwidth}
    \begin{itemize}
        \item \textbf{Multiple Choice Quiz} -- Knowledge check
        \vspace{0.15cm}
        \item \textbf{True/False} -- Conceptual checks
        \vspace{0.15cm}
        \item \textbf{Calculation Exercises} -- Applied practice
        \vspace{0.15cm}
        \item \textbf{Worked Examples} -- Detailed solutions
        \vspace{0.15cm}
        \item \textbf{AI-Assisted Exercise} -- Critical thinking
        \vspace{0.15cm}
        \item \textbf{Summary} -- Key takeaways
    \end{itemize}
    \end{cminipage}
\end{frame}

%=============================================================================
% MULTIPLE CHOICE QUIZ
%=============================================================================
\section{Multiple Choice Quiz}

\begin{frame}{Quiz 1: Multiple Seasonality Challenge}
    \begin{cminipage}{0.95\textwidth}
    \begin{alertblock}{Question}
        Why can't standard SARIMA handle hourly electricity demand data?
    \end{alertblock}

    \vspace{0.4cm}

    \begin{block}{Answer choices}
        \textcolor{MainBlue}{\textbf{(A)}} SARIMA can only handle monthly data\\[3pt]
        \textcolor{MainBlue}{\textbf{(B)}} SARIMA allows only one seasonal period ($m$ parameter)\\[3pt]
        \textcolor{MainBlue}{\textbf{(C)}} SARIMA doesn't support trend components\\[3pt]
        \textcolor{MainBlue}{\textbf{(D)}} SARIMA requires normally distributed data
    \end{block}

    \vspace{0.5cm}

    \begin{center}
        \textit{Answer on next slide...}
    \end{center}
    \end{cminipage}
\end{frame}

\begin{frame}{Quiz 1: Answer}
    \begin{cminipage}{0.95\textwidth}
    \begin{exampleblock}{Answer: B -- SARIMA allows only one seasonal period}
        \textbf{Question:} Why can't standard SARIMA handle hourly electricity demand data?

        \vspace{0.3cm}

    \begin{block}{Answer choices}
        \textcolor{MainBlue}{\textbf{(A)}} SARIMA can only handle monthly data \incorrect\\[3pt]
        \textcolor{MainBlue}{\textbf{(B)}} \textbf{\textcolor{Forest}{SARIMA allows only one seasonal period ($m$ parameter)}} \correct\\[3pt]
        \textcolor{MainBlue}{\textbf{(C)}} SARIMA doesn't support trend components \incorrect\\[3pt]
        \textcolor{MainBlue}{\textbf{(D)}} SARIMA requires normally distributed data \incorrect
    \end{block}

        \vspace{0.3cm}

        \begin{itemize}
            \item Hourly data has daily (24h), weekly (168h), and annual (8760h) patterns
            \item SARIMA's single $m$ parameter cannot capture all these simultaneously
            \item Use TBATS or Prophet for \textbf{multiple seasonality}
        \end{itemize}
    \end{exampleblock}

    \end{cminipage}
    \quantlet{TSA\_ch9\_quiz1\_multiple\_seasonality}{https://github.com/QuantLet/TSA/tree/main/TSA_ch9/TSA_ch9_quiz1_multiple_seasonality}
\end{frame}

\begin{frame}{Quiz 2: TBATS Acronym}
    \begin{cminipage}{0.95\textwidth}
    \begin{alertblock}{Question}
        What does TBATS stand for?
    \end{alertblock}

    \vspace{0.4cm}

    \begin{block}{Answer choices}
        \textcolor{MainBlue}{\textbf{(A)}} Trend, Baseline, ARMA, Transform, Seasonal\\[3pt]
        \textcolor{MainBlue}{\textbf{(B)}} Trigonometric, Box-Cox, ARMA, Trend, Seasonal\\[3pt]
        \textcolor{MainBlue}{\textbf{(C)}} Time-Based Automatic Time Series\\[3pt]
        \textcolor{MainBlue}{\textbf{(D)}} Temporal Bayesian Adaptive Trend System
    \end{block}

    \vspace{0.5cm}

    \begin{center}
        \textit{Answer on next slide...}
    \end{center}
    \end{cminipage}
\end{frame}

\begin{frame}{Quiz 2: Answer}
    \begin{cminipage}{0.95\textwidth}
    \begin{exampleblock}{Answer: B -- Trigonometric, Box-Cox, ARMA, Trend, Seasonal}
        \textbf{Question:} What does TBATS stand for?

        \vspace{0.3cm}

    \begin{block}{Answer choices}
        \textcolor{MainBlue}{\textbf{(A)}} Trend, Baseline, ARMA, Transform, Seasonal \incorrect\\[3pt]
        \textcolor{MainBlue}{\textbf{(B)}} \textbf{\textcolor{Forest}{Trigonometric, Box-Cox, ARMA, Trend, Seasonal}} \correct\\[3pt]
        \textcolor{MainBlue}{\textbf{(C)}} Time-Based Automatic Time Series \incorrect\\[3pt]
        \textcolor{MainBlue}{\textbf{(D)}} Temporal Bayesian Adaptive Trend System \incorrect
    \end{block}

        \vspace{0.3cm}

        \begin{itemize}
            \item \textbf{T}rigonometric (Fourier for seasonality), \textbf{B}ox-Cox (variance stabilization)
            \item \textbf{A}RMA (error autocorrelation), \textbf{T}rend (damped local), \textbf{S}easonal (multiple periods)
        \end{itemize}
    \end{exampleblock}

    \end{cminipage}
    \quantlet{TSA\_ch9\_quiz2\_tbats\_components}{https://github.com/QuantLet/TSA/tree/main/TSA_ch9/TSA_ch9_quiz2_tbats_components}
\end{frame}

\begin{frame}{Quiz 3: Fourier Terms}
    \begin{cminipage}{0.95\textwidth}
    \begin{alertblock}{Question}
        In TBATS, increasing the number of Fourier harmonics ($K$) for a seasonal pattern:
    \end{alertblock}

    \vspace{0.4cm}

    \begin{block}{Answer choices}
        \textcolor{MainBlue}{\textbf{(A)}} Always improves forecast accuracy\\[3pt]
        \textcolor{MainBlue}{\textbf{(B)}} Allows more flexible (complex) seasonal shapes\\[3pt]
        \textcolor{MainBlue}{\textbf{(C)}} Reduces the model complexity\\[3pt]
        \textcolor{MainBlue}{\textbf{(D)}} Eliminates the need for Box-Cox transformation
    \end{block}

    \vspace{0.5cm}

    \begin{center}
        \textit{Answer on next slide...}
    \end{center}
    \end{cminipage}
\end{frame}

\begin{frame}{Quiz 3: Answer}
    \begin{cminipage}{0.95\textwidth}
    \begin{exampleblock}{Answer: B -- Allows more flexible seasonal shapes}
        \textbf{Question:} In TBATS, increasing the number of Fourier harmonics ($K$) for a seasonal pattern:

        \vspace{0.3cm}

    \begin{block}{Answer choices}
        \textcolor{MainBlue}{\textbf{(A)}} Always improves forecast accuracy \incorrect\\[3pt]
        \textcolor{MainBlue}{\textbf{(B)}} \textbf{\textcolor{Forest}{Allows more flexible (complex) seasonal shapes}} \correct\\[3pt]
        \textcolor{MainBlue}{\textbf{(C)}} Reduces the model complexity \incorrect\\[3pt]
        \textcolor{MainBlue}{\textbf{(D)}} Eliminates the need for Box-Cox transformation \incorrect
    \end{block}

        \vspace{0.3cm}

        \begin{itemize}
            \item \textbf{Trade-off}: More harmonics = more flexibility but also more parameters
            \item $s_t^{(i)} = \sum_{j=1}^{K_i} \left[ a_j^{(i)} \cos\left(\frac{2\pi j t}{m_i}\right) + b_j^{(i)} \sin\left(\frac{2\pi j t}{m_i}\right) \right]$
        \end{itemize}
    \end{exampleblock}

    \end{cminipage}
    \quantlet{TSA\_ch9\_quiz3\_fourier\_harmonics}{https://github.com/QuantLet/TSA/tree/main/TSA_ch9/TSA_ch9_quiz3_fourier_harmonics}
\end{frame}

\begin{frame}{Quiz 4: Prophet Decomposition}
    \begin{cminipage}{0.95\textwidth}
    \begin{alertblock}{Question}
        Prophet decomposes a time series into which components?
    \end{alertblock}

    \vspace{0.4cm}

    \begin{block}{Answer choices}
        \textcolor{MainBlue}{\textbf{(A)}} AR, MA, and seasonal components\\[3pt]
        \textcolor{MainBlue}{\textbf{(B)}} Trend, seasonality, holidays, and error\\[3pt]
        \textcolor{MainBlue}{\textbf{(C)}} Mean, variance, and autocorrelation\\[3pt]
        \textcolor{MainBlue}{\textbf{(D)}} Level, slope, and curvature
    \end{block}

    \vspace{0.5cm}

    \begin{center}
        \textit{Answer on next slide...}
    \end{center}
    \end{cminipage}
\end{frame}

\begin{frame}{Quiz 4: Answer}
    \begin{cminipage}{0.95\textwidth}
    \begin{exampleblock}{Answer: B -- Trend, seasonality, holidays, and error}
        \textbf{Question:} Prophet decomposes a time series into which components?

        \vspace{0.3cm}

    \begin{block}{Answer choices}
        \textcolor{MainBlue}{\textbf{(A)}} AR, MA, and seasonal components \incorrect\\[3pt]
        \textcolor{MainBlue}{\textbf{(B)}} \textbf{\textcolor{Forest}{Trend, seasonality, holidays, and error}} \correct\\[3pt]
        \textcolor{MainBlue}{\textbf{(C)}} Mean, variance, and autocorrelation \incorrect\\[3pt]
        \textcolor{MainBlue}{\textbf{(D)}} Level, slope, and curvature \incorrect
    \end{block}

        \vspace{0.3cm}

        \begin{itemize}
            \item $y(t) = g(t) + s(t) + h(t) + \varepsilon_t$
            \item $g(t)$ = trend, $s(t)$ = seasonality (Fourier), $h(t)$ = holidays, $\varepsilon_t$ = error
        \end{itemize}
    \end{exampleblock}

    \end{cminipage}
    \quantlet{TSA\_ch9\_quiz4\_prophet\_decomposition}{https://github.com/QuantLet/TSA/tree/main/TSA_ch9/TSA_ch9_quiz4_prophet_decomposition}
\end{frame}

\begin{frame}{Quiz 5: Prophet vs TBATS}
    \begin{cminipage}{0.95\textwidth}
    \begin{alertblock}{Question}
        When would you choose Prophet over TBATS?
    \end{alertblock}

    \vspace{0.4cm}

    \begin{block}{Answer choices}
        \textcolor{MainBlue}{\textbf{(A)}} When you need automatic model selection\\[3pt]
        \textcolor{MainBlue}{\textbf{(B)}} When you have known holidays and changepoints to incorporate\\[3pt]
        \textcolor{MainBlue}{\textbf{(C)}} When you need the most parsimonious model\\[3pt]
        \textcolor{MainBlue}{\textbf{(D)}} When your data has no trend
    \end{block}

    \vspace{0.5cm}

    \begin{center}
        \textit{Answer on next slide...}
    \end{center}
    \end{cminipage}
\end{frame}

\begin{frame}{Quiz 5: Answer}
    \begin{cminipage}{0.95\textwidth}
    \begin{exampleblock}{Answer: B -- Known holidays and changepoints}
        \textbf{Question:} When would you choose Prophet over TBATS?

        \vspace{0.3cm}

    \begin{block}{Answer choices}
        \textcolor{MainBlue}{\textbf{(A)}} When you need automatic model selection \incorrect\\[3pt]
        \textcolor{MainBlue}{\textbf{(B)}} \textbf{\textcolor{Forest}{When you have known holidays and changepoints to incorporate}} \correct\\[3pt]
        \textcolor{MainBlue}{\textbf{(C)}} When you need the most parsimonious model \incorrect\\[3pt]
        \textcolor{MainBlue}{\textbf{(D)}} When your data has no trend \incorrect
    \end{block}

        \vspace{0.3cm}

        \begin{itemize}
            \item \textbf{Prophet}: Easy holiday integration, analyst-in-the-loop, handles missing data, interpretable
            \item \textbf{TBATS}: Automatic model selection, handles complex seasonality without domain expertise
        \end{itemize}
    \end{exampleblock}

    \end{cminipage}
    \quantlet{TSA\_ch9\_quiz5\_prophet\_vs\_tbats}{https://github.com/QuantLet/TSA/tree/main/TSA_ch9/TSA_ch9_quiz5_prophet_vs_tbats}
\end{frame}

\begin{frame}{Quiz 6: Seasonality Mode}
    \begin{cminipage}{0.95\textwidth}
    \begin{alertblock}{Question}
        For retail sales data where December sales are 3x the monthly average, which seasonality mode is more appropriate in Prophet?
    \end{alertblock}

    \vspace{0.4cm}

    \begin{block}{Answer choices}
        \textcolor{MainBlue}{\textbf{(A)}} Additive seasonality\\[3pt]
        \textcolor{MainBlue}{\textbf{(B)}} Multiplicative seasonality\\[3pt]
        \textcolor{MainBlue}{\textbf{(C)}} Both work equally well\\[3pt]
        \textcolor{MainBlue}{\textbf{(D)}} Neither---use ARIMA instead
    \end{block}

    \vspace{0.5cm}

    \begin{center}
        \textit{Answer on next slide...}
    \end{center}
    \end{cminipage}
\end{frame}

\begin{frame}{Quiz 6: Answer}
    \begin{cminipage}{0.95\textwidth}
    \begin{exampleblock}{Answer: B -- Multiplicative seasonality}

        \vspace{0.1cm}

    \begin{block}{Answer choices}
        \textcolor{MainBlue}{\textbf{(A)}} Additive seasonality \incorrect\\[3pt]
        \textcolor{MainBlue}{\textbf{(B)}} \textbf{\textcolor{Forest}{Multiplicative seasonality}} \correct\\[3pt]
        \textcolor{MainBlue}{\textbf{(C)}} Both work equally well \incorrect\\[3pt]
        \textcolor{MainBlue}{\textbf{(D)}} Neither---use ARIMA instead \incorrect
    \end{block}

        \vspace{0.2cm}

        \begin{itemize}\setlength{\itemsep}{0pt}
            \item When seasonal amplitude scales with the level, use multiplicative
            \item \textbf{Additive}: $y = g(t) + s(t)$ (constant seasonal effect)
            \item \textbf{Multiplicative}: $y = g(t) \cdot (1 + s(t))$ (proportional seasonal effect)
        \end{itemize}
    \end{exampleblock}

    \end{cminipage}
    \quantlet{TSA\_ch9\_additive\_vs\_multiplicative}{https://github.com/QuantLet/TSA/tree/main/TSA_ch9/TSA_ch9_additive_vs_multiplicative}
\end{frame}

\begin{frame}{Quiz 7: Prophet Changepoints}
    \begin{cminipage}{0.95\textwidth}
    \begin{alertblock}{Question}
        In Prophet, changepoints allow the model to:
    \end{alertblock}

    \vspace{0.4cm}

    \begin{block}{Answer choices}
        \textcolor{MainBlue}{\textbf{(A)}} Change the seasonal period automatically\\[3pt]
        \textcolor{MainBlue}{\textbf{(B)}} Adjust the trend slope at specific points in time\\[3pt]
        \textcolor{MainBlue}{\textbf{(C)}} Switch between additive and multiplicative modes\\[3pt]
        \textcolor{MainBlue}{\textbf{(D)}} Detect and remove outliers
    \end{block}

    \vspace{0.5cm}

    \begin{center}
        \textit{Answer on next slide...}
    \end{center}
    \end{cminipage}
\end{frame}

\begin{frame}{Quiz 7: Answer}
    \begin{cminipage}{0.95\textwidth}
    \begin{exampleblock}{Answer: B -- Adjust trend slope at specific points}
        \textbf{Question:} In Prophet, changepoints allow the model to:

        \vspace{0.3cm}

    \begin{block}{Answer choices}
        \textcolor{MainBlue}{\textbf{(A)}} Change the seasonal period automatically \incorrect\\[3pt]
        \textcolor{MainBlue}{\textbf{(B)}} \textbf{\textcolor{Forest}{Adjust the trend slope at specific points in time}} \correct\\[3pt]
        \textcolor{MainBlue}{\textbf{(C)}} Switch between additive and multiplicative modes \incorrect\\[3pt]
        \textcolor{MainBlue}{\textbf{(D)}} Detect and remove outliers \incorrect
    \end{block}

        \vspace{0.3cm}

        \begin{itemize}
            \item Changepoints allow piecewise linear trend with different slopes
            \item $g(t) = (k + \mathbf{a}(t)^\top \boldsymbol{\delta}) \cdot t + (m + \mathbf{a}(t)^\top \boldsymbol{\gamma})$
            \item Prophet automatically detects changepoints or you can specify them manually
        \end{itemize}
    \end{exampleblock}

    \end{cminipage}
    \quantlet{TSA\_ch9\_changepoint\_detection}{https://github.com/QuantLet/TSA/tree/main/TSA_ch9/TSA_ch9_changepoint_detection}
\end{frame}

\begin{frame}{Quiz 8: Model Selection}
    \begin{cminipage}{0.95\textwidth}
    \begin{alertblock}{Question}
        You have daily call center data with weekly seasonality only. Which model is most appropriate?
    \end{alertblock}

    \vspace{0.4cm}

    \begin{block}{Answer choices}
        \textcolor{MainBlue}{\textbf{(A)}} TBATS (designed for multiple seasonality)\\[3pt]
        \textcolor{MainBlue}{\textbf{(B)}} Prophet (handles any seasonality well)\\[3pt]
        \textcolor{MainBlue}{\textbf{(C)}} Standard SARIMA (simpler and sufficient)\\[3pt]
        \textcolor{MainBlue}{\textbf{(D)}} LSTM neural network (most flexible)
    \end{block}

    \vspace{0.5cm}

    \begin{center}
        \textit{Answer on next slide...}
    \end{center}
    \end{cminipage}
\end{frame}

\begin{frame}{Quiz 8: Answer}
    \begin{cminipage}{0.95\textwidth}
    \begin{exampleblock}{Answer: C -- Standard SARIMA is sufficient}
        \textbf{Question:} You have daily call center data with weekly seasonality only. Which model is most appropriate?

        \vspace{0.3cm}

    \begin{block}{Answer choices}
        \textcolor{MainBlue}{\textbf{(A)}} TBATS (designed for multiple seasonality) \incorrect\\[3pt]
        \textcolor{MainBlue}{\textbf{(B)}} Prophet (handles any seasonality well) \incorrect\\[3pt]
        \textcolor{MainBlue}{\textbf{(C)}} \textbf{\textcolor{Forest}{Standard SARIMA (simpler and sufficient)}} \correct\\[3pt]
        \textcolor{MainBlue}{\textbf{(D)}} LSTM neural network (most flexible) \incorrect
    \end{block}

        \vspace{0.3cm}

        \begin{itemize}
            \item \textbf{Principle of parsimony}: Use the simplest model that fits the data
            \item With only weekly seasonality ($m=7$), SARIMA works fine
            \item Use TBATS/Prophet when you \textit{need} multiple seasonalities or special features
        \end{itemize}
    \end{exampleblock}

    \end{cminipage}
    \quantlet{TSA\_ch9\_model\_selection\_guide}{https://github.com/QuantLet/TSA/tree/main/TSA_ch9/TSA_ch9_model_selection_guide}
\end{frame}

\begin{frame}{Quiz 9: Prophet Uncertainty}
    \begin{cminipage}{0.95\textwidth}
    \begin{alertblock}{Question}
        Prophet generates prediction intervals by:
    \end{alertblock}

    \vspace{0.4cm}

    \begin{block}{Answer choices}
        \textcolor{MainBlue}{\textbf{(A)}} Assuming normally distributed residuals\\[3pt]
        \textcolor{MainBlue}{\textbf{(B)}} Sampling from the posterior distribution of parameters\\[3pt]
        \textcolor{MainBlue}{\textbf{(C)}} Using bootstrap resampling of historical errors\\[3pt]
        \textcolor{MainBlue}{\textbf{(D)}} Applying a fixed multiplier to point forecasts
    \end{block}

    \vspace{0.5cm}

    \begin{center}
        \textit{Answer on next slide...}
    \end{center}
    \end{cminipage}
\end{frame}

\begin{frame}{Quiz 9: Answer}
    \begin{cminipage}{0.95\textwidth}
    \begin{exampleblock}{Answer: B -- Sampling from posterior distribution}
        \textbf{Question:} Prophet generates prediction intervals by:

        \vspace{0.3cm}

    \begin{block}{Answer choices}
        \textcolor{MainBlue}{\textbf{(A)}} Assuming normally distributed residuals \incorrect\\[3pt]
        \textcolor{MainBlue}{\textbf{(B)}} \textbf{\textcolor{Forest}{Sampling from the posterior distribution of parameters}} \correct\\[3pt]
        \textcolor{MainBlue}{\textbf{(C)}} Using bootstrap resampling of historical errors \incorrect\\[3pt]
        \textcolor{MainBlue}{\textbf{(D)}} Applying a fixed multiplier to point forecasts \incorrect
    \end{block}

        \vspace{0.3cm}

        \begin{itemize}
            \item Prophet uses \textbf{Bayesian estimation}: MAP for point forecasts, MCMC/simulation for intervals
            \item Uncertainty from both trend (changepoints) and observation noise
        \end{itemize}
        Note: by default, Prophet uses MAP estimation. Prediction intervals come from simulating future trend changepoints and adding historical residual variability, not from full posterior sampling (MCMC is optional via \texttt{mcmc\_samples > 0}).
    \end{exampleblock}

    \end{cminipage}
    \quantlet{TSA\_ch9\_prophet\_components}{https://github.com/QuantLet/TSA/tree/main/TSA_ch9/TSA_ch9_prophet_components}
\end{frame}

\begin{frame}{Quiz 10: Practical Application}
    \begin{cminipage}{0.95\textwidth}
    \begin{alertblock}{Question}
        For forecasting hourly energy demand with daily, weekly, and annual patterns plus holiday effects, which approach is best?
    \end{alertblock}

    \vspace{0.4cm}

    \begin{block}{Answer choices}
        \textcolor{MainBlue}{\textbf{(A)}} SARIMA with $m=24$\\[3pt]
        \textcolor{MainBlue}{\textbf{(B)}} TBATS with three seasonal periods\\[3pt]
        \textcolor{MainBlue}{\textbf{(C)}} Prophet with custom holidays\\[3pt]
        \textcolor{MainBlue}{\textbf{(D)}} Either TBATS or Prophet, depending on whether holidays are important
    \end{block}

    \vspace{0.5cm}

    \begin{center}
        \textit{Answer on next slide...}
    \end{center}
    \end{cminipage}
\end{frame}

\begin{frame}{Quiz 10: Answer}
    \begin{cminipage}{0.95\textwidth}
    \begin{exampleblock}{Answer: D -- TBATS or Prophet depending on needs}

        \vspace{0.1cm}

    \begin{block}{Answer choices}
        \textcolor{MainBlue}{\textbf{(A)}} SARIMA with $m=24$ \incorrect\\[3pt]
        \textcolor{MainBlue}{\textbf{(B)}} TBATS with three seasonal periods \incorrect\\[3pt]
        \textcolor{MainBlue}{\textbf{(C)}} Prophet with custom holidays \incorrect\\[3pt]
        \textcolor{MainBlue}{\textbf{(D)}} \textbf{\textcolor{Forest}{Either TBATS or Prophet, depending on whether holidays are important}} \correct
    \end{block}

        \vspace{0.2cm}

        \begin{itemize}\setlength{\itemsep}{0pt}
            \item Both handle multiple seasonality
            \item Holiday effects crucial $\Rightarrow$ Prophet; automatic selection $\Rightarrow$ TBATS
            \item Often try both and compare via cross-validation
        \end{itemize}
    \end{exampleblock}

    \end{cminipage}
    \quantlet{TSA\_ch9\_electricity\_demand}{https://github.com/QuantLet/TSA/tree/main/TSA_ch9/TSA_ch9_electricity_demand}
\end{frame}

%=============================================================================
% TRUE/FALSE
%=============================================================================
\section{True/False}

\begin{frame}{True or False? --- Questions}
    \begin{cminipage}{0.95\textwidth}
    \footnotesize
    \begin{center}
    \begin{tabular}{p{9cm}c}
        \toprule
        \textbf{Statement} & \textbf{T/F?} \\
        \midrule
        1. Prophet was developed by Facebook (Meta) for business forecasting. & ? \\[0.15cm]
        2. TBATS can only handle two seasonal periods at most. & ? \\[0.15cm]
        3. In Prophet, the default trend is logistic growth. & ? \\[0.15cm]
        4. Fourier terms approximate seasonality using sine and cosine functions. & ? \\[0.15cm]
        5. Prophet requires equally spaced time series data. & ? \\[0.15cm]
        6. The Box-Cox transformation in TBATS stabilizes variance. & ? \\
        \bottomrule
    \end{tabular}
    \end{center}
    \end{cminipage}
\end{frame}

\begin{frame}{True or False? --- Answers}
    \begin{cminipage}{0.95\textwidth}
    \scriptsize
    \begin{center}
    \begin{tabular}{p{7.5cm}cc}
        \toprule
        \textbf{Statement} & \textbf{T/F} & \textbf{Explanation} \\
        \midrule
        1. Prophet was developed by Facebook (Meta) for business forecasting. & \textcolor{Forest}{\textbf{T}} & {\tiny Released 2017, analyst-in-the-loop} \\[0.08cm]
        2. TBATS can only handle two seasonal periods at most. & \textcolor{Crimson}{\textbf{F}} & {\tiny Any number of periods} \\[0.08cm]
        3. In Prophet, the default trend is logistic growth. & \textcolor{Crimson}{\textbf{F}} & {\tiny Default is piecewise linear} \\[0.08cm]
        4. Fourier terms approximate seasonality using sine and cosine functions. & \textcolor{Forest}{\textbf{T}} & {\tiny $\sum[a_k\cos + b_k\sin]$} \\[0.08cm]
        5. Prophet requires equally spaced time series data. & \textcolor{Crimson}{\textbf{F}} & {\tiny Handles missing/irregular} \\[0.08cm]
        6. The Box-Cox transformation in TBATS stabilizes variance. & \textcolor{Forest}{\textbf{T}} & {\tiny $y^{(\lambda)}$ transformation} \\
        \bottomrule
    \end{tabular}
    \end{center}
    \end{cminipage}
\end{frame}

%=============================================================================
% CALCULATION EXERCISES
%=============================================================================
\section{Calculation Exercises}

\begin{frame}{Exercise 1: Fourier Terms Calculation}
    \begin{cminipage}{0.95\textwidth}
    \begin{alertblock}{Problem}
        \begin{itemize}\setlength{\itemsep}{0pt}
            \item \textbf{Data}: Daily data with weekly seasonality ($m=7$), using $K=3$ Fourier harmonics
            \item \textbf{Calculate}: How many parameters does this add to the model?
            \item \textbf{Formula}: $s(t) = \sum_{k=1}^{K} \left[ a_k \cos\left(\frac{2\pi k t}{m}\right) + b_k \sin\left(\frac{2\pi k t}{m}\right) \right]$
        \end{itemize}
    \end{alertblock}

    \vspace{0.2cm}
    \begin{exampleblock}{Solution}
        \begin{itemize}\setlength{\itemsep}{0pt}
            \item Each harmonic requires 2 parameters (sine and cosine coefficients)
            \item $k=1$: $a_1, b_1$ (fundamental frequency $1/7$ cycles per day)
            \item $k=2$: $a_2, b_2$ (first overtone, $2/7$ cycles per day)
            \item $k=3$: $a_3, b_3$ (second overtone, $3/7$ cycles per day)
            \item \textbf{Total}: $2 \times K = 2 \times 3 = \textbf{6}$ parameters
        \end{itemize}
    \end{exampleblock}
    \end{cminipage}
    \quantlet{TSA\_ch9\_fourier\_approximation}{https://github.com/QuantLet/TSA/tree/main/TSA_ch9/TSA_ch9_fourier_approximation}
\end{frame}

\begin{frame}{Exercise 2: Choosing Seasonality Mode}
    \begin{cminipage}{0.95\textwidth}
    \begin{alertblock}{Problem}
        \begin{itemize}\setlength{\itemsep}{0pt}
            \item \textbf{Data}: Monthly hotel bookings --- July 2020: 1000, Jan 2020: 400, July 2023: 2000, Jan 2023: 800
            \item \textbf{Question}: Should you use additive or multiplicative seasonality? Why?
        \end{itemize}
    \end{alertblock}

    \vspace{0.2cm}
    \begin{exampleblock}{Solution}
        \begin{center}
        \small
        \begin{tabular}{lccc}
            \toprule
            Year & July & January & Ratio (Jul/Jan) \\
            \midrule
            2020 & 1000 & 400 & 2.5 \\
            2023 & 2000 & 800 & 2.5 \\
            \bottomrule
        \end{tabular}
        \end{center}
        \begin{itemize}\setlength{\itemsep}{0pt}
            \item The \textit{ratio} stays constant (2.5), not the difference!
            \item Additive: $1000 - 400 = 600$ vs $2000 - 800 = 1200$ (not constant)
            \item \textbf{Conclusion}: Use multiplicative: \texttt{seasonality\_mode='multiplicative'}
        \end{itemize}
    \end{exampleblock}
    \end{cminipage}
    \quantlet{TSA\_ch9\_additive\_vs\_multiplicative}{https://github.com/QuantLet/TSA/tree/main/TSA_ch9/TSA_ch9_additive_vs_multiplicative}
\end{frame}

\begin{frame}{Exercise 3: TBATS Model Interpretation}
    \begin{cminipage}{0.95\textwidth}
    \begin{alertblock}{Problem}
        \begin{itemize}\setlength{\itemsep}{0pt}
            \item \textbf{TBATS model}: Box-Cox $\lambda = 0.5$, seasonal periods $m_1 = 24$, $m_2 = 168$, Fourier terms $K_1 = 5$, $K_2 = 3$
            \item \textbf{Calculate}: What does each component tell you? Total seasonal parameters?
        \end{itemize}
    \end{alertblock}

    \vspace{0.2cm}
    \begin{exampleblock}{Solution}
        \begin{itemize}\setlength{\itemsep}{0pt}
            \item \textbf{Box-Cox $\lambda = 0.5$}: Square root transformation ($y^{(0.5)} = \sqrt{y}$), data had increasing variance
            \item \textbf{$m_1 = 24$}: Daily pattern (24 hours); \textbf{$m_2 = 168$}: Weekly pattern ($7 \times 24$)
            \item \textbf{$K_1 = 5$}: Complex intraday pattern (5 harmonics for peaks, valleys)
            \item \textbf{$K_2 = 3$}: Simpler weekly pattern (weekday vs weekend)
            \item \textbf{Total seasonal parameters}: $2(K_1 + K_2) = 2(5+3) = \textbf{16}$
        \end{itemize}
    \end{exampleblock}
    \end{cminipage}
    \quantlet{TSA\_ch9\_tbats\_decomposition}{https://github.com/QuantLet/TSA/tree/main/TSA_ch9/TSA_ch9_tbats_decomposition}
\end{frame}

\begin{frame}{Exercise 4: Prophet Holiday Effects}
    \begin{cminipage}{0.95\textwidth}
    {\small
    \begin{alertblock}{Problem}
        \begin{itemize}\setlength{\itemsep}{0pt}
            \item \textbf{Task}: Forecast daily restaurant revenue with holiday effects
            \item \textbf{Holidays}: Valentine's Day (Feb 14, boost), Easter (variable, closed), Christmas (Dec 25, closed)
            \item \textbf{Write}: Python code to create the holidays dataframe for 2024--2025
        \end{itemize}
    \end{alertblock}

    \vspace{0.1cm}
    \begin{exampleblock}{Solution}
        {\scriptsize\ttfamily
        holidays = pd.DataFrame(\{\\
        \quad 'holiday': ['valentines', 'valentines', 'easter', 'easter',\\
        \quad\quad\quad\quad\quad\quad 'christmas', 'christmas'],\\
        \quad 'ds': pd.to\_datetime(['2024-02-14', '2025-02-14',\\
        \quad\quad\quad '2024-03-31', '2025-04-20', '2024-12-25', '2025-12-25']),\\
        \quad 'lower\_window': [-1, -1, 0, 0, -1, -1],\\
        \quad 'upper\_window': [0, 0, 0, 0, 0, 0]\})\\
        model = Prophet(holidays=holidays)
        }
    \end{exampleblock}
    }
    \end{cminipage}
    \quantlet{TSA\_ch9\_prophet\_components}{https://github.com/QuantLet/TSA/tree/main/TSA_ch9/TSA_ch9_prophet_components}
\end{frame}

%=============================================================================
% WORKED EXAMPLES
%=============================================================================
\section{Worked Examples}

\begin{frame}{Visual: Retail Sales Forecasting with Prophet}
    \begin{cminipage}{0.95\textwidth}
    {\footnotesize
    \begin{block}{Scenario}
        Monthly retail sales (2018-2023): December peaks, COVID-19 break in 2020, growing trend.
    \end{block}
    \begin{exampleblock}{Prophet Configuration}
        {\scriptsize\ttfamily
        model = Prophet(seasonality\_mode='multiplicative',\\
        \quad changepoint\_prior\_scale=0.5, yearly\_seasonality=True)\\
        model.add\_country\_holidays(country\_name='US')
        }
    \end{exampleblock}
    \begin{alertblock}{Key Decision}
        Multiplicative seasonality: December effect proportional to baseline level.
    \end{alertblock}
    }

    \end{cminipage}
    \quantlet{TSA\_ch9\_retail\_sales}{https://github.com/QuantLet/TSA/tree/main/TSA_ch9/TSA_ch9_retail_sales}
\end{frame}

\begin{frame}{Visual: Energy Demand with TBATS}
    \begin{cminipage}{0.95\textwidth}
    {\footnotesize
    \begin{block}{Scenario}
        Hourly electricity: intraday (24h), weekly (168h), annual (8760h) patterns.
    \end{block}
    \begin{exampleblock}{TBATS in R}
        {\scriptsize\ttfamily
        library(forecast)\\
        energy\_msts <- msts(energy\_data, seasonal.periods = c(24, 168, 8760))\\
        fit <- tbats(energy\_msts); fc <- forecast(fit, h = 168)
        }
    \end{exampleblock}
    \begin{alertblock}{Note}
        TBATS automatically selects $K$ for each seasonal period via AIC.
    \end{alertblock}
    }

    \end{cminipage}
    \quantlet{TSA\_ch9\_electricity\_demand}{https://github.com/QuantLet/TSA/tree/main/TSA_ch9/TSA_ch9_electricity_demand}
\end{frame}

\begin{frame}{Visual: Cross-Validation Comparison}
    \begin{cminipage}{0.95\textwidth}
    \vspace{-0.3cm}
    {\footnotesize
    \begin{block}{Objective}
        Compare Prophet, TBATS, and SARIMA on 2 years of daily sales data.
    \end{block}
    \begin{exampleblock}{Prophet Cross-Validation}
        {\scriptsize\ttfamily
        from prophet.diagnostics import cross\_validation, performance\_metrics\\[0.3em]
        df\_cv = cross\_validation(model, initial='365 days',\\
        \quad period='90 days', horizon='30 days')\\
        metrics = performance\_metrics(df\_cv)\\
        print(f"MAPE: \{metrics['mape'].mean():.2\%\}")
        }
    \end{exampleblock}
    \begin{block}{Typical Results}
        \begin{center}
        \begin{tabular}{lcc}
            \toprule
            Model & MAPE & Computation Time \\
            \midrule
            SARIMA (weekly only) & 8.5\% & Fast \\
            TBATS (weekly + yearly) & 6.2\% & Moderate \\
            Prophet (weekly + yearly + holidays) & 5.8\% & Fast \\
            \bottomrule
        \end{tabular}
        \end{center}
    \end{block}
    }
    \end{cminipage}
    \quantlet{TSA\_ch9\_prophet\_vs\_tbats}{https://github.com/QuantLet/TSA/tree/main/TSA_ch9/TSA_ch9_prophet_vs_tbats}
\end{frame}

%=============================================================================
% DISCUSSION
%=============================================================================
\section{Discussion Topics}

\begin{frame}{Discussion: When to Use Which Model?}
    \begin{cminipage}{0.95\textwidth}
    {\small
    \begin{alertblock}{Key Question}
        You have a new forecasting task. How do you choose between SARIMA, TBATS, and Prophet?
    \end{alertblock}
    \begin{block}{Decision Framework}
        \begin{enumerate}\setlength{\itemsep}{0pt}
            \item \textbf{How many seasonal periods?}
                \begin{itemize}\setlength{\itemsep}{0pt}
                    \item One $\Rightarrow$ SARIMA may suffice
                    \item Multiple $\Rightarrow$ TBATS or Prophet
                \end{itemize}
            \item \textbf{Do you have domain knowledge to encode?}
                \begin{itemize}\setlength{\itemsep}{0pt}
                    \item Holidays, events, changepoints $\Rightarrow$ Prophet
                    \item Let the data speak $\Rightarrow$ TBATS
                \end{itemize}
            \item \textbf{Interpretability requirements?}
                \begin{itemize}\setlength{\itemsep}{0pt}
                    \item Need to explain components $\Rightarrow$ Prophet
                    \item Just need forecasts $\Rightarrow$ Either
                \end{itemize}
        \end{enumerate}
    \end{block}
    }
    \end{cminipage}
    \quantlet{TSA\_ch9\_model\_selection\_guide}{https://github.com/QuantLet/TSA/tree/main/TSA_ch9/TSA_ch9_model_selection_guide}
\end{frame}

\begin{frame}{Discussion: Overfitting with Fourier Terms}
    \begin{cminipage}{0.95\textwidth}
    \vspace{-0.3cm}
    {\scriptsize
    \begin{alertblock}{Key Question}
        Can you have too many Fourier terms? What are the symptoms?
    \end{alertblock}
    \begin{block}{Answer: Yes!}
        \textbf{Symptoms of overfitting}:
        \begin{itemize}\setlength{\itemsep}{0pt}
            \item In-sample fit is excellent, but out-of-sample is poor
            \item Seasonality looks ``jagged'' or unrealistic
            \item Forecasts oscillate wildly
        \end{itemize}
    \end{block}
    \begin{exampleblock}{Guidelines}
        \begin{itemize}\setlength{\itemsep}{0pt}
            \item Maximum $K \leq m/2$ (Nyquist limit)
            \item Start with $K = 3$--$5$ for most applications
            \item Use cross-validation to select $K$
            \item Prophet default: $K=10$ for yearly, $K=3$ for weekly
        \end{itemize}
    \end{exampleblock}
    }
    \end{cminipage}
    \quantlet{TSA\_ch9\_fourier\_approximation}{https://github.com/QuantLet/TSA/tree/main/TSA_ch9/TSA_ch9_fourier_approximation}
\end{frame}

\begin{frame}{Discussion: Handling Structural Breaks}
    \begin{cminipage}{0.95\textwidth}
    {\small
    \begin{alertblock}{Scenario}
        Your historical data includes COVID-19 period (2020-2021). How do you handle this when forecasting 2024?
    \end{alertblock}
    \begin{block}{Options}
        \begin{enumerate}\setlength{\itemsep}{0pt}
            \item \textbf{Exclude COVID period}: Train only on pre-COVID and post-COVID data
            \item \textbf{Use changepoints}: Let Prophet detect/specify breaks
            \item \textbf{Add regressors}: Include COVID indicator variable
            \item \textbf{Adjustment}: Manually adjust 2020-2021 values to ``normal''
        \end{enumerate}
    \end{block}
    \begin{exampleblock}{Prophet Approach}
        {\scriptsize\ttfamily
        model = Prophet(changepoints=['2020-03-15', '2021-06-01'])\\
        df['covid'] = (df['ds'] >= '2020-03-15') \& (df['ds'] < '2021-06-01')\\
        model.add\_regressor('covid')
        }
    \end{exampleblock}
    }
    \end{cminipage}
    \quantlet{TSA\_ch9\_changepoint\_detection}{https://github.com/QuantLet/TSA/tree/main/TSA_ch9/TSA_ch9_changepoint_detection}
\end{frame}

%=============================================================================
% AI-ASSISTED EXERCISE
%=============================================================================
\section{AI-Assisted Exercise}

\begin{frame}{AI Exercise: Critical Thinking}
    \begin{cminipage}{0.95\textwidth}
    \vspace{-0.3cm}
    \begin{block}{\footnotesize Prompt to test in ChatGPT / Claude / Copilot}
        {\footnotesize
        ``Use Prophet to forecast daily Wikipedia page views for the 'Bitcoin' article. Include holiday effects, identify changepoints, and compare with a TBATS model. Evaluate using time series cross-validation.''
        }
    \end{block}
    \vspace{-2mm}
    {\footnotesize
    \textbf{Exercise}:
    \begin{enumerate}\setlength{\itemsep}{0pt}
        \item Did the AI correctly specify the seasonality periods (weekly, yearly)?
        \item Are the changepoints reasonable? Do they correspond to real events?
        \item Is the cross-validation properly implemented (expanding window)?
        \item How does Prophet handle the extreme spikes in Bitcoin-related traffic?
        \item Does the TBATS comparison use the same evaluation metrics?
    \end{enumerate}
    }
    \vspace{-2mm}
    \begin{alertblock}{}
        {\footnotesize \textbf{Warning}: AI-generated code may run without errors and look professional. \textit{That does not mean it is correct.}}
    \end{alertblock}
    \end{cminipage}
    \quantlet{TSA\_ch9\_prophet\_vs\_tbats}{https://github.com/QuantLet/TSA/tree/main/TSA_ch9/TSA_ch9_prophet_vs_tbats}
\end{frame}

%=============================================================================
% TAKE-HOME EXERCISES
%=============================================================================

\begin{frame}{Take-Home Exercises}
    \begin{cminipage}{0.95\textwidth}
    {\footnotesize
    \begin{enumerate}\setlength{\itemsep}{2pt}
        \item \textbf{Theoretical}: Prove that $K = m/2$ Fourier terms can represent any periodic function with period $m$ (for even $m$).

        \item \textbf{Computation}: For the seasonal pattern below (daily data, weekly cycle), determine the minimum number of Fourier harmonics needed:
            \begin{center}
            Mon: 100, Tue: 110, Wed: 115, Thu: 110, Fri: 120, Sat: 80, Sun: 65
            \end{center}

        \item \textbf{Applied}: Download hourly electricity demand data from a public source:
            \begin{itemize}\setlength{\itemsep}{0pt}
                \item Fit both TBATS (in R) and Prophet (in Python)
                \item Compare forecast accuracy using RMSE and MAPE
                \item Visualize the component decompositions
            \end{itemize}

        \item \textbf{Critical Thinking}: Why might Prophet perform poorly on high-frequency financial data (e.g., minute-by-minute stock prices)?
    \end{enumerate}
    }
    \end{cminipage}
    \quantlet{TSA\_ch9\_tbats\_decomposition}{https://github.com/QuantLet/TSA/tree/main/TSA_ch9/TSA_ch9_tbats_decomposition}
\end{frame}

\begin{frame}{Exercise Solutions Hints}
    \begin{cminipage}{0.95\textwidth}
    \vspace{-0.3cm}
    {\footnotesize
    \begin{block}{Hints}
        \begin{enumerate}\setlength{\itemsep}{1pt}
            \item By Fourier theorem, any periodic function can be represented as sum of sines and cosines. With period $m$, frequencies are $k/m$ for $k = 1, \ldots, m/2$.

            \item The pattern has:
                \begin{itemize}\setlength{\itemsep}{0pt}
                    \item One peak (Friday) and one trough (Sunday)
                    \item Fairly smooth transitions
                    \item $K=2$ or $K=3$ likely sufficient (try and compare)
                \end{itemize}

            \item For electricity data:
                \begin{itemize}\setlength{\itemsep}{0pt}
                    \item Include daily (24h) and weekly (168h) patterns
                    \item Add holidays for your region in Prophet
                    \item Expect MAPE around 3-5\% for hourly forecasts
                \end{itemize}

            \item Financial data issues:
                \begin{itemize}\setlength{\itemsep}{0pt}
                    \item No clear seasonality (market efficiency)
                    \item High noise-to-signal ratio
                    \item Prophet designed for ``business'' data with trends and seasons
                \end{itemize}
        \end{enumerate}
    \end{block}
    }
    \end{cminipage}
    \quantlet{TSA\_ch9\_fourier\_approximation}{https://github.com/QuantLet/TSA/tree/main/TSA_ch9/TSA_ch9_fourier_approximation}
\end{frame}

%=============================================================================
% END
%=============================================================================
\section{Summary}

\begin{frame}{Summary: Chapter 9}
    \begin{cminipage}{0.95\textwidth}
    \begin{exampleblock}{Key Concepts}
        \begin{itemize}\setlength{\itemsep}{2pt}
            \item[\textcolor{MainBlue}{\textbf{1.}}] \textbf{TBATS}: Automatic, Fourier-based, handles any number of seasonal periods
            \item[\textcolor{MainBlue}{\textbf{2.}}] \textbf{Prophet}: Analyst-friendly, explicit holiday/event handling, interpretable
            \item[\textcolor{MainBlue}{\textbf{3.}}] \textbf{Seasonality mode}: Additive (constant amplitude) vs Multiplicative (proportional)
            \item[\textcolor{MainBlue}{\textbf{4.}}] \textbf{Fourier terms}: More = flexible but risk overfitting; use CV to select
            \item[\textcolor{MainBlue}{\textbf{5.}}] \textbf{Changepoints}: Allow trend to adapt to structural breaks
        \end{itemize}
    \end{exampleblock}

    \vspace{0.5cm}
    \begin{center}
        \Large\textcolor{MainBlue}{Questions?}
    \end{center}
    \end{cminipage}
\end{frame}


%=============================================================================
% BIBLIOGRAPHY (same references as in the course)
%=============================================================================
\section{Bibliography}

\begin{frame}{Bibliography I}
    \begin{cminipage}{0.95\textwidth}
    \begin{block}{Time Series Fundamentals}
        {\small
        \begin{itemize}\setlength{\itemsep}{0pt}
            \item Hyndman, R.J., \& Athanasopoulos, G. (2021). \textit{Forecasting: Principles and Practice}, 3rd ed., OTexts.
            \item Shumway, R.H., \& Stoffer, D.S. (2017). \textit{Time Series Analysis and Its Applications}, 4th ed., Springer.
            \item Brockwell, P.J., \& Davis, R.A. (2016). \textit{Introduction to Time Series and Forecasting}, 3rd ed., Springer.
        \end{itemize}
        }
    \end{block}

    \begin{exampleblock}{Financial Time Series}
        {\small
        \begin{itemize}\setlength{\itemsep}{0pt}
            \item Tsay, R.S. (2010). \textit{Analysis of Financial Time Series}, 3rd ed., Wiley.
            \item Franke, J., H\"ardle, W.K., \& Hafner, C.M. (2019). \textit{Statistics of Financial Markets}, 4th ed., Springer.
        \end{itemize}
        }
    \end{exampleblock}
    \end{cminipage}
\end{frame}

\begin{frame}{Bibliography II}
    \begin{cminipage}{0.95\textwidth}
    \begin{block}{Modern Approaches and Machine Learning}
        {\small
        \begin{itemize}\setlength{\itemsep}{0pt}
            \item Nielsen, A. (2019). \textit{Practical Time Series Analysis}, O'Reilly Media.
            \item Petropoulos, F., et al. (2022). \textit{Forecasting: Theory and Practice}, International Journal of Forecasting.
            \item Makridakis, S., Spiliotis, E., \& Assimakopoulos, V. (2020). The M4 Competition, International Journal of Forecasting.
        \end{itemize}
        }
    \end{block}

    \begin{exampleblock}{Online Resources and Code}
        {\small
        \begin{itemize}\setlength{\itemsep}{0pt}
            \item \textbf{Quantlet}: \url{https://quantlet.com} --- Code repository for statistics
            \item \textbf{Quantinar}: \url{https://quantinar.com} --- Quantitative methods learning platform
            \item \textbf{GitHub TSA}: \url{https://github.com/QuantLet/TSA/tree/main/TSA_ch9} --- Python code for this seminar
        \end{itemize}
        }
    \end{exampleblock}
    \end{cminipage}
\end{frame}

\begin{frame}{}
    \begin{cminipage}{0.95\textwidth}
    \centering
    \Huge\textcolor{IDAred}{Thank You!}

    \vspace{1cm}

    \Large\textcolor{MainBlue}{Questions?}

    \vspace{0.8cm}

    \normalsize
    Seminar materials are available at: \url{https://danpele.github.io/Time-Series-Analysis/}

    \vspace{0.2cm}

    \href{https://quantlet.com}{\raisebox{-0.15em}{\includegraphics[height=0.8em]{ql_logo.png}} Quantlet} \hspace{0.5cm}
    \href{https://quantinar.com}{\raisebox{-0.15em}{\includegraphics[height=0.8em]{qr_logo.png}} Quantinar}
    \end{cminipage}
\end{frame}

\end{document}
