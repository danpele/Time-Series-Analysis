% Seminar 4: SARIMA Models

\documentclass[9pt, aspectratio=169, t]{beamer}
%=============================================================================
% SHARED PREAMBLE - Time Series Analysis and Forecasting
% Harvard-quality academic presentations
% Bachelor program, Bucharest University of Economic Studies
%
% Usage: \documentclass[9pt, aspectratio=169, t]{beamer}
%            %=============================================================================
% SHARED PREAMBLE - Time Series Analysis and Forecasting
% Harvard-quality academic presentations
% Bachelor program, Bucharest University of Economic Studies
%
% Usage: \documentclass[9pt, aspectratio=169, t]{beamer}
%            %=============================================================================
% SHARED PREAMBLE - Time Series Analysis and Forecasting
% Harvard-quality academic presentations
% Bachelor program, Bucharest University of Economic Studies
%
% Usage: \documentclass[9pt, aspectratio=169, t]{beamer}
%            \input{preamble}
%            \subtitle{Seminar X: Seminar Title}
%            \begin{document} ...
%=============================================================================

% Ensure content fits on slides
\setbeamersize{text margin left=8mm, text margin right=8mm}

%=============================================================================
% THEME AND STYLE CONFIGURATION
%=============================================================================
\usetheme{default}
% Using default theme for clean header/footer control

% Color Palette (matching Redispatch PDF)
\definecolor{MainBlue}{RGB}{26, 58, 110}
\definecolor{AccentBlue}{RGB}{26, 58, 110}
\definecolor{IDAred}{RGB}{205, 0, 0}
\definecolor{DarkGray}{RGB}{51, 51, 51}
\definecolor{MediumGray}{RGB}{128, 128, 128}
\definecolor{LightGray}{RGB}{248, 248, 248}
\definecolor{VeryLightGray}{RGB}{235, 235, 235}
\definecolor{KeynoteGray}{RGB}{218, 218, 218}
\definecolor{SectionGray}{RGB}{120, 120, 120}
\definecolor{FooterGray}{RGB}{100, 100, 100}
\definecolor{Crimson}{RGB}{220, 53, 69}
\definecolor{Forest}{RGB}{46, 125, 50}
\definecolor{Amber}{RGB}{181, 133, 63}
\definecolor{Orange}{RGB}{230, 126, 34}
\definecolor{Purple}{RGB}{142, 68, 173}

% Gradient background (exact Keynote 315° gradient: white to RGB 218,218,218)
\setbeamertemplate{background}{%
    \begin{tikzpicture}[remember picture, overlay]
        \shade[shading=axis, shading angle=315,
        top color=white, bottom color=KeynoteGray]
        (current page.south west) rectangle (current page.north east);
    \end{tikzpicture}%
}
% Fallback solid color for compatibility
\setbeamercolor{background canvas}{bg=}

\setbeamercolor{palette primary}{bg=MainBlue, fg=white}
\setbeamercolor{palette secondary}{bg=MainBlue!85, fg=white}
\setbeamercolor{palette tertiary}{bg=MainBlue!70, fg=white}
\setbeamercolor{structure}{fg=MainBlue}
\setbeamercolor{title}{fg=IDAred}
\setbeamercolor{frametitle}{fg=IDAred, bg=}
\setbeamercolor{block title}{bg=MainBlue, fg=white}
\setbeamercolor{block body}{bg=VeryLightGray, fg=DarkGray}
\setbeamercolor{block title alerted}{bg=Crimson, fg=white}
\setbeamercolor{block body alerted}{bg=Crimson!8, fg=DarkGray}
\setbeamercolor{block title example}{bg=Forest, fg=white}
\setbeamercolor{block body example}{bg=Forest!8, fg=DarkGray}
\setbeamercolor{item}{fg=MainBlue}

% Smaller institute font to avoid overfull hbox on title page
\setbeamerfont{institute}{size=\footnotesize}

% Footer colors (override Madrid theme blue)
\setbeamercolor{author in head/foot}{fg=FooterGray, bg=}
\setbeamercolor{title in head/foot}{fg=FooterGray, bg=}
\setbeamercolor{date in head/foot}{fg=FooterGray, bg=}
\setbeamercolor{section in head/foot}{fg=FooterGray, bg=}
\setbeamercolor{subsection in head/foot}{fg=FooterGray, bg=}

% Bullet styles (apply everywhere including blocks)
\setbeamertemplate{itemize item}{\color{MainBlue}$\boxdot$}
\setbeamertemplate{itemize subitem}{\color{MainBlue}$\blacktriangleright$}
\setbeamertemplate{itemize subsubitem}{\color{MainBlue}\tiny$\bullet$}
\setbeamertemplate{itemize/enumerate body begin}{\normalsize}
\setbeamertemplate{itemize/enumerate subbody begin}{\normalsize}

% Item spacing - compact style
\setlength{\leftmargini}{10pt}       % Level 1: minimal indent
\setlength{\leftmarginii}{10pt}      % Level 2: minimal additional indent
% Compact list spacing (zero extra space before/after lists in blocks)
\makeatletter
\def\@listi{\leftmargin\leftmargini \topsep 0pt \parsep 0pt \itemsep 0pt}
\def\@listii{\leftmargin\leftmarginii \topsep 0pt \parsep 0pt \itemsep 0pt}
\makeatother

\setbeamertemplate{navigation symbols}{}

%=============================================================================
% CUSTOM HEADLINE
%=============================================================================
\setbeamertemplate{headline}{%
    \vskip10pt%
    \hbox to \paperwidth{%
        \hskip0.5cm%
        {\small\color{FooterGray}\renewcommand{\hyperlink}[2]{##2}\insertsectionhead}%
        \hfill%
        \textcolor{FooterGray}{\small\insertframenumber}%
        \hskip0.5cm%
    }%
    \vskip4pt%
    {\color{FooterGray}\hrule height 0.4pt}%
}

%=============================================================================
% CUSTOM FOOTER
%=============================================================================
\usepackage{fontawesome5}

\setbeamertemplate{footline}{%
    {\color{FooterGray}\hrule height 0.4pt}%
    \vskip4pt%
    \hbox to \paperwidth{%
        \hskip0.5cm%
        \textcolor{FooterGray}{\small Time Series Analysis and Forecasting}%
        \hfill%
        \raisebox{-0.1em}{%
            \begin{tikzpicture}[x=0.08em, y=0.08em, line width=0.4pt]
                \draw[FooterGray] (0,3) -- (1,4) -- (2,3.5) -- (3,5) -- (4,4) -- (5,6) -- (6,5.5) -- (7,4) -- (8,5) -- (9,7) -- (10,6) -- (11,5) -- (12,6.5) -- (13,8) -- (14,7) -- (15,6) -- (16,7.5) -- (17,9) -- (18,8) -- (19,7) -- (20,8.5) -- (21,10) -- (22,9) -- (23,8) -- (24,9.5);
            \end{tikzpicture}%
        }%
        \hskip0.5cm%
    }%
    \vskip6pt%
}

%=============================================================================
% PACKAGES
%=============================================================================
\usepackage[utf8]{inputenc}
\usepackage[T1]{fontenc}
\usepackage[english]{babel}
\usepackage{amsmath, amssymb, amsthm}
\usepackage{mathtools}
\usepackage{bm}
\usepackage{tikz}
\usetikzlibrary{arrows.meta, positioning, shapes, calc, decorations.pathreplacing, shadings}
\usepackage{booktabs}
\usepackage{multirow}
\usepackage{array}
\usepackage{graphicx}
\usepackage{hyperref}
\usepackage{colortbl}
\usepackage{listings}
\lstset{basicstyle=\ttfamily\small, breaklines=true, frame=single, backgroundcolor=\color{VeryLightGray}}
\hypersetup{colorlinks=true, linkcolor=MainBlue, urlcolor=MainBlue}
\graphicspath{{../../logos/}{../../charts/}{../../photos/}}
\hfuzz=2pt  % Suppress tiny overfull warnings (<2pt)
\vfuzz=2pt  % Suppress tiny vertical overfull warnings (<2pt)

%=============================================================================
% QUANTLET COMMAND
%=============================================================================
\newcommand{\quantlet}[2]{%
    \hfill\href{#2}{%
        \raisebox{-0.15em}{\includegraphics[height=0.7em]{ql_logo.png}}%
        \textcolor{MainBlue}{\tiny\ #1}%
    }%
}

%=============================================================================
% CUSTOM TITLE PAGE
%=============================================================================
\defbeamertemplate*{title page}{hybrid}[1][]
{
    \vspace{0.2cm}
    % Logos row - top header (with clickable links)
    \begin{center}
        \href{https://www.ase.ro}{\includegraphics[height=1.0cm]{ase_logo.png}}\hspace{0.25cm}%
        \href{https://theida.net}{\includegraphics[height=1.0cm]{ida_logo.png}}\hspace{0.25cm}%
        \href{https://blockchain-research-center.com}{\includegraphics[height=1.0cm]{brc_logo.png}}\hspace{0.25cm}%
        \href{https://www.ai4efin.ase.ro}{\includegraphics[height=1.0cm]{ai4efin_logo.png}}\hspace{0.25cm}%
        \href{https://ipe.ro/new}{\includegraphics[height=1.0cm]{acad_logo.png}}\hspace{0.25cm}%
        \href{https://www.digital-finance-msca.com}{\includegraphics[height=1.0cm]{msca_logo.png}}%
    \end{center}

    \vspace{0.6cm}

    % Main title with Q logos on sides (with clickable links)
    \begin{center}
        \begin{minipage}{0.1\textwidth}
            \centering
            \href{https://quantlet.com}{\includegraphics[height=1.1cm]{ql_logo.png}}
        \end{minipage}%
        \begin{minipage}{0.78\textwidth}
            \centering
            {\LARGE\bfseries\usebeamercolor[fg]{title}\inserttitle}

            \vspace{0.3cm}

            {\usebeamerfont{subtitle}\usebeamercolor[fg]{title}\insertsubtitle}
        \end{minipage}%
        \begin{minipage}{0.1\textwidth}
            \centering
            \href{https://quantinar.com}{\includegraphics[height=1.1cm]{qr_logo.png}}
        \end{minipage}
    \end{center}

    \vspace{0.6cm}

    % Authors (left aligned)
    \hspace{0.5cm}{\usebeamerfont{author}\insertauthor}

    \vspace{0.3cm}

    % Institute/Affiliations (left aligned)
    \hspace{0.5cm}\begin{minipage}[t]{0.9\textwidth}
        \raggedright\small\insertinstitute
    \end{minipage}
}

%=============================================================================
% THEOREM ENVIRONMENTS
%=============================================================================
\theoremstyle{definition}
\setbeamertemplate{theorems}[numbered]
\newtheorem{defn}{Definition}
\newtheorem{thm}{Theorem}
\newtheorem{prop}{Proposition}
\newtheorem{rmk}{Remark}

%=============================================================================
% CENTRED MINIPAGE (no extra vertical space)
%=============================================================================
\newenvironment{cminipage}[1]{%
    \par\noindent\hfill\begin{minipage}{#1}\ignorespaces
}{%
    \end{minipage}\hfill\null\par
}

%=============================================================================
% CUSTOM COMMANDS
%=============================================================================
\newcommand{\E}{\mathbb{E}}
\newcommand{\Var}{\text{Var}}
\newcommand{\Cov}{\text{Cov}}
\newcommand{\Corr}{\text{Corr}}
\newcommand{\R}{\mathbb{R}}
\newcommand{\N}{\mathbb{N}}
\newcommand{\Z}{\mathbb{Z}}
\newcommand{\B}{\mathbf{B}}
\newcommand{\imark}{\textcolor{MainBlue}{\textbullet}}
\newcommand{\RMSE}{\text{RMSE}}
\newcommand{\MAE}{\text{MAE}}
\newcommand{\MAPE}{\text{MAPE}}
\newcommand{\correct}{\textcolor{Forest}{\checkmark}}
\newcommand{\incorrect}{\textcolor{Crimson}{\texttimes}}

% Boldface vector/matrix commands
\newcommand{\bY}{\mathbf{Y}}
\newcommand{\bX}{\mathbf{X}}
\newcommand{\bA}{\mathbf{A}}
\newcommand{\bB}{\mathbf{B}}
\newcommand{\bepsilon}{\boldsymbol{\varepsilon}}
\newcommand{\bvarepsilon}{\boldsymbol{\varepsilon}}
\newcommand{\bSigma}{\boldsymbol{\Sigma}}
\newcommand{\bPhi}{\boldsymbol{\Phi}}
\newcommand{\bGamma}{\boldsymbol{\Gamma}}
\newcommand{\bPi}{\boldsymbol{\Pi}}
\newcommand{\bc}{\mathbf{c}}
\newcommand{\balpha}{\boldsymbol{\alpha}}
\newcommand{\bbeta}{\boldsymbol{\beta}}

%=============================================================================
% TITLE INFORMATION
%=============================================================================
\title[Time Series Analysis]{Time Series Analysis and Forecasting}
\author[D.T. Pele]{Daniel Traian PELE}
\institute{Bucharest University of Economic Studies\\
IDA Institute Digital Assets\\
Blockchain Research Center\\
AI4EFin Artificial Intelligence for Energy Finance\\
Romanian Academy, Institute for Economic Forecasting\\
MSCA Digital Finance}
\date{}

%            \subtitle{Seminar X: Seminar Title}
%            \begin{document} ...
%=============================================================================

% Ensure content fits on slides
\setbeamersize{text margin left=8mm, text margin right=8mm}

%=============================================================================
% THEME AND STYLE CONFIGURATION
%=============================================================================
\usetheme{default}
% Using default theme for clean header/footer control

% Color Palette (matching Redispatch PDF)
\definecolor{MainBlue}{RGB}{26, 58, 110}
\definecolor{AccentBlue}{RGB}{26, 58, 110}
\definecolor{IDAred}{RGB}{205, 0, 0}
\definecolor{DarkGray}{RGB}{51, 51, 51}
\definecolor{MediumGray}{RGB}{128, 128, 128}
\definecolor{LightGray}{RGB}{248, 248, 248}
\definecolor{VeryLightGray}{RGB}{235, 235, 235}
\definecolor{KeynoteGray}{RGB}{218, 218, 218}
\definecolor{SectionGray}{RGB}{120, 120, 120}
\definecolor{FooterGray}{RGB}{100, 100, 100}
\definecolor{Crimson}{RGB}{220, 53, 69}
\definecolor{Forest}{RGB}{46, 125, 50}
\definecolor{Amber}{RGB}{181, 133, 63}
\definecolor{Orange}{RGB}{230, 126, 34}
\definecolor{Purple}{RGB}{142, 68, 173}

% Gradient background (exact Keynote 315° gradient: white to RGB 218,218,218)
\setbeamertemplate{background}{%
    \begin{tikzpicture}[remember picture, overlay]
        \shade[shading=axis, shading angle=315,
        top color=white, bottom color=KeynoteGray]
        (current page.south west) rectangle (current page.north east);
    \end{tikzpicture}%
}
% Fallback solid color for compatibility
\setbeamercolor{background canvas}{bg=}

\setbeamercolor{palette primary}{bg=MainBlue, fg=white}
\setbeamercolor{palette secondary}{bg=MainBlue!85, fg=white}
\setbeamercolor{palette tertiary}{bg=MainBlue!70, fg=white}
\setbeamercolor{structure}{fg=MainBlue}
\setbeamercolor{title}{fg=IDAred}
\setbeamercolor{frametitle}{fg=IDAred, bg=}
\setbeamercolor{block title}{bg=MainBlue, fg=white}
\setbeamercolor{block body}{bg=VeryLightGray, fg=DarkGray}
\setbeamercolor{block title alerted}{bg=Crimson, fg=white}
\setbeamercolor{block body alerted}{bg=Crimson!8, fg=DarkGray}
\setbeamercolor{block title example}{bg=Forest, fg=white}
\setbeamercolor{block body example}{bg=Forest!8, fg=DarkGray}
\setbeamercolor{item}{fg=MainBlue}

% Smaller institute font to avoid overfull hbox on title page
\setbeamerfont{institute}{size=\footnotesize}

% Footer colors (override Madrid theme blue)
\setbeamercolor{author in head/foot}{fg=FooterGray, bg=}
\setbeamercolor{title in head/foot}{fg=FooterGray, bg=}
\setbeamercolor{date in head/foot}{fg=FooterGray, bg=}
\setbeamercolor{section in head/foot}{fg=FooterGray, bg=}
\setbeamercolor{subsection in head/foot}{fg=FooterGray, bg=}

% Bullet styles (apply everywhere including blocks)
\setbeamertemplate{itemize item}{\color{MainBlue}$\boxdot$}
\setbeamertemplate{itemize subitem}{\color{MainBlue}$\blacktriangleright$}
\setbeamertemplate{itemize subsubitem}{\color{MainBlue}\tiny$\bullet$}
\setbeamertemplate{itemize/enumerate body begin}{\normalsize}
\setbeamertemplate{itemize/enumerate subbody begin}{\normalsize}

% Item spacing - compact style
\setlength{\leftmargini}{10pt}       % Level 1: minimal indent
\setlength{\leftmarginii}{10pt}      % Level 2: minimal additional indent
% Compact list spacing (zero extra space before/after lists in blocks)
\makeatletter
\def\@listi{\leftmargin\leftmargini \topsep 0pt \parsep 0pt \itemsep 0pt}
\def\@listii{\leftmargin\leftmarginii \topsep 0pt \parsep 0pt \itemsep 0pt}
\makeatother

\setbeamertemplate{navigation symbols}{}

%=============================================================================
% CUSTOM HEADLINE
%=============================================================================
\setbeamertemplate{headline}{%
    \vskip10pt%
    \hbox to \paperwidth{%
        \hskip0.5cm%
        {\small\color{FooterGray}\renewcommand{\hyperlink}[2]{##2}\insertsectionhead}%
        \hfill%
        \textcolor{FooterGray}{\small\insertframenumber}%
        \hskip0.5cm%
    }%
    \vskip4pt%
    {\color{FooterGray}\hrule height 0.4pt}%
}

%=============================================================================
% CUSTOM FOOTER
%=============================================================================
\usepackage{fontawesome5}

\setbeamertemplate{footline}{%
    {\color{FooterGray}\hrule height 0.4pt}%
    \vskip4pt%
    \hbox to \paperwidth{%
        \hskip0.5cm%
        \textcolor{FooterGray}{\small Time Series Analysis and Forecasting}%
        \hfill%
        \raisebox{-0.1em}{%
            \begin{tikzpicture}[x=0.08em, y=0.08em, line width=0.4pt]
                \draw[FooterGray] (0,3) -- (1,4) -- (2,3.5) -- (3,5) -- (4,4) -- (5,6) -- (6,5.5) -- (7,4) -- (8,5) -- (9,7) -- (10,6) -- (11,5) -- (12,6.5) -- (13,8) -- (14,7) -- (15,6) -- (16,7.5) -- (17,9) -- (18,8) -- (19,7) -- (20,8.5) -- (21,10) -- (22,9) -- (23,8) -- (24,9.5);
            \end{tikzpicture}%
        }%
        \hskip0.5cm%
    }%
    \vskip6pt%
}

%=============================================================================
% PACKAGES
%=============================================================================
\usepackage[utf8]{inputenc}
\usepackage[T1]{fontenc}
\usepackage[english]{babel}
\usepackage{amsmath, amssymb, amsthm}
\usepackage{mathtools}
\usepackage{bm}
\usepackage{tikz}
\usetikzlibrary{arrows.meta, positioning, shapes, calc, decorations.pathreplacing, shadings}
\usepackage{booktabs}
\usepackage{multirow}
\usepackage{array}
\usepackage{graphicx}
\usepackage{hyperref}
\usepackage{colortbl}
\usepackage{listings}
\lstset{basicstyle=\ttfamily\small, breaklines=true, frame=single, backgroundcolor=\color{VeryLightGray}}
\hypersetup{colorlinks=true, linkcolor=MainBlue, urlcolor=MainBlue}
\graphicspath{{../../logos/}{../../charts/}{../../photos/}}
\hfuzz=2pt  % Suppress tiny overfull warnings (<2pt)
\vfuzz=2pt  % Suppress tiny vertical overfull warnings (<2pt)

%=============================================================================
% QUANTLET COMMAND
%=============================================================================
\newcommand{\quantlet}[2]{%
    \hfill\href{#2}{%
        \raisebox{-0.15em}{\includegraphics[height=0.7em]{ql_logo.png}}%
        \textcolor{MainBlue}{\tiny\ #1}%
    }%
}

%=============================================================================
% CUSTOM TITLE PAGE
%=============================================================================
\defbeamertemplate*{title page}{hybrid}[1][]
{
    \vspace{0.2cm}
    % Logos row - top header (with clickable links)
    \begin{center}
        \href{https://www.ase.ro}{\includegraphics[height=1.0cm]{ase_logo.png}}\hspace{0.25cm}%
        \href{https://theida.net}{\includegraphics[height=1.0cm]{ida_logo.png}}\hspace{0.25cm}%
        \href{https://blockchain-research-center.com}{\includegraphics[height=1.0cm]{brc_logo.png}}\hspace{0.25cm}%
        \href{https://www.ai4efin.ase.ro}{\includegraphics[height=1.0cm]{ai4efin_logo.png}}\hspace{0.25cm}%
        \href{https://ipe.ro/new}{\includegraphics[height=1.0cm]{acad_logo.png}}\hspace{0.25cm}%
        \href{https://www.digital-finance-msca.com}{\includegraphics[height=1.0cm]{msca_logo.png}}%
    \end{center}

    \vspace{0.6cm}

    % Main title with Q logos on sides (with clickable links)
    \begin{center}
        \begin{minipage}{0.1\textwidth}
            \centering
            \href{https://quantlet.com}{\includegraphics[height=1.1cm]{ql_logo.png}}
        \end{minipage}%
        \begin{minipage}{0.78\textwidth}
            \centering
            {\LARGE\bfseries\usebeamercolor[fg]{title}\inserttitle}

            \vspace{0.3cm}

            {\usebeamerfont{subtitle}\usebeamercolor[fg]{title}\insertsubtitle}
        \end{minipage}%
        \begin{minipage}{0.1\textwidth}
            \centering
            \href{https://quantinar.com}{\includegraphics[height=1.1cm]{qr_logo.png}}
        \end{minipage}
    \end{center}

    \vspace{0.6cm}

    % Authors (left aligned)
    \hspace{0.5cm}{\usebeamerfont{author}\insertauthor}

    \vspace{0.3cm}

    % Institute/Affiliations (left aligned)
    \hspace{0.5cm}\begin{minipage}[t]{0.9\textwidth}
        \raggedright\small\insertinstitute
    \end{minipage}
}

%=============================================================================
% THEOREM ENVIRONMENTS
%=============================================================================
\theoremstyle{definition}
\setbeamertemplate{theorems}[numbered]
\newtheorem{defn}{Definition}
\newtheorem{thm}{Theorem}
\newtheorem{prop}{Proposition}
\newtheorem{rmk}{Remark}

%=============================================================================
% CENTRED MINIPAGE (no extra vertical space)
%=============================================================================
\newenvironment{cminipage}[1]{%
    \par\noindent\hfill\begin{minipage}{#1}\ignorespaces
}{%
    \end{minipage}\hfill\null\par
}

%=============================================================================
% CUSTOM COMMANDS
%=============================================================================
\newcommand{\E}{\mathbb{E}}
\newcommand{\Var}{\text{Var}}
\newcommand{\Cov}{\text{Cov}}
\newcommand{\Corr}{\text{Corr}}
\newcommand{\R}{\mathbb{R}}
\newcommand{\N}{\mathbb{N}}
\newcommand{\Z}{\mathbb{Z}}
\newcommand{\B}{\mathbf{B}}
\newcommand{\imark}{\textcolor{MainBlue}{\textbullet}}
\newcommand{\RMSE}{\text{RMSE}}
\newcommand{\MAE}{\text{MAE}}
\newcommand{\MAPE}{\text{MAPE}}
\newcommand{\correct}{\textcolor{Forest}{\checkmark}}
\newcommand{\incorrect}{\textcolor{Crimson}{\texttimes}}

% Boldface vector/matrix commands
\newcommand{\bY}{\mathbf{Y}}
\newcommand{\bX}{\mathbf{X}}
\newcommand{\bA}{\mathbf{A}}
\newcommand{\bB}{\mathbf{B}}
\newcommand{\bepsilon}{\boldsymbol{\varepsilon}}
\newcommand{\bvarepsilon}{\boldsymbol{\varepsilon}}
\newcommand{\bSigma}{\boldsymbol{\Sigma}}
\newcommand{\bPhi}{\boldsymbol{\Phi}}
\newcommand{\bGamma}{\boldsymbol{\Gamma}}
\newcommand{\bPi}{\boldsymbol{\Pi}}
\newcommand{\bc}{\mathbf{c}}
\newcommand{\balpha}{\boldsymbol{\alpha}}
\newcommand{\bbeta}{\boldsymbol{\beta}}

%=============================================================================
% TITLE INFORMATION
%=============================================================================
\title[Time Series Analysis]{Time Series Analysis and Forecasting}
\author[D.T. Pele]{Daniel Traian PELE}
\institute{Bucharest University of Economic Studies\\
IDA Institute Digital Assets\\
Blockchain Research Center\\
AI4EFin Artificial Intelligence for Energy Finance\\
Romanian Academy, Institute for Economic Forecasting\\
MSCA Digital Finance}
\date{}

%            \subtitle{Seminar X: Seminar Title}
%            \begin{document} ...
%=============================================================================

% Ensure content fits on slides
\setbeamersize{text margin left=8mm, text margin right=8mm}

%=============================================================================
% THEME AND STYLE CONFIGURATION
%=============================================================================
\usetheme{default}
% Using default theme for clean header/footer control

% Color Palette (matching Redispatch PDF)
\definecolor{MainBlue}{RGB}{26, 58, 110}
\definecolor{AccentBlue}{RGB}{26, 58, 110}
\definecolor{IDAred}{RGB}{205, 0, 0}
\definecolor{DarkGray}{RGB}{51, 51, 51}
\definecolor{MediumGray}{RGB}{128, 128, 128}
\definecolor{LightGray}{RGB}{248, 248, 248}
\definecolor{VeryLightGray}{RGB}{235, 235, 235}
\definecolor{KeynoteGray}{RGB}{218, 218, 218}
\definecolor{SectionGray}{RGB}{120, 120, 120}
\definecolor{FooterGray}{RGB}{100, 100, 100}
\definecolor{Crimson}{RGB}{220, 53, 69}
\definecolor{Forest}{RGB}{46, 125, 50}
\definecolor{Amber}{RGB}{181, 133, 63}
\definecolor{Orange}{RGB}{230, 126, 34}
\definecolor{Purple}{RGB}{142, 68, 173}

% Gradient background (exact Keynote 315° gradient: white to RGB 218,218,218)
\setbeamertemplate{background}{%
    \begin{tikzpicture}[remember picture, overlay]
        \shade[shading=axis, shading angle=315,
        top color=white, bottom color=KeynoteGray]
        (current page.south west) rectangle (current page.north east);
    \end{tikzpicture}%
}
% Fallback solid color for compatibility
\setbeamercolor{background canvas}{bg=}

\setbeamercolor{palette primary}{bg=MainBlue, fg=white}
\setbeamercolor{palette secondary}{bg=MainBlue!85, fg=white}
\setbeamercolor{palette tertiary}{bg=MainBlue!70, fg=white}
\setbeamercolor{structure}{fg=MainBlue}
\setbeamercolor{title}{fg=IDAred}
\setbeamercolor{frametitle}{fg=IDAred, bg=}
\setbeamercolor{block title}{bg=MainBlue, fg=white}
\setbeamercolor{block body}{bg=VeryLightGray, fg=DarkGray}
\setbeamercolor{block title alerted}{bg=Crimson, fg=white}
\setbeamercolor{block body alerted}{bg=Crimson!8, fg=DarkGray}
\setbeamercolor{block title example}{bg=Forest, fg=white}
\setbeamercolor{block body example}{bg=Forest!8, fg=DarkGray}
\setbeamercolor{item}{fg=MainBlue}

% Smaller institute font to avoid overfull hbox on title page
\setbeamerfont{institute}{size=\footnotesize}

% Footer colors (override Madrid theme blue)
\setbeamercolor{author in head/foot}{fg=FooterGray, bg=}
\setbeamercolor{title in head/foot}{fg=FooterGray, bg=}
\setbeamercolor{date in head/foot}{fg=FooterGray, bg=}
\setbeamercolor{section in head/foot}{fg=FooterGray, bg=}
\setbeamercolor{subsection in head/foot}{fg=FooterGray, bg=}

% Bullet styles (apply everywhere including blocks)
\setbeamertemplate{itemize item}{\color{MainBlue}$\boxdot$}
\setbeamertemplate{itemize subitem}{\color{MainBlue}$\blacktriangleright$}
\setbeamertemplate{itemize subsubitem}{\color{MainBlue}\tiny$\bullet$}
\setbeamertemplate{itemize/enumerate body begin}{\normalsize}
\setbeamertemplate{itemize/enumerate subbody begin}{\normalsize}

% Item spacing - compact style
\setlength{\leftmargini}{10pt}       % Level 1: minimal indent
\setlength{\leftmarginii}{10pt}      % Level 2: minimal additional indent
% Compact list spacing (zero extra space before/after lists in blocks)
\makeatletter
\def\@listi{\leftmargin\leftmargini \topsep 0pt \parsep 0pt \itemsep 0pt}
\def\@listii{\leftmargin\leftmarginii \topsep 0pt \parsep 0pt \itemsep 0pt}
\makeatother

\setbeamertemplate{navigation symbols}{}

%=============================================================================
% CUSTOM HEADLINE
%=============================================================================
\setbeamertemplate{headline}{%
    \vskip10pt%
    \hbox to \paperwidth{%
        \hskip0.5cm%
        {\small\color{FooterGray}\renewcommand{\hyperlink}[2]{##2}\insertsectionhead}%
        \hfill%
        \textcolor{FooterGray}{\small\insertframenumber}%
        \hskip0.5cm%
    }%
    \vskip4pt%
    {\color{FooterGray}\hrule height 0.4pt}%
}

%=============================================================================
% CUSTOM FOOTER
%=============================================================================
\usepackage{fontawesome5}

\setbeamertemplate{footline}{%
    {\color{FooterGray}\hrule height 0.4pt}%
    \vskip4pt%
    \hbox to \paperwidth{%
        \hskip0.5cm%
        \textcolor{FooterGray}{\small Time Series Analysis and Forecasting}%
        \hfill%
        \raisebox{-0.1em}{%
            \begin{tikzpicture}[x=0.08em, y=0.08em, line width=0.4pt]
                \draw[FooterGray] (0,3) -- (1,4) -- (2,3.5) -- (3,5) -- (4,4) -- (5,6) -- (6,5.5) -- (7,4) -- (8,5) -- (9,7) -- (10,6) -- (11,5) -- (12,6.5) -- (13,8) -- (14,7) -- (15,6) -- (16,7.5) -- (17,9) -- (18,8) -- (19,7) -- (20,8.5) -- (21,10) -- (22,9) -- (23,8) -- (24,9.5);
            \end{tikzpicture}%
        }%
        \hskip0.5cm%
    }%
    \vskip6pt%
}

%=============================================================================
% PACKAGES
%=============================================================================
\usepackage[utf8]{inputenc}
\usepackage[T1]{fontenc}
\usepackage[english]{babel}
\usepackage{amsmath, amssymb, amsthm}
\usepackage{mathtools}
\usepackage{bm}
\usepackage{tikz}
\usetikzlibrary{arrows.meta, positioning, shapes, calc, decorations.pathreplacing, shadings}
\usepackage{booktabs}
\usepackage{multirow}
\usepackage{array}
\usepackage{graphicx}
\usepackage{hyperref}
\usepackage{colortbl}
\usepackage{listings}
\lstset{basicstyle=\ttfamily\small, breaklines=true, frame=single, backgroundcolor=\color{VeryLightGray}}
\hypersetup{colorlinks=true, linkcolor=MainBlue, urlcolor=MainBlue}
\graphicspath{{../../logos/}{../../charts/}{../../photos/}}
\hfuzz=2pt  % Suppress tiny overfull warnings (<2pt)
\vfuzz=2pt  % Suppress tiny vertical overfull warnings (<2pt)

%=============================================================================
% QUANTLET COMMAND
%=============================================================================
\newcommand{\quantlet}[2]{%
    \hfill\href{#2}{%
        \raisebox{-0.15em}{\includegraphics[height=0.7em]{ql_logo.png}}%
        \textcolor{MainBlue}{\tiny\ #1}%
    }%
}

%=============================================================================
% CUSTOM TITLE PAGE
%=============================================================================
\defbeamertemplate*{title page}{hybrid}[1][]
{
    \vspace{0.2cm}
    % Logos row - top header (with clickable links)
    \begin{center}
        \href{https://www.ase.ro}{\includegraphics[height=1.0cm]{ase_logo.png}}\hspace{0.25cm}%
        \href{https://theida.net}{\includegraphics[height=1.0cm]{ida_logo.png}}\hspace{0.25cm}%
        \href{https://blockchain-research-center.com}{\includegraphics[height=1.0cm]{brc_logo.png}}\hspace{0.25cm}%
        \href{https://www.ai4efin.ase.ro}{\includegraphics[height=1.0cm]{ai4efin_logo.png}}\hspace{0.25cm}%
        \href{https://ipe.ro/new}{\includegraphics[height=1.0cm]{acad_logo.png}}\hspace{0.25cm}%
        \href{https://www.digital-finance-msca.com}{\includegraphics[height=1.0cm]{msca_logo.png}}%
    \end{center}

    \vspace{0.6cm}

    % Main title with Q logos on sides (with clickable links)
    \begin{center}
        \begin{minipage}{0.1\textwidth}
            \centering
            \href{https://quantlet.com}{\includegraphics[height=1.1cm]{ql_logo.png}}
        \end{minipage}%
        \begin{minipage}{0.78\textwidth}
            \centering
            {\LARGE\bfseries\usebeamercolor[fg]{title}\inserttitle}

            \vspace{0.3cm}

            {\usebeamerfont{subtitle}\usebeamercolor[fg]{title}\insertsubtitle}
        \end{minipage}%
        \begin{minipage}{0.1\textwidth}
            \centering
            \href{https://quantinar.com}{\includegraphics[height=1.1cm]{qr_logo.png}}
        \end{minipage}
    \end{center}

    \vspace{0.6cm}

    % Authors (left aligned)
    \hspace{0.5cm}{\usebeamerfont{author}\insertauthor}

    \vspace{0.3cm}

    % Institute/Affiliations (left aligned)
    \hspace{0.5cm}\begin{minipage}[t]{0.9\textwidth}
        \raggedright\small\insertinstitute
    \end{minipage}
}

%=============================================================================
% THEOREM ENVIRONMENTS
%=============================================================================
\theoremstyle{definition}
\setbeamertemplate{theorems}[numbered]
\newtheorem{defn}{Definition}
\newtheorem{thm}{Theorem}
\newtheorem{prop}{Proposition}
\newtheorem{rmk}{Remark}

%=============================================================================
% CENTRED MINIPAGE (no extra vertical space)
%=============================================================================
\newenvironment{cminipage}[1]{%
    \par\noindent\hfill\begin{minipage}{#1}\ignorespaces
}{%
    \end{minipage}\hfill\null\par
}

%=============================================================================
% CUSTOM COMMANDS
%=============================================================================
\newcommand{\E}{\mathbb{E}}
\newcommand{\Var}{\text{Var}}
\newcommand{\Cov}{\text{Cov}}
\newcommand{\Corr}{\text{Corr}}
\newcommand{\R}{\mathbb{R}}
\newcommand{\N}{\mathbb{N}}
\newcommand{\Z}{\mathbb{Z}}
\newcommand{\B}{\mathbf{B}}
\newcommand{\imark}{\textcolor{MainBlue}{\textbullet}}
\newcommand{\RMSE}{\text{RMSE}}
\newcommand{\MAE}{\text{MAE}}
\newcommand{\MAPE}{\text{MAPE}}
\newcommand{\correct}{\textcolor{Forest}{\checkmark}}
\newcommand{\incorrect}{\textcolor{Crimson}{\texttimes}}

% Boldface vector/matrix commands
\newcommand{\bY}{\mathbf{Y}}
\newcommand{\bX}{\mathbf{X}}
\newcommand{\bA}{\mathbf{A}}
\newcommand{\bB}{\mathbf{B}}
\newcommand{\bepsilon}{\boldsymbol{\varepsilon}}
\newcommand{\bvarepsilon}{\boldsymbol{\varepsilon}}
\newcommand{\bSigma}{\boldsymbol{\Sigma}}
\newcommand{\bPhi}{\boldsymbol{\Phi}}
\newcommand{\bGamma}{\boldsymbol{\Gamma}}
\newcommand{\bPi}{\boldsymbol{\Pi}}
\newcommand{\bc}{\mathbf{c}}
\newcommand{\balpha}{\boldsymbol{\alpha}}
\newcommand{\bbeta}{\boldsymbol{\beta}}

%=============================================================================
% TITLE INFORMATION
%=============================================================================
\title[Time Series Analysis]{Time Series Analysis and Forecasting}
\author[D.T. Pele]{Daniel Traian PELE}
\institute{Bucharest University of Economic Studies\\
IDA Institute Digital Assets\\
Blockchain Research Center\\
AI4EFin Artificial Intelligence for Energy Finance\\
Romanian Academy, Institute for Economic Forecasting\\
MSCA Digital Finance}
\date{}

\subtitle{Seminar 4: SARIMA Models}

\begin{document}

{
\setbeamertemplate{headline}{}
\setbeamertemplate{footline}{}
\begin{frame}
    \titlepage
\end{frame}
}


\begin{frame}{Seminar Outline}
    \begin{cminipage}{0.95\textwidth}
    \begin{itemize}
        \item \textbf{Multiple Choice Quiz} -- Knowledge check
        \vspace{0.15cm}
        \item \textbf{True/False} -- Conceptual checks
        \vspace{0.15cm}
        \item \textbf{Calculation Exercises} -- Applied practice
        \vspace{0.15cm}
        \item \textbf{Worked Examples} -- Real-world applications
        \vspace{0.15cm}
        \item \textbf{Real Data Analysis} -- Case study with airline data
        \vspace{0.15cm}
        \item \textbf{AI-Assisted Exercise} -- Critical thinking
        \vspace{0.15cm}
        \item \textbf{Summary} -- Key takeaways
    \end{itemize}
    \end{cminipage}
\end{frame}

%=============================================================================
% MULTIPLE CHOICE QUIZ
%=============================================================================
\section{Multiple Choice Quiz}

\begin{frame}{Quiz 1: Seasonal Differencing}
    \begin{cminipage}{0.95\textwidth}
    \begin{alertblock}{Question}
        For monthly data with annual seasonality, what does the operator $(1-L^{12})$ do?
    \end{alertblock}

    \vspace{0.4cm}

    \begin{block}{Answer choices}
        \textcolor{MainBlue}{\textbf{(A)}} Takes 12 consecutive differences\\[3pt]
        \textcolor{MainBlue}{\textbf{(B)}} Computes $Y_t - Y_{t-12}$\\[3pt]
        \textcolor{MainBlue}{\textbf{(C)}} Averages over 12 months\\[3pt]
        \textcolor{MainBlue}{\textbf{(D)}} Removes the first 12 observations
    \end{block}

    \vspace{0.5cm}

    \begin{center}
        \textit{Answer on next slide...}
    \end{center}
    \end{cminipage}
\end{frame}

\begin{frame}{Quiz 1: Answer}
    \begin{cminipage}{0.95\textwidth}
    \begin{exampleblock}{Answer: B -- Computes $Y_t - Y_{t-12}$}
        \textbf{Question:} What does the operator $(1-L^{12})$ do?

        \vspace{0.3cm}

    \begin{block}{Answer choices}
        \textcolor{MainBlue}{\textbf{(A)}} Takes 12 consecutive differences \incorrect\\[3pt]
        \textcolor{MainBlue}{\textbf{(B)}} \textbf{\textcolor{Forest}{Computes $Y_t - Y_{t-12}$}} \correct\\[3pt]
        \textcolor{MainBlue}{\textbf{(C)}} Averages over 12 months \incorrect\\[3pt]
        \textcolor{MainBlue}{\textbf{(D)}} Removes the first 12 observations \incorrect
    \end{block}

        \vspace{0.3cm}

        \begin{itemize}
            \item \textbf{Seasonal difference operator}: $(1-L^{12})Y_t = Y_t - L^{12}Y_t = Y_t - Y_{t-12}$
            \item \textbf{Example} (January sales): $Y_{Jan2025} - Y_{Jan2024}$
            \item \textbf{Effect}: Removes stable annual seasonal pattern
        \end{itemize}
    \end{exampleblock}

    \end{cminipage}
    \quantlet{TSA\_ch4\_def\_seasonal\_diff}{https://github.com/QuantLet/TSA/tree/main/TSA_ch4/TSA_ch4_def_seasonal_diff}
\end{frame}

\begin{frame}{Visual: Seasonal Difference}
    \begin{cminipage}{0.95\textwidth}
    \begin{center}
        \includegraphics[width=0.92\textwidth, height=0.65\textheight, keepaspectratio]{ch4_def_seasonal_diff.pdf}
    \end{center}
    \vspace{-0.2cm}
    \small Seasonal differencing removes annual patterns by comparing same periods across years.

    \end{cminipage}
    \quantlet{TSA\_ch4\_def\_seasonal\_diff}{https://github.com/QuantLet/TSA/tree/main/TSA_ch4/TSA_ch4_def_seasonal_diff}
\end{frame}

\begin{frame}{Quiz 2: SARIMA Notation}
    \begin{cminipage}{0.95\textwidth}
    \begin{alertblock}{Question}
        What does SARIMA$(1,1,1) \times (1,1,1)_{12}$ represent?
    \end{alertblock}

    \vspace{0.4cm}

    \begin{block}{Answer choices}
        \textcolor{MainBlue}{\textbf{(A)}} 12 different ARIMA models\\[3pt]
        \textcolor{MainBlue}{\textbf{(B)}} ARIMA with 12 AR and 12 MA terms\\[3pt]
        \textcolor{MainBlue}{\textbf{(C)}} ARIMA(1,1,1) with seasonal ARIMA(1,1,1) at period 12\\[3pt]
        \textcolor{MainBlue}{\textbf{(D)}} A model requiring 12 years of data
    \end{block}

    \vspace{0.5cm}

    \begin{center}
        \textit{Answer on next slide...}
    \end{center}
    \end{cminipage}
\end{frame}

\begin{frame}{Quiz 2: Answer}
    \begin{cminipage}{0.95\textwidth}
    \begin{exampleblock}{Answer: C -- ARIMA(1,1,1) with seasonal ARIMA(1,1,1) at period 12}
        \vspace{-0.2cm}
        \begin{center}
            \includegraphics[width=0.95\textwidth, height=0.52\textheight, keepaspectratio]{sem4_sarima_notation.pdf}
        \end{center}
        \vspace{-0.2cm}
        {\footnotesize
        \begin{itemize}\setlength{\itemsep}{0pt}
            \item $(1-\phi_1 L)(1-\Phi_1 L^{12})(1-L)(1-L^{12})Y_t = (1+\theta_1 L)(1+\Theta_1 L^{12})\varepsilon_t$
        \end{itemize}
        }
    \end{exampleblock}

    \end{cminipage}
    \quantlet{TSA\_ch4\_def\_sarima}{https://github.com/QuantLet/TSA/tree/main/TSA_ch4/TSA_ch4_def_sarima}
\end{frame}

\begin{frame}{Visual: SARIMA Model Structure}
    \begin{cminipage}{0.95\textwidth}
    \begin{center}
        \includegraphics[width=0.92\textwidth, height=0.65\textheight, keepaspectratio]{ch4_def_sarima.pdf}
    \end{center}
    \vspace{-0.2cm}
    \small SARIMA combines regular ARIMA components with seasonal components at lag $s$.

    \end{cminipage}
    \quantlet{TSA\_ch4\_def\_sarima}{https://github.com/QuantLet/TSA/tree/main/TSA_ch4/TSA_ch4_def_sarima}
\end{frame}

\begin{frame}{Quiz 3: The Airline Model}
    \begin{cminipage}{0.95\textwidth}
    \begin{alertblock}{Question}
        The ``airline model'' refers to SARIMA$(0,1,1) \times (0,1,1)_{12}$. How many parameters does it have (excluding variance)?
    \end{alertblock}

    \vspace{0.4cm}

    \begin{block}{Answer choices}
        \textcolor{MainBlue}{\textbf{(A)}} 2 parameters\\[3pt]
        \textcolor{MainBlue}{\textbf{(B)}} 4 parameters\\[3pt]
        \textcolor{MainBlue}{\textbf{(C)}} 6 parameters\\[3pt]
        \textcolor{MainBlue}{\textbf{(D)}} 12 parameters
    \end{block}

    \vspace{0.5cm}

    \begin{center}
        \textit{Answer on next slide...}
    \end{center}
    \end{cminipage}
\end{frame}

\begin{frame}{Quiz 3: Answer}
    \begin{cminipage}{0.95\textwidth}
    \begin{exampleblock}{Answer: A -- 2 parameters ($\theta_1$ and $\Theta_1$)}
        \vspace{-0.2cm}
        \begin{center}
            \includegraphics[width=0.95\textwidth, height=0.52\textheight, keepaspectratio]{sem4_airline_model.pdf}
        \end{center}
        \vspace{-0.2cm}
        {\footnotesize
        \begin{itemize}\setlength{\itemsep}{0pt}
            \item \textbf{Airline model}: $(1-L)(1-L^{12})Y_t = (1+\theta_1 L)(1+\Theta_1 L^{12})\varepsilon_t$
            \item Remarkably fits many seasonal economic series (Box \& Jenkins, 1970)
        \end{itemize}
        }
    \end{exampleblock}

    \end{cminipage}
    \quantlet{TSA\_ch4\_def\_sarima}{https://github.com/QuantLet/TSA/tree/main/TSA_ch4/TSA_ch4_def_sarima}
\end{frame}

\begin{frame}{Quiz 4: ACF of Seasonal Data}
    \begin{cminipage}{0.95\textwidth}
    \begin{alertblock}{Question}
        For monthly data with strong seasonality, where would you expect to see significant ACF spikes?
    \end{alertblock}

    \vspace{0.4cm}

    \begin{block}{Answer choices}
        \textcolor{MainBlue}{\textbf{(A)}} Only at lag 1\\[3pt]
        \textcolor{MainBlue}{\textbf{(B)}} Only at lag 12\\[3pt]
        \textcolor{MainBlue}{\textbf{(C)}} At lags 12, 24, 36, ...\\[3pt]
        \textcolor{MainBlue}{\textbf{(D)}} Randomly distributed
    \end{block}

    \vspace{0.5cm}

    \begin{center}
        \textit{Answer on next slide...}
    \end{center}
    \end{cminipage}
\end{frame}

\begin{frame}{Quiz 4: Answer}
    \begin{cminipage}{0.95\textwidth}
    \begin{exampleblock}{Answer: C -- At lags 12, 24, 36, ...}
        \begin{itemize}
            \item \textbf{Intuition}: January 2024 is similar to January 2023, 2022, etc.
            \item \textbf{ACF pattern}: Spikes at lags $s, 2s, 3s, \ldots$ ($\rho_{12}, \rho_{24}, \rho_{36} \neq 0$)
            \item \textbf{Diagnostic}: Slow decay at seasonal lags $\Rightarrow$ $D=1$; Cutoff after lag $s$ $\Rightarrow$ $Q=1$
        \end{itemize}
    \end{exampleblock}

    \end{cminipage}
    \quantlet{TSA\_ch4\_def\_seasonality}{https://github.com/QuantLet/TSA/tree/main/TSA_ch4/TSA_ch4_def_seasonality}
\end{frame}

\begin{frame}{Visual: Seasonality Patterns}
    \begin{cminipage}{0.95\textwidth}
    \begin{center}
        \includegraphics[width=0.92\textwidth, height=0.65\textheight, keepaspectratio]{ch4_def_seasonality.pdf}
    \end{center}
    \vspace{-0.2cm}
    \small Seasonal patterns repeat at regular intervals (monthly, quarterly, etc.) and may be additive or multiplicative.

    \end{cminipage}
    \quantlet{TSA\_ch4\_def\_seasonality}{https://github.com/QuantLet/TSA/tree/main/TSA_ch4/TSA_ch4_def_seasonality}
\end{frame}

\begin{frame}{Quiz 5: Multiplicative Structure}
    \begin{cminipage}{0.95\textwidth}
    \begin{alertblock}{Question}
        In SARIMA, what does ``multiplicative structure'' mean?
    \end{alertblock}

    \vspace{0.4cm}

    \begin{block}{Answer choices}
        \textcolor{MainBlue}{\textbf{(A)}} The seasonal amplitude grows proportionally\\[3pt]
        \textcolor{MainBlue}{\textbf{(B)}} Regular and seasonal polynomials are multiplied\\[3pt]
        \textcolor{MainBlue}{\textbf{(C)}} We multiply the data by seasonal factors\\[3pt]
        \textcolor{MainBlue}{\textbf{(D)}} The model is estimated using multiplication
    \end{block}

    \vspace{0.5cm}

    \begin{center}
        \textit{Answer on next slide...}
    \end{center}
    \end{cminipage}
\end{frame}

\begin{frame}{Quiz 5: Answer}
    \begin{cminipage}{0.95\textwidth}
    \begin{exampleblock}{Answer: B -- Regular and seasonal polynomials are multiplied}
        \begin{itemize}
            \item \textbf{Multiplicative SARIMA}: $\phi(L)\Phi(L^s)(1-L)^d(1-L^s)^D Y_t = \theta(L)\Theta(L^s)\varepsilon_t$
            \item \textbf{Example}: $(1-\phi_1 L)(1-\Phi_1 L^{12}) = 1 - \phi_1 L - \Phi_1 L^{12} + \phi_1\Phi_1 L^{13}$
            \item \textbf{Cross-term} $\phi_1\Phi_1 L^{13}$: Captures interaction between short and long dynamics
        \end{itemize}
    \end{exampleblock}

    \end{cminipage}
    \quantlet{TSA\_ch4\_def\_sarima}{https://github.com/QuantLet/TSA/tree/main/TSA_ch4/TSA_ch4_def_sarima}
\end{frame}

\begin{frame}{Quiz 6: Seasonal vs Regular Differencing}
    \begin{cminipage}{0.95\textwidth}
    \begin{alertblock}{Question}
        When would you apply both regular ($d=1$) and seasonal ($D=1$) differencing?
    \end{alertblock}

    \vspace{0.4cm}

    \begin{block}{Answer choices}
        \textcolor{MainBlue}{\textbf{(A)}} When data has only a trend\\[3pt]
        \textcolor{MainBlue}{\textbf{(B)}} When data has only seasonality\\[3pt]
        \textcolor{MainBlue}{\textbf{(C)}} When data has both trend and seasonal non-stationarity\\[3pt]
        \textcolor{MainBlue}{\textbf{(D)}} Never -- they cancel each other
    \end{block}

    \vspace{0.5cm}

    \begin{center}
        \textit{Answer on next slide...}
    \end{center}
    \end{cminipage}
\end{frame}

\begin{frame}{Quiz 6: Answer}
    \begin{cminipage}{0.95\textwidth}
    \begin{exampleblock}{Answer: C -- Both trend and seasonal non-stationarity}
        \begin{itemize}
            \item \textbf{Combined}: $W_t = (1-L)(1-L^{12})Y_t = Y_t - Y_{t-1} - Y_{t-12} + Y_{t-13}$
            \item \textbf{When needed}: ACF slow decay at lags 1,2,3... $\Rightarrow d=1$; at lags 12,24,36... $\Rightarrow D=1$
            \item \textbf{Examples}: Airline passengers, retail sales, energy demand
        \end{itemize}
    \end{exampleblock}

    \end{cminipage}
    \quantlet{TSA\_ch4\_def\_seasonal\_diff}{https://github.com/QuantLet/TSA/tree/main/TSA_ch4/TSA_ch4_def_seasonal_diff}
\end{frame}

\begin{frame}{Quiz 7: Detecting Seasonality from ACF}
    \begin{cminipage}{0.95\textwidth}
    \begin{alertblock}{Question}
        The ACF of a monthly time series shows slow decay at lags 12, 24, and 36. What does this suggest?
    \end{alertblock}

    \vspace{0.4cm}

    \begin{block}{Answer choices}
        \textcolor{MainBlue}{\textbf{(A)}} The series is stationary\\[3pt]
        \textcolor{MainBlue}{\textbf{(B)}} The series needs regular differencing only\\[3pt]
        \textcolor{MainBlue}{\textbf{(C)}} The series has a seasonal unit root requiring $D=1$\\[3pt]
        \textcolor{MainBlue}{\textbf{(D)}} The series is white noise
    \end{block}

    \vspace{0.5cm}

    \begin{center}
        \textit{Answer on next slide...}
    \end{center}
    \end{cminipage}
\end{frame}

\begin{frame}{Quiz 7: Answer}
    \begin{cminipage}{0.95\textwidth}
    \begin{exampleblock}{Answer: C -- Seasonal unit root requiring $D=1$}
        \vspace{-0.2cm}
        \begin{center}
            \includegraphics[width=0.95\textwidth, height=0.52\textheight, keepaspectratio]{sem4_seasonal_acf.pdf}
        \end{center}
        \vspace{-0.2cm}
        {\footnotesize
        \begin{itemize}\setlength{\itemsep}{0pt}
            \item \textbf{Left}: Stationary seasonal (fast decay at seasonal lags)
            \item \textbf{Right}: Seasonal unit root (slow decay $\Rightarrow$ need $D=1$)
        \end{itemize}
        }
    \end{exampleblock}

    \end{cminipage}
    \quantlet{TSA\_ch4\_def\_seasonality}{https://github.com/QuantLet/TSA/tree/main/TSA_ch4/TSA_ch4_def_seasonality}
\end{frame}

\begin{frame}{Quiz 8: Multiplicative vs Additive Seasonality}
    \begin{cminipage}{0.95\textwidth}
    \begin{alertblock}{Question}
        If the seasonal amplitude of a time series grows proportionally with the level, this indicates:
    \end{alertblock}

    \vspace{0.4cm}

    \begin{block}{Answer choices}
        \textcolor{MainBlue}{\textbf{(A)}} Additive seasonality -- use $(1-L^s)$\\[3pt]
        \textcolor{MainBlue}{\textbf{(B)}} Multiplicative seasonality -- use $\log$ transformation\\[3pt]
        \textcolor{MainBlue}{\textbf{(C)}} No seasonality present\\[3pt]
        \textcolor{MainBlue}{\textbf{(D)}} Need for regular differencing only
    \end{block}

    \vspace{0.5cm}

    \begin{center}
        \textit{Answer on next slide...}
    \end{center}
    \end{cminipage}
\end{frame}

\begin{frame}{Quiz 8: Answer}
    \begin{cminipage}{0.95\textwidth}
    \begin{exampleblock}{Answer: B -- Multiplicative seasonality, use $\log$ transformation}
        \vspace{-0.2cm}
        \begin{center}
            \includegraphics[width=0.95\textwidth, height=0.52\textheight, keepaspectratio]{sem4_mult_add.pdf}
        \end{center}
        \vspace{-0.2cm}
        {\footnotesize
        \begin{itemize}\setlength{\itemsep}{0pt}
            \item \textbf{Multiplicative}: Seasonal amplitude grows with level (diverging lines)
            \item \textbf{Solution}: Apply $\log$ transformation before fitting SARIMA
        \end{itemize}
        }
    \end{exampleblock}

    \end{cminipage}
    \quantlet{TSA\_ch4\_def\_seasonality}{https://github.com/QuantLet/TSA/tree/main/TSA_ch4/TSA_ch4_def_seasonality}
\end{frame}

\begin{frame}{Quiz 9: Seasonal Subseries Plot}
    \begin{cminipage}{0.95\textwidth}
    \begin{alertblock}{Question}
        In a seasonal subseries plot, what indicates multiplicative seasonality?
    \end{alertblock}

    \vspace{0.4cm}

    \begin{block}{Answer choices}
        \textcolor{MainBlue}{\textbf{(A)}} Lines for each month are parallel\\[3pt]
        \textcolor{MainBlue}{\textbf{(B)}} Lines for each month diverge (spread increases over time)\\[3pt]
        \textcolor{MainBlue}{\textbf{(C)}} All months have the same mean\\[3pt]
        \textcolor{MainBlue}{\textbf{(D)}} Lines are horizontal
    \end{block}

    \vspace{0.5cm}

    \begin{center}
        \textit{Answer on next slide...}
    \end{center}
    \end{cminipage}
\end{frame}

\begin{frame}{Quiz 9: Answer}
    \begin{cminipage}{0.95\textwidth}
    \begin{exampleblock}{Answer: B -- Lines diverge (spread increases over time)}
        \begin{itemize}
            \item \textbf{Subseries plot}: Groups data by month, plots each month's values across years
            \item \textbf{Parallel} $\Rightarrow$ Additive; \textbf{Diverging} $\Rightarrow$ Multiplicative; \textbf{Horizontal} $\Rightarrow$ No trend
            \item \textbf{Action}: If multiplicative, apply $\log$ before fitting SARIMA
        \end{itemize}
    \end{exampleblock}

    \end{cminipage}
    \quantlet{TSA\_ch4\_def\_seasonality}{https://github.com/QuantLet/TSA/tree/main/TSA_ch4/TSA_ch4_def_seasonality}
\end{frame}

\begin{frame}{Quiz 10: Invertibility in SARIMA}
    \begin{cminipage}{0.95\textwidth}
    \begin{alertblock}{Question}
        For SARIMA$(0,1,1) \times (0,1,1)_{12}$ to be invertible, which condition must hold?
    \end{alertblock}

    \vspace{0.4cm}

    \begin{block}{Answer choices}
        \textcolor{MainBlue}{\textbf{(A)}} $|\theta_1| < 1$ only\\[3pt]
        \textcolor{MainBlue}{\textbf{(B)}} $|\Theta_1| < 1$ only\\[3pt]
        \textcolor{MainBlue}{\textbf{(C)}} Both $|\theta_1| < 1$ and $|\Theta_1| < 1$\\[3pt]
        \textcolor{MainBlue}{\textbf{(D)}} No invertibility condition exists for MA models
    \end{block}

    \vspace{0.5cm}

    \begin{center}
        \textit{Answer on next slide...}
    \end{center}
    \end{cminipage}
\end{frame}

\begin{frame}{Quiz 10: Answer}
    \begin{cminipage}{0.95\textwidth}
    \begin{exampleblock}{Answer: C -- Both $|\theta_1| < 1$ and $|\Theta_1| < 1$}
        \begin{itemize}
            \item \textbf{Invertibility}: All MA roots outside unit circle
            \item \textbf{Multiplicative MA}: $(1+\theta_1 L)(1+\Theta_1 L^{12})$
            \item \textbf{Roots}: Regular $|z| = |{-1}/{\theta_1}| > 1 \Leftrightarrow |\theta_1| < 1$; Seasonal $|\Theta_1| < 1$
            \item \textbf{Both} conditions required for overall invertibility!
        \end{itemize}
    \end{exampleblock}

    \end{cminipage}
    \quantlet{TSA\_ch4\_def\_sarima}{https://github.com/QuantLet/TSA/tree/main/TSA_ch4/TSA_ch4_def_sarima}
\end{frame}

\begin{frame}{Quiz 11: HEGY Test}
    \begin{cminipage}{0.95\textwidth}
    \begin{alertblock}{Question}
        The HEGY test is used to:
    \end{alertblock}

    \vspace{0.4cm}

    \begin{block}{Answer choices}
        \textcolor{MainBlue}{\textbf{(A)}} Estimate SARIMA parameters\\[3pt]
        \textcolor{MainBlue}{\textbf{(B)}} Test for unit roots at different frequencies (trend and seasonal)\\[3pt]
        \textcolor{MainBlue}{\textbf{(C)}} Check residual normality\\[3pt]
        \textcolor{MainBlue}{\textbf{(D)}} Compare SARIMA models using information criteria
    \end{block}

    \vspace{0.5cm}

    \begin{center}
        \textit{Answer on next slide...}
    \end{center}
    \end{cminipage}
\end{frame}

\begin{frame}{Quiz 11: Answer}
    \begin{cminipage}{0.95\textwidth}
    \begin{exampleblock}{Answer: B -- Test for unit roots at different frequencies}
        \begin{itemize}
            \item \textbf{HEGY test} (Hylleberg-Engle-Granger-Yoo, 1990)
            \item Tests at: Zero freq ($\omega = 0$) $\Rightarrow d = 1$; Nyquist ($\omega = \pi$); Seasonal $\Rightarrow D = 1$
            \item \textbf{Decision}: Reject all $\Rightarrow$ seasonal dummies; Don't reject seasonal $\Rightarrow$ seasonal differencing
        \end{itemize}
    \end{exampleblock}

    \end{cminipage}
    \quantlet{TSA\_ch4\_def\_sarima}{https://github.com/QuantLet/TSA/tree/main/TSA_ch4/TSA_ch4_def_sarima}
\end{frame}

\begin{frame}{Quiz 12: Seasonal MA Identification}
    \begin{cminipage}{0.95\textwidth}
    \begin{alertblock}{Question}
        After applying $(1-L)(1-L^{12})$, the ACF shows a single significant spike at lag 12 only (no spike at lag 1). The PACF decays at seasonal lags. This suggests:
    \end{alertblock}

    \vspace{0.4cm}

    \begin{block}{Answer choices}
        \textcolor{MainBlue}{\textbf{(A)}} SARIMA$(0,1,0) \times (0,1,1)_{12}$\\[3pt]
        \textcolor{MainBlue}{\textbf{(B)}} SARIMA$(0,1,1) \times (0,1,0)_{12}$\\[3pt]
        \textcolor{MainBlue}{\textbf{(C)}} SARIMA$(1,1,0) \times (1,1,0)_{12}$\\[3pt]
        \textcolor{MainBlue}{\textbf{(D)}} SARIMA$(0,1,1) \times (0,1,1)_{12}$
    \end{block}

    \vspace{0.5cm}

    \begin{center}
        \textit{Answer on next slide...}
    \end{center}
    \end{cminipage}
\end{frame}

\begin{frame}{Quiz 12: Answer}
    \begin{cminipage}{0.95\textwidth}
    \begin{exampleblock}{Answer: A -- SARIMA$(0,1,0) \times (0,1,1)_{12}$}
        \begin{itemize}
            \item \textbf{Pattern}: Regular lags -- no spikes in ACF/PACF; Seasonal lags -- ACF cuts off at $s$, PACF decays
            \item \textbf{Interpretation}: No regular MA ($q = 0$); Seasonal MA(1) indicated ($Q = 1$)
            \item \textbf{Model}: $(1-L)(1-L^{12})Y_t = (1 + \Theta_1 L^{12})\varepsilon_t$
        \end{itemize}
    \end{exampleblock}

    \end{cminipage}
    \quantlet{TSA\_ch4\_def\_sarima}{https://github.com/QuantLet/TSA/tree/main/TSA_ch4/TSA_ch4_def_sarima}
\end{frame}

\begin{frame}{Quiz 13: Over-differencing}
    \begin{cminipage}{0.95\textwidth}
    \begin{alertblock}{Question}
        After differencing, the ACF shows a large negative spike at lag 1 or lag $s$. This typically indicates:
    \end{alertblock}

    \vspace{0.4cm}

    \begin{block}{Answer choices}
        \textcolor{MainBlue}{\textbf{(A)}} The model needs more AR terms\\[3pt]
        \textcolor{MainBlue}{\textbf{(B)}} The series has been over-differenced\\[3pt]
        \textcolor{MainBlue}{\textbf{(C)}} The series is perfectly stationary\\[3pt]
        \textcolor{MainBlue}{\textbf{(D)}} Heteroskedasticity is present
    \end{block}

    \vspace{0.5cm}

    \begin{center}
        \textit{Answer on next slide...}
    \end{center}
    \end{cminipage}
\end{frame}

\begin{frame}{Quiz 13: Answer}
    \begin{cminipage}{0.95\textwidth}
    \begin{exampleblock}{Answer: B -- The series has been over-differenced}
        \begin{itemize}
            \item \textbf{Signature}: ACF at lag 1 $\approx -0.5$ $\Rightarrow$ over-diff at $d$; ACF at lag $s$ $\approx -0.5$ $\Rightarrow$ over-diff at $D$
            \item \textbf{Why?} $\Delta^2 Y_t = \varepsilon_t - \varepsilon_{t-1}$ is MA(1) with $\theta = -1$, giving $\rho_1 = -0.5$
            \item \textbf{Fix}: Reduce $d$ or $D$ by one and re-examine ACF/PACF
        \end{itemize}
    \end{exampleblock}

    \end{cminipage}
    \quantlet{TSA\_ch4\_def\_sarima}{https://github.com/QuantLet/TSA/tree/main/TSA_ch4/TSA_ch4_def_sarima}
\end{frame}

\begin{frame}{Quiz 14: Forecasting Horizon}
    \begin{cminipage}{0.95\textwidth}
    \begin{alertblock}{Question}
        For a SARIMA model with $D=1$, what happens to forecast confidence intervals as the horizon $h \to \infty$?
    \end{alertblock}

    \vspace{0.4cm}

    \begin{block}{Answer choices}
        \textcolor{MainBlue}{\textbf{(A)}} They converge to a fixed width\\[3pt]
        \textcolor{MainBlue}{\textbf{(B)}} They grow without bound\\[3pt]
        \textcolor{MainBlue}{\textbf{(C)}} They shrink to zero\\[3pt]
        \textcolor{MainBlue}{\textbf{(D)}} They oscillate seasonally
    \end{block}

    \vspace{0.5cm}

    \begin{center}
        \textit{Answer on next slide...}
    \end{center}
    \end{cminipage}
\end{frame}

\begin{frame}{Quiz 14: Answer}
    \begin{cminipage}{0.95\textwidth}
    \begin{exampleblock}{Answer: B -- They grow without bound}
        \begin{itemize}
            \item \textbf{Unit root property}: Any unit root causes unbounded forecast variance
            \item \textbf{For SARIMA with $D=1$}: $\Var(\hat{Y}_{T+h} - Y_{T+h}) \to \infty$ as $h \to \infty$
            \item \textbf{Intuition}: Seasonal shocks accumulate; long-range forecasts have wide CIs
        \end{itemize}
    \end{exampleblock}

    \end{cminipage}
    \quantlet{TSA\_ch4\_sarima\_forecast}{https://github.com/QuantLet/TSA/tree/main/TSA_ch4/TSA_ch4_sarima_forecast}
\end{frame}

\begin{frame}{Quiz 15: Seasonal Period Selection}
    \begin{cminipage}{0.95\textwidth}
    \begin{alertblock}{Question}
        You have daily data showing clear weekly patterns. What seasonal period $s$ should you use in a SARIMA model?
    \end{alertblock}

    \vspace{0.4cm}

    \begin{block}{Answer choices}
        \textcolor{MainBlue}{\textbf{(A)}} $s = 12$ (monthly)\\[3pt]
        \textcolor{MainBlue}{\textbf{(B)}} $s = 7$ (weekly)\\[3pt]
        \textcolor{MainBlue}{\textbf{(C)}} $s = 365$ (yearly)\\[3pt]
        \textcolor{MainBlue}{\textbf{(D)}} $s = 24$ (hourly)
    \end{block}

    \vspace{0.5cm}

    \begin{center}
        \textit{Answer on next slide...}
    \end{center}
    \end{cminipage}
\end{frame}

\begin{frame}{Quiz 15: Answer}
    \begin{cminipage}{0.95\textwidth}
    \begin{exampleblock}{Answer: B -- $s = 7$ (weekly)}
        {\small
        \begin{center}
        \vspace{-0.2cm}
        \begin{tabular}{lll}
            \textbf{Data} & \textbf{Pattern} & \textbf{Period $s$} \\
            Daily & Weekly & 7 \\
            Monthly & Annual & 12 \\
            Quarterly & Annual & 4
        \end{tabular}
        \end{center}
        \vspace{-0.1cm}
        \begin{itemize}\setlength{\itemsep}{0pt}
            \item \textbf{Rule}: $s$ = observations per cycle of dominant pattern
        \end{itemize}
        }
    \end{exampleblock}

    \end{cminipage}
    \quantlet{TSA\_ch4\_def\_seasonality}{https://github.com/QuantLet/TSA/tree/main/TSA_ch4/TSA_ch4_def_seasonality}
\end{frame}

\begin{frame}{Quiz 16: Seasonal AR Component}
    \begin{cminipage}{0.95\textwidth}
    \begin{alertblock}{Question}
        In the seasonal component $\Phi(L^s) = 1 - \Phi_1 L^s$, what does the coefficient $\Phi_1 = 0.8$ tell us?
    \end{alertblock}

    \vspace{0.4cm}

    \begin{block}{Answer choices}
        \textcolor{MainBlue}{\textbf{(A)}} 80\% of this period's value comes from the previous period\\[3pt]
        \textcolor{MainBlue}{\textbf{(B)}} There is 80\% correlation between consecutive observations\\[3pt]
        \textcolor{MainBlue}{\textbf{(C)}} Strong seasonal persistence: the conditional expectation depends on 80\% of last year's same-period value\\[3pt]
        \textcolor{MainBlue}{\textbf{(D)}} The seasonal pattern explains 80\% of variance
    \end{block}

    \vspace{0.5cm}

    \begin{center}
        \textit{Answer on next slide...}
    \end{center}
    \end{cminipage}
\end{frame}

\begin{frame}{Quiz 16: Answer}
    \begin{cminipage}{0.95\textwidth}
    \begin{exampleblock}{Answer: C -- Strong seasonal persistence ($\Phi_1 = 0.8$)}
        \begin{itemize}
            \item \textbf{SAR(1)}: $Y_t = \Phi_1 Y_{t-12} + \varepsilon_t$
            \item \textbf{With $\Phi_1 = 0.8$}: $Y_{Jan2024} = 0.8 \cdot Y_{Jan2023} + \varepsilon_t$
            \item \textbf{Interpretation}: There is strong linear dependence ($\Phi_1 = 0.8$) on the same period last year. Note: $\Phi_1$ is a regression coefficient, not $R^2$; the explained variance would be $\Phi_1^2 = 0.64$
            \item \textbf{Stationarity}: Requires $|\Phi_1| < 1$ (satisfied here)
        \end{itemize}
    \end{exampleblock}

    \end{cminipage}
    \quantlet{TSA\_ch4\_def\_sarima}{https://github.com/QuantLet/TSA/tree/main/TSA_ch4/TSA_ch4_def_sarima}
\end{frame}

\begin{frame}{Quiz 17: Seasonal Stationarity}
    \begin{cminipage}{0.95\textwidth}
    \begin{alertblock}{Question}
        A seasonal process with $\Phi_1 = 1$ in SARIMA$(0,0,0) \times (1,0,0)_{12}$ is:
    \end{alertblock}

    \vspace{0.4cm}

    \begin{block}{Answer choices}
        \textcolor{MainBlue}{\textbf{(A)}} Stationary\\[3pt]
        \textcolor{MainBlue}{\textbf{(B)}} Has a seasonal unit root (seasonally integrated)\\[3pt]
        \textcolor{MainBlue}{\textbf{(C)}} Explosive\\[3pt]
        \textcolor{MainBlue}{\textbf{(D)}} Undefined
    \end{block}

    \vspace{0.5cm}

    \begin{center}
        \textit{Answer on next slide...}
    \end{center}
    \end{cminipage}
\end{frame}

\begin{frame}{Quiz 17: Answer}
    \begin{cminipage}{0.95\textwidth}
    \begin{exampleblock}{Answer: B -- Has a seasonal unit root}
        \begin{itemize}
            \item \textbf{Model}: $Y_t = Y_{t-12} + \varepsilon_t$ (seasonal random walk)
            \item \textbf{Properties}: Variance grows with time; each month follows its own RW; need $D = 1$
            \item \textbf{Analogy}: Like regular random walk but at seasonal frequency
        \end{itemize}
    \end{exampleblock}

    \end{cminipage}
    \quantlet{TSA\_ch4\_def\_sarima}{https://github.com/QuantLet/TSA/tree/main/TSA_ch4/TSA_ch4_def_sarima}
\end{frame}

\begin{frame}{Quiz 18: Model Comparison}
    \begin{cminipage}{0.95\textwidth}
    \begin{alertblock}{Question}
        Model A: SARIMA$(1,1,1) \times (1,1,1)_{12}$ has AIC = 520. Model B: SARIMA$(0,1,1) \times (0,1,1)_{12}$ has AIC = 525. Which statement is most accurate?
    \end{alertblock}

    \vspace{0.4cm}

    \begin{block}{Answer choices}
        \textcolor{MainBlue}{\textbf{(A)}} Model A is always better since it has lower AIC\\[3pt]
        \textcolor{MainBlue}{\textbf{(B)}} Model B should be preferred due to parsimony despite higher AIC\\[3pt]
        \textcolor{MainBlue}{\textbf{(C)}} The AIC difference of 5 suggests Model A is substantially better\\[3pt]
        \textcolor{MainBlue}{\textbf{(D)}} We cannot compare models with different orders
    \end{block}

    \vspace{0.5cm}

    \begin{center}
        \textit{Answer on next slide...}
    \end{center}
    \end{cminipage}
\end{frame}

\begin{frame}{Quiz 18: Answer}
    \begin{cminipage}{0.95\textwidth}
    \begin{exampleblock}{Answer: C -- AIC difference of 5 suggests Model A is substantially better}
        \begin{itemize}
            \item \textbf{Rule of thumb}: $\Delta$AIC $< 2$: equivalent; $2$--$10$: some evidence; $> 10$: strong evidence
            \item \textbf{Here}: $\Delta$AIC $= 5$ suggests Model A meaningfully better
            \item \textbf{Always}: Also check residual diagnostics and forecast performance!
        \end{itemize}
    \end{exampleblock}

    \end{cminipage}
    \quantlet{TSA\_ch4\_def\_sarima}{https://github.com/QuantLet/TSA/tree/main/TSA_ch4/TSA_ch4_def_sarima}
\end{frame}

\begin{frame}{Quiz 19: Seasonal Patterns in Residuals}
    \begin{cminipage}{0.95\textwidth}
    \begin{alertblock}{Question}
        After fitting a SARIMA model, you notice significant ACF spikes at lags 12 and 24 in the residuals. What does this indicate?
    \end{alertblock}

    \vspace{0.4cm}

    \begin{block}{Answer choices}
        \textcolor{MainBlue}{\textbf{(A)}} The model is correctly specified\\[3pt]
        \textcolor{MainBlue}{\textbf{(B)}} The seasonal component is inadequate\\[3pt]
        \textcolor{MainBlue}{\textbf{(C)}} The data is not seasonal\\[3pt]
        \textcolor{MainBlue}{\textbf{(D)}} Overfitting has occurred
    \end{block}

    \vspace{0.5cm}

    \begin{center}
        \textit{Answer on next slide...}
    \end{center}
    \end{cminipage}
\end{frame}

\begin{frame}{Quiz 19: Answer}
    \begin{cminipage}{0.95\textwidth}
    \begin{exampleblock}{Answer: B -- The seasonal component is inadequate}
        \begin{itemize}
            \item \textbf{Diagnostics}: Good residuals should be white noise (no significant ACF)
            \item \textbf{Seasonal ACF in residuals}: Model hasn't captured seasonal structure; try increasing $P$ or $Q$; verify $D$ is correct
            \item \textbf{Action}: Try SARIMA with higher seasonal order, check Ljung-Box at seasonal lags
        \end{itemize}
    \end{exampleblock}

    \end{cminipage}
    \quantlet{TSA\_ch4\_diagnostics}{https://github.com/QuantLet/TSA/tree/main/TSA_ch4/TSA_ch4_diagnostics}
\end{frame}

\begin{frame}{Quiz 20: Practical Forecasting}
    \begin{cminipage}{0.95\textwidth}
    \begin{alertblock}{Question}
        You're forecasting monthly retail sales with SARIMA$(0,1,1) \times (0,1,1)_{12}$. For the 13-month-ahead forecast, which historical observations are most influential?
    \end{alertblock}

    \vspace{0.4cm}

    \begin{block}{Answer choices}
        \textcolor{MainBlue}{\textbf{(A)}} Only the most recent observation\\[3pt]
        \textcolor{MainBlue}{\textbf{(B)}} The observation from the same month last year\\[3pt]
        \textcolor{MainBlue}{\textbf{(C)}} All observations equally\\[3pt]
        \textcolor{MainBlue}{\textbf{(D)}} Only observations from the same month in all previous years
    \end{block}

    \vspace{0.5cm}

    \begin{center}
        \textit{Answer on next slide...}
    \end{center}
    \end{cminipage}
\end{frame}

\begin{frame}{Quiz 20: Answer}
    \begin{cminipage}{0.95\textwidth}
    \begin{exampleblock}{Answer: B -- The observation from the same month last year}
        \begin{itemize}
            \item \textbf{For 13-month ahead}: Most influential is $Y_{T-11}$ (same month last year), also $Y_T$ and $Y_{T-12}$
            \item \textbf{Intuition}: ``Next January looks like last January, adjusted for recent trend''
        \end{itemize}
    \end{exampleblock}

    \end{cminipage}
    \quantlet{TSA\_ch4\_sarima\_forecast}{https://github.com/QuantLet/TSA/tree/main/TSA_ch4/TSA_ch4_sarima_forecast}
\end{frame}

%=============================================================================
% TRUE/FALSE
%=============================================================================
\section{True/False}

\begin{frame}{True or False? --- Questions}
    \begin{cminipage}{0.95\textwidth}
    \footnotesize
    \begin{center}
    \begin{tabular}{p{9cm}c}
        \toprule
        \textbf{Statement} & \textbf{T/F?} \\
        \midrule
        1. The seasonal period $s$ for quarterly data with annual patterns is $s=4$. & ? \\[0.15cm]
        2. SARIMA models can only handle one seasonal frequency. & ? \\[0.15cm]
        3. If AIC selects SARIMA$(1,1,1) \times (1,1,1)_{12}$ and BIC selects the airline model, BIC is always wrong. & ? \\[0.15cm]
        4. The Kruskal-Wallis test can detect seasonality without assuming normality. & ? \\[0.15cm]
        5. After fitting a SARIMA model, residuals should show no significant ACF at seasonal lags. & ? \\[0.15cm]
        6. Log transformation converts multiplicative seasonality to additive. & ? \\
        \bottomrule
    \end{tabular}
    \end{center}
    \end{cminipage}
\end{frame}

\begin{frame}{True or False? --- Answers}
    \begin{cminipage}{0.95\textwidth}
    \scriptsize
    \begin{center}
    \begin{tabular}{p{7.5cm}cc}
        \toprule
        \textbf{Statement} & \textbf{T/F} & \textbf{Explanation} \\
        \midrule
        1. The seasonal period $s$ for quarterly data with annual patterns is $s=4$. & \textcolor{Forest}{\textbf{T}} & {\tiny 4 quarters/year} \\[0.08cm]
        2. SARIMA models can only handle one seasonal frequency. & \textcolor{Forest}{\textbf{T}} & {\tiny Need TBATS for multiple} \\[0.08cm]
        3. If AIC selects SARIMA$(1,1,1) \times (1,1,1)_{12}$ and BIC selects the airline model, BIC is always wrong. & \textcolor{Crimson}{\textbf{F}} & {\tiny BIC penalizes more} \\[0.08cm]
        4. The Kruskal-Wallis test can detect seasonality without assuming normality. & \textcolor{Forest}{\textbf{T}} & {\tiny Nonparametric test} \\[0.08cm]
        5. After fitting a SARIMA model, residuals should show no significant ACF at seasonal lags. & \textcolor{Forest}{\textbf{T}} & {\tiny White noise residuals} \\[0.08cm]
        6. Log transformation converts multiplicative seasonality to additive. & \textcolor{Forest}{\textbf{T}} & {\tiny $\log(T \times S) = \log T + \log S$} \\
        \bottomrule
    \end{tabular}
    \end{center}
    \end{cminipage}
\end{frame}

%=============================================================================
% CALCULATION EXERCISES
%=============================================================================
\section{Calculation Exercises}

\begin{frame}{Exercise 1: Expanding the Seasonal Difference}
    \begin{cminipage}{0.95\textwidth}
    \begin{alertblock}{Problem}
        \begin{itemize}\setlength{\itemsep}{0pt}
            \item \textbf{Task}: Expand $(1-L)(1-L^{12})Y_t$ fully. What observations are involved?
        \end{itemize}
    \end{alertblock}

    \vspace{0.2cm}
    \begin{exampleblock}{Solution}
        $(1-L)(1-L^{12}) = 1 - L - L^{12} + L^{13}$

        \vspace{0.2cm}
        Therefore:
        $(1-L)(1-L^{12})Y_t = Y_t - Y_{t-1} - Y_{t-12} + Y_{t-13}$

        \vspace{0.2cm}
        \begin{itemize}\setlength{\itemsep}{0pt}
            \item First seasonal difference: $Y_t - Y_{t-12}$ (this year vs last year)
            \item Then regular difference: $(Y_t - Y_{t-12}) - (Y_{t-1} - Y_{t-13})$
        \end{itemize}
    \end{exampleblock}
    \end{cminipage}
\end{frame}

\begin{frame}{Exercise 2: Airline Model Expansion}
    \begin{cminipage}{0.95\textwidth}
    \begin{alertblock}{Problem}
        \begin{itemize}\setlength{\itemsep}{0pt}
            \item \textbf{Task}: Write out the full equation for the airline model SARIMA$(0,1,1) \times (0,1,1)_{12}$:
            \item \textbf{Formula}: $(1-L)(1-L^{12})Y_t = (1+\theta_1 L)(1+\Theta_1 L^{12})\varepsilon_t$
        \end{itemize}
    \end{alertblock}

    \vspace{0.2cm}
    \begin{exampleblock}{Solution}
        Expand the MA side:
        $(1+\theta_1 L)(1+\Theta_1 L^{12}) = 1 + \theta_1 L + \Theta_1 L^{12} + \theta_1 \Theta_1 L^{13}$

        \vspace{0.2cm}
        Full model:
        $Y_t - Y_{t-1} - Y_{t-12} + Y_{t-13} = \varepsilon_t + \theta_1 \varepsilon_{t-1} + \Theta_1 \varepsilon_{t-12} + \theta_1 \Theta_1 \varepsilon_{t-13}$

        \vspace{0.2cm}
        \begin{itemize}\setlength{\itemsep}{0pt}
            \item \textbf{Note}: The cross-term $\theta_1 \Theta_1 L^{13}$ is the multiplicative interaction between regular and seasonal MA components.
        \end{itemize}
    \end{exampleblock}
    \end{cminipage}
\end{frame}

\begin{frame}{Exercise 3: Parameter Count}
    \begin{cminipage}{0.95\textwidth}
    \begin{alertblock}{Problem}
        \begin{itemize}\setlength{\itemsep}{0pt}
            \item \textbf{Task}: How many parameters (excluding $\sigma^2$) are in SARIMA$(2,1,1) \times (1,0,1)_4$?
        \end{itemize}
    \end{alertblock}

    \vspace{0.2cm}
    \begin{exampleblock}{Solution}
        \begin{itemize}\setlength{\itemsep}{0pt}
            \item Regular AR($p=2$): $\phi_1, \phi_2$ $\Rightarrow$ 2 parameters
            \item Regular MA($q=1$): $\theta_1$ $\Rightarrow$ 1 parameter
            \item Seasonal AR($P=1$): $\Phi_1$ $\Rightarrow$ 1 parameter
            \item Seasonal MA($Q=1$): $\Theta_1$ $\Rightarrow$ 1 parameter
        \end{itemize}

        \vspace{0.2cm}
        \textbf{Total: 5 parameters}

        \vspace{0.2cm}
        Note: The differencing orders ($d=1$, $D=0$) don't add parameters -- they're transformations applied to the data.
    \end{exampleblock}
    \end{cminipage}
\end{frame}

\begin{frame}{Exercise 4: SARIMA Forecasting}
    \begin{cminipage}{0.95\textwidth}
    \begin{alertblock}{Problem}
        \begin{itemize}\setlength{\itemsep}{0pt}
            \item \textbf{Data}: Airline model with $\theta_1 = -0.4$ and $\Theta_1 = -0.6$
            \item $Y_T = 500$, $Y_{T-1} = 495$, $Y_{T-11} = 480$, $Y_{T-12} = 470$
            \item $\varepsilon_T = 5$, $\varepsilon_{T-11} = -3$, $\varepsilon_{T-12} = 2$
            \item \textbf{Calculate}: Forecast $Y_{T+1}$
        \end{itemize}
    \end{alertblock}

    \vspace{0.2cm}
    \begin{exampleblock}{Solution}
        From the model: $Y_{T+1} = Y_T + Y_{T-11} - Y_{T-12} + \varepsilon_{T+1} + \theta_1 \varepsilon_T + \Theta_1 \varepsilon_{T-11} + \theta_1 \Theta_1 \varepsilon_{T-12}$

        Setting $\E[\varepsilon_{T+1}] = 0$:

        $\hat{Y}_{T+1} = 500 + 480 - 470 + 0 + (-0.4)(5) + (-0.6)(-3) + (-0.4)(-0.6)(2)$

        $= 510 - 2 + 1.8 + 0.48 = \mathbf{510.28}$
    \end{exampleblock}
    \end{cminipage}
\end{frame}

\begin{frame}{Exercise 5: Identifying Seasonal Period}
    \begin{cminipage}{0.95\textwidth}
    \begin{alertblock}{Problem}
        \begin{itemize}\setlength{\itemsep}{0pt}
            \item \textbf{Task}: Match each data type with its typical seasonal period $s$:
            \item Quarterly GDP, Monthly retail sales, Weekly restaurant reservations, Daily electricity demand
        \end{itemize}
    \end{alertblock}

    \vspace{0.2cm}
    \begin{exampleblock}{Solution}
        {\small
        \begin{itemize}\setlength{\itemsep}{0pt}
            \item Quarterly GDP: $s = 4$ (annual cycle over 4 quarters)
            \item Monthly retail sales: $s = 12$ (annual cycle over 12 months)
            \item Weekly restaurant reservations: $s = 7$ (weekly cycle) or $s = 52$ (annual)
            \item Daily electricity demand: $s = 7$ (weekly pattern) or $s = 365$ (annual)
        \end{itemize}
        \vspace{-0.1cm}
        \textbf{Note}: Some series have multiple seasonal patterns (e.g., daily data may have weekly AND annual cycles).
        }
    \end{exampleblock}
    \end{cminipage}
\end{frame}

%=============================================================================
% WORKED EXAMPLES
%=============================================================================
\section{Worked Examples}

\begin{frame}{Example: Monthly Retail Sales Analysis}
    \begin{cminipage}{0.95\textwidth}
    \begin{alertblock}{Scenario}
        You have 5 years of monthly retail sales data showing clear December peaks and January troughs. Build an appropriate SARIMA model.
    \end{alertblock}

    \vspace{0.2cm}

    \begin{exampleblock}{Step-by-step Approach}
        \begin{enumerate}
            \item \textbf{Visual inspection}: Plot shows upward trend + strong December spikes
            \item \textbf{Seasonal period}: Monthly data with annual pattern $\Rightarrow s = 12$
            \item \textbf{Transformation}: Consider $\log(Y_t)$ if seasonal amplitude grows with level
            \item \textbf{Differencing}: Try $(1-L)(1-L^{12})Y_t$ -- check ACF/PACF
            \item \textbf{Model selection}: Start with airline model, compare via AIC
        \end{enumerate}
    \end{exampleblock}
    \end{cminipage}
\end{frame}

\begin{frame}{Example: ACF/PACF Interpretation for Seasonal Data}
    \begin{cminipage}{0.95\textwidth}
    \begin{alertblock}{Observed Patterns (after differencing)}
        \begin{itemize}\setlength{\itemsep}{0pt}
            \item ACF: Significant at lags 1, 12; cuts off after lag 1 and lag 12
            \item PACF: Significant at lags 1, 12, 13; decays at multiples of 12
        \end{itemize}
    \end{alertblock}

    \vspace{0.2cm}

    \begin{exampleblock}{Interpretation}
        \textbf{Regular component}: ACF cuts off at 1 $\Rightarrow$ MA(1)

        \textbf{Seasonal component}: ACF significant only at lag 12 $\Rightarrow$ seasonal MA(1)

        \textbf{Suggested model}: SARIMA$(0,d,1) \times (0,D,1)_{12}$ -- the airline model!

        \vspace{0.2cm}
        \textbf{Alternative check}: If PACF showed cutoff at seasonal lags instead of ACF, consider seasonal AR terms.
    \end{exampleblock}
    \end{cminipage}
\end{frame}

\begin{frame}[fragile]{Example: Python Implementation}
    \begin{cminipage}{0.95\textwidth}
    \begin{block}{Fitting SARIMA in Python}
        \small
        \begin{verbatim}
from statsmodels.tsa.statespace.sarimax import SARIMAX
import pmdarima as pm

# Manual fit
model = SARIMAX(y, order=(0,1,1), seasonal_order=(0,1,1,12))
results = model.fit()
print(results.summary())

# Automatic selection
auto_model = pm.auto_arima(y, seasonal=True, m=12,
                           start_p=0, max_p=2,
                           start_q=0, max_q=2,
                           d=1, D=1,
                           trace=True)
        \end{verbatim}
    \end{block}
    \end{cminipage}
\end{frame}

\begin{frame}[fragile]{Example: Interpreting SARIMA Output}
    \begin{cminipage}{0.95\textwidth}
    \begin{block}{Sample statsmodels Output}
        \scriptsize
        \begin{verbatim}
                         SARIMAX Results
==============================================================
Model:            SARIMAX(0,1,1)x(0,1,1,12)   AIC:    1348.52
                                               BIC:    1358.21
==============================================================
                 coef    std err      z     P>|z|
--------------------------------------------------------------
ma.L1         -0.4018      0.072   -5.58    0.000
ma.S.L12      -0.5521      0.081   -6.82    0.000
sigma2      1254.3201    142.856    8.78    0.000
        \end{verbatim}
    \end{block}
    \vspace{-0.15cm}
    \begin{block}{Interpretation}
        {\small
        \begin{itemize}\setlength{\itemsep}{0pt}
            \item $\hat{\theta}_1 = -0.40$: Negative MA means positive shocks reduce next period's value
            \item $\hat{\Theta}_1 = -0.55$: Same-season correlation is captured
            \item Both coefficients significant $(p < 0.001)$; $|\theta|, |\Theta| < 1$ -- invertible
        \end{itemize}
        }
    \end{block}
    \end{cminipage}
\end{frame}

%=============================================================================
% REAL DATA ANALYSIS
%=============================================================================
\section{Real Data Analysis}

\begin{frame}{Case Study: Airline Passengers (1949--1960)}
    \begin{cminipage}{0.95\textwidth}
    \vspace{-0.3cm}
    \begin{center}
        \includegraphics[width=0.85\textwidth, height=0.55\textheight, keepaspectratio]{ch4_airline_data.pdf}
    \end{center}
    \vspace{-0.2cm}
    {\footnotesize
    \begin{itemize}\setlength{\itemsep}{0pt}
        \item Classic Box-Jenkins dataset: 144 monthly observations
        \item Clear \textbf{upward trend} and \textbf{seasonal pattern} (summer peaks)
        \item Seasonal amplitude \textbf{grows with level} $\Rightarrow$ multiplicative seasonality
        \item Suggests: log transformation + SARIMA modeling
    \end{itemize}
    }

    \end{cminipage}
    \quantlet{TSA\_ch4\_airline\_data}{https://github.com/QuantLet/TSA/tree/main/TSA_ch4/TSA_ch4_airline_data}
\end{frame}

\begin{frame}{ACF/PACF Analysis After Differencing}
    \begin{cminipage}{0.95\textwidth}
    \vspace{-0.3cm}
    \begin{center}
        \includegraphics[width=0.85\textwidth, height=0.55\textheight, keepaspectratio]{ch4_acf_pacf.pdf}
    \end{center}
    \vspace{-0.2cm}
    {\footnotesize
    \begin{itemize}\setlength{\itemsep}{0pt}
        \item After $(1-L)(1-L^{12})\log(Y_t)$: series appears stationary
        \item Significant spike at lag 1 in ACF $\Rightarrow$ MA(1) component
        \item Significant spike at lag 12 in ACF $\Rightarrow$ Seasonal MA(1) component
        \item Pattern suggests: \textbf{SARIMA$(0,1,1)(0,1,1)_{12}$} (airline model)
    \end{itemize}
    }

    \end{cminipage}
    \quantlet{TSA\_ch4\_acf\_pacf}{https://github.com/QuantLet/TSA/tree/main/TSA_ch4/TSA_ch4_acf_pacf}
\end{frame}

\begin{frame}{SARIMA Estimation Results: Airline Data}
    \begin{cminipage}{0.95\textwidth}
    {\small
    \begin{block}{Model: SARIMA$(0,1,1)(0,1,1)_{12}$ on $\log(\text{Passengers})$}
        \begin{center}
        \begin{tabular}{lcccc}
            \toprule
            \textbf{Parameter} & \textbf{Estimate} & \textbf{Std. Error} & \textbf{z-stat} & \textbf{p-value} \\
            \midrule
            $\theta_1$ (MA.L1) & $-0.4018$ & $0.0896$ & $-4.48$ & $<0.001$ \\
            $\Theta_1$ (MA.S.L12) & $-0.5569$ & $0.0731$ & $-7.62$ & $<0.001$ \\
            $\sigma^2$ & $0.00135$ & -- & -- & -- \\
            \bottomrule
        \end{tabular}
        \end{center}
    \end{block}

    \vspace{0.2cm}

    \begin{exampleblock}{Model Fit Statistics}
        \begin{itemize}\setlength{\itemsep}{0pt}
            \item Log-Likelihood: $244.70$
            \item AIC: $-483.40$, BIC: $-474.53$
            \item Both MA coefficients significant and within invertibility bounds
        \end{itemize}
    \end{exampleblock}
    }
    \end{cminipage}
\end{frame}

\begin{frame}{Forecast: 24 Months Ahead}
    \begin{cminipage}{0.95\textwidth}
    \vspace{-0.3cm}
    \begin{center}
        \includegraphics[width=0.85\textwidth, height=0.55\textheight, keepaspectratio]{ch4_sarima_forecast.pdf}
    \end{center}
    \vspace{-0.2cm}
    {\footnotesize
    \begin{itemize}\setlength{\itemsep}{0pt}
        \item Forecasts capture both trend and seasonal pattern
        \item 95\% confidence intervals widen over forecast horizon
        \item Seasonal peaks (July-August) and troughs (February) clearly visible
        \item Model successfully extrapolates the multiplicative seasonal pattern
    \end{itemize}
    }

    \end{cminipage}
    \quantlet{TSA\_ch4\_sarima\_forecast}{https://github.com/QuantLet/TSA/tree/main/TSA_ch4/TSA_ch4_sarima_forecast}
\end{frame}

\begin{frame}{Model Diagnostics}
    \begin{cminipage}{0.95\textwidth}
    \vspace{-0.3cm}
    \begin{center}
        \includegraphics[width=0.85\textwidth, height=0.55\textheight, keepaspectratio]{ch4_diagnostics.pdf}
    \end{center}
    \vspace{-0.2cm}
    {\footnotesize
    \begin{itemize}\setlength{\itemsep}{0pt}
        \item Residuals appear random with no systematic patterns
        \item Distribution approximately normal (Q-Q plot close to diagonal)
        \item ACF of residuals within confidence bounds -- no significant autocorrelation
        \item Ljung-Box test: $p > 0.05$ at all tested lags $\Rightarrow$ adequate model
    \end{itemize}
    }

    \end{cminipage}
    \quantlet{TSA\_ch4\_diagnostics}{https://github.com/QuantLet/TSA/tree/main/TSA_ch4/TSA_ch4_diagnostics}
\end{frame}

%=============================================================================
% DISCUSSION TOPICS
%=============================================================================
\section{Discussion Topics}

\begin{frame}{Discussion: Deterministic vs Stochastic Seasonality}
    \begin{cminipage}{0.95\textwidth}
    \begin{block}{Key Question}
        When should you use seasonal dummies vs SARIMA for seasonal data?
    \end{block}

    \vspace{0.2cm}

    \begin{block}{Considerations}
        \textbf{Seasonal dummies} (deterministic):
        \begin{itemize}\setlength{\itemsep}{0pt}
            \item Fixed, repeating pattern each year
            \item Same December effect every year
            \item Appropriate when seasonality is stable
        \end{itemize}

        \vspace{0.2cm}
        \textbf{SARIMA} (stochastic):
        \begin{itemize}\setlength{\itemsep}{0pt}
            \item Evolving seasonal pattern
            \item This year's December depends on last year's December
            \item Better when seasonal amplitude varies
        \end{itemize}
    \end{block}
    \end{cminipage}
\end{frame}

\begin{frame}{Discussion: Log Transformation}
    \begin{cminipage}{0.95\textwidth}
    \begin{block}{Key Question}
        When should you take logarithms before fitting SARIMA?
    \end{block}

    \vspace{0.05cm}

    \begin{block}{Guidelines}
        {\small
        \textbf{Use log transformation when}:
        \begin{itemize}\setlength{\itemsep}{0pt}
            \item Seasonal fluctuations grow with the level (multiplicative seasonality)
            \item Variance increases over time
            \item Data is strictly positive (prices, sales, counts)
        \end{itemize}

        \vspace{0.05cm}
        \textbf{Avoid log when}:
        \begin{itemize}\setlength{\itemsep}{0pt}
            \item Seasonal pattern is additive (constant amplitude)
            \item Data contains zeros or negatives
            \item Already on a rate/ratio scale
        \end{itemize}

        \vspace{0.05cm}
        \textbf{Tip}: Compare AIC of models with and without log transformation.
        }
    \end{block}
    \end{cminipage}
\end{frame}

\begin{frame}{Discussion: Multiple Seasonalities}
    \begin{cminipage}{0.95\textwidth}
    \begin{block}{Challenge}
        Daily sales data may have both weekly (7-day) and annual (365-day) seasonal patterns. How do you handle this?
    \end{block}

    \vspace{0.2cm}

    \begin{block}{Approaches}
        \begin{enumerate}
            \item \textbf{Nested SARIMA}: Model at shorter frequency, include longer as exogenous
            \item \textbf{TBATS/BATS models}: Explicitly handle multiple seasonalities
            \item \textbf{Fourier terms}: Add sin/cos terms for each seasonal frequency
            \item \textbf{Prophet/similar}: Modern tools designed for multiple seasonalities
        \end{enumerate}

        \vspace{0.2cm}
        \textbf{Note}: Standard SARIMA handles only one seasonal period. For complex seasonality, consider specialized methods.
    \end{block}
    \end{cminipage}
\end{frame}

\begin{frame}{Discussion: Forecasting Seasonal Data}
    \begin{cminipage}{0.95\textwidth}
    \begin{block}{Key Question}
        What are the unique challenges of forecasting seasonal time series?
    \end{block}

    \vspace{0.2cm}

    \begin{block}{Challenges and Solutions}
        \begin{itemize}\setlength{\itemsep}{0pt}
            \item \textbf{Horizon matters}: 12-month forecast means predicting a full cycle
            \item \textbf{Uncertainty grows}: Seasonal forecasts compound regular uncertainty
            \item \textbf{Turning points}: Capturing when seasons peak/trough
            \item \textbf{Structural breaks}: COVID-19 disrupted many seasonal patterns
        \end{itemize}

        \vspace{0.2cm}
        \textbf{Best practices}:
        \begin{itemize}\setlength{\itemsep}{0pt}
            \item Use rolling-origin cross-validation
            \item Compare against seasonal naive benchmark
            \item Report forecast intervals, especially at seasonal horizons
        \end{itemize}
    \end{block}
    \end{cminipage}
\end{frame}

\begin{frame}{Take-Home Exercises}
    \begin{cminipage}{0.95\textwidth}
    {\small
    \begin{enumerate}
        \item \textbf{Theoretical}: Show that $(1-L)(1-L^4)$ can be written as $(1 - L - L^4 + L^5)$ and explain what this transformation does to quarterly data with annual seasonality.

        \vspace{0.2cm}
        \item \textbf{Computation}: For SARIMA$(1,0,0) \times (1,0,0)_4$ with $\phi_1 = 0.5$ and $\Phi_1 = 0.8$, write out the full AR polynomial and identify all non-zero coefficients.

        \vspace{0.2cm}
        \item \textbf{Applied}: Download monthly airline passenger data and:
            \begin{itemize}\setlength{\itemsep}{0pt}
                \item Plot the series and identify trend/seasonality
                \item Apply appropriate transformations
                \item Fit the airline model and interpret coefficients
                \item Generate 24-month forecasts with confidence intervals
            \end{itemize}

        \vspace{0.2cm}
        \item \textbf{Comparison}: Fit both SARIMA$(0,1,1) \times (0,1,1)_{12}$ and SARIMA$(1,1,0) \times (1,1,0)_{12}$ to the airline data. Compare using AIC, BIC, and residual diagnostics. Which is preferred?
    \end{enumerate}
    }
    \end{cminipage}
\end{frame}

\begin{frame}{Exercise Solutions Hints}
    \begin{cminipage}{0.95\textwidth}
    {\small
    \begin{block}{Hints}
        \begin{enumerate}
            \item Expand $(1-L)(1-L^4) = 1 \cdot 1 - 1 \cdot L^4 - L \cdot 1 + L \cdot L^4 = 1 - L - L^4 + L^5$

            \vspace{0.1cm}
            \item AR polynomial: $(1 - \phi_1 L)(1 - \Phi_1 L^4) = 1 - 0.5L - 0.8L^4 + 0.4L^5$

            \vspace{0.1cm}
            \item For airline data:
                \begin{itemize}\setlength{\itemsep}{0pt}
                    \item Use log transformation (multiplicative seasonality)
                    \item Both $d=1$ and $D=1$ needed
                    \item Typical estimates: $\theta_1 \approx -0.4$, $\Theta_1 \approx -0.6$
                \end{itemize}

            \vspace{0.1cm}
            \item The MA-based airline model typically fits better than pure AR seasonal model for this data (lower AIC).
        \end{enumerate}
    \end{block}
    }
    \end{cminipage}
\end{frame}

%=============================================================================
% AI-ASSISTED EXERCISE
%=============================================================================
\section{AI-Assisted Exercise}

\begin{frame}{AI Exercise: Critical Thinking}
    \begin{cminipage}{0.95\textwidth}
    \vspace{-0.3cm}
    \begin{block}{\footnotesize Prompt to test in ChatGPT / Claude / Copilot}
        {\footnotesize
        ``Download the classic airline passenger dataset (Box-Jenkins, 1970). Check if seasonality is multiplicative (log transform if so), apply seasonal differencing, fit SARIMA using AIC, check residual ACF at seasonal lags. Compare against a seasonal naive benchmark using rolling 1-step forecasts on the last 24 observations. Show plots and a comparison table with RMSE and MAPE.''
        }
    \end{block}
    \vspace{-2mm}
    {\footnotesize
    \textbf{Exercise}:
    \begin{enumerate}\setlength{\itemsep}{0pt}
        \item Run the prompt in an LLM of your choice and critically analyze the response.
        \item Did the AI check for multiplicative seasonality and apply log if needed?
        \item Is the seasonal period $s$ correctly identified from the data frequency?
        \item Did it test for both regular and seasonal unit roots before choosing $d$ and $D$?
        \item Are residuals free of seasonal patterns (check ACF at lags $s, 2s, 3s$)?
    \end{enumerate}
    }
    \vspace{-2mm}
    \begin{alertblock}{}
        {\footnotesize \textbf{Warning}: AI-generated code may run without errors and look professional. \textit{That does not mean it is correct.}}
    \end{alertblock}
    \end{cminipage}
\end{frame}

\begin{frame}{AI Exercise: Critique an AI SARIMA Analysis}
    \begin{cminipage}{0.95\textwidth}
    \begin{block}{Scenario}
        You asked an AI: ``Fit the best SARIMA model to this monthly retail sales data.'' It returned:
        \begin{itemize}\setlength{\itemsep}{0pt}
            \item Fitted SARIMA$(2,1,2)(2,1,2)_{12}$ with AIC = 1520
            \item No check for multiplicative vs additive seasonality
            \item Applied $D=1$ ``because the data is monthly''
            \item Ljung-Box p-value = 0.03 on residuals
            \item 24-month forecast with constant-width confidence intervals
        \end{itemize}
    \end{block}

    \vspace{0.3cm}

    \textbf{Your critique:}
    \begin{enumerate}
        \item Why is the model likely over-parameterized? How many parameters does it have?
        \item Why should multiplicative seasonality be checked before fitting?
        \item Why is Ljung-Box p = 0.03 \textbf{not} acceptable at 5\% level?
        \item Why must SARIMA confidence intervals widen (not stay constant)?
    \end{enumerate}
    \end{cminipage}
\end{frame}

\begin{frame}{AI Exercise: Prompt Refinement for SARIMA}
    \begin{cminipage}{0.95\textwidth}
    \begin{block}{Task}
        Iteratively improve prompts for fitting a SARIMA model to monthly airline data.
    \end{block}

    \vspace{0.2cm}

    \textbf{Round 1} (vague): \textit{``Fit a seasonal model to monthly airline passengers''}
    \begin{itemize}\setlength{\itemsep}{0pt}
        \item What did the AI produce? Did it check for multiplicative seasonality?
    \end{itemize}

    \textbf{Round 2} (better): \textit{``Check if seasonality is multiplicative (log transform if so), apply seasonal differencing, fit SARIMA using AIC, check residual ACF at seasonal lags''}
    \begin{itemize}\setlength{\itemsep}{0pt}
        \item Did the AI follow the seasonal Box-Jenkins methodology?
    \end{itemize}

    \textbf{Round 3} (expert): \textit{``(1) Plot series, check if amplitude grows with level, (2) if multiplicative, apply log, (3) test ADF on levels \& seasonal differences, (4) fit airline model + 2 alternatives, (5) compare AIC/BIC, (6) Ljung-Box on residuals, (7) 24-month rolling forecast with RMSE''}
    \begin{itemize}\setlength{\itemsep}{0pt}
        \item Compare results across all three rounds
    \end{itemize}
    \end{cminipage}
\end{frame}

\begin{frame}{AI Exercise: Model Selection Competition}
    \begin{cminipage}{0.95\textwidth}
    {\small
    \begin{block}{Task}
        Download the classic airline passenger dataset (Box-Jenkins, 1970).
    \end{block}

    \vspace{0.05cm}

    \textbf{Your approach (manual):}
    \begin{itemize}\setlength{\itemsep}{0pt}
        \item Check multiplicative vs additive seasonality $\to$ log if needed
        \item Determine $d$ and $D$ from ADF/KPSS and seasonal ACF
        \item ACF/PACF of transformed series $\to$ candidate models
        \item Compare airline model vs SARIMA$(1,1,0)(1,1,0)_{12}$ via AIC/BIC
        \item Rolling 1-step forecast on last 24 observations
    \end{itemize}

    \vspace{0.05cm}

    \textbf{AI approach:}
    \begin{itemize}\setlength{\itemsep}{0pt}
        \item Ask AI to ``find the best SARIMA model and forecast airline passengers''
    \end{itemize}

    \vspace{0.05cm}

    \textbf{Compare:}
    \begin{itemize}\setlength{\itemsep}{0pt}
        \item Did the AI apply log transformation? Which model did each select?
        \item Compare out-of-sample RMSE; did the AI check seasonal residual patterns?
        \item \textbf{Submit:} 1-page reflection on AI strengths and weaknesses
    \end{itemize}
    }
    \end{cminipage}
\end{frame}

%=============================================================================
% SUMMARY
%=============================================================================
\section{Summary}

\begin{frame}{Summary: Chapter 4}
    \begin{cminipage}{0.95\textwidth}
    \begin{exampleblock}{Key Concepts}
        \begin{itemize}\setlength{\itemsep}{2pt}
            \item[\textcolor{MainBlue}{\textbf{1.}}] \textbf{Seasonal differencing}: $(1-L^s)$ removes stochastic seasonality
            \item[\textcolor{MainBlue}{\textbf{2.}}] \textbf{SARIMA notation}: $(p,d,q) \times (P,D,Q)_s$ separates regular and seasonal
            \item[\textcolor{MainBlue}{\textbf{3.}}] \textbf{The airline model}: SARIMA$(0,1,1)(0,1,1)_{12}$ is surprisingly effective
            \item[\textcolor{MainBlue}{\textbf{4.}}] \textbf{Multiplicative structure}: Creates interaction terms between components
            \item[\textcolor{MainBlue}{\textbf{5.}}] \textbf{ACF/PACF diagnostics}: Patterns at both regular and seasonal lags
            \item[\textcolor{MainBlue}{\textbf{6.}}] \textbf{Log transformation}: Often needed for multiplicative seasonality
        \end{itemize}
    \end{exampleblock}

    \vspace{0.5cm}
    \begin{center}
        \Large\textcolor{MainBlue}{Questions?}
    \end{center}
    \end{cminipage}
\end{frame}

\begin{frame}{Key Formulas Summary}
    \begin{cminipage}{0.95\textwidth}
    \small
    \begin{center}
    \begin{tabular}{ll}
        \toprule
        \textbf{Concept} & \textbf{Formula} \\
        \midrule
        Seasonal difference & $(1-L^s)Y_t = Y_t - Y_{t-s}$ \\[0.1cm]
        Combined differencing & $(1-L)(1-L^s)Y_t = Y_t - Y_{t-1} - Y_{t-s} + Y_{t-s-1}$ \\[0.1cm]
        SARIMA$(p,d,q)(P,D,Q)_s$ & $\phi(L)\Phi(L^s)(1-L)^d(1-L^s)^D Y_t = \theta(L)\Theta(L^s)\varepsilon_t$ \\[0.1cm]
        Airline model & $(1-L)(1-L^{12})Y_t = (1+\theta_1 L)(1+\Theta_1 L^{12})\varepsilon_t$ \\[0.1cm]
        \midrule
        Multiplicative AR & $(1-\phi_1 L)(1-\Phi_1 L^s) = 1 - \phi_1 L - \Phi_1 L^s + \phi_1\Phi_1 L^{s+1}$ \\[0.1cm]
        Multiplicative MA & $(1+\theta_1 L)(1+\Theta_1 L^s) = 1 + \theta_1 L + \Theta_1 L^s + \theta_1\Theta_1 L^{s+1}$ \\[0.1cm]
        \midrule
        Invertibility & $|\theta_i| < 1$ and $|\Theta_j| < 1$ for all $i,j$ \\[0.1cm]
        Stationarity & $|\Phi_j| < 1$ and regular AR roots outside unit circle \\[0.1cm]
        \midrule
        Log transform & $\log(T \times S \times \varepsilon) = \log T + \log S + \log \varepsilon$ \\
        \bottomrule
    \end{tabular}
    \end{center}

    {\scriptsize \textbf{Notation:} $s$ = seasonal period, $\phi/\Phi$ = regular/seasonal AR, $\theta/\Theta$ = regular/seasonal MA, $d/D$ = regular/seasonal differencing order}
    \end{cminipage}
\end{frame}

%=============================================================================
% BIBLIOGRAPHY
%=============================================================================
\section{Bibliography}

\begin{frame}{Bibliography I}
    \begin{cminipage}{0.95\textwidth}
    \begin{block}{Time Series Fundamentals}
        {\small
        \begin{itemize}\setlength{\itemsep}{0pt}
            \item Hyndman, R.J., \& Athanasopoulos, G. (2021). \textit{Forecasting: Principles and Practice}, 3rd ed., OTexts.
            \item Shumway, R.H., \& Stoffer, D.S. (2017). \textit{Time Series Analysis and Its Applications}, 4th ed., Springer.
            \item Brockwell, P.J., \& Davis, R.A. (2016). \textit{Introduction to Time Series and Forecasting}, 3rd ed., Springer.
        \end{itemize}
        }
    \end{block}

    \begin{exampleblock}{Financial Time Series}
        {\small
        \begin{itemize}\setlength{\itemsep}{0pt}
            \item Tsay, R.S. (2010). \textit{Analysis of Financial Time Series}, 3rd ed., Wiley.
            \item Franke, J., H\"ardle, W.K., \& Hafner, C.M. (2019). \textit{Statistics of Financial Markets}, 4th ed., Springer.
        \end{itemize}
        }
    \end{exampleblock}
    \end{cminipage}
\end{frame}

\begin{frame}{Bibliography II}
    \begin{cminipage}{0.95\textwidth}
    \begin{block}{Modern Approaches and Machine Learning}
        {\small
        \begin{itemize}\setlength{\itemsep}{0pt}
            \item Nielsen, A. (2019). \textit{Practical Time Series Analysis}, O'Reilly Media.
            \item Petropoulos, F., et al. (2022). \textit{Forecasting: Theory and Practice}, International Journal of Forecasting.
            \item Makridakis, S., Spiliotis, E., \& Assimakopoulos, V. (2020). The M4 Competition, International Journal of Forecasting.
        \end{itemize}
        }
    \end{block}

    \begin{exampleblock}{Online Resources and Code}
        {\small
        \begin{itemize}\setlength{\itemsep}{0pt}
            \item \textbf{Quantlet}: \url{https://quantlet.com} --- Code repository for statistics
            \item \textbf{Quantinar}: \url{https://quantinar.com} --- Quantitative methods learning platform
            \item \textbf{GitHub TSA}: \url{https://github.com/QuantLet/TSA/tree/main/TSA_ch4} --- Python code for this seminar
        \end{itemize}
        }
    \end{exampleblock}
    \end{cminipage}
\end{frame}

\begin{frame}{}
    \begin{cminipage}{0.95\textwidth}
    \centering
    \Huge\textcolor{IDAred}{Thank You!}

    \vspace{1cm}

    \Large\textcolor{MainBlue}{Questions?}

    \vspace{0.8cm}

    \normalsize
    Seminar materials are available at: \url{https://danpele.github.io/Time-Series-Analysis/}

    \vspace{0.2cm}

    \href{https://quantlet.com}{\raisebox{-0.15em}{\includegraphics[height=0.8em]{ql_logo.png}} Quantlet} \hspace{0.5cm}
    \href{https://quantinar.com}{\raisebox{-0.15em}{\includegraphics[height=0.8em]{qr_logo.png}} Quantinar}
    \end{cminipage}
\end{frame}

\end{document}
