% Seminar 0: Fundamentals of Time Series

\documentclass[9pt, aspectratio=169, t]{beamer}
%=============================================================================
% SHARED PREAMBLE - Time Series Analysis and Forecasting
% Harvard-quality academic presentations
% Bachelor program, Bucharest University of Economic Studies
%
% Usage: \documentclass[9pt, aspectratio=169, t]{beamer}
%            %=============================================================================
% SHARED PREAMBLE - Time Series Analysis and Forecasting
% Harvard-quality academic presentations
% Bachelor program, Bucharest University of Economic Studies
%
% Usage: \documentclass[9pt, aspectratio=169, t]{beamer}
%            %=============================================================================
% SHARED PREAMBLE - Time Series Analysis and Forecasting
% Harvard-quality academic presentations
% Bachelor program, Bucharest University of Economic Studies
%
% Usage: \documentclass[9pt, aspectratio=169, t]{beamer}
%            \input{preamble}
%            \subtitle{Seminar X: Seminar Title}
%            \begin{document} ...
%=============================================================================

% Ensure content fits on slides
\setbeamersize{text margin left=8mm, text margin right=8mm}

%=============================================================================
% THEME AND STYLE CONFIGURATION
%=============================================================================
\usetheme{default}
% Using default theme for clean header/footer control

% Color Palette (matching Redispatch PDF)
\definecolor{MainBlue}{RGB}{26, 58, 110}
\definecolor{AccentBlue}{RGB}{26, 58, 110}
\definecolor{IDAred}{RGB}{205, 0, 0}
\definecolor{DarkGray}{RGB}{51, 51, 51}
\definecolor{MediumGray}{RGB}{128, 128, 128}
\definecolor{LightGray}{RGB}{248, 248, 248}
\definecolor{VeryLightGray}{RGB}{235, 235, 235}
\definecolor{KeynoteGray}{RGB}{218, 218, 218}
\definecolor{SectionGray}{RGB}{120, 120, 120}
\definecolor{FooterGray}{RGB}{100, 100, 100}
\definecolor{Crimson}{RGB}{220, 53, 69}
\definecolor{Forest}{RGB}{46, 125, 50}
\definecolor{Amber}{RGB}{181, 133, 63}
\definecolor{Orange}{RGB}{230, 126, 34}
\definecolor{Purple}{RGB}{142, 68, 173}

% Gradient background (exact Keynote 315° gradient: white to RGB 218,218,218)
\setbeamertemplate{background}{%
    \begin{tikzpicture}[remember picture, overlay]
        \shade[shading=axis, shading angle=315,
        top color=white, bottom color=KeynoteGray]
        (current page.south west) rectangle (current page.north east);
    \end{tikzpicture}%
}
% Fallback solid color for compatibility
\setbeamercolor{background canvas}{bg=}

\setbeamercolor{palette primary}{bg=MainBlue, fg=white}
\setbeamercolor{palette secondary}{bg=MainBlue!85, fg=white}
\setbeamercolor{palette tertiary}{bg=MainBlue!70, fg=white}
\setbeamercolor{structure}{fg=MainBlue}
\setbeamercolor{title}{fg=IDAred}
\setbeamercolor{frametitle}{fg=IDAred, bg=}
\setbeamercolor{block title}{bg=MainBlue, fg=white}
\setbeamercolor{block body}{bg=VeryLightGray, fg=DarkGray}
\setbeamercolor{block title alerted}{bg=Crimson, fg=white}
\setbeamercolor{block body alerted}{bg=Crimson!8, fg=DarkGray}
\setbeamercolor{block title example}{bg=Forest, fg=white}
\setbeamercolor{block body example}{bg=Forest!8, fg=DarkGray}
\setbeamercolor{item}{fg=MainBlue}

% Smaller institute font to avoid overfull hbox on title page
\setbeamerfont{institute}{size=\footnotesize}

% Footer colors (override Madrid theme blue)
\setbeamercolor{author in head/foot}{fg=FooterGray, bg=}
\setbeamercolor{title in head/foot}{fg=FooterGray, bg=}
\setbeamercolor{date in head/foot}{fg=FooterGray, bg=}
\setbeamercolor{section in head/foot}{fg=FooterGray, bg=}
\setbeamercolor{subsection in head/foot}{fg=FooterGray, bg=}

% Bullet styles (apply everywhere including blocks)
\setbeamertemplate{itemize item}{\color{MainBlue}$\boxdot$}
\setbeamertemplate{itemize subitem}{\color{MainBlue}$\blacktriangleright$}
\setbeamertemplate{itemize subsubitem}{\color{MainBlue}\tiny$\bullet$}
\setbeamertemplate{itemize/enumerate body begin}{\normalsize}
\setbeamertemplate{itemize/enumerate subbody begin}{\normalsize}

% Item spacing - compact style
\setlength{\leftmargini}{10pt}       % Level 1: minimal indent
\setlength{\leftmarginii}{10pt}      % Level 2: minimal additional indent
% Compact list spacing (zero extra space before/after lists in blocks)
\makeatletter
\def\@listi{\leftmargin\leftmargini \topsep 0pt \parsep 0pt \itemsep 0pt}
\def\@listii{\leftmargin\leftmarginii \topsep 0pt \parsep 0pt \itemsep 0pt}
\makeatother

\setbeamertemplate{navigation symbols}{}

%=============================================================================
% CUSTOM HEADLINE
%=============================================================================
\setbeamertemplate{headline}{%
    \vskip10pt%
    \hbox to \paperwidth{%
        \hskip0.5cm%
        {\small\color{FooterGray}\renewcommand{\hyperlink}[2]{##2}\insertsectionhead}%
        \hfill%
        \textcolor{FooterGray}{\small\insertframenumber}%
        \hskip0.5cm%
    }%
    \vskip4pt%
    {\color{FooterGray}\hrule height 0.4pt}%
}

%=============================================================================
% CUSTOM FOOTER
%=============================================================================
\usepackage{fontawesome5}

\setbeamertemplate{footline}{%
    {\color{FooterGray}\hrule height 0.4pt}%
    \vskip4pt%
    \hbox to \paperwidth{%
        \hskip0.5cm%
        \textcolor{FooterGray}{\small Time Series Analysis and Forecasting}%
        \hfill%
        \raisebox{-0.1em}{%
            \begin{tikzpicture}[x=0.08em, y=0.08em, line width=0.4pt]
                \draw[FooterGray] (0,3) -- (1,4) -- (2,3.5) -- (3,5) -- (4,4) -- (5,6) -- (6,5.5) -- (7,4) -- (8,5) -- (9,7) -- (10,6) -- (11,5) -- (12,6.5) -- (13,8) -- (14,7) -- (15,6) -- (16,7.5) -- (17,9) -- (18,8) -- (19,7) -- (20,8.5) -- (21,10) -- (22,9) -- (23,8) -- (24,9.5);
            \end{tikzpicture}%
        }%
        \hskip0.5cm%
    }%
    \vskip6pt%
}

%=============================================================================
% PACKAGES
%=============================================================================
\usepackage[utf8]{inputenc}
\usepackage[T1]{fontenc}
\usepackage[english]{babel}
\usepackage{amsmath, amssymb, amsthm}
\usepackage{mathtools}
\usepackage{bm}
\usepackage{tikz}
\usetikzlibrary{arrows.meta, positioning, shapes, calc, decorations.pathreplacing, shadings}
\usepackage{booktabs}
\usepackage{multirow}
\usepackage{array}
\usepackage{graphicx}
\usepackage{hyperref}
\usepackage{colortbl}
\usepackage{listings}
\lstset{basicstyle=\ttfamily\small, breaklines=true, frame=single, backgroundcolor=\color{VeryLightGray}}
\hypersetup{colorlinks=true, linkcolor=MainBlue, urlcolor=MainBlue}
\graphicspath{{../../logos/}{../../charts/}{../../photos/}}
\hfuzz=2pt  % Suppress tiny overfull warnings (<2pt)
\vfuzz=2pt  % Suppress tiny vertical overfull warnings (<2pt)

%=============================================================================
% QUANTLET COMMAND
%=============================================================================
\newcommand{\quantlet}[2]{%
    \hfill\href{#2}{%
        \raisebox{-0.15em}{\includegraphics[height=0.7em]{ql_logo.png}}%
        \textcolor{MainBlue}{\tiny\ #1}%
    }%
}

%=============================================================================
% CUSTOM TITLE PAGE
%=============================================================================
\defbeamertemplate*{title page}{hybrid}[1][]
{
    \vspace{0.2cm}
    % Logos row - top header (with clickable links)
    \begin{center}
        \href{https://www.ase.ro}{\includegraphics[height=1.0cm]{ase_logo.png}}\hspace{0.25cm}%
        \href{https://theida.net}{\includegraphics[height=1.0cm]{ida_logo.png}}\hspace{0.25cm}%
        \href{https://blockchain-research-center.com}{\includegraphics[height=1.0cm]{brc_logo.png}}\hspace{0.25cm}%
        \href{https://www.ai4efin.ase.ro}{\includegraphics[height=1.0cm]{ai4efin_logo.png}}\hspace{0.25cm}%
        \href{https://ipe.ro/new}{\includegraphics[height=1.0cm]{acad_logo.png}}\hspace{0.25cm}%
        \href{https://www.digital-finance-msca.com}{\includegraphics[height=1.0cm]{msca_logo.png}}%
    \end{center}

    \vspace{0.6cm}

    % Main title with Q logos on sides (with clickable links)
    \begin{center}
        \begin{minipage}{0.1\textwidth}
            \centering
            \href{https://quantlet.com}{\includegraphics[height=1.1cm]{ql_logo.png}}
        \end{minipage}%
        \begin{minipage}{0.78\textwidth}
            \centering
            {\LARGE\bfseries\usebeamercolor[fg]{title}\inserttitle}

            \vspace{0.3cm}

            {\usebeamerfont{subtitle}\usebeamercolor[fg]{title}\insertsubtitle}
        \end{minipage}%
        \begin{minipage}{0.1\textwidth}
            \centering
            \href{https://quantinar.com}{\includegraphics[height=1.1cm]{qr_logo.png}}
        \end{minipage}
    \end{center}

    \vspace{0.6cm}

    % Authors (left aligned)
    \hspace{0.5cm}{\usebeamerfont{author}\insertauthor}

    \vspace{0.3cm}

    % Institute/Affiliations (left aligned)
    \hspace{0.5cm}\begin{minipage}[t]{0.9\textwidth}
        \raggedright\small\insertinstitute
    \end{minipage}
}

%=============================================================================
% THEOREM ENVIRONMENTS
%=============================================================================
\theoremstyle{definition}
\setbeamertemplate{theorems}[numbered]
\newtheorem{defn}{Definition}
\newtheorem{thm}{Theorem}
\newtheorem{prop}{Proposition}
\newtheorem{rmk}{Remark}

%=============================================================================
% CENTRED MINIPAGE (no extra vertical space)
%=============================================================================
\newenvironment{cminipage}[1]{%
    \par\noindent\hfill\begin{minipage}{#1}\ignorespaces
}{%
    \end{minipage}\hfill\null\par
}

%=============================================================================
% CUSTOM COMMANDS
%=============================================================================
\newcommand{\E}{\mathbb{E}}
\newcommand{\Var}{\text{Var}}
\newcommand{\Cov}{\text{Cov}}
\newcommand{\Corr}{\text{Corr}}
\newcommand{\R}{\mathbb{R}}
\newcommand{\N}{\mathbb{N}}
\newcommand{\Z}{\mathbb{Z}}
\newcommand{\B}{\mathbf{B}}
\newcommand{\imark}{\textcolor{MainBlue}{\textbullet}}
\newcommand{\RMSE}{\text{RMSE}}
\newcommand{\MAE}{\text{MAE}}
\newcommand{\MAPE}{\text{MAPE}}
\newcommand{\correct}{\textcolor{Forest}{\checkmark}}
\newcommand{\incorrect}{\textcolor{Crimson}{\texttimes}}

% Boldface vector/matrix commands
\newcommand{\bY}{\mathbf{Y}}
\newcommand{\bX}{\mathbf{X}}
\newcommand{\bA}{\mathbf{A}}
\newcommand{\bB}{\mathbf{B}}
\newcommand{\bepsilon}{\boldsymbol{\varepsilon}}
\newcommand{\bvarepsilon}{\boldsymbol{\varepsilon}}
\newcommand{\bSigma}{\boldsymbol{\Sigma}}
\newcommand{\bPhi}{\boldsymbol{\Phi}}
\newcommand{\bGamma}{\boldsymbol{\Gamma}}
\newcommand{\bPi}{\boldsymbol{\Pi}}
\newcommand{\bc}{\mathbf{c}}
\newcommand{\balpha}{\boldsymbol{\alpha}}
\newcommand{\bbeta}{\boldsymbol{\beta}}

%=============================================================================
% TITLE INFORMATION
%=============================================================================
\title[Time Series Analysis]{Time Series Analysis and Forecasting}
\author[D.T. Pele]{Daniel Traian PELE}
\institute{Bucharest University of Economic Studies\\
IDA Institute Digital Assets\\
Blockchain Research Center\\
AI4EFin Artificial Intelligence for Energy Finance\\
Romanian Academy, Institute for Economic Forecasting\\
MSCA Digital Finance}
\date{}

%            \subtitle{Seminar X: Seminar Title}
%            \begin{document} ...
%=============================================================================

% Ensure content fits on slides
\setbeamersize{text margin left=8mm, text margin right=8mm}

%=============================================================================
% THEME AND STYLE CONFIGURATION
%=============================================================================
\usetheme{default}
% Using default theme for clean header/footer control

% Color Palette (matching Redispatch PDF)
\definecolor{MainBlue}{RGB}{26, 58, 110}
\definecolor{AccentBlue}{RGB}{26, 58, 110}
\definecolor{IDAred}{RGB}{205, 0, 0}
\definecolor{DarkGray}{RGB}{51, 51, 51}
\definecolor{MediumGray}{RGB}{128, 128, 128}
\definecolor{LightGray}{RGB}{248, 248, 248}
\definecolor{VeryLightGray}{RGB}{235, 235, 235}
\definecolor{KeynoteGray}{RGB}{218, 218, 218}
\definecolor{SectionGray}{RGB}{120, 120, 120}
\definecolor{FooterGray}{RGB}{100, 100, 100}
\definecolor{Crimson}{RGB}{220, 53, 69}
\definecolor{Forest}{RGB}{46, 125, 50}
\definecolor{Amber}{RGB}{181, 133, 63}
\definecolor{Orange}{RGB}{230, 126, 34}
\definecolor{Purple}{RGB}{142, 68, 173}

% Gradient background (exact Keynote 315° gradient: white to RGB 218,218,218)
\setbeamertemplate{background}{%
    \begin{tikzpicture}[remember picture, overlay]
        \shade[shading=axis, shading angle=315,
        top color=white, bottom color=KeynoteGray]
        (current page.south west) rectangle (current page.north east);
    \end{tikzpicture}%
}
% Fallback solid color for compatibility
\setbeamercolor{background canvas}{bg=}

\setbeamercolor{palette primary}{bg=MainBlue, fg=white}
\setbeamercolor{palette secondary}{bg=MainBlue!85, fg=white}
\setbeamercolor{palette tertiary}{bg=MainBlue!70, fg=white}
\setbeamercolor{structure}{fg=MainBlue}
\setbeamercolor{title}{fg=IDAred}
\setbeamercolor{frametitle}{fg=IDAred, bg=}
\setbeamercolor{block title}{bg=MainBlue, fg=white}
\setbeamercolor{block body}{bg=VeryLightGray, fg=DarkGray}
\setbeamercolor{block title alerted}{bg=Crimson, fg=white}
\setbeamercolor{block body alerted}{bg=Crimson!8, fg=DarkGray}
\setbeamercolor{block title example}{bg=Forest, fg=white}
\setbeamercolor{block body example}{bg=Forest!8, fg=DarkGray}
\setbeamercolor{item}{fg=MainBlue}

% Smaller institute font to avoid overfull hbox on title page
\setbeamerfont{institute}{size=\footnotesize}

% Footer colors (override Madrid theme blue)
\setbeamercolor{author in head/foot}{fg=FooterGray, bg=}
\setbeamercolor{title in head/foot}{fg=FooterGray, bg=}
\setbeamercolor{date in head/foot}{fg=FooterGray, bg=}
\setbeamercolor{section in head/foot}{fg=FooterGray, bg=}
\setbeamercolor{subsection in head/foot}{fg=FooterGray, bg=}

% Bullet styles (apply everywhere including blocks)
\setbeamertemplate{itemize item}{\color{MainBlue}$\boxdot$}
\setbeamertemplate{itemize subitem}{\color{MainBlue}$\blacktriangleright$}
\setbeamertemplate{itemize subsubitem}{\color{MainBlue}\tiny$\bullet$}
\setbeamertemplate{itemize/enumerate body begin}{\normalsize}
\setbeamertemplate{itemize/enumerate subbody begin}{\normalsize}

% Item spacing - compact style
\setlength{\leftmargini}{10pt}       % Level 1: minimal indent
\setlength{\leftmarginii}{10pt}      % Level 2: minimal additional indent
% Compact list spacing (zero extra space before/after lists in blocks)
\makeatletter
\def\@listi{\leftmargin\leftmargini \topsep 0pt \parsep 0pt \itemsep 0pt}
\def\@listii{\leftmargin\leftmarginii \topsep 0pt \parsep 0pt \itemsep 0pt}
\makeatother

\setbeamertemplate{navigation symbols}{}

%=============================================================================
% CUSTOM HEADLINE
%=============================================================================
\setbeamertemplate{headline}{%
    \vskip10pt%
    \hbox to \paperwidth{%
        \hskip0.5cm%
        {\small\color{FooterGray}\renewcommand{\hyperlink}[2]{##2}\insertsectionhead}%
        \hfill%
        \textcolor{FooterGray}{\small\insertframenumber}%
        \hskip0.5cm%
    }%
    \vskip4pt%
    {\color{FooterGray}\hrule height 0.4pt}%
}

%=============================================================================
% CUSTOM FOOTER
%=============================================================================
\usepackage{fontawesome5}

\setbeamertemplate{footline}{%
    {\color{FooterGray}\hrule height 0.4pt}%
    \vskip4pt%
    \hbox to \paperwidth{%
        \hskip0.5cm%
        \textcolor{FooterGray}{\small Time Series Analysis and Forecasting}%
        \hfill%
        \raisebox{-0.1em}{%
            \begin{tikzpicture}[x=0.08em, y=0.08em, line width=0.4pt]
                \draw[FooterGray] (0,3) -- (1,4) -- (2,3.5) -- (3,5) -- (4,4) -- (5,6) -- (6,5.5) -- (7,4) -- (8,5) -- (9,7) -- (10,6) -- (11,5) -- (12,6.5) -- (13,8) -- (14,7) -- (15,6) -- (16,7.5) -- (17,9) -- (18,8) -- (19,7) -- (20,8.5) -- (21,10) -- (22,9) -- (23,8) -- (24,9.5);
            \end{tikzpicture}%
        }%
        \hskip0.5cm%
    }%
    \vskip6pt%
}

%=============================================================================
% PACKAGES
%=============================================================================
\usepackage[utf8]{inputenc}
\usepackage[T1]{fontenc}
\usepackage[english]{babel}
\usepackage{amsmath, amssymb, amsthm}
\usepackage{mathtools}
\usepackage{bm}
\usepackage{tikz}
\usetikzlibrary{arrows.meta, positioning, shapes, calc, decorations.pathreplacing, shadings}
\usepackage{booktabs}
\usepackage{multirow}
\usepackage{array}
\usepackage{graphicx}
\usepackage{hyperref}
\usepackage{colortbl}
\usepackage{listings}
\lstset{basicstyle=\ttfamily\small, breaklines=true, frame=single, backgroundcolor=\color{VeryLightGray}}
\hypersetup{colorlinks=true, linkcolor=MainBlue, urlcolor=MainBlue}
\graphicspath{{../../logos/}{../../charts/}{../../photos/}}
\hfuzz=2pt  % Suppress tiny overfull warnings (<2pt)
\vfuzz=2pt  % Suppress tiny vertical overfull warnings (<2pt)

%=============================================================================
% QUANTLET COMMAND
%=============================================================================
\newcommand{\quantlet}[2]{%
    \hfill\href{#2}{%
        \raisebox{-0.15em}{\includegraphics[height=0.7em]{ql_logo.png}}%
        \textcolor{MainBlue}{\tiny\ #1}%
    }%
}

%=============================================================================
% CUSTOM TITLE PAGE
%=============================================================================
\defbeamertemplate*{title page}{hybrid}[1][]
{
    \vspace{0.2cm}
    % Logos row - top header (with clickable links)
    \begin{center}
        \href{https://www.ase.ro}{\includegraphics[height=1.0cm]{ase_logo.png}}\hspace{0.25cm}%
        \href{https://theida.net}{\includegraphics[height=1.0cm]{ida_logo.png}}\hspace{0.25cm}%
        \href{https://blockchain-research-center.com}{\includegraphics[height=1.0cm]{brc_logo.png}}\hspace{0.25cm}%
        \href{https://www.ai4efin.ase.ro}{\includegraphics[height=1.0cm]{ai4efin_logo.png}}\hspace{0.25cm}%
        \href{https://ipe.ro/new}{\includegraphics[height=1.0cm]{acad_logo.png}}\hspace{0.25cm}%
        \href{https://www.digital-finance-msca.com}{\includegraphics[height=1.0cm]{msca_logo.png}}%
    \end{center}

    \vspace{0.6cm}

    % Main title with Q logos on sides (with clickable links)
    \begin{center}
        \begin{minipage}{0.1\textwidth}
            \centering
            \href{https://quantlet.com}{\includegraphics[height=1.1cm]{ql_logo.png}}
        \end{minipage}%
        \begin{minipage}{0.78\textwidth}
            \centering
            {\LARGE\bfseries\usebeamercolor[fg]{title}\inserttitle}

            \vspace{0.3cm}

            {\usebeamerfont{subtitle}\usebeamercolor[fg]{title}\insertsubtitle}
        \end{minipage}%
        \begin{minipage}{0.1\textwidth}
            \centering
            \href{https://quantinar.com}{\includegraphics[height=1.1cm]{qr_logo.png}}
        \end{minipage}
    \end{center}

    \vspace{0.6cm}

    % Authors (left aligned)
    \hspace{0.5cm}{\usebeamerfont{author}\insertauthor}

    \vspace{0.3cm}

    % Institute/Affiliations (left aligned)
    \hspace{0.5cm}\begin{minipage}[t]{0.9\textwidth}
        \raggedright\small\insertinstitute
    \end{minipage}
}

%=============================================================================
% THEOREM ENVIRONMENTS
%=============================================================================
\theoremstyle{definition}
\setbeamertemplate{theorems}[numbered]
\newtheorem{defn}{Definition}
\newtheorem{thm}{Theorem}
\newtheorem{prop}{Proposition}
\newtheorem{rmk}{Remark}

%=============================================================================
% CENTRED MINIPAGE (no extra vertical space)
%=============================================================================
\newenvironment{cminipage}[1]{%
    \par\noindent\hfill\begin{minipage}{#1}\ignorespaces
}{%
    \end{minipage}\hfill\null\par
}

%=============================================================================
% CUSTOM COMMANDS
%=============================================================================
\newcommand{\E}{\mathbb{E}}
\newcommand{\Var}{\text{Var}}
\newcommand{\Cov}{\text{Cov}}
\newcommand{\Corr}{\text{Corr}}
\newcommand{\R}{\mathbb{R}}
\newcommand{\N}{\mathbb{N}}
\newcommand{\Z}{\mathbb{Z}}
\newcommand{\B}{\mathbf{B}}
\newcommand{\imark}{\textcolor{MainBlue}{\textbullet}}
\newcommand{\RMSE}{\text{RMSE}}
\newcommand{\MAE}{\text{MAE}}
\newcommand{\MAPE}{\text{MAPE}}
\newcommand{\correct}{\textcolor{Forest}{\checkmark}}
\newcommand{\incorrect}{\textcolor{Crimson}{\texttimes}}

% Boldface vector/matrix commands
\newcommand{\bY}{\mathbf{Y}}
\newcommand{\bX}{\mathbf{X}}
\newcommand{\bA}{\mathbf{A}}
\newcommand{\bB}{\mathbf{B}}
\newcommand{\bepsilon}{\boldsymbol{\varepsilon}}
\newcommand{\bvarepsilon}{\boldsymbol{\varepsilon}}
\newcommand{\bSigma}{\boldsymbol{\Sigma}}
\newcommand{\bPhi}{\boldsymbol{\Phi}}
\newcommand{\bGamma}{\boldsymbol{\Gamma}}
\newcommand{\bPi}{\boldsymbol{\Pi}}
\newcommand{\bc}{\mathbf{c}}
\newcommand{\balpha}{\boldsymbol{\alpha}}
\newcommand{\bbeta}{\boldsymbol{\beta}}

%=============================================================================
% TITLE INFORMATION
%=============================================================================
\title[Time Series Analysis]{Time Series Analysis and Forecasting}
\author[D.T. Pele]{Daniel Traian PELE}
\institute{Bucharest University of Economic Studies\\
IDA Institute Digital Assets\\
Blockchain Research Center\\
AI4EFin Artificial Intelligence for Energy Finance\\
Romanian Academy, Institute for Economic Forecasting\\
MSCA Digital Finance}
\date{}

%            \subtitle{Seminar X: Seminar Title}
%            \begin{document} ...
%=============================================================================

% Ensure content fits on slides
\setbeamersize{text margin left=8mm, text margin right=8mm}

%=============================================================================
% THEME AND STYLE CONFIGURATION
%=============================================================================
\usetheme{default}
% Using default theme for clean header/footer control

% Color Palette (matching Redispatch PDF)
\definecolor{MainBlue}{RGB}{26, 58, 110}
\definecolor{AccentBlue}{RGB}{26, 58, 110}
\definecolor{IDAred}{RGB}{205, 0, 0}
\definecolor{DarkGray}{RGB}{51, 51, 51}
\definecolor{MediumGray}{RGB}{128, 128, 128}
\definecolor{LightGray}{RGB}{248, 248, 248}
\definecolor{VeryLightGray}{RGB}{235, 235, 235}
\definecolor{KeynoteGray}{RGB}{218, 218, 218}
\definecolor{SectionGray}{RGB}{120, 120, 120}
\definecolor{FooterGray}{RGB}{100, 100, 100}
\definecolor{Crimson}{RGB}{220, 53, 69}
\definecolor{Forest}{RGB}{46, 125, 50}
\definecolor{Amber}{RGB}{181, 133, 63}
\definecolor{Orange}{RGB}{230, 126, 34}
\definecolor{Purple}{RGB}{142, 68, 173}

% Gradient background (exact Keynote 315° gradient: white to RGB 218,218,218)
\setbeamertemplate{background}{%
    \begin{tikzpicture}[remember picture, overlay]
        \shade[shading=axis, shading angle=315,
        top color=white, bottom color=KeynoteGray]
        (current page.south west) rectangle (current page.north east);
    \end{tikzpicture}%
}
% Fallback solid color for compatibility
\setbeamercolor{background canvas}{bg=}

\setbeamercolor{palette primary}{bg=MainBlue, fg=white}
\setbeamercolor{palette secondary}{bg=MainBlue!85, fg=white}
\setbeamercolor{palette tertiary}{bg=MainBlue!70, fg=white}
\setbeamercolor{structure}{fg=MainBlue}
\setbeamercolor{title}{fg=IDAred}
\setbeamercolor{frametitle}{fg=IDAred, bg=}
\setbeamercolor{block title}{bg=MainBlue, fg=white}
\setbeamercolor{block body}{bg=VeryLightGray, fg=DarkGray}
\setbeamercolor{block title alerted}{bg=Crimson, fg=white}
\setbeamercolor{block body alerted}{bg=Crimson!8, fg=DarkGray}
\setbeamercolor{block title example}{bg=Forest, fg=white}
\setbeamercolor{block body example}{bg=Forest!8, fg=DarkGray}
\setbeamercolor{item}{fg=MainBlue}

% Smaller institute font to avoid overfull hbox on title page
\setbeamerfont{institute}{size=\footnotesize}

% Footer colors (override Madrid theme blue)
\setbeamercolor{author in head/foot}{fg=FooterGray, bg=}
\setbeamercolor{title in head/foot}{fg=FooterGray, bg=}
\setbeamercolor{date in head/foot}{fg=FooterGray, bg=}
\setbeamercolor{section in head/foot}{fg=FooterGray, bg=}
\setbeamercolor{subsection in head/foot}{fg=FooterGray, bg=}

% Bullet styles (apply everywhere including blocks)
\setbeamertemplate{itemize item}{\color{MainBlue}$\boxdot$}
\setbeamertemplate{itemize subitem}{\color{MainBlue}$\blacktriangleright$}
\setbeamertemplate{itemize subsubitem}{\color{MainBlue}\tiny$\bullet$}
\setbeamertemplate{itemize/enumerate body begin}{\normalsize}
\setbeamertemplate{itemize/enumerate subbody begin}{\normalsize}

% Item spacing - compact style
\setlength{\leftmargini}{10pt}       % Level 1: minimal indent
\setlength{\leftmarginii}{10pt}      % Level 2: minimal additional indent
% Compact list spacing (zero extra space before/after lists in blocks)
\makeatletter
\def\@listi{\leftmargin\leftmargini \topsep 0pt \parsep 0pt \itemsep 0pt}
\def\@listii{\leftmargin\leftmarginii \topsep 0pt \parsep 0pt \itemsep 0pt}
\makeatother

\setbeamertemplate{navigation symbols}{}

%=============================================================================
% CUSTOM HEADLINE
%=============================================================================
\setbeamertemplate{headline}{%
    \vskip10pt%
    \hbox to \paperwidth{%
        \hskip0.5cm%
        {\small\color{FooterGray}\renewcommand{\hyperlink}[2]{##2}\insertsectionhead}%
        \hfill%
        \textcolor{FooterGray}{\small\insertframenumber}%
        \hskip0.5cm%
    }%
    \vskip4pt%
    {\color{FooterGray}\hrule height 0.4pt}%
}

%=============================================================================
% CUSTOM FOOTER
%=============================================================================
\usepackage{fontawesome5}

\setbeamertemplate{footline}{%
    {\color{FooterGray}\hrule height 0.4pt}%
    \vskip4pt%
    \hbox to \paperwidth{%
        \hskip0.5cm%
        \textcolor{FooterGray}{\small Time Series Analysis and Forecasting}%
        \hfill%
        \raisebox{-0.1em}{%
            \begin{tikzpicture}[x=0.08em, y=0.08em, line width=0.4pt]
                \draw[FooterGray] (0,3) -- (1,4) -- (2,3.5) -- (3,5) -- (4,4) -- (5,6) -- (6,5.5) -- (7,4) -- (8,5) -- (9,7) -- (10,6) -- (11,5) -- (12,6.5) -- (13,8) -- (14,7) -- (15,6) -- (16,7.5) -- (17,9) -- (18,8) -- (19,7) -- (20,8.5) -- (21,10) -- (22,9) -- (23,8) -- (24,9.5);
            \end{tikzpicture}%
        }%
        \hskip0.5cm%
    }%
    \vskip6pt%
}

%=============================================================================
% PACKAGES
%=============================================================================
\usepackage[utf8]{inputenc}
\usepackage[T1]{fontenc}
\usepackage[english]{babel}
\usepackage{amsmath, amssymb, amsthm}
\usepackage{mathtools}
\usepackage{bm}
\usepackage{tikz}
\usetikzlibrary{arrows.meta, positioning, shapes, calc, decorations.pathreplacing, shadings}
\usepackage{booktabs}
\usepackage{multirow}
\usepackage{array}
\usepackage{graphicx}
\usepackage{hyperref}
\usepackage{colortbl}
\usepackage{listings}
\lstset{basicstyle=\ttfamily\small, breaklines=true, frame=single, backgroundcolor=\color{VeryLightGray}}
\hypersetup{colorlinks=true, linkcolor=MainBlue, urlcolor=MainBlue}
\graphicspath{{../../logos/}{../../charts/}{../../photos/}}
\hfuzz=2pt  % Suppress tiny overfull warnings (<2pt)
\vfuzz=2pt  % Suppress tiny vertical overfull warnings (<2pt)

%=============================================================================
% QUANTLET COMMAND
%=============================================================================
\newcommand{\quantlet}[2]{%
    \hfill\href{#2}{%
        \raisebox{-0.15em}{\includegraphics[height=0.7em]{ql_logo.png}}%
        \textcolor{MainBlue}{\tiny\ #1}%
    }%
}

%=============================================================================
% CUSTOM TITLE PAGE
%=============================================================================
\defbeamertemplate*{title page}{hybrid}[1][]
{
    \vspace{0.2cm}
    % Logos row - top header (with clickable links)
    \begin{center}
        \href{https://www.ase.ro}{\includegraphics[height=1.0cm]{ase_logo.png}}\hspace{0.25cm}%
        \href{https://theida.net}{\includegraphics[height=1.0cm]{ida_logo.png}}\hspace{0.25cm}%
        \href{https://blockchain-research-center.com}{\includegraphics[height=1.0cm]{brc_logo.png}}\hspace{0.25cm}%
        \href{https://www.ai4efin.ase.ro}{\includegraphics[height=1.0cm]{ai4efin_logo.png}}\hspace{0.25cm}%
        \href{https://ipe.ro/new}{\includegraphics[height=1.0cm]{acad_logo.png}}\hspace{0.25cm}%
        \href{https://www.digital-finance-msca.com}{\includegraphics[height=1.0cm]{msca_logo.png}}%
    \end{center}

    \vspace{0.6cm}

    % Main title with Q logos on sides (with clickable links)
    \begin{center}
        \begin{minipage}{0.1\textwidth}
            \centering
            \href{https://quantlet.com}{\includegraphics[height=1.1cm]{ql_logo.png}}
        \end{minipage}%
        \begin{minipage}{0.78\textwidth}
            \centering
            {\LARGE\bfseries\usebeamercolor[fg]{title}\inserttitle}

            \vspace{0.3cm}

            {\usebeamerfont{subtitle}\usebeamercolor[fg]{title}\insertsubtitle}
        \end{minipage}%
        \begin{minipage}{0.1\textwidth}
            \centering
            \href{https://quantinar.com}{\includegraphics[height=1.1cm]{qr_logo.png}}
        \end{minipage}
    \end{center}

    \vspace{0.6cm}

    % Authors (left aligned)
    \hspace{0.5cm}{\usebeamerfont{author}\insertauthor}

    \vspace{0.3cm}

    % Institute/Affiliations (left aligned)
    \hspace{0.5cm}\begin{minipage}[t]{0.9\textwidth}
        \raggedright\small\insertinstitute
    \end{minipage}
}

%=============================================================================
% THEOREM ENVIRONMENTS
%=============================================================================
\theoremstyle{definition}
\setbeamertemplate{theorems}[numbered]
\newtheorem{defn}{Definition}
\newtheorem{thm}{Theorem}
\newtheorem{prop}{Proposition}
\newtheorem{rmk}{Remark}

%=============================================================================
% CENTRED MINIPAGE (no extra vertical space)
%=============================================================================
\newenvironment{cminipage}[1]{%
    \par\noindent\hfill\begin{minipage}{#1}\ignorespaces
}{%
    \end{minipage}\hfill\null\par
}

%=============================================================================
% CUSTOM COMMANDS
%=============================================================================
\newcommand{\E}{\mathbb{E}}
\newcommand{\Var}{\text{Var}}
\newcommand{\Cov}{\text{Cov}}
\newcommand{\Corr}{\text{Corr}}
\newcommand{\R}{\mathbb{R}}
\newcommand{\N}{\mathbb{N}}
\newcommand{\Z}{\mathbb{Z}}
\newcommand{\B}{\mathbf{B}}
\newcommand{\imark}{\textcolor{MainBlue}{\textbullet}}
\newcommand{\RMSE}{\text{RMSE}}
\newcommand{\MAE}{\text{MAE}}
\newcommand{\MAPE}{\text{MAPE}}
\newcommand{\correct}{\textcolor{Forest}{\checkmark}}
\newcommand{\incorrect}{\textcolor{Crimson}{\texttimes}}

% Boldface vector/matrix commands
\newcommand{\bY}{\mathbf{Y}}
\newcommand{\bX}{\mathbf{X}}
\newcommand{\bA}{\mathbf{A}}
\newcommand{\bB}{\mathbf{B}}
\newcommand{\bepsilon}{\boldsymbol{\varepsilon}}
\newcommand{\bvarepsilon}{\boldsymbol{\varepsilon}}
\newcommand{\bSigma}{\boldsymbol{\Sigma}}
\newcommand{\bPhi}{\boldsymbol{\Phi}}
\newcommand{\bGamma}{\boldsymbol{\Gamma}}
\newcommand{\bPi}{\boldsymbol{\Pi}}
\newcommand{\bc}{\mathbf{c}}
\newcommand{\balpha}{\boldsymbol{\alpha}}
\newcommand{\bbeta}{\boldsymbol{\beta}}

%=============================================================================
% TITLE INFORMATION
%=============================================================================
\title[Time Series Analysis]{Time Series Analysis and Forecasting}
\author[D.T. Pele]{Daniel Traian PELE}
\institute{Bucharest University of Economic Studies\\
IDA Institute Digital Assets\\
Blockchain Research Center\\
AI4EFin Artificial Intelligence for Energy Finance\\
Romanian Academy, Institute for Economic Forecasting\\
MSCA Digital Finance}
\date{}

\subtitle{Seminar 0: Fundamentals}

\begin{document}

{
\setbeamertemplate{headline}{}
\setbeamertemplate{footline}{}
\begin{frame}
    \titlepage
\end{frame}
}


\begin{frame}{Seminar Outline}
    \begin{cminipage}{0.95\textwidth}
    \begin{itemize}
        \item \textbf{Multiple Choice Quiz} -- Knowledge check
        \vspace{0.15cm}
        \item \textbf{True/False} -- Conceptual checks
        \vspace{0.15cm}
        \item \textbf{Calculation Exercises} -- Applied practice
        \vspace{0.15cm}
        \item \textbf{AI-Assisted Exercise} -- Critical thinking
        \vspace{0.15cm}
        \item \textbf{Summary} -- Key takeaways
    \end{itemize}
    \end{cminipage}
\end{frame}

%=============================================================================
% MULTIPLE CHOICE QUIZ
%=============================================================================
\section{Multiple Choice Quiz}

\begin{frame}{Quiz 1: Time Series Basics}
    \begin{cminipage}{0.95\textwidth}
    \begin{alertblock}{Question}
        Which of the following is NOT a characteristic of time series data?
    \end{alertblock}

    \vspace{0.4cm}

    \begin{block}{Answer choices}
        \textcolor{MainBlue}{\textbf{(A)}} Observations are ordered in time\\[3pt]
        \textcolor{MainBlue}{\textbf{(B)}} Consecutive observations are usually correlated\\[3pt]
        \textcolor{MainBlue}{\textbf{(C)}} Observations are independent and identically distributed\\[3pt]
        \textcolor{MainBlue}{\textbf{(D)}} Data has a natural temporal ordering
    \end{block}

    \vspace{0.5cm}

    \begin{center}
        \textit{Answer on next slide...}
    \end{center}
    \end{cminipage}
\end{frame}

\begin{frame}{Quiz 1: Answer}
    \begin{cminipage}{0.95\textwidth}
    \begin{exampleblock}{Answer: C -- Observations are independent and identically distributed}
        \textbf{Question:} Which is NOT a characteristic of time series data?

        \vspace{0.3cm}

    \begin{block}{Answer choices}
        \textcolor{MainBlue}{\textbf{(A)}} Observations are ordered in time \incorrect\\[3pt]
        \textcolor{MainBlue}{\textbf{(B)}} Consecutive observations are usually correlated \incorrect\\[3pt]
        \textcolor{MainBlue}{\textbf{(C)}} \textbf{\textcolor{Forest}{Observations are independent and identically distributed}} \correct\\[3pt]
        \textcolor{MainBlue}{\textbf{(D)}} Data has a natural temporal ordering \incorrect
    \end{block}

        \vspace{0.3cm}

        \begin{itemize}
            \item Time series observations are \textbf{dependent} (autocorrelated), not independent
            \item The i.i.d.\ assumption is fundamental to cross-sectional analysis but is \textbf{violated} in time series
            \item This temporal dependence requires \textbf{specialized methods}
        \end{itemize}
    \end{exampleblock}

    \end{cminipage}
    \quantlet{TSA\_ch0\_definition}{https://github.com/QuantLet/TSA/tree/main/TSA_ch0/TSA_ch0_definition}
\end{frame}

\begin{frame}{Quiz 2: Decomposition}
    \begin{cminipage}{0.95\textwidth}
    \begin{alertblock}{Question}
        When should you use multiplicative decomposition instead of additive?
    \end{alertblock}

    \vspace{0.4cm}

    \begin{block}{Answer choices}
        \textcolor{MainBlue}{\textbf{(A)}} When the seasonal pattern has constant amplitude\\[3pt]
        \textcolor{MainBlue}{\textbf{(B)}} When the variance of the series is stable over time\\[3pt]
        \textcolor{MainBlue}{\textbf{(C)}} When seasonal fluctuations grow proportionally with the level\\[3pt]
        \textcolor{MainBlue}{\textbf{(D)}} When the series has no trend component
    \end{block}

    \vspace{0.5cm}

    \begin{center}
        \textit{Answer on next slide...}
    \end{center}
    \end{cminipage}
\end{frame}

\begin{frame}{Quiz 2: Answer}
    \begin{cminipage}{0.95\textwidth}
    \begin{exampleblock}{Answer: C -- When seasonal fluctuations grow proportionally with the level}
        \vspace{-0.2cm}
        \begin{center}
            \includegraphics[width=0.95\textwidth, height=0.52\textheight, keepaspectratio]{additive_vs_multiplicative.png}
        \end{center}
        \vspace{-0.2cm}
        {\footnotesize
        \begin{itemize}\setlength{\itemsep}{0pt}
            \item \textbf{Multiplicative}: $X_t = T_t \times S_t \times \varepsilon_t$ --- seasonal amplitude \textbf{scales with the level}
            \item \textbf{Additive}: $X_t = T_t + S_t + \varepsilon_t$ --- constant amplitude
        \end{itemize}
        }
    \end{exampleblock}

    \end{cminipage}
    \quantlet{TSA\_ch0\_decomposition}{https://github.com/QuantLet/TSA/tree/main/TSA_ch0/TSA_ch0_decomposition}
\end{frame}

\begin{frame}{Quiz 3: Exponential Smoothing}
    \begin{cminipage}{0.95\textwidth}
    \begin{alertblock}{Question}
        In Simple Exponential Smoothing with $\alpha = 0.9$, what happens?
    \end{alertblock}

    \vspace{0.4cm}

    \begin{block}{Answer choices}
        \textcolor{MainBlue}{\textbf{(A)}} Forecasts are very smooth and stable\\[3pt]
        \textcolor{MainBlue}{\textbf{(B)}} Recent observations have very little weight\\[3pt]
        \textcolor{MainBlue}{\textbf{(C)}} Forecasts react quickly to recent changes\\[3pt]
        \textcolor{MainBlue}{\textbf{(D)}} The forecast is essentially a long-term average
    \end{block}

    \vspace{0.5cm}

    \begin{center}
        \textit{Answer on next slide...}
    \end{center}
    \end{cminipage}
\end{frame}

\begin{frame}{Quiz 3: Answer}
    \begin{cminipage}{0.95\textwidth}
    \begin{exampleblock}{Answer: C -- Forecasts react quickly to recent changes}
        With $\alpha = 0.9$: $\hat{X}_{t+1} = 0.9 X_t + 0.1 \hat{X}_t$
        \begin{itemize}
            \item \textbf{High $\alpha$} (e.g.\ 0.9): 90\% weight on the last observation
                \begin{itemize}
                    \item Forecasts very responsive to new data
                \end{itemize}
            \item \textbf{Low $\alpha$} (e.g.\ 0.1): smoother, more stable forecasts
                \begin{itemize}
                    \item Averages over more history
                \end{itemize}
        \end{itemize}
    \end{exampleblock}

    \end{cminipage}
    \quantlet{TSA\_ch0\_smoothing}{https://github.com/QuantLet/TSA/tree/main/TSA_ch0/TSA_ch0_smoothing}
\end{frame}

\begin{frame}{Quiz 4: Moving Averages}
    \begin{cminipage}{0.95\textwidth}
    \begin{alertblock}{Question}
        A centered moving average of order 5 (MA-5) uses which observations to estimate the trend at time $t$?
    \end{alertblock}

    \vspace{0.4cm}

    \begin{block}{Answer choices}
        \textcolor{MainBlue}{\textbf{(A)}} $X_t, X_{t+1}, X_{t+2}, X_{t+3}, X_{t+4}$\\[3pt]
        \textcolor{MainBlue}{\textbf{(B)}} $X_{t-4}, X_{t-3}, X_{t-2}, X_{t-1}, X_t$\\[3pt]
        \textcolor{MainBlue}{\textbf{(C)}} $X_{t-2}, X_{t-1}, X_t, X_{t+1}, X_{t+2}$\\[3pt]
        \textcolor{MainBlue}{\textbf{(D)}} $X_{t-1}, X_t, X_{t+1}$
    \end{block}

    \vspace{0.5cm}

    \begin{center}
        \textit{Answer on next slide...}
    \end{center}
    \end{cminipage}
\end{frame}

\begin{frame}{Quiz 4: Answer}
    \begin{cminipage}{0.95\textwidth}
    \begin{exampleblock}{Answer: C -- $X_{t-2}, X_{t-1}, X_t, X_{t+1}, X_{t+2}$}
        \vspace{-0.2cm}
        \begin{center}
            \includegraphics[width=0.95\textwidth, height=0.52\textheight, keepaspectratio]{ch1_moving_average.png}
        \end{center}
        \vspace{-0.2cm}
        {\footnotesize
        \begin{itemize}\setlength{\itemsep}{0pt}
            \item \textbf{Centered MA}: uses $(k-1)/2$ observations on each side of $t$
            \item \textbf{MA-5}: 2 before + $t$ + 2 after $\Rightarrow$ larger window = smoother
        \end{itemize}
        }
    \end{exampleblock}

    \end{cminipage}
    \quantlet{TSA\_ch0\_definition}{https://github.com/QuantLet/TSA/tree/main/TSA_ch0/TSA_ch0_definition}
\end{frame}

\begin{frame}{Quiz 5: Forecast Evaluation}
    \begin{cminipage}{0.95\textwidth}
    \begin{alertblock}{Question}
        Which metric is most appropriate for comparing forecast accuracy across series with different scales?
    \end{alertblock}

    \vspace{0.4cm}

    \begin{block}{Answer choices}
        \textcolor{MainBlue}{\textbf{(A)}} Mean Absolute Error (MAE)\\[3pt]
        \textcolor{MainBlue}{\textbf{(B)}} Root Mean Squared Error (RMSE)\\[3pt]
        \textcolor{MainBlue}{\textbf{(C)}} Mean Absolute Percentage Error (MAPE)\\[3pt]
        \textcolor{MainBlue}{\textbf{(D)}} Mean Squared Error (MSE)
    \end{block}

    \vspace{0.5cm}

    \begin{center}
        \textit{Answer on next slide...}
    \end{center}
    \end{cminipage}
\end{frame}

\begin{frame}{Quiz 5: Answer}
    \begin{cminipage}{0.95\textwidth}
    \begin{exampleblock}{Answer: C -- Mean Absolute Percentage Error (MAPE)}
        MAPE $= \frac{100}{n}\sum\left|\frac{e_t}{X_t}\right|$ expresses errors as \textbf{percentages}.

        \begin{itemize}
            \item MAE, RMSE, MSE are \textbf{scale-dependent} (units of $X_t$)
            \item MAPE is \textbf{scale-independent} (always in \%)
            \item Caveat: MAPE becomes unstable when $X_t \approx 0$
        \end{itemize}
    \end{exampleblock}

    \end{cminipage}
    \quantlet{TSA\_ch0\_forecast\_eval}{https://github.com/QuantLet/TSA/tree/main/TSA_ch0/TSA_ch0_forecast_eval}
\end{frame}

\begin{frame}{Quiz 6: Cross-Validation}
    \begin{cminipage}{0.95\textwidth}
    \begin{alertblock}{Question}
        Why can't we use standard k-fold cross-validation for time series?
    \end{alertblock}

    \vspace{0.4cm}

    \begin{block}{Answer choices}
        \textcolor{MainBlue}{\textbf{(A)}} Time series data is too small\\[3pt]
        \textcolor{MainBlue}{\textbf{(B)}} It would violate temporal ordering (future predicting past)\\[3pt]
        \textcolor{MainBlue}{\textbf{(C)}} Cross-validation is always invalid\\[3pt]
        \textcolor{MainBlue}{\textbf{(D)}} Time series doesn't need validation
    \end{block}

    \vspace{0.5cm}

    \begin{center}
        \textit{Answer on next slide...}
    \end{center}
    \end{cminipage}
\end{frame}

\begin{frame}{Quiz 6: Answer}
    \begin{cminipage}{0.95\textwidth}
    \begin{exampleblock}{Answer: B -- It would violate temporal ordering}
        \vspace{-0.2cm}
        \begin{center}
            \includegraphics[width=0.95\textwidth, height=0.52\textheight, keepaspectratio]{ch8_timeseries_cv.png}
        \end{center}
        \vspace{-0.2cm}
        {\footnotesize
        \textbf{Principle}: future data cannot be used to predict the past! Rolling/expanding window CV is recommended.
        }
    \end{exampleblock}

    \end{cminipage}
    \quantlet{TSA\_ch0\_forecast\_eval}{https://github.com/QuantLet/TSA/tree/main/TSA_ch0/TSA_ch0_forecast_eval}
\end{frame}

\begin{frame}{Visual: Time Series Decomposition}
    \begin{cminipage}{0.95\textwidth}
    \vspace{-0.3cm}
    \begin{center}
        \includegraphics[width=0.88\textwidth, height=0.55\textheight, keepaspectratio]{ch1_decomposition.png}
    \end{center}
    \vspace{-0.3cm}
    {\footnotesize
    \begin{exampleblock}{Decomposition Components}
        \begin{itemize}\setlength{\itemsep}{0pt}
            \item \textbf{Trend}: long-term movement \quad \textbf{Seasonality}: periodic pattern \quad \textbf{Residual}: random noise
        \end{itemize}
    \end{exampleblock}
    }

    \end{cminipage}
    \quantlet{TSA\_ch0\_decomposition}{https://github.com/QuantLet/TSA/tree/main/TSA_ch0/TSA_ch0_decomposition}
\end{frame}

%=============================================================================
% TRUE/FALSE
%=============================================================================
\section{True/False}

\begin{frame}{True or False? --- Questions}
    \begin{cminipage}{0.95\textwidth}
    \footnotesize
    \begin{center}
    \begin{tabular}{p{9cm}c}
        \toprule
        \textbf{Statement} & \textbf{T/F?} \\
        \midrule
        1. SES forecasts are flat (constant for all horizons). & ? \\[0.15cm]
        2. RMSE penalizes large errors more than MAE. & ? \\[0.15cm]
        3. Multiplicative decomposition requires positive data. & ? \\[0.15cm]
        4. A larger $\alpha$ means more smoothing. & ? \\[0.15cm]
        5. The test set is used for hyperparameter tuning. & ? \\[0.15cm]
        6. Seasonal naive uses the value from one season ago. & ? \\[0.15cm]
        7. MAPE can be infinite if actual values are zero. & ? \\
        \bottomrule
    \end{tabular}
    \end{center}
    \end{cminipage}
\end{frame}

\begin{frame}{True or False? --- Answers}
    \begin{cminipage}{0.95\textwidth}
    \scriptsize
    \begin{center}
    \begin{tabular}{p{7.5cm}cc}
        \toprule
        \textbf{Statement} & \textbf{T/F} & \textbf{Explanation} \\
        \midrule
        1. SES forecasts are flat (constant for all horizons). & \textcolor{Forest}{\textbf{T}} & {\tiny No trend} \\[0.08cm]
        2. RMSE penalizes large errors more than MAE. & \textcolor{Forest}{\textbf{T}} & {\tiny Squared errors} \\[0.08cm]
        3. Multiplicative decomposition requires positive data. & \textcolor{Forest}{\textbf{T}} & {\tiny Cannot $\times$ negative} \\[0.08cm]
        4. A larger $\alpha$ means more smoothing. & \textcolor{Crimson}{\textbf{F}} & {\tiny Large $\alpha$ = less smooth} \\[0.08cm]
        5. The test set is used for hyperparameter tuning. & \textcolor{Crimson}{\textbf{F}} & {\tiny Use validation!} \\[0.08cm]
        6. Seasonal naive uses the value from one season ago. & \textcolor{Forest}{\textbf{T}} & {\tiny $\hat{X}_{t+h} = X_{t+h-m}$} \\[0.08cm]
        7. MAPE can be infinite if actual values are zero. & \textcolor{Forest}{\textbf{T}} & {\tiny Division by zero} \\
        \bottomrule
    \end{tabular}
    \end{center}
    \end{cminipage}
\end{frame}

%=============================================================================
% CALCULATION EXERCISES
%=============================================================================
\section{Calculation Exercises}

\begin{frame}{Exercise 1: Simple Exponential Smoothing}
    \begin{cminipage}{0.95\textwidth}
    \begin{alertblock}{Problem}
        \begin{itemize}\setlength{\itemsep}{0pt}
            \item \textbf{Data}: $X = [10, 12, 11, 14, 13]$ with $\alpha = 0.3$, $\hat{X}_1 = 10$
            \item \textbf{Calculate}: a) Forecasts $\hat{X}_2$ through $\hat{X}_6$; b) MAE and RMSE
            \item \textbf{Formula}: $\hat{X}_{t+1} = \alpha X_t + (1-\alpha)\hat{X}_t$
        \end{itemize}
    \end{alertblock}

    \vspace{0.2cm}
    \begin{exampleblock}{Solution}
        \begin{center}
        \small
        \begin{tabular}{c|ccccc|c}
            $t$ & 1 & 2 & 3 & 4 & 5 & 6\\
            \hline
            $X_t$ & 10 & 12 & 11 & 14 & 13 & ?\\
            $\hat{X}_t$ & 10 & 10 & 10.6 & 10.72 & 11.70 & \textbf{12.09}\\
        \end{tabular}
        \end{center}
        \begin{itemize}\setlength{\itemsep}{0pt}
            \item \textbf{MAE} $= 1.745$ \quad \textbf{RMSE} $= 2.04$
        \end{itemize}
    \end{exampleblock}
    \end{cminipage}
\end{frame}

\begin{frame}{Exercise 2: Error Metrics}
    \begin{cminipage}{0.95\textwidth}
    \begin{alertblock}{Problem}
        \begin{itemize}\setlength{\itemsep}{0pt}
            \item \textbf{Data}: $X = [100, 110, 105, 120]$, $\hat{X} = [95, 108, 110, 115]$
            \item \textbf{Calculate}: MAE, MSE, RMSE, MAPE
        \end{itemize}
    \end{alertblock}

    \vspace{0.2cm}
    \begin{exampleblock}{Solution}
        \begin{itemize}\setlength{\itemsep}{0pt}
            \item \textbf{Errors}: $e = [5, 2, -5, 5]$
            \item \textbf{MAE} $= (|5|+|2|+|-5|+|5|)/4 = 4.25$
            \item \textbf{MSE} $= (25+4+25+25)/4 = 19.75$
            \item \textbf{RMSE} $= \sqrt{19.75} = 4.44$
            \item \textbf{MAPE} $= 25 \times (0.05+0.018+0.048+0.042) = 3.95\%$
        \end{itemize}
    \end{exampleblock}
    \end{cminipage}
\end{frame}

\begin{frame}{Exercise 3: Seasonal Indices}
    \begin{cminipage}{0.95\textwidth}
    \begin{alertblock}{Problem}
        \begin{itemize}\setlength{\itemsep}{0pt}
            \item \textbf{Data}: Seasonal indices: $S = [0.85, 1.05, 0.90, 1.20]$, Trend Q4: $T = 1000$
            \item \textbf{Calculate}: a) Verify normalization. b) Q4 forecast. c) Deseasonalize $X_{Q4} = 1150$
        \end{itemize}
    \end{alertblock}

    \vspace{0.2cm}
    \begin{exampleblock}{Solution}
        \begin{itemize}\setlength{\itemsep}{0pt}
            \item \textbf{a) Normalization}: $\sum S_i = 0.85+1.05+0.90+1.20 = 4.00$ \checkmark
            \item \textbf{b) Forecast}: $\hat{X}_{Q4} = 1000 \times 1.20 = \textbf{1200}$
            \item \textbf{c) Deseasonalization}: $X_{deseasonalized} = 1150/1.20 = \textbf{958.33}$ (below trend)
        \end{itemize}
    \end{exampleblock}
    \end{cminipage}
\end{frame}

%=============================================================================
% AI-ASSISTED EXERCISE
%=============================================================================
\section{AI-Assisted Exercise}

\begin{frame}{AI Exercise: Critical Thinking}
    \begin{cminipage}{0.95\textwidth}
    \vspace{-0.3cm}
    \begin{block}{\footnotesize Prompt to test in ChatGPT / Claude / Copilot}
        {\footnotesize
        ``Using yfinance, download adjusted close prices for SPY from 2015 to 2024. Apply seasonal decomposition (additive and multiplicative) and compare the results visually. Then split the data into training set (2015--2023) and test set (2024). Fit three exponential smoothing models (SES, Holt, Holt-Winters) on the training set and evaluate them on the test set using RMSE and MAPE. Which model is best? Show the plots and comparison table.''
        }
    \end{block}
    \vspace{-2mm}
    {\footnotesize
    \textbf{Exercise}:
    \begin{enumerate}\setlength{\itemsep}{0pt}
        \item Run the prompt in an LLM of your choice and critically analyze the response.
        \item Does the AI choose additive or multiplicative decomposition? Is the choice justified?
        \item How does it evaluate models --- does it use train or test RMSE?
        \item Check the smoothing parameters ($\alpha$, $\beta$, $\gamma$). Are values near 1.0 problematic?
        \item Does the code properly split data into train/test, or does it evaluate on training data?
    \end{enumerate}
    }
    \vspace{-2mm}
    \begin{alertblock}{}
        {\footnotesize \textbf{Warning}: AI-generated code may run without errors and look professional. \textit{That does not mean it is correct.}}
    \end{alertblock}
    \end{cminipage}
\end{frame}

%=============================================================================
% END
%=============================================================================
\section{Summary}

\begin{frame}{Summary: Chapter 0}
    \begin{cminipage}{0.95\textwidth}
    \begin{exampleblock}{Key Concepts}
        \begin{itemize}\setlength{\itemsep}{2pt}
            \item[\textcolor{MainBlue}{\textbf{1.}}] \textbf{Time series}: temporally ordered observations, with dependence (autocorrelation)
            \item[\textcolor{MainBlue}{\textbf{2.}}] \textbf{Decomposition}: additive ($X_t = T_t + S_t + \varepsilon_t$) vs multiplicative ($X_t = T_t \times S_t \times \varepsilon_t$)
            \item[\textcolor{MainBlue}{\textbf{3.}}] \textbf{Exponential smoothing}: SES, Holt, Holt-Winters --- parameter $\alpha$ controls reactivity
            \item[\textcolor{MainBlue}{\textbf{4.}}] \textbf{Forecast evaluation}: MAE, RMSE, MAPE --- the choice depends on context
            \item[\textcolor{MainBlue}{\textbf{5.}}] \textbf{Seasonality}: seasonal indices, forecasting and deseasonalization
        \end{itemize}
    \end{exampleblock}

    \vspace{0.5cm}
    \begin{center}
        \Large\textcolor{MainBlue}{Questions?}
    \end{center}
    \end{cminipage}
\end{frame}


%=============================================================================
% BIBLIOGRAPHY (same references as in the course)
%=============================================================================
\section{Bibliography}

\begin{frame}{Bibliography I}
    \begin{cminipage}{0.95\textwidth}
    \begin{block}{Time Series Fundamentals}
        {\small
        \begin{itemize}\setlength{\itemsep}{0pt}
            \item Hyndman, R.J., \& Athanasopoulos, G. (2021). \textit{Forecasting: Principles and Practice}, 3rd ed., OTexts.
            \item Shumway, R.H., \& Stoffer, D.S. (2017). \textit{Time Series Analysis and Its Applications}, 4th ed., Springer.
            \item Brockwell, P.J., \& Davis, R.A. (2016). \textit{Introduction to Time Series and Forecasting}, 3rd ed., Springer.
        \end{itemize}
        }
    \end{block}

    \begin{exampleblock}{Financial Time Series}
        {\small
        \begin{itemize}\setlength{\itemsep}{0pt}
            \item Tsay, R.S. (2010). \textit{Analysis of Financial Time Series}, 3rd ed., Wiley.
            \item Franke, J., H\"ardle, W.K., \& Hafner, C.M. (2019). \textit{Statistics of Financial Markets}, 4th ed., Springer.
        \end{itemize}
        }
    \end{exampleblock}
    \end{cminipage}
\end{frame}

\begin{frame}{Bibliography II}
    \begin{cminipage}{0.95\textwidth}
    \begin{block}{Modern Approaches and Machine Learning}
        {\small
        \begin{itemize}\setlength{\itemsep}{0pt}
            \item Nielsen, A. (2019). \textit{Practical Time Series Analysis}, O'Reilly Media.
            \item Petropoulos, F., et al. (2022). \textit{Forecasting: Theory and Practice}, International Journal of Forecasting.
            \item Makridakis, S., Spiliotis, E., \& Assimakopoulos, V. (2020). The M4 Competition, International Journal of Forecasting.
        \end{itemize}
        }
    \end{block}

    \begin{exampleblock}{Online Resources and Code}
        {\small
        \begin{itemize}\setlength{\itemsep}{0pt}
            \item \textbf{Quantlet}: \url{https://quantlet.com} --- Code repository for statistics
            \item \textbf{Quantinar}: \url{https://quantinar.com} --- Quantitative methods learning platform
            \item \textbf{GitHub TSA}: \url{https://github.com/QuantLet/TSA/tree/main/TSA_ch0} --- Python code for this seminar
        \end{itemize}
        }
    \end{exampleblock}
    \end{cminipage}
\end{frame}

\begin{frame}{}
    \begin{cminipage}{0.95\textwidth}
    \centering
    \Huge\textcolor{IDAred}{Thank You!}

    \vspace{1cm}

    \Large\textcolor{MainBlue}{Questions?}

    \vspace{0.8cm}

    \normalsize
    Seminar materials are available at: \url{https://danpele.github.io/Time-Series-Analysis/}

    \vspace{0.2cm}

    \href{https://quantlet.com}{\raisebox{-0.15em}{\includegraphics[height=0.8em]{ql_logo.png}} Quantlet} \hspace{0.5cm}
    \href{https://quantinar.com}{\raisebox{-0.15em}{\includegraphics[height=0.8em]{qr_logo.png}} Quantinar}
    \end{cminipage}
\end{frame}

\end{document}
