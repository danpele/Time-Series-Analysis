% Seminar 1: Stochastic Processes and Stationarity

\documentclass[9pt, aspectratio=169, t]{beamer}
%=============================================================================
% SHARED PREAMBLE - Time Series Analysis and Forecasting
% Harvard-quality academic presentations
% Bachelor program, Bucharest University of Economic Studies
%
% Usage: \documentclass[9pt, aspectratio=169, t]{beamer}
%            %=============================================================================
% SHARED PREAMBLE - Time Series Analysis and Forecasting
% Harvard-quality academic presentations
% Bachelor program, Bucharest University of Economic Studies
%
% Usage: \documentclass[9pt, aspectratio=169, t]{beamer}
%            %=============================================================================
% SHARED PREAMBLE - Time Series Analysis and Forecasting
% Harvard-quality academic presentations
% Bachelor program, Bucharest University of Economic Studies
%
% Usage: \documentclass[9pt, aspectratio=169, t]{beamer}
%            \input{preamble}
%            \subtitle{Seminar X: Seminar Title}
%            \begin{document} ...
%=============================================================================

% Ensure content fits on slides
\setbeamersize{text margin left=8mm, text margin right=8mm}

%=============================================================================
% THEME AND STYLE CONFIGURATION
%=============================================================================
\usetheme{default}
% Using default theme for clean header/footer control

% Color Palette (matching Redispatch PDF)
\definecolor{MainBlue}{RGB}{26, 58, 110}
\definecolor{AccentBlue}{RGB}{26, 58, 110}
\definecolor{IDAred}{RGB}{205, 0, 0}
\definecolor{DarkGray}{RGB}{51, 51, 51}
\definecolor{MediumGray}{RGB}{128, 128, 128}
\definecolor{LightGray}{RGB}{248, 248, 248}
\definecolor{VeryLightGray}{RGB}{235, 235, 235}
\definecolor{KeynoteGray}{RGB}{218, 218, 218}
\definecolor{SectionGray}{RGB}{120, 120, 120}
\definecolor{FooterGray}{RGB}{100, 100, 100}
\definecolor{Crimson}{RGB}{220, 53, 69}
\definecolor{Forest}{RGB}{46, 125, 50}
\definecolor{Amber}{RGB}{181, 133, 63}
\definecolor{Orange}{RGB}{230, 126, 34}
\definecolor{Purple}{RGB}{142, 68, 173}

% Gradient background (exact Keynote 315° gradient: white to RGB 218,218,218)
\setbeamertemplate{background}{%
    \begin{tikzpicture}[remember picture, overlay]
        \shade[shading=axis, shading angle=315,
        top color=white, bottom color=KeynoteGray]
        (current page.south west) rectangle (current page.north east);
    \end{tikzpicture}%
}
% Fallback solid color for compatibility
\setbeamercolor{background canvas}{bg=}

\setbeamercolor{palette primary}{bg=MainBlue, fg=white}
\setbeamercolor{palette secondary}{bg=MainBlue!85, fg=white}
\setbeamercolor{palette tertiary}{bg=MainBlue!70, fg=white}
\setbeamercolor{structure}{fg=MainBlue}
\setbeamercolor{title}{fg=IDAred}
\setbeamercolor{frametitle}{fg=IDAred, bg=}
\setbeamercolor{block title}{bg=MainBlue, fg=white}
\setbeamercolor{block body}{bg=VeryLightGray, fg=DarkGray}
\setbeamercolor{block title alerted}{bg=Crimson, fg=white}
\setbeamercolor{block body alerted}{bg=Crimson!8, fg=DarkGray}
\setbeamercolor{block title example}{bg=Forest, fg=white}
\setbeamercolor{block body example}{bg=Forest!8, fg=DarkGray}
\setbeamercolor{item}{fg=MainBlue}

% Smaller institute font to avoid overfull hbox on title page
\setbeamerfont{institute}{size=\footnotesize}

% Footer colors (override Madrid theme blue)
\setbeamercolor{author in head/foot}{fg=FooterGray, bg=}
\setbeamercolor{title in head/foot}{fg=FooterGray, bg=}
\setbeamercolor{date in head/foot}{fg=FooterGray, bg=}
\setbeamercolor{section in head/foot}{fg=FooterGray, bg=}
\setbeamercolor{subsection in head/foot}{fg=FooterGray, bg=}

% Bullet styles (apply everywhere including blocks)
\setbeamertemplate{itemize item}{\color{MainBlue}$\boxdot$}
\setbeamertemplate{itemize subitem}{\color{MainBlue}$\blacktriangleright$}
\setbeamertemplate{itemize subsubitem}{\color{MainBlue}\tiny$\bullet$}
\setbeamertemplate{itemize/enumerate body begin}{\normalsize}
\setbeamertemplate{itemize/enumerate subbody begin}{\normalsize}

% Item spacing - compact style
\setlength{\leftmargini}{10pt}       % Level 1: minimal indent
\setlength{\leftmarginii}{10pt}      % Level 2: minimal additional indent
% Compact list spacing (zero extra space before/after lists in blocks)
\makeatletter
\def\@listi{\leftmargin\leftmargini \topsep 0pt \parsep 0pt \itemsep 0pt}
\def\@listii{\leftmargin\leftmarginii \topsep 0pt \parsep 0pt \itemsep 0pt}
\makeatother

\setbeamertemplate{navigation symbols}{}

%=============================================================================
% CUSTOM HEADLINE
%=============================================================================
\setbeamertemplate{headline}{%
    \vskip10pt%
    \hbox to \paperwidth{%
        \hskip0.5cm%
        {\small\color{FooterGray}\renewcommand{\hyperlink}[2]{##2}\insertsectionhead}%
        \hfill%
        \textcolor{FooterGray}{\small\insertframenumber}%
        \hskip0.5cm%
    }%
    \vskip4pt%
    {\color{FooterGray}\hrule height 0.4pt}%
}

%=============================================================================
% CUSTOM FOOTER
%=============================================================================
\usepackage{fontawesome5}

\setbeamertemplate{footline}{%
    {\color{FooterGray}\hrule height 0.4pt}%
    \vskip4pt%
    \hbox to \paperwidth{%
        \hskip0.5cm%
        \textcolor{FooterGray}{\small Time Series Analysis and Forecasting}%
        \hfill%
        \raisebox{-0.1em}{%
            \begin{tikzpicture}[x=0.08em, y=0.08em, line width=0.4pt]
                \draw[FooterGray] (0,3) -- (1,4) -- (2,3.5) -- (3,5) -- (4,4) -- (5,6) -- (6,5.5) -- (7,4) -- (8,5) -- (9,7) -- (10,6) -- (11,5) -- (12,6.5) -- (13,8) -- (14,7) -- (15,6) -- (16,7.5) -- (17,9) -- (18,8) -- (19,7) -- (20,8.5) -- (21,10) -- (22,9) -- (23,8) -- (24,9.5);
            \end{tikzpicture}%
        }%
        \hskip0.5cm%
    }%
    \vskip6pt%
}

%=============================================================================
% PACKAGES
%=============================================================================
\usepackage[utf8]{inputenc}
\usepackage[T1]{fontenc}
\usepackage[english]{babel}
\usepackage{amsmath, amssymb, amsthm}
\usepackage{mathtools}
\usepackage{bm}
\usepackage{tikz}
\usetikzlibrary{arrows.meta, positioning, shapes, calc, decorations.pathreplacing, shadings}
\usepackage{booktabs}
\usepackage{multirow}
\usepackage{array}
\usepackage{graphicx}
\usepackage{hyperref}
\usepackage{colortbl}
\usepackage{listings}
\lstset{basicstyle=\ttfamily\small, breaklines=true, frame=single, backgroundcolor=\color{VeryLightGray}}
\hypersetup{colorlinks=true, linkcolor=MainBlue, urlcolor=MainBlue}
\graphicspath{{../../logos/}{../../charts/}{../../photos/}}
\hfuzz=2pt  % Suppress tiny overfull warnings (<2pt)
\vfuzz=2pt  % Suppress tiny vertical overfull warnings (<2pt)

%=============================================================================
% QUANTLET COMMAND
%=============================================================================
\newcommand{\quantlet}[2]{%
    \hfill\href{#2}{%
        \raisebox{-0.15em}{\includegraphics[height=0.7em]{ql_logo.png}}%
        \textcolor{MainBlue}{\tiny\ #1}%
    }%
}

%=============================================================================
% CUSTOM TITLE PAGE
%=============================================================================
\defbeamertemplate*{title page}{hybrid}[1][]
{
    \vspace{0.2cm}
    % Logos row - top header (with clickable links)
    \begin{center}
        \href{https://www.ase.ro}{\includegraphics[height=1.0cm]{ase_logo.png}}\hspace{0.25cm}%
        \href{https://theida.net}{\includegraphics[height=1.0cm]{ida_logo.png}}\hspace{0.25cm}%
        \href{https://blockchain-research-center.com}{\includegraphics[height=1.0cm]{brc_logo.png}}\hspace{0.25cm}%
        \href{https://www.ai4efin.ase.ro}{\includegraphics[height=1.0cm]{ai4efin_logo.png}}\hspace{0.25cm}%
        \href{https://ipe.ro/new}{\includegraphics[height=1.0cm]{acad_logo.png}}\hspace{0.25cm}%
        \href{https://www.digital-finance-msca.com}{\includegraphics[height=1.0cm]{msca_logo.png}}%
    \end{center}

    \vspace{0.6cm}

    % Main title with Q logos on sides (with clickable links)
    \begin{center}
        \begin{minipage}{0.1\textwidth}
            \centering
            \href{https://quantlet.com}{\includegraphics[height=1.1cm]{ql_logo.png}}
        \end{minipage}%
        \begin{minipage}{0.78\textwidth}
            \centering
            {\LARGE\bfseries\usebeamercolor[fg]{title}\inserttitle}

            \vspace{0.3cm}

            {\usebeamerfont{subtitle}\usebeamercolor[fg]{title}\insertsubtitle}
        \end{minipage}%
        \begin{minipage}{0.1\textwidth}
            \centering
            \href{https://quantinar.com}{\includegraphics[height=1.1cm]{qr_logo.png}}
        \end{minipage}
    \end{center}

    \vspace{0.6cm}

    % Authors (left aligned)
    \hspace{0.5cm}{\usebeamerfont{author}\insertauthor}

    \vspace{0.3cm}

    % Institute/Affiliations (left aligned)
    \hspace{0.5cm}\begin{minipage}[t]{0.9\textwidth}
        \raggedright\small\insertinstitute
    \end{minipage}
}

%=============================================================================
% THEOREM ENVIRONMENTS
%=============================================================================
\theoremstyle{definition}
\setbeamertemplate{theorems}[numbered]
\newtheorem{defn}{Definition}
\newtheorem{thm}{Theorem}
\newtheorem{prop}{Proposition}
\newtheorem{rmk}{Remark}

%=============================================================================
% CENTRED MINIPAGE (no extra vertical space)
%=============================================================================
\newenvironment{cminipage}[1]{%
    \par\noindent\hfill\begin{minipage}{#1}\ignorespaces
}{%
    \end{minipage}\hfill\null\par
}

%=============================================================================
% CUSTOM COMMANDS
%=============================================================================
\newcommand{\E}{\mathbb{E}}
\newcommand{\Var}{\text{Var}}
\newcommand{\Cov}{\text{Cov}}
\newcommand{\Corr}{\text{Corr}}
\newcommand{\R}{\mathbb{R}}
\newcommand{\N}{\mathbb{N}}
\newcommand{\Z}{\mathbb{Z}}
\newcommand{\B}{\mathbf{B}}
\newcommand{\imark}{\textcolor{MainBlue}{\textbullet}}
\newcommand{\RMSE}{\text{RMSE}}
\newcommand{\MAE}{\text{MAE}}
\newcommand{\MAPE}{\text{MAPE}}
\newcommand{\correct}{\textcolor{Forest}{\checkmark}}
\newcommand{\incorrect}{\textcolor{Crimson}{\texttimes}}

% Boldface vector/matrix commands
\newcommand{\bY}{\mathbf{Y}}
\newcommand{\bX}{\mathbf{X}}
\newcommand{\bA}{\mathbf{A}}
\newcommand{\bB}{\mathbf{B}}
\newcommand{\bepsilon}{\boldsymbol{\varepsilon}}
\newcommand{\bvarepsilon}{\boldsymbol{\varepsilon}}
\newcommand{\bSigma}{\boldsymbol{\Sigma}}
\newcommand{\bPhi}{\boldsymbol{\Phi}}
\newcommand{\bGamma}{\boldsymbol{\Gamma}}
\newcommand{\bPi}{\boldsymbol{\Pi}}
\newcommand{\bc}{\mathbf{c}}
\newcommand{\balpha}{\boldsymbol{\alpha}}
\newcommand{\bbeta}{\boldsymbol{\beta}}

%=============================================================================
% TITLE INFORMATION
%=============================================================================
\title[Time Series Analysis]{Time Series Analysis and Forecasting}
\author[D.T. Pele]{Daniel Traian PELE}
\institute{Bucharest University of Economic Studies\\
IDA Institute Digital Assets\\
Blockchain Research Center\\
AI4EFin Artificial Intelligence for Energy Finance\\
Romanian Academy, Institute for Economic Forecasting\\
MSCA Digital Finance}
\date{}

%            \subtitle{Seminar X: Seminar Title}
%            \begin{document} ...
%=============================================================================

% Ensure content fits on slides
\setbeamersize{text margin left=8mm, text margin right=8mm}

%=============================================================================
% THEME AND STYLE CONFIGURATION
%=============================================================================
\usetheme{default}
% Using default theme for clean header/footer control

% Color Palette (matching Redispatch PDF)
\definecolor{MainBlue}{RGB}{26, 58, 110}
\definecolor{AccentBlue}{RGB}{26, 58, 110}
\definecolor{IDAred}{RGB}{205, 0, 0}
\definecolor{DarkGray}{RGB}{51, 51, 51}
\definecolor{MediumGray}{RGB}{128, 128, 128}
\definecolor{LightGray}{RGB}{248, 248, 248}
\definecolor{VeryLightGray}{RGB}{235, 235, 235}
\definecolor{KeynoteGray}{RGB}{218, 218, 218}
\definecolor{SectionGray}{RGB}{120, 120, 120}
\definecolor{FooterGray}{RGB}{100, 100, 100}
\definecolor{Crimson}{RGB}{220, 53, 69}
\definecolor{Forest}{RGB}{46, 125, 50}
\definecolor{Amber}{RGB}{181, 133, 63}
\definecolor{Orange}{RGB}{230, 126, 34}
\definecolor{Purple}{RGB}{142, 68, 173}

% Gradient background (exact Keynote 315° gradient: white to RGB 218,218,218)
\setbeamertemplate{background}{%
    \begin{tikzpicture}[remember picture, overlay]
        \shade[shading=axis, shading angle=315,
        top color=white, bottom color=KeynoteGray]
        (current page.south west) rectangle (current page.north east);
    \end{tikzpicture}%
}
% Fallback solid color for compatibility
\setbeamercolor{background canvas}{bg=}

\setbeamercolor{palette primary}{bg=MainBlue, fg=white}
\setbeamercolor{palette secondary}{bg=MainBlue!85, fg=white}
\setbeamercolor{palette tertiary}{bg=MainBlue!70, fg=white}
\setbeamercolor{structure}{fg=MainBlue}
\setbeamercolor{title}{fg=IDAred}
\setbeamercolor{frametitle}{fg=IDAred, bg=}
\setbeamercolor{block title}{bg=MainBlue, fg=white}
\setbeamercolor{block body}{bg=VeryLightGray, fg=DarkGray}
\setbeamercolor{block title alerted}{bg=Crimson, fg=white}
\setbeamercolor{block body alerted}{bg=Crimson!8, fg=DarkGray}
\setbeamercolor{block title example}{bg=Forest, fg=white}
\setbeamercolor{block body example}{bg=Forest!8, fg=DarkGray}
\setbeamercolor{item}{fg=MainBlue}

% Smaller institute font to avoid overfull hbox on title page
\setbeamerfont{institute}{size=\footnotesize}

% Footer colors (override Madrid theme blue)
\setbeamercolor{author in head/foot}{fg=FooterGray, bg=}
\setbeamercolor{title in head/foot}{fg=FooterGray, bg=}
\setbeamercolor{date in head/foot}{fg=FooterGray, bg=}
\setbeamercolor{section in head/foot}{fg=FooterGray, bg=}
\setbeamercolor{subsection in head/foot}{fg=FooterGray, bg=}

% Bullet styles (apply everywhere including blocks)
\setbeamertemplate{itemize item}{\color{MainBlue}$\boxdot$}
\setbeamertemplate{itemize subitem}{\color{MainBlue}$\blacktriangleright$}
\setbeamertemplate{itemize subsubitem}{\color{MainBlue}\tiny$\bullet$}
\setbeamertemplate{itemize/enumerate body begin}{\normalsize}
\setbeamertemplate{itemize/enumerate subbody begin}{\normalsize}

% Item spacing - compact style
\setlength{\leftmargini}{10pt}       % Level 1: minimal indent
\setlength{\leftmarginii}{10pt}      % Level 2: minimal additional indent
% Compact list spacing (zero extra space before/after lists in blocks)
\makeatletter
\def\@listi{\leftmargin\leftmargini \topsep 0pt \parsep 0pt \itemsep 0pt}
\def\@listii{\leftmargin\leftmarginii \topsep 0pt \parsep 0pt \itemsep 0pt}
\makeatother

\setbeamertemplate{navigation symbols}{}

%=============================================================================
% CUSTOM HEADLINE
%=============================================================================
\setbeamertemplate{headline}{%
    \vskip10pt%
    \hbox to \paperwidth{%
        \hskip0.5cm%
        {\small\color{FooterGray}\renewcommand{\hyperlink}[2]{##2}\insertsectionhead}%
        \hfill%
        \textcolor{FooterGray}{\small\insertframenumber}%
        \hskip0.5cm%
    }%
    \vskip4pt%
    {\color{FooterGray}\hrule height 0.4pt}%
}

%=============================================================================
% CUSTOM FOOTER
%=============================================================================
\usepackage{fontawesome5}

\setbeamertemplate{footline}{%
    {\color{FooterGray}\hrule height 0.4pt}%
    \vskip4pt%
    \hbox to \paperwidth{%
        \hskip0.5cm%
        \textcolor{FooterGray}{\small Time Series Analysis and Forecasting}%
        \hfill%
        \raisebox{-0.1em}{%
            \begin{tikzpicture}[x=0.08em, y=0.08em, line width=0.4pt]
                \draw[FooterGray] (0,3) -- (1,4) -- (2,3.5) -- (3,5) -- (4,4) -- (5,6) -- (6,5.5) -- (7,4) -- (8,5) -- (9,7) -- (10,6) -- (11,5) -- (12,6.5) -- (13,8) -- (14,7) -- (15,6) -- (16,7.5) -- (17,9) -- (18,8) -- (19,7) -- (20,8.5) -- (21,10) -- (22,9) -- (23,8) -- (24,9.5);
            \end{tikzpicture}%
        }%
        \hskip0.5cm%
    }%
    \vskip6pt%
}

%=============================================================================
% PACKAGES
%=============================================================================
\usepackage[utf8]{inputenc}
\usepackage[T1]{fontenc}
\usepackage[english]{babel}
\usepackage{amsmath, amssymb, amsthm}
\usepackage{mathtools}
\usepackage{bm}
\usepackage{tikz}
\usetikzlibrary{arrows.meta, positioning, shapes, calc, decorations.pathreplacing, shadings}
\usepackage{booktabs}
\usepackage{multirow}
\usepackage{array}
\usepackage{graphicx}
\usepackage{hyperref}
\usepackage{colortbl}
\usepackage{listings}
\lstset{basicstyle=\ttfamily\small, breaklines=true, frame=single, backgroundcolor=\color{VeryLightGray}}
\hypersetup{colorlinks=true, linkcolor=MainBlue, urlcolor=MainBlue}
\graphicspath{{../../logos/}{../../charts/}{../../photos/}}
\hfuzz=2pt  % Suppress tiny overfull warnings (<2pt)
\vfuzz=2pt  % Suppress tiny vertical overfull warnings (<2pt)

%=============================================================================
% QUANTLET COMMAND
%=============================================================================
\newcommand{\quantlet}[2]{%
    \hfill\href{#2}{%
        \raisebox{-0.15em}{\includegraphics[height=0.7em]{ql_logo.png}}%
        \textcolor{MainBlue}{\tiny\ #1}%
    }%
}

%=============================================================================
% CUSTOM TITLE PAGE
%=============================================================================
\defbeamertemplate*{title page}{hybrid}[1][]
{
    \vspace{0.2cm}
    % Logos row - top header (with clickable links)
    \begin{center}
        \href{https://www.ase.ro}{\includegraphics[height=1.0cm]{ase_logo.png}}\hspace{0.25cm}%
        \href{https://theida.net}{\includegraphics[height=1.0cm]{ida_logo.png}}\hspace{0.25cm}%
        \href{https://blockchain-research-center.com}{\includegraphics[height=1.0cm]{brc_logo.png}}\hspace{0.25cm}%
        \href{https://www.ai4efin.ase.ro}{\includegraphics[height=1.0cm]{ai4efin_logo.png}}\hspace{0.25cm}%
        \href{https://ipe.ro/new}{\includegraphics[height=1.0cm]{acad_logo.png}}\hspace{0.25cm}%
        \href{https://www.digital-finance-msca.com}{\includegraphics[height=1.0cm]{msca_logo.png}}%
    \end{center}

    \vspace{0.6cm}

    % Main title with Q logos on sides (with clickable links)
    \begin{center}
        \begin{minipage}{0.1\textwidth}
            \centering
            \href{https://quantlet.com}{\includegraphics[height=1.1cm]{ql_logo.png}}
        \end{minipage}%
        \begin{minipage}{0.78\textwidth}
            \centering
            {\LARGE\bfseries\usebeamercolor[fg]{title}\inserttitle}

            \vspace{0.3cm}

            {\usebeamerfont{subtitle}\usebeamercolor[fg]{title}\insertsubtitle}
        \end{minipage}%
        \begin{minipage}{0.1\textwidth}
            \centering
            \href{https://quantinar.com}{\includegraphics[height=1.1cm]{qr_logo.png}}
        \end{minipage}
    \end{center}

    \vspace{0.6cm}

    % Authors (left aligned)
    \hspace{0.5cm}{\usebeamerfont{author}\insertauthor}

    \vspace{0.3cm}

    % Institute/Affiliations (left aligned)
    \hspace{0.5cm}\begin{minipage}[t]{0.9\textwidth}
        \raggedright\small\insertinstitute
    \end{minipage}
}

%=============================================================================
% THEOREM ENVIRONMENTS
%=============================================================================
\theoremstyle{definition}
\setbeamertemplate{theorems}[numbered]
\newtheorem{defn}{Definition}
\newtheorem{thm}{Theorem}
\newtheorem{prop}{Proposition}
\newtheorem{rmk}{Remark}

%=============================================================================
% CENTRED MINIPAGE (no extra vertical space)
%=============================================================================
\newenvironment{cminipage}[1]{%
    \par\noindent\hfill\begin{minipage}{#1}\ignorespaces
}{%
    \end{minipage}\hfill\null\par
}

%=============================================================================
% CUSTOM COMMANDS
%=============================================================================
\newcommand{\E}{\mathbb{E}}
\newcommand{\Var}{\text{Var}}
\newcommand{\Cov}{\text{Cov}}
\newcommand{\Corr}{\text{Corr}}
\newcommand{\R}{\mathbb{R}}
\newcommand{\N}{\mathbb{N}}
\newcommand{\Z}{\mathbb{Z}}
\newcommand{\B}{\mathbf{B}}
\newcommand{\imark}{\textcolor{MainBlue}{\textbullet}}
\newcommand{\RMSE}{\text{RMSE}}
\newcommand{\MAE}{\text{MAE}}
\newcommand{\MAPE}{\text{MAPE}}
\newcommand{\correct}{\textcolor{Forest}{\checkmark}}
\newcommand{\incorrect}{\textcolor{Crimson}{\texttimes}}

% Boldface vector/matrix commands
\newcommand{\bY}{\mathbf{Y}}
\newcommand{\bX}{\mathbf{X}}
\newcommand{\bA}{\mathbf{A}}
\newcommand{\bB}{\mathbf{B}}
\newcommand{\bepsilon}{\boldsymbol{\varepsilon}}
\newcommand{\bvarepsilon}{\boldsymbol{\varepsilon}}
\newcommand{\bSigma}{\boldsymbol{\Sigma}}
\newcommand{\bPhi}{\boldsymbol{\Phi}}
\newcommand{\bGamma}{\boldsymbol{\Gamma}}
\newcommand{\bPi}{\boldsymbol{\Pi}}
\newcommand{\bc}{\mathbf{c}}
\newcommand{\balpha}{\boldsymbol{\alpha}}
\newcommand{\bbeta}{\boldsymbol{\beta}}

%=============================================================================
% TITLE INFORMATION
%=============================================================================
\title[Time Series Analysis]{Time Series Analysis and Forecasting}
\author[D.T. Pele]{Daniel Traian PELE}
\institute{Bucharest University of Economic Studies\\
IDA Institute Digital Assets\\
Blockchain Research Center\\
AI4EFin Artificial Intelligence for Energy Finance\\
Romanian Academy, Institute for Economic Forecasting\\
MSCA Digital Finance}
\date{}

%            \subtitle{Seminar X: Seminar Title}
%            \begin{document} ...
%=============================================================================

% Ensure content fits on slides
\setbeamersize{text margin left=8mm, text margin right=8mm}

%=============================================================================
% THEME AND STYLE CONFIGURATION
%=============================================================================
\usetheme{default}
% Using default theme for clean header/footer control

% Color Palette (matching Redispatch PDF)
\definecolor{MainBlue}{RGB}{26, 58, 110}
\definecolor{AccentBlue}{RGB}{26, 58, 110}
\definecolor{IDAred}{RGB}{205, 0, 0}
\definecolor{DarkGray}{RGB}{51, 51, 51}
\definecolor{MediumGray}{RGB}{128, 128, 128}
\definecolor{LightGray}{RGB}{248, 248, 248}
\definecolor{VeryLightGray}{RGB}{235, 235, 235}
\definecolor{KeynoteGray}{RGB}{218, 218, 218}
\definecolor{SectionGray}{RGB}{120, 120, 120}
\definecolor{FooterGray}{RGB}{100, 100, 100}
\definecolor{Crimson}{RGB}{220, 53, 69}
\definecolor{Forest}{RGB}{46, 125, 50}
\definecolor{Amber}{RGB}{181, 133, 63}
\definecolor{Orange}{RGB}{230, 126, 34}
\definecolor{Purple}{RGB}{142, 68, 173}

% Gradient background (exact Keynote 315° gradient: white to RGB 218,218,218)
\setbeamertemplate{background}{%
    \begin{tikzpicture}[remember picture, overlay]
        \shade[shading=axis, shading angle=315,
        top color=white, bottom color=KeynoteGray]
        (current page.south west) rectangle (current page.north east);
    \end{tikzpicture}%
}
% Fallback solid color for compatibility
\setbeamercolor{background canvas}{bg=}

\setbeamercolor{palette primary}{bg=MainBlue, fg=white}
\setbeamercolor{palette secondary}{bg=MainBlue!85, fg=white}
\setbeamercolor{palette tertiary}{bg=MainBlue!70, fg=white}
\setbeamercolor{structure}{fg=MainBlue}
\setbeamercolor{title}{fg=IDAred}
\setbeamercolor{frametitle}{fg=IDAred, bg=}
\setbeamercolor{block title}{bg=MainBlue, fg=white}
\setbeamercolor{block body}{bg=VeryLightGray, fg=DarkGray}
\setbeamercolor{block title alerted}{bg=Crimson, fg=white}
\setbeamercolor{block body alerted}{bg=Crimson!8, fg=DarkGray}
\setbeamercolor{block title example}{bg=Forest, fg=white}
\setbeamercolor{block body example}{bg=Forest!8, fg=DarkGray}
\setbeamercolor{item}{fg=MainBlue}

% Smaller institute font to avoid overfull hbox on title page
\setbeamerfont{institute}{size=\footnotesize}

% Footer colors (override Madrid theme blue)
\setbeamercolor{author in head/foot}{fg=FooterGray, bg=}
\setbeamercolor{title in head/foot}{fg=FooterGray, bg=}
\setbeamercolor{date in head/foot}{fg=FooterGray, bg=}
\setbeamercolor{section in head/foot}{fg=FooterGray, bg=}
\setbeamercolor{subsection in head/foot}{fg=FooterGray, bg=}

% Bullet styles (apply everywhere including blocks)
\setbeamertemplate{itemize item}{\color{MainBlue}$\boxdot$}
\setbeamertemplate{itemize subitem}{\color{MainBlue}$\blacktriangleright$}
\setbeamertemplate{itemize subsubitem}{\color{MainBlue}\tiny$\bullet$}
\setbeamertemplate{itemize/enumerate body begin}{\normalsize}
\setbeamertemplate{itemize/enumerate subbody begin}{\normalsize}

% Item spacing - compact style
\setlength{\leftmargini}{10pt}       % Level 1: minimal indent
\setlength{\leftmarginii}{10pt}      % Level 2: minimal additional indent
% Compact list spacing (zero extra space before/after lists in blocks)
\makeatletter
\def\@listi{\leftmargin\leftmargini \topsep 0pt \parsep 0pt \itemsep 0pt}
\def\@listii{\leftmargin\leftmarginii \topsep 0pt \parsep 0pt \itemsep 0pt}
\makeatother

\setbeamertemplate{navigation symbols}{}

%=============================================================================
% CUSTOM HEADLINE
%=============================================================================
\setbeamertemplate{headline}{%
    \vskip10pt%
    \hbox to \paperwidth{%
        \hskip0.5cm%
        {\small\color{FooterGray}\renewcommand{\hyperlink}[2]{##2}\insertsectionhead}%
        \hfill%
        \textcolor{FooterGray}{\small\insertframenumber}%
        \hskip0.5cm%
    }%
    \vskip4pt%
    {\color{FooterGray}\hrule height 0.4pt}%
}

%=============================================================================
% CUSTOM FOOTER
%=============================================================================
\usepackage{fontawesome5}

\setbeamertemplate{footline}{%
    {\color{FooterGray}\hrule height 0.4pt}%
    \vskip4pt%
    \hbox to \paperwidth{%
        \hskip0.5cm%
        \textcolor{FooterGray}{\small Time Series Analysis and Forecasting}%
        \hfill%
        \raisebox{-0.1em}{%
            \begin{tikzpicture}[x=0.08em, y=0.08em, line width=0.4pt]
                \draw[FooterGray] (0,3) -- (1,4) -- (2,3.5) -- (3,5) -- (4,4) -- (5,6) -- (6,5.5) -- (7,4) -- (8,5) -- (9,7) -- (10,6) -- (11,5) -- (12,6.5) -- (13,8) -- (14,7) -- (15,6) -- (16,7.5) -- (17,9) -- (18,8) -- (19,7) -- (20,8.5) -- (21,10) -- (22,9) -- (23,8) -- (24,9.5);
            \end{tikzpicture}%
        }%
        \hskip0.5cm%
    }%
    \vskip6pt%
}

%=============================================================================
% PACKAGES
%=============================================================================
\usepackage[utf8]{inputenc}
\usepackage[T1]{fontenc}
\usepackage[english]{babel}
\usepackage{amsmath, amssymb, amsthm}
\usepackage{mathtools}
\usepackage{bm}
\usepackage{tikz}
\usetikzlibrary{arrows.meta, positioning, shapes, calc, decorations.pathreplacing, shadings}
\usepackage{booktabs}
\usepackage{multirow}
\usepackage{array}
\usepackage{graphicx}
\usepackage{hyperref}
\usepackage{colortbl}
\usepackage{listings}
\lstset{basicstyle=\ttfamily\small, breaklines=true, frame=single, backgroundcolor=\color{VeryLightGray}}
\hypersetup{colorlinks=true, linkcolor=MainBlue, urlcolor=MainBlue}
\graphicspath{{../../logos/}{../../charts/}{../../photos/}}
\hfuzz=2pt  % Suppress tiny overfull warnings (<2pt)
\vfuzz=2pt  % Suppress tiny vertical overfull warnings (<2pt)

%=============================================================================
% QUANTLET COMMAND
%=============================================================================
\newcommand{\quantlet}[2]{%
    \hfill\href{#2}{%
        \raisebox{-0.15em}{\includegraphics[height=0.7em]{ql_logo.png}}%
        \textcolor{MainBlue}{\tiny\ #1}%
    }%
}

%=============================================================================
% CUSTOM TITLE PAGE
%=============================================================================
\defbeamertemplate*{title page}{hybrid}[1][]
{
    \vspace{0.2cm}
    % Logos row - top header (with clickable links)
    \begin{center}
        \href{https://www.ase.ro}{\includegraphics[height=1.0cm]{ase_logo.png}}\hspace{0.25cm}%
        \href{https://theida.net}{\includegraphics[height=1.0cm]{ida_logo.png}}\hspace{0.25cm}%
        \href{https://blockchain-research-center.com}{\includegraphics[height=1.0cm]{brc_logo.png}}\hspace{0.25cm}%
        \href{https://www.ai4efin.ase.ro}{\includegraphics[height=1.0cm]{ai4efin_logo.png}}\hspace{0.25cm}%
        \href{https://ipe.ro/new}{\includegraphics[height=1.0cm]{acad_logo.png}}\hspace{0.25cm}%
        \href{https://www.digital-finance-msca.com}{\includegraphics[height=1.0cm]{msca_logo.png}}%
    \end{center}

    \vspace{0.6cm}

    % Main title with Q logos on sides (with clickable links)
    \begin{center}
        \begin{minipage}{0.1\textwidth}
            \centering
            \href{https://quantlet.com}{\includegraphics[height=1.1cm]{ql_logo.png}}
        \end{minipage}%
        \begin{minipage}{0.78\textwidth}
            \centering
            {\LARGE\bfseries\usebeamercolor[fg]{title}\inserttitle}

            \vspace{0.3cm}

            {\usebeamerfont{subtitle}\usebeamercolor[fg]{title}\insertsubtitle}
        \end{minipage}%
        \begin{minipage}{0.1\textwidth}
            \centering
            \href{https://quantinar.com}{\includegraphics[height=1.1cm]{qr_logo.png}}
        \end{minipage}
    \end{center}

    \vspace{0.6cm}

    % Authors (left aligned)
    \hspace{0.5cm}{\usebeamerfont{author}\insertauthor}

    \vspace{0.3cm}

    % Institute/Affiliations (left aligned)
    \hspace{0.5cm}\begin{minipage}[t]{0.9\textwidth}
        \raggedright\small\insertinstitute
    \end{minipage}
}

%=============================================================================
% THEOREM ENVIRONMENTS
%=============================================================================
\theoremstyle{definition}
\setbeamertemplate{theorems}[numbered]
\newtheorem{defn}{Definition}
\newtheorem{thm}{Theorem}
\newtheorem{prop}{Proposition}
\newtheorem{rmk}{Remark}

%=============================================================================
% CENTRED MINIPAGE (no extra vertical space)
%=============================================================================
\newenvironment{cminipage}[1]{%
    \par\noindent\hfill\begin{minipage}{#1}\ignorespaces
}{%
    \end{minipage}\hfill\null\par
}

%=============================================================================
% CUSTOM COMMANDS
%=============================================================================
\newcommand{\E}{\mathbb{E}}
\newcommand{\Var}{\text{Var}}
\newcommand{\Cov}{\text{Cov}}
\newcommand{\Corr}{\text{Corr}}
\newcommand{\R}{\mathbb{R}}
\newcommand{\N}{\mathbb{N}}
\newcommand{\Z}{\mathbb{Z}}
\newcommand{\B}{\mathbf{B}}
\newcommand{\imark}{\textcolor{MainBlue}{\textbullet}}
\newcommand{\RMSE}{\text{RMSE}}
\newcommand{\MAE}{\text{MAE}}
\newcommand{\MAPE}{\text{MAPE}}
\newcommand{\correct}{\textcolor{Forest}{\checkmark}}
\newcommand{\incorrect}{\textcolor{Crimson}{\texttimes}}

% Boldface vector/matrix commands
\newcommand{\bY}{\mathbf{Y}}
\newcommand{\bX}{\mathbf{X}}
\newcommand{\bA}{\mathbf{A}}
\newcommand{\bB}{\mathbf{B}}
\newcommand{\bepsilon}{\boldsymbol{\varepsilon}}
\newcommand{\bvarepsilon}{\boldsymbol{\varepsilon}}
\newcommand{\bSigma}{\boldsymbol{\Sigma}}
\newcommand{\bPhi}{\boldsymbol{\Phi}}
\newcommand{\bGamma}{\boldsymbol{\Gamma}}
\newcommand{\bPi}{\boldsymbol{\Pi}}
\newcommand{\bc}{\mathbf{c}}
\newcommand{\balpha}{\boldsymbol{\alpha}}
\newcommand{\bbeta}{\boldsymbol{\beta}}

%=============================================================================
% TITLE INFORMATION
%=============================================================================
\title[Time Series Analysis]{Time Series Analysis and Forecasting}
\author[D.T. Pele]{Daniel Traian PELE}
\institute{Bucharest University of Economic Studies\\
IDA Institute Digital Assets\\
Blockchain Research Center\\
AI4EFin Artificial Intelligence for Energy Finance\\
Romanian Academy, Institute for Economic Forecasting\\
MSCA Digital Finance}
\date{}

\subtitle{Seminar 1: Stochastic Processes and Stationarity}

\begin{document}

{
\setbeamertemplate{headline}{}
\setbeamertemplate{footline}{}
\begin{frame}
    \titlepage
\end{frame}
}


\begin{frame}{Seminar Outline}
    \begin{cminipage}{0.95\textwidth}
    \begin{itemize}
        \item \textbf{Multiple Choice Quiz} -- Knowledge check
        \vspace{0.15cm}
        \item \textbf{True/False} -- Conceptual checks
        \vspace{0.15cm}
        \item \textbf{Calculation Exercises} -- Applied practice
        \vspace{0.15cm}
        \item \textbf{Worked Examples} -- Coding practice
        \vspace{0.15cm}
        \item \textbf{AI-Assisted Exercise} -- Critical thinking
        \vspace{0.15cm}
        \item \textbf{Summary} -- Key takeaways
    \end{itemize}
    \end{cminipage}
\end{frame}

%=============================================================================
% OVERVIEW
%=============================================================================
\section{Overview}

\begin{frame}{Key Formulas}
    \begin{cminipage}{0.95\textwidth}
    \begin{columns}[T]
        \begin{column}{0.48\textwidth}
            \textbf{Decomposition:}
            \begin{itemize}\setlength{\itemsep}{0pt}
                \item Additive: $X_t = T_t + S_t + \varepsilon_t$
                \item Multiplicative: $X_t = T_t \times S_t \times \varepsilon_t$
            \end{itemize}

            \vspace{0.3cm}

            \textbf{Exponential Smoothing:}
            \begin{itemize}\setlength{\itemsep}{0pt}
                \item SES: $\hat{X}_{t+1} = \alpha X_t + (1-\alpha)\hat{X}_t$
                \item Holt: adds trend $b_t$
                \item HW: adds seasonality $S_t$
            \end{itemize}
        \end{column}
        \begin{column}{0.48\textwidth}
            \textbf{Stationarity:}
            \begin{itemize}\setlength{\itemsep}{0pt}
                \item $\E[X_t] = \mu$ (constant)
                \item $\Var(X_t) = \sigma^2$ (constant)
                \item $\Cov(X_t, X_{t+h}) = \gamma(h)$
            \end{itemize}

            \vspace{0.3cm}

            \textbf{Random walk:}
            \begin{itemize}\setlength{\itemsep}{0pt}
                \item $X_t = X_{t-1} + \varepsilon_t$
                \item $\Var(X_t) = t\sigma^2$ (grows with time)
            \end{itemize}
        \end{column}
    \end{columns}
    \end{cminipage}
\end{frame}

\begin{frame}{Summary: Concepts and Methods}
    \begin{cminipage}{0.95\textwidth}
    \begin{center}
    \small
    \begin{tabular}{lll}
        \toprule
        \textbf{Concept} & \textbf{Key idea} & \textbf{When to use} \\
        \midrule
        Additive decomposition & Constant seasonal amplitude & Stable variance \\
        Multiplicative decomposition & Seasonality grows with level & Increasing variance \\
        SES & Level only ($\alpha$) & No trend, no seasonality \\
        Holt & Level + Trend ($\alpha, \beta$) & Trend, no seasonality \\
        Holt-Winters & Level + Trend + Seasonality & Trend and seasonality \\
        \midrule
        ADF Test & $H_0$: unit root & Test for non-stationarity \\
        KPSS Test & $H_0$: stationary & Confirm stationarity \\
        \midrule
        Differencing & Remove stochastic trend & Random walk, unit root \\
        Regression & Remove deterministic trend & Linear/polynomial trend \\
        \bottomrule
    \end{tabular}
    \end{center}
    \end{cminipage}
\end{frame}

%=============================================================================
% MULTIPLE CHOICE QUIZ
%=============================================================================
\section{Multiple Choice Quiz}

\begin{frame}{Quiz 1: Stationarity}
    \begin{cminipage}{0.95\textwidth}
    \begin{alertblock}{Question}
        A random walk process $X_t = X_{t-1} + \varepsilon_t$ is:
    \end{alertblock}

    \vspace{0.4cm}

    \begin{block}{Answer choices}
        \textcolor{MainBlue}{\textbf{(A)}} Strictly stationary\\[3pt]
        \textcolor{MainBlue}{\textbf{(B)}} Weakly stationary\\[3pt]
        \textcolor{MainBlue}{\textbf{(C)}} Non-stationary because variance grows with time\\[3pt]
        \textcolor{MainBlue}{\textbf{(D)}} Stationary after adding a constant
    \end{block}

    \vspace{0.5cm}

    \begin{center}
        \textit{Answer on next slide...}
    \end{center}
    \end{cminipage}
\end{frame}

\begin{frame}{Quiz 1: Answer}
    \begin{cminipage}{0.95\textwidth}
    \begin{exampleblock}{Answer: C -- Non-stationary because variance grows with time}
    {\small
    \begin{block}{Answer choices}
        \textcolor{MainBlue}{\textbf{(A)}} Strictly stationary \incorrect\\[3pt]
        \textcolor{MainBlue}{\textbf{(B)}} Weakly stationary \incorrect\\[3pt]
        \textcolor{MainBlue}{\textbf{(C)}} \textbf{\textcolor{Forest}{Non-stationary because variance grows with time}} \correct\\[3pt]
        \textcolor{MainBlue}{\textbf{(D)}} Stationary after adding a constant \incorrect
    \end{block}

        \vspace{0.1cm}

        \begin{itemize}
            \item For random walk: $X_t = \sum_{i=1}^{t} \varepsilon_i$
            \item $\E[X_t] = 0$ (constant mean -- OK)
            \item $\Var(X_t) = t\sigma^2$ (variance depends on $t$ -- NOT OK!)
            \item \textbf{Solution}: differencing gives $\Delta X_t = \varepsilon_t$ --- stationary
        \end{itemize}
    }
    \end{exampleblock}

    \end{cminipage}
\end{frame}

\begin{frame}{Visual: Random Walk vs Stationary}
    \begin{cminipage}{0.95\textwidth}
    \begin{center}
        \includegraphics[width=0.95\textwidth, height=0.55\textheight, keepaspectratio]{ch3_def_random_walk.pdf}
    \end{center}
    \vspace{-0.2cm}
    {\small
    \begin{itemize}\setlength{\itemsep}{0pt}
        \item Random walk trajectories wander unpredictably
        \item Variance grows linearly with time $\Rightarrow$ non-stationary
    \end{itemize}
    }

    \end{cminipage}
    \quantlet{TSA\_ch1\_random\_walk}{https://github.com/QuantLet/TSA/tree/main/TSA_ch1/TSA_ch1_random_walk}
\end{frame}

\begin{frame}{Quiz 2: Unit Root Tests}
    \begin{cminipage}{0.95\textwidth}
    \begin{alertblock}{Question}
        You run ADF and KPSS tests. ADF fails to reject $H_0$, and KPSS rejects $H_0$. What do you conclude?
    \end{alertblock}

    \vspace{0.4cm}

    \begin{block}{Answer choices}
        \textcolor{MainBlue}{\textbf{(A)}} The series is stationary\\[3pt]
        \textcolor{MainBlue}{\textbf{(B)}} The series has a unit root (non-stationary)\\[3pt]
        \textcolor{MainBlue}{\textbf{(C)}} The results are inconclusive\\[3pt]
        \textcolor{MainBlue}{\textbf{(D)}} Additional tests are needed
    \end{block}

    \vspace{0.5cm}

    \begin{center}
        \textit{Answer on next slide...}
    \end{center}
    \end{cminipage}
\end{frame}

\begin{frame}{Quiz 2: Answer}
    \begin{cminipage}{0.95\textwidth}
    \begin{exampleblock}{Answer: B -- The series has a unit root (non-stationary)}
    {\small
    \begin{block}{Answer choices}
        \textcolor{MainBlue}{\textbf{(A)}} The series is stationary \incorrect\\[3pt]
        \textcolor{MainBlue}{\textbf{(B)}} \textbf{\textcolor{Forest}{The series has a unit root (non-stationary)}} \correct\\[3pt]
        \textcolor{MainBlue}{\textbf{(C)}} The results are inconclusive \incorrect\\[3pt]
        \textcolor{MainBlue}{\textbf{(D)}} Additional tests are needed \incorrect
    \end{block}

        \vspace{0.1cm}

        \begin{itemize}
            \item ADF: $H_0$ = unit root. Fail to reject $\Rightarrow$ evidence FOR unit root
            \item KPSS: $H_0$ = stationary. Reject $\Rightarrow$ evidence AGAINST stationarity
            \item Both tests agree: the series is \textbf{non-stationary}
            \item \textbf{Next step}: difference the series before modeling with ARMA
        \end{itemize}
    }
    \end{exampleblock}

    \end{cminipage}
\end{frame}

\begin{frame}{Quiz 3: Trend Types}
    \begin{cminipage}{0.95\textwidth}
    \begin{alertblock}{Question}
        A deterministic trend can be removed by:
    \end{alertblock}

    \vspace{0.4cm}

    \begin{block}{Answer choices}
        \textcolor{MainBlue}{\textbf{(A)}} Differencing\\[3pt]
        \textcolor{MainBlue}{\textbf{(B)}} Regression on time\\[3pt]
        \textcolor{MainBlue}{\textbf{(C)}} Seasonal adjustment\\[3pt]
        \textcolor{MainBlue}{\textbf{(D)}} Moving average smoothing
    \end{block}

    \vspace{0.5cm}

    \begin{center}
        \textit{Answer on next slide...}
    \end{center}
    \end{cminipage}
\end{frame}

\begin{frame}{Quiz 3: Answer}
    \begin{cminipage}{0.95\textwidth}
    \begin{exampleblock}{Answer: B -- Regression on time}
    {\small
    \begin{block}{Answer choices}
        \textcolor{MainBlue}{\textbf{(A)}} Differencing \incorrect\\[3pt]
        \textcolor{MainBlue}{\textbf{(B)}} \textbf{\textcolor{Forest}{Regression on time}} \correct\\[3pt]
        \textcolor{MainBlue}{\textbf{(C)}} Seasonal adjustment \incorrect\\[3pt]
        \textcolor{MainBlue}{\textbf{(D)}} Moving average smoothing \incorrect
    \end{block}

        \vspace{0.1cm}

        \begin{itemize}
            \item \textbf{Deterministic trend}: $Y_t = \alpha + \beta t + \varepsilon_t$ ($\beta$ fixed)
            \item \textbf{Removal method}: regress $Y_t$ on $t$, analyze residuals $\hat{\varepsilon}_t$
            \item \textbf{Why not differencing?}
                \begin{itemize}
                    \item Differencing gives $\Delta Y_t = \beta + \Delta\varepsilon_t$ --- removes the trend but leaves a constant
                    \item Differencing is correct only for \textit{stochastic} trends (unit roots)
                \end{itemize}
        \end{itemize}
    }
    \end{exampleblock}

    \end{cminipage}
\end{frame}

\begin{frame}{Quiz 4: ACF Interpretation}
    \begin{cminipage}{0.95\textwidth}
    \begin{alertblock}{Question}
        If the ACF of a time series decays very slowly (remains significant for many lags), this suggests:
    \end{alertblock}

    \vspace{0.4cm}

    \begin{block}{Answer choices}
        \textcolor{MainBlue}{\textbf{(A)}} The series is white noise\\[3pt]
        \textcolor{MainBlue}{\textbf{(B)}} The series is likely non-stationary\\[3pt]
        \textcolor{MainBlue}{\textbf{(C)}} The series has no autocorrelation\\[3pt]
        \textcolor{MainBlue}{\textbf{(D)}} The series is perfectly predictable
    \end{block}

    \vspace{0.5cm}

    \begin{center}
        \textit{Answer on next slide...}
    \end{center}
    \end{cminipage}
\end{frame}

\begin{frame}{Quiz 4: Answer}
    \begin{cminipage}{0.95\textwidth}
    \begin{exampleblock}{Answer: B -- The series is likely non-stationary}
        \vspace{-0.2cm}
        \begin{center}
            \includegraphics[width=0.95\textwidth, height=0.52\textheight, keepaspectratio]{sem1_acf_decay.pdf}
        \end{center}
        \vspace{-0.2cm}
        {\footnotesize
        \begin{itemize}\setlength{\itemsep}{0pt}
            \item \textbf{Stationary}: ACF decays quickly ($\rho_k = \phi^k \to 0$)
            \item \textbf{Non-stationary}: ACF stays near 1 $\Rightarrow$ differencing needed
        \end{itemize}
        }
    \end{exampleblock}

    \end{cminipage}
    \quantlet{TSA\_ch1\_acf\_patterns}{https://github.com/QuantLet/TSA/tree/main/TSA_ch1/TSA_ch1_acf_patterns}
\end{frame}

\begin{frame}{Quiz 5: Holt's Method}
    \begin{cminipage}{0.95\textwidth}
    \begin{alertblock}{Question}
        Holt's exponential smoothing differs from SES by adding:
    \end{alertblock}

    \vspace{0.4cm}

    \begin{block}{Answer choices}
        \textcolor{MainBlue}{\textbf{(A)}} A seasonal component\\[3pt]
        \textcolor{MainBlue}{\textbf{(B)}} A trend component\\[3pt]
        \textcolor{MainBlue}{\textbf{(C)}} A cyclical component\\[3pt]
        \textcolor{MainBlue}{\textbf{(D)}} An irregular component
    \end{block}

    \vspace{0.5cm}

    \begin{center}
        \textit{Answer on next slide...}
    \end{center}
    \end{cminipage}
\end{frame}

\begin{frame}{Quiz 5: Answer}
    \begin{cminipage}{0.95\textwidth}
    \begin{exampleblock}{Answer: B -- A trend component}
        \vspace{-0.2cm}
        \begin{center}
            \includegraphics[width=0.95\textwidth, height=0.52\textheight, keepaspectratio]{sem1_holt_method.pdf}
        \end{center}
        \vspace{-0.2cm}
        {\footnotesize
        \begin{itemize}\setlength{\itemsep}{0pt}
            \item \textbf{Holt}: $L_t = \alpha Y_t + (1-\alpha)(L_{t-1} + b_{t-1})$; \; $b_t = \beta(L_t - L_{t-1}) + (1-\beta)b_{t-1}$
            \item \textbf{Forecast}: $\hat{Y}_{t+h} = L_t + h \cdot b_t$
        \end{itemize}
        }
    \end{exampleblock}

    \end{cminipage}
    \quantlet{TSA\_ch1\_ts\_basics}{https://github.com/QuantLet/TSA/tree/main/TSA_ch1/TSA_ch1_ts_basics}
\end{frame}

\begin{frame}{Quiz 6: White Noise}
    \begin{cminipage}{0.95\textwidth}
    \begin{alertblock}{Question}
        Which property is NOT required for a process to be white noise?
    \end{alertblock}

    \vspace{0.4cm}

    \begin{block}{Answer choices}
        \textcolor{MainBlue}{\textbf{(A)}} $\E[\varepsilon_t] = 0$\\[3pt]
        \textcolor{MainBlue}{\textbf{(B)}} $\Var(\varepsilon_t) = \sigma^2$ (constant)\\[3pt]
        \textcolor{MainBlue}{\textbf{(C)}} $\Cov(\varepsilon_t, \varepsilon_s) = 0$ for $t \neq s$\\[3pt]
        \textcolor{MainBlue}{\textbf{(D)}} $\varepsilon_t \sim N(0, \sigma^2)$
    \end{block}

    \vspace{0.5cm}

    \begin{center}
        \textit{Answer on next slide...}
    \end{center}
    \end{cminipage}
\end{frame}

\begin{frame}{Quiz 6: Answer}
    \begin{cminipage}{0.95\textwidth}
    \begin{exampleblock}{Answer: D -- Normality is NOT required}
        \vspace{-0.2cm}
        \begin{center}
            \includegraphics[width=0.95\textwidth, height=0.52\textheight, keepaspectratio]{sem1_white_noise.pdf}
        \end{center}
        \vspace{-0.2cm}
        {\footnotesize
        \begin{itemize}\setlength{\itemsep}{0pt}
            \item \textbf{White noise}: zero mean, constant variance, uncorrelated
            \item \textbf{Gaussian white noise}: adds normality $\Rightarrow$ independent (not just uncorrelated)
        \end{itemize}
        }
    \end{exampleblock}

    \end{cminipage}
    \quantlet{TSA\_ch1\_white\_noise}{https://github.com/QuantLet/TSA/tree/main/TSA_ch1/TSA_ch1_white_noise}
\end{frame}

\begin{frame}{Visual: White Noise Properties}
    \begin{cminipage}{0.95\textwidth}
    \begin{center}
        \includegraphics[width=0.95\textwidth, height=0.55\textheight, keepaspectratio]{ch1_def_white_noise.pdf}
    \end{center}
    \vspace{-0.2cm}
    {\small
    \begin{itemize}\setlength{\itemsep}{0pt}
        \item \textbf{Left}: white noise fluctuates around zero
        \item \textbf{Right}: ACF shows no autocorrelation (all values $\approx 0$ after lag 0)
    \end{itemize}
    }

    \end{cminipage}
    \quantlet{TSA\_ch1\_white\_noise}{https://github.com/QuantLet/TSA/tree/main/TSA_ch1/TSA_ch1_white_noise}
\end{frame}

\begin{frame}{Quiz 7: Forecast Horizon}
    \begin{cminipage}{0.95\textwidth}
    \begin{alertblock}{Question}
        As forecast horizon $h$ increases, what typically happens to forecast intervals?
    \end{alertblock}

    \vspace{0.4cm}

    \begin{block}{Answer choices}
        \textcolor{MainBlue}{\textbf{(A)}} They become narrower\\[3pt]
        \textcolor{MainBlue}{\textbf{(B)}} They stay the same width\\[3pt]
        \textcolor{MainBlue}{\textbf{(C)}} They become wider\\[3pt]
        \textcolor{MainBlue}{\textbf{(D)}} They disappear
    \end{block}

    \vspace{0.5cm}

    \begin{center}
        \textit{Answer on next slide...}
    \end{center}
    \end{cminipage}
\end{frame}

\begin{frame}{Quiz 7: Answer}
    \begin{cminipage}{0.95\textwidth}
    \begin{exampleblock}{Answer: C -- They become wider}
        \vspace{-0.2cm}
        \begin{center}
            \includegraphics[width=0.95\textwidth, height=0.52\textheight, keepaspectratio]{sem1_forecast_intervals.pdf}
        \end{center}
        \vspace{-0.2cm}
        {\footnotesize
        \begin{itemize}\setlength{\itemsep}{0pt}
            \item \textbf{Random walk}: $\Var = h\sigma^2$ (grows linearly)
            \item \textbf{95\% CI}: $\hat{Y}_{t+h} \pm 1.96\sqrt{h}\sigma$ (widens with $\sqrt{h}$)
        \end{itemize}
        }
    \end{exampleblock}

    \end{cminipage}
    \quantlet{TSA\_ch1\_ts\_basics}{https://github.com/QuantLet/TSA/tree/main/TSA_ch1/TSA_ch1_ts_basics}
\end{frame}

\begin{frame}{Quiz 8: Seasonality Detection}
    \begin{cminipage}{0.95\textwidth}
    \begin{alertblock}{Question}
        The ACF shows significant spikes at lags 12, 24, and 36 for monthly data. This suggests:
    \end{alertblock}

    \vspace{0.4cm}

    \begin{block}{Answer choices}
        \textcolor{MainBlue}{\textbf{(A)}} No seasonality\\[3pt]
        \textcolor{MainBlue}{\textbf{(B)}} Annual seasonality\\[3pt]
        \textcolor{MainBlue}{\textbf{(C)}} Weekly seasonality\\[3pt]
        \textcolor{MainBlue}{\textbf{(D)}} Random noise
    \end{block}

    \vspace{0.5cm}

    \begin{center}
        \textit{Answer on next slide...}
    \end{center}
    \end{cminipage}
\end{frame}

\begin{frame}{Quiz 8: Answer}
    \begin{cminipage}{0.95\textwidth}
    \begin{exampleblock}{Answer: B -- Annual seasonality}
    {\footnotesize
    \begin{block}{Answer choices}
        \textcolor{MainBlue}{\textbf{(A)}} No seasonality \incorrect\\[2pt]
        \textcolor{MainBlue}{\textbf{(B)}} \textbf{\textcolor{Forest}{Annual seasonality}} \correct\\[2pt]
        \textcolor{MainBlue}{\textbf{(C)}} Weekly seasonality \incorrect\\[2pt]
        \textcolor{MainBlue}{\textbf{(D)}} Random noise \incorrect
    \end{block}

        \vspace{0.1cm}

        \begin{itemize}\setlength{\itemsep}{1pt}
            \item \textbf{Pattern recognition}:
                \begin{itemize}\setlength{\itemsep}{0pt}
                    \item Lag 12: correlation with the same month last year
                    \item Lag 24: same month two years ago
                    \item Lag 36: same month three years ago
                \end{itemize}
            \item \textbf{Seasonal period}: $s = 12$ (monthly data with annual cycle)
            \item \textbf{Common patterns}: retail sales (December), energy consumption (summer/winter), tourism
        \end{itemize}
    }
    \end{exampleblock}

    \end{cminipage}
\end{frame}

\begin{frame}{Quiz 9: MAPE Limitation}
    \begin{cminipage}{0.95\textwidth}
    \begin{alertblock}{Question}
        MAPE (Mean Absolute Percentage Error) should NOT be used when:
    \end{alertblock}

    \vspace{0.4cm}

    \begin{block}{Answer choices}
        \textcolor{MainBlue}{\textbf{(A)}} Comparing models on the same dataset\\[3pt]
        \textcolor{MainBlue}{\textbf{(B)}} The actual values can be zero or near zero\\[3pt]
        \textcolor{MainBlue}{\textbf{(C)}} Forecasting stock prices\\[3pt]
        \textcolor{MainBlue}{\textbf{(D)}} The data has a trend
    \end{block}

    \vspace{0.5cm}

    \begin{center}
        \textit{Answer on next slide...}
    \end{center}
    \end{cminipage}
\end{frame}

\begin{frame}{Quiz 9: Answer}
    \begin{cminipage}{0.95\textwidth}
    \begin{exampleblock}{Answer: B -- When actual values can be zero or near zero}
    {\footnotesize
    \begin{block}{Answer choices}
        \textcolor{MainBlue}{\textbf{(A)}} Comparing models on the same dataset \incorrect\\[2pt]
        \textcolor{MainBlue}{\textbf{(B)}} \textbf{\textcolor{Forest}{The actual values can be zero or near zero}} \correct\\[2pt]
        \textcolor{MainBlue}{\textbf{(C)}} Forecasting stock prices \incorrect\\[2pt]
        \textcolor{MainBlue}{\textbf{(D)}} The data has a trend \incorrect
    \end{block}

        \vspace{0.1cm}

        \begin{itemize}\setlength{\itemsep}{1pt}
            \item \textbf{Formula}: $\text{MAPE} = \frac{100\%}{n}\sum_{t=1}^{n}\left|\frac{Y_t - \hat{Y}_t}{Y_t}\right|$
            \item \textbf{Problem}: $Y_t \approx 0$ $\Rightarrow$ MAPE $\to \infty$
            \item \textbf{Alternatives}:
                \begin{itemize}\setlength{\itemsep}{0pt}
                    \item \textbf{SMAPE}: $\frac{200\%}{n}\sum\frac{|Y_t - \hat{Y}_t|}{|Y_t| + |\hat{Y}_t|}$ (bounded 0--200\%)
                    \item \textbf{MASE}: $\frac{1}{n}\sum\frac{|e_t|}{\frac{1}{n-1}\sum|Y_t - Y_{t-1}|}$ (scale-free)
                \end{itemize}
        \end{itemize}
    }
    \end{exampleblock}

    \end{cminipage}
\end{frame}

%=============================================================================
% TRUE/FALSE
%=============================================================================
\section{True/False}

\begin{frame}{True or False? (Set 1)}
    \begin{cminipage}{0.95\textwidth}
    \footnotesize
    \begin{center}
    \begin{tabular}{p{9cm}c}
        \toprule
        \textbf{Statement} & \textbf{T/F?} \\
        \midrule
        1. A time series with constant mean is always stationary. & ? \\[0.15cm]
        2. The variance of a random walk increases linearly with time. & ? \\[0.15cm]
        3. A stationary process can have time-varying variance. & ? \\[0.15cm]
        4. ADF and KPSS tests have the same null hypothesis. & ? \\[0.15cm]
        5. Lower RMSE always means better forecasts. & ? \\[0.15cm]
        6. Autocorrelation at lag 0 is always equal to 1. & ? \\
        \bottomrule
    \end{tabular}
    \end{center}
    \end{cminipage}
\end{frame}

\begin{frame}{True or False? --- Answers (Set 1)}
    \begin{cminipage}{0.95\textwidth}
    \scriptsize
    \begin{center}
    \begin{tabular}{p{7.5cm}cc}
        \toprule
        \textbf{Statement} & \textbf{T/F} & \textbf{Explanation} \\
        \midrule
        1. Constant mean $\Rightarrow$ stationary. & \textcolor{Crimson}{\textbf{F}} & {\tiny Also need constant var.\ \& cov.} \\[0.08cm]
        2. $\Var$ of random walk grows linearly. & \textcolor{Forest}{\textbf{T}} & {\tiny $\Var(X_t) = t\sigma^2$} \\[0.08cm]
        3. Stationary can have time-varying var. & \textcolor{Crimson}{\textbf{F}} & {\tiny $\Var(X_t) = \sigma^2$ constant} \\[0.08cm]
        4. ADF and KPSS same $H_0$. & \textcolor{Crimson}{\textbf{F}} & {\tiny Opposite $H_0$!} \\[0.08cm]
        5. Lower RMSE $\Rightarrow$ better. & \textcolor{Crimson}{\textbf{F}} & {\tiny Scale-dependent; may overfit} \\[0.08cm]
        6. $\rho(0) = 1$ always. & \textcolor{Forest}{\textbf{T}} & {\tiny $\rho(0) = \gamma(0)/\gamma(0) = 1$} \\
        \bottomrule
    \end{tabular}
    \end{center}
    \end{cminipage}
\end{frame}

\begin{frame}{True or False? (Set 2)}
    \begin{cminipage}{0.95\textwidth}
    \footnotesize
    \begin{center}
    \begin{tabular}{p{9cm}c}
        \toprule
        \textbf{Statement} & \textbf{T/F?} \\
        \midrule
        1. The ACF of a stationary AR(1) process decays exponentially. & ? \\[0.15cm]
        2. White noise is always normally distributed. & ? \\[0.15cm]
        3. Differencing can make a non-stationary series stationary. & ? \\[0.15cm]
        4. The PACF of a MA(1) process cuts off after lag 1. & ? \\[0.15cm]
        5. Zero correlation between two variables implies independence. & ? \\[0.15cm]
        6. Holt-Winters is appropriate for data with no seasonality. & ? \\
        \bottomrule
    \end{tabular}
    \end{center}
    \end{cminipage}
\end{frame}

\begin{frame}{True or False? --- Answers (Set 2)}
    \begin{cminipage}{0.95\textwidth}
    \scriptsize
    \begin{center}
    \begin{tabular}{p{7.5cm}cc}
        \toprule
        \textbf{Statement} & \textbf{T/F} & \textbf{Explanation} \\
        \midrule
        1. ACF of stationary AR(1) decays exp. & \textcolor{Forest}{\textbf{T}} & {\tiny $\rho(h) = \phi^h$} \\[0.08cm]
        2. White noise is always normal. & \textcolor{Crimson}{\textbf{F}} & {\tiny Gaussian = special case} \\[0.08cm]
        3. Differencing $\Rightarrow$ stationary. & \textcolor{Forest}{\textbf{T}} & {\tiny Removes stochastic trends} \\[0.08cm]
        4. PACF of MA(1) cuts off after lag 1. & \textcolor{Crimson}{\textbf{F}} & {\tiny ACF cuts off for MA; PACF decays} \\[0.08cm]
        5. Zero corr.\ $\Rightarrow$ independence. & \textcolor{Crimson}{\textbf{F}} & {\tiny Nonlinear dep.\ may exist} \\[0.08cm]
        6. HW appropriate for no seasonality. & \textcolor{Crimson}{\textbf{F}} & {\tiny Use Holt or SES instead} \\
        \bottomrule
    \end{tabular}
    \end{center}
    \end{cminipage}
\end{frame}

%=============================================================================
% CALCULATION EXERCISES
%=============================================================================
\section{Calculation Exercises}

\begin{frame}{Exercise 1: Autocovariance}
    \begin{cminipage}{0.95\textwidth}
    \begin{alertblock}{Problem}
        \begin{itemize}\setlength{\itemsep}{0pt}
            \item \textbf{Data}: Stationary process with $\E[X_t] = 5$, $\gamma(0) = 4$, $\gamma(1) = 2$, $\gamma(2) = 1$
            \item \textbf{Calculate}: a) $\rho(0), \rho(1), \rho(2)$; b) $\Cov(X_t, X_{t-1})$; c) $\Corr(X_5, X_7)$; d) $\E[X_{t+1} | X_t = 6]$ assuming AR(1)
        \end{itemize}
    \end{alertblock}

    \vspace{0.2cm}
    \begin{exampleblock}{Solution}
        \textbf{a) Autocorrelations:} $\rho(h) = \gamma(h)/\gamma(0)$
        \begin{itemize}\setlength{\itemsep}{0pt}
            \item $\rho(0) = 1$, \quad $\rho(1) = 2/4 = 0.5$, \quad $\rho(2) = 1/4 = 0.25$
        \end{itemize}
        \textbf{b)} $\Cov(X_t, X_{t-1}) = \gamma(1) = 2$ \quad (by stationarity, lag 1 covariance)

        \textbf{c)} $\Corr(X_5, X_7) = \rho(|7-5|) = \rho(2) = 0.25$

        \textbf{d)} For AR(1) with $\phi = \rho(1) = 0.5$:
        $\E[X_{t+1} | X_t] = \mu + \phi(X_t - \mu) = 5 + 0.5(6-5) = 5.5$
    \end{exampleblock}
    \end{cminipage}
    \quantlet{TSA\_ch1\_ex1\_autocovariance}{https://github.com/QuantLet/TSA/tree/main/TSA_ch1/TSA_ch1_ex1_autocovariance}
\end{frame}

\begin{frame}{Exercise 2: Random Walk Properties}
    \begin{cminipage}{0.95\textwidth}
    \begin{alertblock}{Problem}
        \begin{itemize}\setlength{\itemsep}{0pt}
            \item \textbf{Data}: Random walk $X_t = X_{t-1} + \varepsilon_t$ where $\varepsilon_t \sim WN(0, 4)$ and $X_0 = 100$
            \item \textbf{Calculate}: a) $\E[X_{10}]$; b) $\Var(X_{10})$; c) $\Cov(X_5, X_{10})$; d) 95\% CI for $X_{100}$; e) Optimal forecast for $X_6$ given $X_5 = 108$
        \end{itemize}
    \end{alertblock}

    \vspace{0.2cm}
    \begin{exampleblock}{Solution}
        $X_t = X_0 + \sum_{i=1}^{t} \varepsilon_i$ with $\sigma^2 = 4$
        \begin{itemize}\setlength{\itemsep}{0pt}
            \item \textbf{a)} $\E[X_{10}] = X_0 = 100$ \quad (mean stays at starting value)
            \item \textbf{b)} $\Var(X_{10}) = 10 \times 4 = 40$
            \item \textbf{c)} $\Cov(X_5, X_{10}) = \min(5, 10) \times 4 = 20$
            \item \textbf{d)} $\E[X_{100}] = 100$, $SD = \sqrt{400} = 20$; \; 95\% CI: $100 \pm 1.96 \times 20 = [60.8, 139.2]$
            \item \textbf{e)} $\hat{X}_6 = X_5 = 108$ \quad (random walk: optimal forecast = last value)
        \end{itemize}
    \end{exampleblock}
    \end{cminipage}
    \quantlet{TSA\_ch1\_ex2\_random\_walk}{https://github.com/QuantLet/TSA/tree/main/TSA_ch1/TSA_ch1_ex2_random_walk}
\end{frame}

%=============================================================================
% WORKED EXAMPLES
%=============================================================================
\section{Worked Examples}

\begin{frame}[fragile]{Python Exercise 1: Import and Visualization}
    \begin{cminipage}{0.95\textwidth}
    \textbf{Task:} Import S\&P 500 data and create a basic time series plot.
    \vspace{0.2cm}
    \begin{block}{Starter Code}
    \footnotesize
    \begin{verbatim}
import yfinance as yf
import matplotlib.pyplot as plt
sp500 = yf.download('^GSPC', start='2020-01-01', end='2025-01-01')
# TODO: Plot the closing prices
# TODO: Add title and labels
# TODO: Calculate and display basic statistics
    \end{verbatim}
    \end{block}
    \textbf{Questions:}
    \begin{enumerate}\setlength{\itemsep}{0pt}
        \item What is the mean and standard deviation of returns?
        \item Does the series appear stationary? Justify your answer.
    \end{enumerate}

    \end{cminipage}
    \quantlet{TSA\_ch1\_ts\_basics}{https://github.com/QuantLet/TSA/tree/main/TSA_ch1/TSA_ch1_ts_basics}
\end{frame}

\begin{frame}[fragile]{Python Exercise 2: Decomposition}
    \begin{cminipage}{0.95\textwidth}
    \textbf{Task:} Apply STL decomposition on airline passengers data.

    \vspace{0.3cm}

    \begin{block}{Starter Code}
    \small
    \begin{verbatim}
from statsmodels.tsa.seasonal import STL
import pandas as pd

# Load airline passengers
url = 'https://raw.githubusercontent.com/..../airline.csv'
airline = pd.read_csv(url, parse_dates=['Month'],
                      index_col='Month')

# TODO: Apply STL decomposition with period=12
# TODO: Plot all components
# TODO: What percentage of variance is explained by trend?
    \end{verbatim}
    \end{block}

    \textbf{Hint:} \texttt{STL(data, period=12).fit()}
    \end{cminipage}
    \quantlet{TSA\_ch1\_python\_decomposition}{https://github.com/QuantLet/TSA/tree/main/TSA_ch1/TSA_ch1_python_decomposition}
\end{frame}

\begin{frame}[fragile]{Python Exercise 3: Exponential Smoothing}
    \begin{cminipage}{0.95\textwidth}
    \textbf{Task:} Compare SES, Holt, and Holt-Winters methods on real data.

    \vspace{0.3cm}

    \begin{block}{Starter Code}
    \small
    \begin{verbatim}
from statsmodels.tsa.holtwinters import (SimpleExpSmoothing,
    ExponentialSmoothing)

# Split data: 80% train, 20% test
train = airline[:'1958']
test = airline['1959':]

# TODO: Fit SES, Holt, and Holt-Winters
# TODO: Generate forecasts for the test period
# TODO: Calculate RMSE for each method
# TODO: Which method performs best? Why?
    \end{verbatim}
    \end{block}
    \end{cminipage}
    \quantlet{TSA\_ch1\_python\_smoothing}{https://github.com/QuantLet/TSA/tree/main/TSA_ch1/TSA_ch1_python_smoothing}
\end{frame}

\begin{frame}[fragile]{Python Exercise 4: Stationarity Testing}
    \begin{cminipage}{0.95\textwidth}
    \textbf{Task:} Test for stationarity using ADF and KPSS tests.
    \vspace{0.2cm}
    \begin{block}{Starter Code}
    \footnotesize
    \begin{verbatim}
from statsmodels.tsa.stattools import adfuller, kpss
prices = sp500['Close']
returns = prices.pct_change().dropna()
# TODO: Run ADF test on prices and returns
# TODO: Run KPSS test on prices and returns
# TODO: Interpret the results
# ADF: adfuller(series)  |  KPSS: kpss(series, regression='c')
    \end{verbatim}
    \end{block}
    \textbf{Questions:}
    \begin{enumerate}\setlength{\itemsep}{0pt}
        \item Are prices stationary? Are returns stationary?
        \item Do ADF and KPSS results agree?
    \end{enumerate}

    \end{cminipage}
    \quantlet{TSA\_ch1\_unit\_root\_tests}{https://github.com/QuantLet/TSA/tree/main/TSA_ch1/TSA_ch1_unit_root_tests}
\end{frame}

%=============================================================================
% REAL DATA ANALYSIS
%=============================================================================
\section{Real Data Analysis}

\begin{frame}{Case Study: S\&P 500 Index}
    \begin{cminipage}{0.95\textwidth}
    \vspace{0.3cm}
    \begin{center}
        \includegraphics[width=0.75\textwidth, height=0.38\textheight, keepaspectratio]{sp500_prices_returns.pdf}
    \end{center}
    \vspace{-0.1cm}
    \begin{block}{Observations}
    {\small
        \begin{itemize}\setlength{\itemsep}{2pt}
            \item \textbf{Top}: S\&P 500 prices --- clear upward trend (non-stationary)
            \item \textbf{Bottom}: Returns $r_t = \log(P_t/P_{t-1})$ --- stationary, fluctuations around zero mean
            \item Volatility clustering visible
        \end{itemize}
    }
    \end{block}

    \end{cminipage}
    \quantlet{TSA\_ch1\_ts\_basics}{https://github.com/QuantLet/TSA/tree/main/TSA_ch1/TSA_ch1_ts_basics}
\end{frame}

\begin{frame}{Time Series Decomposition: Real Example}
    \begin{cminipage}{0.95\textwidth}
    \vspace{0.3cm}
    \begin{center}
        \includegraphics[width=0.75\textwidth, height=0.38\textheight, keepaspectratio]{decomposition.pdf}
    \end{center}
    \vspace{-0.1cm}
    \begin{block}{Observations}
    {\small
        \begin{itemize}\setlength{\itemsep}{2pt}
            \item \textbf{Trend}: Long-term direction
            \item \textbf{Seasonality}: Regular periodic patterns
            \item \textbf{Residual}: What remains after removing trend and seasonality
            \item Decomposition $\Rightarrow$ understanding structure before modeling
        \end{itemize}
    }
    \end{block}

    \end{cminipage}
    \quantlet{TSA\_ch1\_case\_gdp}{https://github.com/QuantLet/TSA/tree/main/TSA_ch1/TSA_ch1_case_gdp}
\end{frame}

\begin{frame}{Stationarity Testing: ADF Results}
    \begin{cminipage}{0.95\textwidth}
    \vspace{0.3cm}
    \begin{center}
        \includegraphics[width=0.75\textwidth, height=0.38\textheight, keepaspectratio]{adf_test_visualization.pdf}
    \end{center}
    \vspace{-0.1cm}
    \begin{block}{Observations}
    {\small
        \begin{itemize}\setlength{\itemsep}{2pt}
            \item ADF compares test statistic to critical values
            \item Test stat.\ $<$ crit.\ value $\Rightarrow$ reject $H_0$ (stationary)
            \item \textbf{Prices}: ADF $> -2.86$ $\Rightarrow$ non-stationary
            \item \textbf{Returns}: ADF $< -2.86$ $\Rightarrow$ stationary
        \end{itemize}
    }
    \end{block}

    \end{cminipage}
    \quantlet{TSA\_ch1\_unit\_root\_tests}{https://github.com/QuantLet/TSA/tree/main/TSA_ch1/TSA_ch1_unit_root_tests}
\end{frame}

\begin{frame}{Stationarity Comparison: Prices vs Returns}
    \begin{cminipage}{0.95\textwidth}
    {\small
    \begin{block}{ADF Test Results}
        \begin{center}
        \begin{tabular}{lccc}
            \toprule
            \textbf{Series} & \textbf{ADF Statistic} & \textbf{p-value} & \textbf{Conclusion} \\
            \midrule
            S\&P 500 Prices & $-0.82$ & $0.812$ & Non-stationary \\
            S\&P 500 Returns & $-45.3$ & $<0.001$ & Stationary \\
            \bottomrule
        \end{tabular}
        \end{center}
    \end{block}

    \vspace{0.3cm}

    \begin{exampleblock}{Key Insight}
        \begin{itemize}\setlength{\itemsep}{0pt}
            \item Financial prices are typically $I(1)$ -- integrated of order 1
            \item Taking first differences (returns) achieves stationarity
            \item This is why we model \textbf{returns}, not prices!
        \end{itemize}
    \end{exampleblock}
    }

    \end{cminipage}
    \quantlet{TSA\_ch1\_unit\_root\_tests}{https://github.com/QuantLet/TSA/tree/main/TSA_ch1/TSA_ch1_unit_root_tests}
\end{frame}

\begin{frame}{Exponential Smoothing Forecast}
    \begin{cminipage}{0.95\textwidth}
    \vspace{0.3cm}
    \begin{center}
        \includegraphics[width=0.75\textwidth, height=0.38\textheight, keepaspectratio]{holt_winters.pdf}
    \end{center}
    \vspace{-0.1cm}
    \begin{block}{Observations}
    {\small
        \begin{itemize}\setlength{\itemsep}{2pt}
            \item Holt-Winters: data with trend + seasonality
            \item $\alpha, \beta, \gamma$ control adaptiveness
            \item Captures trend continuation + seasonal pattern
            \item Simple yet effective for business applications
        \end{itemize}
    }
    \end{block}

    \end{cminipage}
    \quantlet{TSA\_ch1\_ts\_basics}{https://github.com/QuantLet/TSA/tree/main/TSA_ch1/TSA_ch1_ts_basics}
\end{frame}

%=============================================================================
% DISCUSSION TOPICS
%=============================================================================
\section{Discussion Topics}

\begin{frame}{Discussion Question 1}
    \begin{cminipage}{0.95\textwidth}
    \begin{block}{Scenario}
        \begin{itemize}\setlength{\itemsep}{1pt}
            \item Monthly sales data for a retail company
            \item Clear seasonality (high sales in December) + upward trend
            \item Seasonal peaks have been getting larger over time
        \end{itemize}
    \end{block}

    \vspace{0.4cm}

    \textbf{Discuss:}
    \begin{enumerate}
        \item Additive or multiplicative decomposition? Justify your answer.
        \item Which exponential smoothing method would you recommend?
        \item How would you evaluate forecast performance?
        \item What are the risks of choosing the wrong decomposition?
    \end{enumerate}
    \end{cminipage}
\end{frame}

\begin{frame}{Discussion Question 2}
    \begin{cminipage}{0.95\textwidth}
    \begin{block}{Scenario}
        A colleague claims:
        \begin{itemize}\setlength{\itemsep}{1pt}
            \item ``I ran the ADF test on my stock price data''
            \item ``I got a p-value of 0.65''
            \item ``So my data is stationary and I can fit an ARMA model directly''
        \end{itemize}
    \end{block}

    \vspace{0.4cm}

    \textbf{Discuss:}
    \begin{enumerate}
        \item Where is the reasoning flawed?
        \item What are the hypotheses of the ADF test?
        \item What steps should be taken before estimating an ARMA model?
        \item What role does the KPSS test play in clarifying the situation?
    \end{enumerate}
    \end{cminipage}
\end{frame}

\begin{frame}{Discussion Question 3}
    \begin{cminipage}{0.95\textwidth}
    \begin{block}{Scenario}
        \begin{itemize}\setlength{\itemsep}{1pt}
            \item You build a forecasting model: MAPE of 2\%
            \item The manager is impressed and wants immediate deployment
        \end{itemize}
    \end{block}

    \vspace{0.4cm}

    \textbf{Discuss:}
    \begin{enumerate}
        \item What checks are needed before deployment?
        \item Is the train/validation/test split correct?
        \item Is there a risk of data leakage?
        \item What additional checks are needed?
        \item How would you monitor model performance in production?
    \end{enumerate}
    \end{cminipage}
\end{frame}

\begin{frame}{Discussion Question 4}
    \begin{cminipage}{0.95\textwidth}
    \begin{block}{Scenario}
        Daily electricity demand forecast for the next week:
        \begin{itemize}\setlength{\itemsep}{1pt}
            \item Strong daily patterns (peaks at 6pm)
            \item Weekly patterns (lower on weekends)
            \item Annual patterns (higher in summer/winter)
        \end{itemize}
    \end{block}

    \vspace{0.4cm}

    \textbf{Discuss:}
    \begin{enumerate}
        \item How would you handle multiple seasonality?
        \item Is Holt-Winters appropriate? Justify your answer.
        \item What is the advantage of Fourier terms in this case?
        \item How would you set up train/validation/test samples?
    \end{enumerate}
    \end{cminipage}
\end{frame}

%=============================================================================
% AI-ASSISTED EXERCISE
%=============================================================================
\section{AI-Assisted Exercise}

\begin{frame}{AI Exercise: Critical Thinking}
    \begin{cminipage}{0.95\textwidth}
    \vspace{-0.3cm}
    \begin{block}{\footnotesize Prompt to test in ChatGPT / Claude / Copilot}
        {\footnotesize
        ``Using yfinance, download the BTC-USD price series. Is the price series stationary? If not, transform it so that it becomes stationary. Can I use the returns series to predict future prices? Show me.''
        }
    \end{block}
    \vspace{-2mm}
    {\footnotesize
    \textbf{Exercise}:
    \begin{enumerate}\setlength{\itemsep}{0pt}
        \item Run the prompt in an LLM of your choice and critically analyze the response.
        \item Does the AI correctly interpret the ADF and KPSS null hypotheses?
        \item Does it stop at first differencing or over-difference? Why does this matter?
        \item Check the conclusion about returns --- does ``white noise'' mean ``no structure''?
        \item Does it account for volatility clustering (ARCH effects) in the returns?
    \end{enumerate}
    }
    \vspace{-2mm}
    \begin{alertblock}{}
        {\footnotesize \textbf{Warning}: AI-generated code may run without errors and look professional. \textit{That does not mean it is correct.}}
    \end{alertblock}
    \end{cminipage}
\end{frame}

%=============================================================================
% SUMMARY
%=============================================================================
\section{Summary}

\begin{frame}{Summary: Chapter 1}
    \begin{cminipage}{0.95\textwidth}
    \begin{exampleblock}{Key Concepts}
        \begin{itemize}\setlength{\itemsep}{2pt}
            \item[\textcolor{MainBlue}{\textbf{1.}}] \textbf{Time series are dependent}: not i.i.d.\ like cross-sectional data --- autocorrelation is key
            \item[\textcolor{MainBlue}{\textbf{2.}}] \textbf{Decomposition}: choose wisely --- multiplicative when seasonal amplitude grows with the level
            \item[\textcolor{MainBlue}{\textbf{3.}}] \textbf{Smoothing parameters}: high $\alpha$ = reactive, low $\alpha$ = smooth
            \item[\textcolor{MainBlue}{\textbf{4.}}] \textbf{Stationarity testing}: use both ADF and KPSS together
            \item[\textcolor{MainBlue}{\textbf{5.}}] \textbf{Random walk is non-stationary}: variance grows with time: $\Var(X_t) = t\sigma^2$
        \end{itemize}
    \end{exampleblock}

    \vspace{0.5cm}
    \begin{center}
        \Large\textcolor{MainBlue}{Questions?}
    \end{center}
    \end{cminipage}
\end{frame}

%=============================================================================
% BIBLIOGRAPHY
%=============================================================================
\section{Bibliography}

\begin{frame}{Bibliography I}
    \begin{cminipage}{0.95\textwidth}
    \begin{block}{Time Series Fundamentals}
        {\small
        \begin{itemize}\setlength{\itemsep}{0pt}
            \item Hyndman, R.J., \& Athanasopoulos, G. (2021). \textit{Forecasting: Principles and Practice}, 3rd ed., OTexts.
            \item Shumway, R.H., \& Stoffer, D.S. (2017). \textit{Time Series Analysis and Its Applications}, 4th ed., Springer.
            \item Brockwell, P.J., \& Davis, R.A. (2016). \textit{Introduction to Time Series and Forecasting}, 3rd ed., Springer.
        \end{itemize}
        }
    \end{block}

    \begin{exampleblock}{Financial Time Series}
        {\small
        \begin{itemize}\setlength{\itemsep}{0pt}
            \item Tsay, R.S. (2010). \textit{Analysis of Financial Time Series}, 3rd ed., Wiley.
            \item Franke, J., H\"ardle, W.K., \& Hafner, C.M. (2019). \textit{Statistics of Financial Markets}, 4th ed., Springer.
        \end{itemize}
        }
    \end{exampleblock}
    \end{cminipage}
\end{frame}

\begin{frame}{Bibliography II}
    \begin{cminipage}{0.95\textwidth}
    \begin{block}{Modern Approaches and Machine Learning}
        {\small
        \begin{itemize}\setlength{\itemsep}{0pt}
            \item Nielsen, A. (2019). \textit{Practical Time Series Analysis}, O'Reilly Media.
            \item Petropoulos, F., et al. (2022). \textit{Forecasting: Theory and Practice}, International Journal of Forecasting.
            \item Makridakis, S., Spiliotis, E., \& Assimakopoulos, V. (2020). The M4 Competition, International Journal of Forecasting.
        \end{itemize}
        }
    \end{block}

    \begin{exampleblock}{Online Resources and Code}
        {\small
        \begin{itemize}\setlength{\itemsep}{0pt}
            \item \textbf{Quantlet}: \url{https://quantlet.com} --- Code repository for statistics
            \item \textbf{Quantinar}: \url{https://quantinar.com} --- Quantitative methods learning platform
            \item \textbf{GitHub TSA}: \url{https://github.com/QuantLet/TSA/tree/main/TSA_ch1} --- Python code for this seminar
        \end{itemize}
        }
    \end{exampleblock}
    \end{cminipage}
\end{frame}

\begin{frame}{}
    \begin{cminipage}{0.95\textwidth}
    \centering
    \Huge\textcolor{IDAred}{Thank You!}

    \vspace{1cm}

    \Large\textcolor{MainBlue}{Questions?}

    \vspace{0.8cm}

    \normalsize
    Seminar materials are available at: \url{https://danpele.github.io/Time-Series-Analysis/}

    \vspace{0.2cm}

    \href{https://quantlet.com}{\raisebox{-0.15em}{\includegraphics[height=0.8em]{ql_logo.png}} Quantlet} \hspace{0.5cm}
    \href{https://quantinar.com}{\raisebox{-0.15em}{\includegraphics[height=0.8em]{qr_logo.png}} Quantinar}
    \end{cminipage}
\end{frame}

\end{document}
