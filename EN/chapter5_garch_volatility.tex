% Chapter 5: ARCH/GARCH Models for Volatility
% Harvard-quality academic presentation
% Bachelor program, Bucharest University of Economic Studies

\documentclass[9pt, aspectratio=169, t]{beamer}

% Ensure content fits on slides
\setbeamersize{text margin left=8mm, text margin right=8mm}

%=============================================================================
% THEME AND STYLE CONFIGURATION
%=============================================================================
\usetheme{Madrid}
\usecolortheme{seahorse}

% Professional Color Palette
\definecolor{MainBlue}{RGB}{26, 58, 110}
\definecolor{AccentBlue}{RGB}{42, 82, 140}
\definecolor{IDAred}{RGB}{220, 53, 69}
\definecolor{DarkGray}{RGB}{51, 51, 51}
\definecolor{MediumGray}{RGB}{128, 128, 128}
\definecolor{LightGray}{RGB}{248, 248, 248}
\definecolor{VeryLightGray}{RGB}{235, 235, 235}
\definecolor{Crimson}{RGB}{220, 53, 69}
\definecolor{Forest}{RGB}{46, 125, 50}
\definecolor{Amber}{RGB}{181, 133, 63}
\definecolor{Orange}{RGB}{230, 126, 34}
\definecolor{HarvardCrimson}{RGB}{165, 28, 48}

\setbeamercolor{palette primary}{bg=MainBlue, fg=white}
\setbeamercolor{palette secondary}{bg=MainBlue!85, fg=white}
\setbeamercolor{palette tertiary}{bg=MainBlue!70, fg=white}
\setbeamercolor{structure}{fg=MainBlue}
\setbeamercolor{title}{fg=MainBlue}
\setbeamercolor{frametitle}{fg=MainBlue, bg=white}
\setbeamercolor{block title}{bg=MainBlue, fg=white}
\setbeamercolor{block body}{bg=VeryLightGray, fg=DarkGray}
\setbeamercolor{block title alerted}{bg=Crimson, fg=white}
\setbeamercolor{block body alerted}{bg=Crimson!8, fg=DarkGray}
\setbeamercolor{block title example}{bg=Forest, fg=white}
\setbeamercolor{block body example}{bg=Forest!8, fg=DarkGray}
\setbeamercolor{item}{fg=MainBlue}

\setbeamertemplate{navigation symbols}{}

\setbeamertemplate{footline}{
    \leavevmode%
    \hbox{%
        \begin{beamercolorbox}[wd=.333333\paperwidth,ht=2.5ex,dp=1ex,center]{author in head/foot}%
            \usebeamerfont{author in head/foot}\insertshortauthor
        \end{beamercolorbox}%
        \begin{beamercolorbox}[wd=.333333\paperwidth,ht=2.5ex,dp=1ex,center]{title in head/foot}%
            \usebeamerfont{title in head/foot}\insertshorttitle
        \end{beamercolorbox}%
        \begin{beamercolorbox}[wd=.333333\paperwidth,ht=2.5ex,dp=1ex,right]{date in head/foot}%
            \usebeamerfont{date in head/foot}\insertshortdate{}\hspace*{2em}
            \insertframenumber{} / \inserttotalframenumber\hspace*{2ex}
        \end{beamercolorbox}}%
    \vskip0pt%
}

%=============================================================================
% PACKAGES
%=============================================================================
\usepackage[utf8]{inputenc}
\usepackage[T1]{fontenc}
\usepackage{amsmath, amssymb, amsthm}
\usepackage{mathtools}
\usepackage{bm}
\usepackage{tikz}
\usetikzlibrary{arrows.meta, positioning, shapes, calc, decorations.pathreplacing}
\usepackage{booktabs}
\usepackage{multirow}
\usepackage{array}
\usepackage{graphicx}
\usepackage{hyperref}
\usepackage{colortbl}
\hypersetup{colorlinks=false, pdfborder={0 0 0}}
\graphicspath{{../logos/}{../charts/}}

%=============================================================================
% THEOREM ENVIRONMENTS
%=============================================================================
\theoremstyle{definition}
\setbeamertemplate{theorems}[numbered]
\newtheorem{defn}{Definition}
\newtheorem{thm}{Theorem}
\newtheorem{prop}{Proposition}
\newtheorem{rmk}{Remark}

%=============================================================================
% CUSTOM COMMANDS
%=============================================================================
\newcommand{\E}{\mathbb{E}}
\newcommand{\Var}{\text{Var}}
\newcommand{\Cov}{\text{Cov}}
\newcommand{\Corr}{\text{Corr}}
\newcommand{\R}{\mathbb{R}}
\newcommand{\N}{\mathbb{N}}
\newcommand{\Z}{\mathbb{Z}}
\newcommand{\RMSE}{\text{RMSE}}
\newcommand{\MAE}{\text{MAE}}
\newcommand{\MAPE}{\text{MAPE}}

%=============================================================================
% TITLE INFORMATION
%=============================================================================
\title[Chapter 5: GARCH Models]{Chapter 5: ARCH/GARCH Models for Volatility}
\subtitle{Bachelor Program, Faculty of Cybernetics, Statistics and Economic Informatics, Bucharest University of Economic Studies}
\author[Prof. dr. Daniel Traian Pele]{Prof. dr. Daniel Traian Pele\\[0.2cm]\footnotesize\texttt{danpele@ase.ro}}
\institute{Bucharest University of Economic Studies}
\date{Academic Year 2025--2026}

\begin{document}

%=============================================================================
% TITLE SLIDE
%=============================================================================
\begin{frame}[plain]
    \begin{tikzpicture}[remember picture, overlay]
        \fill[IDAred] (current page.north west) rectangle ([yshift=-0.15cm]current page.north east);
        \node[anchor=north west] at ([xshift=0.5cm, yshift=-0.3cm]current page.north west) {
            \href{https://www.ase.ro}{\includegraphics[height=1.1cm]{ase_logo.png}}
        };
        \node[anchor=north] at ([yshift=-0.3cm]current page.north) {
            \href{https://ai4efin.ase.ro}{\includegraphics[height=1.1cm]{ai4efin_logo.png}}
        };
        \node[anchor=north east] at ([xshift=-0.5cm, yshift=-0.3cm]current page.north east) {
            \href{https://www.digital-finance-msca.com}{\includegraphics[height=1.1cm]{msca_logo.png}}
        };
    \end{tikzpicture}
    \vfill
    \begin{center}
        {\Large\textcolor{MediumGray}{Time Series Analysis and Forecasting}}\\[0.3cm]
        {\Huge\textbf{\textcolor{MainBlue}{Chapter 5: Volatility Models}}}\\[0.5cm]
        {\Large\textcolor{IDAred}{ARCH, GARCH, EGARCH, TGARCH}}
    \end{center}
    \vfill

    \begin{tikzpicture}[remember picture, overlay]
        \fill[IDAred] (current page.south west) rectangle ([yshift=0.15cm]current page.south east);
        \node[anchor=south west] at ([xshift=0.5cm, yshift=0.8cm]current page.south west) {
            \href{https://theida.net}{\includegraphics[height=0.9cm]{ida_logo.png}}
        };
        \node[anchor=south] at ([xshift=-3cm, yshift=0.8cm]current page.south) {
            \href{https://blockchain-research-center.com}{\includegraphics[height=0.9cm]{brc_logo.png}}
        };
        \node[anchor=south] at ([yshift=0.8cm]current page.south) {
            \href{https://quantinar.com}{\includegraphics[height=0.9cm]{qr_logo.png}}
        };
        \node[anchor=south] at ([xshift=3cm, yshift=0.8cm]current page.south) {
            \href{https://quantlet.com}{\includegraphics[height=0.9cm]{ql_logo.png}}
        };
        \node[anchor=south east] at ([xshift=-0.5cm, yshift=0.8cm]current page.south east) {
            \href{https://ipe.ro/new}{\includegraphics[height=0.9cm]{acad_logo.png}}
        };
    \end{tikzpicture}
\end{frame}

%=============================================================================
% TABLE OF CONTENTS
%=============================================================================
\begin{frame}{Table of Contents}
    \tableofcontents
\end{frame}

%=============================================================================
% LEARNING OBJECTIVES
%=============================================================================
\begin{frame}{Learning Objectives}
    \begin{block}{By the end of this chapter, you will be able to:}
        \begin{enumerate}
            \item Understand \textbf{volatility clustering} and stylized facts of financial returns
            \item Estimate and interpret \textbf{ARCH} and \textbf{GARCH} models
            \item Apply asymmetric models (\textbf{EGARCH}, \textbf{GJR-GARCH}) for the leverage effect
            \item Perform model diagnostics and selection
            \item Forecast volatility and calculate \textbf{Value at Risk (VaR)}
        \end{enumerate}
    \end{block}

    \vspace{0.3cm}

    \begin{alertblock}{Practical Skills}
        \begin{itemize}
            \item Python implementation with the \texttt{arch} package
            \item Interpretation of parameters and volatility persistence
            \item VaR calculation for risk management
        \end{itemize}
    \end{alertblock}
\end{frame}

%=============================================================================
% SECTION 1: INTRODUCTION TO VOLATILITY
%=============================================================================
\section{Introduction to Volatility Modeling}

\begin{frame}{Why Model Volatility?}
    \begin{block}{Empirical Observations in Financial Series}
        \begin{itemize}
            \item Financial returns exhibit \textbf{volatility clustering} --- periods of high volatility tend to be followed by periods of high volatility
            \item The distribution of returns has \textbf{fat tails} (leptokurtosis)
            \item Return correlation is nearly zero, but correlation of squares is significant
            \item Volatility responds \textbf{asymmetrically} to shocks (leverage effect)
        \end{itemize}
    \end{block}

    \vspace{0.3cm}

    \begin{alertblock}{Limitation of ARIMA Models}
        ARIMA models assume \textbf{constant variance} (homoskedasticity), which is not realistic for financial series!
    \end{alertblock}
\end{frame}

\begin{frame}{Volatility Clustering}
    \begin{center}
        \includegraphics[width=0.85\textwidth]{garch_volatility_clustering.pdf}
    \end{center}

    \begin{itemize}
        \item Periods of high volatility are followed by periods of high volatility
        \item Periods of calm are followed by periods of calm
        \item This suggests that \textbf{conditional variance} is predictable
    \end{itemize}
\end{frame}

\begin{frame}{Stylized Facts of Financial Returns}
    \begin{columns}[T]
        \begin{column}{0.5\textwidth}
            \begin{block}{Observed Properties}
                \begin{enumerate}
                    \item \textbf{Absence of autocorrelation} in returns
                    \item \textbf{Significant autocorrelation} in $r_t^2$ and $|r_t|$
                    \item \textbf{Fat tails} (kurtosis $> 3$)
                    \item \textbf{Leverage effect} --- negative correlation between returns and volatility
                    \item \textbf{Volatility clustering}
                \end{enumerate}
            \end{block}
        \end{column}
        \begin{column}{0.5\textwidth}
            \begin{center}
                \includegraphics[width=\textwidth]{garch_stylized_facts.pdf}
            \end{center}
        \end{column}
    \end{columns}
\end{frame}

\begin{frame}{Conditional Heteroskedasticity}
    \begin{defn}[Conditional Variance]
        Let $\{r_t\}$ be a return series. The \textbf{conditional variance} at time $t$ is:
        \[
            \sigma_t^2 = \Var(r_t | \mathcal{F}_{t-1}) = \E[(r_t - \mu_t)^2 | \mathcal{F}_{t-1}]
        \]
        where $\mathcal{F}_{t-1}$ represents the information available up to time $t-1$.
    \end{defn}

    \vspace{0.3cm}

    \begin{block}{General Model}
        \[
            r_t = \mu_t + \varepsilon_t, \quad \varepsilon_t = \sigma_t z_t, \quad z_t \sim \text{i.i.d.}(0, 1)
        \]
        where:
        \begin{itemize}
            \item $\mu_t$ = conditional mean (can be modeled as ARMA)
            \item $\sigma_t^2$ = conditional variance (modeled as ARCH/GARCH)
            \item $z_t$ = standardized innovations (Normal, Student-t, GED)
        \end{itemize}
    \end{block}
\end{frame}

%=============================================================================
% SECTION 2: ARCH MODEL
%=============================================================================
\section{The ARCH Model}

\begin{frame}{The ARCH(q) Model --- Engle (1982)}
    \begin{defn}[ARCH(q)]
        The \textbf{Autoregressive Conditional Heteroskedasticity} model of order $q$:
        \[
            \varepsilon_t = \sigma_t z_t, \quad z_t \sim \text{i.i.d.}(0, 1)
        \]
        \[
            \sigma_t^2 = \omega + \alpha_1 \varepsilon_{t-1}^2 + \alpha_2 \varepsilon_{t-2}^2 + \cdots + \alpha_q \varepsilon_{t-q}^2
        \]
    \end{defn}

    \vspace{0.3cm}

    \begin{block}{Stationarity Restrictions}
        \begin{itemize}
            \item $\omega > 0$ (positive base variance)
            \item $\alpha_i \geq 0$ for $i = 1, \ldots, q$ (non-negativity)
            \item $\sum_{i=1}^{q} \alpha_i < 1$ (stationarity)
        \end{itemize}
    \end{block}

    \begin{rmk}
        Robert Engle received the \textbf{Nobel Prize in Economics} in 2003 for developing the ARCH model!
    \end{rmk}
\end{frame}

\begin{frame}{Properties of the ARCH(1) Model}
    \begin{block}{ARCH(1): $\sigma_t^2 = \omega + \alpha_1 \varepsilon_{t-1}^2$}
        \begin{itemize}
            \item \textbf{Unconditional variance}: $\E[\varepsilon_t^2] = \dfrac{\omega}{1 - \alpha_1}$ (if $\alpha_1 < 1$)
            \item \textbf{Kurtosis}: $\kappa = 3 \cdot \dfrac{1 - \alpha_1^2}{1 - 3\alpha_1^2}$ (if $\alpha_1^2 < 1/3$)
            \item Kurtosis $> 3$ for $\alpha_1 > 0$ $\Rightarrow$ \textbf{fat tails}!
        \end{itemize}
    \end{block}

    \vspace{0.3cm}

    \begin{exampleblock}{Numerical Example}
        If $\omega = 0.0001$ and $\alpha_1 = 0.3$:
        \begin{itemize}
            \item Unconditional variance: $\sigma^2 = \frac{0.0001}{1 - 0.3} = 0.000143$
            \item Kurtosis: $\kappa = 3 \cdot \frac{1 - 0.09}{1 - 0.27} = 3.74 > 3$
        \end{itemize}
    \end{exampleblock}
\end{frame}

\begin{frame}{Testing for ARCH Effects}
    \begin{block}{Engle's Test for ARCH Effects}
        \textbf{Procedure}:
        \begin{enumerate}
            \item Estimate the mean model and obtain residuals $\hat{\varepsilon}_t$
            \item Calculate $\hat{\varepsilon}_t^2$
            \item Regress $\hat{\varepsilon}_t^2$ on its lags:
            \[
                \hat{\varepsilon}_t^2 = \beta_0 + \beta_1 \hat{\varepsilon}_{t-1}^2 + \cdots + \beta_q \hat{\varepsilon}_{t-q}^2 + u_t
            \]
            \item Calculate the statistic $LM = T \cdot R^2 \sim \chi^2(q)$
        \end{enumerate}
    \end{block}

    \vspace{0.3cm}

    \begin{alertblock}{Hypotheses}
        \begin{itemize}
            \item $H_0$: No ARCH effects ($\alpha_1 = \cdots = \alpha_q = 0$)
            \item $H_1$: ARCH effects present (at least one $\alpha_i \neq 0$)
        \end{itemize}
    \end{alertblock}
\end{frame}

\begin{frame}{Limitations of the ARCH Model}
    \begin{alertblock}{Practical Problems}
        \begin{enumerate}
            \item \textbf{High order} --- many lags are usually needed (large $q$)
            \item \textbf{Many parameters} --- estimation difficulties
            \item \textbf{Non-negativity constraints} --- difficult to impose for large $q$
            \item \textbf{Does not capture persistence} --- observed volatility is very persistent
        \end{enumerate}
    \end{alertblock}

    \vspace{0.5cm}

    \begin{block}{The Solution}
        \textbf{The GARCH Model} --- introduces lags of conditional variance to capture persistence with fewer parameters!
    \end{block}
\end{frame}

%=============================================================================
% SECTION 3: GARCH MODEL
%=============================================================================
\section{The GARCH Model}

\begin{frame}{The GARCH(p,q) Model --- Bollerslev (1986)}
    \begin{defn}[GARCH(p,q)]
        The \textbf{Generalized ARCH} model:
        \[
            \varepsilon_t = \sigma_t z_t, \quad z_t \sim \text{i.i.d.}(0, 1)
        \]
        \[
            \sigma_t^2 = \omega + \sum_{i=1}^{q} \alpha_i \varepsilon_{t-i}^2 + \sum_{j=1}^{p} \beta_j \sigma_{t-j}^2
        \]
    \end{defn}

    \vspace{0.3cm}

    \begin{block}{Interpretation}
        \begin{itemize}
            \item $\omega$ = base level of volatility
            \item $\alpha_i$ = reaction to recent shocks (news coefficients)
            \item $\beta_j$ = volatility persistence (memory)
            \item $\alpha + \beta$ = total persistence
        \end{itemize}
    \end{block}
\end{frame}

\begin{frame}{The GARCH(1,1) Model}
    \begin{block}{The Most Popular Volatility Model}
        \[
            \sigma_t^2 = \omega + \alpha \varepsilon_{t-1}^2 + \beta \sigma_{t-1}^2
        \]
    \end{block}

    \begin{columns}[T]
        \begin{column}{0.5\textwidth}
            \begin{block}{Restrictions}
                \begin{itemize}
                    \item $\omega > 0$
                    \item $\alpha \geq 0$, $\beta \geq 0$
                    \item $\alpha + \beta < 1$ (stationarity)
                \end{itemize}
            \end{block}

            \begin{block}{Properties}
                \begin{itemize}
                    \item Unconditional variance: $\bar{\sigma}^2 = \dfrac{\omega}{1 - \alpha - \beta}$
                    \item Half-life: $HL = \dfrac{\ln(0.5)}{\ln(\alpha + \beta)}$
                \end{itemize}
            \end{block}
        \end{column}
        \begin{column}{0.5\textwidth}
            \begin{center}
                \includegraphics[width=\textwidth]{garch_conditional_variance.pdf}
            \end{center}
        \end{column}
    \end{columns}
\end{frame}

\begin{frame}{GARCH(1,1) as ARMA for $\varepsilon_t^2$}
    \begin{block}{ARMA(1,1) Representation}
        Define $\nu_t = \varepsilon_t^2 - \sigma_t^2$ (variance shock). Then:
        \[
            \varepsilon_t^2 = \omega + (\alpha + \beta) \varepsilon_{t-1}^2 + \nu_t - \beta \nu_{t-1}
        \]

        This is an \textbf{ARMA(1,1)} for $\varepsilon_t^2$!
    \end{block}

    \vspace{0.3cm}

    \begin{exampleblock}{Implications}
        \begin{itemize}
            \item ACF of $\varepsilon_t^2$ decays exponentially (like ARMA)
            \item Persistence is given by $\alpha + \beta$
            \item PACF can help identify the order
        \end{itemize}
    \end{exampleblock}
\end{frame}

\begin{frame}{Estimation of GARCH Models}
    \begin{block}{Maximum Likelihood Estimation (MLE)}
        Log-likelihood function (normal distribution):
        \[
            \ell(\theta) = -\frac{T}{2} \ln(2\pi) - \frac{1}{2} \sum_{t=1}^{T} \left[ \ln(\sigma_t^2) + \frac{\varepsilon_t^2}{\sigma_t^2} \right]
        \]
    \end{block}

    \vspace{0.3cm}

    \begin{block}{Alternative Distributions for $z_t$}
        \begin{itemize}
            \item \textbf{Student-t}: captures fat tails
            \[
                f(z; \nu) = \frac{\Gamma\left(\frac{\nu+1}{2}\right)}{\sqrt{(\nu-2)\pi} \Gamma\left(\frac{\nu}{2}\right)} \left(1 + \frac{z^2}{\nu-2}\right)^{-\frac{\nu+1}{2}}
            \]
            \item \textbf{GED (Generalized Error Distribution)}: flexibility for kurtosis
            \item \textbf{Skewed Student-t}: asymmetry and fat tails
        \end{itemize}
    \end{block}
\end{frame}

\begin{frame}{Typical Values for GARCH(1,1)}
    \begin{center}
        \begin{tabular}{lccc}
            \toprule
            \textbf{Series} & $\bm{\alpha}$ & $\bm{\beta}$ & $\bm{\alpha + \beta}$ \\
            \midrule
            S\&P 500 daily & 0.05--0.10 & 0.85--0.95 & 0.95--0.99 \\
            EUR/USD daily & 0.03--0.08 & 0.90--0.95 & 0.95--0.99 \\
            Bitcoin daily & 0.10--0.20 & 0.75--0.85 & 0.90--0.98 \\
            Bonds & 0.02--0.05 & 0.90--0.97 & 0.95--0.99 \\
            \bottomrule
        \end{tabular}
    \end{center}

    \vspace{0.5cm}

    \begin{alertblock}{Observations}
        \begin{itemize}
            \item $\alpha + \beta$ close to 1 $\Rightarrow$ \textbf{very persistent volatility}
            \item Small $\alpha$, large $\beta$ $\Rightarrow$ slow reaction to shocks, long memory
            \item Bitcoin: larger $\alpha$ $\Rightarrow$ faster reaction to news
        \end{itemize}
    \end{alertblock}
\end{frame}

\begin{frame}{IGARCH --- Integrated GARCH}
    \begin{defn}[IGARCH(1,1)]
        When $\alpha + \beta = 1$:
        \[
            \sigma_t^2 = \omega + \alpha \varepsilon_{t-1}^2 + (1 - \alpha) \sigma_{t-1}^2
        \]
    \end{defn}

    \vspace{0.3cm}

    \begin{block}{Properties}
        \begin{itemize}
            \item Unconditional variance does not exist (infinite)
            \item Shocks have \textbf{permanent} effect on volatility
            \item Used for series with extreme persistence
            \item Useful for \textbf{RiskMetrics} (J.P. Morgan): $\alpha = 0.06$, $\beta = 0.94$
        \end{itemize}
    \end{block}

    \begin{rmk}
        IGARCH is analogous to a unit root in variance!
    \end{rmk}
\end{frame}

%=============================================================================
% SECTION 4: ASYMMETRIC GARCH MODELS
%=============================================================================
\section{Asymmetric GARCH Models}

\begin{frame}{Leverage Effect}
    \begin{center}
        \includegraphics[width=0.7\textwidth]{garch_leverage_effect.pdf}
    \end{center}

    \begin{block}{Definition}
        \textbf{Leverage effect}: Negative shocks (price declines) tend to increase volatility \textbf{more} than positive shocks of the same magnitude.
    \end{block}

    \begin{alertblock}{Problem with Standard GARCH}
        GARCH(p,q) depends on $\varepsilon_{t-1}^2$, so it treats positive and negative shocks \textbf{symmetrically}!
    \end{alertblock}
\end{frame}

\begin{frame}{The EGARCH Model --- Nelson (1991)}
    \begin{defn}[EGARCH(1,1)]
        \textbf{Exponential GARCH}:
        \[
            \ln(\sigma_t^2) = \omega + \alpha \left( |z_{t-1}| - \E[|z_{t-1}|] \right) + \gamma z_{t-1} + \beta \ln(\sigma_{t-1}^2)
        \]
        where $z_t = \varepsilon_t / \sigma_t$.
    \end{defn}

    \vspace{0.3cm}

    \begin{block}{EGARCH Advantages}
        \begin{itemize}
            \item \textbf{No non-negativity constraints required} --- models $\ln(\sigma_t^2)$
            \item \textbf{Captures leverage effect} through parameter $\gamma$
            \begin{itemize}
                \item $\gamma < 0$: negative shocks $\Rightarrow$ higher volatility
                \item $\gamma = 0$: symmetric effect (like GARCH)
            \end{itemize}
            \item Persistence is given by $\beta$
        \end{itemize}
    \end{block}
\end{frame}

\begin{frame}{News Impact Curve --- EGARCH}
    \begin{center}
        \includegraphics[width=0.75\textwidth]{garch_news_impact_curve.pdf}
    \end{center}

    \begin{block}{Interpretation}
        \textbf{News Impact Curve}: shows how future volatility $\sigma_{t+1}^2$ depends on the current shock $\varepsilon_t$, holding $\sigma_t^2$ constant.
    \end{block}
\end{frame}

\begin{frame}{The GJR-GARCH (TGARCH) Model}
    \begin{defn}[GJR-GARCH(1,1)]
        Glosten, Jagannathan \& Runkle (1993):
        \[
            \sigma_t^2 = \omega + \alpha \varepsilon_{t-1}^2 + \gamma \varepsilon_{t-1}^2 \cdot I_{t-1} + \beta \sigma_{t-1}^2
        \]
        where $I_{t-1} = \begin{cases} 1 & \text{if } \varepsilon_{t-1} < 0 \\ 0 & \text{otherwise} \end{cases}$
    \end{defn}

    \vspace{0.3cm}

    \begin{block}{Interpretation}
        \begin{itemize}
            \item Positive shocks ($\varepsilon_{t-1} > 0$): impact = $\alpha$
            \item Negative shocks ($\varepsilon_{t-1} < 0$): impact = $\alpha + \gamma$
            \item Leverage effect present if $\gamma > 0$
        \end{itemize}
    \end{block}

    \begin{exampleblock}{Stationarity}
        $\alpha + \gamma/2 + \beta < 1$
    \end{exampleblock}
\end{frame}

\begin{frame}{TGARCH --- Threshold GARCH}
    \begin{defn}[TGARCH(1,1)]
        Zakoian (1994) --- models standard deviation:
        \[
            \sigma_t = \omega + \alpha^+ \varepsilon_{t-1}^+ + \alpha^- \varepsilon_{t-1}^- + \beta \sigma_{t-1}
        \]
        where $\varepsilon_t^+ = \max(\varepsilon_t, 0)$ and $\varepsilon_t^- = \max(-\varepsilon_t, 0)$.
    \end{defn}

    \vspace{0.3cm}

    \begin{block}{Comparison of Asymmetric Models}
        \begin{center}
            \begin{tabular}{lcc}
                \toprule
                \textbf{Model} & \textbf{Specification} & \textbf{Leverage} \\
                \midrule
                GARCH & $\sigma_t^2$ & No \\
                EGARCH & $\ln(\sigma_t^2)$ & Yes ($\gamma < 0$) \\
                GJR-GARCH & $\sigma_t^2$ with indicator & Yes ($\gamma > 0$) \\
                TGARCH & $\sigma_t$ & Yes ($\alpha^- > \alpha^+$) \\
                \bottomrule
            \end{tabular}
        \end{center}
    \end{block}
\end{frame}

%=============================================================================
% SECTION 5: MODEL SELECTION AND DIAGNOSTICS
%=============================================================================
\section{Model Selection and Diagnostics}

\begin{frame}{Order Selection}
    \begin{block}{Information Criteria}
        \begin{itemize}
            \item \textbf{AIC} = $-2\ell + 2k$
            \item \textbf{BIC} = $-2\ell + k \ln(T)$
            \item \textbf{HQIC} = $-2\ell + 2k \ln(\ln(T))$
        \end{itemize}
        where $\ell$ = maximized log-likelihood, $k$ = number of parameters.
    \end{block}

    \vspace{0.3cm}

    \begin{alertblock}{Practical Recommendations}
        \begin{itemize}
            \item GARCH(1,1) is sufficient in \textbf{90\% of cases}
            \item Check if asymmetric model significantly improves fit
            \item Choose innovation distribution that minimizes AIC/BIC
        \end{itemize}
    \end{alertblock}
\end{frame}

\begin{frame}{GARCH Model Diagnostics}
    \begin{block}{Standardized Residuals}
        \[
            \hat{z}_t = \frac{\hat{\varepsilon}_t}{\hat{\sigma}_t}
        \]

        If the model is correctly specified, $\hat{z}_t$ should be i.i.d.(0,1).
    \end{block}

    \vspace{0.3cm}

    \begin{block}{Diagnostic Checks}
        \begin{enumerate}
            \item \textbf{Ljung-Box on $\hat{z}_t$}: check absence of autocorrelation in mean
            \item \textbf{Ljung-Box on $\hat{z}_t^2$}: check absence of residual ARCH effects
            \item \textbf{ARCH-LM test on $\hat{z}_t$}: confirm absence of heteroskedasticity
            \item \textbf{Histogram + QQ-plot}: verify assumed distribution
        \end{enumerate}
    \end{block}
\end{frame}

\begin{frame}{Diagnostic Example}
    \begin{center}
        \includegraphics[width=0.85\textwidth]{garch_diagnostics.pdf}
    \end{center}
\end{frame}

%=============================================================================
% SECTION 6: FORECASTING VOLATILITY
%=============================================================================
\section{Volatility Forecasting}

\begin{frame}{Forecasting with GARCH(1,1)}
    \begin{block}{One-Step-Ahead Forecast}
        \[
            \hat{\sigma}_{T+1}^2 = \omega + \alpha \varepsilon_T^2 + \beta \sigma_T^2
        \]
    \end{block}

    \begin{block}{Multi-Step Forecast}
        For $h > 1$:
        \[
            \E_T[\sigma_{T+h}^2] = \bar{\sigma}^2 + (\alpha + \beta)^{h-1} (\sigma_{T+1}^2 - \bar{\sigma}^2)
        \]
        where $\bar{\sigma}^2 = \frac{\omega}{1 - \alpha - \beta}$ = unconditional variance.
    \end{block}

    \vspace{0.3cm}

    \begin{exampleblock}{Convergence}
        \[
            \lim_{h \to \infty} \E_T[\sigma_{T+h}^2] = \bar{\sigma}^2
        \]
        Forecast converges to unconditional variance!
    \end{exampleblock}
\end{frame}

\begin{frame}{Volatility Forecast --- Visualization}
    \begin{center}
        \includegraphics[width=0.85\textwidth]{garch_forecast.pdf}
    \end{center}

    \begin{itemize}
        \item Forecast converges exponentially to $\bar{\sigma}^2$
        \item Convergence speed depends on $\alpha + \beta$
        \item The closer $\alpha + \beta$ is to 1, the slower the convergence
    \end{itemize}
\end{frame}

\begin{frame}{Applications of Volatility Forecasting}
    \begin{columns}[T]
        \begin{column}{0.5\textwidth}
            \begin{block}{Value at Risk (VaR)}
                \[
                    \text{VaR}_\alpha = -\mu_{T+1} + z_\alpha \cdot \sigma_{T+1}
                \]

                The probability of losing more than VaR is $\alpha$ (e.g., 1\%, 5\%).
            \end{block}

            \begin{block}{Expected Shortfall}
                \[
                    \text{ES}_\alpha = \E[-r_{T+1} | r_{T+1} < -\text{VaR}_\alpha]
                \]
            \end{block}
        \end{column}
        \begin{column}{0.5\textwidth}
            \begin{block}{Other Applications}
                \begin{itemize}
                    \item Option pricing
                    \item Dynamic hedging
                    \item Portfolio allocation
                    \item Stress testing
                    \item Scenario analysis
                \end{itemize}
            \end{block}
        \end{column}
    \end{columns}
\end{frame}

\begin{frame}{Value at Risk --- Numerical Example}
    \begin{exampleblock}{VaR Calculation for a Portfolio}
        \textbf{Data:} Portfolio of \textbf{1,000,000 EUR}, forecasted volatility $\hat{\sigma}_{T+1} = 1.5\%$
    \end{exampleblock}

    \vspace{0.2cm}

    \begin{block}{VaR with Normal Distribution}
        \begin{center}
            \begin{tabular}{lccc}
                \toprule
                \textbf{Level} & $\bm{z_\alpha}$ & \textbf{VaR (\%)} & \textbf{VaR (EUR)} \\
                \midrule
                95\% (1 day) & 1.645 & 2.47\% & 24,675 \\
                99\% (1 day) & 2.326 & 3.49\% & 34,890 \\
                99\% (10 days) & $2.326 \cdot \sqrt{10}$ & 11.03\% & 110,314 \\
                \bottomrule
            \end{tabular}
        \end{center}
    \end{block}

    \begin{alertblock}{Scaling for Longer Periods}
        \[
            \text{VaR}_{h\text{ days}} = \text{VaR}_{1\text{ day}} \cdot \sqrt{h}
        \]
        This rule assumes i.i.d. returns --- an approximation for short horizons.
    \end{alertblock}
\end{frame}

\begin{frame}{Value at Risk --- Student-t Distribution}
    \begin{block}{Why Student-t?}
        \begin{itemize}
            \item The normal distribution \textbf{underestimates} tail risk
            \item Financial returns have \textbf{fat tails} (kurtosis $> 3$)
            \item Student-t with $\nu$ degrees of freedom better captures extremes
        \end{itemize}
    \end{block}

    \vspace{0.2cm}

    \begin{exampleblock}{VaR 99\% (1 day) Comparison for $\sigma = 1.5\%$, Portfolio = 1M EUR}
        \begin{center}
            \begin{tabular}{lcc}
                \toprule
                \textbf{Distribution} & \textbf{Quantile} & \textbf{VaR (EUR)} \\
                \midrule
                Normal & 2.326 & 34,890 \\
                Student-t ($\nu = 10$) & 2.764 & 41,460 \\
                Student-t ($\nu = 6$) & 3.143 & 47,145 \\
                Student-t ($\nu = 4$) & 3.747 & 56,205 \\
                \bottomrule
            \end{tabular}
        \end{center}
    \end{exampleblock}

    \begin{alertblock}{Observation}
        With $\nu = 6$ (typical for stocks), VaR is \textbf{35\% higher} than normal!
    \end{alertblock}
\end{frame}

\begin{frame}{VaR --- Complete Example with GARCH}
    \begin{block}{VaR Calculation Procedure}
        \begin{enumerate}
            \item Estimate GARCH(1,1) model with Student-t distribution
            \item Obtain volatility forecast: $\hat{\sigma}_{T+1}$
            \item Calculate VaR: $\text{VaR}_\alpha = t_\alpha(\nu) \cdot \hat{\sigma}_{T+1} \cdot \sqrt{\frac{\nu-2}{\nu}}$
        \end{enumerate}
    \end{block}

    \vspace{0.2cm}

    \begin{exampleblock}{Example: S\&P 500}
        \begin{itemize}
            \item Estimated parameters: $\alpha = 0.088$, $\beta = 0.900$, $\nu = 6.4$
            \item Forecasted volatility: $\hat{\sigma}_{T+1} = 1.2\%$
            \item Portfolio: 10,000,000 EUR
        \end{itemize}

        \vspace{0.2cm}

        \textbf{VaR 99\% (1 day):}
        \[
            \text{VaR} = 3.05 \times 0.012 \times 10,000,000 = \textbf{366,000 EUR}
        \]
    \end{exampleblock}
\end{frame}

%=============================================================================
% SECTION 7: PYTHON IMPLEMENTATION
%=============================================================================
\section{Python Implementation}

\begin{frame}[fragile]{GARCH Estimation in Python --- arch}
\begin{block}{Installation and Import}
\begin{verbatim}
pip install arch

import numpy as np
import pandas as pd
from arch import arch_model
from arch.univariate import GARCH, EGARCH, ConstantMean
\end{verbatim}
\end{block}

\begin{block}{GARCH(1,1) Estimation}
\begin{verbatim}
# Assume returns = return series (%)
model = arch_model(returns, vol='Garch', p=1, q=1,
                   dist='normal')
results = model.fit(disp='off')
print(results.summary())
\end{verbatim}
\end{block}
\end{frame}

\begin{frame}[fragile]{Asymmetric Models in Python}
\begin{block}{EGARCH}
\begin{verbatim}
model_egarch = arch_model(returns, vol='EGARCH', p=1, q=1)
res_egarch = model_egarch.fit(disp='off')
\end{verbatim}
\end{block}

\begin{block}{GJR-GARCH}
\begin{verbatim}
model_gjr = arch_model(returns, vol='Garch', p=1, o=1, q=1)
res_gjr = model_gjr.fit(disp='off')
\end{verbatim}
\end{block}

\begin{block}{Alternative Distributions}
\begin{verbatim}
# Student-t
model_t = arch_model(returns, vol='Garch', p=1, q=1,
                     dist='t')
# Skewed Student-t
model_skewt = arch_model(returns, vol='Garch', p=1, q=1,
                         dist='skewt')
\end{verbatim}
\end{block}
\end{frame}

\begin{frame}[fragile]{Forecasting and Diagnostics}
\begin{block}{Volatility Forecast}
\begin{verbatim}
# Forecast 10 steps ahead
forecasts = results.forecast(horizon=10)
volatility_forecast = np.sqrt(forecasts.variance.values[-1, :])
\end{verbatim}
\end{block}

\begin{block}{Standardized Residuals}
\begin{verbatim}
std_resid = results.std_resid

# Ljung-Box Test
from statsmodels.stats.diagnostic import acorr_ljungbox
lb_test = acorr_ljungbox(std_resid**2, lags=10)
\end{verbatim}
\end{block}

\begin{block}{VaR Calculation}
\begin{verbatim}
sigma_forecast = np.sqrt(forecasts.variance.values[-1, 0])
VaR_95 = 1.645 * sigma_forecast  # for alpha = 5%
\end{verbatim}
\end{block}
\end{frame}

%=============================================================================
% SECTION 8: CASE STUDY
%=============================================================================
\section{Case Study: S\&P 500}

\begin{frame}{S\&P 500 Volatility Analysis}
    \begin{center}
        \includegraphics[width=0.85\textwidth]{garch_sp500_returns.pdf}
    \end{center}

    \begin{itemize}
        \item S\&P 500 daily returns (2000--2024)
        \item Volatility clustering visible during crisis periods
        \item 2008 (financial crisis), 2020 (COVID-19), 2022 (inflation)
    \end{itemize}
\end{frame}

\begin{frame}{GARCH(1,1) Estimation --- S\&P 500}
    \begin{columns}[T]
        \begin{column}{0.5\textwidth}
            \begin{block}{Estimation Results}
                \begin{center}
                    \begin{tabular}{lc}
                        \toprule
                        \textbf{Parameter} & \textbf{Value} \\
                        \midrule
                        $\omega$ & 0.0108 \\
                        $\alpha$ & 0.0883 \\
                        $\beta$ & 0.9002 \\
                        $\alpha + \beta$ & 0.9885 \\
                        \midrule
                        $\nu$ (Student-t df) & 6.42 \\
                        \bottomrule
                    \end{tabular}
                \end{center}
            \end{block}

            \begin{block}{Interpretation}
                \begin{itemize}
                    \item Very persistent volatility
                    \item Half-life $\approx$ 60 days
                    \item Fat tails (Student-t)
                \end{itemize}
            \end{block}
        \end{column}
        \begin{column}{0.5\textwidth}
            \begin{center}
                \includegraphics[width=\textwidth]{garch_sp500_volatility.pdf}
            \end{center}
        \end{column}
    \end{columns}
\end{frame}

\begin{frame}{GARCH vs EGARCH Comparison --- S\&P 500}
    \begin{center}
        \includegraphics[width=0.85\textwidth]{garch_sp500_comparison.pdf}
    \end{center}

    \begin{block}{Leverage Effect Confirmed}
        EGARCH: $\gamma = -0.12$ (significantly negative) $\Rightarrow$ negative shocks amplify volatility more than positive ones
    \end{block}
\end{frame}

%=============================================================================
% KEY FORMULAS
%=============================================================================
\begin{frame}{Key Formulas}
    \begin{block}{Volatility Models}
        \begin{itemize}
            \item \textbf{ARCH(q):} $\sigma_t^2 = \omega + \sum_{i=1}^{q} \alpha_i \varepsilon_{t-i}^2$
            \item \textbf{GARCH(1,1):} $\sigma_t^2 = \omega + \alpha \varepsilon_{t-1}^2 + \beta \sigma_{t-1}^2$
            \item \textbf{EGARCH:} $\ln(\sigma_t^2) = \omega + \alpha(|z_{t-1}| - \E[|z|]) + \gamma z_{t-1} + \beta \ln(\sigma_{t-1}^2)$
            \item \textbf{GJR-GARCH:} $\sigma_t^2 = \omega + \alpha \varepsilon_{t-1}^2 + \gamma \varepsilon_{t-1}^2 I_{t-1} + \beta \sigma_{t-1}^2$
        \end{itemize}
    \end{block}

    \vspace{0.2cm}

    \begin{block}{Properties and Measures}
        \begin{columns}[T]
            \begin{column}{0.5\textwidth}
                \begin{itemize}
                    \item \textbf{Unconditional variance:} $\bar{\sigma}^2 = \dfrac{\omega}{1-\alpha-\beta}$
                    \item \textbf{Half-life:} $HL = \dfrac{\ln(0.5)}{\ln(\alpha+\beta)}$
                \end{itemize}
            \end{column}
            \begin{column}{0.5\textwidth}
                \begin{itemize}
                    \item \textbf{VaR:} $\text{VaR}_\alpha = z_\alpha \cdot \sigma_{T+1}$
                    \item \textbf{Stationarity:} $\alpha + \beta < 1$
                \end{itemize}
            \end{column}
        \end{columns}
    \end{block}

    \begin{alertblock}{ARCH-LM Test}
        $LM = T \cdot R^2 \sim \chi^2(q)$ where $R^2$ comes from regressing $\hat{\varepsilon}_t^2$ on its lags
    \end{alertblock}
\end{frame}

%=============================================================================
% SUMMARY
%=============================================================================
\section{Summary}

\begin{frame}{Summary --- Chapter 5: Volatility Models}
    \begin{block}{Key Concepts}
        \begin{itemize}
            \item \textbf{ARCH(q)}: conditional variance depends on past squared errors
            \item \textbf{GARCH(p,q)}: adds variance lags for persistence
            \item \textbf{EGARCH}: allows leverage effect, no positivity constraints
            \item \textbf{GJR-GARCH/TGARCH}: captures asymmetry with indicator variables
        \end{itemize}
    \end{block}

    \begin{block}{Applications}
        \begin{itemize}
            \item Risk measurement and forecasting (VaR, ES)
            \item Derivative pricing
            \item Portfolio management
        \end{itemize}
    \end{block}

    \begin{alertblock}{Practical Tip}
        Start with GARCH(1,1), check for leverage effect, choose innovation distribution that minimizes AIC/BIC!
    \end{alertblock}
\end{frame}

\begin{frame}{References}
    \begin{thebibliography}{10}
        \bibitem{engle1982} Engle, R.F. (1982). \textit{Autoregressive Conditional Heteroscedasticity with Estimates of the Variance of United Kingdom Inflation}. Econometrica, 50(4), 987-1007.

        \bibitem{bollerslev1986} Bollerslev, T. (1986). \textit{Generalized Autoregressive Conditional Heteroskedasticity}. Journal of Econometrics, 31(3), 307-327.

        \bibitem{nelson1991} Nelson, D.B. (1991). \textit{Conditional Heteroskedasticity in Asset Returns: A New Approach}. Econometrica, 59(2), 347-370.

        \bibitem{gjr1993} Glosten, L.R., Jagannathan, R., \& Runkle, D.E. (1993). \textit{On the Relation between the Expected Value and the Volatility of the Nominal Excess Return on Stocks}. The Journal of Finance, 48(5), 1779-1801.

        \bibitem{tsay2010} Tsay, R.S. (2010). \textit{Analysis of Financial Time Series}. 3rd Edition, Wiley.
    \end{thebibliography}
\end{frame}

\end{document}
