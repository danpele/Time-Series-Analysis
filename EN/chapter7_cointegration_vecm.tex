% Chapter 7: Cointegration and VECM
% Long-Run Relationships in Non-Stationary Time Series
% Bachelor program, Bucharest University of Economic Studies

\documentclass[9pt, aspectratio=169, t]{beamer}

% Ensure content fits on slides
\setbeamersize{text margin left=8mm, text margin right=8mm}

%=============================================================================
% THEME AND STYLE CONFIGURATION
%=============================================================================
\usetheme{Madrid}
\usecolortheme{seahorse}

% IDA-Inspired Color Palette
\definecolor{MainBlue}{RGB}{26, 58, 110}
\definecolor{AccentBlue}{RGB}{42, 82, 140}
\definecolor{IDAred}{RGB}{220, 53, 69}
\definecolor{DarkGray}{RGB}{51, 51, 51}
\definecolor{MediumGray}{RGB}{128, 128, 128}
\definecolor{LightGray}{RGB}{248, 248, 248}
\definecolor{VeryLightGray}{RGB}{235, 235, 235}
\definecolor{Crimson}{RGB}{220, 53, 69}
\definecolor{Forest}{RGB}{46, 125, 50}
\definecolor{Amber}{RGB}{181, 133, 63}
\definecolor{Orange}{RGB}{230, 126, 34}

\setbeamercolor{palette primary}{bg=MainBlue, fg=white}
\setbeamercolor{palette secondary}{bg=MainBlue!85, fg=white}
\setbeamercolor{palette tertiary}{bg=MainBlue!70, fg=white}
\setbeamercolor{structure}{fg=MainBlue}
\setbeamercolor{title}{fg=MainBlue}
\setbeamercolor{frametitle}{fg=MainBlue, bg=white}
\setbeamercolor{block title}{bg=MainBlue, fg=white}
\setbeamercolor{block body}{bg=VeryLightGray, fg=DarkGray}
\setbeamercolor{block title alerted}{bg=Crimson, fg=white}
\setbeamercolor{block body alerted}{bg=Crimson!8, fg=DarkGray}
\setbeamercolor{block title example}{bg=Forest, fg=white}
\setbeamercolor{block body example}{bg=Forest!8, fg=DarkGray}
\setbeamercolor{item}{fg=MainBlue}

\setbeamertemplate{navigation symbols}{}

\setbeamertemplate{footline}{
    \leavevmode%
    \hbox{%
        \begin{beamercolorbox}[wd=.333333\paperwidth,ht=2.5ex,dp=1ex,center]{author in head/foot}%
            \usebeamerfont{author in head/foot}\insertshortauthor
        \end{beamercolorbox}%
        \begin{beamercolorbox}[wd=.333333\paperwidth,ht=2.5ex,dp=1ex,center]{title in head/foot}%
            \usebeamerfont{title in head/foot}\insertshorttitle
        \end{beamercolorbox}%
        \begin{beamercolorbox}[wd=.333333\paperwidth,ht=2.5ex,dp=1ex,right]{date in head/foot}%
            \usebeamerfont{date in head/foot}\insertshortdate{}\hspace*{2em}
            \insertframenumber{} / \inserttotalframenumber\hspace*{2ex}
        \end{beamercolorbox}}%
    \vskip0pt%
}

%=============================================================================
% PACKAGES
%=============================================================================
\usepackage{amsmath, amssymb, amsthm}
\usepackage{mathtools}
\usepackage{bm}
\usepackage{tikz}
\usetikzlibrary{arrows.meta, positioning, shapes, calc}
\usepackage{booktabs}
\usepackage{multirow}
\usepackage{array}
\usepackage{graphicx}
\usepackage{hyperref}
\hypersetup{colorlinks=false, pdfborder={0 0 0}}
\graphicspath{{../logos/}{../charts/}}

%=============================================================================
% THEOREM ENVIRONMENTS
%=============================================================================
\theoremstyle{definition}
\setbeamertemplate{theorems}[numbered]
\newtheorem{defn}{Definition}
\newtheorem{thm}{Theorem}
\newtheorem{prop}{Proposition}

%=============================================================================
% CUSTOM COMMANDS
%=============================================================================
\newcommand{\E}{\mathbb{E}}
\newcommand{\Var}{\text{Var}}
\newcommand{\Cov}{\text{Cov}}
\newcommand{\Corr}{\text{Corr}}
\newcommand{\R}{\mathbb{R}}
\newcommand{\bY}{\mathbf{Y}}
\newcommand{\bX}{\mathbf{X}}
\newcommand{\bA}{\mathbf{A}}
\newcommand{\bB}{\mathbf{B}}
\newcommand{\bepsilon}{\boldsymbol{\varepsilon}}
\newcommand{\bSigma}{\boldsymbol{\Sigma}}
\newcommand{\balpha}{\boldsymbol{\alpha}}
\newcommand{\bbeta}{\boldsymbol{\beta}}
\newcommand{\bPi}{\boldsymbol{\Pi}}
\newcommand{\bGamma}{\boldsymbol{\Gamma}}

%=============================================================================
% TITLE INFORMATION
%=============================================================================
\title[Chapter 7: Cointegration \& VECM]{Chapter 7: Cointegration and VECM}
\subtitle{Bachelor program Faculty of Cybernetics, Statistics and Economic Informatics, Bucharest University of Economic Studies, Romania}
\author[Prof. dr. Daniel Traian Pele]{Prof. dr. Daniel Traian Pele\\[0.2cm]\footnotesize\texttt{danpele@ase.ro}}
\institute{Bucharest University of Economic Studies}
\date{Academic Year 2025--2026}

\begin{document}

%=============================================================================
% TITLE SLIDE
%=============================================================================
\begin{frame}[plain]
    \begin{tikzpicture}[remember picture, overlay]
        \fill[IDAred] (current page.north west) rectangle ([yshift=-0.15cm]current page.north east);
        \node[anchor=north west] at ([xshift=0.5cm, yshift=-0.3cm]current page.north west) {
            \href{https://www.ase.ro}{\includegraphics[height=1.1cm]{ase_logo.png}}
        };
        \node[anchor=north] at ([yshift=-0.3cm]current page.north) {
            \href{https://ai4efin.ase.ro}{\includegraphics[height=1.1cm]{ai4efin_logo.png}}
        };
        \node[anchor=north east] at ([xshift=-0.5cm, yshift=-0.3cm]current page.north east) {
            \href{https://www.digital-finance-msca.com}{\includegraphics[height=1.1cm]{msca_logo.png}}
        };
    \end{tikzpicture}
    \vfill
    \begin{center}
        {\Large\textcolor{MediumGray}{Time Series Analysis and Forecasting}}\\[0.3cm]
        {\Huge\textbf{\textcolor{MainBlue}{Chapter 7: Cointegration \& VECM}}}\\[0.5cm]
        {\Large\textcolor{IDAred}{Long-Run Equilibrium Relationships}}
    \end{center}
    \vfill

    \begin{tikzpicture}[remember picture, overlay]
        \fill[IDAred] (current page.south west) rectangle ([yshift=0.15cm]current page.south east);
        \node[anchor=south west] at ([xshift=0.5cm, yshift=0.8cm]current page.south west) {
            \href{https://theida.net}{\includegraphics[height=0.9cm]{ida_logo.png}}
        };
        \node[anchor=south] at ([xshift=-3cm, yshift=0.8cm]current page.south) {
            \href{https://blockchain-research-center.com}{\includegraphics[height=0.9cm]{brc_logo.png}}
        };
        \node[anchor=south] at ([yshift=0.8cm]current page.south) {
            \href{https://quantinar.com}{\includegraphics[height=0.9cm]{qr_logo.png}}
        };
        \node[anchor=south] at ([xshift=3cm, yshift=0.8cm]current page.south) {
            \href{https://quantlet.com}{\includegraphics[height=0.9cm]{ql_logo.png}}
        };
        \node[anchor=south east] at ([xshift=-0.5cm, yshift=0.8cm]current page.south east) {
            \href{https://ipe.ro/new}{\includegraphics[height=0.9cm]{acad_logo.png}}
        };
    \end{tikzpicture}
\end{frame}

%=============================================================================
% OUTLINE
%=============================================================================
\begin{frame}{Lecture Outline}
    \vspace{-0.3cm}
    {\small
    \begin{columns}[T]
        \begin{column}{0.48\textwidth}
            \tableofcontents[sections={1-4}, hideallsubsections]
        \end{column}
        \begin{column}{0.48\textwidth}
            \tableofcontents[sections={5-8}, hideallsubsections]
        \end{column}
    \end{columns}
    }
\end{frame}

%=============================================================================
\section{Motivation}
%=============================================================================

\begin{frame}{Why Cointegration Matters}
    \begin{block}{The Challenge}
        \begin{itemize}
            \item Many economic/financial time series are \textbf{non-stationary} (I(1))
            \item GDP, stock prices, exchange rates, interest rates all have unit roots
            \item Standard regression with I(1) variables $\Rightarrow$ \textbf{spurious results}
            \item Differencing removes non-stationarity but loses \textbf{long-run information}
        \end{itemize}
    \end{block}

    \vspace{0.2cm}

    \begin{alertblock}{The Solution: Cointegration}
        Some non-stationary series share a \textbf{common stochastic trend}---they move together in the long run. This long-run relationship can be modeled!
    \end{alertblock}

    \vspace{0.2cm}

    \begin{exampleblock}{Nobel Prize 2003}
        Clive Granger received the Nobel Prize in Economics (with Robert Engle) for developing cointegration analysis---``methods for analyzing economic time series with common trends.''
    \end{exampleblock}
\end{frame}

\begin{frame}{Real-World Applications}
    \begin{block}{Finance}
        \begin{itemize}
            \item \textbf{Pairs Trading}: Trade the spread between cointegrated stocks
            \item \textbf{Term Structure}: Short and long interest rates
            \item \textbf{Spot-Futures}: Arbitrage relationships
        \end{itemize}
    \end{block}

    \vspace{0.15cm}

    \begin{block}{Macroeconomics}
        \begin{itemize}
            \item \textbf{Consumption \& Income}: Permanent income hypothesis
            \item \textbf{Money \& Prices}: Quantity theory of money
            \item \textbf{PPP}: Exchange rates and price levels
        \end{itemize}
    \end{block}

    \vspace{0.15cm}

    \begin{block}{Policy Analysis}
        \begin{itemize}
            \item \textbf{Fiscal Policy}: Government spending and tax revenues
            \item \textbf{Monetary Policy}: Interest rate pass-through
            \item \textbf{Labor Markets}: Wages and productivity
        \end{itemize}
    \end{block}
\end{frame}

%=============================================================================
\section{Spurious Regression}
%=============================================================================

\begin{frame}{The Spurious Regression Problem}
    \begin{block}{Granger \& Newbold (1974)}
        Regressing one random walk on another \textbf{independent} random walk:
        $$Y_t = \alpha + \beta X_t + u_t$$
        where $Y_t$ and $X_t$ are independent I(1) processes.
    \end{block}

    \vspace{0.2cm}

    \begin{alertblock}{Symptoms of Spurious Regression}
        \begin{itemize}
            \item High $R^2$ (often $> 0.9$) even though variables are \textbf{unrelated}!
            \item Highly significant $t$-statistics (reject $H_0: \beta = 0$)
            \item Very low Durbin-Watson statistic ($DW \approx 0$)
            \item Residuals are non-stationary (have unit root)
        \end{itemize}
    \end{alertblock}

    \vspace{0.2cm}

    {\footnotesize
    \begin{exampleblock}{Rule of Thumb (Granger)}
        If $R^2 > DW$, suspect spurious regression!
    \end{exampleblock}
    }
\end{frame}

\begin{frame}{Spurious Regression: Visual Example}
    \begin{center}
        \includegraphics[width=0.95\textwidth]{charts/spurious_regression.pdf}
    \end{center}

    \vspace{0.1cm}

    {\footnotesize
    \textbf{Warning}: Two completely independent random walks show high correlation ($R^2 > 0.8$) purely by chance! This is why we need cointegration analysis.
    }
\end{frame}

%=============================================================================
\section{Cointegration Concept}
%=============================================================================

\begin{frame}{Definition of Cointegration}
    \begin{defn}[Cointegration (Engle \& Granger, 1987)]
        Variables $Y_{1t}, Y_{2t}, \ldots, Y_{kt}$ are \textbf{cointegrated of order $(d,b)$}, written $CI(d,b)$, if:
        \begin{enumerate}
            \item All variables are integrated of order $d$: $Y_{it} \sim I(d)$
            \item There exists a linear combination $\bbeta' \bY_t = \beta_1 Y_{1t} + \cdots + \beta_k Y_{kt}$ that is integrated of order $(d-b)$, where $b > 0$
        \end{enumerate}
    \end{defn}

    \vspace{0.3cm}

    \begin{block}{Most Common Case: $CI(1,1)$}
        \begin{itemize}
            \item Variables are $I(1)$ (have unit roots)
            \item Linear combination is $I(0)$ (stationary)
            \item Vector $\bbeta = (\beta_1, \ldots, \beta_k)'$ is the \textbf{cointegrating vector}
        \end{itemize}
    \end{block}

    \vspace{0.2cm}

    {\footnotesize
    The cointegrating vector is unique only up to scalar multiplication. Usually normalized: $\beta_1 = 1$.
    }
\end{frame}

\begin{frame}{Cointegration: Visual Example}
    \begin{center}
        \includegraphics[width=0.95\textwidth]{charts/cointegrated_series.pdf}
    \end{center}

    \vspace{0.1cm}

    {\footnotesize
    \textbf{Key insight}: Both series are I(1) and trend together, but their linear combination (spread) is stationary---this is cointegration!
    }
\end{frame}

\begin{frame}{Intuition: Common Stochastic Trends}
    \begin{block}{Why Does Cointegration Occur?}
        Cointegrated variables share \textbf{common stochastic trends}:
        $$Y_{1t} = \gamma_1 \tau_t + S_{1t}, \qquad Y_{2t} = \gamma_2 \tau_t + S_{2t}$$
        where $\tau_t$ is a common random walk and $S_{it}$ are stationary components.
    \end{block}

    \vspace{0.2cm}

    \begin{exampleblock}{Linear Combination Eliminates the Trend}
        $$\gamma_2 Y_{1t} - \gamma_1 Y_{2t} = \gamma_2 S_{1t} - \gamma_1 S_{2t} \sim I(0)$$
    \end{exampleblock}

    \vspace{0.2cm}

    \begin{alertblock}{Economic Interpretation}
        \begin{itemize}
            \item Cointegration represents a \textbf{long-run equilibrium relationship}
            \item Variables may deviate in the short run
            \item But they are ``pulled back'' to equilibrium over time
            \item The cointegrating vector defines the equilibrium
        \end{itemize}
    \end{alertblock}
\end{frame}

\begin{frame}{Cointegrating Rank}
    \begin{block}{How Many Cointegrating Relationships?}
        For $k$ variables that are $I(1)$:
        \begin{itemize}
            \item Maximum possible cointegrating relationships: $r = k - 1$
            \item If $r = 0$: No cointegration (variables drift apart)
            \item If $r = k$: All variables are $I(0)$ (contradiction)
        \end{itemize}
    \end{block}

    \vspace{0.3cm}

    \begin{exampleblock}{Example: 3 Variables}
        \begin{itemize}
            \item $r = 0$: No cointegration
            \item $r = 1$: One cointegrating relationship
            \item $r = 2$: Two cointegrating relationships (only 1 common trend)
        \end{itemize}
    \end{exampleblock}

    \vspace{0.2cm}

    {\footnotesize
    The number of common stochastic trends = $k - r$
    }
\end{frame}

%=============================================================================
\section{Engle-Granger Method}
%=============================================================================

\begin{frame}{Engle-Granger Two-Step Method}
    \begin{block}{Step 1: Estimate Cointegrating Regression}
        Run OLS regression (assuming $Y_t$ is the dependent variable):
        $$Y_t = \alpha + \beta X_t + e_t$$
        Save the residuals: $\hat{e}_t = Y_t - \hat{\alpha} - \hat{\beta} X_t$
    \end{block}

    \vspace{0.2cm}

    \begin{block}{Step 2: Test Residuals for Stationarity}
        Test if $\hat{e}_t$ is $I(0)$ using ADF test:
        $$\Delta \hat{e}_t = \rho \hat{e}_{t-1} + \sum_{j=1}^{p} \gamma_j \Delta \hat{e}_{t-j} + v_t$$

        \begin{itemize}
            \item $H_0$: $\rho = 0$ (residuals have unit root $\Rightarrow$ no cointegration)
            \item $H_1$: $\rho < 0$ (residuals are stationary $\Rightarrow$ cointegration)
        \end{itemize}
    \end{block}

    \vspace{0.2cm}

    \begin{alertblock}{Important}
        Use \textbf{Engle-Granger critical values}, not standard ADF critical values! (More negative because residuals are estimated)
    \end{alertblock}
\end{frame}

\begin{frame}{Engle-Granger Critical Values}
    \begin{block}{Critical Values for Cointegration Test}
        {\small
        \begin{center}
        \begin{tabular}{lccc}
            \toprule
            \textbf{Number of Variables} & \textbf{1\%} & \textbf{5\%} & \textbf{10\%} \\
            \midrule
            2 & $-3.90$ & $-3.34$ & $-3.04$ \\
            3 & $-4.29$ & $-3.74$ & $-3.45$ \\
            4 & $-4.64$ & $-4.10$ & $-3.81$ \\
            5 & $-4.96$ & $-4.42$ & $-4.13$ \\
            \bottomrule
        \end{tabular}
        \end{center}
        }
        {\footnotesize Based on MacKinnon (1991) response surface estimates, $T = 100$}
    \end{block}

    \vspace{0.3cm}

    \begin{alertblock}{Limitations of Engle-Granger}
        \begin{itemize}
            \item Only tests for \textbf{one} cointegrating relationship
            \item Results depend on which variable is chosen as dependent
            \item Small sample bias in estimated cointegrating vector
            \item Cannot test hypotheses on the cointegrating vector
        \end{itemize}
    \end{alertblock}
\end{frame}

%=============================================================================
\section{Johansen Method}
%=============================================================================

\begin{frame}{Johansen Cointegration Test}
    \begin{block}{Advantages over Engle-Granger}
        \begin{itemize}
            \item Tests for \textbf{multiple} cointegrating relationships
            \item Maximum likelihood estimation (more efficient)
            \item Can test restrictions on cointegrating vectors
            \item Does not require choosing a dependent variable
        \end{itemize}
    \end{block}

    \vspace{0.3cm}

    \begin{block}{Starting Point: VAR in Levels}
        $$\bY_t = \mathbf{c} + \bA_1 \bY_{t-1} + \bA_2 \bY_{t-2} + \cdots + \bA_p \bY_{t-p} + \bepsilon_t$$
    \end{block}

    \vspace{0.2cm}

    Rewrite in \textbf{Vector Error Correction} form...
\end{frame}

\begin{frame}{VECM Representation}
    \begin{block}{Vector Error Correction Model}
        $$\Delta \bY_t = \mathbf{c} + \bPi \bY_{t-1} + \sum_{j=1}^{p-1} \bGamma_j \Delta \bY_{t-j} + \bepsilon_t$$

        where:
        \begin{itemize}
            \item $\bPi = \bA_1 + \bA_2 + \cdots + \bA_p - \mathbf{I}$ (long-run impact matrix)
            \item $\bGamma_j = -(\bA_{j+1} + \cdots + \bA_p)$ (short-run dynamics)
        \end{itemize}
    \end{block}

    \vspace{0.3cm}

    \begin{alertblock}{Key Insight: Rank of $\bPi$}
        The \textbf{rank of $\bPi$} determines cointegration:
        \begin{itemize}
            \item $\text{rank}(\bPi) = 0$: No cointegration (VAR in differences)
            \item $\text{rank}(\bPi) = k$: All variables are $I(0)$ (VAR in levels)
            \item $0 < \text{rank}(\bPi) = r < k$: Cointegration with $r$ cointegrating vectors
        \end{itemize}
    \end{alertblock}
\end{frame}

\begin{frame}{Decomposition of $\bPi$}
    \begin{block}{When $\text{rank}(\bPi) = r < k$}
        The matrix $\bPi$ can be decomposed as:
        $$\bPi = \balpha \bbeta'$$
        where:
        \begin{itemize}
            \item $\bbeta$ is $k \times r$ matrix of \textbf{cointegrating vectors}
            \item $\balpha$ is $k \times r$ matrix of \textbf{adjustment coefficients}
        \end{itemize}
    \end{block}

    \vspace{0.2cm}

    \begin{exampleblock}{Interpretation}
        \begin{itemize}
            \item $\bbeta' \bY_{t-1}$ = deviations from long-run equilibrium (error correction terms)
            \item $\balpha$ = speed of adjustment to equilibrium
            \item Each row of $\balpha$ shows how each variable responds to disequilibrium
        \end{itemize}
    \end{exampleblock}

    \vspace{0.2cm}

    {\footnotesize
    VECM: $\Delta \bY_t = \mathbf{c} + \balpha(\bbeta' \bY_{t-1}) + \sum_{j=1}^{p-1} \bGamma_j \Delta \bY_{t-j} + \bepsilon_t$
    }
\end{frame}

\begin{frame}{Johansen Test Statistics}
    \begin{block}{Two Test Statistics}
        Based on eigenvalues $\hat{\lambda}_1 > \hat{\lambda}_2 > \cdots > \hat{\lambda}_k$ of a certain matrix:

        \vspace{0.2cm}

        \textbf{Trace Test:}
        $$\lambda_{\text{trace}}(r) = -T \sum_{i=r+1}^{k} \ln(1 - \hat{\lambda}_i)$$
        Tests $H_0$: rank $\leq r$ vs $H_1$: rank $> r$

        \vspace{0.2cm}

        \textbf{Maximum Eigenvalue Test:}
        $$\lambda_{\max}(r, r+1) = -T \ln(1 - \hat{\lambda}_{r+1})$$
        Tests $H_0$: rank $= r$ vs $H_1$: rank $= r + 1$
    \end{block}

    \vspace{0.2cm}

    {\footnotesize
    Critical values from Johansen \& Juselius (1990), depend on:
    \begin{itemize}
        \item Number of variables $k$
        \item Deterministic components (constant, trend)
    \end{itemize}
    }
\end{frame}

\begin{frame}{Johansen Test: Visual Interpretation}
    \begin{center}
        \includegraphics[width=0.95\textwidth]{charts/johansen_eigenvalues.pdf}
    \end{center}

    \vspace{0.1cm}

    {\footnotesize
    \textbf{Eigenvalue interpretation}: Significant eigenvalues (above critical threshold) indicate cointegrating relationships. In this example, only the first eigenvalue is significant, suggesting $r = 1$ cointegrating vector.
    }
\end{frame}

\begin{frame}{Testing Procedure}
    \begin{block}{Sequential Testing (Trace Test)}
        \begin{enumerate}
            \item Test $H_0$: $r = 0$ vs $H_1$: $r > 0$
            \begin{itemize}
                \item If not rejected: No cointegration. Stop.
                \item If rejected: At least one cointegrating vector. Continue.
            \end{itemize}
            \item Test $H_0$: $r \leq 1$ vs $H_1$: $r > 1$
            \begin{itemize}
                \item If not rejected: $r = 1$. Stop.
                \item If rejected: At least two cointegrating vectors. Continue.
            \end{itemize}
            \item Continue until $H_0$ is not rejected...
        \end{enumerate}
    \end{block}

    \vspace{0.2cm}

    \begin{alertblock}{Deterministic Components}
        Choose specification carefully:
        \begin{itemize}
            \item No constant, no trend (rarely used)
            \item Constant in cointegrating relation only
            \item Constant in both (most common)
            \item Constant + trend in cointegrating relation
            \item Constant + trend in both
        \end{itemize}
    \end{alertblock}
\end{frame}

%=============================================================================
\section{VECM Estimation}
%=============================================================================

\begin{frame}{VECM Structure}
    \begin{block}{Full VECM Specification}
        For $k = 2$ variables with $r = 1$ cointegrating relation:
        \begin{align*}
            \Delta Y_{1t} &= c_1 + \alpha_1 (Y_{1,t-1} - \beta Y_{2,t-1}) + \gamma_{11} \Delta Y_{1,t-1} + \gamma_{12} \Delta Y_{2,t-1} + \varepsilon_{1t} \\
            \Delta Y_{2t} &= c_2 + \alpha_2 (Y_{1,t-1} - \beta Y_{2,t-1}) + \gamma_{21} \Delta Y_{1,t-1} + \gamma_{22} \Delta Y_{2,t-1} + \varepsilon_{2t}
        \end{align*}
    \end{block}

    \vspace{0.2cm}

    \begin{exampleblock}{Components}
        \begin{itemize}
            \item $(Y_{1,t-1} - \beta Y_{2,t-1})$ = error correction term (deviation from equilibrium)
            \item $\alpha_1, \alpha_2$ = adjustment speeds (should have opposite signs)
            \item $\gamma_{ij}$ = short-run dynamics
            \item $\varepsilon_{it}$ = innovations
        \end{itemize}
    \end{exampleblock}
\end{frame}

\begin{frame}{Error Correction Mechanism: Visual}
    \begin{center}
        \includegraphics[width=0.95\textwidth]{charts/error_correction.pdf}
    \end{center}

    \vspace{0.1cm}

    {\footnotesize
    \textbf{Error correction in action}: When series deviate from equilibrium (shaded regions), the adjustment mechanism pulls them back. Positive deviations lead to downward adjustment, negative deviations lead to upward adjustment.
    }
\end{frame}

\begin{frame}{Interpreting Adjustment Coefficients}
    \begin{block}{The $\alpha$ Coefficients}
        If the cointegrating relation is $Y_1 - \beta Y_2 = 0$ (equilibrium):

        \vspace{0.2cm}

        \begin{itemize}
            \item $\alpha_1 < 0$: $Y_1$ adjusts downward when above equilibrium
            \item $\alpha_2 > 0$: $Y_2$ adjusts upward when $Y_1$ is above equilibrium
        \end{itemize}
    \end{block}

    \vspace{0.3cm}

    \begin{alertblock}{Weak Exogeneity}
        If $\alpha_i = 0$, variable $Y_i$ does \textbf{not} respond to disequilibrium.
        \begin{itemize}
            \item $Y_i$ is \textbf{weakly exogenous} for the long-run parameters
            \item The other variable does all the adjusting
            \item Can simplify estimation (single-equation approach)
        \end{itemize}
    \end{alertblock}

    \vspace{0.2cm}

    {\footnotesize
    Test weak exogeneity: $H_0: \alpha_i = 0$ using likelihood ratio test.
    }
\end{frame}

\begin{frame}{VECM vs VAR in Differences}
    \begin{block}{When Variables are Cointegrated}
        \begin{center}
        \begin{tabular}{lcc}
            \toprule
            & \textbf{VAR in Differences} & \textbf{VECM} \\
            \midrule
            Long-run info & Lost & Preserved \\
            Short-run dynamics & Yes & Yes \\
            Error correction & No & Yes \\
            Forecasting & Poor (long-run) & Better \\
            IRF interpretation & Short-run only & Both \\
            \bottomrule
        \end{tabular}
        \end{center}
    \end{block}

    \vspace{0.3cm}

    \begin{alertblock}{Granger Representation Theorem}
        If variables are cointegrated, there \textbf{must} exist an error correction representation. Ignoring cointegration = model misspecification!
    \end{alertblock}
\end{frame}

\begin{frame}{VECM Impulse Response Functions}
    \begin{center}
        \includegraphics[width=0.95\textwidth]{charts/vecm_irf.pdf}
    \end{center}

    \vspace{0.1cm}

    {\footnotesize
    \textbf{IRF interpretation}: In a cointegrated system, shocks have \textbf{permanent effects} on levels but the system returns to equilibrium. Unlike stationary VAR, effects don't decay to zero---they converge to a new long-run value.
    }
\end{frame}

%=============================================================================
\section{Practical Considerations}
%=============================================================================

\begin{frame}{Practical Workflow}
    \begin{block}{Step-by-Step Procedure}
        \begin{enumerate}
            \item \textbf{Unit Root Tests}: Verify all variables are $I(1)$
            \begin{itemize}
                \item ADF, KPSS on levels and first differences
            \end{itemize}
            \item \textbf{Lag Length Selection}: Choose $p$ for VAR in levels
            \begin{itemize}
                \item Use AIC, BIC, or sequential LR tests
            \end{itemize}
            \item \textbf{Cointegration Test}: Johansen trace/max-eigenvalue tests
            \begin{itemize}
                \item Determine cointegrating rank $r$
            \end{itemize}
            \item \textbf{Estimate VECM}: If $0 < r < k$
            \begin{itemize}
                \item Estimate $\balpha$, $\bbeta$, $\bGamma_j$
            \end{itemize}
            \item \textbf{Diagnostics}: Check residuals for autocorrelation, normality
            \item \textbf{Analysis}: IRF, FEVD, hypothesis tests
        \end{enumerate}
    \end{block}
\end{frame}

\begin{frame}{Common Pitfalls}
    \begin{alertblock}{Things to Watch Out For}
        \begin{itemize}
            \item \textbf{Structural breaks}: Can cause spurious unit roots or cointegration
            \item \textbf{Near-unit-root processes}: Tests have low power
            \item \textbf{Too many lags}: Over-parameterization, loss of efficiency
            \item \textbf{Too few lags}: Residual autocorrelation, biased estimates
            \item \textbf{Wrong deterministic specification}: Affects critical values
            \item \textbf{Small samples}: Johansen test oversized in small samples
        \end{itemize}
    \end{alertblock}

    \vspace{0.3cm}

    \begin{exampleblock}{Recommendation}
        Always check:
        \begin{itemize}
            \item Residual diagnostics (Portmanteau test, normality)
            \item Stability of estimated cointegrating relationship over time
            \item Sensitivity to lag length and deterministic specification
        \end{itemize}
    \end{exampleblock}
\end{frame}

%=============================================================================
\section{Real-World Examples}
%=============================================================================

\begin{frame}{Example 1: Term Structure of Interest Rates}
    \begin{center}
        \includegraphics[width=0.95\textwidth]{charts/interest_rates_coint.pdf}
    \end{center}

    \vspace{0.1cm}

    {\footnotesize
    \textbf{Expectations Hypothesis}: Short and long rates share common trend. The spread (term premium) is stationary---evidence of cointegration!
    }
\end{frame}

\begin{frame}{Interest Rates: Economic Theory}
    \begin{block}{Expectations Hypothesis of Term Structure}
        $$R_t^{(n)} = \frac{1}{n} \sum_{i=0}^{n-1} E_t[r_{t+i}] + \text{term premium}$$

        If term premium is constant, short rate $r_t$ and long rate $R_t$ should be cointegrated with vector $(1, -1)$.
    \end{block}

    \vspace{0.2cm}

    \begin{exampleblock}{Empirical Findings}
        \begin{enumerate}
            \item Both rates are $I(1)$ (unit root tests)
            \item One cointegrating relationship (Johansen test)
            \item Cointegrating vector $\approx (1, -1)$: spread is stationary
            \item Short rate adjusts to disequilibrium (long rate is weakly exogenous)
        \end{enumerate}
    \end{exampleblock}
\end{frame}

\begin{frame}{Example 2: Pairs Trading in Finance}
    \begin{center}
        \includegraphics[width=0.95\textwidth]{charts/pairs_trading.pdf}
    \end{center}

    \vspace{0.1cm}

    {\footnotesize
    \textbf{Strategy}: Find cointegrated stock pairs (e.g., Coca-Cola \& Pepsi). When spread deviates from mean, trade expecting mean reversion. Sell spread when high, buy when low.
    }
\end{frame}

\begin{frame}{Example 3: Purchasing Power Parity (PPP)}
    \begin{center}
        \includegraphics[width=0.95\textwidth]{charts/ppp_cointegration.pdf}
    \end{center}

    \vspace{0.1cm}

    {\footnotesize
    \textbf{PPP Theory}: $e_t = p_t - p_t^*$ (log exchange rate equals price differential). Real exchange rate should be stationary in the long run.
    }
\end{frame}

\begin{frame}{Interest Rate VECM Results}
    \begin{block}{Typical Findings}
        \begin{itemize}
            \item Both rates are $I(1)$
            \item One cointegrating relationship found
            \item Cointegrating vector close to $(1, -1)$: spread is stationary
            \item Short rate adjusts to long rate (not vice versa)
        \end{itemize}
    \end{block}

    \vspace{0.2cm}

    \begin{exampleblock}{VECM Equations (stylized)}
        \begin{align*}
            \Delta r_t &= 0.02 - 0.15(r_{t-1} - R_{t-1}) + \text{lags} + \varepsilon_{1t} \\
            \Delta R_t &= 0.01 - 0.02(r_{t-1} - R_{t-1}) + \text{lags} + \varepsilon_{2t}
        \end{align*}

        \begin{itemize}
            \item Short rate adjusts faster ($\alpha_1 = -0.15$)
            \item Long rate nearly weakly exogenous ($\alpha_2 \approx 0$)
        \end{itemize}
    \end{exampleblock}
\end{frame}

%=============================================================================
\section{Case Study: Interest Rates}
%=============================================================================

\begin{frame}{Case Study: Cointegration of Interest Rates}
    \begin{block}{Research Question}
        Are short-term and long-term interest rates cointegrated?
        Does the expectations hypothesis of the term structure hold?
    \end{block}

    \vspace{0.3cm}

    \begin{columns}[T]
        \begin{column}{0.5\textwidth}
            \textbf{Data}
            \begin{itemize}
                \item US Monthly Data (1962-2023)
                \item 3-Month Treasury Bill Rate
                \item 10-Year Treasury Bond Yield
                \item Source: FRED Database
            \end{itemize}
        \end{column}
        \begin{column}{0.5\textwidth}
            \textbf{Methodology}
            \begin{itemize}
                \item Unit root tests (ADF, PP)
                \item Engle-Granger cointegration test
                \item Johansen procedure
                \item VECM estimation
                \item Impulse response analysis
            \end{itemize}
        \end{column}
    \end{columns}
\end{frame}

\begin{frame}{Step 1: Data Visualization}
    \begin{center}
        \includegraphics[width=0.95\textwidth, height=0.78\textheight, keepaspectratio]{charts/ch7_case_raw_data.pdf}
    \end{center}
\end{frame}

\begin{frame}{Step 2: Unit Root Tests}
    \begin{center}
        \includegraphics[width=0.95\textwidth, height=0.78\textheight, keepaspectratio]{charts/ch7_case_unit_root.pdf}
    \end{center}
\end{frame}

\begin{frame}{Step 3: Engle-Granger Cointegration Test}
    \begin{center}
        \includegraphics[width=0.95\textwidth, height=0.78\textheight, keepaspectratio]{charts/ch7_case_cointegration.pdf}
    \end{center}
\end{frame}

\begin{frame}{Step 4: VECM Estimation}
    \begin{center}
        \includegraphics[width=0.95\textwidth, height=0.78\textheight, keepaspectratio]{charts/ch7_case_vecm.pdf}
    \end{center}
\end{frame}

\begin{frame}{Step 5: Impulse Response Functions}
    \begin{center}
        \includegraphics[width=0.95\textwidth, height=0.78\textheight, keepaspectratio]{charts/ch7_case_irf.pdf}
    \end{center}
\end{frame}

\begin{frame}{Step 6: Forecasting}
    \begin{center}
        \includegraphics[width=0.95\textwidth, height=0.78\textheight, keepaspectratio]{charts/ch7_case_forecast.pdf}
    \end{center}
\end{frame}

%=============================================================================
\section{Summary}
%=============================================================================

\begin{frame}{Key Takeaways}
    \begin{block}{Main Concepts}
        \begin{itemize}
            \item \textbf{Cointegration}: $I(1)$ variables with stationary linear combination
            \item \textbf{Spurious regression}: High $R^2$ with unrelated $I(1)$ variables
            \item \textbf{Error correction}: Adjustment toward long-run equilibrium
            \item \textbf{VECM}: VAR with error correction terms for cointegrated systems
        \end{itemize}
    \end{block}

    \vspace{0.2cm}

    \begin{block}{Testing Methods}
        \begin{itemize}
            \item \textbf{Engle-Granger}: Simple, but only one cointegrating vector
            \item \textbf{Johansen}: Multiple vectors, more powerful, MLE-based
        \end{itemize}
    \end{block}

    \vspace{0.2cm}

    \begin{alertblock}{Remember}
        Cointegration tests have low power in small samples. Economic theory should guide specification. Always check diagnostics!
    \end{alertblock}
\end{frame}

\begin{frame}{What's Next?}
    \begin{block}{Extensions and Related Topics}
        \begin{itemize}
            \item \textbf{Structural VECM}: Identifying structural shocks
            \item \textbf{Threshold cointegration}: Nonlinear adjustment
            \item \textbf{Panel cointegration}: Multiple cross-sections
            \item \textbf{Fractional cointegration}: Long memory
            \item \textbf{Time-varying cointegration}: Regime changes
        \end{itemize}
    \end{block}

    \vspace{0.5cm}

    \begin{center}
        \Large\textcolor{MainBlue}{Questions?}
    \end{center}
\end{frame}

%=============================================================================
\section{Quiz}
%=============================================================================

\begin{frame}{Quick Quiz}
    \begin{enumerate}
        \item What does it mean for two $I(1)$ variables to be cointegrated?

        \vspace{0.3cm}

        \item What is the ``spurious regression'' problem?

        \vspace{0.3cm}

        \item In a VECM, what do the $\alpha$ coefficients represent?

        \vspace{0.3cm}

        \item What is the main advantage of Johansen over Engle-Granger?

        \vspace{0.3cm}

        \item If $\alpha_i = 0$ for variable $Y_i$, what does this imply?
    \end{enumerate}
\end{frame}

\begin{frame}{Quiz Answers}
    {\small
    \begin{enumerate}
        \item \textbf{Cointegration}: A linear combination of the variables is $I(0)$ (stationary). They share a common stochastic trend.

        \vspace{0.2cm}

        \item \textbf{Spurious regression}: Regressing one $I(1)$ variable on another unrelated $I(1)$ variable gives high $R^2$ and significant coefficients even though there's no true relationship.

        \vspace{0.2cm}

        \item \textbf{$\alpha$ coefficients}: Speed of adjustment---how quickly each variable responds to deviations from long-run equilibrium.

        \vspace{0.2cm}

        \item \textbf{Johansen advantage}: Can test for multiple cointegrating relationships, uses MLE (more efficient), doesn't require choosing dependent variable.

        \vspace{0.2cm}

        \item \textbf{$\alpha_i = 0$}: Variable $Y_i$ is weakly exogenous---it doesn't respond to disequilibrium. Other variables do all the adjusting.
    \end{enumerate}
    }
\end{frame}

\end{document}
