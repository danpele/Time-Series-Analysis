% Chapter 4: SARIMA Models for Seasonal Time Series
% Comprehensive Beamer Presentation
% Bachelor program, Bucharest University of Economic Studies

\documentclass[9pt, aspectratio=169, t]{beamer}

% Ensure content fits on slides
\setbeamersize{text margin left=8mm, text margin right=8mm}

%=============================================================================
% THEME AND STYLE CONFIGURATION
%=============================================================================
\usetheme{Madrid}
\usecolortheme{seahorse}

% IDA-Inspired Color Palette
\definecolor{MainBlue}{RGB}{26, 58, 110}
\definecolor{AccentBlue}{RGB}{42, 82, 140}
\definecolor{IDAred}{RGB}{220, 53, 69}
\definecolor{DarkGray}{RGB}{51, 51, 51}
\definecolor{MediumGray}{RGB}{128, 128, 128}
\definecolor{LightGray}{RGB}{248, 248, 248}
\definecolor{VeryLightGray}{RGB}{235, 235, 235}
\definecolor{Crimson}{RGB}{220, 53, 69}
\definecolor{Forest}{RGB}{46, 125, 50}
\definecolor{Amber}{RGB}{181, 133, 63}

\setbeamercolor{palette primary}{bg=MainBlue, fg=white}
\setbeamercolor{palette secondary}{bg=MainBlue!85, fg=white}
\setbeamercolor{palette tertiary}{bg=MainBlue!70, fg=white}
\setbeamercolor{structure}{fg=MainBlue}
\setbeamercolor{title}{fg=MainBlue}
\setbeamercolor{frametitle}{fg=MainBlue, bg=white}
\setbeamercolor{block title}{bg=MainBlue, fg=white}
\setbeamercolor{block body}{bg=VeryLightGray, fg=DarkGray}
\setbeamercolor{block title alerted}{bg=Crimson, fg=white}
\setbeamercolor{block body alerted}{bg=Crimson!8, fg=DarkGray}
\setbeamercolor{block title example}{bg=Forest, fg=white}
\setbeamercolor{block body example}{bg=Forest!8, fg=DarkGray}
\setbeamercolor{item}{fg=MainBlue}

\setbeamertemplate{navigation symbols}{}

\setbeamertemplate{footline}{
    \leavevmode%
    \hbox{%
        \begin{beamercolorbox}[wd=.333333\paperwidth,ht=2.5ex,dp=1ex,center]{author in head/foot}%
            \usebeamerfont{author in head/foot}\insertshortauthor
        \end{beamercolorbox}%
        \begin{beamercolorbox}[wd=.333333\paperwidth,ht=2.5ex,dp=1ex,center]{title in head/foot}%
            \usebeamerfont{title in head/foot}\insertshorttitle
        \end{beamercolorbox}%
        \begin{beamercolorbox}[wd=.333333\paperwidth,ht=2.5ex,dp=1ex,right]{date in head/foot}%
            \usebeamerfont{date in head/foot}\insertshortdate{}\hspace*{2em}
            \insertframenumber{} / \inserttotalframenumber\hspace*{2ex}
        \end{beamercolorbox}}%
    \vskip0pt%
}

%=============================================================================
% PACKAGES
%=============================================================================
\usepackage{amsmath, amssymb, amsthm}
\usepackage{mathtools}
\usepackage{bm}
\usepackage{tikz}
\usetikzlibrary{arrows.meta, positioning, shapes, calc}
\usepackage{booktabs}
\usepackage{multirow}
\usepackage{array}
\usepackage{graphicx}
\usepackage{hyperref}
\hypersetup{colorlinks=false, pdfborder={0 0 0}}
\graphicspath{{logos/}{charts/}}

%=============================================================================
% THEOREM ENVIRONMENTS
%=============================================================================
\theoremstyle{definition}
\setbeamertemplate{theorems}[numbered]
\newtheorem{defn}{Definition}
\newtheorem{thm}{Theorem}
\newtheorem{prop}{Proposition}
\newtheorem{rmk}{Remark}

%=============================================================================
% CUSTOM COMMANDS
%=============================================================================
\newcommand{\E}{\mathbb{E}}
\newcommand{\Var}{\text{Var}}
\newcommand{\Cov}{\text{Cov}}
\newcommand{\Corr}{\text{Corr}}
\newcommand{\R}{\mathbb{R}}
\newcommand{\N}{\mathbb{N}}
\newcommand{\Z}{\mathbb{Z}}
\newcommand{\B}{\mathbf{B}}

%=============================================================================
% TITLE INFORMATION
%=============================================================================
\title[Chapter 4: SARIMA Models]{Chapter 4: SARIMA Models for Seasonal Time Series}
\subtitle{Bachelor program Faculty of Cybernetics, Statistics and Economic Informatics, Bucharest University of Economic Studies, Romania}
\author[Prof. dr. Daniel Traian Pele]{Prof. dr. Daniel Traian Pele\\[0.2cm]\footnotesize\texttt{danpele@ase.ro}}
\institute{Bucharest University of Economic Studies}
\date{Academic Year 2025--2026}

\begin{document}

%=============================================================================
% TITLE SLIDE
%=============================================================================
\begin{frame}[plain]
    \begin{tikzpicture}[remember picture, overlay]
        \node[anchor=north west] at ([xshift=0.5cm, yshift=-0.3cm]current page.north west) {
            \href{https://www.ase.ro}{\includegraphics[height=1.1cm]{ase_logo.png}}
        };
        \node[anchor=north] at ([yshift=-0.3cm]current page.north) {
            \href{https://ai4efin.ase.ro}{\includegraphics[height=1.1cm]{ai4efin_logo.png}}
        };
        \node[anchor=north east] at ([xshift=-0.5cm, yshift=-0.3cm]current page.north east) {
            \href{https://www.digital-finance-msca.com}{\includegraphics[height=1.1cm]{msca_logo.png}}
        };
    \end{tikzpicture}
    \vfill
    \begin{center}
        {\Huge\textbf{\textcolor{MainBlue}{Chapter 4: SARIMA Models}}}\\[0.5cm]
        {\Large\textcolor{MainBlue}{Seasonal Time Series}}
    \end{center}
    \vfill

    \begin{tikzpicture}[remember picture, overlay]
        \node[anchor=south west] at ([xshift=1cm, yshift=0.8cm]current page.south west) {
            \href{https://theida.net}{\includegraphics[height=0.9cm]{ida_logo.png}}
        };
        \node[anchor=south] at ([yshift=0.8cm]current page.south) {
            \href{https://blockchain-research-center.com}{\includegraphics[height=0.9cm]{brc_logo.png}}
        };
        \node[anchor=south east] at ([xshift=-1cm, yshift=0.8cm]current page.south east) {
            \href{https://ipe.ro/new}{\includegraphics[height=0.9cm]{acad_logo.png}}
        };
    \end{tikzpicture}
\end{frame}

%=============================================================================
% TABLE OF CONTENTS
%=============================================================================
\begin{frame}{Outline}
    \tableofcontents
\end{frame}

%=============================================================================
% SECTION 1: SEASONALITY IN TIME SERIES
%=============================================================================
\section{Seasonality in Time Series}

\begin{frame}{What is Seasonality?}
    \begin{defn}[Seasonality]
        A time series exhibits \textbf{seasonality} when it shows regular, periodic fluctuations that repeat over a fixed period $s$ (the seasonal period).
    \end{defn}

    \vspace{0.3cm}

    \begin{exampleblock}{Common Seasonal Periods}
        \begin{itemize}
            \item Monthly data: $s = 12$ (annual cycle)
            \item Quarterly data: $s = 4$ (annual cycle)
            \item Weekly data: $s = 52$ (annual) or $s = 7$ (weekly pattern)
            \item Daily data: $s = 7$ (weekly pattern)
        \end{itemize}
    \end{exampleblock}
\end{frame}

\begin{frame}{Examples of Seasonal Data}
    \begin{columns}[T]
        \begin{column}{0.48\textwidth}
            \begin{block}{Economic Series}
                \begin{itemize}
                    \item Retail sales (holiday peaks)
                    \item Tourism (summer/winter)
                    \item Agricultural production
                    \item Energy consumption
                    \item Employment (hiring cycles)
                \end{itemize}
            \end{block}
        \end{column}
        \begin{column}{0.48\textwidth}
            \begin{block}{Other Domains}
                \begin{itemize}
                    \item Weather/temperature
                    \item Website traffic
                    \item Hospital admissions
                    \item Transportation usage
                    \item Electricity demand
                \end{itemize}
            \end{block}
        \end{column}
    \end{columns}

    \vspace{0.3cm}

    \begin{alertblock}{Why It Matters}
        Ignoring seasonality leads to biased forecasts and invalid inference!
    \end{alertblock}
\end{frame}

\begin{frame}{Deterministic vs Stochastic Seasonality}
    \begin{columns}[T]
        \begin{column}{0.48\textwidth}
            \begin{block}{Deterministic Seasonality}
                Fixed seasonal pattern:
                $$Y_t = \sum_{j=1}^{s} \gamma_j D_{jt} + \varepsilon_t$$
                where $D_{jt}$ are seasonal dummies.

                \vspace{0.2cm}
                \textbf{Properties:}
                \begin{itemize}
                    \item Pattern is constant over time
                    \item Can be removed by regression
                \end{itemize}
            \end{block}
        \end{column}
        \begin{column}{0.48\textwidth}
            \begin{block}{Stochastic Seasonality}
                Evolving seasonal pattern:
                $$\Delta_s Y_t = Y_t - Y_{t-s}$$
                exhibits dependence structure.

                \vspace{0.2cm}
                \textbf{Properties:}
                \begin{itemize}
                    \item Pattern evolves over time
                    \item Requires seasonal differencing
                \end{itemize}
            \end{block}
        \end{column}
    \end{columns}
\end{frame}

\begin{frame}{Detecting Seasonality: Overview}
    {\small
    \begin{block}{Visual Methods}
        \begin{itemize}
            \item Time series plot -- look for repeating patterns
            \item Seasonal subseries plot -- compare same seasons across years
            \item Seasonal box plot -- distribution by season
            \item ACF plot -- spikes at seasonal lags ($s, 2s, 3s, \ldots$)
        \end{itemize}
    \end{block}

    \vspace{0.2cm}

    \begin{block}{Statistical Tests}
        \begin{itemize}
            \item Seasonal unit root tests (HEGY, CH, OCSB)
            \item F-test for seasonal dummies
            \item Kruskal-Wallis test (nonparametric)
        \end{itemize}
    \end{block}

    \vspace{0.2cm}

    \begin{alertblock}{Key Principle}
        Always use \textbf{multiple methods} to confirm seasonality before modeling!
    \end{alertblock}
    }
\end{frame}

\begin{frame}{Visual Method 1: ACF for Seasonality Detection}
    \vspace{-0.3cm}
    \begin{center}
        \includegraphics[width=0.85\textwidth, height=0.55\textheight, keepaspectratio]{charts/ch4_acf_seasonality.pdf}
    \end{center}
    \vspace{-0.2cm}
    {\footnotesize
    \begin{itemize}
        \item \textbf{Left}: ACF of original series shows spikes at lags 12, 24, 36 (seasonal lags)
        \item \textbf{Right}: Slow decay at seasonal lags $\Rightarrow$ indicates \textbf{seasonal unit root}
        \item When ACF decays slowly at seasonal lags, apply seasonal differencing $(1-L^s)$
    \end{itemize}
    }
\end{frame}

\begin{frame}{Visual Method 2: Seasonal Subseries Plot}
    \vspace{-0.3cm}
    \begin{center}
        \includegraphics[width=0.85\textwidth, height=0.55\textheight, keepaspectratio]{charts/ch4_seasonal_subseries.pdf}
    \end{center}
    \vspace{-0.2cm}
    {\footnotesize
    \begin{itemize}
        \item Each line shows one month's values across all years
        \item Reveals: (1) Seasonal pattern (summer months higher), (2) Trend within each month
        \item If lines are roughly parallel $\Rightarrow$ additive seasonality
        \item If lines diverge (spread increases) $\Rightarrow$ multiplicative seasonality
    \end{itemize}
    }
\end{frame}

\begin{frame}{Visual Method 3: Seasonal Box Plot}
    \vspace{-0.3cm}
    \begin{center}
        \includegraphics[width=0.78\textwidth, height=0.55\textheight, keepaspectratio]{charts/ch4_seasonal_boxplot.pdf}
    \end{center}
    \vspace{-0.2cm}
    {\footnotesize
    \begin{itemize}
        \item Shows distribution of values for each month (season)
        \item Clear pattern: July-August peaks (summer travel), lower in winter
        \item Increasing variance by month $\Rightarrow$ suggests log transformation
        \item Red line shows monthly means -- reveals the seasonal shape
    \end{itemize}
    }
\end{frame}

\begin{frame}{Additive vs Multiplicative Seasonality}
    \begin{columns}[T]
        \begin{column}{0.48\textwidth}
            \begin{block}{Additive Model}
                $$Y_t = T_t + S_t + \varepsilon_t$$
                \begin{itemize}
                    \item Seasonal amplitude is \textbf{constant}
                    \item Use when variance is stable
                    \item Difference: $Y_t - Y_{t-s}$
                \end{itemize}
            \end{block}
        \end{column}
        \begin{column}{0.48\textwidth}
            \begin{block}{Multiplicative Model}
                $$Y_t = T_t \times S_t \times \varepsilon_t$$
                \begin{itemize}
                    \item Seasonal amplitude \textbf{grows with level}
                    \item Use when variance increases
                    \item Log transform: $\log(Y_t)$
                \end{itemize}
            \end{block}
        \end{column}
    \end{columns}

    \vspace{0.4cm}

    \begin{exampleblock}{Airline Data}
        Seasonal amplitude grows over time $\Rightarrow$ \textbf{multiplicative}

        Solution: Model $\log(Y_t)$ instead of $Y_t$
    \end{exampleblock}
\end{frame}

\begin{frame}{Statistical Tests for Seasonality}
    {\small
    \begin{block}{F-test for Seasonal Dummies}
        Model: $Y_t = \alpha + \sum_{j=1}^{s-1} \gamma_j D_{jt} + \varepsilon_t$

        Test $H_0: \gamma_1 = \gamma_2 = \cdots = \gamma_{s-1} = 0$ using F-statistic
    \end{block}

    \vspace{0.2cm}

    \begin{block}{Kruskal-Wallis Test (Nonparametric)}
        \begin{itemize}
            \item Compares distributions across seasons
            \item $H_0$: All seasonal distributions are identical
            \item Robust to outliers and non-normality
        \end{itemize}
    \end{block}

    \vspace{0.2cm}

    \begin{block}{Friedman Test}
        \begin{itemize}
            \item Nonparametric alternative for repeated measures
            \item Tests if rankings differ systematically across seasons
        \end{itemize}
    \end{block}
    }
\end{frame}

\begin{frame}{Seasonal Unit Root Tests}
    {\small
    \begin{block}{HEGY Test (Hylleberg, Engle, Granger, Yoo)}
        Tests for unit roots at \textbf{different frequencies}:
        \begin{itemize}
            \item Zero frequency (trend unit root)
            \item Seasonal frequencies (various harmonics)
        \end{itemize}
        Provides separate decisions for each frequency.
    \end{block}

    \vspace{0.2cm}

    \begin{block}{Canova-Hansen (CH) Test}
        \begin{itemize}
            \item $H_0$: Deterministic seasonality (stationary)
            \item $H_1$: Stochastic seasonality (unit root)
        \end{itemize}
    \end{block}

    \vspace{0.2cm}

    \begin{alertblock}{Practical Guidance}
        \begin{itemize}
            \item If HEGY rejects seasonal unit root $\Rightarrow$ use seasonal dummies
            \item If HEGY fails to reject $\Rightarrow$ apply seasonal differencing $(1-L^s)$
        \end{itemize}
    \end{alertblock}
    }
\end{frame}

\begin{frame}[fragile]{Testing for Seasonality in Python}
    {\small
    \begin{block}{Visual and Statistical Tests}
        \begin{verbatim}
import numpy as np
from scipy import stats
from statsmodels.tsa.stattools import acf

# Kruskal-Wallis test for seasonality
groups = [y[y.index.month == m] for m in range(1, 13)]
stat, pvalue = stats.kruskal(*groups)
print(f"Kruskal-Wallis: stat={stat:.2f}, p={pvalue:.4f}")

# Check ACF at seasonal lags
acf_vals = acf(y, nlags=36)
seasonal_acf = [acf_vals[12], acf_vals[24], acf_vals[36]]
print(f"ACF at lags 12,24,36: {seasonal_acf}")
        \end{verbatim}
    \end{block}

    \begin{exampleblock}{Rule of Thumb}
        If $|\rho_{12}| > 2/\sqrt{n}$ (outside 95\% CI), seasonality is significant.
    \end{exampleblock}
    }
\end{frame}

%=============================================================================
% SECTION 2: SEASONAL DIFFERENCING
%=============================================================================
\section{Seasonal Differencing}

\begin{frame}{The Seasonal Difference Operator}
    \begin{defn}[Seasonal Difference]
        The \textbf{seasonal difference operator} $\Delta_s$ is defined as:
        $$\Delta_s Y_t = (1 - L^s) Y_t = Y_t - Y_{t-s}$$
        where $L^s Y_t = Y_{t-s}$ is the seasonal lag operator.
    \end{defn}

    \vspace{0.3cm}

    \begin{exampleblock}{Examples}
        \begin{itemize}
            \item Monthly data ($s=12$): $\Delta_{12} Y_t = Y_t - Y_{t-12}$

            Compares each month to the same month last year
            \item Quarterly data ($s=4$): $\Delta_4 Y_t = Y_t - Y_{t-4}$

            Compares each quarter to the same quarter last year
        \end{itemize}
    \end{exampleblock}
\end{frame}

\begin{frame}{Combining Regular and Seasonal Differencing}
    \begin{block}{Full Differencing}
        For series with both trend and seasonality:
        $$\Delta \Delta_s Y_t = (1-L)(1-L^s) Y_t$$
    \end{block}

    \vspace{0.2cm}

    \begin{exampleblock}{Expansion}
        $(1-L)(1-L^s) Y_t = Y_t - Y_{t-1} - Y_{t-s} + Y_{t-s-1}$

        For monthly data ($s=12$):
        $$\Delta \Delta_{12} Y_t = Y_t - Y_{t-1} - Y_{t-12} + Y_{t-13}$$
    \end{exampleblock}

    \vspace{0.2cm}

    \begin{alertblock}{Order of Differencing}
        \begin{itemize}
            \item $d$: number of regular differences (trend removal)
            \item $D$: number of seasonal differences (seasonal trend removal)
        \end{itemize}
    \end{alertblock}
\end{frame}

\begin{frame}{Seasonal Integration}
    \begin{defn}[Seasonally Integrated Process]
        A series $Y_t$ is \textbf{seasonally integrated} of order $(d, D)_s$, written $Y_t \sim I(d, D)_s$, if:
        $$(1-L)^d (1-L^s)^D Y_t$$
        is stationary.
    \end{defn}

    \vspace{0.3cm}

    \begin{exampleblock}{Common Cases}
        \begin{itemize}
            \item $I(1,0)_{12}$: Regular unit root only (monthly)
            \item $I(0,1)_{12}$: Seasonal unit root only
            \item $I(1,1)_{12}$: Both regular and seasonal unit roots
        \end{itemize}
    \end{exampleblock}
\end{frame}

\begin{frame}{Effect of Differencing: Time Series View}
    \vspace{-0.3cm}
    \begin{center}
        \includegraphics[width=0.85\textwidth, height=0.6\textheight, keepaspectratio]{charts/ch4_differencing_effect.pdf}
    \end{center}
    \vspace{-0.2cm}
    {\footnotesize
    \begin{itemize}
        \item Regular differencing $\Delta Y_t$ removes trend but seasonality remains
        \item Seasonal differencing $\Delta_{12} Y_t$ removes seasonality but trend remains
        \item Both needed: $\Delta\Delta_{12} Y_t$ appears stationary
    \end{itemize}
    }
\end{frame}

\begin{frame}{Effect of Differencing: ACF View}
    \vspace{-0.3cm}
    \begin{center}
        \includegraphics[width=0.85\textwidth, height=0.6\textheight, keepaspectratio]{charts/ch4_acf_differencing.pdf}
    \end{center}
    \vspace{-0.2cm}
    {\footnotesize
    \begin{itemize}
        \item Original: slow decay (non-stationary)
        \item After $\Delta$: seasonal spikes at 12, 24 remain
        \item After $\Delta_{12}$: decay at low lags remains
        \item After both: ACF cuts off quickly $\Rightarrow$ ready for SARIMA modeling
    \end{itemize}
    }
\end{frame}

\begin{frame}{Deciding the Order of Differencing}
    {\small
    \begin{block}{Step-by-Step Procedure}
        \begin{enumerate}
            \item Plot the series -- does it have trend? Seasonality?
            \item Check ACF -- slow decay at lags 1,2,3... $\Rightarrow$ need $d \geq 1$
            \item Check ACF at seasonal lags -- slow decay at $s, 2s, 3s$ $\Rightarrow$ need $D \geq 1$
            \item Apply differencing and repeat until ACF cuts off
        \end{enumerate}
    \end{block}

    \vspace{0.2cm}

    \begin{alertblock}{Common Pitfalls}
        \begin{itemize}
            \item \textbf{Over-differencing}: Introduces artificial patterns; ACF shows negative spike at lag 1 or $s$
            \item \textbf{Under-differencing}: ACF remains slowly decaying
            \item Rule: Rarely need $d > 1$ or $D > 1$
        \end{itemize}
    \end{alertblock}

    \vspace{0.2cm}

    \begin{exampleblock}{Airline Data Decision}
        ACF shows: (1) slow decay $\Rightarrow d=1$, (2) spikes at 12, 24 $\Rightarrow D=1$

        Result: Apply $(1-L)(1-L^{12})$ before modeling
    \end{exampleblock}
    }
\end{frame}

%=============================================================================
% SECTION 3: SARIMA MODEL DEFINITION
%=============================================================================
\section{The SARIMA Model}

\begin{frame}{SARIMA Model Definition}
    \begin{defn}[SARIMA$(p,d,q)\times(P,D,Q)_s$]
        The \textbf{Seasonal ARIMA} model is:
        $$\phi(L)\Phi(L^s)(1-L)^d(1-L^s)^D Y_t = c + \theta(L)\Theta(L^s)\varepsilon_t$$
    \end{defn}

    \vspace{0.2cm}

    {\small
    \begin{block}{Components}
        \begin{itemize}
            \item $\phi(L) = 1 - \phi_1 L - \cdots - \phi_p L^p$: Non-seasonal AR
            \item $\Phi(L^s) = 1 - \Phi_1 L^s - \cdots - \Phi_P L^{Ps}$: Seasonal AR
            \item $\theta(L) = 1 + \theta_1 L + \cdots + \theta_q L^q$: Non-seasonal MA
            \item $\Theta(L^s) = 1 + \Theta_1 L^s + \cdots + \Theta_Q L^{Qs}$: Seasonal MA
            \item $(1-L)^d$: Regular differencing
            \item $(1-L^s)^D$: Seasonal differencing
        \end{itemize}
    \end{block}
    }
\end{frame}

\begin{frame}{SARIMA Notation}
    \begin{block}{Full Specification}
        SARIMA$(p,d,q)\times(P,D,Q)_s$ has 7 parameters to specify:
    \end{block}

    \vspace{0.2cm}

    \begin{table}
        \centering
        \small
        \begin{tabular}{ll}
            \toprule
            \textbf{Parameter} & \textbf{Meaning} \\
            \midrule
            $p$ & Non-seasonal AR order \\
            $d$ & Non-seasonal differencing order \\
            $q$ & Non-seasonal MA order \\
            $P$ & Seasonal AR order \\
            $D$ & Seasonal differencing order \\
            $Q$ & Seasonal MA order \\
            $s$ & Seasonal period \\
            \bottomrule
        \end{tabular}
    \end{table}

    \vspace{0.2cm}

    \begin{exampleblock}{Example}
        SARIMA$(1,1,1)\times(1,1,1)_{12}$: Monthly data with AR(1), MA(1), seasonal AR(1), seasonal MA(1), regular and seasonal differencing.
    \end{exampleblock}
\end{frame}

\begin{frame}{Common SARIMA Models}
    {\small
    \begin{block}{Airline Model: SARIMA$(0,1,1)\times(0,1,1)_s$}
        $$(1-L)(1-L^s)Y_t = (1+\theta L)(1+\Theta L^s)\varepsilon_t$$
        Classic model for many economic series (Box \& Jenkins, 1970).
    \end{block}

    \vspace{0.2cm}

    \begin{block}{SARIMA$(1,0,0)\times(1,0,0)_s$}
        $$(1-\phi L)(1-\Phi L^s)Y_t = \varepsilon_t$$
        Pure seasonal and non-seasonal autoregressive model.
    \end{block}

    \vspace{0.2cm}

    \begin{block}{SARIMA$(0,1,1)\times(0,1,0)_s$}
        $$(1-L)(1-L^s)Y_t = (1+\theta L)\varepsilon_t$$
        Random walk with seasonal differencing and MA(1) errors.
    \end{block}
    }
\end{frame}

\begin{frame}{The Multiplicative Structure}
    \begin{block}{Why Multiplicative?}
        The seasonal and non-seasonal parts \textbf{multiply}:
        $$\phi(L)\Phi(L^s) \quad \text{and} \quad \theta(L)\Theta(L^s)$$
    \end{block}

    \vspace{0.2cm}

    \begin{exampleblock}{Example: SARIMA$(1,0,0)\times(1,0,0)_{12}$}
        $(1-\phi L)(1-\Phi L^{12})Y_t = \varepsilon_t$

        Expanding:
        $Y_t - \phi Y_{t-1} - \Phi Y_{t-12} + \phi\Phi Y_{t-13} = \varepsilon_t$

        The cross-term $\phi\Phi Y_{t-13}$ captures interaction!
    \end{exampleblock}

    \vspace{0.2cm}

    \begin{alertblock}{Interpretation}
        Multiplicative structure allows parsimonious modeling of complex seasonal patterns with few parameters.
    \end{alertblock}
\end{frame}

%=============================================================================
% SECTION 4: SEASONAL ACF AND PACF
%=============================================================================
\section{Seasonal ACF and PACF Patterns}

\begin{frame}{ACF/PACF for Seasonal Models}
    \begin{block}{Key Insight}
        Seasonal models show patterns at both:
        \begin{itemize}
            \item Non-seasonal lags: $1, 2, 3, \ldots$
            \item Seasonal lags: $s, 2s, 3s, \ldots$
        \end{itemize}
    \end{block}

    \vspace{0.3cm}

    \begin{table}
        \centering
        \small
        \begin{tabular}{lcc}
            \toprule
            \textbf{Model} & \textbf{ACF} & \textbf{PACF} \\
            \midrule
            SAR($P$) & Decays at $s, 2s, \ldots$ & Cuts off after $Ps$ \\
            SMA($Q$) & Cuts off after $Qs$ & Decays at $s, 2s, \ldots$ \\
            SARMA & Decays at seasonal lags & Decays at seasonal lags \\
            \bottomrule
        \end{tabular}
    \end{table}
\end{frame}

\begin{frame}{Example: Airline Model ACF/PACF}
    \begin{block}{SARIMA$(0,1,1)\times(0,1,1)_{12}$}
        After differencing $W_t = (1-L)(1-L^{12})Y_t$:
        $$W_t = (1+\theta L)(1+\Theta L^{12})\varepsilon_t$$
    \end{block}

    \vspace{0.2cm}

    {\small
    \begin{exampleblock}{Expected ACF Pattern}
        \begin{itemize}
            \item Spike at lag 1 (from $\theta$)
            \item Spike at lag 12 (from $\Theta$)
            \item Spike at lag 13 (from $\theta \cdot \Theta$ interaction)
            \item All other lags near zero
        \end{itemize}
    \end{exampleblock}

    \begin{exampleblock}{Expected PACF Pattern}
        \begin{itemize}
            \item Exponential decay at lags $1, 2, 3, \ldots$
            \item Exponential decay at lags $12, 24, 36, \ldots$
        \end{itemize}
    \end{exampleblock}
    }
\end{frame}

\begin{frame}{Model Identification Guidelines}
    {\small
    \begin{block}{Step-by-Step Process}
        \begin{enumerate}
            \item Examine ACF for slow decay at seasonal lags $\Rightarrow$ seasonal differencing
            \item After differencing, check ACF/PACF patterns
            \item Non-seasonal behavior at lags $1, 2, \ldots, s-1$
            \item Seasonal behavior at lags $s, 2s, 3s, \ldots$
        \end{enumerate}
    \end{block}

    \vspace{0.2cm}

    \begin{alertblock}{Practical Tips}
        \begin{itemize}
            \item Start with $d \leq 1$ and $D \leq 1$
            \item Usually $P, Q \leq 2$ is sufficient
            \item Use information criteria (AIC, BIC) for final selection
            \item Auto-SARIMA algorithms can help
        \end{itemize}
    \end{alertblock}
    }
\end{frame}

%=============================================================================
% SECTION 5: ESTIMATION AND DIAGNOSTICS
%=============================================================================
\section{Estimation and Diagnostics}

\begin{frame}{Estimation Methods}
    \begin{block}{Maximum Likelihood Estimation}
        Standard approach for SARIMA:
        \begin{itemize}
            \item Conditional MLE (conditional on initial values)
            \item Exact MLE (via Kalman filter)
        \end{itemize}
    \end{block}

    \vspace{0.3cm}

    \begin{block}{Computational Considerations}
        \begin{itemize}
            \item More parameters than ARIMA $\Rightarrow$ more data needed
            \item Seasonal parameters estimated from lags $s, 2s, \ldots$
            \item Need sufficient seasonal cycles (at least 3-4 years of monthly data)
        \end{itemize}
    \end{block}
\end{frame}

\begin{frame}{Stationarity and Invertibility}
    \begin{block}{Stationarity Conditions}
        Both non-seasonal and seasonal AR polynomials must have roots outside the unit circle:
        \begin{itemize}
            \item $\phi(z) = 0 \Rightarrow |z| > 1$
            \item $\Phi(z^s) = 0 \Rightarrow |z| > 1$
        \end{itemize}
    \end{block}

    \vspace{0.3cm}

    \begin{block}{Invertibility Conditions}
        Both non-seasonal and seasonal MA polynomials must have roots outside the unit circle:
        \begin{itemize}
            \item $\theta(z) = 0 \Rightarrow |z| > 1$
            \item $\Theta(z^s) = 0 \Rightarrow |z| > 1$
        \end{itemize}
    \end{block}
\end{frame}

\begin{frame}{Diagnostic Checking}
    \begin{block}{Residual Analysis}
        After fitting SARIMA, check that residuals are white noise:
        \begin{enumerate}
            \item Plot residuals over time (no patterns)
            \item ACF of residuals (no significant spikes)
            \item Ljung-Box test at multiple lags including seasonal
            \item Normality tests (Q-Q plot, Jarque-Bera)
        \end{enumerate}
    \end{block}

    \vspace{0.3cm}

    \begin{alertblock}{Important}
        Check ACF at \textbf{both} non-seasonal and seasonal lags!

        Significant ACF at lag 12 suggests inadequate seasonal modeling.
    \end{alertblock}
\end{frame}

\begin{frame}{Model Selection Criteria}
    \begin{block}{Information Criteria}
        Compare competing SARIMA models using:
        \begin{itemize}
            \item AIC = $-2\ln(L) + 2k$
            \item BIC = $-2\ln(L) + k\ln(n)$
            \item AICc = AIC + $\frac{2k(k+1)}{n-k-1}$ (corrected for small samples)
        \end{itemize}
        where $k = p + q + P + Q + 1$ (plus 1 for variance).
    \end{block}

    \vspace{0.3cm}

    \begin{exampleblock}{Auto-SARIMA}
        Python's \texttt{pmdarima.auto\_arima()} with \texttt{seasonal=True} automatically searches for optimal $(p,d,q)\times(P,D,Q)_s$.
    \end{exampleblock}
\end{frame}

%=============================================================================
% SECTION 6: FORECASTING
%=============================================================================
\section{Forecasting with SARIMA}

\begin{frame}{Point Forecasts}
    \begin{block}{Forecast Computation}
        SARIMA forecasts are computed recursively:
        \begin{itemize}
            \item Replace future $\varepsilon_{T+h}$ with 0
            \item Replace future $Y_{T+h}$ with forecasts $\hat{Y}_{T+h|T}$
            \item Use known past values $Y_T, Y_{T-1}, \ldots$
        \end{itemize}
    \end{block}

    \vspace{0.3cm}

    \begin{exampleblock}{Seasonal Pattern in Forecasts}
        SARIMA forecasts naturally capture seasonality:
        \begin{itemize}
            \item Short-term: influenced by recent values
            \item Long-term: revert to seasonal pattern
        \end{itemize}
    \end{exampleblock}
\end{frame}

\begin{frame}{Forecast Intervals}
    \begin{block}{Uncertainty Quantification}
        $(1-\alpha)$\% prediction interval:
        $$\hat{Y}_{T+h|T} \pm z_{\alpha/2} \sqrt{\Var(e_{T+h})}$$

        Variance computed from MA($\infty$) representation.
    \end{block}

    \vspace{0.3cm}

    \begin{alertblock}{Key Properties}
        \begin{itemize}
            \item Intervals widen with forecast horizon
            \item For $I(1,1)_s$ series: intervals grow without bound
            \item Seasonal pattern visible in point forecasts
            \item Uncertainty captures both trend and seasonal variation
        \end{itemize}
    \end{alertblock}
\end{frame}

\begin{frame}{Long-Horizon Forecasts}
    \begin{block}{Behavior as $h \to \infty$}
        \begin{itemize}
            \item Point forecasts converge to deterministic seasonal pattern
            \item If drift present: linear trend + seasonal pattern
            \item Forecast intervals continue to widen
        \end{itemize}
    \end{block}

    \vspace{0.3cm}

    \begin{exampleblock}{Practical Implication}
        \begin{itemize}
            \item Short-term: SARIMA captures both level and season
            \item Medium-term: Good seasonal forecasts, growing uncertainty
            \item Long-term: Mainly reflects seasonal pattern, wide intervals
        \end{itemize}
    \end{exampleblock}
\end{frame}

%=============================================================================
% SECTION 7: REAL DATA APPLICATION
%=============================================================================
\section{Real Data Application: Airline Passengers}

\begin{frame}{Airline Passengers Data}
    \vspace{-0.3cm}
    \begin{center}
        \includegraphics[width=0.75\textwidth, height=0.55\textheight, keepaspectratio]{charts/ch4_airline_data.pdf}
    \end{center}
    \vspace{-0.2cm}
    {\small
    \begin{itemize}
        \item Classic dataset: Monthly international airline passengers (1949-1960)
        \item Clear upward trend and growing seasonal amplitude
    \end{itemize}
    }
\end{frame}

\begin{frame}{Seasonal Decomposition}
    \vspace{-0.3cm}
    \begin{center}
        \includegraphics[width=0.78\textwidth, height=0.6\textheight, keepaspectratio]{charts/ch4_decomposition.pdf}
    \end{center}
    \vspace{-0.2cm}
    {\footnotesize
    \begin{itemize}
        \item Trend: Strong upward growth
        \item Seasonality: Summer peaks (vacation travel)
        \item Residual: Random variation after removing trend and season
    \end{itemize}
    }
\end{frame}

\begin{frame}{ACF/PACF Analysis}
    \vspace{-0.3cm}
    \begin{center}
        \includegraphics[width=0.75\textwidth, height=0.58\textheight, keepaspectratio]{charts/ch4_acf_pacf.pdf}
    \end{center}
    \vspace{-0.2cm}
    {\footnotesize
    \begin{itemize}
        \item After $\Delta\Delta_{12}$ differencing: spikes at lags 1 and 12
        \item Suggests SARIMA$(0,1,1)\times(0,1,1)_{12}$ (Airline model)
    \end{itemize}
    }
\end{frame}

\begin{frame}{SARIMA Forecast Results}
    \vspace{-0.3cm}
    \begin{center}
        \includegraphics[width=0.78\textwidth, height=0.55\textheight, keepaspectratio]{charts/ch4_sarima_forecast.pdf}
    \end{center}
    \vspace{-0.2cm}
    {\small
    \begin{itemize}
        \item SARIMA captures both trend and seasonal pattern
        \item Forecasts show appropriate seasonal peaks and troughs
    \end{itemize}
    }
\end{frame}

\begin{frame}{Model Diagnostics}
    \vspace{-0.3cm}
    \begin{center}
        \includegraphics[width=0.72\textwidth, height=0.6\textheight, keepaspectratio]{charts/ch4_diagnostics.pdf}
    \end{center}
    \vspace{-0.2cm}
    {\footnotesize
    \begin{itemize}
        \item Residuals appear random; ACF within bounds at all lags
        \item Model adequately captures seasonal structure
    \end{itemize}
    }
\end{frame}

\begin{frame}[fragile]{Python Implementation}
    {\small
    \begin{block}{Fitting SARIMA in Python}
        \begin{verbatim}
from statsmodels.tsa.statespace.sarimax import SARIMAX

# Fit SARIMA(0,1,1)(0,1,1)[12]
model = SARIMAX(y, order=(0,1,1),
                seasonal_order=(0,1,1,12))
results = model.fit()
print(results.summary())

# Forecast
forecast = results.get_forecast(steps=24)
        \end{verbatim}
    \end{block}
    }
\end{frame}

%=============================================================================
% SECTION 8: SUMMARY
%=============================================================================
\section{Summary}

\begin{frame}{Key Takeaways}
    {\small
    \begin{block}{Main Points}
        \begin{enumerate}
            \item \textbf{Seasonality} is common in economic and business data
            \item \textbf{Seasonal differencing} $(1-L^s)$ removes stochastic seasonality
            \item \textbf{SARIMA}$(p,d,q)\times(P,D,Q)_s$ extends ARIMA for seasonal data
            \item \textbf{Multiplicative structure} captures seasonal-trend interactions
            \item \textbf{ACF/PACF} show patterns at both regular and seasonal lags
            \item \textbf{Model selection}: Use AIC/BIC or auto-SARIMA algorithms
        \end{enumerate}
    \end{block}

    \vspace{0.2cm}

    \begin{alertblock}{Next Steps}
        Chapter 5 will cover multivariate time series: VAR models, Granger causality, and cointegration.
    \end{alertblock}
    }
\end{frame}

\begin{frame}{References}
    \begin{thebibliography}{9}
        \bibitem{boxjenkins} Box, G.E.P., Jenkins, G.M., Reinsel, G.C., \& Ljung, G.M. (2015). \textit{Time Series Analysis: Forecasting and Control}. 5th ed. Wiley.

        \bibitem{hyndman} Hyndman, R.J. \& Athanasopoulos, G. (2021). \textit{Forecasting: Principles and Practice}. 3rd ed. OTexts.

        \bibitem{hamilton} Hamilton, J.D. (1994). \textit{Time Series Analysis}. Princeton University Press.

        \bibitem{brockwell} Brockwell, P.J. \& Davis, R.A. (2016). \textit{Introduction to Time Series and Forecasting}. 3rd ed. Springer.
    \end{thebibliography}
\end{frame}

\end{document}
