% Chapter 4: SARIMA Models for Seasonal Time Series
% Comprehensive Beamer Presentation
% Bachelor program, Bucharest University of Economic Studies

\documentclass[9pt, aspectratio=169, t]{beamer}

% Ensure content fits on slides
\setbeamersize{text margin left=18mm, text margin right=12mm}

%=============================================================================
% THEME AND STYLE CONFIGURATION
%=============================================================================
\usetheme{Madrid}
\usecolortheme{seahorse}

% IDA-Inspired Color Palette
\definecolor{MainBlue}{RGB}{26, 58, 110}
\definecolor{AccentBlue}{RGB}{42, 82, 140}
\definecolor{IDAred}{RGB}{220, 53, 69}
\definecolor{DarkGray}{RGB}{51, 51, 51}
\definecolor{MediumGray}{RGB}{128, 128, 128}
\definecolor{LightGray}{RGB}{248, 248, 248}
\definecolor{VeryLightGray}{RGB}{235, 235, 235}
\definecolor{Crimson}{RGB}{220, 53, 69}
\definecolor{Forest}{RGB}{46, 125, 50}
\definecolor{Amber}{RGB}{181, 133, 63}

\setbeamercolor{palette primary}{bg=MainBlue, fg=white}
\setbeamercolor{palette secondary}{bg=MainBlue!85, fg=white}
\setbeamercolor{palette tertiary}{bg=MainBlue!70, fg=white}
\setbeamercolor{structure}{fg=MainBlue}
\setbeamercolor{title}{fg=MainBlue}
\setbeamercolor{frametitle}{fg=MainBlue, bg=white}
\setbeamercolor{block title}{bg=MainBlue, fg=white}
\setbeamercolor{block body}{bg=VeryLightGray, fg=DarkGray}
\setbeamercolor{block title alerted}{bg=Crimson, fg=white}
\setbeamercolor{block body alerted}{bg=Crimson!8, fg=DarkGray}
\setbeamercolor{block title example}{bg=Forest, fg=white}
\setbeamercolor{block body example}{bg=Forest!8, fg=DarkGray}
\setbeamercolor{item}{fg=MainBlue}

\setbeamertemplate{navigation symbols}{}

\setbeamertemplate{footline}{
    \leavevmode%
    \hbox{%
        \begin{beamercolorbox}[wd=.333333\paperwidth,ht=2.5ex,dp=1ex,center]{author in head/foot}%
            \usebeamerfont{author in head/foot}\insertshortauthor
        \end{beamercolorbox}%
        \begin{beamercolorbox}[wd=.333333\paperwidth,ht=2.5ex,dp=1ex,center]{title in head/foot}%
            \usebeamerfont{title in head/foot}\insertshorttitle
        \end{beamercolorbox}%
        \begin{beamercolorbox}[wd=.333333\paperwidth,ht=2.5ex,dp=1ex,right]{date in head/foot}%
            \usebeamerfont{date in head/foot}\insertshortdate{}\hspace*{2em}
            \insertframenumber{} / \inserttotalframenumber\hspace*{2ex}
        \end{beamercolorbox}}%
    \vskip0pt%
}

%=============================================================================
% PACKAGES
%=============================================================================
\usepackage{amsmath, amssymb, amsthm}
\usepackage{mathtools}
\usepackage{bm}
\usepackage{tikz}
\usetikzlibrary{arrows.meta, positioning, shapes, calc}
\usepackage{booktabs}
\usepackage{multirow}
\usepackage{array}
\usepackage{graphicx}
\usepackage{hyperref}
\hypersetup{colorlinks=false, pdfborder={0 0 0}}
\graphicspath{{logos/}{charts/}}

%=============================================================================
% THEOREM ENVIRONMENTS
%=============================================================================
\theoremstyle{definition}
\setbeamertemplate{theorems}[numbered]
\newtheorem{defn}{Definition}
\newtheorem{thm}{Theorem}
\newtheorem{prop}{Proposition}
\newtheorem{rmk}{Remark}

%=============================================================================
% CUSTOM COMMANDS
%=============================================================================
\newcommand{\E}{\mathbb{E}}
\newcommand{\Var}{\text{Var}}
\newcommand{\Cov}{\text{Cov}}
\newcommand{\Corr}{\text{Corr}}
\newcommand{\R}{\mathbb{R}}
\newcommand{\N}{\mathbb{N}}
\newcommand{\Z}{\mathbb{Z}}
\newcommand{\B}{\mathbf{B}}

%=============================================================================
% TITLE INFORMATION
%=============================================================================
\title[Chapter 4: SARIMA Models]{Chapter 4: SARIMA Models for Seasonal Time Series}
\subtitle{Bachelor program Faculty of Cybernetics, Statistics and Economic Informatics, Bucharest University of Economic Studies, Romania}
\author[Prof. dr. Daniel Traian Pele]{Prof. dr. Daniel Traian Pele\\[0.2cm]\footnotesize\texttt{danpele@ase.ro}}
\institute{Bucharest University of Economic Studies}
\date{Academic Year 2025--2026}

\begin{document}

%=============================================================================
% TITLE SLIDE
%=============================================================================
\begin{frame}[plain]
    \begin{tikzpicture}[remember picture, overlay]
        \fill[IDAred] (current page.north west) rectangle ([yshift=-0.15cm]current page.north east);
        \node[anchor=north west] at ([xshift=0.5cm, yshift=-0.3cm]current page.north west) {
            \href{https://www.ase.ro}{\includegraphics[height=1.1cm]{ase_logo.png}}
        };
        \node[anchor=north] at ([yshift=-0.3cm]current page.north) {
            \href{https://ai4efin.ase.ro}{\includegraphics[height=1.1cm]{ai4efin_logo.png}}
        };
        \node[anchor=north east] at ([xshift=-0.5cm, yshift=-0.3cm]current page.north east) {
            \href{https://www.digital-finance-msca.com}{\includegraphics[height=1.1cm]{msca_logo.png}}
        };
    \end{tikzpicture}
    \vfill
    \begin{center}
        {\Huge\textbf{\textcolor{MainBlue}{Chapter 4: SARIMA Models}}}\\[0.5cm]
        {\Large\textcolor{IDAred}{Seasonal Time Series}}
    \end{center}
    \vfill

    \begin{tikzpicture}[remember picture, overlay]
        \fill[IDAred] (current page.south west) rectangle ([yshift=0.15cm]current page.south east);
        \node[anchor=south west] at ([xshift=0.5cm, yshift=0.8cm]current page.south west) {
            \href{https://theida.net}{\includegraphics[height=0.9cm]{ida_logo.png}}
        };
        \node[anchor=south] at ([xshift=-3cm, yshift=0.8cm]current page.south) {
            \href{https://blockchain-research-center.com}{\includegraphics[height=0.9cm]{brc_logo.png}}
        };
        \node[anchor=south] at ([yshift=0.8cm]current page.south) {
            \href{https://quantinar.com}{\includegraphics[height=0.9cm]{qr_logo.png}}
        };
        \node[anchor=south] at ([xshift=3cm, yshift=0.8cm]current page.south) {
            \href{https://quantlet.com}{\includegraphics[height=0.9cm]{ql_logo.png}}
        };
        \node[anchor=south east] at ([xshift=-0.5cm, yshift=0.8cm]current page.south east) {
            \href{https://ipe.ro/new}{\includegraphics[height=0.9cm]{acad_logo.png}}
        };
    \end{tikzpicture}
\end{frame}

%=============================================================================
% TABLE OF CONTENTS
%=============================================================================
\begin{frame}{Outline}
    \vspace{-0.3cm}
    {\small
    \begin{columns}[T]
        \begin{column}{0.48\textwidth}
            \tableofcontents[sections={1-5}, hideallsubsections]
        \end{column}
        \begin{column}{0.48\textwidth}
            \tableofcontents[sections={6-9}, hideallsubsections]
        \end{column}
    \end{columns}
    }
\end{frame}

%=============================================================================
% MOTIVATION
%=============================================================================
\begin{frame}{Motivating Example: Seasonality Is Everywhere}
    \vspace{-0.3cm}
    \begin{center}
        \includegraphics[width=0.88\textwidth, height=0.62\textheight, keepaspectratio]{charts/ch4_motivation_seasonal.pdf}
    \end{center}
    \vspace{-0.2cm}
    {\footnotesize
    \begin{itemize}
        \item Retail sales exhibit clear \textbf{annual patterns}: December peaks, January troughs
        \item Standard ARIMA models cannot capture these \textbf{repeating seasonal cycles}
        \item Ignoring seasonality leads to systematic forecast errors
    \end{itemize}
    }
\end{frame}

\begin{frame}{Understanding Seasonal Components}
    \vspace{-0.3cm}
    \begin{center}
        \includegraphics[width=0.88\textwidth, height=0.62\textheight, keepaspectratio]{charts/ch4_motivation_decomposition.pdf}
    \end{center}
    \vspace{-0.2cm}
    {\footnotesize
    \begin{itemize}
        \item Seasonal time series = \textbf{Trend} + \textbf{Seasonal Pattern} + \textbf{Residuals}
        \item Decomposition helps visualize each component separately
        \item SARIMA models capture both trend dynamics and seasonal behavior
    \end{itemize}
    }
\end{frame}

\begin{frame}{Real-World Application: Monthly Patterns}
    \vspace{-0.3cm}
    \begin{center}
        \includegraphics[width=0.88\textwidth, height=0.62\textheight, keepaspectratio]{charts/ch4_motivation_monthly.pdf}
    \end{center}
    \vspace{-0.2cm}
    {\footnotesize
    \begin{itemize}
        \item Energy demand shows strong \textbf{monthly seasonality} (heating/cooling cycles)
        \item Pattern repeats predictably each year with slight variations
        \item Utility companies use SARIMA forecasts for capacity planning
    \end{itemize}
    }
\end{frame}

\begin{frame}{Why Do We Need SARIMA?}
    \vspace{-0.3cm}
    \begin{center}
        \includegraphics[width=0.88\textwidth, height=0.62\textheight, keepaspectratio]{charts/ch4_motivation_why_sarima.pdf}
    \end{center}
    \vspace{-0.2cm}
    {\footnotesize
    \begin{itemize}
        \item \textbf{Left}: Seasonal ACF shows spikes at lags 12, 24, 36... (annual pattern)
        \item \textbf{Right}: ARIMA residuals still show seasonal autocorrelation --- model is incomplete
        \item SARIMA adds \textbf{seasonal AR and MA terms} to capture these patterns
    \end{itemize}
    }
\end{frame}

\begin{frame}{What We'll Learn Today}
    {\small
    \hfill\begin{minipage}{0.9\textwidth}
        \begin{columns}[T]
            \begin{column}{0.48\textwidth}
                \begin{block}{Concepts}
                    \begin{itemize}\setlength{\itemsep}{1pt}
                        \item Identifying seasonal patterns
                        \item Seasonal differencing operator
                        \item SARIMA$(p,d,q)(P,D,Q)_s$ notation
                        \item The famous ``Airline Model''
                        \item Model selection for seasonal data
                    \end{itemize}
                \end{block}
            \end{column}
            \begin{column}{0.48\textwidth}
                \begin{block}{Skills}
                    \begin{itemize}\setlength{\itemsep}{1pt}
                        \item Diagnose seasonality from ACF/PACF
                        \item Determine seasonal period $s$
                        \item Choose $(P, D, Q)$ seasonal orders
                        \item Implement SARIMA in Python/R
                        \item Forecast seasonal time series
                    \end{itemize}
                \end{block}
            \end{column}
        \end{columns}
    \end{minipage}
    }

    \vspace{0.3cm}

    \begin{alertblock}{Key Insight}
        SARIMA = ARIMA applied at \textbf{two frequencies}: the regular (short-term) and seasonal (long-term) levels
    \end{alertblock}
\end{frame}

%=============================================================================
% SECTION 1: SEASONALITY IN TIME SERIES
%=============================================================================
\section{Seasonality in Time Series}

\begin{frame}{What is Seasonality?}
    \begin{defn}[Seasonality]
        A time series exhibits \textbf{seasonality} when it shows regular, periodic fluctuations that repeat over a fixed period $s$ (the seasonal period).
    \end{defn}

    \vspace{0.1cm}

    \begin{exampleblock}{Common Seasonal Periods}
        \begin{itemize}
            \item Monthly data: $s = 12$ (annual cycle)
            \item Quarterly data: $s = 4$ (annual cycle)
            \item Weekly data: $s = 52$ (annual) or $s = 7$ (weekly pattern)
            \item Daily data: $s = 7$ (weekly pattern)
        \end{itemize}
    \end{exampleblock}
\end{frame}

\begin{frame}{Seasonality: Visual Illustration}
    \begin{center}
        \includegraphics[width=0.95\textwidth]{charts/ch4_def_seasonality.pdf}
    \end{center}
    \vspace{-0.2cm}
    \small Left: monthly data with $s=12$ seasonal period. Right: quarterly data with $s=4$.
\end{frame}

\begin{frame}{Examples of Seasonal Data}
	{\small
		\hfill\begin{minipage}{0.9\textwidth}
			\begin{columns}[T]
				\begin{column}{0.48\textwidth}
					\begin{block}{Economic Series}
						\begin{itemize}\setlength{\itemsep}{0pt}
							\item Retail sales (holiday peaks)
							\item Tourism (summer/winter)
							\item Agricultural production
							\item Energy consumption
							\item Employment (hiring cycles)
						\end{itemize}
					\end{block}
				\end{column}
				\begin{column}{0.48\textwidth}
					\begin{block}{Other Domains}
						\begin{itemize}\setlength{\itemsep}{0pt}
							\item Weather/temperature
							\item Website traffic
							\item Hospital admissions
							\item Transportation usage
							\item Electricity demand
						\end{itemize}
					\end{block}
				\end{column}
			\end{columns}
		\end{minipage}
	}
	\begin{alertblock}{Why It Matters}
		Ignoring seasonality leads to biased forecasts and invalid inference!
	\end{alertblock}
\end{frame}

\begin{frame}{Example: Airline Passengers Data}
    \vspace{-0.3cm}
    \begin{center}
        \includegraphics[width=0.78\textwidth, height=0.55\textheight, keepaspectratio]{charts/ch4_airline_data.pdf}
    \end{center}
    \vspace{-0.2cm}
    {\small
    \begin{itemize}
        \item Monthly international airline passengers (1949--1960)
        \item Clear \textbf{upward trend} and \textbf{growing seasonal amplitude}
        \item Summer peaks reflect vacation travel patterns
    \end{itemize}
    }
\end{frame}

\begin{frame}{Visualizing Seasonal Patterns}
    \vspace{-0.3cm}
    \begin{center}
        \includegraphics[width=0.78\textwidth, height=0.58\textheight, keepaspectratio]{charts/ch4_seasonal_boxplot.pdf}
    \end{center}
    \vspace{-0.2cm}
    {\footnotesize
    \begin{itemize}
        \item Box plot reveals consistent seasonal pattern across years
        \item July--August show highest passenger counts (summer travel)
        \item November--February show lowest counts (winter months)
    \end{itemize}
    }
\end{frame}

\begin{frame}{Deterministic vs Stochastic Seasonality}
    {\small
    \hfill\begin{minipage}{0.9\textwidth}
    \begin{columns}[T]
        \begin{column}{0.48\textwidth}
            \begin{block}{Deterministic Seasonality}
                Fixed seasonal pattern:
                $Y_t = \sum_{j=1}^{s} \gamma_j D_{jt} + \varepsilon_t$
                where $D_{jt}$ are seasonal dummies.

                \textbf{Properties:}
                \begin{itemize}\setlength{\itemsep}{0pt}
                    \item Pattern constant over time
                    \item Removed by regression
                \end{itemize}
            \end{block}
        \end{column}
        \begin{column}{0.48\textwidth}
            \begin{block}{Stochastic Seasonality}
                Evolving seasonal pattern:
                $\Delta_s Y_t = Y_t - Y_{t-s}$
                exhibits dependence structure.

                \textbf{Properties:}
                \begin{itemize}\setlength{\itemsep}{0pt}
                    \item Pattern evolves over time
                    \item Requires seasonal differencing
                \end{itemize}
            \end{block}
        \end{column}
    \end{columns}
    \end{minipage}
    }
\end{frame}

\begin{frame}{Detecting Seasonality}
    {\small
    \hfill\begin{minipage}{0.9\textwidth}
    \begin{block}{Visual Methods}
        \begin{itemize}
            \item Time series plot -- look for repeating patterns
            \item Seasonal subseries plot -- compare same seasons across years
            \item ACF plot -- spikes at seasonal lags ($s, 2s, 3s, \ldots$)
        \end{itemize}
    \end{block}

    \vspace{0.1cm}

    \begin{block}{Statistical Tests}
        \begin{itemize}
            \item Seasonal unit root tests (HEGY, CH, OCSB)
            \item F-test for seasonal dummies
            \item Kruskal-Wallis test (nonparametric)
        \end{itemize}
    \end{block}

    \vspace{0.1cm}

    \begin{exampleblock}{ACF Signature}
        Strong seasonality: ACF shows significant spikes at lags $s, 2s, 3s, \ldots$
    \end{exampleblock}
    \end{minipage}
    }
\end{frame}

\begin{frame}{ACF Reveals Seasonal Structure}
    \vspace{-0.3cm}
    \begin{center}
        \includegraphics[width=0.82\textwidth, height=0.55\textheight, keepaspectratio]{charts/ch4_acf_seasonality.pdf}
    \end{center}
    \vspace{-0.2cm}
    {\footnotesize
    \begin{itemize}
        \item \textbf{Slow decay} at all lags indicates non-stationarity (trend)
        \item \textbf{Spikes at lags 12, 24, 36} confirm seasonal pattern ($s=12$)
        \item ACF at seasonal lags shows slow decay $\Rightarrow$ needs seasonal differencing
    \end{itemize}
    }
\end{frame}

\begin{frame}{F-test for Seasonal Dummies: Intuition}
    {\small
    \hfill\begin{minipage}{0.9\textwidth}
    \begin{block}{What Does This Test Do?}
        Tests whether the \textbf{mean values differ significantly across seasons}.
        \begin{itemize}\setlength{\itemsep}{0pt}
            \item If January mean $\neq$ February mean $\neq$ ... $\neq$ December mean $\Rightarrow$ seasonality
            \item Compares a model WITH seasonal dummies vs. a model WITHOUT
        \end{itemize}
    \end{block}

    \vspace{0.1cm}

    \begin{block}{The Models Being Compared}
        \textbf{Restricted}: $Y_t = \alpha + \varepsilon_t$ \quad \textbf{Unrestricted}: $Y_t = \alpha + \sum_{j=1}^{s-1} \gamma_j D_{jt} + \varepsilon_t$

        where $D_{jt} = 1$ if observation $t$ is in season $j$, 0 otherwise.
    \end{block}

    \vspace{0.1cm}

    \begin{alertblock}{Key Idea}
        If adding seasonal dummies \textbf{significantly reduces} prediction errors, then seasonality is present.
    \end{alertblock}
    \end{minipage}
    }
\end{frame}

\begin{frame}{F-test for Seasonal Dummies: Formula and Example}
    \begin{block}{F-statistic Formula}
        $$F = \frac{(SSR_R - SSR_U)/(s-1)}{SSR_U/(n-s)} \sim F_{s-1, n-s}$$
        \begin{itemize}
            \item $SSR_R$ = Sum of Squared Residuals from restricted model (no dummies)
            \item $SSR_U$ = Sum of Squared Residuals from unrestricted model (with dummies)
            \item $s-1$ = number of restrictions (monthly: 11, quarterly: 3)
        \end{itemize}
    \end{block}

    \vspace{0.1cm}

    \begin{exampleblock}{Numerical Example (Monthly Data, n=120)}
        $SSR_R = 15000$, $SSR_U = 8500$, $s = 12$

        $$F = \frac{(15000 - 8500)/11}{8500/108} = \frac{590.9}{78.7} = 7.51$$

        Critical value $F_{0.05, 11, 108} \approx 1.87$. Since $7.51 > 1.87$: \textbf{Reject $H_0$} $\Rightarrow$ Seasonality present!
    \end{exampleblock}
\end{frame}

\begin{frame}{Kruskal-Wallis Test: Intuition}
    {\small
    \hfill\begin{minipage}{0.9\textwidth}
    \begin{block}{What Does This Test Do?}
        A \textbf{nonparametric} test that checks if observations from different seasons come from the same distribution.
        \begin{itemize}\setlength{\itemsep}{0pt}
            \item Works by \textbf{ranking} all observations from smallest to largest
            \item Checks if ranks are evenly distributed across seasons
            \item If one season consistently has higher/lower ranks $\Rightarrow$ seasonality
        \end{itemize}
    \end{block}

    \vspace{0.1cm}

    \begin{exampleblock}{Why Use It Instead of F-test?}
        \begin{itemize}\setlength{\itemsep}{0pt}
            \item \textbf{No normality assumption} -- works with any distribution
            \item \textbf{Robust to outliers} -- extreme values don't distort results
        \end{itemize}
    \end{exampleblock}

    \vspace{0.1cm}

    \begin{alertblock}{Limitation}
        Less powerful than F-test when data IS normally distributed.
    \end{alertblock}
    \end{minipage}
    }
\end{frame}

\begin{frame}{Kruskal-Wallis Test: Formula and Example}
    {\footnotesize
    \hfill\begin{minipage}{0.9\textwidth}
    \begin{block}{Test Statistic}
        $H = \frac{12}{N(N+1)} \sum_{j=1}^{s} \frac{R_j^2}{n_j} - 3(N+1)$ \quad where $N$ = total obs., $n_j$ = obs. in season $j$, $R_j$ = sum of ranks.
    \end{block}

    \begin{exampleblock}{Example: Quarterly Sales (n=20, s=4)}
        Data ranked 1-20. Rank sums: Q1: $R_1 = 15$, Q2: $R_2 = 35$, Q3: $R_3 = 70$, Q4: $R_4 = 90$
        $$H = \frac{12}{20 \times 21}\left(\frac{15^2}{5} + \frac{35^2}{5} + \frac{70^2}{5} + \frac{90^2}{5}\right) - 3(21) = 12.6$$
        Critical value $\chi^2_{0.05, 3} = 7.81$. Since $12.6 > 7.81$: \textbf{Reject $H_0$} $\Rightarrow$ Seasonality!
    \end{exampleblock}

    \begin{alertblock}{In Python}
        \texttt{scipy.stats.kruskal(q1, q2, q3, q4)}
    \end{alertblock}
    \end{minipage}
    }
\end{frame}

\begin{frame}{HEGY Test: What Problem Does It Solve?}
    {\small
    \hfill\begin{minipage}{0.9\textwidth}
    \begin{block}{The Key Question}
        Given a seasonal time series, we need to know:
        \begin{enumerate}\setlength{\itemsep}{0pt}
            \item Does it need \textbf{regular differencing} $(1-L)$? $\Rightarrow$ set $d=1$
            \item Does it need \textbf{seasonal differencing} $(1-L^s)$? $\Rightarrow$ set $D=1$
        \end{enumerate}
        HEGY tests for \textbf{both} types of unit roots simultaneously!
    \end{block}

    \begin{exampleblock}{Why Not Just Use ADF?}
        ADF only tests for a \textbf{regular} unit root at frequency zero. Seasonal data can have unit roots at \textbf{seasonal frequencies} that ADF misses!
    \end{exampleblock}

    \begin{alertblock}{HEGY Tests Multiple Frequencies}
        Quarterly: tests at 0, $\pi$, $\pm\pi/2$. Monthly: tests at 0, $\pi$, $\pm\pi/6$, $\pm\pi/3$, $\pm\pi/2$, $\pm 2\pi/3$, $\pm 5\pi/6$.
    \end{alertblock}
    \end{minipage}
    }
\end{frame}

\begin{frame}{HEGY Test: The Regression Formula (Quarterly)}
    {\footnotesize
    \hfill\begin{minipage}{0.9\textwidth}
    \begin{block}{HEGY Auxiliary Regression}
        For quarterly data ($s=4$), estimate:
        $$\Delta_4 y_t = \pi_1 z_{1,t-1} + \pi_2 z_{2,t-1} + \pi_3 z_{3,t-2} + \pi_4 z_{4,t-2} + \sum_{j=1}^{k} \phi_j \Delta_4 y_{t-j} + \varepsilon_t$$
    \end{block}
    \vspace{-0.1cm}
    \begin{block}{Transformed Variables}
        \vspace{-0.3cm}
        \begin{align*}
            z_{1t} &= (1+L+L^2+L^3)y_t = y_t + y_{t-1} + y_{t-2} + y_{t-3} \\[-0.1cm]
            z_{2t} &= -(1-L+L^2-L^3)y_t = -y_t + y_{t-1} - y_{t-2} + y_{t-3} \\[-0.1cm]
            z_{3t} &= -(1-L^2)y_t = -y_t + y_{t-2} \quad;\quad z_{4t} = -(L-L^3)y_t = -y_{t-1} + y_{t-3}
        \end{align*}
        \vspace{-0.3cm}
    \end{block}
    \vspace{-0.1cm}
    \begin{alertblock}{Hypotheses}
        $H_0: \pi_1=0$ (freq.\ 0), $H_0: \pi_2=0$ (freq.\ $\pi$), $H_0: \pi_3=\pi_4=0$ (freq.\ $\pm\pi/2$)
    \end{alertblock}
    \end{minipage}
    }
\end{frame}

\begin{frame}{HEGY Test: Decision Rules with Examples}
    {\footnotesize
    \hfill\begin{minipage}{0.9\textwidth}
    \begin{block}{HEGY Critical Values (5\%, n=100, with constant)}
        \begin{tabular}{lccc}
            \toprule
            Test & Statistic & Critical Value & If NOT rejected... \\
            \midrule
            $t_1$ ($\pi_1=0$) & t-stat & $-2.88$ & Need $d=1$ \\
            $t_2$ ($\pi_2=0$) & t-stat & $-2.88$ & Need $D=1$ \\
            $F_{34}$ ($\pi_3=\pi_4=0$) & F-stat & $6.57$ & Need $D=1$ \\
            \bottomrule
        \end{tabular}
    \end{block}

    \begin{exampleblock}{Example: Quarterly GDP}
        Suppose HEGY gives: $t_1 = -1.52$, $t_2 = -4.21$, $F_{34} = 2.15$
        \begin{itemize}\setlength{\itemsep}{0pt}
            \item $t_1 = -1.52 > -2.88$: Cannot reject $\Rightarrow$ \textbf{need $d=1$}
            \item $t_2 = -4.21 < -2.88$: Reject $\Rightarrow$ no unit root at $\pi$
            \item $F_{34} = 2.15 < 6.57$: Cannot reject $\Rightarrow$ \textbf{need $D=1$}
        \end{itemize}
        \textbf{Conclusion}: Use SARIMA with $d=1, D=1$
    \end{exampleblock}
    \end{minipage}
    }
\end{frame}

\begin{frame}{Canova-Hansen Test: The Opposite of HEGY}
    {\footnotesize
    \hfill\begin{minipage}{0.9\textwidth}
    \begin{block}{HEGY vs Canova-Hansen: Different Null Hypotheses!}
        \begin{center}
        \begin{tabular}{lcc}
            \toprule
            & \textbf{HEGY} & \textbf{Canova-Hansen} \\
            \midrule
            $H_0$ & Seasonal unit root & \textbf{No} seasonal unit root \\
            $H_1$ & No seasonal unit root & Seasonal unit root \\
            \midrule
            Reject $H_0$ & Use seasonal dummies & Use $(1-L^s)$ differencing \\
            Don't reject & Use $(1-L^s)$ differencing & Use seasonal dummies \\
            \bottomrule
        \end{tabular}
        \end{center}
    \end{block}

    \begin{alertblock}{Why Does This Matter?}
        \begin{itemize}\setlength{\itemsep}{0pt}
            \item HEGY: ``Prove there's NO unit root'' (conservative toward differencing)
            \item CH: ``Prove there IS a unit root'' (conservative toward dummies)
            \item Use \textbf{both} tests for robust conclusions!
        \end{itemize}
    \end{alertblock}
    \end{minipage}
    }
\end{frame}

\begin{frame}{Canova-Hansen Test: Formula}
    {\footnotesize
    \hfill\begin{minipage}{0.9\textwidth}
    \begin{block}{Test Procedure}
        1. Regress $y_t$ on seasonal dummies: $y_t = \sum_{j=1}^{s} \gamma_j D_{jt} + u_t$

        2. Compute partial sums at seasonal frequency $\lambda_i$:
        $S_{it}^{(c)} = \sum_{j=1}^{t} \hat{u}_j \cos(\lambda_i j)$, \; $S_{it}^{(s)} = \sum_{j=1}^{t} \hat{u}_j \sin(\lambda_i j)$
    \end{block}
    \vspace{-0.1cm}
    \begin{block}{LM Test Statistic}
        $$LM_i = \frac{1}{T^2 \hat{\omega}_i} \left[ \sum_{t=1}^{T} (S_{it}^{(c)})^2 + \sum_{t=1}^{T} (S_{it}^{(s)})^2 \right]$$
        where $\hat{\omega}_i$ = consistent estimate of spectral density at frequency $\lambda_i$.
    \end{block}
    \vspace{-0.1cm}
    \begin{alertblock}{Decision}
        Reject $H_0$ (stationarity) if $LM > $ critical value $\Rightarrow$ seasonal differencing needed.
    \end{alertblock}
    \end{minipage}
    }
\end{frame}

\begin{frame}{Summary: Choosing the Right Seasonality Test}
    {\footnotesize
    \hfill\begin{minipage}{0.9\textwidth}
    \begin{center}
    \begin{tabular}{p{2cm}p{2.5cm}p{2.5cm}p{3cm}}
        \toprule
        \textbf{Test} & \textbf{$H_0$} & \textbf{If Reject} & \textbf{Best For} \\
        \midrule
        F-test & No seasonality & Seasonality exists & Normal data \\
        Kruskal-Wallis & No seasonal diff. & Seasonality exists & Non-normal, outliers \\
        HEGY & Unit root exists & Use dummies & Determining $d$, $D$ \\
        Canova-Hansen & No unit root & Use $(1-L^s)$ & Confirming stability \\
        \bottomrule
    \end{tabular}
    \end{center}

    \vspace{0.1cm}

    \begin{alertblock}{Key Insight}
        F-test/Kruskal-Wallis: ``\textit{Is there seasonality?}'' \\
        HEGY/Canova-Hansen: ``\textit{What type?}'' (deterministic vs stochastic)
    \end{alertblock}
    \end{minipage}
    }
\end{frame}

%=============================================================================
% SECTION 2: SEASONAL DIFFERENCING
%=============================================================================
\section{Seasonal Differencing}

\begin{frame}{The Seasonal Difference Operator}
    \begin{defn}[Seasonal Difference]
        The \textbf{seasonal difference operator} $\Delta_s$ is defined as:
        $$\Delta_s Y_t = (1 - L^s) Y_t = Y_t - Y_{t-s}$$
        where $L^s Y_t = Y_{t-s}$ is the seasonal lag operator.
    \end{defn}

    \vspace{0.1cm}

    \begin{exampleblock}{Examples}
        \begin{itemize}
            \item Monthly data ($s=12$): $\Delta_{12} Y_t = Y_t - Y_{t-12}$

            Compares each month to the same month last year
            \item Quarterly data ($s=4$): $\Delta_4 Y_t = Y_t - Y_{t-4}$

            Compares each quarter to the same quarter last year
        \end{itemize}
    \end{exampleblock}
\end{frame}

\begin{frame}{Seasonal Difference: Visual Illustration}
    \begin{center}
        \includegraphics[width=0.95\textwidth]{charts/ch4_def_seasonal_diff.pdf}
    \end{center}
    \vspace{-0.2cm}
    \small Left: original series with seasonal pattern. Right: after $\Delta_{12}$, seasonal pattern is removed.
\end{frame}

\begin{frame}{Combining Regular and Seasonal Differencing}
    \begin{block}{Full Differencing}
        For series with both trend and seasonality:
        $$\Delta \Delta_s Y_t = (1-L)(1-L^s) Y_t$$
    \end{block}

    \vspace{0.1cm}

    \begin{exampleblock}{Expansion}
        $(1-L)(1-L^s) Y_t = Y_t - Y_{t-1} - Y_{t-s} + Y_{t-s-1}$

        For monthly data ($s=12$):
        $$\Delta \Delta_{12} Y_t = Y_t - Y_{t-1} - Y_{t-12} + Y_{t-13}$$
    \end{exampleblock}

    \vspace{0.1cm}

    \begin{alertblock}{Order of Differencing}
        \begin{itemize}
            \item $d$: number of regular differences (trend removal)
            \item $D$: number of seasonal differences (seasonal trend removal)
        \end{itemize}
    \end{alertblock}
\end{frame}

\begin{frame}{Effect of Differencing Operations}
    \vspace{-0.3cm}
    \begin{center}
        \includegraphics[width=0.82\textwidth, height=0.6\textheight, keepaspectratio]{charts/ch4_differencing_effect.pdf}
    \end{center}
    \vspace{-0.2cm}
    {\footnotesize
    \begin{itemize}
        \item Regular differencing removes trend but seasonal pattern remains
        \item Seasonal differencing removes seasonality but trend pattern remains
        \item \textbf{Both differences} needed to achieve stationarity
    \end{itemize}
    }
\end{frame}

\begin{frame}{ACF Before and After Differencing}
    \vspace{-0.3cm}
    \begin{center}
        \includegraphics[width=0.82\textwidth, height=0.6\textheight, keepaspectratio]{charts/ch4_acf_differencing.pdf}
    \end{center}
    \vspace{-0.2cm}
    {\footnotesize
    \begin{itemize}
        \item Original ACF: slow decay indicates non-stationarity
        \item After $\Delta$: seasonal spikes remain at lags 12, 24, 36
        \item After $\Delta_{12}$: trend decay remains at early lags
        \item After $\Delta\Delta_{12}$: ACF cuts off $\Rightarrow$ \textbf{stationary}
    \end{itemize}
    }
\end{frame}

\begin{frame}{Seasonal Integration}
    \begin{defn}[Seasonally Integrated Process]
        A series $Y_t$ is \textbf{seasonally integrated} of order $(d, D)_s$, written $Y_t \sim I(d, D)_s$, if:
        $$(1-L)^d (1-L^s)^D Y_t$$
        is stationary.
    \end{defn}

    \vspace{0.1cm}

    \begin{exampleblock}{Common Cases}
        \begin{itemize}
            \item $I(1,0)_{12}$: Regular unit root only (monthly)
            \item $I(0,1)_{12}$: Seasonal unit root only
            \item $I(1,1)_{12}$: Both regular and seasonal unit roots
        \end{itemize}
    \end{exampleblock}
\end{frame}

%=============================================================================
% SECTION 3: SARIMA MODEL DEFINITION
%=============================================================================
\section{The SARIMA Model}

\begin{frame}{SARIMA Model Definition}
    \begin{defn}[SARIMA$(p,d,q)\times(P,D,Q)_s$]
        The \textbf{Seasonal ARIMA} model is:
        $$\phi(L)\Phi(L^s)(1-L)^d(1-L^s)^D Y_t = c + \theta(L)\Theta(L^s)\varepsilon_t$$
    \end{defn}

    {\footnotesize
    \begin{block}{Components}
        \begin{itemize}\setlength{\itemsep}{0pt}
            \item $\phi(L) = 1 - \phi_1 L - \cdots - \phi_p L^p$: Non-seasonal AR
            \item $\Phi(L^s) = 1 - \Phi_1 L^s - \cdots - \Phi_P L^{Ps}$: Seasonal AR
            \item $\theta(L) = 1 + \theta_1 L + \cdots + \theta_q L^q$: Non-seasonal MA
            \item $\Theta(L^s) = 1 + \Theta_1 L^s + \cdots + \Theta_Q L^{Qs}$: Seasonal MA
            \item $(1-L)^d$: Regular differencing; $(1-L^s)^D$: Seasonal differencing
        \end{itemize}
    \end{block}
    }
\end{frame}

\begin{frame}{SARIMA: Visual Illustration}
    \begin{center}
        \includegraphics[width=0.95\textwidth]{charts/ch4_def_sarima.pdf}
    \end{center}
    \vspace{-0.2cm}
    \small Progressive differencing: original $\to$ regular diff $\to$ seasonal diff $\to$ both differences.
\end{frame}

\begin{frame}{SARIMA Notation}
    \begin{block}{Full Specification}
        SARIMA$(p,d,q)\times(P,D,Q)_s$ has 7 parameters to specify:
    \end{block}

    \vspace{0.1cm}

    \begin{table}
        \centering
        \small
        \begin{tabular}{ll}
            \toprule
            \textbf{Parameter} & \textbf{Meaning} \\
            \midrule
            $p$ & Non-seasonal AR order \\
            $d$ & Non-seasonal differencing order \\
            $q$ & Non-seasonal MA order \\
            $P$ & Seasonal AR order \\
            $D$ & Seasonal differencing order \\
            $Q$ & Seasonal MA order \\
            $s$ & Seasonal period \\
            \bottomrule
        \end{tabular}
    \end{table}

    \vspace{0.1cm}

    \begin{exampleblock}{Example}
        {\small SARIMA$(1,1,1)\times(1,1,1)_{12}$: Monthly data with AR(1), MA(1), seasonal AR(1), seasonal MA(1), and both regular and seasonal differencing.}
    \end{exampleblock}
\end{frame}

\begin{frame}{Common SARIMA Models}
    {\small
    \hfill\begin{minipage}{0.9\textwidth}
    \begin{block}{Airline Model: SARIMA$(0,1,1)\times(0,1,1)_s$}
        $(1-L)(1-L^s)Y_t = (1+\theta L)(1+\Theta L^s)\varepsilon_t$ -- Classic model (Box \& Jenkins, 1970)
    \end{block}

    \begin{block}{SARIMA$(1,0,0)\times(1,0,0)_s$}
        $(1-\phi L)(1-\Phi L^s)Y_t = \varepsilon_t$ -- Pure seasonal and non-seasonal AR
    \end{block}

    \begin{block}{SARIMA$(0,1,1)\times(0,1,0)_s$}
        $(1-L)(1-L^s)Y_t = (1+\theta L)\varepsilon_t$ -- Random walk + seasonal diff + MA(1)
    \end{block}
    \end{minipage}
    }
\end{frame}

\begin{frame}{The Multiplicative Structure}
    \begin{block}{Why Multiplicative?}
        The seasonal and non-seasonal parts \textbf{multiply}:
        $$\phi(L)\Phi(L^s) \quad \text{and} \quad \theta(L)\Theta(L^s)$$
    \end{block}

    \vspace{0.1cm}

    \begin{exampleblock}{Example: SARIMA$(1,0,0)\times(1,0,0)_{12}$}
        $(1-\phi L)(1-\Phi L^{12})Y_t = \varepsilon_t$

        Expanding:
        $Y_t - \phi Y_{t-1} - \Phi Y_{t-12} + \phi\Phi Y_{t-13} = \varepsilon_t$

        The cross-term $\phi\Phi Y_{t-13}$ captures interaction!
    \end{exampleblock}

    \vspace{0.1cm}

    \begin{alertblock}{Interpretation}
        Multiplicative structure allows parsimonious modeling of complex seasonal patterns with few parameters.
    \end{alertblock}
\end{frame}

%=============================================================================
% SECTION 4: SEASONAL ACF AND PACF
%=============================================================================
\section{Seasonal ACF and PACF Patterns}

\begin{frame}{ACF/PACF for Seasonal Models}
    \begin{block}{Key Insight}
        Seasonal models show patterns at both:
        \begin{itemize}
            \item Non-seasonal lags: $1, 2, 3, \ldots$
            \item Seasonal lags: $s, 2s, 3s, \ldots$
        \end{itemize}
    \end{block}

    \vspace{0.1cm}

    \begin{table}
        \centering
        \small
        \begin{tabular}{lcc}
            \toprule
            \textbf{Model} & \textbf{ACF} & \textbf{PACF} \\
            \midrule
            SAR($P$) & Decays at $s, 2s, \ldots$ & Cuts off after $Ps$ \\
            SMA($Q$) & Cuts off after $Qs$ & Decays at $s, 2s, \ldots$ \\
            SARMA & Decays at seasonal lags & Decays at seasonal lags \\
            \bottomrule
        \end{tabular}
    \end{table}
\end{frame}

\begin{frame}{Example: Airline Model ACF/PACF}
    \begin{block}{SARIMA$(0,1,1)\times(0,1,1)_{12}$}
        After differencing $W_t = (1-L)(1-L^{12})Y_t$:
        $$W_t = (1+\theta L)(1+\Theta L^{12})\varepsilon_t$$
    \end{block}

    \vspace{0.1cm}

    {\small
    \begin{exampleblock}{Expected ACF Pattern}
        \begin{itemize}
            \item Spike at lag 1 (from $\theta$)
            \item Spike at lag 12 (from $\Theta$)
            \item Spike at lag 13 (from $\theta \cdot \Theta$ interaction)
            \item All other lags near zero
        \end{itemize}
    \end{exampleblock}

    \begin{exampleblock}{Expected PACF Pattern}
        \begin{itemize}
            \item Exponential decay at lags $1, 2, 3, \ldots$
            \item Exponential decay at lags $12, 24, 36, \ldots$
        \end{itemize}
    \end{exampleblock}
    }
\end{frame}

\begin{frame}{Model Identification Guidelines}
    {\small
    \hfill\begin{minipage}{0.9\textwidth}
    \begin{block}{Step-by-Step Process}
        \begin{enumerate}
            \item Examine ACF for slow decay at seasonal lags $\Rightarrow$ seasonal differencing
            \item After differencing, check ACF/PACF patterns
            \item Non-seasonal behavior at lags $1, 2, \ldots, s-1$
            \item Seasonal behavior at lags $s, 2s, 3s, \ldots$
        \end{enumerate}
    \end{block}

    \vspace{0.1cm}

    \begin{alertblock}{Practical Tips}
        \begin{itemize}
            \item Start with $d \leq 1$ and $D \leq 1$
            \item Usually $P, Q \leq 2$ is sufficient
            \item Use information criteria (AIC, BIC) for final selection
            \item Auto-SARIMA algorithms can help
        \end{itemize}
    \end{alertblock}
    \end{minipage}
    }
\end{frame}

%=============================================================================
% SECTION 5: ESTIMATION AND DIAGNOSTICS
%=============================================================================
\section{Estimation and Diagnostics}

\begin{frame}{Estimation Methods}
    \begin{block}{Maximum Likelihood Estimation}
        Standard approach for SARIMA:
        \begin{itemize}
            \item Conditional MLE (conditional on initial values)
            \item Exact MLE (via Kalman filter)
        \end{itemize}
    \end{block}

    \vspace{0.1cm}

    \begin{block}{Computational Considerations}
        \begin{itemize}
            \item More parameters than ARIMA $\Rightarrow$ more data needed
            \item Seasonal parameters estimated from lags $s, 2s, \ldots$
            \item Need sufficient seasonal cycles (at least 3-4 years of monthly data)
        \end{itemize}
    \end{block}
\end{frame}

\begin{frame}{Stationarity and Invertibility}
    \begin{block}{Stationarity Conditions}
        Both non-seasonal and seasonal AR polynomials must have roots outside the unit circle:
        \begin{itemize}
            \item $\phi(z) = 0 \Rightarrow |z| > 1$
            \item $\Phi(z^s) = 0 \Rightarrow |z| > 1$
        \end{itemize}
    \end{block}

    \vspace{0.1cm}

    \begin{block}{Invertibility Conditions}
        Both non-seasonal and seasonal MA polynomials must have roots outside the unit circle:
        \begin{itemize}
            \item $\theta(z) = 0 \Rightarrow |z| > 1$
            \item $\Theta(z^s) = 0 \Rightarrow |z| > 1$
        \end{itemize}
    \end{block}
\end{frame}

\begin{frame}{Diagnostic Checking}
    \begin{block}{Residual Analysis}
        After fitting SARIMA, check that residuals are white noise:
        \begin{enumerate}
            \item Plot residuals over time (no patterns)
            \item ACF of residuals (no significant spikes)
            \item Ljung-Box test at multiple lags including seasonal
            \item Normality tests (Q-Q plot, Jarque-Bera)
        \end{enumerate}
    \end{block}

    \vspace{0.1cm}

    \begin{alertblock}{Important}
        Check ACF at \textbf{both} non-seasonal and seasonal lags!

        Significant ACF at lag 12 suggests inadequate seasonal modeling.
    \end{alertblock}
\end{frame}

\begin{frame}{Model Selection Criteria}
    \begin{block}{Information Criteria}
        Compare competing SARIMA models using:
        \begin{itemize}
            \item AIC = $-2\ln(L) + 2k$
            \item BIC = $-2\ln(L) + k\ln(n)$
            \item AICc = AIC + $\frac{2k(k+1)}{n-k-1}$ (corrected for small samples)
        \end{itemize}
        where $k = p + q + P + Q + 1$ (plus 1 for variance).
    \end{block}

    \vspace{0.1cm}

    \begin{exampleblock}{Auto-SARIMA}
        Python's \texttt{pmdarima.auto\_arima()} with \texttt{seasonal=True} automatically searches for optimal $(p,d,q)\times(P,D,Q)_s$.
    \end{exampleblock}
\end{frame}

%=============================================================================
% SECTION 6: FORECASTING
%=============================================================================
\section{Forecasting with SARIMA}

\begin{frame}{Point Forecasts}
    \begin{block}{Forecast Computation}
        SARIMA forecasts are computed recursively:
        \begin{itemize}
            \item Replace future $\varepsilon_{T+h}$ with 0
            \item Replace future $Y_{T+h}$ with forecasts $\hat{Y}_{T+h|T}$
            \item Use known past values $Y_T, Y_{T-1}, \ldots$
        \end{itemize}
    \end{block}

    \vspace{0.1cm}

    \begin{exampleblock}{Seasonal Pattern in Forecasts}
        SARIMA forecasts naturally capture seasonality:
        \begin{itemize}
            \item Short-term: influenced by recent values
            \item Long-term: revert to seasonal pattern
        \end{itemize}
    \end{exampleblock}
\end{frame}

\begin{frame}{Forecast Intervals}
    \begin{block}{Uncertainty Quantification}
        $(1-\alpha)$\% prediction interval:
        $$\hat{Y}_{T+h|T} \pm z_{\alpha/2} \sqrt{\Var(e_{T+h})}$$

        Variance computed from MA($\infty$) representation.
    \end{block}

    \vspace{0.1cm}

    \begin{alertblock}{Key Properties}
        \begin{itemize}
            \item Intervals widen with forecast horizon
            \item For $I(1,1)_s$ series: intervals grow without bound
            \item Seasonal pattern visible in point forecasts
            \item Uncertainty captures both trend and seasonal variation
        \end{itemize}
    \end{alertblock}
\end{frame}

\begin{frame}{Long-Horizon Forecasts}
    \begin{block}{Behavior as $h \to \infty$}
        \begin{itemize}
            \item Point forecasts converge to deterministic seasonal pattern
            \item If drift present: linear trend + seasonal pattern
            \item Forecast intervals continue to widen
        \end{itemize}
    \end{block}

    \vspace{0.1cm}

    \begin{exampleblock}{Practical Implication}
        \begin{itemize}
            \item Short-term: SARIMA captures both level and season
            \item Medium-term: Good seasonal forecasts, growing uncertainty
            \item Long-term: Mainly reflects seasonal pattern, wide intervals
        \end{itemize}
    \end{exampleblock}
\end{frame}

%=============================================================================
% SECTION 7: REAL DATA APPLICATION
%=============================================================================
\section{Real Data Application: Airline Passengers}

\begin{frame}{Airline Passengers Data}
    \vspace{-0.3cm}
    \begin{center}
        \includegraphics[width=0.75\textwidth, height=0.55\textheight, keepaspectratio]{charts/ch4_airline_data.pdf}
    \end{center}
    \vspace{-0.2cm}
    {\small
    \begin{itemize}
        \item Classic dataset: Monthly international airline passengers (1949-1960)
        \item Clear upward trend and growing seasonal amplitude
    \end{itemize}
    }
\end{frame}

\begin{frame}{Seasonal Decomposition}
    \vspace{-0.3cm}
    \begin{center}
        \includegraphics[width=0.78\textwidth, height=0.6\textheight, keepaspectratio]{charts/ch4_decomposition.pdf}
    \end{center}
    \vspace{-0.2cm}
    {\footnotesize
    \begin{itemize}
        \item Trend: Strong upward growth
        \item Seasonality: Summer peaks (vacation travel)
        \item Residual: Random variation after removing trend and season
    \end{itemize}
    }
\end{frame}

\begin{frame}{ACF/PACF Analysis}
    \vspace{-0.3cm}
    \begin{center}
        \includegraphics[width=0.75\textwidth, height=0.58\textheight, keepaspectratio]{charts/ch4_acf_pacf.pdf}
    \end{center}
    \vspace{-0.2cm}
    {\footnotesize
    \begin{itemize}
        \item After $\Delta\Delta_{12}$ differencing: spikes at lags 1 and 12
        \item Suggests SARIMA$(0,1,1)\times(0,1,1)_{12}$ (Airline model)
    \end{itemize}
    }
\end{frame}

\begin{frame}{SARIMA Forecast Results}
    \vspace{-0.3cm}
    \begin{center}
        \includegraphics[width=0.78\textwidth, height=0.55\textheight, keepaspectratio]{charts/ch4_sarima_forecast.pdf}
    \end{center}
    \vspace{-0.2cm}
    {\small
    \begin{itemize}
        \item SARIMA captures both trend and seasonal pattern
        \item Forecasts show appropriate seasonal peaks and troughs
    \end{itemize}
    }
\end{frame}

\begin{frame}{Model Diagnostics}
    \vspace{-0.3cm}
    \begin{center}
        \includegraphics[width=0.72\textwidth, height=0.6\textheight, keepaspectratio]{charts/ch4_diagnostics.pdf}
    \end{center}
    \vspace{-0.2cm}
    {\footnotesize
    \begin{itemize}
        \item Residuals appear random; ACF within bounds at all lags
        \item Model adequately captures seasonal structure
    \end{itemize}
    }
\end{frame}

\begin{frame}{Python Implementation}
    {\small
    \begin{block}{Fitting SARIMA in Python}
        \texttt{from statsmodels.tsa.statespace.sarimax import SARIMAX}

        \vspace{0.1cm}
        \texttt{model = SARIMAX(y, order=(0,1,1), seasonal\_order=(0,1,1,12))}

        \texttt{results = model.fit()}

        \texttt{forecast = results.get\_forecast(steps=24)}
    \end{block}

    \vspace{0.1cm}

    \begin{alertblock}{Note}
        Complete Python examples with comments are provided in the Jupyter notebooks.
    \end{alertblock}
    }
\end{frame}

%=============================================================================
% SECTION 8: SUMMARY
%=============================================================================
\section{Summary}

\begin{frame}{Key Takeaways}
    {\small
    \hfill\begin{minipage}{0.9\textwidth}
    \begin{block}{Main Points}
        \begin{enumerate}
            \item \textbf{Seasonality} is common in economic and business data
            \item \textbf{Seasonal differencing} $(1-L^s)$ removes stochastic seasonality
            \item \textbf{SARIMA}$(p,d,q)\times(P,D,Q)_s$ extends ARIMA for seasonal data
            \item \textbf{Multiplicative structure} captures seasonal-trend interactions
            \item \textbf{ACF/PACF} show patterns at both regular and seasonal lags
            \item \textbf{Model selection}: Use AIC/BIC or auto-SARIMA algorithms
        \end{enumerate}
    \end{block}

    \vspace{0.1cm}

    \begin{alertblock}{Next Steps}
        Chapter 5 will cover multivariate time series: VAR models, Granger causality, and cointegration.
    \end{alertblock}
    \end{minipage}
    }
\end{frame}

%=============================================================================
% SECTION 9: QUIZ
%=============================================================================
\section{Quiz}

\begin{frame}{Quiz Question 1}
    \begin{alertblock}{Question}
        For monthly data with annual seasonality, what is the seasonal period $s$?
    \end{alertblock}

    \vspace{0.3cm}

    \begin{enumerate}[(A)]
        \item $s = 4$
        \item $s = 7$
        \item $s = 12$
        \item $s = 52$
    \end{enumerate}
\end{frame}

\begin{frame}{Quiz Question 1: Answer}
    \begin{exampleblock}{Correct Answer: (C) $s = 12$ (12 months per year)}
        Common periods: Quarterly=4, Monthly=12, Weekly=52, Daily=7, Hourly=24
    \end{exampleblock}
    \vspace{0.2cm}
    \begin{center}
        \includegraphics[width=0.95\textwidth, height=0.55\textheight, keepaspectratio]{charts/ch4_quiz1_seasonal_periods.pdf}
    \end{center}
\end{frame}

\begin{frame}{Quiz Question 2}
    \begin{alertblock}{Question}
        What does the seasonal difference operator $(1 - L^{12})$ do to a monthly series?
    \end{alertblock}

    \vspace{0.3cm}

    \begin{enumerate}[(A)]
        \item Computes $Y_t - Y_{t-1}$ (month-to-month change)
        \item Computes $Y_t - Y_{t-12}$ (year-over-year change)
        \item Computes the 12-month moving average
        \item Removes the trend component only
    \end{enumerate}
\end{frame}

\begin{frame}{Quiz Question 2: Answer}
    \begin{exampleblock}{Correct Answer: (B) Year-over-year change}
        $(1 - L^{12})Y_t = Y_t - Y_{t-12}$ removes the seasonal pattern by comparing same months.
    \end{exampleblock}
    \vspace{0.2cm}
    \begin{center}
        \includegraphics[width=0.95\textwidth, height=0.55\textheight, keepaspectratio]{charts/ch4_quiz2_seasonal_diff.pdf}
    \end{center}
\end{frame}

\begin{frame}{Quiz Question 3}
    \begin{alertblock}{Question}
        In SARIMA$(1,1,1)\times(1,1,1)_{12}$ notation, what does the $(1,1,1)_{12}$ part represent?
    \end{alertblock}

    \vspace{0.3cm}

    \begin{enumerate}[(A)]
        \item AR(1), differencing once, MA(1) at the regular level
        \item Seasonal AR(1), seasonal differencing once, seasonal MA(1)
        \item 12 AR terms, 12 differences, 12 MA terms
        \item The model has 12 parameters in total
    \end{enumerate}
\end{frame}

\begin{frame}{Quiz Question 3: Answer}
    \begin{exampleblock}{Correct Answer: (B)}
        Seasonal AR(1), seasonal differencing once, seasonal MA(1)
    \end{exampleblock}

    \begin{block}{SARIMA Notation Breakdown}
        SARIMA$(p,d,q)\times(P,D,Q)_s$:

        \vspace{0.2cm}
        \begin{tabular}{ll}
            $(p,d,q)$ & Non-seasonal: AR($p$), $d$ differences, MA($q$) \\
            $(P,D,Q)_s$ & Seasonal: SAR($P$), $D$ seasonal diffs, SMA($Q$) \\
        \end{tabular}

        \vspace{0.3cm}
        For $(1,1,1)\times(1,1,1)_{12}$:
        \begin{itemize}
            \item Non-seasonal: AR(1), one regular difference, MA(1)
            \item Seasonal: SAR(1) at lag 12, one $\Delta_{12}$, SMA(1) at lag 12
        \end{itemize}
    \end{block}
\end{frame}

\begin{frame}{Quiz Question 4}
    \begin{alertblock}{Question}
        The ``Airline Model'' is SARIMA$(0,1,1)\times(0,1,1)_{12}$. How many parameters need to be estimated (excluding variance)?
    \end{alertblock}

    \vspace{0.3cm}

    \begin{enumerate}[(A)]
        \item 1
        \item 2
        \item 4
        \item 12
    \end{enumerate}
\end{frame}

\begin{frame}{Quiz Question 4: Answer}
    \begin{exampleblock}{Correct Answer: (B)}
        2 parameters
    \end{exampleblock}

    \begin{block}{Model Structure}
        SARIMA$(0,1,1)\times(0,1,1)_{12}$:
        $$(1-L)(1-L^{12})Y_t = (1 + \theta_1 L)(1 + \Theta_1 L^{12})\varepsilon_t$$

        Parameters:
        \begin{itemize}
            \item $\theta_1$: non-seasonal MA coefficient
            \item $\Theta_1$: seasonal MA coefficient
        \end{itemize}

        Total: \textbf{2 parameters} (plus $\sigma^2$)
    \end{block}

    {\footnotesize
    \begin{alertblock}{Why ``Airline Model''?}
        Box \& Jenkins (1970) used this model to forecast international airline passengers. It's remarkably effective for many seasonal economic series!
    \end{alertblock}
    }
\end{frame}

\begin{frame}{Quiz Question 5}
    \begin{alertblock}{Question}
        You observe significant ACF spikes at lags 12, 24, and 36 in a monthly series. What does this suggest?
    \end{alertblock}

    \vspace{0.3cm}

    \begin{enumerate}[(A)]
        \item The series has a unit root
        \item The series has annual seasonality that needs seasonal differencing
        \item The series follows an AR(36) process
        \item The series is already stationary
    \end{enumerate}
\end{frame}

\begin{frame}{Quiz Question 5: Answer}
    \begin{exampleblock}{Correct Answer: (B) Needs seasonal differencing}
        ACF spikes at 12, 24, 36 = stochastic seasonality. Apply $(1 - L^{12})$ to remove it.
    \end{exampleblock}
    \vspace{0.2cm}
    \begin{center}
        \includegraphics[width=0.95\textwidth, height=0.55\textheight, keepaspectratio]{charts/ch4_quiz5_seasonal_acf.pdf}
    \end{center}
\end{frame}

\begin{frame}{Quiz Question 6}
    \begin{alertblock}{Question}
        After applying $(1-L)(1-L^{12})$ to a monthly series, the ACF shows a significant spike only at lag 1 and lag 12. What SARIMA model is suggested?
    \end{alertblock}

    \vspace{0.3cm}

    \begin{enumerate}[(A)]
        \item SARIMA$(1,1,0)\times(1,1,0)_{12}$
        \item SARIMA$(0,1,1)\times(0,1,1)_{12}$
        \item SARIMA$(1,1,1)\times(1,1,1)_{12}$
        \item SARIMA$(0,1,0)\times(0,1,0)_{12}$
    \end{enumerate}
\end{frame}

\begin{frame}{Quiz Question 6: Answer}
    \begin{exampleblock}{Correct Answer: (B)}
        SARIMA$(0,1,1)\times(0,1,1)_{12}$ (The Airline Model)
    \end{exampleblock}

    \begin{block}{ACF/PACF Identification Rules}
        For MA processes, ACF \textbf{cuts off} after lag $q$:

        \vspace{0.2cm}
        \begin{tabular}{ll}
            \textbf{Pattern} & \textbf{Suggests} \\
            \hline
            ACF spike at lag 1 only & MA(1) for non-seasonal part \\
            ACF spike at lag 12 only & SMA(1) for seasonal part \\
        \end{tabular}

        \vspace{0.2cm}
        Combined: MA(1) $\times$ SMA(1) = $(0,d,1)\times(0,D,1)_{12}$

        With $d=1$ and $D=1$ (already differenced): $(0,1,1)\times(0,1,1)_{12}$
    \end{block}
\end{frame}

\begin{frame}{References}
    \begin{thebibliography}{9}
        \bibitem{boxjenkins} Box, G.E.P., Jenkins, G.M., Reinsel, G.C., \& Ljung, G.M. (2015). \textit{Time Series Analysis: Forecasting and Control}. 5th ed. Wiley.

        \bibitem{hyndman} Hyndman, R.J. \& Athanasopoulos, G. (2021). \textit{Forecasting: Principles and Practice}. 3rd ed. OTexts.

        \bibitem{hamilton} Hamilton, J.D. (1994). \textit{Time Series Analysis}. Princeton University Press.

        \bibitem{brockwell} Brockwell, P.J. \& Davis, R.A. (2016). \textit{Introduction to Time Series and Forecasting}. 3rd ed. Springer.
    \end{thebibliography}
\end{frame}

\end{document}
