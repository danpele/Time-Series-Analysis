% Capitolul: Modele ARCH/GARCH pentru Volatilitate
% Prezentare Beamer
% Program de licenta, Academia de Studii Economice din Bucuresti

\documentclass[9pt, aspectratio=169, t]{beamer}

% Ensure content fits on slides
\setbeamersize{text margin left=18mm, text margin right=12mm}

%=============================================================================
% THEME AND STYLE CONFIGURATION
%=============================================================================
\usetheme{Madrid}
\usecolortheme{seahorse}

% IDA-Inspired Color Palette
\definecolor{MainBlue}{RGB}{26, 58, 110}
\definecolor{AccentBlue}{RGB}{42, 82, 140}
\definecolor{IDAred}{RGB}{220, 53, 69}
\definecolor{DarkGray}{RGB}{51, 51, 51}
\definecolor{MediumGray}{RGB}{128, 128, 128}
\definecolor{LightGray}{RGB}{248, 248, 248}
\definecolor{VeryLightGray}{RGB}{235, 235, 235}
\definecolor{Crimson}{RGB}{220, 53, 69}
\definecolor{Forest}{RGB}{46, 125, 50}
\definecolor{Amber}{RGB}{181, 133, 63}

\setbeamercolor{palette primary}{bg=MainBlue, fg=white}
\setbeamercolor{palette secondary}{bg=MainBlue!85, fg=white}
\setbeamercolor{palette tertiary}{bg=MainBlue!70, fg=white}
\setbeamercolor{structure}{fg=MainBlue}
\setbeamercolor{title}{fg=MainBlue}
\setbeamercolor{frametitle}{fg=MainBlue, bg=white}
\setbeamercolor{block title}{bg=MainBlue, fg=white}
\setbeamercolor{block body}{bg=VeryLightGray, fg=DarkGray}
\setbeamercolor{block title alerted}{bg=Crimson, fg=white}
\setbeamercolor{block body alerted}{bg=Crimson!8, fg=DarkGray}
\setbeamercolor{block title example}{bg=Forest, fg=white}
\setbeamercolor{block body example}{bg=Forest!8, fg=DarkGray}
\setbeamercolor{item}{fg=MainBlue}

\setbeamertemplate{navigation symbols}{}

\setbeamertemplate{footline}{
    \leavevmode%
    \hbox{%
        \begin{beamercolorbox}[wd=.333333\paperwidth,ht=2.5ex,dp=1ex,center]{author in head/foot}%
            \usebeamerfont{author in head/foot}\insertshortauthor
        \end{beamercolorbox}%
        \begin{beamercolorbox}[wd=.333333\paperwidth,ht=2.5ex,dp=1ex,center]{title in head/foot}%
            \usebeamerfont{title in head/foot}\insertshorttitle
        \end{beamercolorbox}%
        \begin{beamercolorbox}[wd=.333333\paperwidth,ht=2.5ex,dp=1ex,right]{date in head/foot}%
            \usebeamerfont{date in head/foot}\insertshortdate{}\hspace*{2em}
            \insertframenumber{} / \inserttotalframenumber\hspace*{2ex}
        \end{beamercolorbox}}%
    \vskip0pt%
}

%=============================================================================
% PACKAGES
%=============================================================================
\usepackage[utf8]{inputenc}
\usepackage[T1]{fontenc}
\usepackage{amsmath, amssymb, amsthm}
\usepackage{mathtools}
\usepackage{bm}
\usepackage{tikz}
\usetikzlibrary{arrows.meta, positioning, shapes, calc}
\usepackage{booktabs}
\usepackage{multirow}
\usepackage{array}
\usepackage{graphicx}
\usepackage{hyperref}
\hypersetup{colorlinks=false, pdfborder={0 0 0}}
\graphicspath{{logos/}{charts/}}

%=============================================================================
% THEOREM ENVIRONMENTS
%=============================================================================
\theoremstyle{definition}
\setbeamertemplate{theorems}[numbered]
\newtheorem{defn}{Definitie}
\newtheorem{thm}{Teorema}
\newtheorem{prop}{Propozitie}
\newtheorem{rmk}{Observatie}

%=============================================================================
% CUSTOM COMMANDS
%=============================================================================
\newcommand{\E}{\mathbb{E}}
\newcommand{\Var}{\text{Var}}
\newcommand{\Cov}{\text{Cov}}
\newcommand{\Corr}{\text{Corr}}
\newcommand{\R}{\mathbb{R}}
\newcommand{\N}{\mathbb{N}}
\newcommand{\Z}{\mathbb{Z}}
\newcommand{\B}{\mathbf{B}}

%=============================================================================
% TITLE INFORMATION
%=============================================================================
\title[Capitolul 5: GARCH]{Capitolul 5: Modele de Volatilitate: ARCH, GARCH și Extensii}
\subtitle{Program de licenta, Facultatea de Cibernetica, Statistica si Informatica Economica, Academia de Studii Economice din Bucuresti}
\author[Prof. dr. Daniel Traian Pele]{Prof. dr. Daniel Traian Pele\\[0.2cm]\footnotesize\texttt{danpele@ase.ro}}
\institute{Academia de Studii Economice din Bucuresti}
\date{An Universitar 2025--2026}

\begin{document}

%=============================================================================
% TITLE SLIDE
%=============================================================================
\begin{frame}[plain]
    \begin{tikzpicture}[remember picture, overlay]
        \fill[IDAred] (current page.north west) rectangle ([yshift=-0.15cm]current page.north east);
        \node[anchor=north west] at ([xshift=0.5cm, yshift=-0.3cm]current page.north west) {
            \href{https://www.ase.ro}{\includegraphics[height=1.1cm]{ase_logo.png}}
        };
        \node[anchor=north] at ([yshift=-0.3cm]current page.north) {
            \href{https://ai4efin.ase.ro}{\includegraphics[height=1.1cm]{ai4efin_logo.png}}
        };
        \node[anchor=north east] at ([xshift=-0.5cm, yshift=-0.3cm]current page.north east) {
            \href{https://www.digital-finance-msca.com}{\includegraphics[height=1.1cm]{msca_logo.png}}
        };
    \end{tikzpicture}
    \vfill
    \begin{center}
        {\Large\textcolor{MediumGray}{Analiza și Prognoza Seriilor de Timp}}\\[0.3cm]
        {\Huge\textbf{\textcolor{MainBlue}{Capitolul 5: Modele de Volatilitate}}}\\[0.5cm]
        {\Large\textcolor{IDAred}{ARCH, GARCH, EGARCH, TGARCH}}
    \end{center}
    \vfill

    \begin{tikzpicture}[remember picture, overlay]
        \fill[IDAred] (current page.south west) rectangle ([yshift=0.15cm]current page.south east);
        \node[anchor=south west] at ([xshift=0.5cm, yshift=0.8cm]current page.south west) {
            \href{https://theida.net}{\includegraphics[height=0.9cm]{ida_logo.png}}
        };
        \node[anchor=south] at ([xshift=-3cm, yshift=0.8cm]current page.south) {
            \href{https://blockchain-research-center.com}{\includegraphics[height=0.9cm]{brc_logo.png}}
        };
        \node[anchor=south] at ([yshift=0.8cm]current page.south) {
            \href{https://quantinar.com}{\includegraphics[height=0.9cm]{qr_logo.png}}
        };
        \node[anchor=south] at ([xshift=3cm, yshift=0.8cm]current page.south) {
            \href{https://quantlet.com}{\includegraphics[height=0.9cm]{ql_logo.png}}
        };
        \node[anchor=south east] at ([xshift=-0.5cm, yshift=0.8cm]current page.south east) {
            \href{https://ipe.ro/new}{\includegraphics[height=0.9cm]{acad_logo.png}}
        };
    \end{tikzpicture}
\end{frame}

%=============================================================================
% TABLE OF CONTENTS
%=============================================================================
\begin{frame}{Cuprins}
    \tableofcontents
\end{frame}

%=============================================================================
% LEARNING OBJECTIVES
%=============================================================================
\begin{frame}{Obiective de Învățare}
    \begin{block}{La finalul acestui capitol, veți fi capabili să:}
        \begin{enumerate}
            \item Înțelegeți \textbf{volatility clustering} și faptele stilizate ale randamentelor financiare
            \item Estimați și interpretați modele \textbf{ARCH} și \textbf{GARCH}
            \item Aplicați modele asimetrice (\textbf{EGARCH}, \textbf{GJR-GARCH}) pentru efectul de levier
            \item Efectuați diagnosticarea și selectarea modelelor
            \item Prognozați volatilitatea și calculați \textbf{Value at Risk (VaR)}
        \end{enumerate}
    \end{block}

    \vspace{0.3cm}

    \begin{alertblock}{Competențe Practice}
        \begin{itemize}
            \item Implementare Python cu pachetul \texttt{arch}
            \item Interpretarea parametrilor și a persistenței volatilității
            \item Calculul VaR pentru managementul riscului
        \end{itemize}
    \end{alertblock}
\end{frame}

%=============================================================================
% SECTION 1: INTRODUCTION TO VOLATILITY
%=============================================================================
\section{Introducere în Modelarea Volatilității}

\begin{frame}{De ce Modelăm Volatilitatea?}
    \begin{block}{Observații Empirice în Seriile Financiare}
        \begin{itemize}
            \item Randamentele financiare prezintă \textbf{volatility clustering} --- perioadele de volatilitate ridicată tind să fie urmate de perioade de volatilitate ridicată
            \item Distribuția randamentelor are \textbf{cozi groase} (leptokurtosis)
            \item Corelația randamentelor este aproape zero, dar corelația pătratelor este semnificativă
            \item Volatilitatea răspunde \textbf{asimetric} la șocuri (leverage effect)
        \end{itemize}
    \end{block}

    \vspace{0.3cm}

    \begin{alertblock}{Limitarea Modelelor ARIMA}
        Modelele ARIMA presupun \textbf{varianță constantă} (homoscedasticitate), ceea ce nu este realist pentru seriile financiare!
    \end{alertblock}
\end{frame}

\begin{frame}{Volatility Clustering}
    \begin{center}
        \includegraphics[width=0.85\textwidth]{garch_volatility_clustering.pdf}
    \end{center}

    \begin{itemize}
        \item Perioadele de volatilitate mare sunt urmate de perioade de volatilitate mare
        \item Perioadele de calm sunt urmate de perioade de calm
        \item Aceasta sugerează că \textbf{varianța condiționată} este predictibilă
    \end{itemize}
\end{frame}

\begin{frame}{Fapte Stilizate ale Randamentelor Financiare}
    \begin{columns}[T]
        \begin{column}{0.5\textwidth}
            \begin{block}{Proprietăți Observate}
                \begin{enumerate}
                    \item \textbf{Absența autocorrelației} în randamente
                    \item \textbf{Autocorrelație semnificativă} în $r_t^2$ și $|r_t|$
                    \item \textbf{Cozi groase} (kurtosis $> 3$)
                    \item \textbf{Leverage effect} --- corelație negativă între randamente și volatilitate
                    \item \textbf{Volatility clustering}
                \end{enumerate}
            \end{block}
        \end{column}
        \begin{column}{0.5\textwidth}
            \begin{center}
                \includegraphics[width=\textwidth]{garch_stylized_facts.pdf}
            \end{center}
        \end{column}
    \end{columns}
\end{frame}

\begin{frame}{Heteroscedasticitate Condiționată}
    \begin{defn}[Varianță Condiționată]
        Fie $\{r_t\}$ o serie de randamente. \textbf{Varianța condiționată} la momentul $t$ este:
        \[
            \sigma_t^2 = \Var(r_t | \mathcal{F}_{t-1}) = \E[(r_t - \mu_t)^2 | \mathcal{F}_{t-1}]
        \]
        unde $\mathcal{F}_{t-1}$ reprezintă informația disponibilă până la momentul $t-1$.
    \end{defn}

    \vspace{0.3cm}

    \begin{block}{Modelul General}
        \[
            r_t = \mu_t + \varepsilon_t, \quad \varepsilon_t = \sigma_t z_t, \quad z_t \sim \text{i.i.d.}(0, 1)
        \]
        unde:
        \begin{itemize}
            \item $\mu_t$ = media condiționată (poate fi modelată ARMA)
            \item $\sigma_t^2$ = varianța condiționată (modelată ARCH/GARCH)
            \item $z_t$ = inovații standardizate (Normal, Student-t, GED)
        \end{itemize}
    \end{block}
\end{frame}

%=============================================================================
% SECTION 2: ARCH MODEL
%=============================================================================
\section{Modelul ARCH}

\begin{frame}{Modelul ARCH(q) --- Engle (1982)}
    \begin{defn}[ARCH(q)]
        Modelul \textbf{Autoregressive Conditional Heteroskedasticity} de ordin $q$:
        \[
            \varepsilon_t = \sigma_t z_t, \quad z_t \sim \text{i.i.d.}(0, 1)
        \]
        \[
            \sigma_t^2 = \omega + \alpha_1 \varepsilon_{t-1}^2 + \alpha_2 \varepsilon_{t-2}^2 + \cdots + \alpha_q \varepsilon_{t-q}^2
        \]
    \end{defn}

    \vspace{0.3cm}

    \begin{block}{Restricții pentru Stationaritate}
        \begin{itemize}
            \item $\omega > 0$ (varianța de bază pozitivă)
            \item $\alpha_i \geq 0$ pentru $i = 1, \ldots, q$ (non-negativitate)
            \item $\sum_{i=1}^{q} \alpha_i < 1$ (stationaritate)
        \end{itemize}
    \end{block}

    \begin{rmk}
        Robert Engle a primit \textbf{Premiul Nobel pentru Economie} în 2003 pentru dezvoltarea modelului ARCH!
    \end{rmk}
\end{frame}

\begin{frame}{Proprietăți ale Modelului ARCH(1)}
    \begin{block}{ARCH(1): $\sigma_t^2 = \omega + \alpha_1 \varepsilon_{t-1}^2$}
        \begin{itemize}
            \item \textbf{Varianța necondiționată}: $\E[\varepsilon_t^2] = \dfrac{\omega}{1 - \alpha_1}$ (dacă $\alpha_1 < 1$)
            \item \textbf{Kurtosis}: $\kappa = 3 \cdot \dfrac{1 - \alpha_1^2}{1 - 3\alpha_1^2}$ (dacă $\alpha_1^2 < 1/3$)
            \item Kurtosis $> 3$ pentru $\alpha_1 > 0$ $\Rightarrow$ \textbf{cozi groase}!
        \end{itemize}
    \end{block}

    \vspace{0.3cm}

    \begin{exampleblock}{Exemplu Numeric}
        Dacă $\omega = 0.0001$ și $\alpha_1 = 0.3$:
        \begin{itemize}
            \item Varianța necondiționată: $\sigma^2 = \frac{0.0001}{1 - 0.3} = 0.000143$
            \item Kurtosis: $\kappa = 3 \cdot \frac{1 - 0.09}{1 - 0.27} = 3.74 > 3$
        \end{itemize}
    \end{exampleblock}
\end{frame}

\begin{frame}{Testarea Efectelor ARCH}
    \begin{block}{Testul Engle pentru Efecte ARCH}
        \textbf{Procedură}:
        \begin{enumerate}
            \item Estimează modelul pentru medie și obține reziduurile $\hat{\varepsilon}_t$
            \item Calculează $\hat{\varepsilon}_t^2$
            \item Regresează $\hat{\varepsilon}_t^2$ pe lag-urile sale:
            \[
                \hat{\varepsilon}_t^2 = \beta_0 + \beta_1 \hat{\varepsilon}_{t-1}^2 + \cdots + \beta_q \hat{\varepsilon}_{t-q}^2 + u_t
            \]
            \item Calculează statistica $LM = T \cdot R^2 \sim \chi^2(q)$
        \end{enumerate}
    \end{block}

    \vspace{0.3cm}

    \begin{alertblock}{Ipoteze}
        \begin{itemize}
            \item $H_0$: Nu există efecte ARCH ($\alpha_1 = \cdots = \alpha_q = 0$)
            \item $H_1$: Există efecte ARCH (cel puțin un $\alpha_i \neq 0$)
        \end{itemize}
    \end{alertblock}
\end{frame}

\begin{frame}{Limitări ale Modelului ARCH}
    \begin{alertblock}{Probleme Practice}
        \begin{enumerate}
            \item \textbf{Ordine mare} --- de obicei sunt necesare multe lag-uri ($q$ mare)
            \item \textbf{Mulți parametri} --- dificultăți de estimare
            \item \textbf{Restricții de non-negativitate} --- greu de impus pentru $q$ mare
            \item \textbf{Nu capturează persistența} --- volatilitatea observată este foarte persistentă
        \end{enumerate}
    \end{alertblock}

    \vspace{0.5cm}

    \begin{block}{Soluția}
        \textbf{Modelul GARCH} --- introduce lag-uri ale varianței condiționate pentru a captura persistența cu mai puțini parametri!
    \end{block}
\end{frame}

%=============================================================================
% SECTION 3: GARCH MODEL
%=============================================================================
\section{Modelul GARCH}

\begin{frame}{Modelul GARCH(p,q) --- Bollerslev (1986)}
    \begin{defn}[GARCH(p,q)]
        Modelul \textbf{Generalized ARCH}:
        \[
            \varepsilon_t = \sigma_t z_t, \quad z_t \sim \text{i.i.d.}(0, 1)
        \]
        \[
            \sigma_t^2 = \omega + \sum_{i=1}^{q} \alpha_i \varepsilon_{t-i}^2 + \sum_{j=1}^{p} \beta_j \sigma_{t-j}^2
        \]
    \end{defn}

    \vspace{0.3cm}

    \begin{block}{Interpretare}
        \begin{itemize}
            \item $\omega$ = nivel de bază al volatilității
            \item $\alpha_i$ = reacția la șocuri recente (news coefficients)
            \item $\beta_j$ = persistența volatilității (memory)
            \item $\alpha + \beta$ = persistența totală
        \end{itemize}
    \end{block}
\end{frame}

\begin{frame}{Modelul GARCH(1,1)}
    \begin{block}{Cel Mai Popular Model de Volatilitate}
        \[
            \sigma_t^2 = \omega + \alpha \varepsilon_{t-1}^2 + \beta \sigma_{t-1}^2
        \]
    \end{block}

    \begin{columns}[T]
        \begin{column}{0.5\textwidth}
            \begin{block}{Restricții}
                \begin{itemize}
                    \item $\omega > 0$
                    \item $\alpha \geq 0$, $\beta \geq 0$
                    \item $\alpha + \beta < 1$ (stationaritate)
                \end{itemize}
            \end{block}

            \begin{block}{Proprietăți}
                \begin{itemize}
                    \item Varianța necondiționată: $\bar{\sigma}^2 = \dfrac{\omega}{1 - \alpha - \beta}$
                    \item Half-life: $HL = \dfrac{\ln(0.5)}{\ln(\alpha + \beta)}$
                \end{itemize}
            \end{block}
        \end{column}
        \begin{column}{0.5\textwidth}
            \begin{center}
                \includegraphics[width=\textwidth]{garch_conditional_variance.pdf}
            \end{center}
        \end{column}
    \end{columns}
\end{frame}

\begin{frame}{GARCH(1,1) ca ARMA pentru $\varepsilon_t^2$}
    \begin{block}{Reprezentare ARMA(1,1)}
        Definim $\nu_t = \varepsilon_t^2 - \sigma_t^2$ (șocul varianței). Atunci:
        \[
            \varepsilon_t^2 = \omega + (\alpha + \beta) \varepsilon_{t-1}^2 + \nu_t - \beta \nu_{t-1}
        \]

        Aceasta este un \textbf{ARMA(1,1)} pentru $\varepsilon_t^2$!
    \end{block}

    \vspace{0.3cm}

    \begin{exampleblock}{Implicații}
        \begin{itemize}
            \item ACF al $\varepsilon_t^2$ decade exponențial (ca ARMA)
            \item Persistența este dată de $\alpha + \beta$
            \item PACF poate ajuta la identificarea ordinului
        \end{itemize}
    \end{exampleblock}
\end{frame}

\begin{frame}{Estimarea Modelelor GARCH}
    \begin{block}{Metoda Verosimilității Maxime (MLE)}
        Funcția de log-verosimilitate (distribuție normală):
        \[
            \ell(\theta) = -\frac{T}{2} \ln(2\pi) - \frac{1}{2} \sum_{t=1}^{T} \left[ \ln(\sigma_t^2) + \frac{\varepsilon_t^2}{\sigma_t^2} \right]
        \]
    \end{block}

    \vspace{0.3cm}

    \begin{block}{Distribuții Alternative pentru $z_t$}
        \begin{itemize}
            \item \textbf{Student-t}: capturează cozile groase
            \[
                f(z; \nu) = \frac{\Gamma\left(\frac{\nu+1}{2}\right)}{\sqrt{(\nu-2)\pi} \Gamma\left(\frac{\nu}{2}\right)} \left(1 + \frac{z^2}{\nu-2}\right)^{-\frac{\nu+1}{2}}
            \]
            \item \textbf{GED (Generalized Error Distribution)}: flexibilitate pentru kurtosis
            \item \textbf{Skewed Student-t}: asimetrie și cozi groase
        \end{itemize}
    \end{block}
\end{frame}

\begin{frame}{Valori Tipice pentru GARCH(1,1)}
    \begin{center}
        \begin{tabular}{lccc}
            \toprule
            \textbf{Serie} & $\bm{\alpha}$ & $\bm{\beta}$ & $\bm{\alpha + \beta}$ \\
            \midrule
            S\&P 500 zilnic & 0.05--0.10 & 0.85--0.95 & 0.95--0.99 \\
            EUR/USD zilnic & 0.03--0.08 & 0.90--0.95 & 0.95--0.99 \\
            Bitcoin zilnic & 0.10--0.20 & 0.75--0.85 & 0.90--0.98 \\
            Obligațiuni & 0.02--0.05 & 0.90--0.97 & 0.95--0.99 \\
            \bottomrule
        \end{tabular}
    \end{center}

    \vspace{0.5cm}

    \begin{alertblock}{Observații}
        \begin{itemize}
            \item $\alpha + \beta$ aproape de 1 $\Rightarrow$ \textbf{volatilitate foarte persistentă}
            \item $\alpha$ mic, $\beta$ mare $\Rightarrow$ reacție lentă la șocuri, memorie lungă
            \item Bitcoin: $\alpha$ mai mare $\Rightarrow$ reacție mai rapidă la news
        \end{itemize}
    \end{alertblock}
\end{frame}

\begin{frame}{IGARCH --- Integrated GARCH}
    \begin{defn}[IGARCH(1,1)]
        Când $\alpha + \beta = 1$:
        \[
            \sigma_t^2 = \omega + \alpha \varepsilon_{t-1}^2 + (1 - \alpha) \sigma_{t-1}^2
        \]
    \end{defn}

    \vspace{0.3cm}

    \begin{block}{Proprietăți}
        \begin{itemize}
            \item Varianța necondiționată nu există (infinită)
            \item Șocurile au efect \textbf{permanent} asupra volatilității
            \item Folosit pentru serii cu persistență extremă
            \item Util pentru \textbf{RiskMetrics} (J.P. Morgan): $\alpha = 0.06$, $\beta = 0.94$
        \end{itemize}
    \end{block}

    \begin{rmk}
        IGARCH este analog cu o rădăcină unitară în varianță!
    \end{rmk}
\end{frame}

%=============================================================================
% SECTION 4: ASYMMETRIC GARCH MODELS
%=============================================================================
\section{Modele GARCH Asimetrice}

\begin{frame}{Leverage Effect}
    \begin{center}
        \includegraphics[width=0.7\textwidth]{garch_leverage_effect.pdf}
    \end{center}

    \begin{block}{Definiție}
        \textbf{Leverage effect}: Șocurile negative (scăderi de preț) tind să crească volatilitatea \textbf{mai mult} decât șocurile pozitive de aceeași magnitudine.
    \end{block}

    \begin{alertblock}{Problema GARCH Standard}
        GARCH(p,q) depinde de $\varepsilon_{t-1}^2$, deci tratează șocurile pozitive și negative \textbf{simetric}!
    \end{alertblock}
\end{frame}

\begin{frame}{Modelul EGARCH --- Nelson (1991)}
    \begin{defn}[EGARCH(1,1)]
        \textbf{Exponential GARCH}:
        \[
            \ln(\sigma_t^2) = \omega + \alpha \left( |z_{t-1}| - \E[|z_{t-1}|] \right) + \gamma z_{t-1} + \beta \ln(\sigma_{t-1}^2)
        \]
        unde $z_t = \varepsilon_t / \sigma_t$.
    \end{defn}

    \vspace{0.3cm}

    \begin{block}{Avantaje EGARCH}
        \begin{itemize}
            \item \textbf{Nu necesită restricții de non-negativitate} --- modelează $\ln(\sigma_t^2)$
            \item \textbf{Captură leverage effect} prin parametrul $\gamma$
            \begin{itemize}
                \item $\gamma < 0$: șocuri negative $\Rightarrow$ volatilitate mai mare
                \item $\gamma = 0$: efect simetric (ca GARCH)
            \end{itemize}
            \item Persistența este dată de $\beta$
        \end{itemize}
    \end{block}
\end{frame}

\begin{frame}{News Impact Curve --- EGARCH}
    \begin{center}
        \includegraphics[width=0.75\textwidth]{garch_news_impact_curve.pdf}
    \end{center}

    \begin{block}{Interpretare}
        \textbf{News Impact Curve}: arată cum volatilitatea viitoare $\sigma_{t+1}^2$ depinde de șocul curent $\varepsilon_t$, menținând $\sigma_t^2$ constant.
    \end{block}
\end{frame}

\begin{frame}{Modelul GJR-GARCH (TGARCH)}
    \begin{defn}[GJR-GARCH(1,1)]
        Glosten, Jagannathan \& Runkle (1993):
        \[
            \sigma_t^2 = \omega + \alpha \varepsilon_{t-1}^2 + \gamma \varepsilon_{t-1}^2 \cdot I_{t-1} + \beta \sigma_{t-1}^2
        \]
        unde $I_{t-1} = \begin{cases} 1 & \text{dacă } \varepsilon_{t-1} < 0 \\ 0 & \text{altfel} \end{cases}$
    \end{defn}

    \vspace{0.3cm}

    \begin{block}{Interpretare}
        \begin{itemize}
            \item Șocuri pozitive ($\varepsilon_{t-1} > 0$): impact = $\alpha$
            \item Șocuri negative ($\varepsilon_{t-1} < 0$): impact = $\alpha + \gamma$
            \item Leverage effect prezent dacă $\gamma > 0$
        \end{itemize}
    \end{block}

    \begin{exampleblock}{Stationaritate}
        $\alpha + \gamma/2 + \beta < 1$
    \end{exampleblock}
\end{frame}

\begin{frame}{TGARCH --- Threshold GARCH}
    \begin{defn}[TGARCH(1,1)]
        Zakoian (1994) --- modelează deviația standard:
        \[
            \sigma_t = \omega + \alpha^+ \varepsilon_{t-1}^+ + \alpha^- \varepsilon_{t-1}^- + \beta \sigma_{t-1}
        \]
        unde $\varepsilon_t^+ = \max(\varepsilon_t, 0)$ și $\varepsilon_t^- = \max(-\varepsilon_t, 0)$.
    \end{defn}

    \vspace{0.3cm}

    \begin{block}{Comparație Modele Asimetrice}
        \begin{center}
            \begin{tabular}{lcc}
                \toprule
                \textbf{Model} & \textbf{Specificație} & \textbf{Leverage} \\
                \midrule
                GARCH & $\sigma_t^2$ & Nu \\
                EGARCH & $\ln(\sigma_t^2)$ & Da ($\gamma < 0$) \\
                GJR-GARCH & $\sigma_t^2$ cu indicator & Da ($\gamma > 0$) \\
                TGARCH & $\sigma_t$ & Da ($\alpha^- > \alpha^+$) \\
                \bottomrule
            \end{tabular}
        \end{center}
    \end{block}
\end{frame}

%=============================================================================
% SECTION 5: MODEL SELECTION AND DIAGNOSTICS
%=============================================================================
\section{Selectarea și Diagnosticarea Modelelor}

\begin{frame}{Selectarea Ordinului}
    \begin{block}{Criterii Informaționale}
        \begin{itemize}
            \item \textbf{AIC} = $-2\ell + 2k$
            \item \textbf{BIC} = $-2\ell + k \ln(T)$
            \item \textbf{HQIC} = $-2\ell + 2k \ln(\ln(T))$
        \end{itemize}
        unde $\ell$ = log-verosimilitate maximizată, $k$ = nr. parametri.
    \end{block}

    \vspace{0.3cm}

    \begin{alertblock}{Recomandări Practice}
        \begin{itemize}
            \item GARCH(1,1) este suficient în \textbf{90\% din cazuri}
            \item Verifică dacă modelul asimetric îmbunătățește semnificativ fit-ul
            \item Alege distribuția inovațiilor care minimizează AIC/BIC
        \end{itemize}
    \end{alertblock}
\end{frame}

\begin{frame}{Diagnosticarea Modelelor GARCH}
    \begin{block}{Reziduuri Standardizate}
        \[
            \hat{z}_t = \frac{\hat{\varepsilon}_t}{\hat{\sigma}_t}
        \]

        Dacă modelul este corect specificat, $\hat{z}_t$ ar trebui să fie i.i.d.(0,1).
    \end{block}

    \vspace{0.3cm}

    \begin{block}{Verificări Diagnostic}
        \begin{enumerate}
            \item \textbf{Ljung-Box pe $\hat{z}_t$}: verifică absența autocorrelației în medie
            \item \textbf{Ljung-Box pe $\hat{z}_t^2$}: verifică absența efectelor ARCH reziduale
            \item \textbf{Test ARCH-LM pe $\hat{z}_t$}: confirmare absență heteroscedasticitate
            \item \textbf{Histogramă + QQ-plot}: verifică distribuția asumată
        \end{enumerate}
    \end{block}
\end{frame}

\begin{frame}{Exemplu Diagnostic}
    \begin{center}
        \includegraphics[width=0.85\textwidth]{garch_diagnostics.pdf}
    \end{center}
\end{frame}

%=============================================================================
% SECTION 6: FORECASTING VOLATILITY
%=============================================================================
\section{Prognoza Volatilității}

\begin{frame}{Prognoza cu GARCH(1,1)}
    \begin{block}{Prognoză Un Pas Înainte}
        \[
            \hat{\sigma}_{T+1}^2 = \omega + \alpha \varepsilon_T^2 + \beta \sigma_T^2
        \]
    \end{block}

    \begin{block}{Prognoză Multi-Pas}
        Pentru $h > 1$:
        \[
            \E_T[\sigma_{T+h}^2] = \bar{\sigma}^2 + (\alpha + \beta)^{h-1} (\sigma_{T+1}^2 - \bar{\sigma}^2)
        \]
        unde $\bar{\sigma}^2 = \frac{\omega}{1 - \alpha - \beta}$ = varianța necondiționată.
    \end{block}

    \vspace{0.3cm}

    \begin{exampleblock}{Convergență}
        \[
            \lim_{h \to \infty} \E_T[\sigma_{T+h}^2] = \bar{\sigma}^2
        \]
        Prognoza converge către varianța necondiționată!
    \end{exampleblock}
\end{frame}

\begin{frame}{Prognoza Volatilității --- Vizualizare}
    \begin{center}
        \includegraphics[width=0.85\textwidth]{garch_forecast.pdf}
    \end{center}

    \begin{itemize}
        \item Prognoza converge exponențial către $\bar{\sigma}^2$
        \item Viteza de convergență depinde de $\alpha + \beta$
        \item Cu cât $\alpha + \beta$ mai aproape de 1, cu atât convergența mai lentă
    \end{itemize}
\end{frame}

\begin{frame}{Aplicații ale Prognozei Volatilității}
    \begin{columns}[T]
        \begin{column}{0.5\textwidth}
            \begin{block}{Value at Risk (VaR)}
                \[
                    \text{VaR}_\alpha = -\mu_{T+1} + z_\alpha \cdot \sigma_{T+1}
                \]

                Probabilitatea de a pierde mai mult decât VaR este $\alpha$ (ex: 1\%, 5\%).
            \end{block}

            \begin{block}{Expected Shortfall}
                \[
                    \text{ES}_\alpha = \E[-r_{T+1} | r_{T+1} < -\text{VaR}_\alpha]
                \]
            \end{block}
        \end{column}
        \begin{column}{0.5\textwidth}
            \begin{block}{Alte Aplicații}
                \begin{itemize}
                    \item Prețul opțiunilor
                    \item Hedging dinamic
                    \item Alocarea portofoliului
                    \item Stress testing
                    \item Analiza scenariilor
                \end{itemize}
            \end{block}
        \end{column}
    \end{columns}
\end{frame}

\begin{frame}{Value at Risk --- Exemplu Numeric}
    \begin{exampleblock}{Calculul VaR pentru un Portofoliu}
        \textbf{Date:} Portofoliu de \textbf{1.000.000 EUR}, volatilitate prognozată $\hat{\sigma}_{T+1} = 1.5\%$
    \end{exampleblock}

    \vspace{0.2cm}

    \begin{block}{VaR cu Distribuție Normală}
        \begin{center}
            \begin{tabular}{lccc}
                \toprule
                \textbf{Nivel} & $\bm{z_\alpha}$ & \textbf{VaR (\%)} & \textbf{VaR (EUR)} \\
                \midrule
                95\% (1 zi) & 1.645 & 2.47\% & 24.675 \\
                99\% (1 zi) & 2.326 & 3.49\% & 34.890 \\
                99\% (10 zile) & $2.326 \cdot \sqrt{10}$ & 11.03\% & 110.314 \\
                \bottomrule
            \end{tabular}
        \end{center}
    \end{block}

    \begin{alertblock}{Scalare pentru Perioade Mai Lungi}
        \[
            \text{VaR}_{h\text{ zile}} = \text{VaR}_{1\text{ zi}} \cdot \sqrt{h}
        \]
        Această regulă presupune randamente i.i.d. --- o aproximare pentru orizonturi scurte.
    \end{alertblock}
\end{frame}

\begin{frame}{Value at Risk --- Distribuție Student-t}
    \begin{block}{De ce Student-t?}
        \begin{itemize}
            \item Distribuția normală \textbf{subestimează} riscul de coadă
            \item Randamentele financiare au \textbf{cozi groase} (kurtosis $> 3$)
            \item Student-t cu $\nu$ grade de libertate capturează mai bine extremele
        \end{itemize}
    \end{block}

    \vspace{0.2cm}

    \begin{exampleblock}{Comparație VaR 99\% (1 zi) pentru $\sigma = 1.5\%$, Portofoliu = 1M EUR}
        \begin{center}
            \begin{tabular}{lcc}
                \toprule
                \textbf{Distribuție} & \textbf{Cuantilă} & \textbf{VaR (EUR)} \\
                \midrule
                Normal & 2.326 & 34.890 \\
                Student-t ($\nu = 10$) & 2.764 & 41.460 \\
                Student-t ($\nu = 6$) & 3.143 & 47.145 \\
                Student-t ($\nu = 4$) & 3.747 & 56.205 \\
                \bottomrule
            \end{tabular}
        \end{center}
    \end{exampleblock}

    \begin{alertblock}{Observație}
        Cu $\nu = 6$ (tipic pentru acțiuni), VaR este cu \textbf{35\% mai mare} decât cel normal!
    \end{alertblock}
\end{frame}

\begin{frame}{VaR --- Exemplu Complet cu GARCH}
    \begin{block}{Procedura de Calcul VaR}
        \begin{enumerate}
            \item Estimează modelul GARCH(1,1) cu distribuție Student-t
            \item Obține prognoza volatilității: $\hat{\sigma}_{T+1}$
            \item Calculează VaR: $\text{VaR}_\alpha = t_\alpha(\nu) \cdot \hat{\sigma}_{T+1} \cdot \sqrt{\frac{\nu-2}{\nu}}$
        \end{enumerate}
    \end{block}

    \vspace{0.2cm}

    \begin{exampleblock}{Exemplu: S\&P 500}
        \begin{itemize}
            \item Parametri estimați: $\alpha = 0.088$, $\beta = 0.900$, $\nu = 6.4$
            \item Volatilitate prognozată: $\hat{\sigma}_{T+1} = 1.2\%$
            \item Portofoliu: 10.000.000 EUR
        \end{itemize}

        \vspace{0.2cm}

        \textbf{VaR 99\% (1 zi):}
        \[
            \text{VaR} = 3.05 \times 0.012 \times 10.000.000 = \textbf{366.000 EUR}
        \]
    \end{exampleblock}
\end{frame}

%=============================================================================
% SECTION 7: PYTHON IMPLEMENTATION
%=============================================================================
\section{Implementare în Python}

\begin{frame}[fragile]{Estimare GARCH în Python --- arch}
\begin{block}{Instalare și Import}
\begin{verbatim}
pip install arch

import numpy as np
import pandas as pd
from arch import arch_model
from arch.univariate import GARCH, EGARCH, ConstantMean
\end{verbatim}
\end{block}

\begin{block}{Estimare GARCH(1,1)}
\begin{verbatim}
# Presupunem returns = seria de randamente (%)
model = arch_model(returns, vol='Garch', p=1, q=1,
                   dist='normal')
results = model.fit(disp='off')
print(results.summary())
\end{verbatim}
\end{block}
\end{frame}

\begin{frame}[fragile]{Modele Asimetrice în Python}
\begin{block}{EGARCH}
\begin{verbatim}
model_egarch = arch_model(returns, vol='EGARCH', p=1, q=1)
res_egarch = model_egarch.fit(disp='off')
\end{verbatim}
\end{block}

\begin{block}{GJR-GARCH}
\begin{verbatim}
model_gjr = arch_model(returns, vol='Garch', p=1, o=1, q=1)
res_gjr = model_gjr.fit(disp='off')
\end{verbatim}
\end{block}

\begin{block}{Distribuții Alternative}
\begin{verbatim}
# Student-t
model_t = arch_model(returns, vol='Garch', p=1, q=1,
                     dist='t')
# Skewed Student-t
model_skewt = arch_model(returns, vol='Garch', p=1, q=1,
                         dist='skewt')
\end{verbatim}
\end{block}
\end{frame}

\begin{frame}[fragile]{Prognoză și Diagnostic}
\begin{block}{Prognoză Volatilitate}
\begin{verbatim}
# Prognoză 10 pași înainte
forecasts = results.forecast(horizon=10)
volatility_forecast = np.sqrt(forecasts.variance.values[-1, :])
\end{verbatim}
\end{block}

\begin{block}{Reziduuri Standardizate}
\begin{verbatim}
std_resid = results.std_resid

# Test Ljung-Box
from statsmodels.stats.diagnostic import acorr_ljungbox
lb_test = acorr_ljungbox(std_resid**2, lags=10)
\end{verbatim}
\end{block}

\begin{block}{VaR Calculat}
\begin{verbatim}
sigma_forecast = np.sqrt(forecasts.variance.values[-1, 0])
VaR_95 = 1.645 * sigma_forecast  # pentru alpha = 5%
\end{verbatim}
\end{block}
\end{frame}

%=============================================================================
% SECTION 8: CASE STUDY
%=============================================================================
\section{Studiu de Caz: S\&P 500}

\begin{frame}{Analiza Volatilității S\&P 500}
    \begin{center}
        \includegraphics[width=0.85\textwidth]{garch_sp500_returns.pdf}
    \end{center}

    \begin{itemize}
        \item Randamente zilnice S\&P 500 (2000--2024)
        \item Volatility clustering vizibil în perioadele de criză
        \item 2008 (criză financiară), 2020 (COVID-19), 2022 (inflație)
    \end{itemize}
\end{frame}

\begin{frame}{Estimare GARCH(1,1) --- S\&P 500}
    \begin{columns}[T]
        \begin{column}{0.5\textwidth}
            \begin{block}{Rezultate Estimare}
                \begin{center}
                    \begin{tabular}{lc}
                        \toprule
                        \textbf{Parametru} & \textbf{Valoare} \\
                        \midrule
                        $\omega$ & 0.0108 \\
                        $\alpha$ & 0.0883 \\
                        $\beta$ & 0.9002 \\
                        $\alpha + \beta$ & 0.9885 \\
                        \midrule
                        $\nu$ (Student-t df) & 6.42 \\
                        \bottomrule
                    \end{tabular}
                \end{center}
            \end{block}

            \begin{block}{Interpretare}
                \begin{itemize}
                    \item Volatilitate foarte persistentă
                    \item Half-life $\approx$ 60 zile
                    \item Cozi groase (Student-t)
                \end{itemize}
            \end{block}
        \end{column}
        \begin{column}{0.5\textwidth}
            \begin{center}
                \includegraphics[width=\textwidth]{garch_sp500_volatility.pdf}
            \end{center}
        \end{column}
    \end{columns}
\end{frame}

\begin{frame}{Comparație GARCH vs EGARCH --- S\&P 500}
    \begin{center}
        \includegraphics[width=0.85\textwidth]{garch_sp500_comparison.pdf}
    \end{center}

    \begin{block}{Leverage Effect Confirmat}
        EGARCH: $\gamma = -0.12$ (semnificativ negativ) $\Rightarrow$ șocurile negative amplifică volatilitatea mai mult decât cele pozitive
    \end{block}
\end{frame}

%=============================================================================
% KEY FORMULAS
%=============================================================================
\begin{frame}{Formule Cheie}
    \begin{block}{Modele de Volatilitate}
        \begin{itemize}
            \item \textbf{ARCH(q):} $\sigma_t^2 = \omega + \sum_{i=1}^{q} \alpha_i \varepsilon_{t-i}^2$
            \item \textbf{GARCH(1,1):} $\sigma_t^2 = \omega + \alpha \varepsilon_{t-1}^2 + \beta \sigma_{t-1}^2$
            \item \textbf{EGARCH:} $\ln(\sigma_t^2) = \omega + \alpha(|z_{t-1}| - \E[|z|]) + \gamma z_{t-1} + \beta \ln(\sigma_{t-1}^2)$
            \item \textbf{GJR-GARCH:} $\sigma_t^2 = \omega + \alpha \varepsilon_{t-1}^2 + \gamma \varepsilon_{t-1}^2 I_{t-1} + \beta \sigma_{t-1}^2$
        \end{itemize}
    \end{block}

    \vspace{0.2cm}

    \begin{block}{Proprietăți și Măsuri}
        \begin{columns}[T]
            \begin{column}{0.5\textwidth}
                \begin{itemize}
                    \item \textbf{Varianță necondiționată:} $\bar{\sigma}^2 = \dfrac{\omega}{1-\alpha-\beta}$
                    \item \textbf{Half-life:} $HL = \dfrac{\ln(0.5)}{\ln(\alpha+\beta)}$
                \end{itemize}
            \end{column}
            \begin{column}{0.5\textwidth}
                \begin{itemize}
                    \item \textbf{VaR:} $\text{VaR}_\alpha = z_\alpha \cdot \sigma_{T+1}$
                    \item \textbf{Stationaritate:} $\alpha + \beta < 1$
                \end{itemize}
            \end{column}
        \end{columns}
    \end{block}

    \begin{alertblock}{Test ARCH-LM}
        $LM = T \cdot R^2 \sim \chi^2(q)$ unde $R^2$ provine din regresia $\hat{\varepsilon}_t^2$ pe lag-urile sale
    \end{alertblock}
\end{frame}

%=============================================================================
% SUMMARY
%=============================================================================
\section{Rezumat}

\begin{frame}{Rezumat --- Capitolul 5: Modele de Volatilitate}
    \begin{block}{Concepte Cheie}
        \begin{itemize}
            \item \textbf{ARCH(q)}: varianța condiționată depinde de pătratele erorilor trecute
            \item \textbf{GARCH(p,q)}: adaugă lag-uri ale varianței pentru persistență
            \item \textbf{EGARCH}: permite leverage effect, fără restricții de pozitivitate
            \item \textbf{GJR-GARCH/TGARCH}: captură asimetria cu variabile indicator
        \end{itemize}
    \end{block}

    \begin{block}{Aplicații}
        \begin{itemize}
            \item Măsurarea și prognoza riscului (VaR, ES)
            \item Prețul derivatelor
            \item Managementul portofoliului
        \end{itemize}
    \end{block}

    \begin{alertblock}{Sfat Practic}
        Începe cu GARCH(1,1), verifică leverage effect, alege distribuția inovațiilor care minimizează AIC/BIC!
    \end{alertblock}
\end{frame}

\begin{frame}{Referințe}
    \begin{thebibliography}{10}
        \bibitem{engle1982} Engle, R.F. (1982). \textit{Autoregressive Conditional Heteroscedasticity with Estimates of the Variance of United Kingdom Inflation}. Econometrica, 50(4), 987-1007.

        \bibitem{bollerslev1986} Bollerslev, T. (1986). \textit{Generalized Autoregressive Conditional Heteroskedasticity}. Journal of Econometrics, 31(3), 307-327.

        \bibitem{nelson1991} Nelson, D.B. (1991). \textit{Conditional Heteroskedasticity in Asset Returns: A New Approach}. Econometrica, 59(2), 347-370.

        \bibitem{gjr1993} Glosten, L.R., Jagannathan, R., \& Runkle, D.E. (1993). \textit{On the Relation between the Expected Value and the Volatility of the Nominal Excess Return on Stocks}. The Journal of Finance, 48(5), 1779-1801.

        \bibitem{tsay2010} Tsay, R.S. (2010). \textit{Analysis of Financial Time Series}. 3rd Edition, Wiley.
    \end{thebibliography}
\end{frame}

\end{document}
